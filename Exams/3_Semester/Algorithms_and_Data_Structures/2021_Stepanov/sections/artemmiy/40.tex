
\subsubsection*{Расстояние от точки до прямой, проекция}

Пусть дана прямая $ax + by + c = 0$ и точка $p (x0, y0)$.

\Statement Рассстояние от точки $p$ до прямой $ax + by + c$ будет равно $\frac{|ax_0 + by_0 + c|}{\sqrt{a^2 + b^2}}$.

\Proof Рассмотрим точку на прямой $ax + by + c = 0$, заданную как $q = p + \vec{n} \cdot \lambda$, где $\vec{n} = (a, b)$ - вектор нормали к прямой.  В матричном виде: $q = \begin{matrix} x_0 + a \cdot \lambda \\ y_0 + b \cdot \lambda \end{matrix}$. Далее подставляем $a (x_0 + a \cdot \lambda) + b (y_0 + b \lambda) + c = 0$, $\lambda (a^2 + b^2) = -ax_0 - by_0 - c$, откуда $\lambda = - \frac{ax_0 + by_0 + c}{a^2 + b^2}$. Расстоянием до прямой будет $|\lambda \vec{n}| = |\lambda| \cdot |\vec{n}| = \frac{|ax_0 + by_0 + c|}{a^2 + b^2} \cdot \sqrt{a^2 + b^2}$. $\square$

Точка $q = p + \vec{n} \cdot \lambda$, будет проекцией точки $p$ на эту прямую.

\subsubsection*{Пересечение прямых}

Пригодится метод Крамера. Пусть у нас есть две прямые $a_1 x + b_1 y + c_1 = 0$ и $a_2 x + b_2 y + c_2 = 0$. По сути мы решаем систему линейных уравнений:

\begin{center}
    $\begin{cases}
    a_1 x + b_1 y = -c_1; \\
    a_2 x + b_2 y = -c_2.
    \end{cases}$
\end{center}

Считаем определитель матрицы, составленной из $a_1$, $a_2$, $b_1$, $b_2$ как $\Delta = a_1 b_2 - a_2 b_1$, посчитаем ещё $\Delta_1 = -c_1 b_2 + c_2 b_1$, $\Delta_2 = -a_1 c_2 + a_2 c_1$. Если $\Delta = 0$, то прямые либо параллельны, либо совпадают, так как направляющие векторы либо параллельны, либо совпадают. Если $\Delta \neq 0$, то единственным решением системы будет $(\frac{\Delta_1}{\Delta}, \frac{\Delta_2}{\Delta})$.

Как понять, что прямые совпадают? Прямые совпадают тогда и только тогда, когда существует такое число $k$, что $a_1 = k a_2$, $b_1 = k b_2$, $c_1 = k c_2$. Если $a_2 \neq 0$, то подойдёт $k = \frac{a_1}{a_2}$.
