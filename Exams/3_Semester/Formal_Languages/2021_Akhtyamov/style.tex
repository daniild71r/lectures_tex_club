\documentclass[a4paper,12pt]{article}
\usepackage{cmap}
\usepackage{amssymb}
\usepackage{amsmath}
\usepackage[T2A]{fontenc}
\usepackage[utf8]{inputenc}
\usepackage[russian]{babel}
\usepackage{indentfirst}
\usepackage{epigraph}
\renewcommand{\epigraphsize}{\small}
% \pagestyle{empty}

\usepackage[left=1.5cm,right=1.5cm,
    top=2cm,bottom=2cm]{geometry}
\usepackage{graphicx}
\graphicspath{ {./images/} }
\newcommand{\RomanNumeralCaps}[1] {\MakeUppercase{\romannumeral #1}}

\newcommand{\mysection}[2]{\setcounter{section}{#1}\addtocounter{section}{-1}\section{#2}}

\setcounter{secnumdepth}{0} 

\linespread{1.15}                          % коэффициент межстрочного интервала
\setlength{\parskip}{0.4em}                % вертик. интервал между абзацами
\binoppenalty=1000 

\usepackage{graphicx}%Вставка картинок правильная
\usepackage{float}%"Плавающие" картинки
\usepackage{wrapfig}%Обтекание фигур (таблиц, картинок и прочего)

\DeclareRobustCommand{\svdots}{% s for `scaling' - знак кратности выровненный по высоте букв
  \, \vcenter{%
    \offinterlineskip
    \hbox{.}
    \vskip0.25\normalbaselineskip
    \hbox{.}
    \vskip0.25\normalbaselineskip
    \hbox{.}%
  }% 
  \,
}
\usepackage{listings}
\usepackage[unicode, pdftex]{hyperref}
\usepackage{xcolor}

\definecolor{linkcolor}{HTML}{50006b} % цвет ссылок
%\definecolor{urlcolor}{HTML}{107896} % цвет гиперссылок
\definecolor{urlcolor}{HTML}{50006b} % цвет гиперссылок
 
\hypersetup{pdfstartview=FitH,  linkcolor=linkcolor,urlcolor=urlcolor, colorlinks=true}

\definecolor{codegreen}{rgb}{0,0.6,0}
\definecolor{codegray}{rgb}{0.5,0.5,0.5}
\definecolor{codepurple}{rgb}{0.58,0,0.82}
\definecolor{backcolour}{cmyk}{0,0,0,0.05}

\lstdefinestyle{mystyle}{
    backgroundcolor=\color{backcolour},
    commentstyle=\color{codegreen},
    keywordstyle=\color{magenta},
    numberstyle=\tiny\color{codegray},
    stringstyle=\color{codepurple},
    basicstyle=\ttfamily\footnotesize,
    breakatwhitespace=false,
    breaklines=true,
    captionpos=b,
    keepspaces=true,
    numbers=left,
    numbersep=5pt,
    showspaces=false,
    showstringspaces=false,
    showtabs=false,
    tabsize=2,
    texcl=true
}

\lstset{extendedchars=\true, style=mystyle}

\usepackage[pages = some]{background}
\backgroundsetup{
	scale = 1,
	angle = 0,
	opacity = 1,
	contents = {\includegraphics[height = \paperheight, keepaspectratio]{background.jpg}}}
	

% нужно для стрелочки в билете 5.21
\newcommand{\leftrarrows}{\mathrel{\raise.75ex\hbox{\oalign{%
  $\scriptstyle\leftarrow$\cr
  \vrule width0pt height.5ex$\hfil\scriptstyle\relbar$\cr}}}}
\newcommand{\lrightarrows}{\mathrel{\raise.75ex\hbox{\oalign{%
  $\scriptstyle\relbar$\hfil\cr
  $\scriptstyle\vrule width0pt height.5ex\smash\rightarrow$\cr}}}}
\newcommand{\Rrelbar}{\mathrel{\raise.75ex\hbox{\oalign{%
  $\scriptstyle\relbar$\cr
  \vrule width0pt height.5ex$\scriptstyle\relbar$}}}}
\newcommand{\longleftrightarrows}{\leftrarrows\joinrel\Rrelbar\joinrel\lrightarrows}

\makeatletter
\def\leftrightarrowsfill@{\arrowfill@\leftrarrows\Rrelbar\lrightarrows}
\newcommand{\xleftrightarrows}[2][]{\ext@arrow 3399\leftrightarrowsfill@{#1}{#2}}
\makeatother


\newcommand{\Def}{\textbf{Опеределение:} }  % объявление новых макрокоманд
\newcommand{\Statement}{\textbf{Утверждение:} }
\newcommand{\Lemma}{\textbf{Лемма:} }
\newcommand{\Th}{\textbf{Теорема:} }
\newcommand{\Task}{\textbf{Задача:} }
\newcommand{\Solution}{\textbf{Решение:} }
\newcommand{\Example}{\textbf{Пример:} }
\newcommand{\Note}{\textbf{Замечание:} } 
\newcommand{\Vars}{\textbf{Введем обозначения:} } 
\newcommand{\Proof}{$\blacktriangle$ }
\newcommand{\EndProof}{$\blacksquare$ }
\newcommand{\N}{\mathbb{N}}

\newcommand{\Le}{\leqslant}                % русский стиль нестрогих неравенств
\newcommand{\Ge}{\geqslant}
\newcommand{\brackets}[1]{\left({#1}\right)} % автоматический размер скобочек

\newcommand{\angles}[1]{\left\langle{#1}\right\rangle}
\newcommand{\abs}[1]{\left|{#1}\right|}
\newcommand{\bracketss}[1]{\left({#1}\right)}


\newcommand{\drawsome}[1]{
    \begin{figure}[h!]
            \centering
            \includegraphics[scale=0.7]{#1}
            \label{fig:first}
    \end{figure}
}

\newcommand{\drawsomebig}[1]{
    \begin{figure}[h!]
            \centering
            \includegraphics[scale=1.15]{#1}
            \label{fig:first}
    \end{figure}
}

\newcommand{\drawsomemedium}[1]{
    \begin{figure}[h!]
            \centering
            \includegraphics[scale=0.45]{#1}
            \label{fig:first}
    \end{figure}
}

\newcommand{\drawsomesmall}[1]{
    \begin{figure}[h!]
            \centering
            \includegraphics[scale=0.3]{#1}
            \label{fig:first}
    \end{figure}
}