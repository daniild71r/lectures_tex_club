\subsection{32. Замкнутость класса контекстно-свободных языков относительно конечных преобразований.}

\textbf{Теорема. }

Если $L$ -- контекстно-свободный язык, $\psi$ -- конечное преобразование, то
$\psi(L)$ -- контекстно-свободный язык.

\underline{Доказательство}

Так как $\psi$ -- конечное преобразование, то по теореме Нива
существуют такие $\eta, R, \varphi$, что 
$\psi = \eta \circ \idR \circ \varphi^{-1}$, 
где $\eta, \varphi$ -- неудлиняющие гомоморфизмы,
$R$ -- регулярный (автоматный) язык.

Тогда достаточно доказать 3 утверждения:

1) Если $\varphi$ -- неудлиняющий гомоморфизм и $L$ -- КС-язык, то $\varphi^{-1} (L)$ -- КС-язык.

2) Если $R$ -- регулярный язык и $L$ -- КС-язык, то $\idR (L) = L \cap R$ -- КС-язык.

3) Если $\varphi$ -- неудлиняющий гомоморфизм и $L$ -- КС-язык, то $\varphi (L)$ -- КС-язык.

1) Введем обозначения:

$\Gamma_{\varepsilon} = \{ u \in \Gamma \ |\ \varphi(u) = \varepsilon \}$

$\Gamma_a = \{ u \in \Gamma \ |\ \varphi(u) = a \}$

Тогда $\varphi^{-1}(\varepsilon) = \Gamma_{\varepsilon}^{*}$, а 
$\varphi^{-1}(a) = \Gamma_{\varepsilon}^{*} \Gamma_a \Gamma_{\varepsilon}^{*}$

Заведём нетерминалы $S_a$:

$S_a \vdash \varphi^{-1}(a) = \Gamma_{\varepsilon}^{*} \Gamma_a \Gamma_{\varepsilon}^{*}$

Так как $\Gamma_{\varepsilon}^{*} \Gamma_a \Gamma_{\varepsilon}^{*}$ -- регулярный, то построим $G_a = \angles{\ldots, S_a}$, распознающий этот язык.

$L$ -- КС-язык, следовательно, существует КС-грамматика в нормальной форме Хомского $G = \angles{N, \Sigma, P, S}\ :\ L(G) = L$, то есть в ней есть только правила вида $S \to \varepsilon, A \to BC, A \to a$.

Заменим правила в ней следующим образом:

$S \to \varepsilon$ заменим на $S \to S_{\varepsilon}$, при этом добавим правила $S_{\varepsilon} \to x S_{\varepsilon} \forall x \in \Gamma_{\varepsilon}$ и $S_{\varepsilon} \to \varepsilon$.

$S \to AB$ заменим на $S \to S_{\varepsilon} A S_{\varepsilon} B S_{\varepsilon}$.

$A \to a$ заменим на $A \to S_a$

2) Если $L$ -- КС-язык, $R$ -- регулярный язык, то $L \cap R$ -- КС-язык.

Пусть $R$ -- регулярный язык, заданный своим ДКА $\angles{\Sigma, Q_1, s_1, T_1, \delta_1⟩}$, и $L$ -- КС-язык, заданный своим МП-автоматом: $\angles{\Sigma, \Gamma, Q_2, s_2, T_2, z_0, \delta_2}$. Тогда прямым произведением назовем следующий автомат:

\begin{itemize}
    \item $Q = \{ \angles{q_1, q_2}\ |\ q_1 \in Q_1, q_2 \in Q_2 \}$. Иначе говоря, состояние в новом автомате -- пара из состояния первого автомата и состояния второго автомата.
    
    \item $s = \angles{s_1, s_2}$
    
    \item Стековый алфавит $\Gamma$ остается неизменным.
    
    \item $T = \{ \angles{t_1,t_2}\ |\ t_1 \in T_1, t_2 \in T_2 \}$. Допускающие состояния нового автомата -- пары состояний, где оба состояния были допускающими в своем автомате.
    
    \item $\delta(\angles{q_1, q_2}, c, d) = \angles{\delta_{1}(q_1, c), \delta_{2}(q_2, c, d)}$. 
    При этом на стек кладется то, что положил бы изначальный МП-автомат при совершении перехода из состояния $q_2$, видя на ленте символ $c$ и символ $d$ на вершине стека.
    
\end{itemize}


Этот автомат использует в качестве состояний пары из двух состояний каждого автомата, а за операции со стеком отвечает только МП-автомат. Слово допускается этим автоматом $\Leftrightarrow$ слово допускается и ДКА, и МП-автоматом, то есть язык данного автомата совпадает с $R \cap L$.

3) $L$ -- КС-язык, следовательно, существует КС-грамматика в нормальной форме Хомского $G = \angles{N, \Sigma, P, S}\ :\ L(G) = L$, то есть в ней есть только правила вида $S \to \varepsilon, A \to BC, A \to a$. Для того, чтобы эта грамматика задавала язык $\varphi(L)$ достаточно заменить правила вида $A \to a$ на $A \to \varphi(a)$ (остальные правила оставить без изменений).

$S \vdash_G w \Leftrightarrow S \vdash_{G'} \varphi(w)$

Идея доказательства: по индукции можно доказать, что

$A \vdash_G w \Leftrightarrow A \vdash_{G'} \varphi(w)$.
