\subsection{6. Минимальный ДКА, его единственность.}
% 1) Пусть $q_1, q_2 \in [q]$. Рассмотрим произвольное $w \in \Sigma^*$. Мы знаем, что $\Delta(q_1, w) \in F \Longleftrightarrow \Delta(q_2, w) \in F$
%  Пусть $p_1 = \Delta(q_1, w)$, $p_2 = \Delta(q_2, w)$. 
 
% Докажем, что $p_1 \sim_M p_2$. Пусть это не так, тогда (без ограничения общности)
% $\exists v \in \Sigma^*: \Delta(p_1, v) \in F, \Delta(p_2, v) \not\in F$. Тогда заметим, что неверно $\Delta(q_1, wv) \in F \Longleftrightarrow \Delta(q_2, wv) \in F$, то есть $q_1 \nsim_M q_2$. Противоречие.\\

% 2)  Пусть $q_1, q_2 \in [q]$. Рассмотрим произвольное $w \in \Sigma^*$. Мы знаем, что $\Delta(q_1, w) \in F \Longleftrightarrow \Delta(q_2, w) \in F$. Возьмем $w = \varepsilon$. Получим, что $q_1 \in F \Longleftrightarrow q_2 \in F$.\\

% Поэтому, если рассмотрим автомат на классах эквивалентности, он будет построен корректно. Уточнение: для него $\Delta([u], a) = [ua], u \in \Sigma^*, a \in \Sigma$.

% 3) Понятно, что любое слово из исходного языка читается, ведь мы добавляли ребро в новом автомате в случае, когда из состояний одного класса такие ребра шли в состояния другого класса. А еще завершающими состояниями являются классы, в которых лежат завершающие состояния. Поэтому каждому пути из стартовой вершины в завершающее состояние в исходном автомате соответствует путь из стартовой вершины в завершающую в новом автомате.\\

% Заметим теперь, что никакое новое слово приниматься не начало. Потому что для каждого ребра между классами состояний можно найти ребро с таким же словом между любыми двумя представителями этих классов (следует из первого пункта и способа построения автомата). Поэтому для любого принимаемого новым автоматом слова $w$ в исходном автомате можно построить путь из стартового состояния в завершающее, пройдя по которому, можно прочитать $w$.\\

% 4) В автомате, построенном на классах эквивалентности состояний никакие два состояния не эквивалентны, потому что тогда бы они лежали в одном классе, т.е. были бы одним состоянием.
% \EndProof

% Что необходимо показать?\\
% 1) Как его получить? (это будет в следующем билете)\\
% 2) Почему он существует?\\
% 3) Почему он единственный с точностью до изоморфизма?\\

% \Th $M$ -- минимальный ПДКА, распознающий язык $L$ тогда и только тогда, когда любые два состояния попарно неэквивалентны и все состояния достижимы из стартового.

% \Proof
% Состояния, не достижимые из стартовой вершины, очевидно, не нужны, и их можно убрать.

% Обозначим $q_1 = \Delta(q_0, w_1)$, $q_2 = \Delta(q_0, w_2)$.

% Заметим, что $\big(\Delta(q_1, w) \in F \Longleftrightarrow \Delta(q_2, w) \in F\big) \Longleftrightarrow \big(w_1w \in L \Longleftrightarrow w_2w \in L\big)$, поэтому $q_1 \sim_M q_2 \Longleftrightarrow w_1 \sim_L w_2$.

% (Посмотрим на определение $w_1 \sim_L w_2$: $\forall w \in \Sigma^*: w_1w \in L \Longleftrightarrow w_2w \in L$ Поймем, что это условие по смыслу соответствует условию эквивалентности состояний, ведь слово $w_1w$ лежит в языке в том и только в том случае, когда его можно прочитать из $q_0$ и попасть в состояние из $F$).

\Def $M$ -- минимальный ПДКА, распознающий язык $L$, если $M$ минимальный по количеству состояний.

В предыдущем билете мы доказали существование минимального ПДКА (это логически следует из леммы и теоремы).\\

\textit{Замечание 1:} в МПДКА $M = \langle Q, \ldots \rangle \; \hookrightarrow \; \arrowvert \Sigma^* /_{\sim_L} \arrowvert = \arrowvert Q \arrowvert$

\textit{Замечание 2:} $[u] \in \Sigma^* /_{\sim_L} \Longrightarrow [u] = \cup_{q}{L_q}$, где $L_q := \{w \,\arrowvert\, \Delta(q_0, w) = q\}$ в МПДКА. Тогда получаем, что $[u] \in \Sigma^* /_{\sim_L} \Longrightarrow \exists!\, q :\; [u]=L_q$.

\Th Для любого автоматного языка $L$ существует единственный с точностью до изоморфизма минимальный ПДКА $M$, такой что $L = L \brackets{M}$.

\Proof Пусть $M$ — минимальный ПДКА, $Q_M$ — множество его состояний.

Построим автомат $M_0 = \langle \Sigma^* /_{\sim_L}, \Sigma, \Delta, [\varepsilon], \{ [w] | w \in L \} \rangle$. Для любых $u \in \Sigma^*$, $a \in \Sigma$ верно, что $\Delta \brackets{[u], a} = [ua]$ (факт 1). Так как количество состоянии автомата, соответствующему языку $L$, конечно, то по следствию из леммы о классах эквивалентности и $L_q$: $|\Sigma^* /_{\sim_L}| < \infty$. 

Факт 1 верен вследствие следующего:

\begin{center}
    $u \sim_L v \Longrightarrow ua \sim_L va \Longleftrightarrow \forall w \brackets{uaw \in L \Longleftrightarrow vaw \in L}$
    
    $w' = aw$, $\forall w' \brackets{uw' \in L \Longleftrightarrow vw' \in L} \Longleftrightarrow u \sim_L v$
\end{center}

Так как $u \sim_L v$, то если $u \in L$, то $v \in L$, и наоборот (просто берем $w=\varepsilon$).

Теперь рассмотрим $\psi : Q_M \rightarrow \Sigma^* /_{\sim_L}$, $\psi \brackets{q} = \{ w | \Delta \brackets{q_0, w} = q \} = L_q$. Из замечаний $\Longrightarrow \, \psi $ -- взаимнооднозначное отображение. Более того, покажем, что $\psi$ — изоморфизм между автоматами, как графами. Для этого нужно показать, что:

\begin{enumerate}
    \item $\Delta \brackets{\psi(q), a} = [\psi\brackets{q}a]$;
    \item $q \in F \Longleftrightarrow \psi \brackets{q} \subseteq L$.
\end{enumerate}

Покажем, почему выполняется (1).

\begin{center}
    $\psi \brackets{q} = [u]$, $\psi \brackets{q'} = [u']$, $\Delta \brackets{q, a} = q'$
    
    $\Delta \brackets{\psi(q), a} = \Delta \brackets{\{ w | \Delta(q_0, w) = q \}, a} = \{ w' = wa | \langle q_0, w' \rangle = q'\} = [wa]$
    
    $[u] = [w]$, $[u'] = [wa]$ по транзитивности переходов в автомате
\end{center}

Покажем, почему выполняется (2). Из того, что $q \in F$, следует, что слова из множества $\psi \brackets{q} = \{ w | \Delta \brackets{q_0, w} = q \}$ принадлежат языку $L$, так они распознаются автоматом, поскольку $q$ является завершающим состоянием. А так как $\psi \brackets{q} = \{ w | \Delta \brackets{q_0, w} = q \} \subseteq L$, то так как они распознаются автоматом, соответствующему языку $L$, то $q \in F$.

Пусть $\psi_1$ — изоморфизм между минимальными ПДКА $M_1$ и $M_0$, $\psi_2$ — изоморфизм между минимальными ПДКА $M_2$ и $M_0$. Тогда $M_1$ и $M_2$ изоморфны между собой — этому соответствует изоморфизм $\psi_2^{-1} \circ \psi_1$, композиция изоморфизмов является изоморфизмом.
\EndProof