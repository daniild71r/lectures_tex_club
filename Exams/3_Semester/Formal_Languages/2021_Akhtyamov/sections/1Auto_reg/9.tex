\subsection{9. Алгоритм проверки равенства регулярных выражений. Теорема Майхилла-Нероуда.}
\textbf{Лемма (из билета 6):} Для любых $u \in \Sigma^*$, $a \in \Sigma$ верно, что $\Delta \brackets{[u], a} = [ua]$

\Proof
 $u \sim_L v \Longrightarrow \forall w' \brackets{uw' \in L \Longleftrightarrow vw' \in L} \Longrightarrow \forall w \brackets{uaw \in L \Longleftrightarrow vaw \in L} \Longleftrightarrow ua \sim_L va$ \EndProof
\\

\Lemma Для любых $u, v \in \Sigma^*$ верно, что $\Delta \brackets{[u], v} = [uv]$
\newline \Proof
Индукция по $|v|$. База:

$v = \varepsilon \Longrightarrow \Delta([u], \varepsilon) = [u] = [u\varepsilon]$

$v = a \Longrightarrow \Delta([u], a) = [ua]$ — по предыдущей лемме.

Шаг: $v = v'a$
$$\Delta([u], v'a) = \Delta(\Delta([u], v'), a) = \Delta([uv'], a) = [uv'a] = [uv]$$
\EndProof\\ 


\textbf{Теорема Майхилла-Нероуда:} Язык $L$ является автоматным тогда и только тогда, когда $\Sigma^* /_{\sim_L}$ содержит конечное количество классов эквивалентности.

\Proof

$\Longrightarrow$ Так как $L$ является автоматным. то для него существует минимальный ПДКА, $\arrowvert \Sigma^* /_{\sim_L} \arrowvert = \arrowvert Q \arrowvert$, Q - конечно

$\Longleftarrow$ Множество $\Sigma^* /_{\sim_L}$ конечно. Построим канонический ПДКА $M_0 = \langle Q, \Sigma, \Delta, [\varepsilon], F \rangle$, где $Q=\Sigma^* /_{\sim_L}, \ F = \{ [w] \;|\; w \in L \}$, $\Delta = \{\angles{[w], a} \rightarrow [wa] \;|\; w\in\Sigma^*, \, a\in\Sigma\}$. Докажем, что $L(M_0) = L$.
$$
w\in L(M_0) \Longleftrightarrow \Delta([\varepsilon],w)\in F \Longleftrightarrow [w] \in F \Longleftrightarrow \exists u \in L \, (w \sim_L u) \Longleftrightarrow w\in L.
$$
Заметим, что если $u \in L$, $w\notin L$ то $u \nsim w$. Автомат построен. \EndProof

\subsubsection*{Алгоритм проверки регулярных выражений на равенство}

Пусть $R_1$ и $R_2$ — регулярные выражения. Построим по ним минимальные ПДКА $M_1$ и $M_2$ соответственно. По теореме о существовании и единственности минимального ПДКА для автоматного языка $L \brackets{M}$ минимальный ПДКА единственен с точности до изоморфизма. Если $M_1$ и $M_2$ изоморфны, то регулярные выражения равны, иначе нет.