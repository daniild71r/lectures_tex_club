\subsection{18. Совпадение классов КС-языков и языков, распознаваемых МП-автоматами: построение грамматики по автомату.}
\textbf{Automaton $\Longrightarrow$ Gram}

\Proof Пусть $w \in L \brackets{M}$, $M$ — МП-автомат, для которого выполнено соотношение:

\begin{center}
    $\forall \brackets{\langle q_1, u, \alpha \rangle \rightarrow \langle q_2, \beta \rangle} \in \Delta : |u| \leqslant 1, |\alpha| + |\beta| = 1$
\end{center}

Для вывода построим график «длина стека» от префикса $w$:

\begin{center}
    $\langle q_0, w, \varepsilon \rangle \vdash \langle q, u, \gamma \rangle$
\end{center}

На графике — точка $\brackets{|\gamma|, |w| - |u|}$.

\drawsomebig{а4.PNG}

Далее зададим грамматику $G = \langle N, \Sigma, P, S \rangle$, где:

\begin{center}
    $N = S \cup \{ A_{ij} | q_i, q_j \in Q \}$
    \textit{Под $A_{ij}$ подразумевается то, что выводится на пути между $q_i$ и $q_j$ без изменения стека (нет pop()).}
    $P$ — объединение следующих множеств:
    \begin{enumerate}
        \item $\{ A_{ii} \rightarrow \varepsilon | q_i \in Q \}$
        \item $\{ S \rightarrow A_{0j} | q_j \in F \}$
        \item $\{A_{ij} \rightarrow aA_{rs}bA_{tj}\ | \alpha \wedge \beta\}$, где:
        \begin{enumerate}
            \item $\alpha$: условие, что $\langle q_i, a, \varepsilon \rangle \rightarrow \langle q_r, A \rangle \in \Delta$;
            \item $\beta$: условие, что $\langle q_s, b, A \rangle \rightarrow \langle q_t, \varepsilon \rangle \in \Delta$.
        \end{enumerate}
    \end{enumerate}
\end{center}

\drawsomebig{f5.PNG}

Теперь нужно доказать следующую лемму:

\Lemma $A_{ij} \vdash_G w \Longleftrightarrow \langle q_i, w, \varepsilon \rangle \vdash_M \langle q_j, \varepsilon, \varepsilon \rangle$

$\Longrightarrow$ Докажем индукцией по длине вывода в грамматике.

\textbf{База.} Вывод за один шаг. $A_{ij} \rightarrow \varepsilon$. Тогда $i = j$, $w = \varepsilon$, $\langle q_i, \varepsilon, \varepsilon \rangle \vdash \langle q_i, \varepsilon, \varepsilon \rangle$.

\textbf{Переход.} Пусть $A_{ij} \vdash w$ за $k$ шагов. Тогда $A_{ij} \vdash aA_{rs}bA_{tj}$, при этом:

\begin{center}
    $\langle q_i, a, \varepsilon \rangle \rightarrow \langle a_r, A \rangle$ $\brackets{1}$
    
    $\langle q_s, b, A \rangle \rightarrow \langle q_t, \varepsilon \rangle$ $\brackets{3}$
\end{center}

Слово $w$ имеет вид $aubv$. Тогда, используя предположение индукции, можем получить:

\begin{center}
    $A_{rs} \vdash u \Longrightarrow \langle q_r, u, \varepsilon \rangle \overset{\brackets{2}}{\vdash} \langle q_s, \varepsilon, \varepsilon \rangle$
    
    $A_{tj} \vdash v \Longrightarrow \langle q_t, v, \varepsilon \rangle \overset{\brackets{4}}{\vdash} \langle q_j, \varepsilon, \varepsilon \rangle$
\end{center}


Тогда $\langle q_i, aubv, \varepsilon \rangle \overset{\brackets{1}}{\vdash} \langle q_r, ubv, A \rangle \overset{\brackets{2}}{\vdash} \langle q_s, bv, A \rangle \overset{\brackets{3}}{\vdash} \langle q_t, v, \varepsilon \rangle \overset{\brackets{4}}{\vdash} \langle q_j, \varepsilon, \varepsilon \rangle$.

$\Longleftarrow$ Проведём индукцию по количеству переходов $k$, которые необходимы для того, чтобы $\langle q_i, w, \varepsilon \rangle \vdash \langle q_j, \varepsilon, \varepsilon \rangle$. 

\textbf{База.} $k = 0$. Тогда $\langle q_i, w, \varepsilon \rangle \vdash_0 \langle q_j, \varepsilon, \varepsilon \rangle$, откуда $w = \varepsilon$ и $q_i = q_j$, $A_{ij} = A_{ii}$, так как есть правило $A_{ii} \rightarrow \varepsilon$, то $A_{ii} \vdash \varepsilon$.

\textbf{Переход.} $\langle q_i, w, \varepsilon \rangle \vdash_k \langle q_j, \varepsilon, \varepsilon \rangle$. Так как стек пустой, то:

\begin{center}
    $\langle q_i, w, \varepsilon \rangle \vdash_1 \langle q_r, u, A \rangle$
\end{center}

$A$ существует, так как мы либо кладём, либо снимаем со стека. Пусть $q_s \rightarrow q_t$ — это момент, когда $A$ снят со стека. Тогда:

\begin{center}
    $\langle q_s, v, A \rangle \vdash_1 \langle q_t, x, \varepsilon \rangle \vdash \langle q_j, \varepsilon, \varepsilon \rangle$
    
    $\langle q_i, a, \varepsilon \rangle \rightarrow \langle q_r, A \rangle \in \Delta : w = au$
    
    $\exists u' : u = u'v$, $\langle q_r, u', \varepsilon \rangle \vdash \langle q_s. \varepsilon, \varepsilon \rangle$
\end{center}

Пользуясь предположением индукции, получаем:

\begin{center}
    $A_{rs} \vdash u'$
    
    $\langle q_s, b, A \rangle \vdash \langle q_t, \varepsilon \rangle \in \Delta : v = bx$
    
    $\langle q_t, x, \varepsilon \rangle \vdash \langle q_j, \varepsilon, \varepsilon \rangle \Longrightarrow A_{tj} \vdash x$
\end{center}

Так как правило $A_{ij} \rightarrow aA_{rs}bA_{tj} \in P$, то $A_{ij} \vdash aA_{rs}bA_{tj} \vdash au'bx = au'v = au = w$. Предположение доказано.

\underline{Из леммы будет следовать теорема следующим образом:}

\begin{center}
    $w \in L \brackets{G} \Longleftrightarrow S \vdash w \Longleftrightarrow \exists A_{0j} \brackets{q_j \in F} : S \vdash_1 A_{0j} \vdash w \Longleftrightarrow \text{ (по лемме) } \newline \exists A_{0j} \brackets{q_j \in F} : \langle q_0, w, \varepsilon \rangle \vdash \langle q_j, \varepsilon, \varepsilon \rangle \Longleftrightarrow \exists q_j \in F : \langle q_0, w, \varepsilon \rangle \vdash \langle q_j, \varepsilon, \varepsilon \rangle \Longleftrightarrow w \in L \brackets{M}$
\end{center}

$\square$