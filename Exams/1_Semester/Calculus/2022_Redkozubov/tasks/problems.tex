\begin{enumerate}
    \item Множество действительный чисел $\R$ как полное упорядоченное поле. Точные грани числовых множеств. Принцип полноты Вейерштрасса. Аксиома Архимеда.
    \item Предел числовой последовательности: единственность, ограниченность. Переход к пределу в неравенствах. Теорема о зажатой последовательности. Арифметические операции с пределами. Бесконечные пределы. Теорема о пределе монотонной последовательности. Теорема Кантора о вложенных отрезках. Теорема Больцано--Вейерштрасса. Частичные пределы, теорема о верхнем и нижнем пределах. Фундаментальные последовательности и критерий Коши.
    \item Внутренность, внешность и граница подмножества $\R$. Открытые и замкнутые множества и их свойства. Предельные точки множества, критерии замкнутости. Замкнутость множества частичных пределов. Замыкание множества. Лемма Гейне--Бореля.
    \item Определения предела функции в точке по Коши и по Гейне, их равносильность. Свойства пределов функции. Предел композиции. Критерий Коши существования предела функции. Односторонние пределы функции, теорема об односторонних пределах монотонной функции.
    \item Непрерывность функции в точке. Равносильные определения непрерывности. Теорема о непрерывности композиции. Точки разрыва, их классификация. Теорема о разрывах монотонной функции.
    \item Ограниченность непрерывной на отрезке функции. Теорема Вейерштрасса о достижимости точных граней. Теорема о промежуточных значениях непрерывной функции. Непрерыность монотонной функции, отображающей промежуток на промежуток. Теорема об обратной функции. Равномерная непрерывность. Теорема Кантора о равномерной непрерывности. Экспонента и ее свойства. Второй замечательный предел. Сравнение асимптотического поведения функций, $O$--символика.
    \item Определение и геометрический смысл производной. Линейная аппроксимация и дифференциал функции. Непрерывность дифференцируемой функции. Односторонние производные. Производная суммы, произведения и частного. Производная композиции. Инвариантность формы дифференциала относительно замены переменного. Производная обратной функции. Таблица производных основных элементарных функций.
    \item Локальные экстремумы. Необходимое условие экстремума дифференцируемой функции. Теорема Ролля. Теоремы Лагранжа и Коши о среднем значении. Теорема Дарбу о промежуточных значениях производной. Условия монотонности и постоянства дифференцируемой функции. Достаточное условие экстремума функции в терминах первой производной. Правило Лопиталя для раскрытия неопределенностей вида $\frac{0}{0}$. Правило Лопиталя для раскрытия неопределенностей вида $\frac{\infty}{\infty}$.
    \item Производные высших порядков. Формула Лейбница для $n$--ой производной произведения функций. Формула Тейлора с остаточным членом в форме Пеано. Разложения функций $e^{x}$, $\sin x$, $\cos x$, $\ln(1+x)$ и $(1+x)^{\alpha}$. Достаточное условия локального экстремума в терминах высших производных. Формула Тейлора с остаточным членом в форме Лагранжа. Выпуклые функции. Дифференциальные условия выпуклости. Неравенство Йенсена.
    \item Первообразная. Неопределнный интеграл и его свойства. Интегрирование рациональных функций.
    \item Определенный интеграл Римана. Формула Ньютона-Лейбница. Ограниченность интегрируемой функции. Линейность и монотонность интеграла. Верхние и нижние суммы Дарбу и их свойства. Критерий интегрируемости Дарбу. Интегрируемость по подотрезкам. Аддитивность интеграла. Интегрируемость произведения и модуля интегрируемых функций. Интегральная теорема о среднем. Интегрируемость непрерывных функций, монотонных функций и ограниченных функций с конечным числом точек разрыва. Интеграл с переменным верхним пределом: непрерывность и дифференцируемость. Существование первообразной у непрерывной функции. Замена переменной в интеграле. Формула интегрирования по частям.
    \item Евклидово пространство $\R^{m}$. Предел и производная вектор--функции. Теорема Лагранжа для вектор--функций. Параметризованная кривая в $\R^{m}$. Длина кривой. Аддитивность длины кривой. Достаточное условие спрямляемости. Дифференцируемость переменной длины дуги кривой. Натуральная параметризация.
\end{enumerate}