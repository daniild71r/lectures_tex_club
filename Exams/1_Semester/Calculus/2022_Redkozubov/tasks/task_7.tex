\section{7. Определение и геометрический смысл производной. Линейная аппроксимация и дифференциал функции. Непрерывность дифференцируемой функции. Односторонние производные. Производная суммы, произведения и частного. Производная композиции. Инвариантность формы дифференциала относительно замены переменного. Производная обратной функции. Таблица производных основных элементарных функций.}

    Всюду в этом разделе $I \subset \R$ -- невырожденный промежуток числовой прямой.
    
    \begin{definition}
        Пусть $f: I \to \R$, \ $a \in I$. \textit{Производной} функции $f$ в точке $a$ называется следующий предел:
        \[\lim_{x \rightarrow a} \frac{f(x) - f(a)}{x - a}\]
        Обозначается $f'(a)$, $\frac{df(a)}{dx}$.\\
        Если указанный предел конечен, то говорят, что функция $f$ \textit{дифференцируема} в точке $a$.
    \end{definition}
    
    Выражение $\frac{f(x) - f(a)}{x - a}$ называется \textit{разностным отношением}.
    
    Пусть функция $f$ дифференцируема в точке $a$. 
    
    \[l_{\text{секущая}}: y = f(a) + \frac{f(t) - f(a)}{t - a} (x - a)\]
    \[l_{\text{касательная}}: y = f(a) + f'(a)(x - a)\]
    \[k_{\text{секущая}} = \frac{f(t) - f(a)}{t - a} \Rightarrow k_{\text{касательная}} = f'(a) \]
    
    \begin{center}
        \includegraphics[width=0.55\textwidth]{print_1.png}
    \end{center}
    
    \begin{note}{Геометрический смысл производной}\\
        Угловой коэффициент секущей стремится к угловому коэффициенту касательной.
    \end{note}
    
    \begin{theorem}{О линейной аппроксимации}\\
        Пусть $f: I \to \R$, \ $a \in I$.
        Функция $f$ дифференцируема в точке $a$ $\lra$ $\exists A \in \R$:
        \[f(x) = f(a) + A(x - a) + o(x - a), \ x \to a.\]
        В этом случае $A = f'(a)$.
    \end{theorem}
    
    \begin{proof}
        $\Rightarrow$ Пусть $f$ дифференцируема в $a$. Определим функцию $\alpha: I \rightarrow R$, $\alpha(x) = \frac{f(x) - f(a)}{x - a} - f'(a)$ при $ x \neq a$, и $\alpha(a)$ произвольно.
        Тогда $\lim_{x \to a} \alpha(x) = 0$ и $f(x) - f(a) = f'(a)(x - a) + \alpha(x)(x - a)$. Следовательно, выполнимо условие.
        \\
        $\Leftarrow$ Из условия следует, что $A + o(1) = \frac{f(x) - f(a)}{x - a}$. Переходя к пределу при $x \to a$, получаем $\exists \ f'(a) = A$, т.е. $f$ дифференцируема в точке $a$.
    \end{proof}
    
    \begin{corollary}
        Если $f$ дифференцируема в точке $a$, то $f$ непрерывна в $a$.
    \end{corollary}
    
    \begin{proof}
        Переходя к пределу при $x \to a$ в теореме о линейной аппроксимации, получим $\lim_{x \to a} f(x) = f(a)$. Это означает, что $f$ непрерывана в точке $a$.
    \end{proof}
    
    Рассмотрение односторонних пределов приводит к следующему обобщению производной.
    
    \begin{definition}
        Пусть $f: I \to \R$, $a \in I$.
        \\
        $f_{+}'(a) = \lim_{x \to a + 0} \frac{f(x) - f(a)}{x - a}$ называется \textit{правой производной} $f$ в точке $a$.
        \\
        $f_{-}'(a) = \lim_{x \to a - 0} \frac{f(x) - f(a)}{x - a}$ называется \textit{левой производной} $f$ в точке $a$.
    \end{definition}
    
    \begin{note}
        Если $a$ -- внутренняя точка $I$, то $\exists \ f'(a) \lra f_{+}'(a) = f_{-}'(a)$. В этом случае все три предела равны.
        \\
        Если $a$ -- концевая точка $I$, то $f'(a)$ существует одновременно с соответствующей односторонней производной.
    \end{note}
    
    \begin{theorem} \ \\
        Пусть $f, g: I \to \R$, $a \in I$ и $\alpha, \beta \in \R$.
        Если $f$ и $g$ дифференцируемы в точке $a$, то в этой точке дифференцируемы $\alpha f + \beta g$, $f \cdot g$ и при условии $g \neq 0$ на $I$ также $\frac{f}{g}$. При этом 
        \begin{enumerate}
            \item $(\alpha f + \beta g)'(a) = \alpha f'(a) + \beta g'(a)$
            \item $(f \cdot g)'(a) = f'(a)g(a) + f(a)g'(a)$
            \item $(\frac{f}{g})(a) = \frac{f'(a)g(a) - f(a)g'(a)}{g^{2}(a)}$
        \end{enumerate}
    \end{theorem}
    
    \begin{proof}
        \begin{enumerate}
            \item Следует из свойств линейности предела.
            \item \[(f \cdot g)(x) - (f \cdot g)(a) = g(a)(f(x) - f(a)) + f(x)(g(x) - g(a))\]
        \[\lim_{x \to a} \frac{(f \cdot g)(x) - (f \cdot g)(a)}{x - a} = f'(a) g(a) + f(a) g'(a)\]
        -- непрерывна в точке $a$.
            \item Переходя к пределу при $x \to a$, получим
        \[\frac{(\frac{1}{g})(x) - (\frac{1}{g})(a)}{x - a} = \frac{1}{g(a) \cdot g(x)} \cdot \frac{g(a) - g(x)}{x - a},\]
        получим, что $g(x)$ дифференцируема в точке $a$ и $(\frac{1}{g})'(a) = - \frac{g'(a)}{g^{2}(a)}$. Теперь пункт (3) следует из (2).
        \end{enumerate}
    \end{proof}
    
    \begin{theorem}{Производная композиции}\\
        Пусть $I, J$ -- промежутки, $f: I \to J$, $g: J \to \R$. Если $f$ дифференцируема в точке $a \in I$ и $g$ дифференцируема в точке $b = f(a)$, то композиция $g\circ f: I \to \R$ дифференцируема в точке $a$, причем
        \[(g\circ f)'(a) = g'(b) \cdot f'(a)\]
    \end{theorem}
    
    \begin{proof}
        Определим функцию $h: J \to \R$, 
        \[ h(y) =
        \begin{cases}
            \frac{g(y) - g(b)}{y - b},       & \quad y \neq b\\
            g'(b),  & \quad y = b.
        \end{cases}
        \]
        Тогда $h$ непрерывна в точке $b$. Покажем, что $\forall x \in I$, $x \neq a$, выполнено
        \[\frac{(g\circ f)(x) - (g\circ f)(a)}{x - a} = h(f(x)) \cdot \frac{f(x) - f(a)}{x - a}\]
        Если $f(x) = f(a)$, то $0 = 0$. Если $f(x) \neq f(a)$, то равенство следует из того, что
        \[\frac{(g\circ f)(x) - (g\circ f)(a)}{x - a} = \frac{g(f(x)) - g(f(a)))}{f(x) - f(a)} \cdot \frac{f(x) - f(a)}{x - a}\]
        Перейдем к пределу
        \[\lim_{x \to a} \frac{(g\circ f)(x) - (g\circ f)(a)}{x - a} = g'(b)\cdot f'(a).\]
        (т.к. $h$ непрерывна в точке b, то по свойству предела композиции $\lim_{x \to a} h(f(x)) = h(b) = g'(b)$)
    \end{proof}
    
    \begin{theorem}{Производная обратной функции}\\
        Пусть $f: I \to \R$ непрерывна и строго монотонна на промежутке $I$. Если f дифференцируема в точке $a \in I$ и $f'(a) \neq 0$, то обратная функция $f^{-1}: f(I) \to I$ дифференцируема в точке $b = f(a)$, причем
        \[(f^{-1})'(b) = \frac{1}{f'(a)}\]
    \end{theorem}
    
    \begin{proof}
        По теореме об обратной функции на $J = f(I)$ определена функция $f^{-1}$, которая на $J$ непрерывна и строго монотонна. Следовательно, $f^{-1}(t) \to a$ при $t \to b$, $f^{-1}(t) \neq a$ при $t \neq b$
        \[\lim_{t \to b} \frac{f^{-1}(t) - f^{-1}(b)}{t - b} = \lim_{t \to b} \frac{f^{-1}(t)-f^{-1}(b)}{f(f^{-1}(t)) - f(f^{-1}(b))} = \lim_{x \to a}\frac{x - a}{f(x) - f(a)} = \frac{1}{f'(a)}\]
    \end{proof}
    
    Таблица производных.
    
    \begin{enumerate}
        \item $c' = 0$
        \item $(a^{x})' = a^{x} \ln(a),$ при $a > 0, a \neq 1$
        \item $(\log_{a}x)' = \frac{1}{x \ln(a)},$ при $a > 0, a \neq 1$
        \item $(x^{\alpha})' = \alpha \cdot x^{\alpha - 1},$ при $\alpha \in \R$
        \item $(\sin{x})' = \cos{x}$
        \item $(\cos{x})' = - \sin{x}$
        \item $(\tg{x})' = \frac{1}{\cos^{2}{x}}$
        \item $(\ctg x)' = - \frac{1}{\sin^{2}{x}}$
        \item $(\arcsin{x})' = - (\arccos{x})' = \frac{1}{\sqrt{1 - x^{2}}},$ при $x \in (-1, 1)$
        \item $(\arctg {x})' = - (\arcctg{x})' = \frac{1}{1 + x^{2}}$
    \end{enumerate}
    
    \begin{proof}
        \begin{enumerate}
            \item По определению.
            \item По второму замечательному пределу $(e^{x})' = \lim_{t \to x} \frac{e^{t} - e^{x}}{t - x} = e^{x} \cdot \lim_{t \to x} \frac{e^{t - x} - 1}{t - x} = e^{x}.$\\
            $(a^{x})' = (e^{x \ln(a)})' = e^{x \ln(a)} \cdot (x \cdot \ln(a))' = a^{x} \ln(a).$
            \item $y = \log_{a}x \lra x = a^{y}$, по теореме о производной обратной функции получим: $(\log_{a}x)' = \frac{1}{(a^{y})'} = \frac{1}{a^{y} \ln(a)}.$
            \item $\alpha \in \Z \ \ (x^{\alpha})' = \alpha \cdot x^{\alpha - 1}$ -- по определению. \\
            $\alpha \in \R \setminus \Z \ \ (x^{\alpha})' = (e^{\alpha \ln(x)})' = e^{\alpha \ln(x)} \frac{\alpha}{x} = \alpha \cdot x^{\alpha - 1}.$
            \item $(\sin x)' = \lim_{t \to x} \frac{\sin t - \sin x}{t - x} = \lim_{t \to x} \frac{\sin \frac{t - x}{2} \cos \frac{t + x}{2}}{\frac{t - x}{2}} = \cos x.$
            \item $(\cos x)' = (-1)\cdot \cos (\frac{\pi}{2} - x) = - \sin x.$
            \item $x \neq \frac{\pi}{2} + \pi k, \ k \in \Z$\\
            $(\tg x)' = (\frac{\sin x}{\cos x})' = \frac{\cos^{2}x + \sin^{2}x}{\cos^{2}x} = \frac{1}{\cos^{2}x}.$
            \item Аналогично пункту (7).
            \item $y = \arcsin x \lra x = \sin y \ \ y \in (-\frac{\pi}{2};\frac{\pi}{2})$\\
            $(\arcsin x)' = \frac{1}{(\sin y)'} = \frac{1}{\cos y} = \frac{1}{\sqrt{1 - \sin^{2}y}} = \frac{1}{\sqrt{1 - x^{2}}}.$
            \item Аналогично пункту (9).
        \end{enumerate}
    \end{proof}
    
    \begin{definition}
        Пусть $f: I \to \R$, $I$ -- промежуток, дифференцируема в точке $a$. Линейная функция $h \mapsto f'(a)h$, $h \in \R$, называется \textit{дифференциалом} функции $f$ в точке $a$ и обозначается $df_{a}$.
    \end{definition}
    
    \begin{center}
        \includegraphics[width=0.4\textwidth]{none5.png}
    \end{center}
    
    Для функции $x \mapsto x$ функция $dx(h) = 1\cdot h$ в любой точке. Следовательно, $df_{a}(h) = f'(a)dx(h)$. Или в функциональной записи: $df_{a}=f'(a)dx$.
    
    \begin{note}{Инвариантность дифференциала.}\\
        Формула $df_{x} = f'(x)dx$ верна и в случае, когда $x$ -- независимая переменная, и в случае, когда $x = x(t)$.
    \end{note}