\textbf{Аналитическая геометрия на плоскости и в пространстве.}
\begin{enumerate}
    \item Коллинеарные, компланарные векторы. Линейные операции с векторами и их свойства. Линейно зависимые и независимые системы векторов. Базис, координаты вектора в базисе. Описание базисов на плоскости и в пространстве. Действия над векторами в координатах. Связь между линейной зависимостью, коллинеарностью и компланарностью векторов. Изменение координат при замене базиса.
    \item Общая декартова система координат, прямоугольная система координат. Связь между координатами направленного отрезка и координатами его конца и начала, задание отрезка и прямой в декартовой системе координат. Замена декартовой системы координат, формулы перехода.
    \item Скалярное произведение, его свойства, выражение в ортонормированном и произвольном базисе. Формулы для определения расстояния между точками и угла между векторами.
    \item Ориентация на плоскости и в пространстве. Ориентированные площадь и объем (смешанное произведение). Свойства ориентированных площади и объема. Выражение ориентированных площади и объема в произвольном базисе.
    \item Векторное произведение, его свойства, выражение в правом ортонормированном базисе. Критерии коллинеарности и компланарности векторов. Двойное векторное произведение.
    \item Понятние об уравнении множества. Алгебраические линии и поверхности. Пересечение и объединение алгебраических линий (поверхностей). Сохранение порядка при переходе к другой системе координат.
    \item Прямая на плоскости, различные способы задания, их эквивалентность. Формула для расстояния от точки до прямой в прямоугольной системе координат. Условия пересечения и параллельности двух прямых. Пучок прямых.
    \item Плоскость в пространстве, различные способы задания, их эквивалентность. Условие параллельности двух плоскостей. Направляющий вектор пересечения двух плоскостей. Пучок плоскостей.
    \item Прямая в пространстве, различные способы задания, их эквивалентность. Формулы для расстояния от точки до плоскости и расстояния между скрещивающимися прямыми в прямоугольной системе координат.
    \item Эллипс, гипербола, парабола, их канонические уравнения. Теоремы о фокусах и директрисах. Асимптоты гиперболы. Сопряжённые диаметры.
    \item Вывод общего уравнения касательной к кривой второго порядка. Касательные к эллипсу, параболе и гиперболе.
    \item Классификация линий второго порядка. Приведение уравнения второго порядка с двумя переменными к каноническому виду в прямоугольной системе координат. Центр кривой второго порядка.
    \item Инварианты кривой второго порядка.
\end{enumerate}

\textbf{Линейные пространства. Матрицы и определители.}
\begin{enumerate}
    \item Матрицы, операции с матрицами, их свойства.
    \item Понятия группы, кольца и поля, примеры. Поле комплексных чисел. Характеристика поля, простое подполе. Группа перестановок, знак подстановки. Изоморфизм групп, теорема Кэли. Порядок элемента. Циклические группы, теорема об изоморфизме циклических групп, подгруппы циклических групп. Теорема Лагранжа о порядке подгруппы, её следствия.
    \item Поле комплексных чисел. Модуль и аргумент комплексного числа.
    \item Линейное пространство. Понятие линейно (не)зависимой системы векторов. Подпространство линейного пространства. Линейная оболочка системы векторов, её характеризация.
    \item Системы линейных уравнений. Элементарные преобразования строк и столбцов матрицы, элементарные матрицы, их свойства. Приведение матрицы к ступенчатому и упрощенному виду. Метод Гаусса решения систем линейных уравнений. Основная лемма о линейной зависимости. Фундаментальная система решений и общее решение однородной системы линейных уравнений. Общее решение неоднородной системы.
    \item Базис и размерность линейного пространства, их свойства. Теорема об изоморфизме. Дополнение линейно независимой системы векторов до базиса. Координаты вектора в базисе, запись операций над векторами через координаты. Изменение координат вектора при изменении базиса. Матрица перехода. Мощность конечного векторного пространства и конечного поля.
    \item Ранг системы векторов, его связь с размерностью линейной оболочки. Ранг матрицы. Теорема о ранге матрицы. Ранг произведения матриц. Теорема о базисном миноре. Нахождение ранга с помощью элементарных преобразований. Теорема Кронекера-Капелли. Невырожденные и обратимые матрицы. Нахождение обратной матрицы при помощи элементарных преобразований.
    \item Подпространства в линейном пространстве. Сумма и пересечение подпространств. Прямая сумма подпространств, её характеризации. Прямое дополнение подпространства. Связь размерностей суммы и пересечения подпространств.
    \item Линейные функции (функционалы). Сопряжённое (двойственное) пространство, его размерность. Взаимный (биортогональный) базис, координаты в нём, замена координат при замене базиса. Канонический изоморфизм пространства и дважды сопряжённого к нему. Аннуляторные подпространства, их свойства.
    \item Линейные отображения и линейные преобразования (операторы) линейного пространства. Их матрицы. Ядро и образ линейного отображения, их размерности. Критерий инъективности линейного отображения. Операции над линейными преобразованиями и их матрицами. Изменение матрицы линейного отображения и линейного преобразования при замене базисов.
    \item Полилинейные и кососимметрические функции. Определитель матрицы, задание определителя его свойствами, явное выражение определителя через элементы матрицы. Поведение определителя при элементарных преобразованиях. Определитель произведения матриц и транспонированной матрицы. Определитель с углом нулей.
    \item Миноры и их алгебраические дополнения. Теорема о произведении минора на его алгебраическое дополнение. Теорема Лапласа. Разложение определителя по строке, столбцу. Определитель Вандермонда. Теорема Крамера. Формула обратной матрицы.
\end{enumerate}