\section{24. Ортогональные преобразования и их свойства. Канонический вид ортогонального преобразования. Инвариантные подпространства малых размерностей для линейного оператора в действительном линейном пространстве.}

\begin{definition}
	Оператор $\phi \in \mathcal{L}(V)$ называется \textit{ортогональным} (\textit{унитарным}), если $\forall \overline{u}, \overline{v} \in V: (\phi(\overline{u}), \phi(\overline{v})) = (\overline{u}, \overline{v})$.
\end{definition}

\begin{theorem}
	Пусть $\phi \in \mathcal{L}(V)$. Тогда оператор $\phi$ ортогонален (унитарен) $\hm\Leftrightarrow$ $\phi$ обратим и $\phi^{-1} = \phi^{*}$.
\end{theorem}

\begin{proof}
	По определению, $\phi$ ортогональнен (унитарен) $\hm{\Leftrightarrow}$ для любых векторов $\overline{u}, \overline{v} \in V$ выполнено $(\overline{u}, \overline{v}) = (\phi(\overline{u}), \phi(\overline{v})) = (\overline{u}, (\phi^*\phi)(\overline{v}))$. В силу единственности сопряженного оператора, это равносильно равенству $\phi^*\phi = \id^* = \id$. Это, в свою очередь, равносильно тому, что $\phi$ обратим и $\phi^{-1} \hm{=} \phi^{*}$.
\end{proof}

\begin{proposition}
	Пусть $\phi\in \mathcal{L}(V)$ "--- ортогональный (унитарный), $U \le V$. Тогда $U$ инвариантно относительно $\phi$ $\Leftrightarrow$ $U^\perp$ инвариантно относительно $\phi$.
\end{proposition}

\begin{proof}
	Поскольку $(U^\perp)^\perp = U$, то достаточно доказать импликацию $\ra$. Так как $U$ инвариантно относительно $\phi$, то $U^\perp$ инвариантно относительно $\phi^* = \phi^{-1}$, то есть $\phi^{-1}(U^\perp) \le U^\perp$. Но оператор $\phi$ биективен, поэтому $\phi^{-1}(U^\perp) = U^\perp$ и $\phi(U^\perp) = U^\perp$, откуда $U^\perp$ инвариантно относительно $\phi$.
\end{proof}

\begin{theorem}[о каноническом виде ортогонального оператора]
    Пусть $V$ - евклидово пространство, $\phi: V \to V$ -- ортогональный оператор. Тогда существует ортонормированный базис $e$, в котором матрица $\phi$ состоит из матриц поворота и единиц на главной диагонали.
     \[\phi = \left(\begin{array}{@{}cccc@{}}
		\cline{1-1}
		\multicolumn{1}{|c|}{R(\alpha_1)} & 0 & \dots & 0\\
		\cline{1-2}
		0 & \multicolumn{1}{|c|}{R(\alpha_2)} & \dots & 0\\
		\cline{2-2}
		\vdots & \vdots & \ddots & \vdots\\
		0 & 0 & \dots & 1\\
	\end{array}\right),\]
\end{theorem}

\begin{proof}
    $\phi$ имеет в $V$ одномерные или двумерные инвариантные подпространства. Пусть $U$ - 
    одномерное подпространство, или, если таких нет, двумерное инвариантное подпространство.
    \begin{enumerate}
        \item Пусть $\dim U = 1$, $e \in U$, $|e|= 1$. Покажем, что в таком случае модуль $\lambda$ равен единице. В $U$ верно $\phi(e) = \lambda e$. Тогда $(e, e) = (\phi(e), \phi(e)) = 
        \lambda^2 (e, e)$. Отсюда $\lambda^2 = 1$, а значит $\lambda = \pm 1$.
        \item Пусть $\dim V = 2$, $(e_1, e_2)$ - ортонормированный базис в $U$. Тогда $A^T A = E$. Найдем вид $A$.
        Пусть $ A = \begin{pmatrix}
                        a     & b \\
                        c     & d        
                    \end{pmatrix}$. Тогда:
        \begin{gather*}
            \begin{pmatrix}
                a     & c \\
                b     & d        
            \end{pmatrix} \begin{pmatrix}
                a     & b \\
                c     & d        
            \end{pmatrix} = \begin{pmatrix}
                1     & 0 \\
                0     & 1        
            \end{pmatrix}
        \end{gather*}
        Получим следующую систему уравнений:
        \begin{gather*}
            a^2 + c^2 = 1 (1) \\
            b^2 + d^2 = 1 (2) \\
            ab + cd = 0 (3) \\
        \end{gather*}
        Положим 
        \begin{gather*}
            a = \cos(\alpha), c = \sin(\alpha), b = - \sin(\beta), d = \cos(\beta)
        \end{gather*}
        Условия $(1)$ и $(2)$ очевидно выполняются. Проверим $(3)$ и найдем при помощи него связь 
        между углами $\alpha$ и $\beta$.
        \begin{gather*}
            -\cos(\alpha) \sin(\beta) + \sin(\alpha) \cos(\beta) = 0 \\
            \sin(\alpha - \beta) = 0 \hookrightarrow \alpha - \beta = \pi k, k \in Z
        \end{gather*}
        Рассмотрим случаи:
        \begin{enumerate}
            \item $\alpha = \beta$ -- по модулю $2\pi$.
            \item $\alpha = \beta + \pi$ -- по модулю $2\pi$.
            \item Покажем, что $\alpha + \beta = \pi$ быть не может:
            \begin{gather*}
                \cos(\beta) = \cos(\alpha - \pi) = -\cos(\alpha) \\
                \sin(\beta) = \sin(\alpha - \pi) = -\sin(\alpha) \Mapsto
                A = \begin{pmatrix}
                        \cos(\alpha)     & \sin(\alpha) \\
                        \sin(\alpha)     & -\cos(\alpha)       
                    \end{pmatrix}
            \end{gather*}
            Где $A^T = A$ и получаем два собственных вектора: $v_1 = (\cos(\frac{\alpha}{2}), \sin(\frac{\alpha}{2}))^T$, $v_{-1} = (-\sin(\frac{\alpha}{2}), \cos(\frac{\alpha}{2}))^T$ -- это противоречит с тем, что нет одномерных инвариантных подпространств.
        \end{enumerate}
        Теперь пространство $V$ раскладывается в прямую сумму $V = U \oplus U^{\perp}$. По предположению индукции для ортогонального дополнения $U$ теорема верна. Тогда она верна и для всего $V$.
    \end{enumerate}
\end{proof}

\begin{proposition}
	Пусть $V$ "--- линейное пространство над $\mathbb{R}$, $\dim{V} \ge 1$, $\phi \in \mathcal{L}(V)$. Тогда у $\phi$ существует одномерное или двумерное инвариантное подпространство.
\end{proposition}

\begin{proof}
	По основной теореме алгебры, минимальный многочлен $\mu_\phi$ имеет следующий вид:
	\[\mu_\phi(x) = \prod_{i = 1}^k(x - \alpha_i)\prod_{j = 1}^m(x^2 + \beta_jx + \gamma_j)\]
	
	Поскольку $\mu_\phi(\phi) = 0$, то хотя бы один из операторов $\phi - \alpha_i$, $\phi^2 + \beta_j\phi + \gamma_j$ "--- вырожденный. Более того, все они вырожденные в силу минимальности многочлена $\mu_\phi$. Значит, возможны два случая:
	\begin{enumerate}
		\item Если $\phi - \alpha$ "--- вырожденный, то $\exists \overline{v} \in V$, $\overline{v} \ne \overline{0}$ "--- собственный вектор с собственным значением $\alpha$, и $\langle\overline{v}\rangle \le V$ "--- искомое подпространство.
		\item Если $\phi^2 + \beta\phi + \gamma$ "--- вырожденный, то $\exists \overline{v} \in V$, $\overline{v} \ne \overline{0}: (\phi^2 + \beta\phi \hm{+} \gamma)(\overline{v}) \hm{=} \overline{0}$. Поскольку $\phi^2(\overline{v}) = -\beta\phi(\overline{v}) - \gamma\overline{v}$, то $\langle\overline{v}, \phi(\overline{v})\rangle \hm{\le} V$ "--- искомое подпространство.\qedhere
	\end{enumerate}
\end{proof}