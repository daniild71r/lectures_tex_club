\section{19. Евклидово и эрмитово пространство. Выражение скалярного произведения в координатах. Матрица Грама системы векторов и ее свойства. Неравенства Коши-Буняковского и треугольника.}

\begin{definition}
    Пусть $V$ - линейное пространство над полем действительных чисел. $V$ называется Евклидовым, если 
    на нем определена положительно определенная билинейная симметрическая функция $f(x, y)$. По 
    определению $f(x, y)$ назвается скалярным произведением и обозначается $(x, y)$.
\end{definition}

\begin{definition}
    Пусть $V$ - линейное пространство над $\Cm$. $V$ называется эрмитовым, если на $V$ определена 
    положительно определенная эрмитова полуторалинейная функция $f(x, y)$. Аналогично с евклидовыми 
    пространствами $f(x, y)$ называется скалярным произведением и обозначается $(x, y)$.
\end{definition}

\begin{definition}
    Матрицей Грама системы $a_1, a_2, \dots a_k$ называется матрица 
    \begin{gather*}
        \Gamma(a_1, \dots a_n) = \begin{pmatrix}
        (a_1, a_1)      & (a_1, a_2)      & \dots  & (a_1, a_n)       \\
        (a_2, a_1)      & (a_2, a_2)      & \dots  & (a_2, a_n)       \\
        \vdots & \vdots & \ddots & \vdots   \\
        (a_n, a_1)      & (a_n, a_2)      & \dots  & (a_n, a_n)
        \end{pmatrix}
    \end{gather*}
\end{definition}

\begin{theorem}
    \begin{enumerate}
        \item
        Пусть $e_1, e_2, \dots e_n$ - базис в $V$, $\Gamma = \Gamma(e)$. Тогда $\forall x, y \in V$ верно 
        $(x, y) = x^T \Gamma \tilde{y}$
        \item Пусть $a_1, a_2, \dots a_k$ - произвольная система векторов в $V$. Тогда $|\Gamma(a_1, \dots a_n)| \geq 0$, 
        прричем равенство достигается тогда и только тогда, когда система линейно зависима.
    \end{enumerate}
\end{theorem}

\begin{proof}
    \begin{enumerate}
        \item $f(x, y) = x^T A \tilde{y} = x^T \Gamma \tilde{y}$.
        \item Пусть система линейно независима. Тогда $U = \langle a_1, a_2, \dots a_k \rangle$, 
        $f \vert_{U}$ - положительно определена, а значит по критерию Сильвестра $|\Gamma(a_1, \dots a_n)| > 0$.

        Пусть теперь система линейно зависима и без ограничения общности $a_k = \lambda_1 a_1 + \dots + \lambda_{k-1} a_{k-1}$.
        Тогда нижняя строка будет состоять из нулей в силу того, что $(a_k, a_i) = (\lambda_1 a_1 + \dots + \lambda_{k-1} a_{k-1}, a_i)$
    \end{enumerate}
\end{proof}

\begin{theorem}[Неравенство Коши-Буняковского]
    Пусть $V$ - пространство со скалярным произведением, и пусть $x, y \in V$. Тогда 
    \begin{gather*}
        |(x, y)|^2 \leq (x, x) \cdot (y, y)
    \end{gather*}
\end{theorem}

\begin{proof}
    \begin{enumerate}
        \item Пусть $x$ или $y$ - нулевой вектор, тогда $0 = 0$.
        \item Пусть $x, y \neq 0$ и коллинеарны, то есть $y = \lambda x$. Тогда 
        \begin{gather*}
            |(x, \lambda x)|^2 = |\lambda|^2 |(x, x)|^2 = \lambda \tilde{\lambda} |(x, x)|^2 = 
            (x, x) (y, y)
        \end{gather*}
        \item Пусть $x, y \neq 0$ и неколлинеарны. Тогда система из $x$ и $y$ линейно независима, а значит 
        по теореме 1:
        \begin{gather*}
            0 < |\Gamma(x, y)| = (x, x)(y, y) - (x, y)(y, x) = (x, x)(y, y) - |(x, y)|^2.
        \end{gather*}
    \end{enumerate}
\end{proof}

\begin{corollary}[Неравенство треугольника]
    Для всех $x, y \in V$ верно:
    \begin{gather*}
        |x + y| \leq |x| + |y|
    \end{gather*}
\end{corollary}

\begin{proof}
    Неравенство треугольника эквивалентно неравенству $(x +y, x+y) \leq (x, x) + 2\sqrt{(x, x)(y, y)} + (y, y)$.
    $(x, x) + (y, y) + 2 \Re (x, y)$
\end{proof}
