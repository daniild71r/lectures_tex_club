\setcounter{section}{53}
\section{Матрицы Адамара. (Первая) конструкция Пэли с квадратичными вычетами при $n = p+1, p = 4m+3$.}
\par \textbf{Определение:} Для простого $p$ определим $p \times p$ \textit{матрицу Якобсталя} $Q$ формулой $Q_{jl} = (\frac{j-l}{p})$ (это символ Лежандра).
\par \textbf{I конструкция Пэли:} Пусть $p \equiv 3 \: (\md 4)$. Тогда матрица
$$\left(
\begin{array}{cc}
1 & e^T\\
e &  Q-E_p
\end{array}
\right)$$
где $e$ — столбец из единиц, а $E_p$ — единичная матрица, является матрицей Адамара порядка $p + 1$.
\par $\blacktriangle$ Рассмотрим скалярное произведение строк $a_1$ и $a_2$ матрицы $Q$.
$$\sum_{b=1}^p \left(\frac{a_1-b}{p}\right)\left(\frac{a_2-b}{p}\right)$$
\par Пусть $x=a_1-b, c=a_2-a_1$. Получаем
$$\sum_{x=1}^p \left(\frac{x}{p}\right)\left(\frac{c+x}{p}\right)=\sum_{x=1}^{p-1} \left(\frac{x}{p}\right) \left(\frac{x \cdot x^{-1} (x+c)}{p}\right)=\sum_{x=1}^{p-1} \left(\frac{x}{p}\right)^2\left(\frac{1+x^{-1}c}{p}\right)$$ 
\par При $x \neq 0 \: \left(\frac{x}{p}\right)^2=1$. Положим $y=1+x^{-1}c$. Так как $c \not\equiv 0 (\md \: p)$, то $x^{-1} c$ пробегает все числа $1 \ldots p-1 \Rightarrow y$ пробегает числа $2 \ldots p$. $$\sum_{x \not\equiv 0 (\md p)} \left(\frac{1+x^{-1} c}{p}\right)=\sum_{y \not\equiv 1(\md p)} \left(\frac{y}{p}\right)=0-\left(\frac{1}{p}\right)=-1$$
\par Рассмотрим скалярное произведение строк искомой матрицы. По сравнению со скалярным произведением строк $Q$ добавятся слагаемые $1$, $(-1)\cdot\left(\frac{a_1-a_2}{p}\right)$ и $(-1)\cdot\left(\frac{a_2-a_1}{p}\right)$ (раньше они умножались на нули). Эти символы Лежандра отличаюся в $\left( \frac{-1}{p} \right)$ раз. $\left( \frac{-1}{p} \right)=(-1)^\frac{p-1}{2}=(-1)^\frac{4m+2}{2}=(-1)^{2m+1}=-1 \Rightarrow$ слагаемые с символом Лежандра сократятся $\Rightarrow$ скалярное произведение любых двух строк искомой матрицы равно $(-1)+1=0$. 
\par Очевидно, что если мы рассмотрим скалярное произведение первой строки с любой другой, мы получим 0, так как в $Q$ было поровну единиц и минус единиц (ранее доказывалось, что в $\mathbb{Z}_p$ поровну квадратичных вычетов и невычетов) и у нас добавилась одна единица и одна минус единица. $\Rightarrow$ это матрица Адамара $\blacksquare$ 