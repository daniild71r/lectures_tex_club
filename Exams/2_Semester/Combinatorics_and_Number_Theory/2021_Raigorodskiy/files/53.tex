\setcounter{section}{52}
\section{Матрицы Адамара. Кронекеровское произведение и общая формулировка про $A \cdot B$.}
\par \textbf{Определение:} \textit{Кронекеровским произведением} матриц $A \in M_{a \times a}$ и $B \in M_{b \times b}$ называется матрица $$C=A \otimes B =
\left(
\begin{array}{ccc}
a_{11}B & \ldots & a_{1n}B\\
\vdots & \ddots & \vdots\\
a_{n1}B & \ldots & a_{nn}B
\end{array}
\right) \: (c_{kb+l, pb+q}=a_{kp} b_{lq})$$
\par \textbf{Замечание 1:} Равенство в скобках выполняется в 0-индексации
\par \textbf{Замечание 2:} Иногда определение вводят наоборот, то есть берут матрицу $A$ и умножают ее на элементы из $B$ (так вводилось на лекции), но для наших нужд не важно какое выбирать.
\par \textbf{Утверждение (задача 20.6):} Кронекеровское произведение матриц Адамара является матрицей Адамара.
\par $\blacktriangle$ Рассмотрим две произвольные строки $C$ с номерами $kb+l, pb+q$. Тогда их скалярное произведение равно
$$a_{k1}a_{p1}(b_l, b_q) + a_{k2}a_{p2}(b_l, b_q) + \ldots + a_{ka}a_{pa}(b_l, b_q) = (b_l, b_q)(a_k, a_p)$$
\par Так как мы берем 2 разные строчки, то либо $l \neq q$, либо $k \neq p$. Значит, так как $A, B$ - матрицы Адамара либо $(b_l, b_q)=0$, либо $(a_k, a_p)=0 \Rightarrow (c_{kb+l}, c_{pb+q})=0 \; \forall kb+l \neq pb+q \Rightarrow \quad C$ - матрица Адамара $\blacksquare$