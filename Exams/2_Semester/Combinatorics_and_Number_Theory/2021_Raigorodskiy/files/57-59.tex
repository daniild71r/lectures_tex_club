\setcounter{section}{56}
\section{Распределение простых чисел в натуральном ряде. Функции $\pi(x), \theta(x), \psi(x)$. Теорема о равенстве
нижних и верхних пределов (формулировка). Неравенство $\lambda_2 \geq \lambda_3$.}

\textbf{Постулат Бертрана.} $\forall x \geq 2\quad \exists \text{ простое } p: x < p < 2x$.

\textbf{Асимптотика} $\forall x \quad \exists p : p \in [x; x + O(x^{0,525})]$\\

\textbf{Неравенство Чебышёва} $\exists a, b \in \mathcal{R} : 0 < a < b \text{ (на самом деле, близкие к единице) } : \frac{ax}{ln(x)} \leq \pi(x) \leq \frac{bx}{ln(x)}$

\textbf{Определение} $\pi(x) = \displaystyle\sum_{p \leq x}1$ -- количество простых чисел, не превышающих $x$\\

\textbf{Определение} $\theta(x) = \displaystyle\sum_{p \leq x}ln(p)$\\

\textbf{Определение} $\psi(x) = \displaystyle\sum_{(p, \alpha) : p^{\alpha} \leq x}ln(p)$

\textbf{Теорема о равенстве
нижних и верхних пределов (формулировка)}\\

Введем следующие обозначения:\\

$\lambda_1 = \overline{\lim\limits_{x \to \infty} \frac{\theta(x)}{x}}$\\

$\lambda_2 = \overline{\lim\limits_{x \to \infty} \frac{\psi(x)}{x}}$\\

$\lambda_3 = \overline{\lim\limits_{x \to \infty} \frac{\pi(x)}{x/ln(x)}}$\\

$\mu_1, \mu_2, \mu_3$ -- соответствующие нижние пределы.\\

\textbf{Теорема:} $\lambda_1 = \lambda_2 = \lambda_3, \mu_1 = \mu_2 = \mu_3$

\textbf{Неравенство $\lambda_2 \leq \lambda_3$}\\

Зафиксируем $p$ и $x$. Тогда таких $\alpha$, что $p^{\alpha} < x$, ровно $[log_p x] = [\frac{ln(x)}{ln(p)}]$.\\

Тогда $\psi(x) = \displaystyle\sum_{(p, \alpha) : p^{\alpha} \leq x}ln(p) = \displaystyle\sum_{p \leq x}[\frac{ln(x)}{ln(p)}ln(p) \leq \displaystyle\sum_{(p, \alpha) : p \leq x}ln(x) = ln(x)\displaystyle\sum_{p \leq x}1 = ln(x)\pi(x)$.\\

$\frac{\psi(x)}{x} \leq \frac{\pi(x)ln(x)}{x} = \frac{\pi(x)}{x/ln(x)}$, т.е. $\lambda_2 \leq \lambda_3$\\


\section{Распределение простых чисел в натуральном ряде. Функции $\pi(x), \theta(x), \psi(x)$. Теорема о равенстве
нижних и верхних пределов (формулировка). Неравенство $\lambda_3 \leq \lambda_1$.}

Зафиксируем некоторое $\gamma \in (0;1)$.\\

$\theta(x) = \displaystyle\sum_{p \leq x}ln(p) \geq \displaystyle\sum_{x^\gamma < p \leq x}ln(p) > \displaystyle\sum_{x^{\gamma} < p \leq x}ln(x^{\gamma}) = \gamma ln(x)\displaystyle\sum_{x^\gamma < p \leq x}1 = \gamma ln(x)(\pi(x) - \pi(x^{\gamma})) \geq \gamma ln(x)(\pi(x) - x^{\gamma})$.\\

Получаем неравенство:\\

$\frac{\theta(x)}{x} \geq \gamma(\frac{\pi(x)}{x/ln(x)} - \frac{x^{\gamma}}{x}ln(x))$. Перейдя к верхнему пределу, получим, что $\frac{\theta(x)}{x} \geq \gamma\frac{\pi(x)}{x/ln(x)}$, т.е. $\lambda_1 \geq \gamma\lambda_3 \forall \gamma \in (0;1)$. Значит, $\lambda_1 \geq \lambda_3$, и $\lambda_2 \geq \lambda_1 \geq \lambda_3$.


\section{Порядки(показатели) элементов в системах вычетов. Равенство $ord(g^l) = \frac{ord(g)}{gcd(l, ord(g))}$. Следствие: если есть порядок $k$, то есть порядки и всех делителей $k$.}

\textbf{Порядки(показатели) элементов в системах вычетов}

Рассмотрим систему вычетов по модулю $m$.\\

\textbf{Определение} Пусть $gcd(g, m) = 1$. Тогда показатель $ord(g) = k$ -- минимальное $k > 0, g^k \equiv 1$.\\

Если $gcd(g, m) \neq 1$, то рассматривать $ord(g)$ бессмысленно, т.к. оно равно $\infty$.\\

\textbf{Равенство $ord(g^l) = \frac{ord(g)}{gcd(l, ord(g))}$}\\

Обозначим $ord(g^l)$ за $s$, а $ord(g)$ за $k$. По определению порядка, $s$ -- минимальное натуральное число такое, что $g^{ls} \equiv 1$. Заметим, что т.к. $k$ -- минимальное число такое, что $g^k \equiv 1$, то $k|ls$. Значит, мы ищем минимальное $s$ такое, что $k|ls$, ведь если это верно, то несложно понять, что тогда $s$ -- порядок $g^l$.\\


\textbf{Теперь сформулируем лемму:}\\

Пусть $a, b \in \mathcal{N}, s$ -- минимальное натуральное число, такое что, $b|as$. Тогда $s = \frac{b}{gcd(a, b)}$.

\begin{proof}
$\frac{a}{gcd(a,b)}$ -- целое, поэтому $\frac{ab}{gcd(a,b)}$ $\vdots$ $b$, то есть $\frac{b}{gcd(a, b)} \geq s$.\\

Пусть $a'= \frac{a}{gcd(a, b)}, b' = \frac{b}{gcd(a,b)})$. \\
Тогда т.к. $b|as$, то $b'|a's$, а в силу того, что $gcd(a', b') = 1$, то $b'|s \Rightarrow s \geq b'$. А т.к. $s \leq b'$, то $s = b'$.

\textbf{Следствие: если есть порядок $k$, то есть порядки и всех делителей $k$.}

Пусть существует $ord(k) < \infty \pmod{m}$. Тогда $gcd(k, m) = 1$.\\

Значит, если $a|k$, то $gcd(a, m) = 1$. Тогда рассмотрим $m$ чисел: $a^1, \dotsc, a^{m-1}$. Если среди них все различные, тогда среди них есть $1$, т.к. остатков от деления на $m$, отличных от $0$, ровно $m-1$. В противном случае какие-то два различных числа равны, то есть $a^i \equiv a^{i+t} \pmod{m}$, где $t \neq 0$. Тогда $a^i(a^t - 1) \equiv 0 \pmod{m}$, а т.к. $gcd(a^i, m) = 1$, то $a^t \equiv 1 \pmod{m}$. 
\end{proof}