
\setcounter{section}{28}
\section{Квадратичные иррациональности. Множество Z[$\sqrt{m}$]: сопряжение, замкнутость сложения, умножения. Согласованность сопряжения и умножения. Норма и её свойства.}
\textbf{Опр} Иррациональное число $\overline{\alpha}$ называется \textit{квадратичной иррациональностью}, если $\alpha$ - корень квадратного уравнения с целыми коэффициентами.
\\
\\
\textbf{Опр} Пусть $\alpha = a + b\sqrt{m}$ — квадратичная иррациональность. Назовем число $\alpha = a - b\sqrt{m}$ сопряженным к $\alpha$ числом
\\
\textbf{Утверждение}\\
Множество Z[$\sqrt{m}$] = \{a + b$\sqrt{m}$| a, b $\in Z$ \} $\subset$ R замкнуто относительно операций:
\begin{itemize}
    \item[1] Сопряжения
    \item[2] Сложения
    \item[3] Умножения
\end{itemize}
$\blacktriangle$
\begin{itemize}
    \item[1] a - b$\sqrt{m}$ = a + (-b)$\sqrt{m}$;  a, -b $\in Z \Longrightarrow$ a - b$\sqrt{m} \in $Z[$\sqrt{m}$]
    \item[2] $a_1 + b_1\sqrt{m} + a_2 + b_2\sqrt{m} = (a_1 + a_2) + (b_1 + b_2)\sqrt{m}$; $(a_1 + a_2),\ (b_1 + b_2)$ $\in Z \Longrightarrow$ $a_1 + b_1\sqrt{m} + a_2 + b_2\sqrt{m} \in $Z[$\sqrt{m}$]
    \item[3] $(a_1 + b_1\sqrt{m}) *( a_2 + b_2\sqrt{m}) = (a_1 a_2 + b_1 b_2 m) + (a_1 b_2 + a_2 b_1)\sqrt{m}; (a_1 a_2 + b_1 b_2 m), (a_1 b_2 + a_2 b_1) \in Z \Longrightarrow (a_1 + b_1\sqrt{m}) *( a_2 + b_2\sqrt{m}) \in Z[\sqrt{m}$] $\ \blacksquare$
\end{itemize}

Сопряжённость для квадратичной иррациональности согласована с общим определением. В алгебре сопряженными к элементу $\alpha$ над полем F называются корни неприводимого многочлена f(x) $\in$ F[x],
для которого f($\alpha$) = 0. Это согласовано с определением комплексного сопряжения. А именно, для комплексного числа z $\in$ C \ R его сопряжённое — это второй корень квадратного многочлена, у
которого первый корень — это z.\\
\textbf{Опр}\\
Для $\alpha \in$  Z[$\sqrt{m}$ ] определим норму N($\alpha$) = $\alpha \overline{\alpha}$.
\\
\textbf{Свойства}
\begin{itemize}
    \item[1] $N(\alpha) \in Z \ \blacktriangle \ N(\alpha) =  \alpha \overline{\alpha} = (a + b\sqrt{m})*(a - b\sqrt{m}) = a^2 - b^2m \in Z \ \blacksquare$
    \item[2] $N(\alpha\beta) = N(\alpha)*N(\beta) \ \\ \blacktriangle \ \alpha = a_1 + b_1\sqrt{m}, \ \beta = a_2 + b_1\sqrt{m}. \\ \alpha\beta = (a_1 a_2 + b_1 b_2 m) + (a_1 b_2 + a_2 b_1)\sqrt{m} \\ \overline{\alpha \beta} = (a_1 a_2 + b_1 b_2 m) - (a_1 b_2 + a_2 b_1)\sqrt{m} \\ N(\alpha \beta) = ((a_1 a_2 + b_1 b_2 m) + (a_1 b_2 + a_2 b_1)\sqrt{m}) ((a_1 a_2 + b_1 b_2 m) - (a_1 b_2 + a_2 b_1)\sqrt{m} ) =\\= (a_1 + b_1\sqrt{m})( a_2 + b_2\sqrt{m}) *(a_1 - b_1\sqrt{m})( a_2 - b_2\sqrt{m})  = (a_1 + b_1\sqrt{m})(a_1 - b_1\sqrt{m}) * ( a_2 + b_2\sqrt{m})( a_2 - b_2\sqrt{m}) = N(\alpha)N(\beta) \ \blacksquare $
\end{itemize}


\section{Пара (a, b), где a + b$\sqrt{2}$ = $(1 + \sqrt{2})^n$ является решением уравнения Пелля $a^2 - 2b^2 = \pm1$.}
\textbf{Опр} Уравнение вида $x^2 - my^2 = 1$, где m — натуральное число, не являющееся точным квадратом, называется уравнением Пелля. Решение (1, 0) называется тривиальным. Решение (x, y)
называется положительным, если x $>$ 0 и y $>$ 0.
\\
\\
Определим $a_n$ и $b_n $при помощи равенства $(1 + \sqrt{2})^n$ = $a_n + b_n\sqrt{2}$
\begin{itemize}
    \item [1.] $(1 + \sqrt{2})^n = \sum_{k = 0}^n C_{n}^k (\sqrt{2})^{k}$ \\ $(1 - \sqrt{2})^n = \sum_{k = 0}^n C_{n}^k (-\sqrt{2})^{k}$. При четных k $(-\sqrt{2})^{k} = (\sqrt{2})^{k}  \in N \Longrightarrow (-\sqrt{2})^{k} \in a_n$. При нечетных k $(-\sqrt{2})^{k} = -(\sqrt{2})^{k} \not\in Z \Longrightarrow (-\sqrt{2})^{k} \in -b_n$ \\ Таким образом,  $(1 - \sqrt{2})^n = a_n - b_n\sqrt{2}$
    
    \item[2.] $a_n^2 - 2b_n^2 = (a_n - b_n\sqrt{2})(a_n + b_n\sqrt{2}) = (1 + \sqrt{2})^n(1 - \sqrt{2})^n= (-1)^n$
\end{itemize}
Отсюда заключаем, что такие $a_n$ и $b_n$: $(1 + \sqrt{2})^n$ = $a_n + b_n\sqrt{2}$ являются решениями уравнения Пелля $a^2 - 2b^2 = \pm1$.


\section{Связь между решениями уравнения Пелля $a^2 - 2b^2 = \pm1$ и элементами \texorpdfstring{$Z[\sqrt{2}]$}{Z[sqrt2]} нормой 1.}
\textbf{Утверждение}\\
Любой элемент Z[$\sqrt{2}$] нормы 1 является решением уравнения $a^2 - 2b^2 = 1$, любое решение уравнения $a^2 - 2b^2 = 1$ - элемент Z[$\sqrt{2}$] нормы 1
\\
$\blacksquare$
\begin{itemize}
    \item -> Пусть (a,b) - решение уравнения Пелля $a^2 - 2b^2 = 1$, тогда $(a + b\sqrt{2})(a - b\sqrt{2}) = 1 \Longrightarrow N(a + b\sqrt{2}) = 1 ;  a,b \in Z[\sqrt{2}]$
    \item <- Пусть a, b $\in Z[\sqrt{2}], \ N(a + b\sqrt{2}) = 1 \Longrightarrow (a + b\sqrt{2})(a - \sqrt{2}) = 1 = a^2 - 2b^2 \Longrightarrow (a,b) $ - решение уравнения Пелля $\ \blacksquare$
\end{itemize}
Аналогичное утверждение можно сформулировать для $a^2 - 2b^2 = -1$


\section{Алгебраические и трансцендентные числа. Существование трансцендентных чисел (из соображения мощности). Степень алгебраического числа. Теорема Лиувилля (б/д).}
\textbf{Опр} Число $\alpha$ - алгебраическое, если существует многочлен с целыми коэффициентами, корнем которого является $\alpha$
\\
\\
Обозначим множество алгебраических чисел $A$. Это множество счетно (достаточно занумеровать все многочлены)
\\
\textbf{Опр} $R \backslash A \ (C \backslash A) $ имеет мощность континуум, все числа из этого множества - \textit{трансцендентные числа}
\\
\\
\textbf{Опр} \textit{Степень алгебраического числа} - это минимальная степень уравнения, корнем которого является это число
\\
\\
\textbf{Теорема Лиувилля}
\\ Пусть $\alpha$ -  алгебраическое число степени d, тогда $\exists c = c(\alpha):$ неравенство $|\alpha - \frac{p}{q}| \leq \frac{c}{q^d}$ не имеет решени в $\frac{p}{q}$


\section{Определение решётки (эквивалентность двух определений) и дискретного подмножества. Определитель решётки. Независимость значения определителя от выбора базиса.}
\textbf{Опр} Пусть $( e_1, . . . , e_k)$ — набор линейно независимых векторов в $R^n$. Решётка - абелева группа, порождённая $\{e_i\}$. Иными словами, решётка есть множество $\Lambda = \{a_1 e_1 + ... + a_k e_k\}, a_i \in Z$
\\
\\
\textbf{Эквивалентность}\\
<- Для $\Lambda = \{a_1 e_1 + ... + a_k e_k\}, a_i \in Z$ выполяняются ассоциативность и коммутативность сложения, существует нейтральный по сложению ($\overline{0}$) и к каждому $\overline{a} = a_1 e_1 + ... + a_k e_k$ обратный $-\overline{a} = -a_1 e_1 - ... - a_k e_k$, значит, $\Lambda$ - абелева группа. Причем $\{e_i\}$ - базис\\
-> Любой элемент абелевой группы, порожденной $\{e_i\}$ имеет вид $\overline{a} = a_1 e_1 + ... + a_k e_k$, где $a_i \in Z \Longrightarrow$  $\overline{a} \in \Lambda$
\\
\\
\textbf{Опр} Подмножество X пространства $R^n$ называется дискретным, если для любой точки x $\in$ X существует окрестность этой точки, не содержащая других точек множества X.
\\
\\
\textbf{Опр} Определителем $det \Lambda$ решётки $\Lambda$ называется определитель матрицы, составленной из координат её базисных векторов.( Он равен объёму фундаментального параллелепипеда, то есть
параллелепипеда, составленного из базисных векторов.)
\\
\\
\textbf{Утверждение}
\\
Определитель решетки не зависит от выбора базиса
\\
$\blacktriangle \ $ Пусть А, В - матрицы в разных базисах, S - матрица перехода от А к B. Тогда $B = A*S$. В силу того, что векторы нового безиса - это ЛК векторов старого базиса с какими-то целочисленными коэффициентами, матрица S целочисленная. По этим же соображениям, $S^{-1}$ - целочисленная матрица. Тогда\\
$det B = det A \ det S, det A = det S^{-1} det B \Longrightarrow \frac{1}{det S} = det S^{-1} \Longrightarrow det S^{-1} det S = 1. \Longrightarrow det S = \pm1 \Longrightarrow det A = det B \ \blacksquare $

\section{Определение решётки и его определителя. Решётка $\Lambda_{\overline{a}}$ и её определитель.}
\textbf{Опр} Дано простое число p и зафиксирован вектор $\overline{a}$ = $(\frac{a_1}{p}, ..., \frac{a_n}{p})$, где $a_i \in Z$. Определим множество $\Lambda_{\overline{a}} = \{l\overline{a} + \overline{b}, \ l \in Z, \overline{b} \in Z^n\}$
\\
\textbf{Утверждение }\\
$\Lambda_{\overline{a}}$ - решетка \\
$\blacktriangle \ $Заметим, что все это множество порождается векторами $(\overline{a}, \overline{e_1},\overline{e_2},...,\overline{e_n})$. Покажем, что если убрать из этого набора векторов $\overline{e_1}$, они все равно будут порождать множество $\Lambda_{\overline{a}}$. \\
Заметим, что если все $a_i \vdots p$, то ничего нового мы не получим, т.е. $\overline{a}$ линейно выражается через $(\overline{e_1},...\overline{e_n})$, и мы нашли базис, порождащий это множество, тогда $\Lambda_{\overline{a}}$ - решетка. 
\\
Предположим, какой-то из $a_i \not\vdots p$; Пусть БОО это $a_1$. Научимся из вектора $\gamma = (\frac{1}{p}, ..., \frac{la_n}{p})$ получать вектор $\overline{a} $ и вектор $\overline{e_1}$ . 
\\
Возьмем в качестве $\overline{b} = k\overline{e_1}$, тогда $l\overline{a} + \overline{b} = (\frac{la_1 + kp}{p}, \frac{la_2}{p}, ...., \frac{la_n}{p})$ \\
Заметим, что всегда можно выбрать l и k так, чтобы $la_1 + kp = 1$, т.к. ($a_1, p$) = 1\\
Покажем, что  $(\gamma,e_2,...,e_n)$ образуют базис. Для начала заметим, что $\overline{e_1} = p\gamma - la_2\overline{e_2} - la_3\overline{e_3} -...la_n\overline{e_n}$.  $l\overline{a} = \gamma - \overline{b}$. Мы умеем выражать все базисные векторы и $l\overline{a} \Longrightarrow$  умеем выражать $\overline{a} \Longrightarrow $ нашли базис$\blacksquare$
\\
\textbf{Найдем $det \Lambda_{\overline{a}}$ }\\ Заметим, что матрица, составленная из базисных векторов $\Lambda_{\overline{a}}$ нижняя треугольная, diag($\frac{1}{p}, 1,1,...,1$), исходя из того, какой базис мы нашли. Тогда  $det \Lambda_{\overline{a}} = \frac{1}{p}$