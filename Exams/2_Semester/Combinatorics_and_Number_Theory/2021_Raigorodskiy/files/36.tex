\setcounter{section}{35}
\section{Определение равномерной распределённой последовательности по модулю 1. Является ли р.р. (mod 1) последовательность $a^n$ при a < 1?}

Послед-ность $x_1, x_2, \dots, x_n, \dots$ - \textbf{равномерно распределена по модулю 1}, если:
\[ \forall a, b \in [0, 1] \lim_{N \to \infty} \frac{|\{i = 1, \dots, N: \{x_i\} \in [a, b)\}|}{N} = b - a \]
или, что равносильно (по сути речь про вероятность, что дробная часть числа из первых N окажется на отрезке b - a):
\[ \forall \gamma \in [0, 1] \lim_{N \to \infty} \frac{|\{i = 1, \dots, N: \{x_i\} \leqslant \gamma \}|}{N} = \gamma \] \\
\\
$a^n$ при $a<1$ \textbf{не является} р.р. (mod 1). \par
$\blacktriangle$ Очевидно, что $\{ a^n \} = a^n$. Возьём $\gamma = a + \varepsilon < 1$. Тогда $\forall n$ $a^n < a < \gamma$; для этого $\gamma$: $\lim_{N \to \infty} \frac{|\{i = 1, \dots, N: \{x_i\} \leqslant \gamma \}|}{N} = 1 \neq \gamma$ $\blacksquare$