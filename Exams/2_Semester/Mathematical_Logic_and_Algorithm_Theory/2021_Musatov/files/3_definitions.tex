\section{Вычислимость}

\subsection{Машина Тьюринга}

\textbf{Опр} \textit{Машина Тьюринга}\\ Формальное определние машины Тьюринга - это кортеж ($\Sigma, \Gamma, Q, q_1, q_a, q_r, \delta$), где
\begin{itemize}
    \item [$\Sigma$]
    - конечное непустое множество  - входной алфавит, типично $\{0,1\}$
\item[$\Gamma$] - конечное непустое множество, включающее в себя $\Sigma$, как подмножество, а также, по меньшей мере, еще пустой символ( бланк, пробел) - ленточный алфавит
\item[$Q$] - конечное множество, не пересекающееся с $\Gamma$ - множество внутренних состояний
\item[$q_1$]$\in Q$ - начальное состояние \item[$q_a$]$\in Q$ - принимающее состояние
\item[$q_r$]$\in Q$ - отвергающее состояние
\item[$\delta$] - функция перехода. $\delta: Q\times\Gamma \rightarrow Q\times\Gamma \times\{L, R, N\}$, где L - перемещение влево, R - вправо, N - никуда
\end{itemize}
Для задач с текстовым или числовым ответом вместо $q_r, q_a$ рассматривают одно $q_0$\\
\textbf{Опр} \textit{Конфигурация машины Тьюринга} - данные о содержимом ленты, положении указателя и состоянии упавляющего блока.\\ Начальная конфигурация: на ленте написан вход, машина в состоянии $q$, указывает на первый символ входа. У каждой конфигурации есть однозначно опредляемая следующая. Если состояние завершающее, конфигурация уже не меняется.\\ Иначе производится замена символа, состояния и положения головки.\\
\includegraphics{images/3 (определения)_m31.PNG} AqaB \\ \textit{Вычислением на машине Тьюринга} называется последовательность конфигураций, каждая из которых непосредственно следует из предыдущей по правилам этой машины.\\ \\
\textbf{Пример смены конфигураций}\\
\begin{center}
    \includegraphics[width =17cm]{images/3 (определения)_mmm1.PNG}
\end{center}
q,s - состояния

\subsection{Вычислимая функция}

\textbf{Опр} \textit{Вычислимая функция}

Функция $f: \{0,1\}^* \rightarrow \{0,1\}^*$ называется \textit{вычислимой}, если для некоторой машины Тьюринга выполнено:
\begin{itemize}
    \item [1] Если $f(x)$ определена, то существует вычисление, которое начинается с qx и заканчивается $q_0f(x)$
    \item[2] Если $f(x)$ не определена, то не существует вычисление, которое начинается с qx и заканчивается $q_0f(x)$
\end{itemize}

\textbf{Примеры}

\begin{itemize}
    \item [$\checkmark$] Нигде не определённая функция вычислима (в качестве алгоритма надо взять программу, которая всегда зацикливается).
    \item [$\checkmark$] Пусть $\Sigma$ = \{0, 1\}. Опишите машины, вычисляющие функции: f(x) = x, f(x) = 0,
нигде не определённую.
    \begin{itemize}
        \item [-] $\delta(q_1,0) = (q_0, 0, N)$
        \item [-] $\delta(q_1,1) = (q_0, 1, N)$
        \item [-] $\delta(q_1,\#) = (q_0, \#, N)$
    \end{itemize}
\end{itemize}

\subsection{Разрешимое множество}

\textbf{Опр} \textit{Разрешимое множество}\\
Множество $A \subset \{0,1\}^*$ называется \textit{разрешимым}, если для некоторой машины Тьюринга выполнено:
\begin{itemize}
    \item [1] Если $x \in A$, то существует вычисление на этой машине, которое начинается с $q_1x$ и заканчивается $q_a$
     \item [2] Если $x \in \overline{A}$, то существует вычисление на этой машине, которое начинается с $q_1x$ и заканчивается $q_r$
\end{itemize}

\subsection{Перечислимое множество}

\textbf{Опр} \textit{Перечислимое множество}\\
Будем рассматривать машину, у которой вместо завершающих состояний есть команды печати в поток вывода: печать 0, печать пробела. Результатом работы такой машины будет конечная или бесконечноая цепочка слов, разделенных пробелами.\\
Множество называется \textit{перечислимым}, если существует печатающая машина, такая что:
\begin{itemize}
    \item [] Если $x \in A$, то х встречается в потоке вывода
     \item [] Если $x \not\in A$, то х не встречается в потоке вывода
\end{itemize}
\textbf{Примеры}
\begin{itemize}
    \item [$\checkmark$] Пустое множество является перечислимым 
    \item [$\checkmark$] Область значений/Область определения любой вычислимой функции - перечислимое множество
    \item[$\times$]  $\{n | U(n, x)$ определено при всех x$\}$ - неперечислимо
\end{itemize}

\subsection{Универсальная машина Тьюринга}

\textbf{Опр} \textit{Универсальная машина Тьюринга}\\
Что такое? Это некоторая машина, которая получает на вход описание другой машины и вход для нее, а возвращает результат ее работы.\\
\begin{center}
    $U(<M>,x) = M(x)$
\end{center}

\subsection{Универсальная вычислимая функция}

\textbf{Опр} \textit{Универсальная вычислимая функция.}\\
Функция $u:\{0,1\}^* \times \{0,1\}^* \rightarrow \{0,1\}^*$ называется \textit{универсальной вычислимой функцией}, если:
\begin{itemize}
    \item [1] u вычислима, как функция от двух аргументов
    \item[2] Если $f:\{0,1\}^* \rightarrow \{0,1\}^* $ - вычислимая функция одного аргумента, то $\exists p \forall x \ u(p,x) = f(x)$ 
\end{itemize}

\subsection{Главная универсальная вычислимая функция}

\textbf{Опр} \textit{Главная универсальная вычислимая функция}\\
$U:\mathbb{N}\times\mathbb{N} \rightarrow \mathbb{N}$ - Главная Универсальная Функция, если
\begin{itemize}
    \item[1] U вычислима
     \item[2] U универсальна, т.е. для любой вычислимой $f:\mathbb{N} \rightarrow \mathbb{N} $ найдется p такое, что $\forall x f(x) = U(p,x)$ (говорят, что p - это номер функции f)
     \item[3] U главная, т.е. для любой вычислимой $V: \mathbb{N}\times\mathbb{N} \rightarrow \mathbb{N}$ найдется всюду определенная вычислимая $s:\mathbb{N} \rightarrow \mathbb{N}$, такая что $\forall p \ \forall x \ V(p,x) = U(s(p),x)$
\end{itemize}
Интуитивный смысл: U - универсальный компилятор, V - какой-то вычислимый. Первый аргумент V - \" \ программа\" \ , второй - \" \ данные\" \ , s - \" \ автоматический траснлятор\" \ , переделывающие программу для V в программу для U

\subsection{$m$-сводимость}

\textbf{Опр} \textit{m-сводимость}\\
Говорят, что А m-сводится к В, если существует всюду определенная вычислимая функция $f:\mathbb{N} \rightarrow \mathbb{N}$, такая что $x \in A \Longleftrightarrow f(x) \in B$. Обозначение: $A\leq_m B$

\subsection{Арифметическая иерархия}

\textbf{Опр} \textit{Классы арифметической иерархии}\\
Говорят, что множество $A $ принадлежит классу $\Sigma_n$, если существует такое разрешимое множество $R \in \mathbb{N}^{k+1}$, что  \begin{center}
    $x \in A \Longleftrightarrow \exists y_1 \ \forall y_2 \ \exists y_3 .... \mathcal{Q} y_n [(x, y_1,...,y_k) \in R]$
\\
\end{center} 
Аналогично, говорят, что $A$ принадлежит классу $\Pi_n$, если существует такое разрешимое множество $R \in \mathbb{N}^{k+1}$, что \begin{center}
    $x \in A \Longleftrightarrow \forall y_1 \ \exists y_2 \ \forall y_3 .... \mathcal{Q} y_n [(x, y_1,...,y_k) \in R]$
\end{center} 
Согласно этому определению, $\Sigma_0 = \Pi_0$(классы $\Sigma_0$ и $\Pi_0$ совпадают с классом всех разрешимых множеств)
\\
$\Sigma_1$ - перечислимые, $\Pi_1$ - коперечислимые
\\
$\blacktriangle \ $S перечислимо $\Longleftrightarrow$ для некоторого разрешимого R верно ($x \in S \Longleftrightarrow \exists y \ (x,y) \in R$), Q коперечислимо $\Longleftrightarrow$ для некоторого разрешимого R верно ($x \in S \Longleftrightarrow \forall y \ (x,y) \in R$)\\
\textbf{Примеры}
\begin{itemize}
    \item[1] T - множество всюду определенных функций \\ $p \in T \Longleftrightarrow \forall n \exists k (U(p,n)$ останавливается за k шагов) - разрешимое свойство $\Longrightarrow T \in \Pi_2$
    \item[2] FD - множество функций с конечной областью определения
    
    $p \in FD \Leftrightarrow \exists N \forall n \ \forall k  (n > N \longrightarrow U(p,n)$ останавливается за k шагов) - разрешимое свойство $\Longrightarrow FD \in \Sigma_2$
\end{itemize}

\subsection{$\lambda$-термы, $\alpha$-конверсии, $\beta$-редукции, нормальная форма}

\textbf{Опр} \textit{ $\lambda$-термы}\\
$\lambda$-терм строится по индукции
\begin{itemize}
    \item [1] Переменная является $\lambda$-термом
    \item[2](Операция аппликации): Если M и N суть лямбда-термы, то (MN) - тоже лямбда-терм.
    \\
    Смысл: в функцию M вместо переменное подставляем N
    \item[3](Операция $\lambda$-абстракции): Если M - терм, а x - переменная, то ($\lambda x.M$) - тоже терм \\
    Смысл: выражение M теперь рассматривается как функция от х
\end{itemize}
\textbf{Опр} \textit{$\alpha$-конверсия} \\
$\alpha$-конверсия - это замена связаной переменной. $\lambda x.M \rightarrow \lambda z.M(z/x)$
\\
\textbf{Ограничение: не должно возникать конфликтов имен. Переменные из М не должны попадать под действие $\lambda$-квантора}
\\
\\
\textbf{Пример}
\begin{itemize}
    \item [$\checkmark$] $\lambda x.xy \rightarrow \lambda z.zy$ -  так можно
     \item [$\checkmark$] $\lambda x.xy(\lambda x.xy) \rightarrow \lambda z.zy(\lambda x.xy)$ -  и так можно
     \item[$\times$] $\lambda x.xy \rightarrow \lambda y.yy$ - а вот так нельзя
      \item[$\times$] $\lambda x.x(\lambda y.xy) \rightarrow \lambda y.y(\lambda y.yy)$ - и так тоже нельзя! Тут переменная, полученнная после замены, попала под воздействие уже имеющегося квантора
\end{itemize}
\textbf{Опр} \textit{$\beta$-редукция} \\
Замена аргумента функции на какое-то значение. $(\lambda x.M)N \rightarrow M(N/x)$
\\
\textbf{Ограничение: не должно возникать конфликтов имен. В N не должно быть переменных, по которым стоят $\lambda$-кванторы в М}
\\
\\
\textbf{Пример}
\begin{itemize}
    \item[$\checkmark$] sin x при x = 2 равен sin 2
    \item[$\times $] $(\lambda x.(x \lambda y.xy))y \rightarrow y\lambda y.yy$ - так нельзя
\end{itemize}
\textbf{Опр} \textit{Нормальная форма}\\ Говорят, что терм M находится в \textit{нормальной форме}, если к нему нельзя применить $\beta$-редукцию даже после нескольких $\alpha$-конверсий\\ Говорят, что N - нормальная форма M, если M = N и N находится в нормальной форме.
\\
Не у всех термов есть нормальная форма.\\  \\
\textbf{Пример}\\
$\Omega = (\lambda x.xx)(\lambda x.xx)$

\subsection{Нумералы Чёрча}

\textbf{Опр} \textit{Нумералы Чёрча}

Семантика нумералов Черча.

Формально, число $k$ соответствует преобразованию функции f в ее k-ую итерацию

\begin{center}
    $\overline{0} = fx.x$
    
    $\overline{1} = fx.fx$
    
    $\overline{2} = fx.f(fx)$
    
    $\dots$
    
    $\overline{n} = \lambda fx. f(f.....f(fx))...)$ - $n$ раз $f$
\end{center}

\subsection{Комбинатор неподвижной точки}

\textbf{Опр} \textit{Комбинатор}\\
\textit{Комбинатором} называется замкнутый $\lambda$-терм (без свободных переменных) \\ Говорят, что комбинатор G представляет функцию $g: \mathbb{N}^k \rightarrow \mathbb{N}$, если для любых $n_1, ..., n_k$ выполнено:
\begin{center}
    $G\overline{n_1}\overline{n_2}...\overline{n_k} = \overline{g(n_1,...,n_k)}$. Если g  не определена, то у $G\overline{n_1}\overline{n_2}...\overline{n_k}$ нет нормальной формы
\end{center}
\textbf{Пример}
\begin{itemize}
    \item [1] Inc - прибавление 1. Inc $\overline{n} = \overline{n} + 1$ : \\
    $Inc = \lambda nfx.f(nfx)$
    \item [2] Add - сложение.  Add $\overline{n} \overline{m} = \overline{n + m}: \\
    Add = \lambda mn fx.mf(nfx)$
    \item[3] Mult - умножение:\\
    $Mult = \lambda mn fx. m(nf)x$
     \item[4] Exp - возведение в степень:\\
    $Exp = \lambda mn fx. nmfx$
\end{itemize}
\textbf{Опр} \textit{Комбинатор неподвижной точки} \\
Y - комбинатор неподвижнй точки, если для любого F верно YF = F(YF)


\textbf{Пример}
\includegraphics{images/3 (определения)_m32.PNG}