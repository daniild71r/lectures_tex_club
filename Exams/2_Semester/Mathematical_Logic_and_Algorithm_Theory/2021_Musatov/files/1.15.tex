\subsection{Построение программы, отличие которой ни от какой другой нельзя доказать в арифметике Пеано.}

\begin{definition}
    Две программы называются \textit{различно доказуемыми}, если можно формально доказать (например, в формальной арифметике), что существует вход, на котором они дают разные результаты.
\end{definition}

\begin{theorem}
    Можно построить программу $p$ со свойством "\textit{никакая программа $q$ не является доказуемо различной с $p$}"\\
    Программа $p$ является непознаваемой, так как если мы докажем любое свойство этой программы, то можно было бы отличить $p$ от программы $q$, не обладающей этим свойством.\\
    
    \begin{proof}
        Существование непознаваемой программы $p$ следует из теоремы о неподвижной точке. В самом деле, рассмотрим следующий преобразователь программ: получив на вход программу $u$, мы перебираем все программы $v$ и все доказательства в формальной теории, пока не найдём некоторую программу $v$, доказуемо различную с $u$. Это программа $v$ и будет результатом преобразования. Если бы непозноваемой программы не существовало, то этот преобразователь был бы вычислимой функцией, не имеющей неподвижной точки.
    \end{proof}
    
    Можно отметить, что "на самом деле" наша непознаваемая функция нигде не определена, поскольку если бы она давала результат на каком-то входе. то она бы была доказуема отична от программы, дающей другой результат на этом же входе. Получается парадокс: с одной стороны, мы доказали, что наша функция отличается от всюду нулевой функции;  с другой стороны, непознаваемость означает невозможность такого доказательства. Этот парадокс получается из-за неявно сделанных предположений (в частности, корректности формальной системы: по второй теореме Гёделя непротиворечимость формальной системы нельзя доказать её средствами).
    
\end{theorem}

    