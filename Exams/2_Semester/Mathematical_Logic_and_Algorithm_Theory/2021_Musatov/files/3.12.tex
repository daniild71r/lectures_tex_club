\subsection{Построение комбинаторов логических значений, булевых функций, операций с параметрами, проверки на ноль для нумералов Чёрча}
    
\begin{definition}
    \textit{Комбинатором} называется замкнутый $\lambda$-терм (без свободных переменных).
\end{definition}

\textbf{Представление логических значений и булевых функций}\\

Пусть
$$
    False = \lambda xy.y (= \overline{0})
$$
$$
    True = \lambda xy.x
$$
Получается, что
$$
    True \: M \: N = M
$$
$$
    False \: M \: N = N
$$
Тогда логические функции выражаются следующим образом:
$$
    And = \lambda pq.pqp
$$
$$
    Or = \lambda pq.ppq
$$
$$
    Not = \lambda p.p \: False \: True
$$

\begin{proof}*\\
    1) $And = \lambda pq.pqp$\\
    Если $p = 0$, то $p \wedge q = 0 = p$\\
    Если $p = 1$, то $p \wedge q = q$\\
    2) $Or = \lambda pq.ppq$\\
    Если $p = 0$, то $p \lor q = q$\\
    Если $p = 1$, то $p \lor q = 1 = p$\\
    3) $Not = \lambda p.p \: False \: True$\\
    Если $p = False$, то $False \: False \: True = True$
    Если $p = True$, то $True \: False \: True = False$
\end{proof}


\textbf{Представление арифметических опреаций на нумералах Чёрча}\\

1) $Inc$ -- прибавление единицы ($Inc \overline{n} = \overline{n + 1}$)\\

$Inc = \lambda nfx.f(nfx)$\\

\begin{proof}
    $
        Inc \overline{n} = (\lambda nfx.f(nfx)) \overline{n} = \lambda fx. f(\overline{n}fx) = \lambda fx.f(\lambda gy.\underbrace{g(g(g(...)))}_{n \text{раз}} fx) = \lambda fx. \underbrace{f(f(f(...(f(fx))...)))}_{n + 1 \text{раз}} = \overline{n + 1}
    $
\end{proof}

2) $Add$ -- сложение\\

$Add = \lambda mnfx.mf(nfx)$\\

\begin{proof}*\\
    $
    Add \: \overline{m} \: \overline{n} = (\lambda mnfx.mf(nfx)) \: \overline{m} \: \overline{n} = \lambda fx. \overline{m}f(\overline{n}fx) = \lambda fx. (\lambda gy. \underbrace{g(g(g(...)))}_{m \text{раз}}) f (\underbrace{f(f(...))}_{n \text{раз}}) = \lambda fx. \underbrace{f(f(f(...)))}_{m + n \text{раз}} = \overline{m + n}
    $
\end{proof}

3) $Mult$ -- умножение\\

$Mult = \lambda mnfx.m(nf)x$\\

Доказательство аналогично.\\

\textbf{Проверка на ноль для нумералов Чёрча}
$$
    IsZero = \lambda n.n (\lambda x.False) True
$$
Проверим для нуля:
$$
    IsZero \overline{0} = \overline{0} (\lambda x.False)True = True
$$
Любое число, кроме нуля представимо в виде: $\overline{n + 1}$. Проверим $IsZero$ для таких чисел:
$$
    IsZero(\overline{n + 1}) = \overline{n + 1} (\lambda x. False) True = (\lambda fx.f(...)) (\lambda x. False) True = ((\lambda x. False)(...)) = False 
$$