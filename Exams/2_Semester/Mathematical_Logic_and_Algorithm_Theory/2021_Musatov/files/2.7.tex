\subsection{Теорема о делении с остатком вполне упорядоченных множеств.}

\textbf{Теорема} $\forall \alpha, \beta \quad \exists ! \gamma, \delta : \delta < \alpha$ и $\alpha = \beta \cdot \gamma + \delta$, где $\alpha, \beta, \gamma, \delta$ -- ВУМы.\\

\textbf{Доказательство:}\\

1) Существование.

Рассмотрим $\zeta$ такое, что заведомо $\beta \zeta > \alpha$ (например, подойдет $\zeta = \alpha + 1$).\\

Это значит, что $\alpha$ равняется некоторому начальному отрезку $\beta \zeta$. Этот начальный отрезок представляется в виде $[0;q), q \in \beta \zeta$ и потому $q = (b, g), b \in \beta, g \in \zeta$\\

$\alpha \in [0;q) \Rightarrow \alpha = (s, t) :$ либо $t < g$, а $s$ любое из $\beta$, либо $t = g, s < b$.\\

Для каждого $t < g$ получаем экземпляр $\beta$, порядок на этих экземплярах взят с $[0;g)$

В итоге : $\gamma = [0;g), \delta = [0;b)$\\

2) Единственность.

Если $\gamma_1 = \gamma_2,$ то аналогично единственности вычитания.
Если $\gamma_1 < \gamma_2$, то $\gamma_1 + 1 \leq \gamma_2$ и поэтому $\beta \cdot \gamma_1 + \delta_1 < \beta \cdot \gamma_1 + \beta = \beta \cdot (\gamma_1 + 1) \leq \beta \cdot \gamma_2 \leq \beta \cdot \gamma_2 + \delta_2$.