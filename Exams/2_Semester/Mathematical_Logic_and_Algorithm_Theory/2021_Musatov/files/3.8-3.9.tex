\subsection{Теорема Райса–Успенского о неразрешимости нетривиальных свойств вычислимых функций.}
\par \textbf{Теорема Райса-Успенского:} Пусть $A \subset \mathcal{F}$ — произвольное нетривиальное свойство вычислимых функций (нетривиальность означает, что есть как функции, ему удовлетворяющие, так и функции, ему не удовлетворяющие, то есть что множество $A$ непусто и не совпадает со всем $\mathcal{F}$).
Пусть $U$ — главная универсальная функция. Тогда не существует алгоритма, который по $U$-номеру вычислимой функции проверял бы, обладает ли она свойством $A$. Другими словами, множество $S_a=\{n | U_n \in A\}$ неразрешимо.
\par $\blacktriangle$ Пусть $\zeta(x)$ — нигде не определённая функция. Без ограничения общности $\zeta \in  \overline{A}$ (иначе получим неразрешимость $\overline{A}$, которая влечёт неразрешимость $A$)
\par Пусть $\xi(x)$ — какая-то функция из $A$. Пусть $K$ — какое-то перечислимое неразрешимое множество (например, из проблемы самоприменимости)
\par Рассмотрим $$V(n, x) = \begin{cases}
\xi(x) & n \in K\\
\zeta(x) & n \in \overline{K}
\end{cases}$$ Тогда $V$ — вычислимая функция. Программа, вычисляющая $V$: запустить перечисление $K$, ожидать появления $n$. Если появилось, вернуть $\xi(x)$.
\par По определению главной универсальной вычислимой функции (ГУВФ) существует всюду определённая $s$, такая что $\forall n \forall x V(n,x) = U (s(n),x)$
\begin{enumerate}
    \item Если $n \in K$, то $V(n, x)=\xi(x)=U(s(n),x) \Rightarrow s(n)$ - номер функции из $A$
    \item Если $n \in \overline{K}$, то $U(s(n), x)=\zeta(x)$, т.е. $s(n)$ — номер функции из $\overline{A}$.
\end{enumerate}
\par Получаем $n \in K \Leftrightarrow s(n) \in S_a$. При этом $s$ вычислима и всюду определена, так что ситуация подходит под
определение $m$-сводимости (см. определения) $\Rightarrow K \leq_m S_a$. Так как $K$ неразрешимо, то и $S_a$ неразрешимо $\blacksquare$

\subsection{Теорема Клини о неподвижной точке. Построение программы, на любом входе печатающей некоторый собственный номер.}
\par \textbf{Теорема Клини о неподвижной точке:} Пусть $U$ — ГУВФ, $h$ — всюду определённая вычислимая функция. Тогда существует $p$, т.ч. при всех $x$ верно $U(p,x)=U(h(p),x)$
\par $\blacktriangle$ Пусть $V(x,y):=U(U(x,x), y)$. В силу главности $U$ существует вычислимая всюду определенная $s$, такая что $\forall x,y \: V(x,y)=U(s(x), y)$. 
\par Рассмотрим $t(x)=h(s(x))$ - вычислима и всюду определена как композиция вычислимых всюду определенных. Значит $\exists p \forall x \: t(x)=U(p,x)$. Тогда $$U(s(p),y)=V(p,y)=U(U(p,p),y)=U(t(p),y)=U(h(s(p)),y) \Rightarrow U(s(p),y)=U(h(s(p)),y) \Rightarrow $$ $$ \Rightarrow s(p) \text{ - неподвижная точка } \blacksquare$$
\par \textbf{Замечание:} в условии некоторый собственный номер означает какой-то свой номер в ГУВФ (какой-то так как у каждой функции бесконечное количество номеров)
\par \textbf{Утверждение:}  Пусть $U(n, x)$ — главная вычислимая универсальная
функция для класса всех вычислимых функций одного аргумента. Тогда существует такое число $p$, что $U(p, x) = p$ для любого $x$.
\par $\blacktriangle$ Рассмотрим $V(n,x)=n$. Так как $U$ - ГУВФ, то существует вычислимая всюду определенная функция $s$ такая что $U(s(n),x)=V(n,x)$. Тогда по теореме Клини о неподвижной точке существует $p$ такое что $\forall x \: U(p,x)=U(s(p),x)=V(p,x)=p \: \blacksquare$