\subsection{Любое счётное вполне упорядоченное множество изоморфно некоторому подмножеству $\mathbb{R}$}

Заметим, что \textbf{вполне упорядоченное множество} - это такое \emph{линейно упорядоченное} множество, что в любом его непустом подмножестве есть минимальный элемент; $\mathbb{Q} \subset \mathbb{R}$. Тогда следующая теорема является более сильной, чем этот билет, и из неё, конечно, следует этот билет.

\textbf{Теорема}. Всякое счётное линейно упорядоченное множество
изоморфно некоторому подмножеству множества $\mathbb{Q}$.

$\blacktriangle$
Пусть X — данное нам множество. Требуемый изоморфизм между ним и $\mathbb{Q}$ строится по шагам. После n шагов у нас есть два n-элементных подмножества $X_n \subset X$ и $Q_n \subset \mathbb{Q}$ (элементы которых мы будем называть «охваченными») и взаимно однозначное соответствие между ними, сохраняющее порядок. На очередном шаге мы берём какой-то неохваченный элемент множества Х и сравниваем его со всеми охваченными элементами X. Он может оказаться либо меньше всех, либо больше, либо попасть между какими-то двумя. В каждом из случаев мы можем найти неохваченный элемент в $\mathbb{Q}$ , находящийся в том же положении (больше всех, между первым и вторым охваченным сверху, между вторым и третьим охваченным сверху и т. п.). При этом мы пользуемся тем, что в $\mathbb{Q}$ нет наименьшего элемента, нет наибольшего и нет соседних элементов, — в зависимости от того, какой из трёх случаев имеет место. После этого мы добавляем выбранные элементы к $X_n$ и $Q_n$, считая их соответствующими друг другу.

Чтобы в пределе получить изоморфизм между множествами X
и некоторым подмножеством $\mathbb{Q}$, мы должны позаботиться о том, чтобы все элементы были рано или поздно охвачены. Поскольку X счётно, пронумеруем его элементы и будем выбирать неохваченный элемент с наименьшим номером. Это соображение завершает доказательство.
$\blacksquare$