Лемма~\ref{varcoef-equivlemma} позволяет рассматривать все только для линейных систем так как все рассуждения автоматически перенесутся и на линейное уравнение.

Для нормальной линейной системы уравнений~(\ref{varcoef-sys})

\begin{theorem}[Существования и единственности для системы (\ref{varcoef-sys}) (б/д)]
Зададим начальное условие \(y(x_0) = y_0\), где \(x \in I\) и $y_0$~--- заданный $n$-мерный вектор. Пусть матрица $A(x)$ и $\vec{f}$ непрерывны на $I$. Тогда на всем $I$ решение задачи Коши (\ref{varcoef-sys}) существует и единственно.
\end{theorem}

\begin{theorem}[Существования и единственности для уравнения (\ref{varcoef-eq}) (б/д)]
Пусть все функции \(a_i(x), i \in [1; n]\), и \(f(x)\)~---непрерывны на $I$ и пусть \(x_0 \in I\).\\
Тогда при произвольных начальных значениях \(y_1^{(0)}, \ldots, y_n^{(0)}\) решение задачи Коши (\ref{varcoef-eq}), 
\[y(x_0) = y_1^{(0)}, y'(x_0) = y_2^{(0)}, \ldots, y^{(n-1)}(x_0) = y_n^{(0)}\]
существует и единственно на всем $I$.
\end{theorem}
