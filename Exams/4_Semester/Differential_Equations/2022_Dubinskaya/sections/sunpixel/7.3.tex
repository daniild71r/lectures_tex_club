\textbf{Замечание.}

Функционалы, зависящие от нескольких неизвестных функций, можно
интерпретировать как функционалы, зависящие от вектор-функции.

Пусть $C_n^1 [a, b]$ -- множество непрерывно дифференцируемых
вектор-функций с компонентами \\
$y_1(x), \dots, y_n(x)$, определёнными на
множестве $[a, b]$. Норма на этом пространстве вводится следующим образом:

$$
\left\| \Vec{y}(x) \right\|_{C_n^1 [a, b]} = 
\max_{x \in [a, b]} |y(x)| + \max_{x \in [a, b]} |y'(x)|
$$

Тогда соответствующая метрика определяется как
$\rho(\Vec{y}, \Vec{z}) = \| \Vec{y} - \Vec{z} \|$.

Функционалы для вектор-функций определяются аналогично:

\begin{equation}\label{J_func_vec}
J(\Vec{y}) = \int_a^b F \left[ x; \Vec{y}(x), \Vec{y'}(x) \right] dx,
\end{equation}

где $F \left[ x; \Vec{y}(x), \Vec{y'}(x) \right]$ --
дважды непрерывно дифференцируемая функция на $[a, b] \times \R^{2n}$.

В качестве аргументов функции $J(\Vec{y})$ будем
рассматривать вектор-функции из множества 
$M = \left\{ \Vec{y}(x) \in C_n^1 [a, b] : 
\Vec{y}(a) = \Vec{A}, \Vec{y}(b) = \Vec{B} \right\}$.

\textbf{Определение.}

Функция $\widehat{y}(x) \in M$ называется 
\textit{слабым локальным минимумом (максимумом)} 
функционала (\ref{J_func_vec}), если

$$
\exists \varepsilon > 0 : \forall \Vec{y} \in M :
\left\| \widehat{\Vec{y}} - \Vec{y} \right\|_{C_n^1} < \varepsilon
\ \ J(\Vec{y}) \geqslant J (\widehat{\Vec{y}})
\ \left( J(\Vec{y}) \leqslant J (\widehat{\Vec{y}}) \right)
$$

\textbf{Теорема.}

Если дважды непрерывно дифференцируемая вектор-функция 
$\widehat{\Vec{y}} \in M$ даёт слабый локальный экстремум
функционала (\ref{J_func_vec}), то $\widehat{\Vec{y}}$
удовлетворяет на $[a, b]$ системе уравнений Эйлера:

\begin{equation}
\frac{\partial F}{\partial y_i} 
- \frac{d}{dx} \left( \frac{\partial F}{\partial y_i'} \right) = 0,
\ i = \overline{1, \dots, n}
\end{equation}

\underline{Доказательство.}

Пусть $\widehat{\Vec{y}} = 
\left( y_1(x), \widehat{y_2}(x), \dots, \widehat{y_n}(x) \right)$ --
первая координата свободная, а остальные зафиксируем.
Тогда $J(\Vec{y}) = \widetilde{J}(y_1)$ -- функционал, зависящий от
одной переменной. Это является простейшей вариационной задачей,
экстремум достигается при $\widehat{y_1}(x)$, и необходимым условием
является уравнение Эйлера:

$$
\frac{\partial F}{\partial y_1} 
- \frac{d}{dx} \left( \frac{\partial F}{\partial y_1'} \right) = 0,
$$

Повторив эти действия для каждой координаты, получаем нужную систему.
\bigbreak
\textbf{Функционалы, зависящие от производных высших порядков.}

В этом разделе мы будем работать с функциями из пространства
$C^k [a, b]$. На этом пространстве норму можно ввести следующим образом:

$$
\left\| y \right\|_{C^k [a, b]} = 
\max_{x \in [a, b]} |y(x)| + 
\sum_{i = 1}^k \max_{x \in [a, b]} |y^{(i)}(x)|
$$

Соответствующая метрика вводится следующим образом:
$\rho(y, z) = \| y - z \|_{C^k}$.

Аналогично простейшей вариационной задаче 
введём множество допустимых вариаций:

$$
\mathring{C}^k [a, b] = 
\left\{ y \in C^k [a, b] : y^{(i)}(a) = y^{(i)}(b) = 0, 
i = \overline{0, \dots, k - 1} \right\}
$$
\pagebreak

\textbf{Лемма}

$$
\text{Если } f(x) \in C [a, b] \text{ и } 
\forall \eta \in \mathring{C}^k [a, b]
\int_a^b f(x) \eta(x) dx = 0 \text{, то }
f(x) \equiv 0 \text{ на } [a, b].
$$

\underline{Доказательство.}

Предположим, что $f(x) \not \equiv 0$ на $[a, b]$.
Тогда $\exists x_0 \in (a, b) : f(x_0) \neq 0$ (пусть,
без ограничения общности, $f(x_0) > 0$).
Так как $f(x)$ непрерывна, то 
$\exists \varepsilon > 0 : 
\left( x - (x_0 - \varepsilon) \right)^{2k} \subset (a, b)$ и
и $f(x) > 0 \forall x \in (x_0 - \varepsilon, x_0 + \varepsilon)$.
Подберём такую вариацию:

\begin{equation*}
\eta(x) = 
\begin{cases}
\left( x - (x_0 - \varepsilon) \right)^{2k}
\left( x - (x_0 + \varepsilon) \right)^{2k},
x \in (x_0 - \varepsilon, x_0 + \varepsilon)
\\
0, \text{ иначе}
\end{cases}
\end{equation*}

Можно проверить, что $\eta(x) \in \mathring{C}^k [a, b]$.
Поэтому 

\begin{equation*}
0 = \int_a^b f(x) \eta(x) dx =
\int_{x_0 - \varepsilon}^{x_0 + \varepsilon} f(x) \eta(x) dx > 0
\end{equation*}

Получили противоречие. Получается, что $f(x) \equiv 0$ на $[a, b]$.

\textbf{Определение.}

Рассмотрим функционал

\begin{equation}\label{J_func_der}
J(y) = \int_a^b F \left[ x; y(x), \dots, y^{(k)}(x) \right] dx,
\end{equation}

определённый на множестве
$M = \{ y(x) \in C^k [a, b] : y^{(i)}(a) = A_i, y^{(i)}(b) = B_i, 
i = \overline{1, \dots, k - 1} \}$,\\
где $F \left[ x; y(x), \dots, y^{(k)}(x) \right]$ -- $(k + 1)$ раз
непрерывно дифференцируемая функция.

Скажем, что $\widehat{y} \in M$ даёт 
\textit{слабый локальный минимум (максимум)} 
функционала (\ref{J_func_der}), если

$$
\exists \varepsilon > 0 : \forall y(x) \in M : 
\| \widehat{y} - y\|_{C^k} < \varepsilon \ \ 
J(y) \geqslant J(\widehat{y}) \left( J(y) \leqslant J(\widehat{y}) \right).
$$

Пусть $y(x) \in M$ и $\eta(x) \in \mathring{C}^k [a, b]$.
При фиксированных $y(x)$ и $\eta(x)$ определим $\Phi(\alpha), \alpha \in \R$:

$$
\Phi(\alpha) = J( y(x) + \alpha \eta(x) ) = 
\int_a^b F \left( x; y + \alpha \eta, \dots, 
y^{(k)} + \alpha \eta^{(k)} \right) dx
$$

Аналогично простейшей вариационной задаче,
если $\widehat{y}$ даёт экстремум $J(y)$, 
то $\Phi(\alpha)$ имеет экстремум при $\alpha = 0$:

$$
\Phi'(0) = \left. \frac{dJ(\widehat{y} + \alpha \eta)}{d\alpha} 
\right|_{\alpha = 0} = 
\int_a^b \left[ 
\frac{\partial F}{\partial \widehat{y}} \eta(x) + 
\frac{\partial F}{\partial \widehat{y}'} \eta'(x) + \dots + 
\frac{\partial F}{\partial \widehat{y}^{(k)}} \eta^{(k)}(x)
\right] dx
$$

\textbf{Определение.}
\\
Выражение $\Phi'(0)$, где $\eta(x) \in \mathring{C}^k [a, b]$, называется
\textit{первой вариацией функционала (\ref{J_func_der})} \\
и обозначается $\delta J[y; \eta]$.

\textbf{Теорема (необходимое условие слабого локального экстремума).}

Пусть $\widehat{y} \in M$ является $2k$ дифференцируемой функцией и 
даёт слабый локальный экстремум функционала (\ref{J_func_der}).
Тогда на $[a, b]$ $\widehat{y}(x)$ удовлетворяет 
уравнению Эйлера-Пуассона:

\begin{equation}\label{euler_pois_eq}
\frac{\partial F}{\partial y} - 
\frac{d}{dx} \left( \frac{\partial F}{\partial y'} \right) +
\frac{d^2}{dx^2} \left( \frac{\partial F}{\partial y''} \right) - \dots
+ (-1)^k 
\frac{d^k}{dx^k} \left( \frac{\partial F}{\partial y^{(k)}} \right) = 0
\end{equation}

\underline{Доказательство.}

Так как $\widehat{y}$ даёт слабый локальный экстремум, то
$\delta J [\widehat{y}, \eta(x)] = 0 \ \ 
\forall \eta(x) \in \mathring{C}^k [a, b]$. Распишем $\delta J$:

$$
0 = \delta J [\widehat{y}, \eta(x)] = 
\int_a^b \left[ 
\frac{\partial F}{\partial \widehat{y}} \eta(x) + 
\frac{\partial F}{\partial \widehat{y}'} \eta'(x) + \dots + 
\frac{\partial F}{\partial \widehat{y}^{(k)}} \eta^{(k)}(x)
\right] dx
$$
\pagebreak

Проинтегрируем по частям каждое слагаемое в этой формуле 
столько раз, сколько нужно, чтобы избавиться от всех производных $\eta(x)$.
Например, для $i$-го слагаемого поступим следующим образом:

\begin{multline*}
\int_a^b \frac{\partial F}{\partial y^{(i)}} \eta^{(i)} dx = 
\int_a^b \frac{\partial F}{\partial y^{(i)}} d\eta^{(i - 1)} = 
\underbrace{\left. \frac{\partial F}{\partial y^{(i)}} \eta^{(i - 1)} 
\right|_a^b}_{= 0 \text{, так как } 
\eta^{(i - 1)}(a) = \eta^{(i - 1)}(b) = 0} - 
\int_a^b \frac{d}{dx} \left( \frac{\partial F}{\partial y^{(i)}} \right)
\eta^{(i - 1)} dx = \\
= - \left[ 
\underbrace{\left. \frac{d}{dx} \left( 
\frac{\partial F}{\partial y^{(i)}} \right) 
\eta^{(i - 2)} \right|_a^b}_{ = 0} - 
\int_a^b \frac{d^2}{dx^2} \left( 
\frac{\partial F}{\partial y^{(i)}} \right) \eta^{(i - 2)} dx
\right] = 
\int_a^b \frac{d^2}{dx^2} \left( \frac{\partial F}{\partial y^{(i)}}
\right) \eta^{(i - 2)} dx = \dots
\end{multline*}

Для последнего слагаемого получится так:

$$
\int_a^b \frac{\partial F}{\partial y^{(k)}} \eta^{(k)} dx = 
(-1)^k \int_a^b \frac{d^k}{dx^k} \left( 
\frac{\partial F}{\partial y^{(k)}} \right) \eta(x) dx
$$

Тогда первая вариация перепишется в следующем виде:

$$
0 = \delta J [\widehat{y}, \eta(x)] = 
\int_a^b \left[ 
\frac{\partial F}{\partial y} - 
\frac{d}{dx} \left( \frac{\partial F}{\partial y'} \right) + \dots + 
(-1)^k \frac{d^k}{dx^k} \left( \frac{\partial F}{\partial y^{(k)}} \right)
\right] \eta(x) dx \ \ 
\forall \eta(x) \in \mathring{C}^k [a, b]
$$

Используя доказанную лемму, получаем необходимое уравнение.
