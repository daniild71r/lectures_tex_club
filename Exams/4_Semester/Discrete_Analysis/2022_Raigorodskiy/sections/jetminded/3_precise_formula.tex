\textbf{Унициклический граф} - связный граф из n вершин и n рёбер.

Идея: ровно один цикл, т.к. к дереву без циклов добавили ещё одно ребро.

Количество унициклических графов на n вершинах: $u_n$

\Th $u_n = \sum_{k=3}^n C_n^k \frac{(k-1)!}{2} kn^{n-k-1}$

\Proof
Длина цикла в уницикл. графе равна k. $k \in [3; n]$, т.к. нет ни петель, ни кратных рёбер, тогда: 

$C_n^k$ - количество способов выбрать вершины цикла; 

$\frac{(k-1)!}{2}$ - количество способов зафиксировать цикл после того, как вершины выбраны;  

$F_{n, k}$ - количество лесов, в котором k деревьев, причём вершина $i$ принадлежит $i$-ому дереву; $F_{n, k} = kn^{n-k-1}$ (док-во внизу).
\EndProof

\leftbar

Доказательство формулы $F_{n, k}$ с семинара: Нетрудно построить взаимно-однозначное соответствие между корневыми лесами на $n$ вершинах с $k$ деревьями и деревьями на вершинах ${1, 2, \dots, n+1}$, где вершина $n+1$ имеет степень $k$. Из прошлой задачи мы знаем, что таких деревьев $C_{n-1}^{k-1}n^{n-k}$ (выбираем $k-1$ позиций в коде Прюфера для вершины $n+1$, остальное заполняем произвольно). В задаче нас просят посчитать не просто количество корневых лесов на $n$ вершинах с $k$ деревьями, а количество таких лесов, где корнем $i$-го дерева является вершина $i$. В силу симметрии. Каждое k-элементное подмножество множества ${1, 2, \dots, n+1}$ является множеством корней для одинакового числа лесов, поэтому чтобы получить ответ надо поделить количество корневых лесов на $C_n^k$. Таким образом, ответом будем $kn^{n-k-1}$.
\endleftbar

Следствие: $u_n \sim \sqrt{\frac{\pi}{8}} n^{n - \frac{1}{2}}$ - Асимптотика (б/д).