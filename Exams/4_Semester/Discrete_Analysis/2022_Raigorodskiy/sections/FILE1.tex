\section{(4) Определение графа, графов с петлями и кратными ребрами. Ориентированные графы. Лемма
о рукопожатиях. Четыре определения дерева и их эквивалентность.}
\subsection{1. Недетерминированные конечные автоматы (НКА). Различные варианты определений. Автоматные языки.}

\Def Алфавит $\Sigma$ — непустое конечное множество, элементы которого называются символами. При этом $\Sigma^*$ — множество слов, состоящее из всех слов $\Sigma$, $\varepsilon \in \Sigma^*$.

\Def Формальный язык $L$ — некоторое подмножество $\Sigma^*$.

\Def Недетерминированный конечный автомат (НКА) — кортеж $M = \langle Q, \Sigma, \Delta, q_0, F \rangle$:

\begin{enumerate}
    \item $Q$ — множество состояний, $Q$ — конечное множество, то есть $|Q| < \infty$;
    \item $\Sigma$ — алфавит;
    \item $\Delta \subset Q \times \Sigma^* \times Q$ — множество переходов \textit{(т.е. $\text{состояние}_{1} \xrightarrow{\text{слово}} \text{состояние}_{2}$)};
    \item $q_0 \in Q$ — стартовое состояние;
    \item $F \subset Q$ — множество завершающих состояний.
\end{enumerate}

\Def Конфигурация в автомате $\langle Q, \Sigma, \Delta, q_0, F \rangle$ — элемент $\langle q, w \rangle \in Q \times \Sigma^*$.

\Def Отношение $\vdash$ достижимости по $M$ — наименьшее рефлексивное транзитивное отношение над $Q \times \Sigma^*$, такое что:

\begin{enumerate}
    \item $\forall w \in \Sigma^* : (\langle q_1, w \rangle \rightarrow q_2) \in \Delta \Longrightarrow \langle q_1, w \rangle \vdash \langle q_2, \varepsilon \rangle$
    \item $\forall u, v \in \Sigma^* : \langle q_1, u \rangle \vdash \langle q_2, \varepsilon \rangle$, $\langle q_2, v \rangle \vdash \langle q_3, \varepsilon \rangle \Longrightarrow \langle q_1, uv \rangle \vdash \langle q_3, \varepsilon \rangle$
    \item $\forall u \in \Sigma^* : \langle q_1, u \rangle \vdash \langle q_2, \varepsilon \rangle \Longrightarrow \forall v \in \Sigma^* \langle q_1, uv \rangle \vdash \langle q_2, v \rangle$
\end{enumerate}

\begin{figure}[h]
    \hspace{-4ex} \begin{minipage}[h]{0.6\linewidth}
    \center{\includegraphics[width=0.6\linewidth]{1_1_1.jpg}}
    \end{minipage}
    \hfill
    \hspace{-4ex} \begin{minipage}[h]{0.5\linewidth}
    \Example Рассмотрим как автомат с картинки распознает слово $abab$:
    \newline $\langle q_0, abab \rangle \vdash \langle q_0, bab \rangle \vdash \langle q_1, ab \rangle \vdash \langle q_1, \varepsilon\rangle$ \\
    По транзитивности получаем: $\langle q_0, abab \rangle \vdash \langle q_1, \varepsilon\rangle$ \\
    \newline \textbf{Поясним переходы:} 
    \newline Первый и второй $\vdash$ работают по свойству 3, \newline а третий $\vdash$ работает по свойству 1.
    \end{minipage}
\end{figure}

\Def Для автомата $M = \langle Q, \Sigma, \Delta, q_0, F \rangle$ языком $L(M)$, задаваемым автоматом $M$, является множество $\{ w \in \Sigma^* \,\, | \,\, \exists q \in F : \langle q_0, w \rangle \vdash \langle q, \varepsilon \rangle \}$.

\Def Язык $L$ - автоматный, если существует такой НКА $M$, что $L = L (M)$.

\textbf{Утверждение об НКА с одним завершающим состоянием:} Для любого автоматного языка $L$ существует НКА $M' = \langle Q', \Sigma, \Delta', q_0', F' \rangle$, такой что $L(M') = L$, и $|F'| = 1$.

\textit{Идея доказательства:} добавим одно завершающее состояние $q_f$ на замену остальным и добавим $\varepsilon$-переходы из ''предыдущих'' завершающих состояний в новое. \newline \Vars$\vdash_i$ - достижимость за $i$ переходов

\Proof $L$ — автоматный язык, значит, существует НКА $M = \langle Q, \Sigma, \Delta, q_0, F \rangle$, такой что $L (M) = L$. Введём $M' = \langle Q \cup \{ q_f \}, \Sigma, \Delta', q_0, \{ q_f \} \rangle$, где $\Delta' = \Delta \cup \{ \langle q, \varepsilon \rangle \mapsto q_f \,\,|\,\, q \in F \}$. Далее нужно доказать, что $L (M) = L(M')$.

Докажем, что $L(M) \subset L (M')$. По определению из того, что $w \in L (M)$, следует, что существует состояние $q \in F$, что $\langle q_0, w \rangle \vdash \langle q, \varepsilon \rangle$. В автомате $M'$ есть переход $\langle q, \varepsilon \rangle \mapsto q_f \Longrightarrow \langle q, \varepsilon \rangle \vdash \langle q_f, \varepsilon \rangle$. Так как $\langle q_0, w \rangle \vdash \langle q, \varepsilon \rangle \vdash \langle q_f, \varepsilon \rangle$, то $w \in L (M')$.
\begin{figure}[h]
    \hspace{-4ex} \begin{minipage}[h]{1\linewidth}
    \center{\includegraphics[width=0.6\linewidth]{1_1_2.jpg}}
    \end{minipage}
    \hspace{-4ex}
\end{figure}

Докажем, что $L (M) \supset L (M')$. Из того, что $w \in L (M')$, следует, что $\langle q_0, w \rangle \vdash \langle q_f, \varepsilon \rangle$. Но так как в $q_f$ можно добраться только по $\varepsilon$-переходу, то существует состояние $q'$, что $\langle q_0, w \rangle \vdash \langle q', \varepsilon \rangle \vdash_1 \langle q_f, \varepsilon \rangle \Longrightarrow q' \in F$. А в автомате $M$ $\langle q_0, w \rangle \vdash \langle q', \varepsilon \rangle$. Значит, $w \in L(M)$. \EndProof

\textbf{Утверждение об НКА с не более однобуквенными переходами:} Для любого автоматного языка $L$ существует НКА $M = \langle Q, \Sigma, \Delta, q_0, F \rangle$, такой что $L = L(M)$ и $\forall (\langle q_1, w \rangle \mapsto q_2) \in \Delta$ $|w| \leqslant 1$

\begin{center}
\includegraphics[width=0.6\linewidth]{1_1_3.jpg}
\end{center}

\textbf{Теорема об НКА с однобуквенными переходами:} Для любого НКА $M = \langle Q, \Sigma, \Delta, q_0, F \rangle$ существует НКА $M' = \langle Q, \Sigma, \Delta', q_0, F' \rangle$, такой что $L (M) = L (M')$ и:
$$\forall (\langle q_1, w \rangle \rightarrow q_2) \in \Delta \quad |w| = 1$$
\Proof Обозначим множество вершин, достижимых из $q$ по $w$ как $\Delta (q, w) = \{ q' | \langle q, w \rangle \vdash \langle q', \varepsilon \rangle \}$. Считаем, что в любом переходе $|w| \leqslant 1$. Введём следующие множества:

\begin{center}
    $F' := \{ q \,\,|\,\, \Delta (q, \varepsilon) \cap F \neq \varnothing \}$
    
    $\Delta' = \{ \langle q_1, a \rangle \rightarrow q_2 \,\, |\,\, \exists q_3 \in \Delta (q_1, \varepsilon) : \langle q_3, a \rangle \rightarrow q_2 \}$
\end{center}

Докажем, что $L(M') = L(M)$

Пусть $w\in L(M')$. Тогда $\exists q' \in F' : \langle q_0, w \rangle \vdash_{M'} \langle q', \varepsilon \rangle$. 

Тогда $\exists q \in F : \Delta(q',\varepsilon) = q \Longrightarrow \langle q', \varepsilon \rangle \vdash_{M} \langle q, \varepsilon \rangle$.

Рассмотрим $w = w_1 w_2 \dots w_n$. Тогда верно следующее:
$$
\forall m \; \exists q_m' \; : \; \angles{q_{m-1}', w_m} \rightarrow q_m' \in \Delta' \;\; (q_m' := q') \quad (1)
$$
%m-1
Значит, $\exists q_{m} \; : \; \Delta(q_{m-1}', \varepsilon) = q_{m}, \; \angles{q_{m}, w_m} \rightarrow q_m'$ \ и \ $\angles{q_{m-1}', w_{m}}\vdash_M \angles{q_m', \varepsilon} \quad (2)$

Из (1) и (2) следует, что 
$$
\angles{q_0', w_1\ldots w_m}\vdash_M\angles{q_1', w_2\ldots w_m}\vdash_M\angles{q_2', w_3\ldots w_m}\ldots\vdash_M\angles{q_m', w_m}\vdash_M\angles{q, \varepsilon}
$$
Следовательно, $w\in L(M)$.
% Докажем следующее утверждение:

% \begin{center}
%     $\exists q \in F : \langle q_0, w \rangle \vdash_M \langle q, \varepsilon \rangle \Longleftrightarrow \exists q' \in F' : \langle q_0, w \rangle \vdash_{M'} \langle q', \varepsilon \rangle$
% \end{center}

% Пусть $w = w_1 w_2 \dots w_n$, $w \in L(M') \Longrightarrow \exists q' \in F' : \langle q_0, w \rangle \vdash_{M'} \langle q, \varepsilon \rangle$. Из однобуквенности всех переходов:

% \begin{center}
%     $\exists q_1, \dots, q_n : \langle q_0, w_1 \dots w_n \rangle \vdash_{M'} \langle q_1, w_2 \dots w_n \rangle \vdash_{M'} \dots \vdash_{M'} \langle q_{n - 1}, w_n \rangle \vdash_{M'} \langle q_n, \varepsilon \rangle$
    
%     $q_n \in F' \Longrightarrow \exists q'' \in F : q'' \in \Delta (q_n, \varepsilon) \Longrightarrow \langle q_n, \varepsilon \rangle \vdash_M \langle q'', \varepsilon \rangle$ \qquad $(1)$
    
%     $\brackets{\langle q_{k - 1}, w_k \rangle \overset{M'}{\rightarrow} q_k} \Longrightarrow \exists \overset{\sim}{q_k} \in Q$
    
%     $\overset{\sim}{q_k} = \Delta \brackets{q_{k - 1}, \varepsilon}$
    
%     $\langle \overset{\sim}{q_k}, w_k \rangle \rightarrow q_k \in \Delta$
    
%     $\langle q_{k - 1}, \varepsilon \rangle \vdash_M \langle \overset{\sim}{q_k}, \varepsilon \rangle$
    
%     $\langle q_{k - 1}, w_k \rangle \vdash_M \langle q_k, \varepsilon \rangle$ $\brackets{2}$
% \end{center}

% \noindent Из $\brackets{1}$ и $\brackets{2}$ следует, что существует $q'' \in F$, такой что $\langle q_0, w_1 \dots w_n \rangle \overset{\brackets{2}}{\vdash_M} \langle q_1, w_2 \dots w_k \rangle \overset{\brackets{2}}{\vdash_M} \dots \overset{\brackets{2}}{\vdash_M} \langle q_n, \varepsilon \rangle \overset{\brackets{1}}{\vdash_M} \langle q'', \varepsilon \rangle$, откуда $w = w_1 \dots w_n \in L \brackets{M}$. 

\par В обратную сторону: $w \in L(M) \Rightarrow \exists q \in F: \: \langle q_0, w \rangle \vdash_M \langle q, \varepsilon \rangle$. Пусть $w=w_1w_2$ (для больших n аналогично). Тогда есть цепь \begin{itemize}
    \item $\langle q_0, w_1w_2 \rangle \vdash_M \langle q_1', w_1w_2 \rangle$
    \item $\langle q_1', w_1w_2 \rangle \vdash_{M,1} \langle q_1, w_2 \rangle$ (читаем символ $w_1$)
    \item $\langle q_1, w_2 \rangle \vdash_M \langle q_2', w_2 \rangle$
    \item $\langle q_2', w_2 \rangle \vdash_{M,1} \langle q_2, \varepsilon \rangle$ (читаем символ $w_2$)
    \item $\langle q_2, \varepsilon \rangle \vdash_M \langle q, \varepsilon \rangle$
\end{itemize}
\par Получаем, что $\Delta(q_0, \varepsilon)=q_1', \: \langle q_1', w_1 \rangle \rightarrow q_1 \Rightarrow \langle q_0, w_1 \rangle \rightarrow q_1 \in \Delta'$.

\par Аналогично $\langle q_1, w_2 \rangle \rightarrow q_2 \in \Delta'$. Так как $\Delta(q_2, \varepsilon)=q \in F \Rightarrow q_2 \in F'$.
\par Тогда $\langle q_0, w_1w_2 \rangle \vdash_{M'} \langle q_1, w_2 \rangle \vdash_{M'} \langle q_2, \varepsilon \rangle$ и $w=w_1w_2 \in L(M')$
\EndProof


\newpage{}

\section{(5) Код Прюфера. Формула Кэли.}
\includegraphics[width=1\linewidth]{sections/Polina/imgs/13.jpg}
\newpage \includegraphics[width=1\linewidth]{sections/Polina/imgs/14.jpg}
\newpage{}

\section{(7) Точная ф-ла для числа унициклических графов. Асимптотика (б/д).}
\textbf{Унициклический граф} - связный граф из n вершин и n рёбер.

Идея: ровно один цикл, т.к. к дереву без циклов добавили ещё одно ребро.

Количество унициклических графов на n вершинах: $u_n$

\Th $u_n = \sum_{k=3}^n C_n^k \frac{(k-1)!}{2} kn^{n-k-1}$

\Proof
Длина цикла в уницикл. графе равна k. $k \in [3; n]$, т.к. нет ни петель, ни кратных рёбер, тогда: 

$C_n^k$ - количество способов выбрать вершины цикла; 

$\frac{(k-1)!}{2}$ - количество способов зафиксировать цикл после того, как вершины выбраны;  

$F_{n, k}$ - количество лесов, в котором k деревьев, причём вершина $i$ принадлежит $i$-ому дереву; $F_{n, k} = kn^{n-k-1}$ (док-во внизу).
\EndProof

\leftbar

Доказательство формулы $F_{n, k}$ с семинара: Нетрудно построить взаимно-однозначное соответствие между корневыми лесами на $n$ вершинах с $k$ деревьями и деревьями на вершинах ${1, 2, \dots, n+1}$, где вершина $n+1$ имеет степень $k$. Из прошлой задачи мы знаем, что таких деревьев $C_{n-1}^{k-1}n^{n-k}$ (выбираем $k-1$ позиций в коде Прюфера для вершины $n+1$, остальное заполняем произвольно). В задаче нас просят посчитать не просто количество корневых лесов на $n$ вершинах с $k$ деревьями, а количество таких лесов, где корнем $i$-го дерева является вершина $i$. В силу симметрии. Каждое k-элементное подмножество множества ${1, 2, \dots, n+1}$ является множеством корней для одинакового числа лесов, поэтому чтобы получить ответ надо поделить количество корневых лесов на $C_n^k$. Таким образом, ответом будем $kn^{n-k-1}$.
\endleftbar

Следствие: $u_n \sim \sqrt{\frac{\pi}{8}} n^{n - \frac{1}{2}}$ - Асимптотика (б/д).
\newpage{}

\section{(9) Асимптотика числа унициклических графов.}
\subsection{4. Регулярные выражения. Теорема Клини о совпадении классов регулярных и автоматных языков. Регулярный автомат, алгоритм построения.}

\Vars \\
Regex (регулярное выражение) обозначим за $R$, \newline Language (язык) -- за $L$, \newline $L(R_i)$ (язык, который задается регулярным выражением $R$) -- $L_i$.

\Def Рекурсивное определение регулярного выражения.

% \begin{minipage}[r]{0.1\linewidth} 
% %\begin{flushright}
%     \includegraphics[width=5\linewidth]{images/1_4_1.png}
% %\end{flushright} 
% \end{minipage} 
\begin{center}
    \begin{tabular}{|c|c|}
        \hline
        $Regex (R)$ & $Language (L_i = L(R_i))$ \\
        \hline
        $0$ & $\varnothing$ \\
        $1$ & $\{ \varepsilon \}$ \\
        $a$, $a \in \Sigma$ & $\{ a \}$ \\
        $R_1 + R_2$ & $L_1 \cup L_2$ \\
        $R_1 \cdot R_2$ & $L_1 \cdot L_2$ \\
        $R^*$ & $L^*$ \\
        \hline
    \end{tabular}
\end{center}

Здесь $\varepsilon$ -- пустое слово, <<$\cdot$>> -- операция конкатенации языков (в полученном языке $L_1 \cdot L_2$ лежат слова вида $a_1a_2$, где слово $a_1$ лежит в языке $L_1$, а слово $a_2$ лежит в языке $L_2$), <<$*$>> -- звезда Клини.

Напомним определение звезды Клини: $V^* = \bigcup_{i=0}^{\infty} V^i$ 

\textbf{Приоритет операций} в регулярных выражениях (левее — приоритетнее): $* \rightarrow \cdot \rightarrow +$

\Def Язык $L$ -- регулярный, если он задается регулярным выражением.

\hspace{4ex}

\textbf{Теорема Клини:} Классы регулярных и автоматных языков совпадают.

\Proof Докажем два вложения:

\textbf{1. Регулярные $\subseteq$ Автоматные}

% \begin{figure}[h]
%     \begin{minipage}[h]{0.6\linewidth}
%     Индукция по построению выражения. 
    
%     Немного изменим утверждение -- докажем, что по регулярному выражению можно построить НКА с $1$ завершающим состоянием, который задает тот же язык.\\
    
%     \textit{База}: Построим автоматы для регулярных выражений: 0, 1, a.
%     \end{minipage}
%     \hspace{-4ex} \begin{minipage}[h]{0.5\linewidth}
%     \center{\includegraphics[width=0.6\linewidth]{images/4_base.jpg}}
%     \end{minipage}
% \end{figure}
% Регулярное выражение <<$0$>> -- автомат без завершающих состояний.

% Регулярное выражение <<$1$>> -- в автомате, состоящем из одной вершины, стартовая вершина помечается завершающим состоянием.

% Регулярное выражение <<$a$>> -- в автомате две вершины. Вершина номер $0$ стартовая, вершину номер $1$ помечаем как терминальную. Проводим ребро из $0$ в $1$, на котором пишем букву a.

Индукция по построению выражения. Немного изменим утверждение -- докажем, что по регулярному выражению можно построить НКА с $1$ завершающим состоянием, который задает тот же язык.

\textit{База}: Построим автоматы для регулярных выражений: $0, 1, a \in \Sigma$.
% %картинка%
\begin{figure}[h!]
    \centering
    \includegraphics[scale=0.27]{4_base.jpg}
\end{figure}
% \newline \center{\includegraphics[width=0.29\linewidth]{4_base.jpg}}
% \begin{minipage}[r]{1\linewidth} 
% %\begin{flushright}
%     % \includegraphics[width=2\linewidth]{images/1_4_2.png}
%     \center{\includegraphics[width=1\linewidth]{images/4_base.jpg}}
% %\end{flushright} 
% \end{minipage} 

\textit{Переход:}

1) $R = R_1 + R_2$. Построим автомат $A_1$ для $R_1$, для которого вершина $S_1$ -- стартовая, а вершина $F_1$ -- единственная терминальная. Для $R_2$ это будут автомат $A_2$ со стартовой вершиной $S_2$ и терминальной $F_2$.
Создадим новую вершину $S$, которая и будет стартовой в новом автомате. Из нее проведем два ребра с $\varepsilon$-переходами в $S_1$ и в $S_2$. Аналогично соединим завершающие в автоматах с новой завершающей вершиной $F$. Нетрудно доказать, что такой автомат задаст тот же язык, что и наше регулярное выражение.

2) $R = R_1 \cdot R_2$. Аналогично прошлому пункту получим автоматы для $R_1$ и $R_2$ с теми же обозначениями. Вершина $S_1$ будет стартовой в нашем новом автомате. Добавим также $\varepsilon$-переход из $F_1$ в $S_2$, уберем терминальность $F_1$.

3) $R = R_1^*$. Построим автомат $A_1$ для $R_1$ со стартовой вершиной $S_1$ и терминальной вершиной $F_1$. Создадим вершину $S$ -- новую стартовую вершину, пометим ее терминальной. Добавим из нее и из $F_1$ $\varepsilon$-переход в $S_1$.

\textbf{2. Автоматные $\subseteq$ Регулярные}

\Note{Регулярный автомат -- НКА, в котором на ребрах записаны регулярные выражения. Докажем утверждение для регулярных автоматов.}

\Note{Всякий НКА задается регулярным автоматом с 1 завершающим состоянием.}

Индукция по $|Q|$ (количеству состояний -- вершин) в регулярном автомате.

\textit{База:}

1) $|Q| = 1$. Тогда в регулярном автомате стартовое состояние является завершающим, и можно однозначно построить регулярное выражение. Такому автомату соответсвует регулярное выражение $a^*$
\newline
\begin{minipage}[r]{0.1\linewidth} 
%\begin{flushright}
    \includegraphics[width=4\linewidth]{images/1_4_3.png}
%\end{flushright} 
\end{minipage} 

2) $|Q| = 2$. Cтартовое состояние и завершающее состояние различны, и можно тоже однозначно построить регулярное выражение. Такому автомату соответсвует регулярное выражение $\alpha^*\beta(\gamma + \delta \alpha^* \beta)^*$
\newline
\begin{minipage}[r]{0.1\linewidth} 
%\begin{flushright}
    \includegraphics[width=4\linewidth]{images/1_4_4.png}
%\end{flushright} 
\end{minipage} 

Для случая, когда завершающее состояние -- это начальная вершина, регулярное выражение будет $(\gamma + \delta \alpha^* \beta)^*$.

\textit{Переход:}
\Note{Есть нестартовая и незавершающая вершина!}

1) Удаляем кратные ребра:
\begin{minipage}[r]{0.2\linewidth} 
%\begin{flushright}
    \includegraphics[width=2.5\linewidth]{images/1_4_5.png}
%\end{flushright} 
\end{minipage} 

Кратные ребра означают, что мы можем выбрать, какой символ будем использовать. Именно этот смысл и несет в себе операция <<$+$>>.

2) Добавляем циклы на себя:
\begin{minipage}[r]{0.1\linewidth} 
%\begin{flushright}
    \includegraphics[width=3.5\linewidth]{images/1_4_6.png}
%\end{flushright} 
\end{minipage} 

3) Удаляем нестартовое и незавершающее состояние:

\begin{minipage}[r]{0.1\linewidth} 
%\begin{flushright}
    \includegraphics[width=5.5\linewidth]{images/1_4_7.png}
%\end{flushright} 
\end{minipage} 
\newline Теперь у нас на одно сотояние стало меньше, т.е. мы можем воспользоваться утверждением индукции.
\EndProof
\newpage{}

\section{(5) Определение плоских и планарных графов. Формула Эйлера (б/д). Примеры непланарных графов. Критерий Понтрягина–Куратовского планарности графов (б/д).}
\subsection{3. Свойства класса автоматных языков. Замкнутость относительно булевых операций.}

\Def Полный ДКА.
Полный ДКА (ПДКА) - ДКА, для которого выполнено:
$$\forall a \in \Sigma, q \in Q \,\,\, |\Delta (q, a)| = 1$$

\Statement Для любого автоматного языка $L$ существует ПДКА $M$, такой что $L(M) = M$ (т.е. автоматы распознают одинаковое множество слов);

Метод построения ПДКА из ДКА:\\
1) строим ''стоковую'' вершину.\\
2) Добавляем из всех вершин переходы по недостающим буквам в "сток".

\begin{figure}[h]
    \hspace{-4ex} \begin{minipage}[h]{1\linewidth}
    \center{\includegraphics[width=0.6\linewidth]{1_3_1.png}}
    \end{minipage}
    \hspace{-4ex}
\end{figure}

Появятся ли новые слова? - нет, потому что, если мы попали в стоковую вершину, то не сможем ''выбраться'' из неё.

\Def Итерация Клини для языка L.
$$L^* = \cup_{k = 0}^{\infty}L^k$$

\Th Класс автоматных языков замкнут относительно\\
1. Конкатенации\\
2. Объединения\\
3. Пересечения\\
4. Итерации Клини\\
5. Дополнения

\Proof
Далее будем рассматривать только НКА с одним завершающим состоянием.
Для того чтобы после операции у итогового автомата было одно завершающее состояние, добавляем состояние и соединяем завершающие состояния с ним с помощью $\varepsilon$-переходов. (делаем новое состояние - завершающим, а старые - не завершающими)

1) Конкатенация $M_1$ и $M_2$:

Соединяем $\varepsilon$-переходами завершающее состояние $M_1$ со стартовыми состояниями $M_2$.
\begin{center}
\includegraphics[width=0.45\linewidth]{1_3_2.png}
\end{center}

2) Объединение $M_1$ и $M_2$:

Добавляем стартовое состояние. Соединяем её со стартовыми состояниями $M_1$ и $M_2$ с помощью $\varepsilon$-переходов. 
\begin{center}
\includegraphics[width=0.25\linewidth]{1_3_3.png}
\end{center}

4) Итерации Клини над $M_1$:

Добавляем стартово-завершающее состояние. С помощью $\varepsilon$-переходов соединяем её с начальными состояниями $M_1$, а завершающее состояния $M_1$ с ней.
\begin{center}
\includegraphics[width=0.32\linewidth]{1_3_4.png}
\end{center}

3) Пересечение $M_1$, $M_2$:

Строим "декартово произведение" автоматов с одно буквенными переходами.

\begin{center}
\includegraphics[width=0.55\linewidth]{1_3_5.png}
\end{center}

То есть пересечение будет состоять из состояний, каждому из которых соответствует пара чисел $(i, j)$, это номера состояний из $M_1$ и $M_2$ соответственно, которым это состояние соответствует. И между состояниями $(i_1, j_1)$ и $(i_2, j_2)$ будет проходить ребро с символом $k$, если между $i_1$ и $i_2$ проходило ребро с символом $k$ в $M_1$ и между $j_1$ и $j_2$ проходило ребро с символом $k$ в $M_2$. $(i, j)$ - стартовое состояние, если $i$ - стартовое в $M_1$, $j$ - стартовое в $M_2$. Аналогично с завершающем состоянием. 

5) Дополнение: строим ПДКА и инвертируем терминальность всех состояний.

\newpage{}

\section{(4) Пути и циклы. Простые пути и циклы. Критерии эйлеровости графа и ориентированного
графа.}
\subsection{4. Регулярные выражения. Теорема Клини о совпадении классов регулярных и автоматных языков. Регулярный автомат, алгоритм построения.}

\Vars \\
Regex (регулярное выражение) обозначим за $R$, \newline Language (язык) -- за $L$, \newline $L(R_i)$ (язык, который задается регулярным выражением $R$) -- $L_i$.

\Def Рекурсивное определение регулярного выражения.

% \begin{minipage}[r]{0.1\linewidth} 
% %\begin{flushright}
%     \includegraphics[width=5\linewidth]{images/1_4_1.png}
% %\end{flushright} 
% \end{minipage} 
\begin{center}
    \begin{tabular}{|c|c|}
        \hline
        $Regex (R)$ & $Language (L_i = L(R_i))$ \\
        \hline
        $0$ & $\varnothing$ \\
        $1$ & $\{ \varepsilon \}$ \\
        $a$, $a \in \Sigma$ & $\{ a \}$ \\
        $R_1 + R_2$ & $L_1 \cup L_2$ \\
        $R_1 \cdot R_2$ & $L_1 \cdot L_2$ \\
        $R^*$ & $L^*$ \\
        \hline
    \end{tabular}
\end{center}

Здесь $\varepsilon$ -- пустое слово, <<$\cdot$>> -- операция конкатенации языков (в полученном языке $L_1 \cdot L_2$ лежат слова вида $a_1a_2$, где слово $a_1$ лежит в языке $L_1$, а слово $a_2$ лежит в языке $L_2$), <<$*$>> -- звезда Клини.

Напомним определение звезды Клини: $V^* = \bigcup_{i=0}^{\infty} V^i$ 

\textbf{Приоритет операций} в регулярных выражениях (левее — приоритетнее): $* \rightarrow \cdot \rightarrow +$

\Def Язык $L$ -- регулярный, если он задается регулярным выражением.

\hspace{4ex}

\textbf{Теорема Клини:} Классы регулярных и автоматных языков совпадают.

\Proof Докажем два вложения:

\textbf{1. Регулярные $\subseteq$ Автоматные}

% \begin{figure}[h]
%     \begin{minipage}[h]{0.6\linewidth}
%     Индукция по построению выражения. 
    
%     Немного изменим утверждение -- докажем, что по регулярному выражению можно построить НКА с $1$ завершающим состоянием, который задает тот же язык.\\
    
%     \textit{База}: Построим автоматы для регулярных выражений: 0, 1, a.
%     \end{minipage}
%     \hspace{-4ex} \begin{minipage}[h]{0.5\linewidth}
%     \center{\includegraphics[width=0.6\linewidth]{images/4_base.jpg}}
%     \end{minipage}
% \end{figure}
% Регулярное выражение <<$0$>> -- автомат без завершающих состояний.

% Регулярное выражение <<$1$>> -- в автомате, состоящем из одной вершины, стартовая вершина помечается завершающим состоянием.

% Регулярное выражение <<$a$>> -- в автомате две вершины. Вершина номер $0$ стартовая, вершину номер $1$ помечаем как терминальную. Проводим ребро из $0$ в $1$, на котором пишем букву a.

Индукция по построению выражения. Немного изменим утверждение -- докажем, что по регулярному выражению можно построить НКА с $1$ завершающим состоянием, который задает тот же язык.

\textit{База}: Построим автоматы для регулярных выражений: $0, 1, a \in \Sigma$.
% %картинка%
\begin{figure}[h!]
    \centering
    \includegraphics[scale=0.27]{4_base.jpg}
\end{figure}
% \newline \center{\includegraphics[width=0.29\linewidth]{4_base.jpg}}
% \begin{minipage}[r]{1\linewidth} 
% %\begin{flushright}
%     % \includegraphics[width=2\linewidth]{images/1_4_2.png}
%     \center{\includegraphics[width=1\linewidth]{images/4_base.jpg}}
% %\end{flushright} 
% \end{minipage} 

\textit{Переход:}

1) $R = R_1 + R_2$. Построим автомат $A_1$ для $R_1$, для которого вершина $S_1$ -- стартовая, а вершина $F_1$ -- единственная терминальная. Для $R_2$ это будут автомат $A_2$ со стартовой вершиной $S_2$ и терминальной $F_2$.
Создадим новую вершину $S$, которая и будет стартовой в новом автомате. Из нее проведем два ребра с $\varepsilon$-переходами в $S_1$ и в $S_2$. Аналогично соединим завершающие в автоматах с новой завершающей вершиной $F$. Нетрудно доказать, что такой автомат задаст тот же язык, что и наше регулярное выражение.

2) $R = R_1 \cdot R_2$. Аналогично прошлому пункту получим автоматы для $R_1$ и $R_2$ с теми же обозначениями. Вершина $S_1$ будет стартовой в нашем новом автомате. Добавим также $\varepsilon$-переход из $F_1$ в $S_2$, уберем терминальность $F_1$.

3) $R = R_1^*$. Построим автомат $A_1$ для $R_1$ со стартовой вершиной $S_1$ и терминальной вершиной $F_1$. Создадим вершину $S$ -- новую стартовую вершину, пометим ее терминальной. Добавим из нее и из $F_1$ $\varepsilon$-переход в $S_1$.

\textbf{2. Автоматные $\subseteq$ Регулярные}

\Note{Регулярный автомат -- НКА, в котором на ребрах записаны регулярные выражения. Докажем утверждение для регулярных автоматов.}

\Note{Всякий НКА задается регулярным автоматом с 1 завершающим состоянием.}

Индукция по $|Q|$ (количеству состояний -- вершин) в регулярном автомате.

\textit{База:}

1) $|Q| = 1$. Тогда в регулярном автомате стартовое состояние является завершающим, и можно однозначно построить регулярное выражение. Такому автомату соответсвует регулярное выражение $a^*$
\newline
\begin{minipage}[r]{0.1\linewidth} 
%\begin{flushright}
    \includegraphics[width=4\linewidth]{images/1_4_3.png}
%\end{flushright} 
\end{minipage} 

2) $|Q| = 2$. Cтартовое состояние и завершающее состояние различны, и можно тоже однозначно построить регулярное выражение. Такому автомату соответсвует регулярное выражение $\alpha^*\beta(\gamma + \delta \alpha^* \beta)^*$
\newline
\begin{minipage}[r]{0.1\linewidth} 
%\begin{flushright}
    \includegraphics[width=4\linewidth]{images/1_4_4.png}
%\end{flushright} 
\end{minipage} 

Для случая, когда завершающее состояние -- это начальная вершина, регулярное выражение будет $(\gamma + \delta \alpha^* \beta)^*$.

\textit{Переход:}
\Note{Есть нестартовая и незавершающая вершина!}

1) Удаляем кратные ребра:
\begin{minipage}[r]{0.2\linewidth} 
%\begin{flushright}
    \includegraphics[width=2.5\linewidth]{images/1_4_5.png}
%\end{flushright} 
\end{minipage} 

Кратные ребра означают, что мы можем выбрать, какой символ будем использовать. Именно этот смысл и несет в себе операция <<$+$>>.

2) Добавляем циклы на себя:
\begin{minipage}[r]{0.1\linewidth} 
%\begin{flushright}
    \includegraphics[width=3.5\linewidth]{images/1_4_6.png}
%\end{flushright} 
\end{minipage} 

3) Удаляем нестартовое и незавершающее состояние:

\begin{minipage}[r]{0.1\linewidth} 
%\begin{flushright}
    \includegraphics[width=5.5\linewidth]{images/1_4_7.png}
%\end{flushright} 
\end{minipage} 
\newline Теперь у нас на одно сотояние стало меньше, т.е. мы можем воспользоваться утверждением индукции.
\EndProof
\newpage{}

\section{(5) Последовательности и графы де Брёйна. Случай алфавита 0,1 и подслов произвольной длины. Правило 0 лучше 1 (б/д).}
\includegraphics[width=1\linewidth]{sections/Polina/imgs/17.jpg}
\newpage{}

\section{(6) Гамильтоновы пути и циклы. Достаточное условие Дирака гамильтоновости графа.}
\subsection{6. Минимальный ДКА, его единственность.}
% 1) Пусть $q_1, q_2 \in [q]$. Рассмотрим произвольное $w \in \Sigma^*$. Мы знаем, что $\Delta(q_1, w) \in F \Longleftrightarrow \Delta(q_2, w) \in F$
%  Пусть $p_1 = \Delta(q_1, w)$, $p_2 = \Delta(q_2, w)$. 
 
% Докажем, что $p_1 \sim_M p_2$. Пусть это не так, тогда (без ограничения общности)
% $\exists v \in \Sigma^*: \Delta(p_1, v) \in F, \Delta(p_2, v) \not\in F$. Тогда заметим, что неверно $\Delta(q_1, wv) \in F \Longleftrightarrow \Delta(q_2, wv) \in F$, то есть $q_1 \nsim_M q_2$. Противоречие.\\

% 2)  Пусть $q_1, q_2 \in [q]$. Рассмотрим произвольное $w \in \Sigma^*$. Мы знаем, что $\Delta(q_1, w) \in F \Longleftrightarrow \Delta(q_2, w) \in F$. Возьмем $w = \varepsilon$. Получим, что $q_1 \in F \Longleftrightarrow q_2 \in F$.\\

% Поэтому, если рассмотрим автомат на классах эквивалентности, он будет построен корректно. Уточнение: для него $\Delta([u], a) = [ua], u \in \Sigma^*, a \in \Sigma$.

% 3) Понятно, что любое слово из исходного языка читается, ведь мы добавляли ребро в новом автомате в случае, когда из состояний одного класса такие ребра шли в состояния другого класса. А еще завершающими состояниями являются классы, в которых лежат завершающие состояния. Поэтому каждому пути из стартовой вершины в завершающее состояние в исходном автомате соответствует путь из стартовой вершины в завершающую в новом автомате.\\

% Заметим теперь, что никакое новое слово приниматься не начало. Потому что для каждого ребра между классами состояний можно найти ребро с таким же словом между любыми двумя представителями этих классов (следует из первого пункта и способа построения автомата). Поэтому для любого принимаемого новым автоматом слова $w$ в исходном автомате можно построить путь из стартового состояния в завершающее, пройдя по которому, можно прочитать $w$.\\

% 4) В автомате, построенном на классах эквивалентности состояний никакие два состояния не эквивалентны, потому что тогда бы они лежали в одном классе, т.е. были бы одним состоянием.
% \EndProof

% Что необходимо показать?\\
% 1) Как его получить? (это будет в следующем билете)\\
% 2) Почему он существует?\\
% 3) Почему он единственный с точностью до изоморфизма?\\

% \Th $M$ -- минимальный ПДКА, распознающий язык $L$ тогда и только тогда, когда любые два состояния попарно неэквивалентны и все состояния достижимы из стартового.

% \Proof
% Состояния, не достижимые из стартовой вершины, очевидно, не нужны, и их можно убрать.

% Обозначим $q_1 = \Delta(q_0, w_1)$, $q_2 = \Delta(q_0, w_2)$.

% Заметим, что $\big(\Delta(q_1, w) \in F \Longleftrightarrow \Delta(q_2, w) \in F\big) \Longleftrightarrow \big(w_1w \in L \Longleftrightarrow w_2w \in L\big)$, поэтому $q_1 \sim_M q_2 \Longleftrightarrow w_1 \sim_L w_2$.

% (Посмотрим на определение $w_1 \sim_L w_2$: $\forall w \in \Sigma^*: w_1w \in L \Longleftrightarrow w_2w \in L$ Поймем, что это условие по смыслу соответствует условию эквивалентности состояний, ведь слово $w_1w$ лежит в языке в том и только в том случае, когда его можно прочитать из $q_0$ и попасть в состояние из $F$).

\Def $M$ -- минимальный ПДКА, распознающий язык $L$, если $M$ минимальный по количеству состояний.

В предыдущем билете мы доказали существование минимального ПДКА (это логически следует из леммы и теоремы).\\

\textit{Замечание 1:} в МПДКА $M = \langle Q, \ldots \rangle \; \hookrightarrow \; \arrowvert \Sigma^* /_{\sim_L} \arrowvert = \arrowvert Q \arrowvert$

\textit{Замечание 2:} $[u] \in \Sigma^* /_{\sim_L} \Longrightarrow [u] = \cup_{q}{L_q}$, где $L_q := \{w \,\arrowvert\, \Delta(q_0, w) = q\}$ в МПДКА. Тогда получаем, что $[u] \in \Sigma^* /_{\sim_L} \Longrightarrow \exists!\, q :\; [u]=L_q$.

\Th Для любого автоматного языка $L$ существует единственный с точностью до изоморфизма минимальный ПДКА $M$, такой что $L = L \brackets{M}$.

\Proof Пусть $M$ — минимальный ПДКА, $Q_M$ — множество его состояний.

Построим автомат $M_0 = \langle \Sigma^* /_{\sim_L}, \Sigma, \Delta, [\varepsilon], \{ [w] | w \in L \} \rangle$. Для любых $u \in \Sigma^*$, $a \in \Sigma$ верно, что $\Delta \brackets{[u], a} = [ua]$ (факт 1). Так как количество состоянии автомата, соответствующему языку $L$, конечно, то по следствию из леммы о классах эквивалентности и $L_q$: $|\Sigma^* /_{\sim_L}| < \infty$. 

Факт 1 верен вследствие следующего:

\begin{center}
    $u \sim_L v \Longrightarrow ua \sim_L va \Longleftrightarrow \forall w \brackets{uaw \in L \Longleftrightarrow vaw \in L}$
    
    $w' = aw$, $\forall w' \brackets{uw' \in L \Longleftrightarrow vw' \in L} \Longleftrightarrow u \sim_L v$
\end{center}

Так как $u \sim_L v$, то если $u \in L$, то $v \in L$, и наоборот (просто берем $w=\varepsilon$).

Теперь рассмотрим $\psi : Q_M \rightarrow \Sigma^* /_{\sim_L}$, $\psi \brackets{q} = \{ w | \Delta \brackets{q_0, w} = q \} = L_q$. Из замечаний $\Longrightarrow \, \psi $ -- взаимнооднозначное отображение. Более того, покажем, что $\psi$ — изоморфизм между автоматами, как графами. Для этого нужно показать, что:

\begin{enumerate}
    \item $\Delta \brackets{\psi(q), a} = [\psi\brackets{q}a]$;
    \item $q \in F \Longleftrightarrow \psi \brackets{q} \subseteq L$.
\end{enumerate}

Покажем, почему выполняется (1).

\begin{center}
    $\psi \brackets{q} = [u]$, $\psi \brackets{q'} = [u']$, $\Delta \brackets{q, a} = q'$
    
    $\Delta \brackets{\psi(q), a} = \Delta \brackets{\{ w | \Delta(q_0, w) = q \}, a} = \{ w' = wa | \langle q_0, w' \rangle = q'\} = [wa]$
    
    $[u] = [w]$, $[u'] = [wa]$ по транзитивности переходов в автомате
\end{center}

Покажем, почему выполняется (2). Из того, что $q \in F$, следует, что слова из множества $\psi \brackets{q} = \{ w | \Delta \brackets{q_0, w} = q \}$ принадлежат языку $L$, так они распознаются автоматом, поскольку $q$ является завершающим состоянием. А так как $\psi \brackets{q} = \{ w | \Delta \brackets{q_0, w} = q \} \subseteq L$, то так как они распознаются автоматом, соответствующему языку $L$, то $q \in F$.

Пусть $\psi_1$ — изоморфизм между минимальными ПДКА $M_1$ и $M_0$, $\psi_2$ — изоморфизм между минимальными ПДКА $M_2$ и $M_0$. Тогда $M_1$ и $M_2$ изоморфны между собой — этому соответствует изоморфизм $\psi_2^{-1} \circ \psi_1$, композиция изоморфизмов является изоморфизмом.
\EndProof
\newpage{}

\section{(7) Вершинная связность и число независимости графа. Признак Эрдёша-Хватала (б/д). Гамильтоновость графа 1-пересечений 3-элементных подмножеств n-элементного множества при всех дост. больших n.}
Пусть есть граф $G = (V, E)$, $W \subseteq V$.

\Def W - независимое, если $\forall x, y \in W: (x, y) \notin E$

\Def $\alpha (G) = $ мощность любого самого большого независимого $W$ - число независимости графа.

\Def $\varkappa (G)$ (буква читается каппа) Вершинная связность - min количество вершин, удаление которых приведёт к тому, что граф перестаёт быть связным.

\Th Признак Эрдёша - Хва'тала.

$\alpha (G) \leqslant \varkappa (G)$. Тогда G - гамильтонов.

\Example $V = \{ A \subset \{1, 2, \dots, n \} : |A| = 3\}$. Тогда $|V| = C_n^3 \sim \frac{n^3}{6}$. 

$E = \{ (A, B): |A \cap B| = 1\}$; $\forall A \in V$ $deg A = 3 C_{n-3}^2$ $\Rightarrow$ $|E| = \frac{3 C_{n-3}^2 C_n^3}{2} \sim \frac{n^3}{6} \cdot 3 \cdot \frac{n^2}{4} = \frac{n^5}{8}$

$\alpha(G)$: макс. число "троек", попарные пересечения которых не равны 1, т.е. 0 или 2.

\begin{equation*}
\alpha(G) \geqslant (=) 
 \begin{cases}
   n &\text{$n \equiv 0 (4)$}\\
   n-1 &\text{$n \equiv 1 (4)$} \\
   n-2 &\text{$n \equiv 2, 3 (4)$}
 \end{cases}
\end{equation*}
(Конструкция: разбиваем на четвёрки элементов, внутри каждой четвёрки берём все подмн-ва (они пересекаются по 2 элементам всегда), а разные четвёрки пересекаются по 0).

\Th $\alpha(G) \leqslant n$ для графа в примере.

\Proof Зафиксируем $W = \{ A_1, A_2, \dots, A_t \}$ - независимое подмножество. 

Сопоставим каждому $A_i \rightarrow \overline{x_i} = (x^1, \dots, x^n) \in \Z_2^n, x^j \in \{0, 1\}$, причём 1 стоят на позициях, которые принадлежат множеству $A_i$.

$|A_i \cap A_j| = (\overline{x_i}, \overline{x_j})$. Осталось доказать, что $x_1, \dots, x_t$ ЛНЗ, тогда очевидно, что $|W| = t \leqslant n$, в силу размерности пространства.

Докажем, что $x_1, \dots, x_t$ ЛНЗ в $\Z_2^n$. $c_1x_1 + c_2x_2 + \dots + c_tx_t = \overline{0}$. Домножим обе части равенства на $x_1$, получим: $c_1(x_1, x_1) + c_2 (x_2, x_1) + \dots + c_t (x_t, x_1) = 0$; но все скалярные произведения, кроме $(x_1, x_1)$, равны 0 в $\Z_2$, т.к. они равны либо 0, либо 2, а $(x_1, x_1) \equiv 1 (2)$, значит, $c_1 = 0$. Так можно доказать для любого коэффициента, т.е. это тривиальная линейная комбинация. \EndProof

Оценка на $\varkappa(G)$. Зафиксируем вершины $x, y$, $f(x, y)$ - количество их общих соседок. Тогда $\varkappa(G) \geqslant min_{x, y} f(x, y)$ для любого графа (очевидно). 

Для нашего графа G если $x \cap y = 0$, то $f(x, y) = 3 \cdot 3 \cdot (n - 6)$, т.к. нужно, чтобы из первого множества был один из 3х элементов, из 2-ого один из 3х элементов, а 3й элемент отличен от элементов из этих 2х мн-в.

Если $x \cap y = 1$, то $f(x, y) = C_{n-5}^2 + 2 \cdot 2 \cdot (n - 5)$, первое слагаемое - это если "соседи" содержат элемент, по которому пересечение, второе - не содержат.

Аналогично, если $x \cap y = 2$, то $f(x, y) = 2 \cdot C_{n-4}^2 + n - 4$.

Очевидно, что при больших n минимально $9(n-6) > n \geq \alpha(G)$, значит, такие графы гамильтоновы.
\newpage{}

\section{(7) Вершинная связность и число независимости графа. Признак Эрдёша-Хватала.}
Во всём документе $G=(V, E)$ $-$ зафиксированный граф.

\Def Если $W \subseteq V$ и $\forall x, y \in W((x, y) \notin E)$, то $W$ $-$ \textit{независимое множество}.

\Def \textit{Число независимости графа $G$} $-$ наибольшая мощность независимого множества $G$, обозначается $\alpha(G)$.

\Def множство $W \subseteq V$ \textit{развязывающим}, если $G \Big|_{V \setminus W}$ не связен. \textit{Вершинная связность $G$} $-$ мощность наименьшего развязывающего множества, обозначается $\varkappa(G)$.

\textbf{Теорема(Эрдёша-Хватала):} Если $\alpha(G) \leqslant \varkappa(G)$, то $G$ $-$ гамильтонов.
\begin{wrapfigure}{r}{0.20\linewidth}
				\includegraphics[scale = 0.40]{images/E-H_graph.png}
				\caption{Иллюстрация графа}
\end{wrapfigure} \\
\Proof Предположим противное: в таком графе нет гамильтонового цикла. Тогда рассмотрим любой самый длинный простой цикл $C=\{x_1, ..., x_k\}$. Рассмотрим граф $G'=G \big |_{V \setminus C}$. В $G'$ рассмотрим произвольную компоненту связности $W$. 

\textbf{Множество соседей $W$ в графе $G$} $-$ $N_W(G)= \newline \left \{ y \in V \setminus W : \exists x \in W\left( (x, y) \in E \right) \right \}$.

\underline{Утв. 1}: $N_W(G) \subset C$.

\underline{Утв. 2}: Соседние вершины $C$ не могут принадлежать $N_W(G)$ одновременно (иначе можно удлинить цикл).

\underline{Утв. 3}: $\varkappa(G) \leqslant |N_W(G)|$.

\underline{Утв. 4}: Назовём $M=\{x_{i+1}: x_i \in N_W(G) \}$. По утв. 2 \newline
$M \cap N_W(G) = \varnothing$ и $|M|=|N_W(G)|$. Тогда $M$ $-$ независимое множество вершин. \\
Док-во утв. 4: Предположим противное: в $M$ есть рёбра между вершинами. Тогда выберем в цикле вершины $x_i, x_{i+1}, x_j, x_{j+1}$, при этом между $x_{i+1}$ и $x_{j+1}$ есть ребро.
Тогда можно удлинить цикл: вместо $x_1\!\! \shortrightarrow\!\! x_2\!\! \shortrightarrow\!\! ...\!\! \shortrightarrow\!\! x_i\!\! \shortrightarrow\!\! x_{i+1}\!\! \shortrightarrow\!\! ...\!\! \shortrightarrow\!\! x_j\!\! \shortrightarrow\!\! x_{j+1}\!\! \shortrightarrow\!\! ...\!\! \shortrightarrow\!\! x_k\!\! \shortrightarrow\!\! x_1$ новый цикл $x_1\!\! \shortrightarrow\!\! x_2\!\! \shortrightarrow\!\! ...\!\! \shortrightarrow\!\! x_i\!\! \shortrightarrow\!\! a\!\! \shortrightarrow\!\! ...\!\! \shortrightarrow\!\! b\!\! \shortrightarrow\!\! x_j\!\! \shortrightarrow\!\! ...\!\! \shortrightarrow\!\! x_{i+1}\!\! \shortrightarrow x_{j+1}\!\! \shortrightarrow\!\! ...\!\! \shortrightarrow\!\! x_k\!\! \shortrightarrow\!\! x_1$ ($a, b \in W$). будет как минимум на 1 вершину длиннее, что противоречит предположению, что $C$ - наидлиннейший.

\underline{Утв. 5}: Пусть $x \in W$. Тогда $M \bigcup {x}$ $-$ независимое множество.

Тогда из утв. 5 $\alpha(G) \geqslant |M|+1$, но из утв. 3 $\varkappa(G) \leqslant |M|$. Значит, $\varkappa(G) < \alpha(G)$, что противоречит условию теоремы. \EndProof
\begin{center}
    \includegraphics[scale = 0.40]{images/E-H_cycle_lengthening.png}
\end{center}
\newpage{}


\section{(5) Теорема Турана о числе ребер в графе с данным числом вершин и числом независимости.}
\subsection{7. Минимальный ДКА, алгоритм построения.}
Мы хотим преобразовать автомат так, чтобы его состояния были попарно неэквивалентны. Но у нас есть только слова. Как быть? Введем эквивалентность по словам фиксированной длины.

\Def $q_1 \underset{n}{\sim} q_2$, если для любого слова $w$ : $\arrowvert w \arrowvert \leq n$:
$$\Delta(q_1, w) \in F \Longleftrightarrow \Delta(q_2, w) \in F$$.
Это отношение эквивалентности (аналогично своему $M$-аналогу), поэтому можно ввести $Q/_{\underset{n}{\sim}}$.\\

\Lemma $q_1 \underset{\arrowvert Q \arrowvert - 2}{\sim} q_2 \Longrightarrow q_1 \sim q_2  $

\Proof Если $q_1 \underset{i + 1}{\sim} q_2 \Longrightarrow q_1 \underset{i}{\sim} q_2$, тогда $\arrowvert Q /_{\underset{i}{\sim}} \arrowvert \leqslant \arrowvert Q /_{\underset{i + 1}{\sim}} \arrowvert$.

Покажем, что если $\arrowvert Q /_{\underset{i}{\sim}} \arrowvert = \arrowvert Q /_{\underset{i + 1}{\sim}} \arrowvert$, то $\arrowvert Q /_{\underset{i + 1}{\sim}} \arrowvert = \arrowvert Q /_{\underset{i + 2}{\sim}} \arrowvert$. Пусть существуют состояния $q_1$ и $q_2$. такие что $q_1 \underset{i + 2}{\nsim} q_2$, $q_1 \underset{i + 1}{\sim} q_2$.

Тогда $\exists u, \arrowvert u \arrowvert \leqslant i + 2$, что без ограничения общности $\Delta(q_1, u) \in F$, $\Delta (q_2, u) \notin F$. Заметим, что должно быть верно, что $\arrowvert u \arrowvert = i + 2$, иначе возникнет противоречие с $(i + 1)$-эквивалентностью. Тогда пусть $u = av$, где $a$ -- некоторый символ. 

Рассмотрим $p_1 = \Delta(q_1, a), p_2 = \Delta(q_2, a) \Longrightarrow \Delta(p_1,v)\in F, \; \Delta(p_2,v)\notin F$, при этом $|v| \leqslant i + 1$

Тогда $p_1 \underset{i + 1}{\nsim} p_2$ так как существует слово длины $i+1$, которое их различает $\Longrightarrow p_1 \underset{i}{\nsim} p_2 \Longrightarrow q_1 \underset{i + 1}{\nsim} q_2$ Противоречие.

Отсюда следует, что если $\arrowvert Q /_{\underset{i}{\sim}} \arrowvert = \arrowvert Q /_{\underset{i + 1}{\sim}} \arrowvert$, то $\arrowvert Q /_{\underset{i + 1}{\sim}} \arrowvert = \arrowvert Q /_{\underset{i + 2}{\sim}} \arrowvert$. И так для всех $i$. Поэтому если мы остановились, новые классы эквивалентности не появятся. Понятно, что мы можем увеличить класс эквивалентности O($|Q|$) раз

\EndProof

\textit{Поэтому наш алгоритм выглядит так:}

Множество $Q /_{\sim_0}$ представляет собой $\{ F, Q \setminus F \}$, то есть это множество из множества завершающих состояний и множества состояний, не являющихся завершающими.

До тех пор, пока количество классов эквивалентности меняется, увеличиваем длину слов, по которым проверяем эквивалентность.

Пусть у нас была длина $n$, мы перешли к $n + 1$. Рассмотрим какой-то класс эквивалентности. 

Посмотрим на переход по первой букве. Заметим тогда, что нам останется пройти $n$ символов. Все эти вершины (на расстоянии $1$) мы уже распределили по классам эквивалентности на основе слов длины $n$, поэтому нужно для каждого сотояния найти множество классов эквивалентности его соседей. Если множества различаются, то эти состояния теперь разойдутся по разным классам. Совпадают, значит, останутся в одном классе. 

В соответствии с этим получим новое распределение состояний по классам.


\newpage{}

\section{(4) Соотношения между хроматическим числом, числом независимости и кликовым числом.}
\includegraphics[width=1\linewidth]{sections/Polina/imgs/5.jpg}
\newpage{}

\section{ (4) Числа Рамсея R(s, t): точные значения для s + t $\leq$ 7. Рекуррентная верхняя оценка Эрдёша–
Секереша}
\textbf{Определение} \textit{Число Рамсея R(n,m)}  — наименьшее из таких чисел $x \in N$, что при любой раскраске ребер полного графа на $x$ вершинах в два цвета найдется клика на $n$ вершинах с ребрами цвета 1 или клика на $m$ вершинах с ребрами цвета 2
\\
\includegraphics[]{polina_1.PNG}
\begin{itemize}
    \item [1] $R(1,t) = 1$
    \item[2] $R(2,t) = t$
    \item[3] $R(3,3) = 6$ \\
    Возьмем какую-нибудь вершину графа на 6 вершинах. Тогда по приницпу Дирихле из нее выходит либо 3 ребра первого цвета, либо три ребра второго цвета. БОО пусть первого цвета. Тогда рассмотрим треугольник, образованный концами этих ребер. 
    
    Либо у него есть ребро первого цвета, и тогда мы нашли треугольник первого цвета
    
    Либо таких ребер нет, тогда мы нашли треугольник второго цвета.
    
    Пример, почему не можем взять меньше вершин:\\
    \includegraphics[]{polina_2.png}
    \item[4] То, что $R(4,3) \leq R(3,3)+R(4,2) - 1 = 9$
    Почему 8 вершин нельзя:\\
    \includegraphics[]{polina_3.PNG}
\end{itemize}
\textbf{Теорема}
\\
$\forall n, m \in N \  \exists \ R(n,m)$, при этом $R(n,m)\leq R(n - 1,m) + R(n,m - 1)$. Если $R(n - 1,m), R(n,m - 1)$ - четны, то верно $R(n,m)\leq R(n - 1,m) + R(n,m - 1) - 1$
\\
\\
$\blacktriangle$ 
Обозначим правую часть неравенства за n. Зафиксируем произвольный граф на n вершинах, и докажем, что в таком графе найдется либо n-клика красного цвета, либо m-клика синего цвета. 

Возьмем какую-нибудь вершину графа. Тогда из нее выходит либо $R(n - 1,m)$ ребер первого цвета, либо  $R(n ,m - 1)$ ребер второго цвета. БОО пусть первого цвета, тогда концы ребер - граф на  $R(n-1 ,m)$ вершинах, тогда в нем найдется либо n-1 клика 1 цвета, либо m-клика 2 цвета, и мы все нашли. Второй случай аналогично 
\\
\\
Для четных значений достаточно заметить, что найдется вершина, у которой либо $R(n - 1,m)$ ребер первого цвета, либо  $R(n ,m - 1)$ ребер второго цвета, потому что иначе ребер красного цвета у каждой вершины $R(n - 1,m)$ - 1, то есть нечетное число, при этом всего вершин так же нечетно, что противоречит лемме о рукопожатиях  $\blacksquare$

\section{(4) Следствие рекуррентной верхней оценки Эрдёша–Секереша для недиагональных и диагональных чисел Рамсея. Уточнение Конлона (б/д). Нижняя оценка диагональных чисел Рамсея с
помощью простого вероятностного метода.}
\textbf{Следствие рекуррентной верхней оценки Эрдёша–Секереша}
\begin{center}
$R(s,t) \leq C_{s+t-2}^{s-1}$ 
\end{center}
$\blacktriangle$ Докажем по индукции
\begin{itemize}
    \item [1] $R(1,t) = C_{t-1}^{1-1} = 1$
    \item[2] $R(s,t) \leq R(s-1, t) + R(s, t-1) \leq C_{s+t-3}^{s-2} + C_{s+t-3}^{s-1} = C_{s+t-2}^{s-1}$ $\blacksquare$
\end{itemize}

\textbf{Для диагональных чисел Рамсея}

\begin{center}
    $R(s,s) \leq C_{2s-2}^{s-1} \sim \frac{4^{s-1}}{\sqrt{\pi s}}$ 
\end{center}Доказывается с использованием формулы Стирлинга
\\
\textbf{Теорема (Уточнение Конлона)}
$R(s,s) \leq e^{-\gamma\frac{ln^2(s)}{lnln(s)}}4^s, \gamma > 0$
\\
\\
\textbf{Теорема (Нижняя оценка диагональных чисел Рамсея)}
\begin{center}
    $R(s,s) \geq (1 + \overline{\overline{o}}(1))\frac{s}{e\sqrt{2}}2^{\frac{s}{2}}$
\end{center}
$\blacktriangle$ Рассмотрим случайный граф $G(n, \frac{1}{2})$.  $A_1, ..., A_{C_{n}^s}$ - все s-элементные подмножества множества вершин графа. Рассмотрим событие $E_i$ - $A_i$ образует клику или независимое множество.
\\
$\mathbb{P}(E_i) = 2(\frac{1}{2})^{C_s^2}$
\\
$\mathbb{P}(\bigcup_{i=1}^{C_n^s}E_i) \leqslant \sum\limits_{i=1}^{C_n^s}\mathbb{P}(E_i) = C_n^s \cdot 2^{1-C_s^2} \sim \frac{n^s}{s!}2^{1-C_s^2} 
= \frac{n^s}{s!} 2^{1-s^2/2 + s/2 } 
=\Big / \text{Подставим n как } (1 + \overline{\overline{o}}(1)) \frac{s}{e \sqrt{2}} 2^{\frac{s}{2}} \Big / = (1+\overline{\overline{o}}(1))^s\frac{1}{e^s 2^{s/2}}\frac{s^s 2^{1+s/2}}{\sqrt{2\pi s}(\frac{s}{e})^s}\frac{1}{(1 + \overline{\overline{o}}(1))} = \frac{(1+\overline{\overline{o}}(1))^s}{(1 + \overline{\overline{o}}(1))}\frac{2}{\sqrt{2\pi s}}$
\\
Подбором $\overline{\overline{o}}(1)$ в числителе можно сделать так, чтобы полученное число было < 1 при любом s. Тогда вероятностным методом получим, что найдется граф, в котором при данном n не найдется ни s-клики, ни s-независимого множества. $\blacksquare$
\\
\\
\textbf{Теорема}
\begin{center}
    $\forall n \  R(s,s) \geqslant n - C_n^s 2^{1-C_s^2}$
\end{center}
$\blacktriangle$ Рассмотрим случайный граф $G(n, \frac{1}{2})$. $A_i, E_i$ - те же, что в предыдущей теореме. Введем $X(G)$ - число таких $A_i$, которые образуют в G либо клику, либо независимое множество
\\
$\mathbb{E} X =  C_n^s \cdot 2^{1-C_s^2}$, тогда существует граф на n вершинах, для которого $X(G) \leqslant C_n^s \cdot 2^{1-C_s^2} $
\\
Зафиксируем G  и удалим по одной вершине из каждой s-клики и каждого независимого множества размера s. После удаления в графе осталось $\geqslant n - C_n^s 2^{1-C_s^2}$ вершин, причем мы разрушили все s-клики и s-независимые множества, поэтому их нет, и появиться они не могли $\blacksquare$
\\
\\
\textbf{Следствие}
\begin{center}
    $R(s,s) \geqslant (1+\overline{\overline{o}}(1))\frac{s}{e}2^{\frac{s}{2}}$
\end{center}
$\blacktriangle$ Аналогично теореме 1. $C_n^s2^{1-C_s^2} = \frac{2}{\sqrt{2\pi s}}\frac{(1+\overline{\overline{o}}(1))^s}{(1+\overline{\overline{o}}(1))}2^{s/2}; \\ R(s,s) \geq n - C_n^s2^{1-C_s^2} = (th 2) = (1+\overline{\overline{o}}(1)) \frac{s2^{s/2}}{e} - \frac{2}{\sqrt{2\pi s}}\frac{(1+\overline{\overline{o}}(1))^s}{(1+\overline{\overline{o}}(1))}2^{s/2}$. Уменьшаемое асимптотически больше, чем вычитаемое, поэтому $= (1+\overline{\overline{o}}(1)) \frac{s2^{s/2}}{e} \ \blacksquare$
\section{(7) Двудольные числа Рамсея. Верхняя оценка Конлона для двудольных чисел Рамсея: $2
^k k(1+\overline{\overline{o}}(1))$
c доказательством, $2
^{k+1} log_2 k(1 + \overline{\overline{o}}(1))$ - формулировка}

см. ниже

\section{(10) Двудольные числа Рамсея. Верхняя оценка Конлона для двудольных чисел Рамсея: $2
^k k(1+\overline{\overline{o}}(1))$
- формулировка, $2
^{k+1} log_2 k(1 + \overline{\overline{o}}(1))$ с доказательством}
\textbf{Определение} $b(s,t) = min\{\ n \in \N: $ при любой раскраске ребер $K_{n,n}$ в красный и синий цвета либо существует красный пограф $K_{s,s}$ , либо синий  $K_{t,t}\}$
\\
\textbf{Определение} Рассмотрим граф $K_{m,n}$. Будем считать, что меньшая его доля m, верхняя, а n - нижняя. Рассмотрим произвольный подграф G. Определим плотность графа G = $\frac{|E(G)|}{mn}$
\\
\\
\textbf{Лемма}
\\
Если числа $n,m,p,r,s$ таковы, что $C_m^r(s-1) < nC_{mp}^r$. Тогда для любого подграфа G плотности p в G есть $K_{r,s}$
\\
\Proof
Предположим, в G нет $K_{r,s}$. Посчитаем в G $K_{r,1}$
\begin{itemize}
    \item [1] Их $\leqslant C_m^r(s-1)$. Для каждого r-элементного множества верхней доли не может быть s множества нижней доли, иначе мы бы нашли  $K_{r,s}$ 
    \item [2] Пусть $d_1, ..., d_n$ - степени вершин G в нижней доле. Тогда получаем, что количество $K_{r,1}$ = $C_{d_1}^r + ... + C_{d_n}^r \geqslant nC_{\frac{d_1 + ... + d_n}{n}}^r \geqslant nC_{mp}^r$  (в силу выпуклости бином коэфф)
\end{itemize}
\EndProof
\\
\textbf{Теорема}
\begin{center}
    $b(k,k) \leq 2^kk(1+o(1))$
\end{center}
\Proof Положим n = $2^kk(1+\varepsilon), \varepsilon > 0$. Рассмотрим произвольную раскраску ребер  $K_{n,n}$ в синий и красный цвета. Выберем два графа. Один, у которого все ребра красные, а другой - все ребра синие. Тогда один из этих графов имеет p $\geq 1/2$ в $K_{n,n}$. ББО пусть красный.
\\
Положим r=s=k, m=n. Тогда очень хочется\\
$C_n^k(k-1) < n*C_{n/2}^k$
\\
$C_n^k \sim \frac{n^k}{k!}, C_{n/2}^k \sim \frac{(n/2)^k}{k!} \Longrightarrow (1 +o(1)) \frac{n^k}{k!}(k-1) \ vs\  n* \frac{(n/2)^k}{k!} (1 +o(1))  \Longrightarrow  (1 +o(1))(k-1) \ < \ \frac{n}{2^k}(1+o(1))$, значит, выполняется условие леммы, то есть мы найдем красный подграф $K_{n,n}$, откуда $b(k,k) \leq n$
\EndProof
\\
\textbf{Теорема}
\begin{center}
    $b(k,k) \leq 2
^{k+1} log_2 k(1 + o(1))$
\end{center}
\Proof Положим n = $2
^{k+1} log_2 k(1 + \varepsilon), \varepsilon > 0$ и зафиксируем некоторую раскраску $K_{n,n}$. Рассмотрим вершины второй доли. Назовем вершину из второй доли красной, если из нее выходит красных
ребер больше, чем синих (а иначе - синей). Без ограничения общности считаем, что красных вершин $\geq n/2$
\\
Рассмотрим красный граф $G_{n, n/2}$, где $n/2$ отвечает множеству красных вершин из второй доли. Из определения красной вершины, плотность $G  \geq 1/2 = p$
\\
В G есть $K_{k - 2log_2k, k^2log_2k}, k - 2log_2k = r_1, k^2log_2k = r_2$. 
\\
Проверим по лемме:
\\
\\
$\frac{n}{2}C_{n/2}^{r_1} > (r_2 - 1)C_{n}^{r_1} \Longrightarrow (1 +\varepsilon)(log_2k)2^kC_{n/2}^{k-2log_2k} > (k^2log_2k - 1)C_{n}^{k - 2log_2k} \Longrightarrow (1 + \varepsilon)(1 +o(1))log_2k2^{2log_2k} > (1 +o(1))k^2log_2k \Longrightarrow (1 + \varepsilon)(1 +o(1))k^2 > (1 +o(1))k^2log_2k \Longrightarrow (1 + \varepsilon)(1 +o(1)) > (1 +o(1))log_2k$ - все хорошо.
\\
Пусть $m_1 = n, n_1 = n/2$. По лемме мы нашли в графе $G_{n, n/2}$ подграф $K_{k - 2log_2k, k^2log_2k}$ 
\\
\\
Теперь что мы хотим сделать. У нас есть две сардельки - верхняя и нижняя, которые образуют найденный полный двудольный граф на красных ребрах. Теперь, если мы найдем вне верхней сардельки сардельку на $2log_2k$ элементах, которая образует с подсарделькой размера k нижней сардельки полный двудольный граф на красных ребрах, то мы, собственно, победили, и нашли полный двудольный граф на k вершинах, где верхняя доля - объединение верхних сарделек с рисунка, а нижняя - та самая подсарделька
\\
\includegraphics[width=12cm]{polina_5.PNG}
\\
\\
Пусть $m_2 = k^2log_2k, n_2 = (n - (k - 2log_2k))$
\\
Из каждой вершины верхней доли $G_{m_2, n_2}$ в нижнюю долю этого графа выходят $\geq \frac{n}{2} - (k-2log_2k)$ ребер\\
$p(G_{m_2,n_2}) \geq \frac{n/2 - k + 2log_2k}{n - k + 2log_2k} \geq \frac{n/2 - k}{n} = 1/2 - k/n = p$
\\
Снова воспользуемся леммой\\
$n_2C_{m_2/2 - m_2k/n}^{k} > (2log_2k - 1)C_{m_2}^{k} \\(1+o(1))(1+\varepsilon)log_2k2^{k+1}\frac{(k^2log_2k(1/2 - k/n))^k}{k!} > (1+o(1))(2log_2k)\frac{(k^2log_2k)^k}{k!}$
\\
\textbf{Замечание 1} Мы имели право так асимптотически оценивать $C_a^b \sim \frac{a^b}{b!}$, если $a = o(b)$. Мы прошлись по грани....
\\
\textbf{Замечание 2} $(\frac{1}{2} - \frac{k}{n})^k = (1/2)^k(1 - \frac{2k}{n}) \sim (1/2)^k$ 
\\
$(1+o(1))(1+\varepsilon) > (1+o(1))$
\\
\\
Значит, можно воспользоваться леммой, и найдется в точности подграф $K_{k,k}$
\EndProof
\newpage{}


\section{(8) Конструктивная нижняя оценка Франкла–Уилсона для $R(s, s)$. Лемма для кликового числа без доказательства.}
\section{(8) Конструктивная нижняя оценка Франкла–Уилсона для $R(s, s)$. Лемма для числа независимости без доказательства.}
\setcounter{section}{16}

\section{Отношения на множествах. Свойства бинарных отношений. Отношения эквивалентности, теорема о классах эквивалентности}
\par Любое свойство можно отождествить с множеством всех объектов, которые им обладают. Например, свойство чётности соответсвует множеству чётных чисел.
\par \textbf{Свойством элементов} множества $A$ называется любое подмножество $A$ или, что тоже самое, любая функция из $A$ в $\{0,1\}$.
\par  \textbf{Бинарным отношением} на множестве $A$ называется любое подмножество $A^2 = A \times A$ или, что тоже самое, любая функция из $A^2$ в $\{0,1\}$. Обозначения: $(x,y) \in R \mbox{ или } R(x,y) = 1 \mbox{ или } xRy$.
\par \textbf{Предикатом валентности $k$} на множестве $A$ называется подмножество $A^k$ или, что тоже самое, функция из $A^k$ в $\{0,1\}$.
\subsection*{Классификация отношений:}
\begin{enumerate}
    \item рефлексивные $\forall x \; xRx \quad\color{ForestGreen}\mbox{ ( = \, $\leqslant$ \, $\svdots$ \, $\subset$ \, $\cong$ )} $ 
    \item антирефлексивные $\forall x \; \neg(xRx) \quad\color{ForestGreen}\mbox{ ( < )} $ 
    \item симметричные $\forall x,y \;\; xRy \to yRx \quad\color{ForestGreen}\mbox{ ( = \, $\cong$ \, $\bmod$ \, $\|$ )} $ 
    \item антисимметричные $\forall x,y \;\; (xRy \land yRx) \to (x=y) \quad\color{ForestGreen}\mbox{ ( < \, $\leqslant$ \, $\svdots$ \, $\subset$ )} $
    \item транзитивные $\forall x,y,z \;\; (xRy \land yRz) \to xRz \quad\color{ForestGreen}\mbox{ ( = \, < \, $\svdots$ \, $\subset$ \, $\cong$ )} $ 
    \item антитранзитивные $\forall x,y,z \;\; (xRy \land yRz) \to \neg(xRz) \quad\color{ForestGreen}\mbox{ ( $\bot$ на плоскости )} $ 
    \item евклидово (правое) $\forall x,y,z \;\; (xRy \land xRz) \to yRz \quad\color{ForestGreen}\mbox{ ( нетразитивное R = \{(1,2),(2,2),(2,3),(3,2),(3,3)\} )} $ 
\end{enumerate}    

\subsection*{Наборы свойств:}
\begin{itemize}
    \item отношение эквивалентности: \textit{рефлексив. + симметрич. + транзитив.}
    \item отношение нестрогого (частичного) порядка: \textit{рефлексив. + антисимметрич. + транзитив.}
    \item отношение строгого (частичного) порядка: \textit{антирефлексив. + антисимметрич. + транзитив.}
    \item отношение (нестрогого) предпорядка: \textit{рефлексив. + транзитив.}
\end{itemize}

\subsection*{Отношения эквивалентности}
Примеры: $= \, \bmod \, \sim $ (подобие тругольников) $\, \| $ (паралленльность или совпадение прямых) и тд


Общий пример: задана $F: A \to B$ \quad $x \sim y$, если $f(x) = f(y)$


Тогда: 
\begin{flushleft}
    \hspace{10mm}\fbox{=} $f: A \to A \quad f(x) = x$
    
    \hspace{10mm}\fbox{$\bmod k$} $f$ возвращает остаток при делении на $k$
    
    \hspace{10mm}\fbox{$\|$} $f$ возвращает направление (элемент проективной прямой)
    
    \hspace{10mm}\fbox{$\sim$} $f$ возвращает форму (3 угла, упорядоченных по неубыванию, в сумме дающие $180^{\circ}$)
\end{flushleft}

Пусть на множестве $A$ задано отношение $\sim$ \,, тогда \textbf{классом эквивалентности} элемента $x \in A$ называется множество $K_x = \{y \; | \; y\sim x\}$
\\ \par \textbf{Теорема. (Основная теорема об отношениях эквивалентности) } \par Если на множестве задано отношение эквивалентности, то все множество разбивается на классы эквивалентности (т.е. представляется в виде такого объединения непересекающихся подмножеств, что два элемента эквивалентны тогда и только тогда, когда лежат в одном и том же подмножестве).
\par Иначе говоря, если на множестве $A$ задано отношение эквивалентности, то $A = \cup_{i \in I}A_i$ таких, что 
\begin{enumerate}
    \item если $x \in A_i$, то $A_i = K_x$
    \item если $x \in A_i$ и $y \in A_i$, то $x \sim y$
    \item при $i \neq j$ если $x \in A_i$ и $y \in A_j$, то $\neg(x \sim y)$
\end{enumerate}
Искомая функция (отношение эквивалентности) отображает $x$ в $K_x$ и тем самым делит множество на классы эквивалентности.
\newline $\blacktriangleright$
Так как отношение эквивалентности состоит из рефлексивности, симметричности и транзитивности, то получаем
\begin{enumerate}
\setcounter{enumi}{-1}
    \item из рефлексивности ($\forall x \; xRx$) следует, что $x \in K_x$, т.е. каждый элемент лежит в своем классе эквивалентности
    \item если $y \in K_x$ и $z \in K_x$, то $y \sim z$ получаем следующими рассуждениями. Так как $y \in K_x \rightarrow y \sim x$, $z \in K_x \rightarrow z \sim x$, т.е. $x \sim z$. Тогда по транзитивности $y \sim z$
    \item если $z \in K_x$, $z \in K_y$, то $ K_x = K_y$ получем следующими рассуждениями. Так как $z \in K_x \rightarrow z \sim x$, т.е. $x \sim z$, $z \in K_y \rightarrow z \sim y$. Тогда по транзитивности $x \sim y$. \newlineПусть $t \sim K_x$. Тогда $t \sim x$, по транзитивности $t \sim y$. Таким образом, $K_x \subset K_y$ и $K_y \subset K_x$, следовательно $K_x=K_y$. По контрапозиции если $K_x \neq K_y$, то $K_x \cap K_y = $ \O.
    \item если $K_x \neq K_y$, $z \in K_x$, $t \in K_y$, то $\neg(z \sim t)$ так как, если бы $z \sim t$, то по транзитивности $z \sim y$, откуда $z \in K_x \cap K_y$, что противоречит предыдущему пункту. Следовательно, $\neg(z \sim t)$. $\blacksquare$
\end{enumerate}

\textbf{Задача:} Являются ли следующие отношения отношениями эквивалентности? Если да, то укажите, на какие классы разбиваются соответсвующие множества.
\begin{enumerate}
    \item[(a)] $|x-y| \, \svdots \, k$, \, $x,y \in \mathbb{Z}, k \in \mathbb{N}, k > 0$
    \begin{itemize}
        \item[$\blacktriangle$] $\forall x \;|x-x| \, \svdots \, k$ {\color{ForestGreen}OK}; \quad $\forall x,y \;|x-y| \, \svdots \, k \;\; |y-x| \, \svdots \, k$ {\color{ForestGreen}OK}; \quad $\forall x,y,z \;|x-y| \, \svdots \, k \;\; |y-z| \, \svdots \, k$ имеют одинаковый остаток {\color{ForestGreen}OK}. Следовательно, классы эквивалентности по остаткам. $\blacksquare$
    \end{itemize}
    
    \item[(б)] $\{ ((x_1, y_1), (x_2,y_2)) \;\; | \;\; x_1-x_2 = y_1-y_2 \}$ на плоскости
    \begin{itemize}
        \item[$\blacktriangle$] Две точки на плоскости, удовлетворяющие заданному отношению, определяют прямую параллельную $y=x$. $\forall (x_1, y_1) \;0 = 0$ {\color{ForestGreen}OK}; \quad $\forall (x_1, y_1), (x_2,y_2) \; D_1 = D_2 \;$ и $\; -D_1=-D_2$ {\color{ForestGreen}OK}; \quad $\forall (x_1, y_1), (x_2,y_2), (x_3,y_3) \newline D_1=D_2 \;\; D_2=D_3$ и $D_1=D_3$ {\color{ForestGreen}OK}. Значит, классы эквивалентности прямых вида $y = x + b$. $\blacksquare$
    \end{itemize}
    
    \item[(в)] $|x-y| \, < \, 1$, \, $x \in \mathbb{R}$
    \begin{itemize}
        \item[$\blacktriangle$] $0 \, R \, \frac{2}{3}$ и $\frac{2}{3} \, R \, \frac{4}{3}$, но $0 \, R \, \frac{4}{3} \,$ {\color{Red}WA}. Следовательно, не является отношением эквивалентности. $\blacksquare$
    \end{itemize}
    
    \item[(г)] $\{ (AB, CD) \;\; | \;\; ABCD \, - $ параллелограмм, возможно вырожденный$\}$  на множестве направленных отрезков, возможно вырожденных, на плоскости
    \begin{itemize}
        \item[$\blacktriangle$] Рефлексивность очевидно выполнена {\color{ForestGreen}OK}; симметричность тоже {\color{ForestGreen}OK}; транзитивность для векторов тоже верна, так как возможны вырожденные случаи {\color{ForestGreen}OK}. Следовательно, классы эквивалентности параллелограммов. $\blacksquare$
    \end{itemize}

    \item[(д)] $x\|y$; $x\|y$ или $x=y$ на множестве всех прямых на плоскости
    \begin{itemize}
        \item[$\blacktriangle$] Если считать, что параллельные прямые - это прямые не имеющие общих точек, то рефлексивность не выполняется {\color{Red}WA}. Следовательно, первое не является отношением эквивалентности.
        \newline Во-втором случае рефлексивность уже выполнена {\color{ForestGreen}OK}; симметричность тоже {\color{ForestGreen}OK}; транзитивность для параллельности с совпадением тоже верна {\color{ForestGreen}OK}. Следовательно, классы эквивалентности направления. $\blacksquare$
    \end{itemize}
    
    \item[(е)] $x$ гомотетичен $y$; $x$ подобен $y$ на множестве всех треугольников на плоскости
    \begin{itemize}
        \item[$\blacktriangle$] Для гомотетии транзитивность неверна. В качестве примера можно рассмотреть треугольники X, Y, Z, где X = Z (со сдвигом в плоскости), а X и Y, Y и Z соответсвенно гомотетичны. Тогда транзитивности нет. {\color{Red}WA}. Следовательно, первое не является отношением эквивалентности.
        \newline Во-втором случае рефлексивность выполнена, так как треугольник подобен сам себе {\color{ForestGreen}OK}; симметричность тоже {\color{ForestGreen}OK}; транзитивность тоже выполнена по равенству углов треугольников (3 признак подобия) {\color{ForestGreen}OK}. Следовательно, классы эквивалентности наборов величин углов треугольника. $\blacksquare$
    \end{itemize}
    
    \item[(ж)] из $x$ существует путь в $y$ на множестве всех вершин некоторого графа.
    \begin{itemize}
        \item[$\blacktriangle$] Рефлексивность выполнена, так как существует нулевой путь из вершины в саму себя {\color{ForestGreen}OK}; симметричность тоже выполнена, так как граф не ориентирован {\color{ForestGreen}OK}; транзитивность тоже выполнена, так как можно произвести неформальное сложение путей {\color{ForestGreen}OK}. Следовательно, классы эквивалентности компоненты связности. $\blacksquare$
    \end{itemize}
    
    \item[(з)] последовательность $a_n - b_n$ бесконечно мала на множестве всех последовательностей рациональных чисел.
    \begin{itemize}
        \item[$\blacktriangle$] Рефлексивность выполнена, так как $\lim_{x\to\infty} (a_n-a_n) = 0$ {\color{ForestGreen}OK}; симметричность тоже выполнена, так как граф $\lim_{x\to\infty} (a_n-b_n) = \lim_{x\to\infty} (b_n-a_n) =0$ {\color{ForestGreen}OK}; транзитивность тоже выполнена, так как $\lim_{x\to\infty} (a_n-c_n) = \lim_{x\to\infty} (a_n-b_n+b_n-c_n) =0$ {\color{ForestGreen}OK}. Следовательно, классы эквивалентности действительных чисел, к которым стремятся последовательности. $\blacksquare$
    \end{itemize}
    
    \item[(з)] $f$ и $g$ равны в нуле; $f$ и $g$ равны в некоторой точке на множестве функций из $\mathbb{R}$ в $\mathbb{R}$
    \begin{itemize}
        \item[$\blacktriangle$] В первом случае рефлексивность очевидно выполнена {\color{ForestGreen}OK}; симметричность тоже {\color{ForestGreen}OK}; транзитивность тоже выполнена так как если $f(0)=g(0)=c$ и $g(0)=h(0)=d$, то $c=g(0)=d$ {\color{ForestGreen}OK}. Следовательно, классы эквивалентности по значению в нуле. 
        \newline Во-втором случае транзитивность не выполнена, так как одна функция может совпадать с другой в точке $a$, которая совпадает с третьей в точке $b$, но при этом не факт, что первая и третья функции совпадают в какой-либо точке {\color{Red}WA}. Следовательно, не является отношением эквивалентности. $\blacksquare$
    \end{itemize}
\end{enumerate}
\par \textbf{Задача:} Докажите, что ни одно требование в определении отношения эквивалентности не является лишним: например, существует симметричное, транзитивное, но не рефлексивное отношение.
\par $\blacktriangle$ Возьмем социальное отношение «быть знакомым с ...», или в более чёткой форме отношение «когда-либо встретиться с ...». Тогда рефлексивность выполнена, так как всякий человек встречался с собой {\color{ForestGreen}OK}; симметричность тоже выполнена, так как если один человек встретился с другим, то второй встретился с первым {\color{ForestGreen}OK}; но транзитивность не выполнена, так как  если один человек встретился с другим, который встретился с третьим, то не факт, что первый и третий встречались {\color{Red}WA}. Следовательно, существует симметричное, транзитивное, но не рефлексивное отношение. $\blacksquare$

\section{Отношения частичного и линейного порядка. Примеры отношений. Любое счётное упорядоченное множество можно доупорядочить линейно }
\par Как было указано ранее, отношение называется \textbf{отношением частичного порядка}, если оно рефлексивно, антисимметрично и транзитивно. При этом антисимметричность:  $\forall x,y \;\; (xRy \land yRx) \to (x=y)$
\\ \par \textbf{(Частично) упорядоченным множеством} (сокращенно ч.у.м.) называется пара $\langle A,\leq_A \rangle$ — множество и частичный порядок (отношение порядка) на нем.
\par Частичный порядок называется \textbf{линейным порядоком}, если любые два различных элемента множества сравнимы. (Диаграмма Хассе вырождается в цепочку, т.е. линию). Если порядок линеен, множество называется \textbf{линейно упорядоченным}.
\par Отношение называется \textbf{отношением строгого порядка}, если оно транзитивно и антирефлексивно.
\\ \par \textbf{Задача:} Проверьте, что следующие множества являются упорядоченными. Какие из них упорядочены линейно?
\begin{enumerate}
    \item[(a)] $\langle A, = \rangle$, $\langle 2^A, \subset \rangle$ где $A$ - произвольное множество.
    \begin{itemize}
        \item[$\blacktriangle$] $A=A$ {\color{ForestGreen}OK}; \quad $x=y$ {\color{ForestGreen}OK}; \quad $\big((x=y) \land (y=z)\big) \to (x=z)$ {\color{ForestGreen}OK}; \quad Очевидно, порядок не линеен. {\color{Red}WA}
        \newline Во-втором случае $2^A \subset 2^A$ {\color{ForestGreen}OK}; \quad $\big((x \subset y) \land (y \subset x)\big) \to (x = y)$ {\color{ForestGreen}OK}; \quad $x \subset y \subset z \to x \subset z$ {\color{ForestGreen}OK}; 
        \newline Если $A=\{0,1\}$, то \{0\} и \{1\} не сравнимы, следовательно, порядок не линеен. {\color{Red}WA} $\blacksquare$
    \end{itemize}
    
    \item[(б)] $\langle A, \leq \rangle$, $\langle A, \geq \rangle$ где $A$ - одно из множеств $\mathbb{N,Z,Q,R}$.
    \begin{itemize}
        \item[$\blacktriangle$] $x \leq x$ {\color{ForestGreen}OK}; \quad $\big((x \leq y) \land (y \leq x)\big) \to (x = y)$ {\color{ForestGreen}OK}; \quad $\big((x \leq y) \land (y \leq z)\big) \to (x \leq z)$ {\color{ForestGreen}OK}.
        \newline Аналогично с $\geq$. По очевидным причинам в обоих случаях это линейный порядок. {\color{ForestGreen}OK}  $\blacksquare$
    \end{itemize}
    
    \item[(в)] $\langle \mathbb{N}, \svdots \rangle$
    \begin{itemize}
        \item[$\blacktriangle$] $x \svdots x$ {\color{ForestGreen}OK}; \quad $\big((x \svdots y) \land (y \svdots x)\big) \to (x = y)$ {\color{ForestGreen}OK}; \quad $\big((x \svdots y) \land (y \svdots z)\big) \to (x \svdots z)$ {\color{ForestGreen}OK}.
        \newline Элементы 2 и 3 не сравнимы, следовательно, порядок не линеен. {\color{Red}WA} $\blacksquare$
    \end{itemize}
    
    \item[(г)] $\langle \mathbb{R}^2, \leq_{lex} \rangle$ где $(x_1,y_1) \leq_{lex}(x_2,y_2)$, если либо $x_1 < x_2$, либо $x_1 = x_2$ и $y_1 \leq y_2$
    \begin{itemize}
        \item[$\blacktriangle$] $(x,y)  \leq_{lex}(x,y)$ {\color{ForestGreen}OK}; \quad $\Big(\big((x_1,y_1) \leq_{lex}(x_2,y_2)\big) \land \big((x_2,y_2) \leq_{lex}(x_1,y_1)\big)\Big) \to \big((x_1=x_2) \land (y_1=y_2)\big)$ {\color{ForestGreen}OK}; \quad $\Big(\big((x_1,y_1) \leq_{lex}(x_2,y_2)\big) \land \big((x_2,y_2) \leq_{lex}(x_3,y_3)\big)\Big) \to \big((x_1,y_1) \leq_{lex}(x_3,y_3)\big)$ {\color{ForestGreen}OK}.
        \newline По очевидным причинам это линейный порядок. {\color{ForestGreen}OK}  $\blacksquare$
    \end{itemize}
    
    \item[(д)] $\langle \mathbb{R}^2, \leq \rangle$ где $(x_1,y_1) \leq (x_2,y_2)$, если $x_1 \leq x_2$ и $y_1 \leq y_2$
    \begin{itemize}
        \item[$\blacktriangle$] $(x,y) \leq(x,y)$ {\color{ForestGreen}OK}; \quad $\Big(\big((x_1,y_1) \leq(x_2,y_2)\big) \land \big((x_2,y_2) \leq(x_1,y_1)\big)\Big) \to \big((x_1=x_2) \land (y_1=y_2)\big)$ {\color{ForestGreen}OK}; \quad $\Big(\big((x_1,y_1) \leq(x_2,y_2)\big) \land \big((x_2,y_2) \leq(x_3,y_3)\big)\Big) \to \big((x_1,y_1) \leq(x_3,y_3)\big)$ {\color{ForestGreen}OK}.
        \newline Элементы $(2,3)$ и  $(3,1)$ не сравнимы, следовательно, порядок не линеен. {\color{Red}WA} $\blacksquare$
    \end{itemize}
\end{enumerate}
\par \textbf{Задача:} Докажите, что любое счётное упорядоченное множество можно доупорядочить линейно. (Т.е. для любого отношения порядка $R$ существует отношение линейного порядка $S$, такое что $R \subset S$.)
\par $\blacktriangle$ Занумеруем все пары, которые еще не упорядочены. Далее, рассмотрим первую пару таких $a$ и $b$, что $(a,b),(b,a) \notin R$. Построим новое отношение $R^\prime \; | \; R \subset R^\prime$, в котором положим $cR^\prime d$ для всех таких $c$ и $d$, что $(c,a),(b,d) \in R$. Проверим, что $R^\prime$ является отношением порядка.
\newline Рефлексивность выполнена, так как $R$ рефлексивно. Рассмотрим антисимметричность $(c,d),(d,c)$:
\begin{enumerate}
    \item если оба этих отношения из $R$, тогда $c=d$;
    \item если одно из $R$, а другое из $R^\prime$, тогда имеем $cRa$, $bRd$ и $dRc$, откуда по транзитивности $R$ выполнено $bRa$, т.е. эти элементы сравнимы в $R$ - противоречие;
    \item если оба этих соотношения из $R^\prime$, то имеем $cRa$, $bRd$, $bRc$ и $dRa$, и снова по транзитивности $aRb$ и $bRa$.
\end{enumerate}
Транзитивность $R^\prime$ проверяется аналогично.
\par Таким образом, $R^\prime$ - частичный порядок. Тогда выкинем из нашей нумирации все добавленные пары и рассмотрим следующую по номеру пару, которая не упорядочена. Продолжая этот процесс, мы получим линейный порядок $S$, содержащий данный. $\blacksquare$
\newpage{}


\section{(5) Многоцветные числа Рамсея и числа Рамсея для гиперграфов. Существование и рекуррентные верхние оценки.}
\subsection{19. Нормальная форма Грейбах для КС-грамматик. Модифицированный алгоритм Кока-Янгера-Касами для нормальной формы Грейбах: достоинства и недостатки.}

\Def Грамматика в нормальной форме Грейбах (grammar in Greibach normal form) — контекстно-свободная грамматика $\langle N, \Sigma, P, S \rangle$, в которой каждое правило имеет один из следующих четырёх видов:

1) $S \rightarrow \varepsilon$

2) $A \rightarrow a$

3) $A \rightarrow aB$

4) $A \rightarrow aBC$

причём $B, C \in N$, $B, C \neq S$, $a \in \Sigma$\\

\textbf{Модифицированный алгоритм Кока-Янгера-Касами}

Утверждение: проще парсить из нормальной формы Грейбаха, т.к. каждый раз мы вытаскиваем одну букву $\Rightarrow$ в алгоритме Кока-Янгера-Касами константа будет ниже; 

Минус: большое количество нетерминалов.\\


\textbf{Теорема} Каждая контекстно-свободная грамматика
эквивалентна некоторой грамматике в нормальной форме
Грейбах. (на хор 6 и выше по версии Виталия)

$\blacktriangle$
0. Возьмём G в НФ Хомского. Введём обозначение: $A \backslash B \vdash w \Leftrightarrow A \vdash Bw$

1. Заметим следующее:

\includegraphics[width=15cm]{images/greybach.JPG}

2. Вводим G' =  $\langle N', \Sigma, P', S \rangle$, где $N' = \{S\} \cup \{(A \backslash B) | A, B \in N\}$, 

а $P' = \{(A \backslash A) \rightarrow \varepsilon | A \in N \} \cup \{ S \rightarrow a(S \backslash A) | A \rightarrow a \in P\} \cup \{ (A \backslash B) \rightarrow e (D \backslash E) (A \backslash C) | C \rightarrow BD, E \rightarrow e \in P\} \cup \{ S \rightarrow \varepsilon | S \rightarrow \varepsilon \in P\}$. Отсюда $O(N') = O(N^2)$
$\blacksquare$

Осталось доказать: $\forall A, B \in N$ $\forall w \in \Sigma^*$   $A \vdash_G Bw \Leftrightarrow (A \backslash B) \vdash_{G'} w$

$\Rightarrow$: индукция по длине вывода в G. База: A = B, $w = \varepsilon$. Переход: см. картинку

$\Leftarrow$: индукция по длине вывода в G'. База: $A\backslash B \vdash_1 w \Rightarrow w = \varepsilon, A = B \Rightarrow A \vdash A \varepsilon$.

Переход:  $A\backslash B \vdash_1 e(C \backslash D)(A \backslash F)$; $(C \backslash D) \vdash u$; по предположению индукции, $C \vdash Du$. Аналогично, $A \vdash Fv$. + в P есть правила $D \rightarrow e$ и $F \rightarrow BC$.

Тогда $A \vdash Fv \vdash BCv \vdash BDuv \vdash Beuv$. Почти победа!

\textbf{L(G) = L(G')}. $w \in L(G) \Leftrightarrow S \vdash w$.

$L(G) \subset L(G')$: пусть $S \vdash w$

1) $w = \varepsilon$, переносим правило $S \rightarrow \varepsilon$ в G'

2) $w \neq \varepsilon$. Тогда $w = au$. $S \vdash Au \vdash_1 au$, т.к. НФ Хомского. По лемме $S \backslash A \vdash u$ $\Longrightarrow S \vdash_{G'} a(S \backslash A) \vdash au = w$. 

$L(G') \subset L(G)$:  $w= au$, $S \vdash_{G'} a(S \backslash A)$, где $ A \rightarrow a \in P$, тогда $S \backslash A \vdash u$, в итоге $S \vdash Au = au = w$.

\newpage{}


\section{(7) Гиперграфы. Гиперграфы $t$-пересечений. Теорема Эрдёша–Ко–Радо (о максимальном числе ребер в гиперграфе $1$-пересечений).}
\setcounter{section}{19}

\section{Диаграмма Хассе. Определение цепи и антицепи. Теорема о длине наибольшей цепи в ч.у.м. (б/д). Доказательство теоремы на примерах задач о людоедах и числах.}
\textbf{Диаграммой Хассе} называется ориентированный граф без циклов, по которому отношение порядка строится так: $a \leqslant b$, если из a в b идёт ориентированный путь. (см. билет 19) \\ \par

\textbf{Цепь упорядоченного множества $\langle M, \leqslant_M \rangle$} - упорядоченная последовательность элементов $a_1, a_2, \dots, a_n$, для которой $a_i \leqslant_M a_{i+1} $ для всех $1 \leqslant i \leqslant n-1$. \textbf{Антицепь} - набор элементов, никакие два из которых не находятся в отношении $\leqslant_M$. \\ \par
\textbf{Теорема}. Если d длина наибольшей цепи в упорядоченном множестве, то упорядоченное множество можно разбить на d антицепей. \par
\textbf{Задача 6.2 про людоедов} Некоторые людоеды хотят съесть некоторых других людоедов. Известно, что длина наибольшей
цепочки, в которой каждый людоед хочет съесть последующего, равна n (в частности, циклов нет). Докажите, что людоедов можно рассадить в n пещер, в каждой из которых никто никого не хочет съесть. Можно ли гарантированно рассадить их в меньшее число пещер? \par
$\blacktriangle$
Аналогично задаче 6.3, выбираем сначала все минимальные элементы для этого ч.у.м. Каждая следующая пещера: все людоеды, которых хотят съесть только людоеды из предыдущих пещер. Из условия, что максимальная цепь n, мы разбиваем всех великанов на n пещер, в каждой пещере антицепь. При этом гарантированно меньше пещер получить нельзя, рассмотрим пример с n великанами, где каждый великан пронумерован, и хочет съесть всех великанов с номером, большим, чем у него.
$\blacksquare$ \\ \par

\textbf{Теорема}. В упорядоченном множестве из mn + 1 элемента есть цепь длины n + 1 или антицепь из $m + 1$ элемента. \par
\textbf{Задача 6.3 про числа} (а) Докажите, что среди 36 различных натуральных чисел или найдется 6 чисел, среди которых ни одно не делится на другое, или найдется 8 чисел, которые можно выстроить в цепь, где каждое число делится на следующее. (б) Верен ли аналогичный факт для целых чисел? \par
$\blacktriangle$
(а) Пусть не найдётся ни цепь длины 8, ни антицепь длины 6. Тогда возьмём все числа, которые в данном ч.у.м. являются минимальными. Очевидно, что количество таких чисел $\leqslant 5$. Последовательно будем выбирать множества чисел, которые делятся только на числа из предыдущих множеств; получим следующую схему: \\ 
\includegraphics{images/20} \\

Нашли противоречие. Значит, начальное утверждение неверно. \\
(б) Формально, делимость среди целых чисел не является ч.у.м., т.к. нарушена антисимметричность => теорема не применима с её доказательством. Однако мы можем доопределить целые числа до ч.у.м.: будем считать, что -7 делится на 7, но не наоборот. Тогда делимость на таких целых числах будет ч.у.м. и можно применить теорему (аналогичные рассуждения из пункта а)
$\blacksquare$
\newpage{}


\section{(7) История последовательных продвижений:теорема Эрдёша–Ко–Радо (общий случай), теорема Франкла, теорема Уилсона, теорема Алсведе–Хачатряна. Все б/д, но с подробными комментариями. Нужно продемонстрировать четкое понимание, что за параметры выбираются в теореме АХ (Алсведе–Хачатрян): когда эта теорема превращается в ЭКР (Эрдеш-Ко-Радо); когда оценка становится тривиальной($C_k^n$); примеры конструкций, в которых можно явно посчитать,что оценка ЭКР не самая лучшая и АХ ее превосходит.}
\subsection{21. Линейные и полулинейные множества. Теорема Парика: полулинейность КС-языков.}

\textbf{Лемма:} $\forall X \subseteq \mathbb{N}^{|\Sigma|}$ полулинейного $\exists$ R (полулинейный) - регулярный язык ($\psi(R) = X$)

\textbf{Теорема Па'рика:} L - КС язык $\Rightarrow$ L - полулинейный.

Введём \textbf{лемму о разрастании}: G - КС язык. Тогда: $\forall k \exists p: \forall w \in L$ : $|w| \geqslant p$ $\exists w = xu_1\dots u_kyv_k \dots v_2v_1z$: $|u_iv_i| > 0, |u_1 \dots u_kyv_k \dots v_1| \leqslant p$

\includegraphics[]{images/parikh3.JPG}

$\blacktriangle$ G - в НФ Хомского, $p = 2^{|N|k + 1}$ $\Rightarrow$ глубина дерева $\geqslant |N|k + 1$ по нетерминалам. Тогда $\exists$ нетерминал А, который повторяется k раз. Возьмём самый глубокий А: $(h_0, h_1, \dots, h_k)$
$\blacksquare$

Док-во теоремы Парика:

$\blacktriangle$ $L = L(G)$: G - в НФ Хомского, $= <N, \Sigma, P, S>$. $\forall M \subset N: L_M(G) = \{w|S \vdash w$ при помощи только нетерминалов из M, причём всех $\}$.

Надо доказать, что $L_M(G)$ - полулинейно (тогда $L(G)$ - объединение конечного числа полулинейных множеств и как следствие полулинейно). Возьмём p из леммы о разрастании, $k = |N|$. $X = \{ \psi(w) | w\in L_M(G), |w| \leqslant p\}$, $Y = \{ \psi(uv) | |uv| \leqslant p \&\& \exists B \vdash uBv$ из нетерминалов M $\}$

Лемма: $\psi(L_M(G)) = X + <Y>$.

$\Leftarrow$: $x \in X + <Y>$. $x = x_1 + \alpha_1y_1 + \dots + \alpha_m y_m$; $x_1 \in X \Rightarrow \exists w': w' \in L_M(G), |w'| \leqslant p$. 

$y_i \in Y \Rightarrow \exists w_i = u_iv_i:$ вывод $B_i \vdash u_iBv_i$ из M. Финт ушами: $w' \in L_M(G) \Rightarrow$ $\exists$ вывод со всеми нетерминлами B: рисуем ёлку вида

\includegraphics[]{images/new_year_tree.JPG}

w'' - слово после "подвешивания" $B_i \vdash u_iB_iv_i$ к дереву вывода; $w'' \in (L_M(G): \psi(w'') = x$

Победа!

$\Rightarrow$: $w \in L_M(G) \Rightarrow \psi(w) \in X + <Y>$.

Индукция по длине слова.

База: $|w| \leqslant p \Rightarrow \psi(w) \in X$.

Переход: для $k = |N|$ лемма о разрастании:

\includegraphics[width=2cm]{images/parikh4.JPG}

$M_i$ - множество нетерминалов, не считая А, которые встречаются в выводе: $A^{(i)} \vdash u_{i+1}u_{i+2} \dots u_k y v_k \dots v_{i+1}$. Тогда $|M| > |M_0| \geqslant |M_1| \geqslant |M_2| \geqslant \dots \geqslant |M_k| \geqslant 0$; k+1 число = $|N| + 1$ число от 0 до $|M| - 1$. Тогда по принципу Дирихле $\exists j: M_j = M_{j+1}$.

\includegraphics[]{images/parikh5.JPG}

$\psi(w) = \psi(\omega) + \psi(u_jv_j)$, $\psi(\omega) \in X + <Y>, \psi(u_jv_j) \in Y$ (первое по предположению индукции). Ура!

$\blacksquare$
\newpage{}


\section{(4) Системы общих представителей (с.о.п.). «Тривиальные» нижние и верхние оценки.}
\setcounter{section}{8}
\section{Проверка правильности скобочной последовательности с несколькими типами скобок.}
\par \textbf{Определение}: \textit{Правильные скобочные последовательности с несколькими типами скобок} (рассмотрим с двумя)
\begin{enumerate}
    \item $\varepsilon$ (пустое слово) - ПСП
    \item $S$ - ПСП $\Rightarrow (S), [S]$ - ПСП
    \item $S_1, S_2$ - ПСП $\Rightarrow S_1 S_2$ - ПСП 
\end{enumerate}
\par \textbf{Задача:} Проверить, является ли последовательность из нескольких типов скобок правильной скобочной последовательностью
\par \textbf{Решение:} Храним стек незакрытых открывающих скобок
\lstinputlisting[language=C++,
emph={int,char,double,float,unsigned},
emphstyle={\color{blue}}
]{code/9_psp.cpp}
\par \textbf{Утверждение:} Данный алгоритм корректен, то есть ПСП $\Leftrightarrow$ алгоритм вывел true
\begin{itemize}
    \item[$\blacktriangle \Rightarrow$] Индукция по построению \begin{enumerate}
    \item База: $\varepsilon$ - обработается корректно
    \item $T=(u)$, $u$ - ПСП. По предположению индукции, всё $u$ удалится из стека к моменту прихода закрывающей скобки (можно считать, что стек начинается после первой скобки, она никак не влияет на применение алгоритма к $u$), а внешние скобки обработаются корректно (аналогично для других типов скобок)
    \item $T=T_1 T_2; T_1, T_2$ - ПСП. По предположению индукции, после того как считается $T_1$ стек опустошится $\Rightarrow$ после $T_2$ - тоже $\Rightarrow$ алгоритм сработает корректно $\blacksquare$
    \end{enumerate}
    \item[$\Leftarrow$] Доказываем индукцией по количеству действий, "обращая" предыдущий пункт.
    \par Рассмотрим скобочную последовательность, на которую алгоритм выдаёт true. Алгоритм сопоставил каждой открывающейся скобке одного типа закрывающуюся скобку того же типа. Причём они обязательно идут в правильном порядке. 
\parДля доказательства факта используем индукцию.
Рассмотрим пары соответсвующих скобок в порядке закрытия пары:
\begin{enumerate}
    \item База: Если между парой скобок нет других скобок, то последовательность от одной скобки до другой - правильная.
    \item Шаг: Если между парой скобок (назовём их $a$ и $b$ соответственно) есть непустая подстрока, то все скобки из подстроки уже были рассмотрены индукцией, так как открывающиеся скобки в подстроке были позже, чем $a$ добавлены в стек и по правилу стека должны были раньше из него выйти, а значит они уже были рассмотрены индукцией. Аналогично с закрывающимеся скобками из подстроки - они идут раньше, чем $b$, следовательно, по правилу стека им ставили в соответствие открывающиеся скобки, которые были добавленны позже $a$. (окрывающаяся скобка не могла быть добавленны раньше $a$, потому что $a$ перегородила бы ей выход.) По предположению индукции получаем, что подстрока состоит из одной или нескольких ПСП $\Rightarrow$ сама подстрока ПСП. $\Rightarrow$ Подпоследовательность от $a$ до $b$ - правильная. $\blacksquare$
\end{enumerate}
\end{itemize} 
\newpage{}

\section{(5) Верхняя оценка размера минимальной с.о.п. с помощью жадного алгоритма.}
\includegraphics[width=1\linewidth]{sections/Polina/imgs/21.jpg}
\newpage{}


\section{(8) Конструктивная нижняя оценка размера минимальной с.о.п.}
\Th $\forall n \geqslant 4 \forall k \leqslant \frac{n}{4} \forall s : 2 \leqslant ln \frac{sk}{n} \leqslant k$: $\exists M: \tau (M) \geqslant \frac{1}{32} \cdot  \frac{n}{k} ln \frac{sk}{n}$

Будем пользоваться неравенством $[x] \geqslant \frac{x}{2}$.

\Proof Пусть $m = [\frac{1}{2} ln \frac{sk}{n}] \geqslant 1$. Рассмотрим $N_1, \dots, N_{C_{2m}^m} \subset \{ 1, 2, \dots, 2m\} \subset \{1, 2, \dots, n\} = R_n$, где $N_i$ - все возможные m-элементные подмножества, $|N_i| = m$. 

$\tau (\{N_1, \dots, N_{C_{2m}^m}\}) = m + 1$, потому что если взять любое m-эл. мн-во найдётся обратное к нему. 

$q = [\frac{2k}{m}]; \frac{k}{m} \geqslant \frac{2k}{ln \frac{sk}{n}} \geqslant 2 \Rightarrow q \geqslant 4$

Разобьем на такие множества размера $2m$.
\includegraphics[]{images/est2.JPG}

В каждое разбиение размера $2m$ запихиваем совокупность выше ($N_1, \dots, N_{C_{2m}^{m}}$).

Рассмотрим множества $L_1 = N_1^1 \cup N_1^2 \cup \dots N_1^q$; $\dots$ $L_{C_{2m}^{m}} = N_{C_{2m}^{m}}^1 \cup N_{C_{2m}^{m}}^2 \cup \dots N_{C_{2m}^{m}}^q$; $ \forall i |L_i| = q\cdot m = [\frac{2k}{m}] \cdot m \geqslant \frac{k}{m} \cdot m = k$

$\tau (\{L_1, \dots, L_{C_{2m}^{m}}) = m + 1$ (очевидно, т.к. объединения одинаковые, сохраняется предыдущий результат для $\tau$).

$C_{2m}^m < 2^{2m} \leqslant 2^{ln \frac{sk}{n}} < e^{ln \frac{sk}{n}} = \frac{sk}{n}$

\includegraphics[]{images/est3.JPG}

$\tau(M') = t(m+1) > tm \geqslant \frac{n}{4qm} m \geqslant \frac{nm}{8k} \geqslant \frac{1}{32} \frac{n}{k} ln \frac{sk}{n}$. Но если какое-то из множеств мощности больше, чем k, то просто удалим из него любые элементы (соп от этого не уменьшится). $\tau(M'') \geqslant \tau(M')$
\EndProof
\newpage{}

\section{(10) Вероятностная нижняя оценка размера минимальной с.о.п. Следствие из неё.}
\subsection{25. Анализатор перенос-свёртка, недостатки анализатора.}
\textbf{Мотивировка:} У нас есть алгоритм Эрли, но он работает долго.
\begin{itemize}
    \item Scan: $O(|w|^2|G|)$
 \item Predict: $O(|w|^2|G|^2)$
\item  Complete:$O(|w|^3|G|^2)$
\end{itemize}
Чтобы понять алгоритм, \textbf{рассмотрим грамматику и слово w = ab}\\
\begin{itemize}
    \item $S' \rightarrow S$(добавляем)
 \item $S \rightarrow aB$
 \item $B \rightarrow b$
\end{itemize}
\includegraphics[width=7cm]{form1.PNG}\\
\begin{itemize}
    \item[] Храним стек текущих символов и текущую позицию в слове.

 \item На вершине стека написана правая часть правила - заменяем на левую (reduce)

 \item Иначе - добавляем символ в стек и читаем символ (shift)
\end{itemize} Стоит заметить, что разбираем мы слово справа налево, а вывод в грамматике правсторонний, поэтому мы будем вынуждены поменять порядок правил на обратный для построения дерева разбора
\\
\textbf{А в чем проблема?}
\begin{itemize}
    \item [1] А как понять, какая правая часть правила находится на стеке?
     \item [2] А что делать, если нам подходят несколько правил? (стек можно распарсить двумя и более способами) 
\end{itemize}
\newpage{}

\section{(7) Нижняя оценка размера минимальной с.о.п. с помощью обобщенных с.о.п. Не требуется проверять, то что определенное значение $\ell$ подходит под условие теоремы.}

\Th Пусть даны $n, k, s$. Рассмотрим произвольное $\ell$, обозначим $n' = C_n^k, s' = C_n^{\ell}, k' = C_{n - \ell}^k$. Предположим, что для $\ell$ выполнено неравество:

$$
max\{ \frac{n'}{k'}, \frac{n'}{k'} ln\bigg(\frac{s' k'}{n'}\bigg)\} + \frac{n'}{k'} + 1 \leq s
$$

Тогда $\exists \mathcal{M} = \mathcal{M}(n, k, s): \ \tau(\mathcal{M}) > \ell$.

\Proof Зафиксируем $n, s, k$. Мы хотим построить $\mathcal{M} = \{M_1, M_2, \ldots, M_s\}$ такие, что $\forall i \in \{1, \dots, s\}: \ |M_i| = k, M_i \subseteq \{1, 2, \ldots, n\}: \ \tau(M) > \ell$, то есть нужно, чтобы было верно следующее: 

$$
\forall L \subset \{1, 2, \ldots, n\}: \ |L| = \ell \ \exists M_i \in \mathcal{M}: \ M_i \cap L = \varnothing \Longleftrightarrow M_i \subset \{1, 2, \ldots, n\} \setminus L
$$

То есть нам надо выбрать с.о.п. для каждого $n - \ell$ элементного множества, при этом их должно оказаться не больше $s$. 

Далее для всех $(n-\ell)$-элементных подмножеств $\{1, \ldots, n\}$ мы хотим рассматривать все $k$-элементные множества, которые лежат в них целиком.

Давайте рассмотрим все $k$-элементные подмножества, обозначим их за $K_1, \ldots, K_{C_n^k}$. Перейдем от $K_i$ к их номерам, то есть далее каждому множеству будет сопоставляться число. Рассмотрим каждое $(n - \ell)$ элементное подмножество $L_j : j \in \{1, \ldots, C_n^{\ell}\}$. Их всего $C_n^{n - \ell} = C_n^{\ell}$. Теперь рассмотрим все такие $i$, что $K_i \subset L_j$, то есть множество представителей $L_j$. Таких $i$ будет $C_{n-\ell}^k$, они образуют подмножество в множестве $\{1, 2, \ldots, C_n^k\}$, множество из таких $K_i$ обозначим за $\Lambda_i$.

Теперь заметим, что у нас есть множество $\{1, 2, \ldots, C_n^k\}$ и есть $C_n^{\ell}$ модмножеств из $C_{n - \ell}^k$ элементов, при этом для нее нам надо построить с.о.п. Переобозначим $n' = C_n^k, s' = C_n^{\ell}, k' = C_{n - \ell}^k$. Теперь возьмем любую минимальную с.о.п. для совокупности $\Lambda_1, \ldots, \Lambda_{s'}$. Обозначим ее за $\sigma_1, \ldots, \sigma_{\tau}$. 

По определению с.о.п. 

$$\forall j \in \{1, \ldots, s'\} \ \exists i: \ \sigma_i \in \Lambda_j \Longleftrightarrow \forall j \in \{1, \ldots, s'\} \ \exists i: \ K_{\sigma_i} \subset L_j$$

Рассмотрим $\mathcal{M} = \{K_{\sigma_1}, \ldots, K_{\sigma_{\tau}}\}, \ \tau(\mathcal{M}) > \ell$. Тогда это искомая с.о.п., чей размер оценивается жадным алгоритмом:

$$
\tau(\mathcal{M}) \leq max\{ \frac{n'}{k'}, \frac{n'}{k'} ln\bigg(\frac{s'k'}{n'}\bigg)\} + \frac{n'}{k'} + 1 = max\{ \frac{C_n^k}{C_{n - \ell}^k}, \frac{C_n^k}{C_{n - \ell}^k} ln\bigg(\frac{C_n^{\ell} C_{n - \ell}^k}{C_n^k}\bigg)\} + \frac{C_n^k}{C_{n - \ell}^k} + 1
$$

Что, очевидно, не больше, чем $s$, так как $\tau(\mathcal{M}) \leq |\mathcal{M}| = s$. Теорема доказана.

\EndProof
\newpage

\section{(7) С.о.п. в геометрии (теорема о треугольниках на плоскости, б/д). Размерность Вапника–Червоненкиса. Теорема Радона (б/д). Подсчет размерности семейства полупространств. Лемма о числе областей в пространстве заданной мощности и размерности. Лемма о размерности измельчения (достаточно доказать существование верхней оценки, не обязательно такой, как на лекции)}

\textbf{С.о.п. в геометрии}

Рассмотрим множество точек $S \subset \mathbb{R}^2$ конечной мощности. Начнем пересекать его cо всевозможными треугольниками в этой плоскости и пусть $\mathcal{M}$ это система всех подмножеств $S$, которые можно получить, пересекая $S$ с треугольниками. 

Зафиксируем теперь $\varepsilon \in (0, 1)$ и пусть $\mathcal{M}_\varepsilon \subseteq \mathcal{M} = \{ M \in \mathcal{M} \mid |M| \geqslant \varepsilon|S| \}$

\Th (Вапника-Червоненкиса, частный случай) $\forall \varepsilon\:\: \exists \text{с.о.п.} N $ для совокупности $\mathcal{M}_\varepsilon$, такая что $$ |N| \leqslant \frac{500}{\varepsilon} \log_2 \frac{500}{\varepsilon}$$

\Note Мощность $N$ не зависит от $S$ и $n$.\\

\textbf{Размерность Вапника-Червоненкиса. Теорема Радона (б/д).}\\

\Def Пара $(\mathcal{X},\: R)$, где $\mathcal{X}$ -- произвольное множество, а $R \subseteq 2^{\mathcal{X}}$,  называется {\it ранжированным пространством}. \\ 

\Def Подмножество $A \subseteq \mathcal{X}$ {\it дробится} системой $R$, если $$\forall B \subseteq A\:\: \exists r \in R:\:\: A\cap r = B,$$
причем \textit{проекцией} системы $R$ на $A$ назовем множество $Pr_A R = \{ r \cap A \mid r \in R\}$ всевозможных пересечений $r\in R$ с $A$. Очевидно, что $A$ дробится тогда и только тогда, когда $Pr_A R = 2^A$.

\Def \textit{Размерность Вапника-Червоненкиса} $VC(\mathcal{X}, R)$ ранжированного пространства $(\mathcal{X}, R)$ по определению равна$$ VC(\mathcal{X}, R) := \max \{m \in \mathbb{N}\: \mid\: \exists A \subseteq \mathcal{X},\: |A| = m:\:\: Pr_A R = 2^A \}$$
    (если такого $m$ не существует, то $VC(\mathcal{X}, R) = +\infty$).

\Example $VC(\mathbb{N}, 2^{\mathbb{N}}) = +\infty$.

\Example $VC(\mathbb{R}, \mathcal{H}) = 2$, где $\mathcal{H}$ --- семейство всех лучей

\Th (Радон, б/д)\\
Любое множество из $n+2$ точек $S \subset \mathbb{R}^n$ можно представить как $S = A_1 \sqcup A_2$, причем выпуклые оболочки $A_1$ и $A_2$ пересекаются, т.е. $$ conv(A_1) \cap conv(A_2) \neq \varnothing.$$ 

\textbf{Подсчет размерности семейства полупространств.}\\

\Statement $VC(\mathbb{R}^n, \mathcal{H}) = n + 1$, где $\mathcal{H}$ --- семейство всех открытых полупространств (например для $n=2$ это полупоскости).

\Proof Поскольку $n+1$ вершина симплекса в $\mathbb{R}^n$ дробится, то верно неравенство $VC(\mathbb{R}^n, \mathcal{H}) \ge n+1$. \\ Для множества $S,\: |S| \ge n+2$ найдем предстваление $A_1 \sqcup A_2$ из теоремы Радона. Тогда очевидно, что отдробить $A_1$ не получится.

\EndProof

\textbf{Лемма о числе областей в пространстве заданной мощности и размерности.}

\Lemma \textbf{(1)} Пусть $(\mathcal{X},\: R)$ --- ранжированное пространство, такое что $VC(\mathcal{X}, R) = d,\:\: |\mathcal{X}| = n$. Тогда верно неравенство $$ |R| \le g(n, d) = \sum\limits_{i=0}^{d} C_n^i.$$

\Proof Заметим сначала, что $g(n, d) = g(n-1, d) + g(n-1, d-1)$ --- следствие из треугольника Паскаля.\\
Воспользуемся индукцией по $n$ и $d$. \\

\textit{База:} 

Пусть $n=0$ и $d$ произвольное. Тогда $R$ равно либо $\{ \varnothing\}$, либо $\varnothing$. В любом случае, $|R| \le 1$. В то же время, очевидно, что $VC(\mathcal{X}, R) \le n = 0$. Но тогда $|R| \le 1 = g(0, 0)$. 

Пусть, наоборот, $d=0$ и $n \ge 1$ --- произвольное. Предположим, что $|R| \ge 2$. Тогда $\exists \: R_1 \neq R_2 \in R$, причем либо в $R_1 \setminus R_2$, либо в $R_2 \setminus R_1$ содержится элемент $x \in \mathcal{X}$. Тогда множество $A = \{ x\}$ дробится $R_1$ и $R_2$, что противоречит $d=0$. Получаем $|R| \le 1 = g(n, 0)$, и база доказана. \\

\textit{Переход:} зафиксируем $(\mathcal{X}, R)$ с $VC(\mathcal{X},R) = d \ge 1$ и $|\mathcal{X}| = n \ge 1$. Рассмотрим произвольный $x \in \mathcal{X}$ и определим пространства $(\mathcal{X}_1,\: R_1),\: (\mathcal{X}_2,\: R_2)$ следующим образом:
\begin{align*}
    & \mathcal{X}_1 = \mathcal{X}_2 = \mathcal{X} \setminus \{ x\}\\
    & R_1 = \{r \setminus \{ x\}\: \mid\: r \in R\},\: R_2 = \{r \in R\: \mid\: x\not\in r\text{, но } r\cup\{x\} \in R\}
\end{align*}

Тогда имеем $|R| = |R_1| + |R_2|$. Докажем два неравенства:
\begin{enumerate}
    \item $VC(\mathcal{X}_1, R_1) \le d$ --- очевидно. Тогда по предположению индукции $|R_1| \le g(n-1, d)$.
    \item $VC(\mathcal{X}_2, R_2) \le d-1$.
    \Proof
    Предположим $VC(\mathcal{X}_2, R_2) \ge d$ и рассмотрим $A \subseteq \mathcal{X}_2,\: |A| = d,\: Pr_{R_2} A = 2^A$. По определению $R_2$, множество $A \cup \{x\}$ дробится системой $R$, но $|A \cup \{ x \}| \ge d+1$, что противоречит $VC(\mathcal{X}, R) = d$.
    \EndProof
    
    Тогда по предположению индукции $|R_2| \le g(n-1, d-1)$
\end{enumerate}

Отсюда следует, что $$|R| = |R_1| + |R_2| \le g(n-1, d) +  g(n-1, d-1) = g(n, d).$$

\EndProof

\textbf{Лемма о размерности измельчения}


\Def $h \in \mathbb{N}$. Тогда $h-$\textit{измельчение} $R$ это множество $$R_h := \{r_1 \cap \ldots \cap r_h:\: r_i \in R\}$$ 

(множества в пересечении не обязательно различные)

\Example Для $(\mathbb{R}^n, \mathcal{H})$ измельчение $R_h$ целиком содержит в себе совокупность всех выпуклых многогранников в $\mathbb{R}^n$, имеющих $h$ граней. Для $h = 3, n = 2$ это все треугольники на плоскости.

\Lemma \textbf{(2)} Пусть $d \ge 2,\: h \ge 2$ (эти ограничения наложим для удобства доказательства, остальные случаи очевидны) и $VC(\mathcal{X}, R) \le d$. Тогда $$ \exists N: VC(\mathcal{X}, R_h) \le N $$ (на лекции $N =  2dh \log_2 dh$) 

\Proof Зафиксируем любое $S \subseteq \mathcal{X}$, имеющее $|S| = n$ (на лекции было $n > 2dh \log_2 dh$) и дробящееся $R_h$ (если такое существует). По лемме (1) имеем $|Pr_R S| \le g(n, d)$. Тогда $$\left| Pr_S R_h\right| \le | Pr_S R|^h \le n^{dh}.$$
Поскольку $n \ge 2d$, то в сумме $g(n, d)$ максимально последнее слагаемое и $g(n, d) \le n^d$ ({\it б/д, по индукции}).
Поскольку  $|Pr_S R_h | = 2^n$, то должно быть выполнено неравенство $$2^n = | Pr_S R_h| \le | Pr_S R|^h \le n^{dh},$$
что неверно при достаточно больших $n$ (на лекции -- для $n > 2dh \log_2 dh$), и множества $S$ с $|S| = n$, дробящегося $R_h$, не существует.

\EndProof

\newpage{}

\section{(10) Эпсилон-сети. Теорема Вапника–Червоненкиса об эпсилон-сетях (первая лемма – только формулировка, вторая лемма с доказательством, завершение доказательства) и теорема о треугольниках как частный случай.}
\Def $Z$ - $\varepsilon$-сеть для $M \subset X$, если $Z \subset X$ и $\forall x \in M \exists z \in Z: |x - z| < \varepsilon$. 

\Th Вапника-Червоненкиса: Пусть $VC(X, R) = d < \infty$ Тогда $\forall \varepsilon > 0 \forall A \text{(А - конечное)}: A \subset X \exists \varepsilon$ - сеть $N: |N| \leqslant [\frac{8d}{\varepsilon}log_2\frac{8d}{\varepsilon}]$

\Proof 
\href{https://www.youtube.com/watch?v=dQw4w9WgXcQ}{Док-во по ссылке}
\EndProof


\newpage{}

\section{Тема 29}
Оставлена читателям как простое упражнение.
\newpage{}

\section{(4) Размерность Вапника–Червоненкиса. Теорема Вапника–Червоненкиса (б/д). Приложения в статистике: равномерная сходимость в ЗБЧ (УЗБЧ) (б/д).}

\Def Пара $(\mathcal{X},\: R)$, где $\mathcal{X}$ -- произвольное множество, а $R \subseteq 2^{\mathcal{X}}$,  называется {\it ранжированным пространством}. \\ 

\Def Подмножество $A \subseteq \mathcal{X}$ {\it дробится} системой $R$, если $$\forall B \subseteq A\:\: \exists r \in R:\:\: A\cap r = B,$$
причем \textit{проекцией} системы $R$ на $A$ назовем множество $Pr_A R = \{ r \cap A \mid r \in R\}$ всевозможных пересечений $r\in R$ с $A$. Очевидно, что $A$ дробится тогда и только тогда, когда $Pr_A R = 2^A$.

\Def \textit{Размерность Вапника-Червоненкиса} $VC(\mathcal{X}, R)$ ранжированного пространства $(\mathcal{X}, R)$ по определению равна$$ VC(\mathcal{X}, R) := \max \{m \in \mathbb{N}\: \mid\: \exists A \subseteq \mathcal{X},\: |A| = m:\:\: Pr_A R = 2^A \}$$
    (если такого $m$ не существует, то $VC(\mathcal{X}, R) = +\infty$).
    
\Th (Вапника-Червоненкиса)
$VC(X, R) \le d \Rightarrow \forall \varepsilon \in (0, 1) \ \exists \varepsilon-\text{сеть размера} \le \lceil\frac{8d}{e}log_2{\frac{8d}{e}}\rceil$

\textbf{Напоминание ЗБЧ и УЗБЧ для испытаний Бернулли:}
$A_1, \ldots, A_n, \ldots$ -- независимые в совокупности события

$\forall i: P(A_i) = p:$

$\frac{I_{A_1} + \ldots + I_{A_n}}{n} \stackrel{p}{\longrightarrow} p$ (ЗБЧ)

$\frac{I_{A_1} + \ldots + I_{A_n}}{n} \stackrel{\text{п.н.}}{\longrightarrow} p$ (УЗБЧ)

\textbf{Про мат. статистику}

У нас есть выборка -- набор чисел $x_1, \ldots, x_n$, которые мы пронаблюдали. Мы считаем, что за этими числами стоят какие-то случайные величины $\xi_1, \ldots, \xi_n$, и эти числа служат просто их реализацией.

Эти случайные величины $\xi_1, \ldots, \xi_n$ независимы и одинаково распределены.

$F_{\xi_i}(x)$ -- функция распределения для $i$ случайной величины. Поскольку они все одинаковы, то и функция для них одна и та же. Но мы ее не знаем и хотим аппроксимировать. И вот об этом теорема Гливенко-Кантелли. В ней мы простейшим образом пытаемся приблизить функцию распределения (ступенечками).

Определим $\hat{F}_n(x_1, \ldots, x_n; x) = \frac{1}{n} \sum\limits_{i=1}^{n}I_{\{x_i \le x\}}$ ($A_j := \{x_j \le x\}$). 

УЗБЧ гласит, что $\hat{F}_n(x_1, \ldots, x_n; x) \stackrel{\text{п.н.}}{\longrightarrow} P(\xi_i \le x) = F_{\xi_i}(x)$

\Th (Гливенко-Кантелли)

$$P(sup_{\{x \in \mathbb{R}\}} \mid \hat{F}_n(x_1, \ldots, x_n; x) - F_{\xi_1}(x)\mid \xrightarrow[{n\rightarrow \infty}]{} 0) = 1$$

Переформулируем в терминах индикаторов:

$$A_1^x, \ldots, A_n^x : A_i^x = \{\xi_i \le x\}, x \in \mathbb{R}$$

$\forall x : A_i^x$ взаимно нез. и $\forall i : P(A_i^x) = p^x$

\Th (Вапника-Червоненкиса, применительно к вероятности, мб не нужна, но я оставлю)

Пусть $x \in \mathcal{X}$. Рассмотрим последовательность событий $A_1^x, \ldots, A_n^x, \ldots$ на некотором вероятностном пространстве, для которой $\forall x : A_i^x$ независимы в совокупности и $\forall x \forall n \ P(A_n^x) = p^x$.

Тогда $\frac{1}{n} \sum\limits_{i=1}^{n} I\{A_i^x\}$ сходится по $x$ равномерно к $p^x$ 

(а именно, $P(sup_{x \in \mathcal{X}} \mid \frac{1}{n} \sum\limits_{i=1}^{n} I\{A_i^x\} - p^x \mid \rightarrow 0) = 1$) тогда и только тогда, когда
$$VC(\mathcal{X};\{A_1^x, \ldots, A_n^x, \ldots \}) < + \infty$$.

\newpage{}

\section{(5) Системы различных представителей. Теорема Холла.}

\textbf{Системы различных представителей}

\Def Пусть дан набор $\mathcal{M}$ множеств. В каждом из множеств выбрали по элементу. Если все элементы различны, то такой набор назовем \textit{системой различных представителей (сокращенно с.р.п.)}. \par
Или, формально, \textit{системой различных представителей для набора $\mathcal{M}$ множеств} называется упорядоченный набор различных элементов $x(S) \in S, S \in \mathcal{M}$, т. е., такое инъективное отображение $x : \mathcal{M} \rightarrow \bigcup\limits_{S \in \mathcal{M}}S$, что $x(S) \in S$ для любого $S \in \mathcal{M}$.\\

\Th \textit{(Холла)} Пусть $S_1, \ldots, S_m$~--- конечные множества. В каждом из них можно выбрать по элементу $x_i \in S_i$ так, чтобы все $x_i$ были различны, тогда и только тогда, когда для каждого $k \in \{1, \ldots, m\}$ объединение любых $k$ из этих множеств состоит из не менее $k$ элементов.

\begin{itemize}
    \item[$\Rightarrow$] Допустим нашлись такие различные элементы $x_1, \ldots, x_m$. Тогда рассмотрим $k$-элементное подмножество $\{i_1, \ldots, i_k\} \subset \{1, \ldots, m\}$. Тогда в множестве $A = \bigcup\limits_{n = 1}^k S_{i_n}$ будут обязательно лежать элементы $x_{i_1}, \ldots, x_{i_k}$. Так как они различны, то мощность множества $A$ хотя бы $k$.
    \item[$\Leftarrow$] Индукция по $m$. База очевидна. Предположим, что для всех систем с меньшим количеством элементов утверждение верно. Пусть найдётся такое множество $I \subsetneq \{1, \dots, m\}$, что $|\bigcup\limits_{i \in I} S_i| = |I|$. Применим предположение индукции для $S_I := \{S_i | i \in I\}$. Далее, пересечения множеств из $S_{\{1, \ldots, m\} \setminus I}$ со множеством $(\bigcup\limits_{i = 1}^m S_i) \setminus (\bigcup\limits_{i \in I} S_i)$ <<непокрытых>> элементов образуют систему, для которой выполнено условие леммы Холла. Значит, можно применить предположение индукции.

    Если же такого множества $I$ не найдётся, возьмём произвольный элемент $x$ произвольного множества $S_k$. Дополнения оставшихся множеств до этого элемента образуют систему, для которой выполнено условие леммы Холла. Значит, можно применить предположение индукции.
\end{itemize}
\newpage{}

\section{(4) Перманент. Формула разложения по строке. Связь с количеством систем различных представителей.}

\Def \textit{Перманент} квадратной матрицы $A = (a_{ij})$ размера $n \times n$ определяется формулой:
$$Per(A) := \sum \limits_{\sigma \in S_n}\prod \limits_{i=1}^n a_{i,\sigma(i)},$$
где $S_n$ есть множество всех перестановок $n-$элементного множества.

Формула перманента матрицы отличается от определителя матрицы отсутствием умножения на знак перестановки.

\Def \textit{Подматрицей} данной матрицы называется матрица, полученная из данной вычеркиванием некоторого количества строк и столбцов. Перманент прямоугольной матрицы $A^{m \times n}$ определяется как сумма перманентов всех квадратных подматриц максимального размера. Или, формулой, при $m < n$:
$$Per(A) := \sum \limits_{\sigma}\prod \limits_{i=1}^m a_{i,\sigma(i)},$$
где сумма берется по всем $m-$элементным размещениям чисел от 1 до $n$. При $m > n$ положим $Per(A) = Per(A^T)$

\Th \textit{(Формула разложения по строке)} Если $m \leq n$, то для любого $i \in \{1, \ldots, m\}$
$$Per(A) = \sum \limits_{j=1}^n a_{ij}Per(A_{ij}),$$
где $A_{ij}$~--- матрица, получаемая из исходной вычеркиванием $i-$ой строки и $j-$ого столбца.

Доказательство очевидно -- просто перегруппировать слагаемые в сумме.\\

\textbf{Связь с количеством систем различных представителей}\\

\Def Пусть дан набор $\mathcal M$ множеств. В каждом из множеств выбрали по элементу. Если все элементы различны, то такой набор назовем \textit{системой различных представителей (сокращенно с.р.п.)}.

Или, формально, \textit{системой различных представителей для набора $\mathcal{M}$ множеств} называется упорядоченный набор различных элементов $x(S) \in S, S \in \mathcal{M}$, т. е., такое инъективное отображение $x : \mathcal{M} \rightarrow \bigcup\limits_{S \in \mathcal{M}}S$, что $x(S) \in S$ для любого $S \in \mathcal{M}$.

\Def \textit{Паросочетание} $M$ в двудольном графе~--- произвольное множество рёбер двудольного графа, такое что никакие два ребра не имеют общей вершины.

\Def Вершины двудольного графа, инцидентные рёбрам паросочетания $M$, называются \textit{покрытыми}, а неинцидентные~--- \textit{свободными}.

\Def Паросочетание $M$ графа $G$ называется \textit{полным}, если оно покрывает все вершины графа. (В пределах следующей теоремы имеется в виду необходимость покрытия всех вершин левой доли)

\Th Для любых $m \leq n$ перманент прямоугольной матрицы $m \times n$ из нулей и единиц равен количеству с.р.п. для системы из $m$ подмножеств множества $\{1, \ldots, n\}$, определяемых строками.

\Proof Изобразим двудольный граф, где в левой доле множества $S_1, \ldots, S_m$, а в правой их элементы $x_1, \ldots, x_n$. Тогда нахождение СРП равносильно нахождению полного паросочетания в таком двудольном графе. 

Рассмотрим матрицу смежности данного графа. Осознаем, что ее перманент и есть количество полных паросочетаний, тогда мы доказали требуемое.

Утверждение о том, что в графе будет найдено полное паросочетание равносильно тому, что мы в его матрице смежности смогли расставить единички так, чтобы в каждой строке и в каждом столбце было ровно по одной единице. Тогда заметим, что перманент такой матрицы по формуле выше и будет равен числу полных паросочетаний (суммирование по всем перестановкам нужных произведений).

\EndProof
\newpage{}

\section{(5) Оценки $(\frac{n}{e})^n \leqslant n! \leqslant e^2(\frac{n}{e})^{n+1}, \ C_n^k \leqslant \frac{n^k}{k!}$. Оценки для $C_{2n}^n$ с помощью комбинаторных тождеств. Формула Стирлинга б/д. Асимптотика $C_{2n}^n$}
\
\\
\includegraphics[width=18cm]{polina_9.PNG}\\
\includegraphics[width=18cm]{polina_6.PNG}\\
\includegraphics[width=18cm]{polina_7.PNG}\\
\includegraphics[width=18cm]{polina_8.PNG}
\section{(5) Асимптотика $ln (n!)$ и $\sqrt[n]{n!}$ (с доказательством без использования формулы Cтирлинга).}
\begin{itemize}
    \item [1] $\sqrt[n]{n!}$. Воспользуемся неравенством $n^ne^{-n} \leqslant n! \leqslant n^{n+1}e^{-n+1}$. Извлечём
корень n-ой степени из всех его частей и получим $\frac{n}{e} \leqslant \sqrt[n]{n!} \leqslant (ne)^{\frac{1}{n}}(\frac{n}{e})$. Заметим, что правая и левая части при $n \rightarrow \infty$ стремятся к $\frac{n}{e}$, откуда $\sqrt[n]{n!} \sim \frac{n}{e}$
\item[2] $ln (n!)$  Воспользуемся неравенством из предыдущего билета. Прологарифмируем все его части и получим $nln(n) - n\leqslant ln(n!)\leqslant (n+1)ln(n) - (n-1) \Longrightarrow ln (n!) \sim nln(n)$
\end{itemize}
\section{ (6) Оценки биномиальных коэффициентов вида $C^{[an]}_n, a \in (0, 1)$. Аналогичные результаты для полиномиальных коэффициентов}
\
\\
\includegraphics[width=18cm]{polina_10.PNG}\\
\includegraphics[width=18cm]{polina_11.PNG}\\
Аналогично\\
\includegraphics[width=18cm]{images/polina_12_new.PNG}
\section{(6) Асимптотика для $C_n^k$ при $k^2 = o(n)$. Оценки той же величины при больших $k$. Асимптотики для $C_{2n}^n/C_{2n}^{n-k}$.}
\
\\
\includegraphics[width=18cm]{images/polina_12.PNG}\\
$C_{2n}^n/C_{2n}^{n-k}$\\
\includegraphics[width=15cm]{images/polina_15.PNG}\\
\includegraphics[width=15cm]{images/polina_16.PNG}
\section{(5) Симметричный случай ЛЛЛ (б/д). Вывод оценки диагонального числа Рамсея (теорема Спенсера).}
\textbf{Теорема (ЛЛЛ, симметричный случай)}
\begin{itemize}
    \item [1]Пусть $A_1, ..., A_n$ - события,  $\forall i \ P(A_i) \leq p$.
    \item[2] Пусть $\forall i$ событие $A_i$ не зависит от совокупности остальных событий, кроме не более чем d штук.
    \item[3] $p < \frac{1}{e(d+1)}$. 
\end{itemize}
Тогда $P(\bigcap\limits_{i=1}^{n}\overline{A_i}) > 0$
 

\textbf{Определение} A не зависит от совокупности $B_1, ..., B_n$, если $\forall I \in \{1 ... n\} \ P(A \ |\ \bigcap\limits_{i \in I}B_i) = P(A) $

\textbf{Теорема Спенсера}
\begin{center}
Пусть для фиксированного s число n таково, что 
$e \cdot 2^{1-C_s^2}(C_{n}^{s-2}C_{s}^2) < 1$, тогда $R(s,s) > n$
\end{center}
\Proof 
 $R(s,s) > n \leftrightarrow \ \exists G$ на n вершинах, такой что 
    \begin{itemize}
        \item $w(G) < s$
        \item $\alpha(G) < s$
    \end{itemize}
    
Рассмотрим случайный граф $G(n, \frac{1}{2})$. $A_1, ..., A_{C_{n}^s}$ - все s-элементные подмножества множества вершин графа. Рассмотрим событие $E_i$ - $A_i$ образует клику или независимое множество.
\\
p = $2^{1 - C_s^2}$
\\
$d = |\{ j: |A_i \bigcap A_j| \geqslant 2\}| = \sum\limits_{t=2}^{s-1} C_s^tC_{n-s}^{s-t} < C_s^2C_{n}^{s-2}$ (справа налево. Берём две вершинки в нашем множестве $A_i$, и добиваем множество до множества размера s любыми вершинками из оставшихся, в том числе и вершинками из множества, поэтому мы учтём некоторые выбранные множества несколько раз, откуда получаем такое неравенство). Если $e \cdot 2^{1-C_s^2}(C_{n}^{s-2}C_{s}^2) < 1$, то по ЛЛЛ получаем, что найдется граф на n вершинах, в котором нет ни s-клики, ни s-независимого множества
\EndProof
\\
\textbf{Следствие}
\\
$R(s,s) \geqslant 2^{s/2}s\frac{\sqrt{2}}{e}(1+o(1))$
\\
\Proof

\includegraphics[width=18cm]{images/polina_13.PNG}

\includegraphics[width=18cm]{images/polina_14.PNG}
\EndProof
\section{(8) Симметричный и несимметричный случай ЛЛЛ (с доказательством симметричного либо напрямую, либо с доказательством несимметричного и выводом из него)}
\textbf{Определение} Пусть $A_1, ..., A_n$ - события. $G = (V,E)$ является орграфом зависимостей для этих событий, если $V = \{A_1,..., A_n\}$ и $\ \forall \ i$
\ $A_i$ не зависит от совокупности всех событий, с которыми оно не соединено исходящими ребрами
\\
\\
\textbf{ Теорема (ЛЛЛ, общий случай)} 
\\
Пусть $A_1, ..., A_n$ - события.  $G = (V,E)$ - орграф зависимостей, для которого существуют $x_1, ..., x_n \ \in [0,1): \\ \forall i \ P(A_i) \leq x_i*\prod\limits_{j: (A_i A_j) \in E}(1-x_j)$
\\
Тогда $P(\bigcap\limits_{i=1}^n \overline{A_i}) \geq \prod\limits_{j=1}^n(1-x_j) > 0$
\\
\\
\textbf{Докажем симметричный случай}
\Proof
\begin{itemize}
    \item [1] d = 0
    \\ $P(\bigcap\limits_{i=1}^n \overline{A_i}) = \prod\limits_{i=1}^n(P(\overline{A_i})) = \prod\limits_{1=1}^n(1-P(A_i)) \geq (1-p)^n \geq (1+1/e)^n > 0 $
    \item[2] $d \geq 1$ \\ $x_i = \frac{1}{d+1}$
    \\
    В качестве G возьмем такую конструкцию. Берем $A_i$ и выделяем группу из d событий, от которых $A_i$ может зависить, и проводим стрелки.
    \\
    Нам нужно доказать, что $P(A_i) \leq \frac{1}{d+1}\prod\limits_{j: (A_i, A_j) \in E}(1 - \frac{1}{d+1})$. Достаточно показать, что $p \leq \frac{1}{d+1}(1 - \frac{1}{d+1})^d$. Почему? Стрелочек, то есть ор. ребер не больше чем d, поэтому мы лишь усилили условие. С другой стороны из симметричного ЛЛЛ  $P(A_i) \leq p$. 
    \\
    Но тогда можно еще усилить условие и показать, что  $p \leq \frac{1}{d+1}\frac{1}{e}$, что известно из условия
\end{itemize}
    \EndProof
    \\
    \\
    Теперь докажем общий случай\\
    \Proof
    \\
    $P(\bigcap\limits_{i=1}^n \overline{A_i}) = P(\overline{A_1})P(\overline{A_2}|\overline{A_1})* ... * P(\overline{A_n}| \overline{A_1}\bigcap \overline{A_2} \bigcap...\bigcap \overline{A_{n-1}}) = (1 -P(A_1))(1 - P(A_2|\overline{A_1}))* ... *(1 - P(A_n| \overline{A_1}\bigcap \overline{A_2} \bigcap...\bigcap \overline{A_{n-1}}))$
    \\
    \\
    Нужно показать, что вычитаемое в i-й скобке $\leq x_i$. Тогда получим в точности утверждение теоремы. Чтобы это доказать, воспользуемся леммой ниже. Достаточно будет выбрать лишь нужное J 
    \\
    \textbf{Лемма}
\\
$\forall i \ \forall J \subset \{1,..., n\} \backslash \{i \} \ P(A_i | \bigcap\limits_{j \in J}\overline{A_j}) \leq x_i$\\
\Proof Зафиксируем i. Индукция по |J|.
\begin{itemize}
    \item [1] $|J| = 0$. В этом случае в перечесении мы получим все вероятностное пространство, в котором живут $A_j$. $\Omega \backslash \bigcup A_j = \bigcup \overline{A_j} = \Omega$, откуда $P(A_i) \leq x_i$
    \item [2] Фиксируем $J$
    \item[3]$J_1 = \{j\in J: (A_i, A_j) \in E\}$\\ $J_2 = \{j\in J: (A_i, A_j) \not \in E\}$
    \begin{itemize}
        \item[1] Пусть $J_1$ - пустое. \\ Тогда $P(A_i | \bigcap\limits_{j \in J}\overline{A_j}) = P(A_i|\bigcap\limits_{j \in J_2}\overline{A_j})$ = (в силу независимости в совокупности) $P(A_i) \leq x_i$
        \item[2] $J_1 \neq \emptyset , J_1 = \{j_1,..., j_r\} \\ P(A_i| \bigcap_{j \in J}\overline{A_j}) = \frac{P(A_i \bigcap( \bigcap\limits_{j \in J_1}\overline{A_j})| \bigcap\limits_{j \in J_2}\overline{A_j})}{P(\bigcap\limits_{j \in J_1}\overline{A_j}| \bigcap\limits_{j \in J_2}\overline{A_j})} \leq \frac{P(A_i |\bigcap\limits_{j\in J_2}\overline{A_j})}{P(\bigcap\limits_{j \in J_1}\overline{A_j}| \bigcap\limits_{j \in J_2}\overline{A_j})} \leq \frac{x_i \prod\limits_{j:(A_i, A_j) \in E} (1 - x_j)}{P(\bigcap\limits_{j \in J_1}\overline{A_j}| \bigcap\limits_{j \in J_2}\overline{A_j})}$
        \\
        Осталось показать, что $P(\bigcap\limits_{j \in J_1}\overline{A_j}| \bigcap\limits_{j \in J_2}\overline{A_j}) \geq \prod\limits_{j:(A_i, A_j) \in E} (1 - x_j)$
        \\
        \\
        $P(\bigcap\limits_{j \in J_1}\overline{A_j}| \bigcap\limits_{j \in J_2}\overline{A_j}) =
        P(\overline{A_{j_1}}| \bigcap\limits_{j \in J_2}\overline{A_j})*P(\overline{A_{j_2}}| \bigcap\limits_{j \in J_2}\overline{A_j} \bigcap A_{j_1})* ... * P(\overline{A_{j_r}}| \bigcap\limits_{j \in J_2}\overline{A_j}\bigcap \overline{A_{j_1}} ... \bigcap \overline{A_{j_{r-1}}}) \\ = (1 -P( A_{j_1}| \bigcap\limits_{j \in J_2}\overline{A_j}))(1 -P(A_{j_2}| \bigcap\limits_{j \in J_2}\overline{A_j} \bigcap \overline{A_{j_1}}))* ... * (1 -P(A_{j_r}| \bigcap\limits_{j \in J_2}\overline{A_j}\bigcap \overline{A_{j_1}} ... \bigcap \overline{A_{j_{r-1}}})) \geq \prod\limits_{j \in J_1}(1 - x_j) \geq \prod\limits_{j:(A_i, A_j) \in E}(1 - x_j)$ (Это верно, так как справа мы учитываем не только индексы из J, но и может быть другие, но при этом индексы из  J будут учтены точно)
    \end{itemize}
    \EndProof
\EndProof
\end{itemize}
\section{(8) Пример применения несимметричного случая ЛЛЛ для нижней оценки R(3, t) (вычислять оптимальные параметры не требуется). Самые точные известные оценки для R(3, t) (б/д).}

\textbf{Нижняя оценка $R(3,t)$}

Пусть G(n, p) - случайный граф \\
Рассмотрим события $A_1, ... A_{C_n^3}$ - обозанающие, что соответствующее подмножество вершин графа образует треугольник. $P(A_i) = p^3$
\\
$B_1, ... B_{C_n^t}$ - соответствующее подмножество вершин ггафа образует независимое множество на t вершинах. $P(B_i) = (1-p)^{C_t^2}$
\\
Чтобы оценить $R(3,t)$ снизу числом n, надо показать, что при данном n $P(\bigcap\limits_{i=1}^{C_n^3}(\overline{A_i})\bigcap \bigcap\limits_{j=1}^{C_n^t}(\overline{B_j})) > 0$
\\
\\
Построим граф зависимостей. Он будет иметь стрелки 4х видов:
\\
\includegraphics[]{polina_4.PNG}
\\
\begin{equation*}
\begin{cases}
   \#(A_i \rightarrow A_j) = 3(n-3)\\
  \#(A_i \rightarrow B_j) = 3C_{n-3}^{t-2} + C_{n-3}^{t-3} \\
  \#(B_i \rightarrow B_j) = C_n^t - tC_{n-1}^{t-1} - C_{n-t}^{t} \\
  \#(B_i \rightarrow A_j) = (n-t)C_t^2 + C_t^3
 \end{cases}
\end{equation*}
\\
Для Локальной Леммы Ловаса нужно, чтобы \\
$p^3 \leq x*\prod\limits_{j: (A_i, A_j)}(1-x)*\prod\limits_{j: (A_i, B_j)}(1-y)  =  x(1-x)^{\#(A_i \rightarrow A_j)}(1-y)^{\#(A_i \rightarrow B_j)}\\ (1-p)^{C_t^2} \leq y*\prod\limits_{j: (B_i, B_j)}(1-y)*\prod\limits_{j: (B_i, A_j)}(1-x)  =  y(1-y)^{\#(B_i\rightarrow B_j)}(1-x)^{\#(B_i \rightarrow A_j)}$
\\
\\
Отсюда получаем формулировку 
\\\textbf{Теоремы}
\\
\\
Пусть для данного t число n таково, что \\ $\exists p \in [0,1]\ \exists x \in[0,1) \ \exists y \in[0,1): \\ \\ 1.\ p^3 \leq x(1-x)^{\#(A_i \rightarrow A_j)}(1-y)^{\#(A_i \rightarrow B_j)}
\\
2.\ (1-p)^{C_t^2} \leq y(1-y)^{\#(B_i \rightarrow B_j)}(1-x)^{\#(B_i \rightarrow A_j)}$ \\
Тогда $R(3,t) > n$
\EndProof
\\
\\
\textbf{Теорема}
\begin{center}
    $R(3,t) \geq const*\frac{t^2}{ln^2(t)}$
\end{center}
\textbf{Теорема}
\begin{center}
    $R(3,t) \geq (1 + o(1))\frac{1}{162}\frac{t^2}{ln(t)}$
\end{center}
\textbf{Теорема}
\begin{center}
    $R(3,t) \leq (1 + o(1))\frac{t^2}{ln(t)}$
\end{center}
\textbf{Теорема}
\begin{center}
    $R(3,t) \geq (1 + o(1))\frac{t^2}{4ln(t)}$
\end{center}


% следующие темы см. FILE2.tex