\par \textbf{Неравенство Азумы (оценка отклонения):} Пусть $f$ - липшицева по ребрам. Тогда $$\forall a > 0 \hookrightarrow P(|f - \mathbb{E} f| \geq a) \leq 2e^{-\frac{a^2}{2C_n^2}}$$

\par \Note Для вероятности без модуля справедливо такое же неравенство, но без множителя 2

\par \textbf{Теорема (Боллобаш):} Пусть $G=G(n, \frac{1}{2})$. Тогда $\exists \varphi=\varphi(n)$, что $\varphi=o(\frac{n}{\ln{n}})$ и а.п.н выполнено $$|\chi(G)-\frac{n}{2\log_2 n}| \leq \varphi(n)$$

\par Введем обозначения: \begin{enumerate}
    \item $X_k(G)$ - количество $k$-элементных независимых множеств в $G$
    \item $f_k(n):=\mathbb{E}X_k=C_n^k 2^{-C_k^2}$
    \item $k_0(n):=\min \{k: f_k(n) \geq 1\}$ 
    \par Корректность: $X_k=0$ а.п.н при $k\sim 2\log_2 n \Leftrightarrow \alpha(G) < 2\log_2 n$ а.п.н. Правая часть доказывалась ранее (TODO вставить номер билета).
    \par Значит $k_0$ определено корректно, такое $k$ есть, так как $f_k(n)\rightarrow 0$ при $k\rightarrow \infty$
    \par \textbf{Упражнение:} Если $k \leq 2\log_2 n - 100 \log_2\log_2 n$, то $f_k(n)\rightarrow \infty$. Отсюда следует, что $k_0 \sim 2\log_2 n$
    \item $m:=[\frac{n}{(\ln{n})^2}]$
    \item $k_1(m):=k_0(m)-3 \sim 2\log_2 n$
    \item $Y_k(G):=\max \{t: \exists S_1, \ldots, S_t: \forall i \: |S_i|=k, S_i$ - независимое, $\forall i, j \hookrightarrow |S_i \cap S_j| \leq 1\}$ (интуитивно: максимальный размер гирлянды из $k$-вершинных сарделек (независимых множеств))
\end{enumerate}

\par \Lemma $\mathbb{E}Y_{k_1} \geq \frac{m^2}{2k_1^4}(1+o(1))$
\par \Note В рамках этого доказательства $k:=k_1$
\par \Proof Зафиксируем $G$ на $m$ вершинах. У него есть независимые множества мощности $k$: $\mathcal{K}:=\{K_1, \ldots, K_{X_k(G)}\}$. "Проредим" наши множества: выберем $q^* \in [0; 1]$ - вероятность того, что мы выбираем $K_i, i=1, \ldots, X_k(G)$.

\par Введем обозначения
\begin{enumerate}
    \item $\mu:=\mathbb{E}X_k=f_k(m)=C_m^k \left(\frac{1}{2}\right)^{C_k^2}=m^{3+o(1)}$ (последнее б/д в этом билете)
    \item $W(G):=\{\{K_i, K_j\}: \: K_i, K_j \in \mathcal{K}, \: |K_i \cap K_j| \geq 2\}$
    \item $C(G)$ - прореженное множество
    \item $W'(G, C(G)):=\{\{K_i, K_j\}: \: K_i, K_j \in C(G), \: |K_i \cap K_j| \geq 2\}$
    \item $\mathbb{E}|W|:=\frac{\Delta}{2}$
\end{enumerate}

\par Посчитаем мощности остальных множеств: $\mathbb{E}|C|=\mu q^*$ (разбиваем на индикаторы «данное множество лежит в $C$», всего их $\mathbb{E}X_k$, у каждого вероятность $q^*$), $\mathbb{E}|W'|=\frac{\Delta}{2}(q^*)^2$ (то же самое что $W$, но $K_i, K_j$ должны попасть в $C$, а вероятность этого $(q^*)^2$).

\par Введем $C^*(G)$ - множество, полученное удалением из $C(G)$ по одному множеству из каждой пары из $W'$ Тогда
$$\mathbb{E}|C^*| \geq \mathbb{E}|C|-\mathbb{E}|W'|=\mu q^*-\frac{\Delta}{2} (q^*)^2$$

\par Так как $q^*$ мы можем выбирать сами, нам выгоднее выбрать его так, чтобы оценка была лучше, то есть получился максимум параболы. $$q_{max}=\frac{\mu}{\Delta}<1? \text{ (покажем позже)}$$
$$y_{max}=\frac{\mu^2}{2\Delta}$$

\par $|C^*|$ - это размер какой-то конкретной гирлянды сарделек, а $Y_k$ - максимальный размер. Значит $$\mathbb{E}Y_k \geq \mathbb{E}|C^*|\geq \frac{\mu^2}{2\Delta}$$

\par Хотим показать, что $\frac{\mu^2}{2\Delta} \sim \frac{m^2}{2k^4}$. То есть хотим показать, что $\Delta \sim \frac{\mu^2 k^4}{m^2}$. Если верим в эту эквивалентность, то
$$\frac{\mu}{\Delta}\sim \frac{\mu m^2}{\mu^2 k^4}=\frac{m^2}{k^4 m^{3+o(1)}} \rightarrow 0 \text{ (так как $m, k\rightarrow \infty$ при $n\rightarrow \infty$)}$$
\par то есть $\frac{\mu}{\Delta}$ действительно меньше единицы при больших $n$.

\par Вернемся к асимптотике $\Delta$. Будем выбирать упорядоченные пары $k$-элементных множеств, пересечение которых хотя бы 2 (в $W$ лежат неупорядоченные, то есть мы найдем в 2 раза больше, что и есть $\Delta$).

$$\Delta=2\mathbb{E}|W|=\sum_{t=2}^{k-1} C_m^k C_k^2 C_{m-k}^{k-2} \left(\frac{1}{2}\right)^{2C_k^2 - C_t^2}\sim \frac{\mu^2 k^4}{m^2}$$

\par Первая $C$ - набираем $K_i$, вторая - $K_i \cap K_j$, третья - остатки $K_j$. Асимптотика этой суммы и есть искомая (ей можно пользоваться без доказательства), то есть $\Delta \sim \frac{\mu^2 k^4}{m^2}$ \EndProof
