\begin{lemma} (О наследовании асимптотической нормальности)
	Если $\wh{\theta}_n$ --- асимптотически нормальная оценка $\theta \in \Theta \subseteq \R$ с асимптотической дисперсией $\sigma^2(\theta)$, а также $\tau \colon \R \to \R$ --- дифференцируемая на $\Theta$ функция, тогда $\tau(\wh{\theta}_n)$ --- асимптотически нормальная оценка $\tau(\theta)$ с асимптотической дисперсией $\sigma^2(\theta)(\tau'(\theta))^2$
\end{lemma}

\begin{proof}
	Зафиксируем $\theta \in \Theta$. Из условия мы знаем, что
	\[
		\xi_n := \sqrt{n}(\wh{\theta}_n - \theta) \xrightarrow[n \to \infty]{d_\theta} \xi \sim N(0, \sigma^2(\theta))
	\]
	Воспользуемся подходом из теории вероятностей, который называется \textit{дельта-методом}. Тогда:
	\[
		\frac{\tau(\theta + \xi_nb_n) - \tau(\theta)}{b_n} \xrightarrow[n \to \infty]{d_\theta} \tau'(\theta) \cdot \xi \sim N(0, \sigma^2(\theta)(\tau'(\theta))^2)
	\]
	Несложно понять, что нам нужно взять $b_n = \frac{1}{\sqrt{n}}$, чтобы метод сработал. Отсюда же получаем требуемую сходимость.
\end{proof}

\begin{proposition} (Многомерный случай леммы о наследовании асимптотической нормальности)
	Если $\wh{\theta}_n$ --- асимптотически нормальная оценка $\theta \in \Theta \subseteq \R^k$ с асимптотической матрицей ковариаций $\Sigma(\theta)$, а также $\tau \colon \R^k \to \R^s$ --- дифференцируемая на $\Theta$ функция, тогда $\tau(\wh{\theta}_n)$ --- асимптотически нормальная оценка $\tau(\theta)$ с асимптотической матрицей ковариаций $J_\tau \Sigma J_\tau^T$, где $J_\tau = J_\tau(\theta)$ --- матрица Якоби для $\tau$
\end{proposition}

\begin{proof}
	\textcolor{red}{Аналогично одномерному случаю, просто применяем многомерный дельта-метод}
\end{proof}

\subsection{Методы нахождения оценок}

\begin{note}
	Далее, как обычно, $(\cX, \B(\cX), \cP)$ --- параметрическая модель, причём $\cP = \{P_\theta \colon \theta \in \Theta\}$.
\end{note}

\subsubsection*{Метод подстановки (ОМП)}

\begin{definition}
	Пусть $G$ --- функционал из множества мер над $(\cX, \B(\cX))$ в множество параметров $\Theta$, а также выполнено условие:
	\[
		\forall \theta \in \Theta\ \ \theta = G(P_\theta)
	\]
	Тогда, \textit{оценкой по методу подстановки (ОМП)} называется оценка $\theta_n^*(X_1, \ldots, X_n) = G(P_n^*)$, где $P_n^*$ --- эмпирическое распределение, построенное по выборке $\{X_k\}_{k = 1}^n$
\end{definition}

\begin{example}
	Рассмотрим $\cP = \{Bern(\theta)\}$, $G(P) = \int_\R xdP(x)$. В силу определения распределения Бернулли, $\theta = G(P_\theta)$. При этом для эмпирического распределения имеем такую формулу:
	\[
		G(P_n^*) = \int_\R xdP_n^*(x) = \frac{\sum_{k = 1}^n X_k}{n} = \ol{X}
	\]
\end{example}

\subsubsection*{Метод моментов (ОММ)}

\begin{note}
	Пусть $\cX = \R^m$, $\Theta \subseteq \R^k$. Рассмотрим выборку $\{X_i\}_{i = 1}^n$ из распределения $P \in \cP$, а также борелевские функции $\forall i \in \range{1}{k}\ g_i \colon \R^m \to \R$ (они ещё называются \textit{пробными функциями}). Тогда далее мы принимаем следующие обозначения:
	\begin{itemize}
		\item $m_i(\theta) := \E_\theta g_i(X_1)$
		
		\item $m(\theta) := (m_1(\theta), \ldots, m_k(\theta))^T$
		
		\item $\ol{g} := (\ol{g_1(X)}, \ldots \ol{g_k(X)})^T$
	\end{itemize}
	При этом мы требуем, что $m_i(\theta)$ конечны.
\end{note}

\begin{definition}
	Если существует и единственно решение системы уравнений относительно $\theta$:
	\[
		\System{
			&{m_1(\theta) = \ol{g_1(X)}}
			\\
			&{\vdots}
			\\
			&{m_k(\theta) = \ol{g_k(X)}}
		}
		\Lra m(\theta) = \ol{g}(X)
	\]
	то значение $\theta^* = m^{-1}(\ol{g}(X))$ называется \textit{оценкой по методу моментов (ОММ)}
\end{definition}

\begin{note}
	Стандартными пробными функциями считают $g_i(x) = x^i$
\end{note}

\begin{theorem} (о сильно состоятельности ОММ)
	Пусть выполнены следующие условия на $m \colon \Theta \to m(\Theta)$:
	\begin{itemize}
		\item $m$ --- биекция
		
		\item $m^{-1}$ можно доопределить до функции, заданной на $\R^k$
		
		\item Доопределённая $m^{-1}$ непрерывна в каждой точке $m(\Theta)$
	\end{itemize}
	Тогда оценка по методу моментов $\theta_n^*$ является сильно состоятельной оценкой параметра $\theta$.
\end{theorem}

\begin{proof}
	Зафиксируем $\theta \in \Theta$. По УЗБЧ мы имеем сходимость $\ol{g}(X) \to^{P_\theta\text{ п.н.}} m(\theta)$, а из этого и условия теоремы, по теореме о наследовании сходимости, можем получить такую сходимость:
	\[
		\theta_n^* = m^{-1}(\ol{g}(X)) \xrightarrow{P_\theta\text{ п.н.}} m^{-1}(m(\theta)) = \theta
	\]
\end{proof}

\begin{note}
	Несколько поясним, за что отвечает каждое требование к $m$ в теореме:
	\begin{itemize}
		\item Биективность позволяет заявить, что $m^{-1}(m(\theta)) = m$
		
		\item Доопределение нужно из-за $m^{-1}(\ol{g}(X))$
		
		\item Непрерывность на $m(\theta)$ требуется, чтобы мы могли воспользоваться теоремой о наследовании сходимости
	\end{itemize}
\end{note}

\begin{theorem} (о асимптотической нормальности ОММ)
	Пусть выполнены следующие условия на $m \colon \Theta \to m(\Theta)$:
	\begin{itemize}
		\item $m$ --- биекция
		
		\item $m^{-1}$ можно доопределить до функции, заданной на $\R^k$
		
		\item Доопределённая $m^{-1}$ дифференцируема в каждой точке $m(\Theta)$
		
		\item $\forall i \in \range{1}{k}\ \ \E_\theta g_i^2(X_1) < \infty$
	\end{itemize}
	Тогда оценка по методу моментов $\theta_n^*$ является асимптотически нормальной оценкой параметра $\theta$.
\end{theorem}


\begin{proof}
	По Центральной Предельной Теореме:
	\[
		\sqrt{n}(\ol{g}(X) - m(\theta)) \xrightarrow{d_\theta} N(0, \Sigma)
	\]
	где $\Sigma$ --- матрица ковариаций для вектора $g(X)$. \textcolor{red}{Осталось применить многомерную лемму о наследовании асимптотической сходимости}
\end{proof}

\begin{note}
	Метод моментов на самом деле является частным слуаем метода подстановки, ибо посмотрим на решение системы и реальный вектор $\theta$ в развёрнутом виде:
	\[
		\theta^* = m^{-1} \begin{pmatrix}
			\int_\cX g_1(x)dP_n^*(x)
			\\
			\vdots
			\\
			\int_\cX g_k(x)dP_n^*(x)
		\end{pmatrix}
		=: G(P_n^*); \quad \theta = m^{-1} \begin{pmatrix}
			\int_\cX g_1(x)dP_\theta(x)
			\\
			\vdots
			\\
			\int_\cX g_k(x)dP_\theta(x)
		\end{pmatrix}
		= G(P_\theta)
	\]
\end{note}

\subsubsection*{Метод выборочных квантилей}

\begin{note}
	Здесь мы считаем, что $\cX = \R$
\end{note}

\begin{definition}
	Пусть $P$ --- распределение на $\R$, $F$ --- соответствующая функция распределения и $p \in (0; 1)$. Тогда \textit{p-квантилью распределения $P$} называется следующая точка $z_p$:
	\[
		z_p := \inf \{x \colon F(x) \ge p\}
	\]
\end{definition}

\begin{note}
	В силу правой непрерывности функции распределения $F$ инфинум достигается. При этом $F^{-1}(p)$ совершенно не обязательно всегда подходит:
	\begin{itemize}
		\item Если $F$ непрерывна, то существует точно решение уравнения $F(z_p) = p$, но оно не единственно в силу нестрогого возрастания
		
		\item Если $F$ строго монотонна, то у нас есть гарантия на единственность решения, однако оно не обязательно существует \textcolor{red}{что простите?}
	\end{itemize}
\end{note}

\begin{definition}
	Пусть $\{X_i\}_{i = 1}^n$ --- выборка. Статистика $z_{n, p} = X_{(\ceil{np})}$ называется \textit{выборочной $p$-квантилью}
\end{definition}

\begin{note}
	Несложно понять, что выборочный $p$-квантиль --- это просто $p$-квантиль для эмпирического распределения $P_n^*$
\end{note}

\begin{theorem} (О выборочных квантилях)
	Пусть $\{X_i\}_{i = 1}^n$ --- выборка из распределения $P$ с плотностью $f(x)$. Пусть $z_p$ --- это $p$-квантиль этого распределения, причём $f$ непрерывно дифференцируема в некоторой окрестности $z_p$ и $f(z_p) > 0$. Тогда имеет место сходимость:
	\[
		\sqrt{n}(z_{n, p} - z_p) \xrightarrow{d} N\ps{0, \frac{p(1 - p)}{f^2(z_p)}}
	\]
\end{theorem}

\begin{reminder}
	Пусть $\{X_i\}_{i = 1}^n$ --- выборка из распределения $P$ с плотностью $f(x)$, а $F$ --- функция распределния $P$, то верны следующие факты про порядковые статистики:
	\begin{itemize}
		\item $P(X_{(k)} \le x) = \sum_{m = k}^n C_n^m F^m(x)(1 - F(x))^{n - m}$
		
		\item $p_{X_{(k)}} = n C_{n - 1}^{k - 1} F^{k - 1}(x)(1 - F(x))^{n - k}f(x)$
	\end{itemize}
\end{reminder}

\begin{proof}
	Сходимость по распределению эквивалентна тому, что функции распределения сходятся во всех точках непрерывности своего предела. Мы пронормируем доказываемую сходимость так, чтобы при доказательстве получить справа $N(0, 1)$:
	\[
		\eta_n = \frac{\sqrt{n}(z_{n, p} - z_p)}{\sqrt{\frac{p(1 - p)}{f^2(z_p)}}}
	\]
\end{proof}