% Лекция от 9 октября.

\begin{proof}
	Рассмотрим случаи взаимного расположения $a$ и $\ol \triangle$ (замыкания треугольника).
	\begin{enumerate}
		\item Если $a \not\in \ol \triangle$. Проведем средние линии, получим $\triangle^{(1)}, \triangle^{(2)}, \triangle^{(3)}, \triangle^{(4)}$. Обозначим интеграл по границе треугольника (любого) как $I(\triangle')$:
		\[
			I(\triangle') = \int_{\partial \triangle'} f(z) dz.
		\]
		
		Тогда \begin{align}
			\label{gurse_middle_lines}
			I(\triangle) = I\jleft( \triangle^{(1)} \jright) + I\jleft( \triangle^{(2)} \jright) + I\jleft( \triangle^{(3)} \jright) + I\jleft( \triangle^{(4)} \jright)
		\end{align}
		(к этому равенству есть комментарий после доказательства, посмотрите).

		Оценим модуль интеграла по треугольнику (с помощью неравенства треугольника для модуля):
		\[
			\begin{aligned}
				\mds{I(\triangle)} & = && \mds{I\jleft( \triangle^{(1)} \jright) + I\jleft( \triangle^{(2)} \jright) + I\jleft( \triangle^{(3)} \jright) + I\jleft( \triangle^{(4)} \jright)} & \leq & \\
				& && \mds{I\jleft( \triangle^{(1)} \jright)} + \mds{I\jleft( \triangle^{(2)} \jright)} + \mds{I\jleft( \triangle^{(3)} \jright)} + \mds{I\jleft( \triangle^{(4)} \jright)} & \leq & \\
				& && 4 \mds{I\jleft( \triangle^{(k)} \jright)} && ,
			\end{aligned}
		\]
		где $k$ -- индекс треугольника с наибольшим по модулю интегралом по границе $I\jleft( \triangle^{(k)} \jright)$. Т.е.
		\[
			I \jleft( \triangle^{(k)} \jright) \geqslant \frac{1}{4} I(\triangle).
		\]
		Обозначим $\triangle_1 = \triangle$, $\triangle_2 = \triangle^{(k)}$, $\triangle_3$ и далее построим итеративно таким же образом из предыдущего треугольника в последовательности. (Прим. автора: можно ещё то же самое так сказать: строим последовательность $\triangle_n$, итеративно выбирая из разбиения средними линиями треугольник с максимальным по модулю интегралом по границе, $\triangle_1 = \triangle$.)
		
		Получили последовательность вложенных треугольников:
		\[
			\triangle_1 \subset \triangle_2 \subset \ldots \subset \triangle_n \subset \ldots
		\]
		
		В силу компактности $\ol \triangle$ (прим. автора: он компактен, потому что замкнут и ограничен, при замыкании треугольника, открытый он или замкнутый, мы получаем треугольник с границами; после теоремы есть комментарий, как мы из компактности получаем это свойство):
		\[
			\exists z_0 \in \bigcup_{n = 1}^\infty \triangle_n.
		\]
		
		Т.к. $f$ дифференцируема в $z_0$, то по определению дифференцируемости
		\[
			\begin{aligned}
				f: & (\forall z \in \Cm) \,\, f(z) = f(z_0) + f'(z_0) (z - z_0) + o(z - z_0), \\
				o(z - z_0) = f(z) - f(z_0) - f'(z_0) (z - z_0) : & (\forall \epsilon > 0) (\exists \delta_0 > 0) \jleft( \forall z \in B_{\delta_0}(z_0) \jright) \,\, \mds{o(z - z_0)} \leq \epsilon \mds{z - z_0}.
			\end{aligned}
		\]
		{\color{red} Использовать здесь наше определение дифференцируемости.}
		
		Рассмотрим интеграл по границе для произвольного $\triangle_n$.
		\[
			\int_{\triangle_n} f(z) dz = \int_{\triangle_n} f(z_0) dz + \int_{\triangle_n} f'(z_0) (z - z_0) dz + \int_{\triangle_n} o(z - z_0) dz.
		\]
		Первые два интеграла равны нулю, поскольку интеграл берется по замкнутой кривой и у интегрируемых функций есть первообразные (т.е. под интегралами полные дифференциалы):
		\[
			(z f(z_0))' = f(z_0), \jleft( f'(z_0) \frac{(z - z_0)^2}{2} \jright)' = f'(z_0) (z - z_0).
		\]
		Тогда
		\[
			\int_{\triangle_n} f(z) dz = \int_{\triangle_n} o(z - z_0) dz.
		\]
		
		Зафиксируем $\epsilon > 0$. {\color{red} Продолжение следует.}
		
		
	\end{enumerate}
\end{proof}
\begin{anote}
	По всей видимости в теореме треугольник может быть и без границы, с частичной границей, может быть внутренность треугольника. Иначе мы бы не говорили про замыкание.
\end{anote}
\begin{anote}
	{\color{red} Здесь будет картинка для \ref{gurse_middle_lines}.}
\end{anote}
% Нарисовать картинку с треугольниками в tikz по координатам, объяснить, как получается равенство из свойств интеграла по кривой: надо воспользоваться аддитивностью.