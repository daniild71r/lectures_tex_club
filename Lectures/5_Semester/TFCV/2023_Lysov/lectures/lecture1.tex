\section{Комплексные числа}

\subsection{Базовые определения и свойства}

Напомним некоторые определения из 1-го семестра матана. Этого не было в лекциях, но нужно напомнить, чтобы изложение конспекта было связным.

\begin{reminder}
	\it{Множеством комплексных чисел} называют множество $\Cm = \R^2$.
\end{reminder}

\begin{reminder}
	\it{Мнимой единицей} $i$ называют комплексное число $(0, 1)$. По определению сложения и умножения комплексных чисел оно имеет свойство:
	\[
		i^2 = -1.
	\]
\end{reminder}

\begin{reminder}
	Множество $\R$ вложено в $\Cm$. Для $a \in \R$ можно поставить в соответствие $(a, 0) \in \Cm$. Для комплексных чисел, соответствующих вещественным (так называемых чисто вещественных), операции будут точно такими же, как и с обычными вещественными числами (сумма чисто вещественных $(a, 0)$ и $(b, 0)$ будет переводить в соответствующее сумме $a + b$ комплексное $(a + b, 0)$, например).
\end{reminder}
\begin{reminder}
	Пусть $z = (a, b) \in \Cm$; алгебраической формой $z$ называется запись в виде
	$z = a + bi$, где $a$ называется вещественной частью ($\re z$), $b$ -- мнимой частью ($\im z$).
\end{reminder}
\begin{anote}
	По факту, вещественная часть -- первый элемент пары, мнимая часть -- второй элемент.
\end{anote}

С напоминаниями всё, дальше материал лекций.

\begin{definition}
	Пусть $i \in \Cm$; $i$ называется \it{мнимой единицей}, если $i^2 = -1$.
\end{definition}
\begin{anote}
	Определение как в лекции. Зачем мы так делаем? Можно определить комплексные числа разными способами, на памяти есть два: как пары вещественной и мнимой части, так определяли на матане в первом семестре; как $\R[x] \slash \jleft< x^2 + 1 \jright>$, так могут определить на алгебре (в таких многочленах $x$ можно обозначить как $i$, поскольку $x^2$ отождествляется с $-1$). Потому наше осторожное определение, не смотрящее на структуру $\Cm$, имеет смысл, но в дальнейшем будем предполагать определение с матана.
\end{anote}

\begin{proposition}
	Если $z_1 \in \Cm$, $z_2 \in \Cm$, тогда $z_1 = z_2 \Lra \re(z_1) = \re(z_2) \,\land\, \im(z_1) = \im(z_2)$
\end{proposition}
\begin{proof}
	Моментально из определений комплексного числа и вещественной и мнимой частей, т.к. по определению из анализа $\Cm = \R^2$, а равенство вектора эквивалентно равенству компонент (компоненты -- вещественная и мнимая части, по их определениям).
\end{proof}

Зафиксируем до конца параграфа $z \in \Cm$, $x = \re(z)$, $y = \im(z)$. Тогда $z = x + iy$.

\begin{definition}
	\textit{Комплексным сопряжением} $z$ называется $\overline{z} = x - iy$.
\end{definition}
\begin{note}
	Сопряжение является инволюцией: $\overline{\overline{z}} = z$.
\end{note}
\begin{note}
	Справедливы формулы для вещественной и мнимой частей:
	\[
		\re z = \frac{z + \overline{z}}{2}, \quad \im z = \frac{z - \overline{z}}{2}.
	\]
\end{note}

\begin{definition}
	\it{Модулем} $z$ называется $\jleft\lvert z \right\rvert = \sqrt{z \ol z} = \sqrt{x^2 + y^2}$.
\end{definition}

\begin{proposition}[основные свойства сопряжения]
	Если $a, b \in \Cm$, тогда
	\begin{enumerate}
		\item $\ol{a + b} = \ol a + \ol b$,
		\item $\ol{ab} = \ol a \ol b$,
		\item $\ol{\jleft( \frac{a}{b} \jright)} = \frac{\ol a}{\ol b}$, если $b \neq 0$.
	\end{enumerate}
\end{proposition}
\begin{proof}
	Будет.
\end{proof}

\begin{anote}
	Многое из сказанного далее будет полным аналогом утверждений для векторов из $\R^2$ и стандартной нормы.
\end{anote}

\begin{proposition}[основные свойства модуля]
	Если $a, b \in \Cm$, тогда
	\begin{enumerate}
		\item $\mds{ab} = \mds a \mds b$,
		\item $\mds{\frac{a}{b}} = \frac{\mds a}{\mds b}$,
		\item $\ol{\jleft( \frac{a}{b} \jright)} = \frac{\ol a}{\ol b}$, если $b \neq 0$.
	\end{enumerate}
\end{proposition}
\begin{proof}
	Будет.
\end{proof}

\begin{proposition}[свойства неравенств]
	Если $a, b \in \Cm$, тогда
	\begin{enumerate}
		\item $\mds a = \mds{\ol a}$,
		\item $- \mds a \le \re a \le \mds a$, $- \mds a \le \im a \le \mds a$,
		\item $\mds a \le \re a + \im a$,
		\item $\mds{a + b} \le \mds a + \mds b$ (неравенство треугольника)
	\end{enumerate}
\end{proposition}
\begin{proof}
	Будет.
\end{proof}

\begin{note}
	Комплексное число $z$ можно представлять как точку на плоскости (или, эквивалентно, направляющий вектор к ней из нуля). Тогда сложение и вычитание чисел -- сложение и вычитание векторов, а из полярных координат для вектора появляются расстояние и угол $\varphi$ (от оси $x$ к нему). Будем считать, что угол в полуинтервале $(-\pi, \pi]$. Стандартное расстояние до точки и норма вектора совпадают с модулем (по определению модуля). Из полярных координат число представимо в тригономертрической форме:
	\[
		z = \mds z \cos \phi + i \mds z \sin \phi = \mds z (\cos \phi + i \sin \phi),
	\]
	для $z = 0$ угол можем взять любым.
	{\color{red} Здесь бы картинку с лекции.}
\end{note}
\begin{anote}
	Для ненулевых $z$ тригонометрическое представление можно получить и аналитически. Поделим на модуль числа обе компоненты, а затем возьмём арккосинус и арксинус. Для нулевых $z$, как мы понимаем, угол не однозначен.
\end{anote}

\begin{note}
	Пусть $z \neq 0$; аргументом числа $z$ называется $\arg z$ -- угол из геометрического представления.
\end{note}
\begin{note}
	Пусть $z \neq 0$; множество $\Arg z$ определим так:
	\[
		\Arg z = \{ \arg z + 2 \pi k \mid k \in \Z \}.
	\]
\end{note}

\begin{corollary}[неравенство параллелограмма]
	Если $a, b \in \Cm$, тогда 
	\[
		\mds{a - b}^2 + \mds{a + b}^2 = 2 \mds{a}^2 + 2 \mds{b}^2.
	\]
\end{corollary}

\begin{example}
	Вычислим произведение двух комплексных чисел, записанных в тригонометрической форме:
	\[
		\begin{aligned}
			z_1 z_2 = & \mds{z_1} \mds{z_2} (\cos \phi_1 + i \sin \phi_1) (\cos \phi_2 + i \sin \phi_2) = \\
			& \mds{z_1} \mds{z_2} (\cos \phi_1 \cos \phi_2 - \sin \phi_1 \sin \phi_2 + i (\cos \phi_1 \sin \phi_2 + \cos \phi_2 \sin \phi_1)) = \\
			& \mds{z_1 z_2} (\cos(\phi_1 + \phi_2) + i \sin(\phi_1 + \phi_2)).
		\end{aligned}
	\]
\end{example}

{\color{red} Здесь будет следствие.} \\
%TODO: доделать!

{\color{red} Формула Муавра.} \\
%TODO: доделать!

{\color{red} Здесь будет пример.} \\
%TODO: доделать!


\begin{definition}
	\it{Окрестностью точки} $z$ радиуса $r$ ($r > 0$) называется множество
	\[
		O_r(z) = \jleft\{ z' \in \Cm: \mds{z - a} < r \jright\}.
	\]
\end{definition}

\begin{definition}
	\it{Замкнутой окрестностью} точки $z$ радиуса $r$ ($r > 0$) называется множество
	\[
	{\ol O}_r(z) = \jleft\{ z' \in \Cm: \mds{z - z'} \le r \jright\}.
	\]
\end{definition}
\begin{anote}
	Замыкание окрестности совпадает с замкнутой окрестностью.
\end{anote}
%TODO. Доказательство от автора.

\begin{definition}
	\it{Проколотой окрестностью} точки $z$ радиуса $r$ ($r > 0$) называется множество
	\[
		{\dot O}_r(z) = \jleft\{ z' \in \Cm: 0 < \mds{z - z'} < r \jright\}.
	\]
\end{definition}

\subsection{Сходимость последовательностей}

Для краткости до конца параграфа ${\{ a_n \}}_{n = 1}^\infty \subset \Cm$ -- последовательность комплексных чисел, $a \in \Cm$ -- комплексное число.

\begin{definition}
	Последовательность ${\{ a_n \}}_{n = 1}^\infty$ \it{сходится} к $a$ ($a_n \to a$), если
	\[
		(\forall \epsilon > 0) \, (\exists N \in \N) \, (\forall n > N) \,\, \mds{a_n - a} < \epsilon.
	\]
	Комплексное число $a$ называется пределом последовательности ${\{ a_n \}}_{n = 1}^\infty$.
\end{definition}

\begin{note}
	Сходимость последовательности комплексных чисел эквивалентна одновременной сходимости вещественных и мнимых частей:
	\[
		a_n \to a \Lra \re a_n \to \re a \,\land\, \im a_n \to \im a.
	\] 
\end{note}
%TODO. доказательство от автора; на лекции не было (насколько мне известно.

\begin{definition}
	Последовательность ${\{ a_n \}}_{n = 1}^\infty$ называется \it{бесконечно большой}, если
	\[
		(\forall \epsilon > 0) \, (\exists N \in \N) \, (\forall n > N) \,\, \mds{a_n} > \epsilon.
	\]
\end{definition}

\begin{definition}
	\it{Расширенной комплексной плоскостью} $\ol \Cm$ называется $\Cm \cup \{ \infty \}$.
\end{definition}
\begin{anote}
	Добавляем элемент, который будем считать пределом бесконечно больших последовательностей {\color{red} Спросить, это ли имел в виду лектор.}
\end{anote}

\begin{definition}
	Последовательность ${\{ a_n \}}_{n = 1}^\infty$ называется \it{ограниченной}, если
	\[
		(\exists r > 0) \, (\forall n \in \N) \,\, \mds{a_n} < r.
	\]
\end{definition}

\begin{proposition}
	Любая последовательность в $\ol \Cm$ имеет сходящуюся подпоследовательность.
\end{proposition}

%TODO. Доказательство от автора; на лекции, насколько понимаю, не было.

\begin{definition}
	Последовательность ${\{ a_n \}}_{n = 1}^\infty$ называется \it{фундаментальной}, если
	\[
		(\forall \epsilon > 0) \, (\exists N \in \N) \, (\forall n, m > N) \, \mds{a_n - a_m} < \epsilon.
	\]
\end{definition}

\begin{proposition}
	Последовательность ${\{ a_n \}}_{n = 1}^\infty$ фундаментальна тогда и только тогда, когда имеет конечный предел.
\end{proposition}
%TODO. На лекции пропустили. Доказать от автора, сказать хотя бы пару слов.

\subsection{Сходимость функций}

Для краткости до конца параграфа $a, A \in \Cm$, $r \in (0, \infty)$ (число, которое подойдет для того, чтобы быть радиусом).

\begin{definition}
	Пусть $f: \dot{O}_r(a)$; \it{функция} $f(z)$ \it{сходится} к $A$ при стремлении $z$ к $a$, если
	\[
		(\forall \epsilon > 0) \, (\exists \delta > 0) \, (\forall z \in \dot{O}_\delta(a)) \,\, \mds{f(z) - A} < \epsilon.
	\]
	Обозначается как $\ulim{z \to a} f(z) = A$.
\end{definition}

\begin{definition}
	Пусть $f: O_r(a) \to \Cm$; $f$ непрерывна в точке $a$, если $\ulim{z \to a} f(z) = f(a)$.
\end{definition}

\begin{note}
	Если $f: O_r(a)$, $f$ непрерывна в $a$, тогда $\re f$ и $\im f$ непрерывны в точке $a$.
\end{note}

% А обратное не верно? Разобраться. Для функций нескольких переменных было ли эквивалентно? Если не было, сделать замечание от автора. Спросить у Миши, если он захочет вспоминать.
%TODO. сказать хотя бы пару слов об этом.


