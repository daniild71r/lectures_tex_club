% Лекция от 16 сентября.

\section{Интегрирование}

На лекции по ТФКП мы не определяли снова, что такое кривая. Дальше идет напоминание материала первого курса матана.

\begin{definition}
	Пусть $\alpha, \beta \in \R$, $\alpha \leq \beta$, $z: [\alpha, \beta] \to \Cm$, $(\forall t \in [\alpha, \beta]) \,\, z = x(t) + i y(t)$, $x$ и $y$ непрерывны; $z$ называется \it{параметризацией кривой}. (Модифицировано автором.)
\end{definition}

\begin{definition}
	Пусть $z_1: [\alpha_1, \beta_1] \to \Cm$, $z_2: [\alpha_2, \beta_2] \to \Cm$, $z_1$ и $z_2$ -- параметризации кривых; $z_1$ и $z_2$ эквивалентны, если существует непрерывная функция $\varphi: [\alpha_1, \beta_1] \to [\alpha_2, \beta_2]$ такая, что $z_1 = z_2 \circ \varphi$.
\end{definition}

\begin{anote}
	Получилось отношение эквивалентности: рефлексивно, достаточно взять тождественное отображение, оно непрерывно; симметрично, т.к. $\varphi^{-1}$ существует и непрерывна по свойствам непрерывных функций из $\R$ в $\R$, $z_1 \circ \varphi^{-1} = z_2 \circ \varphi \circ \varphi^{-1} = z_2$; транзитивно: если $z_1 = z_2 \circ \varphi_1$, $z_2 = z_3 \circ \varphi_2$, то $z_1 = (z_3 \circ \varphi_2) \circ \varphi_1 = z_3 \circ (\varphi_2 \circ \varphi_3)$, при этом $\varphi_2 \circ \varphi_3$ непрерывна как композиция непрерывных функций.
\end{anote}

\begin{definition}
	Класс эквивалентности параметризаций кривых по определенному выше отношению эквивалентности называется кривой.
\end{definition}

\begin{anote}
	У любых двух эквивалентных параметризаций одно и то же множество значений: $(\forall t \in [\alpha_1, \beta_1]) \, \, z_1(t) = (z_2 \circ \varphi)(t)$, тогда $z_1([\alpha_1, \beta_1]) = z_2(\varphi([\alpha_1, \beta_1])) \subset z_2([\alpha_2, \beta_2])$. Можно показать включение в другую сторону благодаря симметричности определенного нами отношения.
\end{anote}

\begin{definition}
	Пусть $z: [\alpha, \beta] \to \Cm$, $z$ -- параметризация кривой; \it{образом кривой} называется $z([\alpha, \beta])$.
\end{definition}

Дальше идет материал лекции.

{\color{red} Сказать, что такое минус кривая: это обход её в другом направлении, разворот параметризации; возможно, у нас на лекции кривые ориентированные были, в отношении эквивалентности должны быть неубывающие непрерывные. Определялось ли на матане это -- не знаю. }.

Пусть $D$ -- область в $\Cm$; $z: [\alpha, \beta] \to \Cm$, $z$ -- параметризация кривой $\gamma$, $\gamma \in \Cm$ (прим. автора: подразумевается, что образ кривой лежит в области; кривую считаем в том числе и множеством точек); $f: D \to \Cm$, $f$ непрерывна в $D$, $f(x + iy) = u(x, y) + iv(x, y)$ для любых $x, y \in \R$.

Возьмём разбиение $P = (t_0, \ldots, t_n)$ отрезка $[\alpha, \beta]$ (прим. автора: разбиение определяли на матане, во втором семестре; $t_0 = \alpha$, $t_n = \beta$, $\jleft( \forall k \in \ol{1, n} \jright) \,\, t_{k - 1} < t_k$; в разбиении скобочки, потому что порядок точек важен). Рассмотрим интегральную сумму.
\[
	\begin{aligned}
		S(f, P) = & && \sum_{k = 1}^n \underbrace{f(z(t_k))}_{(u+iv)(z(t_k))} \underbrace{\jleft( z(t_k) - z(t_{k - 1}) \jright)}_{\Delta x_k + i \Delta y_k} = \\
		& && \sum_{k = 1}^n \jleft[ u(z(t_k)) \Delta x_k - v(z(t_k)) \Delta y_k \jright] + \\
		& i && \sum_{k = 1}^n \jleft[ u(z(t_k)) \Delta y_k + v(z(t_k)) \Delta x_k \jright].
	\end{aligned}
\]
(Прим. автора: по ходу дела ввели приращения параметризации по $x$ и по $y$, вещественной и мнимой частям.) Устремим к нулю мелкость дробления $\Delta P$, получим интеграл по кривой.

\begin{definition}[интеграл от функции по кривой]
	Интегралом от функции $f$ по кривой $\gamma$ называется $\int_\gamma f(z) dz = \lim_{\Delta(P) \to 0} S(f, P)$.
\end{definition}
\begin{note}
	При выполнении предположений выше, интеграл всегда существует и
	\[
		\int_\gamma f(z) dz = \int_\gamma (u dx - v dy) + i \int_\gamma (v dx + u dy).
	\]
\end{note}
\begin{anote}
	Перед нами криволинейные интегралы второго рода (здесь мы пользуемся тем, что $\Cm = \R^2$, но с доопределённым умножением). Мы определяли их в третьем семестре матана аналогично криволинейным интегралам первого рода (в интегралах первого рода функции и кривые в трехмерном пространстве). Кому интересно, это страница 67 конспекта КТЛ лекций А. Л. Лукашова за осень 2022-2023 учебного года.
\end{anote}

Рассмотрим ещё другую сумму, теперь домножаем на модуль приращения параметризации.
\[
	\begin{aligned}
		S'(f, P) = & && \sum_{k = 1}^n \underbrace{f(z(t_k))}_{(u+iv)(z(t_k))} \mds{ z(t_k) - z(t_{k - 1})}\\
		= & && \sum_{k = 1}^n \underbrace{f(z(t_k))}_{(u+iv)(z(t_k))} \mds{\Delta z(t_k)}.
	\end{aligned}
\]
При $\Delta P$ стремящемуся к нулю $S'(f, P)$ стремится к интегралу $\int_\gamma f(z) \mds{dz}$:
\[
	\int_\gamma f(z) \mds{dz} = \int_\gamma u ds + i \int_\gamma v ds,
\]
где $ds = \sqrt{{(z_1'(t))}^2 + {(z_2'(t))}^2} dt$.

\begin{anote}
	Комбинируя всё вместе, получим:
	\[
		\int_\gamma f(z) \mds{dz} = \int_\alpha^\beta f(z) \sqrt{{(z_1'(t))}^2 + {(z_2'(t))}^2} dt.
	\]
\end{anote}

{\color{red} Вот бы примеры вычисления сюда от себя добавить, но это как-нибудь потом.}

\begin{note}
	Длину кривой можно вычислить так (обозначение $\mds{\gamma}$ от автора, не с лекции):
	\[
		\mds{\gamma} = \int_\gamma \mds{dz}
	\]
\end{note}
\begin{theorem}[свойства интеграла от функции по кривой, без доказательства]
	Верны следующие свойства.
	\begin{enumerate}
		\item $\int_\gamma f dz$ и $\int_\gamma f \mds{dz}$ не зависят от параметризации.
		\item $\int_{-\gamma} f dz = - \int_{\gamma} f dz$, $\int_\gamma f \mds{dz} = \int_{-\gamma} f \mds{dz}$.
		\item Линейность: $\int_\gamma (af + bg) dz = a \int_\gamma f dz + b \int_\gamma g dz$.
		\item Аддитивность: $\int_{\gamma_1 \cup \gamma_2} f dz = \int_{\gamma_1} f dz + \int_{\gamma_2} f dz$.
		\item Неравенства: $\mds{\int_\gamma f dz} \leq \int_\gamma \mds f \mds{dz} \leq \max_{p \in \gamma} \mds{f(p)} \mds{\gamma}$.
	\end{enumerate}
\end{theorem}

