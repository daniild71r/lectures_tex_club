\section{Дифференцируемость в $\Cm$}

Зафиксируем до конца параграфа $a \in \Cm$, $f: O_r(a) \to \Cm$, $f(z) = u(\re z, \im z) + i v(\re z, \im z)$, где $u, v: \R^2 \to \R$ (разложили функцию на функции вещественной и мнимой части; если считать, что $\Cm = \R^2$, то даже $u, v: \Cm \to \R$).

\begin{definition}
	Функция $f$ дифференцируема в точке $a$, если
	\[
	(\exists A' \in \Cm) \,\, \ulim{z \to a} \frac{f(z) - f(a)}{z - a} = A',
	\]
	(т.е. существует предел). $A'$ обозначается как $f'(z_0)$.
\end{definition}
\begin{note}
	Существование предела эквивалентно утверждению:
	\[
		(\exists A' \in \Cm) \,\, f(z_0 + \Delta z) - f(z_0) = \Delta f = A' \Delta z + o(\Delta z),
	\]
	где o-малое взято при $\Delta z \to 0$, можно использовать его в определении.	
\end{note}
%TODO: доказать, хотя бы пару слов сказать.


