\begin{definition}
	Пространство $E$ называется \textit{линейным нормированным}, если выполнено 2 условия:
	\begin{enumerate}
		\item $E$ --- линейное пространство над $K$
		
		\item В пространстве $E$ существует \textit{оператор нормы} $\|\cdot\| \colon E \to \R_+$. Он удовлетворяет следующим условиям:
		\begin{enumerate}
			\item $\forall x \in E\ \ \|x\| \ge 0 \wedge \|x\| = 0 \Lra x = 0$
			
			\item $\forall x \in E,\ \alpha \in \R\ \ \|\alpha x\| = |\alpha| \cdot \|x\|$
			
			\item $\forall x, y \in E\ \ \|x  + y\| \le \|x\| + \|y\|$
		\end{enumerate}
	\end{enumerate}
\end{definition}

\begin{proposition}
	Любое линейное нормированное пространство является метрическим с индуцированной нормой метрикой $\rho(x, y) = \|x - y\|$.
\end{proposition}

\begin{proof}
	Проверим все свойства метрики:
	\begin{enumerate}
		\item (Неотрицательность и аксиома тождества) $\forall x, y \in X\ \rho(x, y) = \|x - y\| \ge 0$. Если $\rho(x, y) = \|x - y\| = 0$, то по определению нормы $x - y = 0 \Lra x = y$
		
		\item (Симметричность) 
		\[
			\forall x, y \in X\ \ \rho(x, y) = \|x - y\| = \|(-1) \cdot (y - x)\| = |-1| \cdot \|y - x\| = \|y - x\| = \rho(y, x)
		\]
		
		\item (Неравенство треугольника) 
		\[
			\forall x, y, z \in X\ \ \rho(x, z) = \|x - z\| \le \|x - y\| + \|y - z\| = \rho(x, y) + \rho(y, z)
		\]
	\end{enumerate}
\end{proof}

\begin{definition}
	Полное линейное нормированное пространство называется \textit{банаховым}.
\end{definition}

\begin{example}~
	\begin{enumerate}
		\item $\R^n$ является банаховым пространством
		
		\item $C[a; b]$ со своей собственной нормой $\|f\| = \min_{[a; b]} |f|$ является банаховым
		
		\item $C[a; b]$ с нормой $\|f\|_p = \ps{\int_a^b |f(x)|^pd\mu(x)}^{1 / p}$
	\end{enumerate}
\end{example}

\begin{note}
	Далее буква $E$ закрепляется за линейным нормированным пространством. 
\end{note}

\begin{definition}
	\textit{Линейным многообразием} $L \subseteq E$ называется такое подмножество, которое само по себе является линейным пространством.
\end{definition}

\begin{definition}
	Пространство $L \subseteq E$ называется \textit{подпространством в $E$}, если $L$ --- замкнутое линейное многообразие.
\end{definition}

\begin{example}
	Рассмотрим множество многочленов $\cP$ в пространстве $C[a; b]$. По теореме Вейерштрасса $\cl \cP = C[a; b] \neq \cP$, поэтому $\cP$ является только линейным многообразием.
\end{example}

\begin{definition}
	\textit{Линейной оболочкой множества} $S \subseteq E$ называется множество всех конечных линейных комбинаций элементов из $S$. Обозначается как $[S]$ или $\Lin S$
\end{definition}

\begin{anote}
	В этот момент читатель мог заметить, что обозначение с квадратными скобками может обозначать как замкнутость множества, так и теперь линейную оболочку. Здесь могу лишь отослать к странице про обозначения: для замыкания в конспекте везде используется $\cl$, а для линейной оболочки будет $[\cdot]$.
\end{anote}

\begin{definition}
	Норма $\|\cdot\|_1$ \textit{слабее, чем} норма $\|\cdot\|_2$, если выполнено условие:
	\[
		\exists C > 0 \such \forall x \in E\ \ \|x\|_1 \le C\|x\|_2
	\]
\end{definition}

\begin{example}
	В пространстве $C[a; b]$ норма $\|\cdot\|_1$ слабее нормы $\|\cdot\|_{C[a; b]}$:
	\[
		\int_a^b |f(x)|dx \le \max_{x \in [a; b]} |f(x)| \cdot (b - a) = \|f\|_C \cdot (b - a)
	\]
\end{example}

\begin{exercise} (от автора)
	Докажите, что отношение слабости на нормах является предпорядком.
\end{exercise}

\begin{definition}
	Нормы $\|\cdot\|_1$, $\|\cdot\|_2$ эквивалентны, если выполнено условие:
	\[
		\exists C_1, C_2 > 0 \such \forall x \in E\ \ C_1\|x\|_1 \le \|x\|_2 \le C_2\|x\|_1
	\]
\end{definition}

\begin{exercise} (от автора)
	Докажите, что введённое отношение эквивалентности норм соответствует аксиомам отношения эквивалентности.
\end{exercise}

\begin{exercise}
	Верно ли, что если $E$ является линейным нормированным пространством с нормой $\|\cdot\|_1$, то для эквивалентной нормы $\|\cdot\|_2$ это тоже так?
\end{exercise}

\begin{definition}
	Пусть $E$ --- линейно нормированное пространство над $\R$. Тогда множество $S \subseteq E$ называется \textit{выпуклым}, если выполнено утверждение:
	\[
		\forall x, y \in S\ \forall \lambda \in [0; 1]\ \ \lambda x + (1 - \lambda)y \in S
	\]
\end{definition}

\begin{definition}
	\textit{Базисом Гамеля в пространстве $E$} называется набор векторов \\ $\{e_k\}_{k \in \mathcal{K}} \subseteq E$ такой, что верно 2 условия:
	\begin{enumerate}
		\item Любой конечный поднабор $\{e_{k_t}\}_{t = 1}^n$ линейно независим
		
		\item Любой вектор пространства $E$ выражается конечной линейной комбинацией векторов из $E$
	\end{enumerate}
\end{definition}

\begin{anote}
	Базис Гамеля может быть любым с точки зрения мощности $\mathcal{K}$. Наглядный пример: если считать $\R$ линейным пространством над $\Q$, то в нём имеется несчётный базис Гамеля.
\end{anote}

\begin{definition}
	\textit{Базисом Шаудера в пространстве $E$} называется набор векторов $\{e_k\}_{k \in \mathcal{K}}^\infty \subseteq E$, $\mathcal{K}$ не более чем счётно, такой, что любой вектор из $E$ раскладывается по этой системе единственным образом:
	\[
		\forall v \in E\ \ \exists! \{\alpha_k\}_{k \in \mathcal{K}} \subseteq \K \such v = \sum_{k \in \mathcal{K}} \alpha_ke_k
	\]
\end{definition}

\begin{proposition} (от автора)
	Если в пространстве $E$ существует конечный базис Гамеля или Шаудера, то он является также базисом Шаудера или Гамеля соответственно. Стало быть, такую конструкцию можно называть просто базисом.
\end{proposition}

\begin{definition} (Не по лектору)
	\textit{Размерностью} $E$ называется мощность базиса в пространстве $E$. Если конечного базиса не существует, то размерность равна бесконечности. Обозначается как $\dim E$
\end{definition}

\begin{definition}
	Пусть $E_1, E_2$ --- линейные нормированные пространства. Отображение $A \colon E_1 \to E_2$ называется \textit{оператором}.
\end{definition}

\begin{definition}
	Пусть $E$ --- линейное нормированное пространство над $\K$. Тогда отображение $f \colon E \to \K$ называется \textit{функционалом}.
\end{definition}

\begin{anote}
	Норма --- это как раз пример функционала.
\end{anote}

\begin{theorem}
	Пусть $\dim E < \infty$. Тогда на $E$ все нормы эквивалентны.
\end{theorem}

\begin{lemma} (Непрерывность нормы)
	Если $x \in E, \{x_n\}_{n = 1}^\infty \subseteq E$, причём $\lim_{n \to \infty} x_n = x$, то $\lim_{n \to \infty} \|x_n\| = \|x\|$.
\end{lemma}

\begin{proof}
	Для сходимости элементов $E$ есть тривиальная эквивалентность \\ $\lim_{n \to \infty} \|x_n - x\| = 0$. Для доказательства требуемого достаточно воспользоваться неравенством с модулями:
	\[
		\forall a, b \in \K\ \ \big||a| - |b|\big| \le |a - b| \Lora \big|\|x_n\| - \|x\|\big| \le \|x_n - x\| \xrightarrow[n \to \infty]{} 0
	\]
\end{proof}

\begin{corollary}
	Если $\lim_{n \to \infty} \|x_n\| = 0$, то и $\lim_{n \to \infty} x_n = 0$
\end{corollary}

\begin{proof}
	Просто заметить, что $\lim_{n \to \infty} \|x_n - 0\| = \lim_{n \to \infty} \|x_n\| = 0$
\end{proof}

\begin{proof} (теоремы)
	Для простоты, приведём доказательство в случае $\K = \R$ (остаётся не рассмотренным $\K = \Cm$, но он отличается только сопряжением). Рассмотрим ортонормированный базис $\{e_1, \ldots, e_n\}$. Покажем, что произвольная норма $\|\cdot\|_1$ является эквивалентной к евклидовой норме, порождённой ортогональным базисом:
	\[
		\|x\| = \no{\sum_{k = 1}^n \alpha_k e_k} = \sqrt{\sum_{k = 1}^n \alpha_k^2}
	\]
	\begin{itemize}
		\item[$\Ra$] Обозначим $\kappa = \max_{k \in \range{1}{n}} \|e_k\|_1$. Имеет место цепочка неравенств:
		\[
			\|x\|_1 = \no{\sum_{k = 1}^n \alpha_ke_k}_1 \le \sum_{k = 1}^n \|\alpha_ke_k\|_1 = \sum_{k = 1}^n |\alpha_k| \cdot \|e_k\|_1 \le \kappa \sum_{k = 1}^n |\alpha_k| \le \kappa \sqrt{n} \cdot \|x\|
		\]
		Последний переход --- неравенство Коши между средним арифметическим и квадратичным.
		
		\item[$\La$] Предположим противное. Тогда, верно утверждение $\forall C > 0\ \exists x \in E \such C\|x\| \ge \|x\|_1$. Стало быть, для $C_n = \frac{1}{n}$ найдётся соответствующая последовательность $\{x_n\}_{n = 1}^\infty$. По ней построим $y_n = \frac{x_n}{\|x_n\|}$ --- последовательность точек на единичной сфере в евклидовой норме. Заметим, что единичная сфера в таком пространстве является замкнутым и ограниченным множеством, то есть компактом. Стало быть, из $y_n$ можно выделить сходящуюся подпоследовательность $\{y_{n_k}\}_{k = 1}^\infty$, $y_0 = \lim_{k \to \infty} y_{n_k}$, причём $\|y_0\| = 1$. Осталось заметить противоречие:
		\[
			\|y_{n_k}\|_1 = \frac{\|x_{n_k}\|_1}{\|x_{n_k}\|} \le \frac{1}{n_k} \xrightarrow[k \to \infty]{} 0 \Ra \lim_{k \to \infty} y_{n_k} = 0
		\]
		Этого не может быть, коль скоро предел единственен и он уже должен не равняться нулю.
	\end{itemize}
\end{proof}

\begin{theorem}
	Пусть $\{e_1, \ldots, e_n\} \subseteq E$. Тогда $[e_1, \ldots, e_n]$ является подпространством $E$.
\end{theorem}

\begin{proof}
	Для того, чтобы $L = [e_1, \ldots, e_n]$ стало подпространством, нам нехватает замкнутости. Однако её можно тривиально получить, если доказать, что $L$ --- банахово пространство. Итак, мы знаем, что $\dim L = d \le n < \infty$, то есть конечномерное пространство. Значит, в $L$ все нормы эквивалентны. Вспомним об изоморфизме $L \cong \R^d$ из алгебры. Так как в $\R^d$ полно, то то же самое верно и про $L$ с евклидовой метрикой. Осталось сказать, что евклидова норма в $L$ эквивалентна индуцированной из $E$ норме, а тогда $L$ полно и с индуцированной метрикой, что и требовалось показать.
\end{proof}

\begin{theorem} (Ф. Рисса)
	Пусть $E$ --- бесконечномерное пространство. Тогда единичная сфера в $E$ не является вполне ограниченной.
\end{theorem}

\begin{corollary}
	Единичная сфера не компактна.
\end{corollary}

\begin{note}
	Наша цель --- доказать теорему Рисса. Проникнемся интуицией к доказательству, рассмотрев частный случай $E = \ell_2$. Несложно заметить, что единичной нормой (то есть лежат на единичном шаре в $\ell_2$) обладают элементы такого вида:
	\[
		\forall n \in \N\quad e_n = (\underbrace{0, \ldots, 0}_{n - 1}, 1, 0, \ldots)
	\]
	Более того, любые из этих двух элементов всегда удалены на расстояние, большее единицы (а если быть точным, то $\rho(e_n, e_m) = \sqrt{2}$). Стало быть, никакой конечной $\eps$-сетью (для, например, $\eps \le 1$) это множество не покрыть. Возможно ли применить ту же идею в общем случае?
\end{note}

\begin{lemma} (о <<почти перпендикуляре>>)
	Пусть $E_1 \subset E$ --- подпространство. Тогда выполнено утверждение:
	\[
		\forall \eps > 0\ \exists y \in E \colon \System{
			&{\|y\| = 1}
			\\
			&{\rho(y, E_1) > 1 - \eps}
		}
	\]
\end{lemma}

\begin{note}
	Тот факт, что $E_1$ --- не просто линейное многообразие, а подпространство, существенно.
	
	Рассмотрим $E = C[0; 1]$, $E_1 = \mathcal{P}$. Тогда $E_1$ --- не подпространство. Более того, $\cl E_1 = E$, поэтому отдалиться от $E_1$ на хоть какое-то ненулевое расстояние просто невозможно.
\end{note}

\begin{proof}
	Коль скоро $E_1 \neq E$, то существует $y_0 \notin E_1$. Введём обозначение $d = \rho(y_0, E_1)$. Сразу понятно, что в силу замкнутости не может быть $d = 0$, иначе $y_0 \in E_1$. Из определения расстояния, в частности, верно утверждение:
	\[
		\forall \eps > 0\ \exists z_0 \in E_1 \such d \le \|y_0 - z_0\| < d(1 + \eps)
	\]
	Посмотрим на вектор $y = \frac{y_0 - z_0}{\|y_0 - z_0\|}$. Обозначим коэффициент при векторе за $\alpha$. Сразу видно, что $\|y\| = 1$, поэтому осталось проверить только расстояние до $E_1$:
	\[
		\forall z \in E_1\quad \rho(y, z) = \|y - z\| = \|\alpha(y_0 - z_0) - z\| = |\alpha| \cdot \Bigg\|y_0 - \underbrace{\ps{z_0 + \frac{z}{\alpha}}}_{\in E_1}\Bigg\| \ge \alpha \cdot d
	\]
	Выстроим неравенство на $\alpha$ из неравенства, связывающего $y_0$ и $z_0$:
	\[
		d \le \frac{1}{\alpha} < d(1 + \eps) \Lora \frac{1}{d} \ge \alpha > \frac{1}{d(1 + \eps)}
	\]
	Таким образом, $\alpha d > \frac{1}{1 + \eps}$. Последняя оценка стремится к единице с ростом $\eps$, а поэтому для произвольного $\eps'$ мы можем подобрать $\eps$ так, чтобы выполнить неравенство $\frac{1}{1 + \eps} > 1 - \eps'$.
\end{proof}

\begin{proof} (теоремы Рисса)
	Рассмотрим произвольный $x_1 \in E\colon \|x_1\| = 1$. Тогда он не является базисом в силу условия. Рассмотрим $E_1 = \tbr{x_1}$. Тогда $E_1 \neq E$ и можно воспользоваться леммой о почти перпендикуляре. Так мы находим $x_2$. Дальше повторяем то же самое с $E_2 = \tbr{x_1, x_2}$. Получаем в итоге последовательность $\{x_n\}_{n = 1}^\infty$ со следующими свойствами:
	\[
		\System{
			&{\forall n \in \N\ \ \|x_n\| = 1}
			\\
			&{\forall n \neq m\ \ \rho(x_n, x_m) > 1 - \eps}
		}
	\]
	Понятно, что существование такого подмножества сферы полностью ломает вполне ограниченность (нет $\frac{1 - \eps}{2}$-сети, например). Следовательно, сфера не может быть компактом.
\end{proof}
