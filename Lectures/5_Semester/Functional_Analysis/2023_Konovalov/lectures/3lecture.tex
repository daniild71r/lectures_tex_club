\begin{definition}
	Пусть $X, Y$ --- топологические пространства. Тогда мы говорим, что они \textit{гомеоморфны}, если существует функция $f \colon X \to Y$ такая, что она удовлетворяет двум свойствам:
	\begin{enumerate}
		\item $f$ --- биекция
		
		\item $f, f^{-1}$ непрерывны на всём своем пространстве
	\end{enumerate}
	Гомеоморфизм топологических пространств обозначается как $X \simeq Y$
\end{definition}

\begin{exercise}
	Пусть $(X, \rho)$ --- метрическое пространство. Тогда $x \in X$ --- точка прикосновения множества $M$ тогда и только тогда, когда существует последовательность $\{m_k\}_{k = 1}^\infty \subseteq M$ такая, что $\lim_{k \to \infty} m_k = x$
\end{exercise}

\begin{exercise}
	В метрическом пространстве определение предела через шары и произвольные открытые множества эквивалентны.
\end{exercise}

\begin{exercise}
	Если $(X, \Tau)$ --- связное топологическое пространство и $f \colon X \to Y$ --- непрерывное отображение в тоже топологическое пространство $Y$, то $f(X)$ является связаным топологическим пространством.
\end{exercise}
 
\section{Полные метрические пространства}

\begin{definition}
	Метрическое пространство $X$ называется \textit{полным}, если в нём любая фундаментальная последовательность сходится
\end{definition}

\begin{example}~
	\begin{itemize}
		\item Полные метрические пространства:
		\begin{itemize}
			\item $\R$
			
			\item $\Cm$
			
			\item $L_p[a; b],\ p \ge 1$
		\end{itemize}
		
		\item Неполные метрические пространства:
		\begin{itemize}
			\item $\Q$
			
			\item $C_p[a; b],\ p \ge 1$
		\end{itemize}
	\end{itemize}
\end{example}

\begin{theorem} (Принцип вложенных шаров)
	Пусть $(X, \rho)$ --- полное метрическое пространство, а $\{\ole{B}_n(x_n, r_n)\}_{n = 1}^\infty$ --- последовательность замкнутых вложенных шаров, $r_n \to 0$. Тогда существует и единственна точка $x \in \bigcap_{n = 1}^\infty \ole{B}_n$.
\end{theorem}

\begin{proof}
	Интуитивно понятно, что к гипотетической точке $x$ должна сходиться последовательность центров шаров. Действительно, покажем существование такого предела. В силу полноты $X$, нам достаточно показать фундаментальность $\{x_n\}_{n = 1}^\infty$. А это просто, ибо верно неравенство $\rho(x_{n + p}, x_n) \le r_n$. Осталось воспользоваться условием, что $r_n \to 0$, и фундаментальность тривиально установлена. Обозначим $x := \lim_{n \to \infty} x_n$. Если мы снова посмотрим на оценку выше и устремим $p$ в бесконечность (в силу сходимости это уже можно), то получится замечательный факт:
	\[
		\forall n \in \N\ \ \rho(x, x_n) \le r_n \Lra x \in \ole{B}_n(x_n, r_n)
	\]
	Осталось показать, что $x$ --- единственная точка пересечения шаров. Действительно, если есть отличная $y$, то $\rho(x, y) > 0$, а тогда мы знаем, что начиная с некоторого номера все шары находятся в сколь угодно малой окрестности $x$, отсюда противоречие с $y$.
\end{proof}

\begin{note}
	Все условия теоремы существенны:
	\begin{itemize}
		\item При отсутствии замкнутости существует такое равенство: $\bigcap_{n = 1}^\infty \ps{0; \frac{1}{n}} = \emptyset$
		
		\item При отсутствии вложенности совсем тривиально, что пересечение может быть пустым
		
		\item Если радиусы шаров не стремятся к нулю, то можно рассмотреть $X = \N$ с такой метрикой:
		\[
			\rho(m, n) = \System{
				&{1 + \frac{1}{m + n},\ m \neq n}
				\\
				&{0,\ m = n}
			}
		\]
		Тогда последовательность шаров $\ole{B}\ps{1 + \frac{1}{2n}}$ является искомым контрпримером
	\end{itemize}
\end{note}

\begin{note}
	Принцип вложенных шаров является критерием полноты метрического пространства. Доказываться этот факт не будет.
\end{note}

\begin{proposition}
	Следующие свойства эквивалентны:
	\begin{itemize}
		\item $M \subseteq X$ --- нигде не плотное множество
		
		\item $\forall B(x)\ \exists B(y) \subseteq B(x) \such B(y) \cap M = \emptyset$
		
		\item $\forall G\ \exists G_1 \subseteq G \such M \cap G_1 = \emptyset$
	\end{itemize}
\end{proposition}

\begin{theorem} (Бэра)
	Пусть $(X, \rho)$ --- полное метрическое пространство. Тогда $X$ нельзя представить в виде $\bigcup_{n = 1}^\infty M_n$, где $M_n$ --- нигде не плотное множество
\end{theorem}

\begin{proof}
	Предположим противное. Тогда $X = \bigcup_{n = 1}^\infty M_n$, где $M_n$ --- нигде не плотное множество. Дальнейшая идея состоит в том, чтобы воспользоваться принципом вложенных шаров и найти точку, которая не будет принадлежать ни одному $M_n$ (а должна, как точка $X$). Итак, найдём наши шары:
	\begin{enumerate}
		\item Рассмотрим $x_1 \in X$ и шар, скажем, $B(x_1, 1)$. Так как $M_1$ нигде не плотно, то должен найтись шар $B(X_1, r) \subseteq B(x_1, 1)$ такой, что $B(x_1, r) \cap M_1 = \emptyset$ (это тривиально следует из определения, ведь иначе $\cl M_1 \supseteq B(x_1, 1)$). В этом шаре мы можем выбрать замкнутый шар $\ole{B}(x_1, r_1) \subseteq B(x_1, r)$, который послужит отправной точкой
		
		\item Для нахождения следующего замкнутого шара, рассмотрим произвольный шар \\ $B(x_k, r') \subset \ole{B}(x_{k - 1}, r_{k - 1})$ и повторим процедуру выше для него.
	\end{enumerate}
	Итак, мы получили последовательность вложенных замкнутых шаров. В силу полноты метрического пространства, их пересечение соответствует ровно одной точке $x \in \bigcap_{k = 1}^\infty B(x_k, r_k)$. Осталось формально увидеть 2 уже оговоренных факта:
	\begin{itemize}
		\item $x \in X \Ra \exists k_0 \in \N \such x \in M_{k_0}$
		
		\item $x \in \bigcap_{k = 1}^\infty \ole{B}(x_k, r_k) \wedge \big(\forall k \in \N\ \ole{B}(x_k, r_k) \cap M_k\big) = \emptyset \Lora \forall k \in \N\ x \notin M_k$
	\end{itemize}
	Достигнуто явное противоречие.
\end{proof}

\begin{example} (из книги К. Иосида <<Функциональный Анализ>>, 1967г.)
	Покажем пространство, где можно применить теорему Бэра. Пространство $C[0; 1]$ --- это полное метрическое пространство, а $M \subset C[0; 1]$ --- это множество всех непрерывных функций, у которых хотя бы в одной точке есть хотя бы одна односторонняя производная. Несложно показать, что $M$ представимо в виде счётного объединения нигде не плотных множеств. Это интуитивно означает, что даже таких функций невероятно мало.
\end{example}