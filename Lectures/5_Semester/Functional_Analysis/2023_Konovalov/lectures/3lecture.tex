 \textcolor{red}{Дописать первый час лекции}
 
 \section{Полные метрические пространства}
 
 \begin{definition}
 	Метрическое пространство $X$ называется \textit{полным}, если в нём любая фундаментальная последовательность сходится
 \end{definition}

\begin{example}
	\textcolor{red}{Тут должны быть примерчики разных пространств}
\end{example}

\begin{theorem}
	Пусть $(X, \rho)$ --- полное метрическое пространство, а $\{\ole{B}_n(x_n, r_n)\}_{n = 1}^\infty$ --- последовательность замкнутых вложенных шаров, $r_n \to 0$. Тогда существует и единственна точка $x \in \bigcap_{n = 1}^\infty \ole{B}_n$.
\end{theorem}

\begin{proof}
	Интуитивно понятно, что к $x$ должна сходиться последовательность центров шаров. Действительно, покажем существование такого предела. В силу полноты $X$, нам достаточно показат фундаментальность $\{x_n\}_{n = 1}^\infty$. Итак, рассмотрим расстояние:
	\[
		\rho(x_{n + p}, x_n) \le r_n
	\]
	Осталось воспользоваться условием, что $r_n \to 0$, и фундаментальность тривиально установлена. Пусть $x = \lim_{n \to \infty} x_n$. Если мы снова посмотрим на оценку выше и устремим $p$ в бесконечность (в силу сходимости это уже можно), то получится замечательный факт:
	\[
		\forall n \in \N\ \ \rho(x, x_n) \le r_n \Lra x \in \ole{B}_n(x_n, r_n)
	\]
	Осталось показать, что $x$ --- единственная точка пересечения шаров. Действительно, если есть отличная $y$, то $\rho(x, y) > 0$, а тогда мы знаем, что начиная с некоторого номера все шары находятся в сколь угодно малой окрестности $x$, отсюда противоречие с $y$.
\end{proof}

\begin{theorem} (Бэра)
	Пусть $(X, \rho)$ --- полное метрическое пространство. Тогда $X$ нельзя представить в виде $\bigcup_{n = 1}^\infty M_n$, где $M_n$ --- нигде не плотное множество
\end{theorem}

\begin{proof}
	Предположим противное. Тогда $X = \bigcup_{n = 1}^\infty M_n$, где $M_n$ --- нигде не плотное множество 
\end{proof}

\begin{example}
	Покажем пространство, где можно применить теорему Бэра. Пространство $C[0; 1]$ --- это полное метрическое пространство, а $M \subset C[0; 1]$ --- это множество всех неперрывных функций, у которых хотя бы в одной точке есть хотя бы одна односторонняя производная
\end{example}