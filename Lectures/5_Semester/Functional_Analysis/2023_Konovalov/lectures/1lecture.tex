\begin{definition}
	Множество $X$ называется \textit{топологическим пространством}, если ему сопоставлена система множеств $\tau \subseteq 2^X$ --- \textit{топология}, обладающая следующими свойствами:
	\begin{enumerate}
		\item $\emptyset, X \in \Tau$
		
		\item $\forall \goth{A}\ \forall \{G_\alpha\}_{\alpha \in \goth{A}} \subseteq \Tau\ \ \bigcup_{\alpha \in \goth{A}} G_\alpha \in \Tau$
		
		\item $\forall \goth{A}\ \forall \{G_\alpha\}_{\alpha \in \goth{A}} \subseteq \Tau\ \ \bigcap_{\alpha \in \goth{A}} G_\alpha \in \Tau$
	\end{enumerate}
\end{definition}

\begin{example} (Положительный терминатор, строитель)
	Пусть задана система дифференциальных уравнений:
	$$
		\System{
			&{\dot{x} = f(t, x)}
			\\
			&{x(t_0) = x_0}
		}
	$$
	Решением этой системы будет функция (при выполнении теоремы Коши), которую можно записать так:
	$$
		x(t) = x_0 + \int_0^t f(s, x(s))ds
	$$
\end{example}