\begin{proof}~
	\begin{enumerate}
		\item[$1 \Ra 2$] 
		\begin{itemize}
			\item Покажем, что $X$ --- полное. Пусть $\{x_n\}_{n = 1}^\infty$ --- фундаментальная последовательность. Тогда:
			\[
				\forall \eps > 0\ \exists N \in \N \such \forall n \ge N, p \in \N\ \ \rho(x_{n + p}, x_n) < \eps
			\]
			Рассмотрим множества $A_n = \{x_n, x_{n + 1}, \ldots\}$. Если $n \ge N$, то выполнена цепочка вложений:
			\[
				A_n \subseteq B(x_n, \eps) \Lora \cl A_n \subseteq \cl B(x_n, \eps) \subseteq \ole{B}(x_n, \eps)
			\]
			При этом $\{\cl A_n\}_{n = 1}^\infty$ --- центрированная система замкнутых множеств. В силу компактности $X$, пересечение всех $\cl A_n$ непусто. Обозначим $x \in \bigcap_{n = 1}^\infty \cl A_n$. Осталось показать, что $\lim_{n \to \infty} x_n = x$. А из выше сказанного, это уже тривиально:
			\[
				x \in \bigcap_{n = 1}^\infty \cl A_n \Ra x \in \bigcap_{n = N}^\infty A_n \subseteq \bigcap_{n = N}^\infty \ole{B}(x_n, \eps)
			\]
			Это и есть ровно то, что мы хотели: $\forall n \ge N\ \ \rho(x, x_n) \le \eps$.
			
			\item Осталось разобраться с вполне ограниченностью. Зафиксируем $\eps > 0$ и найдём $\eps$-сеть. Рассмотрим самое простое открытое покрытие $X$ --- покрытие шарами с центрами в каждой точке $X$, $X = \bigcup_{x \in X} B(x, \eps)$. В силу компактности, существует конечное подпокрытие $X = \bigcup_{i = 1}^k B(x_i, \eps)$. Несложно понять, что $\{x_i\}_{i = 1}^k$ --- искомая сеть.
		\end{itemize}
	
		\item[$2 \Ra 3$] Пусть $\{x_n\}_{n = 1}^\infty$. В силу полноты, нам достаточно найти фундаментальную подпоследовательность.  Итак, в силу вполне ограниченности $X$:
		\[
			\forall \eps > 0\ \exists \{y_j\}_{j = 1}^k\ \such X = \bigcup_{j = 1}^k \ole{B}(y_j, \eps)
		\]
		Заметим, что любой $\ole{B}(y_j, \eps)$ тоже является полным и вполне ограниченным пространством. Применим итерационный метод, $\eps = \frac{1}{m}$, $m \to \infty$. На итерации $m = 1$ мы находим шар $\ole{B}(y_1, 1)$, в котором содержится бесконечное число элементов $\{x_n\}_{n = 1}^n$. Выделяем эти элементы в последовательность $\{x_{1, m}\}_{n = 1}^\infty$  На итерации $m > 1$ мы уже используем $\frac{1}{m}$-сеть из пространства $\ole{B}(y_{j_{m - 1}}, \frac{1}{m - 1})$, в ней находим шар $\ole{B}(y_{j_m}, \frac{1}{m})$, который снова содержит бесконечное число элементов $\{x_{n, m - 1}\}_{n = 1}^\infty$ и продолжаем так счётное число раз. Осталось применить диагональный метод Кантора и показать, что последовательность $\{t_m := x_{m, m}\}_{m = 1}^\infty$ является фундаментальной. Это действительно так:
		\[
			\forall m_1 \le m_2\ \ t_{m_2} \in \ole{B}\ps{t_{m_1}, \frac{1}{m_1}}
		\]
		
		\item[$3 \Ra 1$] 
		\begin{itemize}
			\item Докажем вполне ограниченность от противного. Предположим существует $\eps_0 > 0$ такое, что любой конечный набор $\{x_n\}_{n = 1}^k \subseteq X$ не является $\eps_0$-сетью. Стало быть, $\{x_1\}$ не является $\eps_0$-сетью, то есть найдётся $x_2 \colon \rho(x_1, x_2) > \eps_0$. Аналогично повторяем для $\{x_1, x_2\}$ и так далее. Этим методом мы нашли последовательность $\{x_n\}_{n = 1}^\infty$ такую, что $\rho(x_n, x_m) > \eps_0$ для любого $n \neq m$. Стало быть, из такой последовательность нельзя выделить сходящуюся подпоследовательность, противоречие
			
			\item Имея на руках доказанную вполне ограниченность $X$, мы готовы доказать исходную импликацию. Снова предположим противное: $X = \bigcup_{\alpha \in \gA} G_\alpha$, причём в этом открытом покрытии нельзя выделить конечное подпокрытие. С другой стороны, у нас есть $\eps$-сети $X = \bigcup_{i = 1}^k \ole{B}(x_k, \eps)$. Несложно понять, что в силу определения открытого покрытия, должен существовать хотя бы один шар $\ole{B}(x_k, \eps)$, который не может быть покрыт конечным числом $G_\alpha$. Снова прибегнем к итерационному методу $\eps = \frac{1}{n}$ и будем искать соответствующие шары из $\eps$-сетей для $X$. Получим $\ole{B}(x_n, 1 / n)$ --- последовательность шаров, каждый из которых нельзя покрыть конечным числом $G_\alpha$. По условию, мы можем выделить сходящуюся подпоследовательность $\{x_{n_k}\}_{k = 1}^\infty$, $\lim_{k \to \infty} x_{n_k} = x_0$. Коль скоро $x_0 \in X = \bigcup_{\alpha \in \gA} G_\alpha$, то существует $\alpha_0 \colon x_0 \in G_{\alpha_0}$. Так как $G_{\alpha_0}$ --- открытое, то существует $r_0 \colon B(x_0, r_0) \subseteq G_{\alpha_0}$. В силу уже упомянутого предела, $\exists K \in \N \colon B(x_{n_K}, \frac{1}{n_K}) \subseteq B(x_0, r_0) \subseteq G_{\alpha_0}$, получили противоречие с построением $\{x_n\}_{n = 1}^\infty$
		\end{itemize}
		
		\item[$3 \Lra 4$] В сторону из $3 \La 4$ доказательство тривиальное, поэтому нас интересует только $3 \Ra 4$. Пусть $M$ бесконечно. Раз так, то в нём есть как минимум 1 элемент $x_1 \in M$, причём $M \bs x_1 \neq \emptyset$. Повторив эту итерацию счётное число раз (при этом требуя $x_n \in M \bs \{x_k\}_{k = 1}^{n - 1}$), мы получили какую-то произвольную последовательность $\{x_n\}_{n = 1}^\infty \subseteq M$, причём все элементы различны. Стало быть, из неё можно выделить сходящуюся подпоследовательность, предел которой будет сразу предельной точкой $M$.
		
		\item[$2 \La 5$] Докажем отрицание: если $X$ - неполное или не вполне ограниченное, то $C(X, \K) \centernot\subseteq B(X, \K)$. Приведём соответствующие 2 конкретных примера, когда в таких пространствах найдётся соотвествующая непрерывная неограниченная функция, и мы обобщим их идеи на общий случай:
		\begin{enumerate}
			\item $X = (0; 1)$ - неполное пространство. На нём определена $f(x) = 1 / x$. Заметим, что эту функцию можно переписать так:
			\[
				f(x) = \frac{1}{x} = \frac{1}{\rho(x, 0)}
			\]
			Похожую функцию $f$ мы будем строить. Если $X$ - произвольное неполное пространство, то по теореме Хаусдорфа есть изометрия $\pi \colon X \to Y$ такая, что \\ $\cl \pi(X) = Y$. Коль скоро $X$ неполно, то должен существовать $y_0 \in Y \colon \forall x \in X\ \pi(x) \neq y_0$, при этом $y_0$ является предельной точкой для $X$. Стало быть, мы можем рассмотреть функцию $f(x) = \frac{1}{\rho(x, y_0)}$. Эта функция обязана быть непрерывной в $X$, но при этом она неограничена (в силу, опять же, возможности стремления к $y_0$ из $X$)
			
			\item $X = \R$ - не вполне ограниченное пространство. На нём есть $f(x) = x$. Будем действовать схожим образом: раз $X$ не вполне ограничено, то можно найти $x_1 \in X$ с открытым шаром $B(x_1)$ такой, что эта окрестность не попадает в покрытие $1$-сети. Более того, мы можем найти и $x_2 \in X$ со своей окрестностью, ибо если не смогли, то мы просто достроили $\eps$-сеть \textcolor{red}{Требует лучшего пояснения}. Зададим функцию на этих окрестностях так, чтобы было равенство $f(x_n) = n$, а от соответствующей точки происходит убывание до нуля, причём ноль достигается на границе $U(x_n)$. Формально, такую функцию можно записать так:
			\[
				f(x) = \sum_{n = 1}^\infty n(r_n - \rho(x, x_n))^+
			\]
			где $a^+ = \max\{a, 0\}$. Тривиальным образом эта функция непрерывна, но неограничена в силу последовательности $\{x_n\}_{n = 1}^\infty$
		\end{enumerate}
	
		\item[$1 \Ra 5$] \textcolor{red}{Основывается на теореме Кантора}.
	\end{enumerate}
\end{proof}