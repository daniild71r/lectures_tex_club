\begin{exercise}~
	\begin{enumerate}
		\item Если $Y \subseteq X$ --- компакт в метрическом пространстве, то $Y$ замкнуто в $X$
		
		\item Если $X$ --- компактное метрическое пространство, то $X$ сепарабельно
		
		\item Если $X$ --- полное метрическое пространство, тогда $Y \subseteq X$ компактно тогда и только тогда, когда оно замкнуто и вполне ограничено
	\end{enumerate}
\end{exercise}

\begin{theorem} (Кантора)
	Пусть $X$ --- компактное метрическое пространство, $f \in C(X, K)$, где $K \in \{\R, \Cm\}$. Тогда $f \in \hat{C}(X, K)$:
	\[
		\forall \eps > 0\ \exists \delta > 0 \such \forall x, y \in X,\ \rho(x, y) < \delta\ \ |fx - fy| < \eps
	\]
\end{theorem}

\begin{proof} (Прямое доказательство)
	Раз $X$ --- компактное метрическое пространство, то мы можем рассмотреть любой $x \in X$ и для него существует шар $B(x, r_x) \subseteq X$. В силу непрерывности $f$ на $X$, у нас имеется непрерывность и в каждой точке:
	\[
		\forall \eps > 0\ \exists B(x, r_x) \such \forall y \in B(x, r_x)\ \ |fx - fy| < \eps
	\]
	Чтобы найти $r$, подходящее для каждой точки при фиксированном $\eps > 0$, вспомним о компактности и наличии покрытия $X = \bigcup_{x \in X} B(x, r_x)$. Выделим из покрытия конечное, пусть в нём $n$ элементов. Тогда $\delta := \min_{k \in \range{1}{n}} \frac{r_k}{2}$ и его достаточно. Действительно, пусть $\rho(x, y) < \delta$. В таком случае, есть шар $B(x_k, r_k)$, содержащий обе этих точки. Но шары выбирались исходя из непрерывности $f$, то есть:
	\[
		|fx - fy| \le |fx - fx_k| + |fx_k - fy| < 2\eps
	\]
\end{proof}

\begin{proof} (Косвенное доказательство)
	Предположим противное. Тогда для $f$ не выполнено условие равномерной непрерывности:
	\[
		\exists \eps_0 > 0\ \forall \delta > 0 \such \exists x, y \in X,\ \rho(x, y) < \delta\ \ |fx - fy| \ge \eps_0
	\]
	Дальше схема обычная. Рассмотрим $\delta = \frac{1}{n}$, получим последовательности $\{x_n\}_{n = 1}^\infty$ и $\{y_n\}_{n = 1}^\infty$. В силу компактности $X$, можем выделить сходящуюся к некоторому $x_0$ подпоследовательность $\{x_{n_k}\}_{k = 1}^\infty$. Но тогда есть предел $\lim_{k \to \infty} y_{n_k} = x_0$, а это уже ведёт к противоречию с фактом $|fx_{n_k} - fy_{n_k}| \ge \eps_0$.
\end{proof}

\begin{definition}
	Пусть $X$ --- метрическое пространство. $Y \subseteq X$ называется \textit{предкомпактным}, если $\cl Y$ --- компактно. В такой ситуации говорят, что \textit{$Y$ компактно относительно $X$}.
\end{definition}

\begin{theorem} (Арц\'{е}ла-Аск\'{о}ли)
	Пусть $X$ --- компактное метрическое пространство, $M \subseteq C(X, K)$. Множество $M$ является предкомпактным тогда и только тогда, когда выполнено 2 условия:
	\begin{enumerate}
		\item $M$ ограничено
		
		\item $M$ равностепенно непрерывно, то есть выполено условие:
		\[
			\forall \eps > 0\ \exists \delta >0 \such \forall x, y \in X,\ \rho(x, y) < \delta\ \forall f \in M\ \ |fx - fy| < \eps
		\] 
	\end{enumerate}
\end{theorem}

\begin{proof}~
	\begin{itemize}
		\item[$\Ra$] Будем доказывать каждый пункт отдельно:
		\begin{enumerate}
			\item Коль скоро $X$ компактно, оно также вполне ограничено. Отсюда следующее:
			\[
				\forall \eps > 0\ \exists \{\phi_k\}_{k = 1}^n \such \forall f \in M\ \exists \phi_k\ \ \rho(f, \phi_k) = \|f - \phi_k\| < \eps
			\]
			Стало быть, $\|f\|$ можно оценить таким образом:
			\[
				\forall f \in M\ \ \|f\| \le \|f - \phi_k\| + \|\phi_k\| < \eps + \max_{k \in \range{1}{n}} \|\phi_k\|
			\]
			
			\item В силу компактности, $\{\phi_k\}_{k = 1}^n$ --- равномерно непрерывные функции, то есть
			\[
				\forall \eps > 0\ \exists \delta_k > 0 \such \forall x, y \in X,\ \rho(x, y) < \delta_k\ \ |\phi_kx - \phi_ky| < \eps
			\]
		\end{enumerate}
		Положим $\delta := \min_{k \in \range{1}{n}} \delta_k$. Тогда оценку получить несложно:
		\[
			|fx - fy| \le |fx - \phi_k x| + |\phi_k x - \phi_k y| + |\phi_k y - fy| < 3\eps
		\]
		
		\item[$\La$] 
	\end{itemize}
\end{proof}

\section{Линейно нормированные пространства}

\begin{exercise}
	Пусть $E$ --- линейно нормированное пространство над $K$, $M = \tbr{e_1, \ldots, e_n}$
\end{exercise}