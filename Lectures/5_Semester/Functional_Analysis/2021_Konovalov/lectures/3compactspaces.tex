\section{Компактные метрические пространства}

\begin{definition}
    Метрическое пространство $X$ называется \textit{компактным}, если для любой системы открытых множеств $\{G_\alpha\}_{\alpha \in \mf A} \subset 2^X$ такой, что $\bigcup_{\alpha \in \mf A} G_\alpha = X$, существует конечный набор $\alpha_1, \dotsc, \alpha_n \in \mf A$ такой, что $\bigcup_{k = 1}^n G_{\alpha_k} = X$.
\end{definition}

\begin{note}
    Определение выше можно сформулировать так: из любого открытого покрытия пространства $X$ можно выделить конечное подпокрытие. Или, эквивалнентно, если система открытых множеств не содержит конечного покрытия пространства $X$, то она не является покрытием пространства $X$.
\end{note}

\begin{definition}
    Пусть $X$ "--- метрическое пространство. Система $\{B_\alpha\}_{\alpha \in \mf A} \subset 2^X$ называется центрированной, если для любого конечного набора $\alpha_1, \dotsc, \alpha_n \in \mf A$ выполнено $\bigcap_{k = 1}^n B_{\alpha_k} \ne \emptyset$.
\end{definition}

\begin{theorem}\label{thm3.1}
    Метрическое пространство $X$ компактно $\lra$ любая центрированная система замкнутых множеств в $X$ имеет непустое пересечение.
\end{theorem}

\begin{proof}
    Каждой системе открытых множеств $\{G_\alpha\}_{\alpha \in \mf A} \subset 2^X$ можно поставить в соответствие систему замкнутых множеств $\{F_\alpha\}_{\alpha \in \mf A} := \{X \bs G_\alpha\}_{\alpha \in \mf A}$, и наоборот. Тогда любая система открытых множеств $\{G_\alpha\}_{\alpha \in \mf A}$, не содержащая конечного покрытия, не является покрытием $\lra$ любая центрированная система замкнутых множеств $\{F_{\alpha}\}_{\alpha \in \mf A}$ имеет непустое пересеченеие.
\end{proof}

\begin{definition}
    Пусть $X$ "--- метрическое пространство. Множество $M \subset X$ называется \textit{вполне ограниченным}, если для любого $\epsilon > 0$ существует конечный набор точек $x_1, \dotsc, x_n \in X$, называемый \textit{$\epsilon$-сетью}, такой, что $\bigcup_{k = 1}^n B(x_k, \epsilon) = X$.
\end{definition}

\begin{theorem}[критерий компактности]\label{thm3.2}
    Пусть $X$ "--- метрическое пространство. Тогда следующие условия эквивалентны:
    \begin{enumerate}
        \item $X$ компактно
        \item $X$ полно и вполне ограниченно
        \item Из любой последовательности $\{x_n\} \subset X$ можно выделить сходящуюся подпоследовательность $\{x_{n_k}\}$
        \item Любое бесконечное множество $M \subset X$ имеет предельную точку
    \end{enumerate}
\end{theorem}

\begin{proof}~
    \begin{itemize}
        \item\imp{1}{2}$X$ вполне ограниченно, поскольку для любого $\epsilon > 0$ из открытого покрытия $\{B(x, \epsilon)\}_{x \in X}$ по условию можно выделить конечное подпокрытие, дающее требуемую $\epsilon$-сеть. Остается доказать полноту пространства $X$.
        
        Пусть последовательность $\{x_n\} \subset X$ фундаментальна. Для каждого $n \in \N$ положим $A_n := \{x_n, x_{n+1}, \dotsc \}$, тогда система $\{\overline{A_n}\}$ является центрированной системой замкнутых множеств, поэтому можно выбрать точку $x_0 \in \bigcap_{n=1}^\infty \overline{A_n}$. В силу фундаментальности, для любого $\epsilon > 0$ существует $N \in \N$ такое, что при всех $n > N$ выполнено $\overline{A_n} \subset \overline B(x_n, \epsilon)$, откуда и $\rho(x_n, x_0) < \epsilon$. Значит, $x_n \to_X x_0$.
        
        \item\imp{2}{3}Зафиксируем произвольную последовательность $\{x_n\} \subset X$. Поскольку $X$ вполне ограниченно, то для любого $\epsilon > 0$ существует $y \in X$ такое, что шар $B(y, \epsilon)$ содержит бесконечно много точек из $\{x_n\}$. Будем применять это рассуждение сначала для всего пространства $X$, потом для шаров в $X$, содержащих бесконечно много точек из $\{x_n\}$:
        \begin{itemize}
            \item[$\bullet$] Для $\epsilon := 1$ выберем $\{x^1_k\} \subset \{x_n\}$ так, что $\{x^1_k\} \subset B(y_1, 1)$
            \item[$\bullet$] Для $\epsilon := \frac 12$ выберем $\{x^2_k\} \subset \{x^1_n\}$ так, что $\{x^2_k\} \subset B(y_2, \frac 12) \subset B(y_1, 1)$
            \item[$\bullet$] Для $\epsilon := \frac 13$ выберем $\{x^3_k\} \subset \{x^2_n\}$ так, что $\{x^3_k\} \subset B(y_3, \frac 13) \subset B(y_2, \frac 12)$, и так далее
        \end{itemize}

        Рассмотрим <<диагональную>> последовательность $\{x_k^k\} \subset \{x_n\}$. По построению, она является фундаментальной, и в силу полноты пространства $X$ она сходится.
        
        \item\imp{3}{1}Проверим сначала, что $X$ вполне ограниченно. Предположим противное, то есть существует $\epsilon_0 > 0$ такое, что никакой конечный набор шаров радиуса $\epsilon_0$ не покрывает $X$. Тогда можно выбрать точку $x_1 \in X$, затем точку $x_2 \in \big(X \bs B(x_1, \epsilon_0)\big)$, точку $x_3 \in \big(X \bs (B(x_1, \epsilon_0) \cup B(x_2, \epsilon_0))\big)$, и так далее. Процесс не закончится, и будет получена последовательность $\{x_n\}$ с попарными расстояниями между точками не меньше $\epsilon_0$, из которой нельзя выделить сходящуюся подпоследовательность, --- противоречие.
        
        Теперь проверим, что $X$ компактно. Предположим противное, то есть существует открытое покрытие $\{G_\alpha\}_{\alpha \in \mf A}$, не имеющее конечного подпокрытия. Значит, для любого $\epsilon > 0$ найдется $x \in X$ такое, что шар $B(x, \epsilon)$ не покрывается конечным числом множеств из $\{G_\alpha\}$. Выбирая такую точку $x_n$ для $\epsilon := \frac 1n$ при каждом $n \in \N$, получим последовательность $\{x_n\}$, из которой можно выделить сходящуюся подпоследовательность $\{x_{n_k}\}$. Пусть $x_{n_k} \to_X x_0 \in X$, тогда существует $\alpha_0 \in \mf A$ такое, что $x_0 \in G_{\alpha_0}$. Но множество $G_{\alpha_0}$ открыто, поэтому оно покрывает некоторую окрестность точки $x_0$, а значит и все шары $B(x_{n_k}, \frac1{n_k})$, начиная с некоторого номера, --- противоречие.
        
        \item\imp{3}{4}Зафиксируем бесконечное множество $M \subset X$, тогда, выбирая произвольным образом последовательность $\{x_n\} \subset M$ и выделяя из нее сходящуюся подпоследовательность, получим требуемое.
        
        \item\imp{4}{3}Зафиксируем последовательность $\{x_n\}$. Если множество ее значений конечно, то в ней можно выделить стационарную подпоследовательность. Если же множество ее значений бесконечно, то оно имеет предельную точку $x_0 \in X$, поэтому можно выбрать подпоследовательность $\{x_{n_k}\}$ такую, что $x_{n_k} \to_X x_0$.\qedhere
    \end{itemize}
\end{proof}

\begin{note}
    Условию компактности пространства $X$ также эквивалентно такое условие: любая непрерывная функция $f : X \to \R$ является ограниченной на $X$, то есть выполнено включение $C(X, \R) \subset B(X, \R)$.
\end{note}

\begin{theorem}[Кантора]\label{thm3.3}
    Пусть $X$ "--- компактное метрическое пространство, и функция $f: X \to \R$ непрерывна на $X$. Тогда $f$ равномерно непрерывна на $X$.
\end{theorem}

\begin{proof}[Прямое доказательство]
    По условию, $f$ непрерывна на $X$, то есть выполнено следующее:
    \[\forall x \in X: \forall \epsilon > 0: \exists \delta_{x, \epsilon} > 0: \forall y \in B(x, \delta): |f(x) - f(y)| < \epsilon\]

    Для фиксированного $\epsilon$ выполнено $X = \bigcup_{x \in X}B(x, \delta_{x, \epsilon})$, поэтому можно выбрать конечный набор $x_1, \dotsc, x_n \in X$ такой, что $X = \bigcup_{k = 1}^nB(x_n, \delta_{x_n, \epsilon})$. Тогда для данного $\epsilon > 0$ в условии равномерной непрерывности подойдет $\delta := \min\{\delta_{x_1, \epsilon}, \dotsc, \delta_{x_n, \epsilon}\}$.
\end{proof}

\begin{proof}[Косвенное доказательство]
    Предположим противное, то есть выполнено следующее:
    \[\exists \epsilon_0 > 0: \forall \delta > 0: \exists x, y \in X, \rho(x, y) < \delta: |f(x) - f(y)| \ge \epsilon_0\]

    Выбирая $\delta := \frac 1n$ для каждого $n \in \N$, получим последовательности $\{x_n\}$ и $\{y_n\}$. Поскольку $X$ компактно, можно выделить из них сходящуюся подпоследовательности $\{x_{n_k}\}$ и $\{y_{n_k}\}$, причем сходятся они к одной и той же точке $x_0 \in X$ по построению, однако для любого $k \in \N$ выполнено $|f(x_{n_k}) - f(y_{n_k})| \ge \epsilon_0$ --- противоречие.
\end{proof}

\begin{definition}
    Пусть $X$ "--- метрическое пространство. Множество $M \subset X$ называется \textit{предкомпактом}, если множество $\overline M$ компактно как подпространство в $X$.
\end{definition}

\begin{note}
    Пусть $M \subset X$. Нетрудно проверить, что справедливы следующие утверждения:
    \begin{itemize}
        \item Если множество $M$ вполне ограниченно как подмножество в $X$, то оно также вполне ограниченно как пространство, то есть конечную $\epsilon$-сеть можно выбирать не в объемлющем пространстве $X$, а в самом пространстве $M$. Обратное, очевидно, тоже верно.
        
        \item Если замыкание $\overline M$ множества $M$ вполне ограниченно, то множество $M$ тоже вполне ограниченно.
        
        \item Если пространство $X$ полно, то условия полноты и замкнутости множества $M$ эквивалентны.
    \end{itemize}

    Из теоремы \ref{thm3.2} и утверждений выше следует, что в полном метрическом пространстве свойства предкомпактности и вполне ограниченности эквивалентны.
\end{note}

\begin{theorem}[Арцела---Асколи]
    Пусть $X$ "--- компактное метрическое пространство, $M \subset C(X, \R)$. Тогда множество $M$ вполне ограниченно $\lra$ множество $M$ ограниченно и выполнено условие равностепенной непрерывности:
    \[
        \forall \epsilon > 0: \exists \delta > 0: \forall x, y \in X, \rho(x, y) < \delta: \forall f \in M: |f(x) - f(y)| < \epsilon 
    \]
\end{theorem}

\begin{proof}[Доказательство для случая, когда {$X = [a, b]$}]~
    \begin{itemize}
        \item[$\ra$] Ограниченность множеств $M$ очевидна, проверим условие равностепенной непрерывности. Зафиксируем произвольное $\epsilon > 0$ и выберем конечный набор функций $\phi_1, \dotsc, \phi_n \in C(X, \R)$, образующий $\epsilon$-сеть. По теореме \ref{thm3.3}, каждая из этих функций равномерно непрерывна. Пусть $\delta_1, \dotsc, \delta_n > 0$ "--- числа, соответствующие выбранному $\epsilon$ в определении равномерной непрерывности. Тогда для $\delta := \min \{\delta_1, \dotsc, \delta_n\}$ выполнено требуемое.
        
        \item[$\la$] Поскольку множество $M$ ограниченно, то существует $C > 0$ такое, что для любой функции $f \in M$ выполнено условие $\sup_{x \in [a, b]}|f(x)| \le C$. Зафиксируем произвольное $\epsilon > 0$ и выберем по нему $\delta > 0$ из условия равностепенной непрерывности. Разобьем отрезок $[a, b]$ на части длины меньше $\delta$ точками $a = x_0 < x_1 < \dotsb < x_n = b$, отрезок $[-C, C]$ --- на части длины меньше $\epsilon$ точками $-C = y_0 < y_1 < \dotsb < y_m = C$, и рассмотрим конечное множество $L$ кусочно-линейных функций, построенных по всевозможным наборам точек вида $\{(x_0, y_{i_i}), \dotsc, (x_n, y_{i_n})\}$, $i_0, \dotsc, i_n \in \{0, \dotsc, m\}$.
        
        Для любой функции $f \in M$ можно выбрать ломаную $\phi \in L$ такую, что для всех $i \in \{0, \dotsc, n\}$ выполнено $|f(x_i) - \phi(x_i)| < \epsilon$. Рассмотрим произвольную точку $x \in [a, b]$ и выберем $i \in \{0, \dotsc, n - 1\}$ такое, что $x \in [x_i, x_{i + 1}]$, тогда:
        \[
            |f(x) - \phi(x)| \le |f(x) - f(x_i)| + |f(x_i) - \phi(x_i)| + |\phi(x) - \phi(x_i)| <  2\epsilon + |\phi(x_{i+1}) - \phi(x_i)|
        \]

        Оценим слагаемое $|\phi(x_{i+1}) - \phi(x_i)|$ отдельно:
        \[|\phi(x_{i+1}) - \phi(x_i)| \le |f(x_{i+1}) - \phi(x_{i+1})| + |f(x_{i+1}) - f(x_i)| + |f(x_i) - \phi(x_i)| < 3\epsilon\]

        Таким образом, $\sup_{x \in [a, b]}|f(x) - \phi(x)| < 5\epsilon$. Значит, построенное множество $L$ образует конечную $5\epsilon$-сеть для множества $M$, тогда, в силу произвольности выбора числа $\epsilon$, множество $M$ вполне ограниченно.\qedhere
    \end{itemize}
\end{proof}