\subsection{Делимость многочленов}

\begin{definition}
	Пусть $K$ "--- коммутативное кольцо, $a, b \in K$. Говорят, что \textit{$a$ делит $b$}, или \textit{$b$ делится на $a$}, если существует элемент $c \in K$ такой, что $ac = b$. Обозначение "--- $a\mid b$, или $b\divby a$.
\end{definition}

\begin{definition}
	Пусть $K$ "--- коммутативное кольцо, $a, b \in K$. Элемент $c \hm{\in} K$ называется \textit{наибольшим общим делителем} элементов $a$ и $b$, если выполнены следующие условия:
	\begin{enumerate}
		\item $c\mid a, b$
		\item Для любого $d \in K$ такого, что $d\mid a, b$, выполнено $d\mid c$
	\end{enumerate}
	
	Обозначение "--- $c = \nd(a, b)$.
\end{definition}

\begin{note}
	Наибольший общий делитель двух элементов произвольного коммутативного кольца не всегда существует и не всегда единственен.
\end{note}

\begin{definition}
	Пусть $K$ "--- коммутативное кольцо. Элементы $a, b \in K$ называются \textit{ассоциированными}, если существует $\alpha \in K^*$ такое, что $a = \alpha b$.
\end{definition}

\begin{note}
	Если $a$ и $b$ ассоциированы, то для любого элемента $c \in K$ выполнены равносильности $a\mid c \Leftrightarrow b\mid c$ и $c\mid a \Leftrightarrow c\mid b$.
\end{note}

\begin{example} Справедливы следующие утверждения о делимости:
	\begin{itemize}
		\item Если $K$ "--- коммутативное кольцо, $a \in K$ и $0\mid a$, то $a = 0$.
		\item Если $K$ "--- коммутативное кольцо, то $\forall a \in K: a\mid 0$.
		\item $2 \nmid 3$ в $\mathbb{Z}$, но $2\mid 3$ в $\mathbb{Q}$.
		\item Если $K$ "--- коммутативное кольцо и $a, b \in K$, то $\nd(a, b) \hm{=} a \Leftrightarrow a\mid b$.
	\end{itemize}
\end{example}

\begin{proposition}
	Пусть $K$ "--- целостное кольцо, $a, b \in K$. Тогда любые два наибольших общих элементов делителя $a$ и $b$ ассоциированы.
\end{proposition}

\begin{proof}
	Пусть $c = \nd(a, b)$ и $d = \nd(a, b)$. Тогда, по определению наибольшего общего делителя, $c\mid d$ и $d\mid c$, то есть $c = \alpha d$ и $d = \beta c$ для некоторых $\alpha, \beta \in K$. Следовательно, $\alpha \beta c = c$. Если $c \ne 0$, то $\alpha \beta = 1$ и $\alpha, \beta \in K^*$. Если же $c = 0$, то $a = b = c = d = 0$.
\end{proof}

\begin{note}
	Пусть $F$ "--- поле. Поскольку для любых $P, Q \in F[x]$ выполнено равенство $\deg{PQ} = \deg{P} + \deg{Q}$, то $F[x]^* = F^*$, то есть обратимы лишь многочлены, являющиеся ненулевыми скалярами. Значит, ассоциированные многочлены в $F[x]$ отличаются умножением на ненулевой скаляр.
\end{note}

\begin{theorem}
	Пусть $F$ "--- поле, $A, B \in F[x]$ и $B \ne 0$. Тогда существует единственная пара многочленов $Q, R \in F[x]$ такая, что $A = QB + R$ и $\deg{R} < \deg{B}$.
\end{theorem}

\begin{proof}~
	\begin{enumerate}
		\item Пусть $n := \deg{A}$, $k := \deg B$. Докажем существование индукцией по $n$. База, $n < k$, тривиальна: $A = 0B + A$. Теперь докажем переход, $n \ge k$. Перепишем $A$ в виде $A = ax^n + A'$, $B$ --- в виде $B = bx^k + B'$, где $A', B' \in F[x]$ "--- многочлены такие, что $\deg{A'} < n$, $\deg{B'} \hm{<} k$. Определим многочлен $C \in F[x]$ следующим образом:
		\[C := A - ab^{-1}x^{n - k}B = A' \hm{-} ab^{-1}x^{n - k}B'\]
		
		Поскольку $\deg{C} \le \deg{A'} < n$, то, по предположению, существуют многочлены $Q', R' \in F[x]$ такие, что $C = Q'B + R'$, тогда $A = (Q' + ab^{-1}x^{n - k})B + R'$.
		\item Покажем, что набор $(Q, R)$ единственен. Пусть $A = Q_1B \hm{+} R_1 \hm{=} Q_2B + R_2$ для некоторых двух наборов многочленов $(Q_1, R_1)$ и $(Q_2, R_2)$, тогда $(Q_1 - Q_2)B = R_1 - R_2$. Поскольку $\deg{(R_1 - R_2)} < \deg{B}$, то равенство может выполняться только в том случае, когда $Q_1 - Q_2 = 0$, откуда $Q_1 = Q_2$ и $R_1 = R_2$.\qedhere
	\end{enumerate}
\end{proof}

\begin{definition}
	Пусть $F$ "--- поле, $A, B \in F[x]$, $B \ne 0$, а многочлены $Q, R \in F[x]$ таковы, что $A = QB + R$ и $\deg{R} < \deg{B}$. Многочлен $Q$ называется \textit{неполным частным}, а $R$ "--- \textit{остатком} при делении $A$ на $B$.
\end{definition}

\begin{theorem}[алгоритм Евклида]
	Пусть $F$ "--- поле, $A, B \in F[x]$. Тогда существует многочлен $C \hm{=} \nd(A, B)$, причем для некоторых $\exists P, Q \in F[x]$ выполнено равенство $C = AP + BQ$.
\end{theorem}

\begin{proof}
	Проведем индукцию по величине $k \hm{:=} \min\{\deg{A}, \deg{B}\}$. База, ${k = -\infty}$, тривиальна: если без ограничения общности $B = 0$, то $\nd(A, 0) = A$ и $A = 1A + 0B$. Докажем переход. Пусть без ограничения общности $\deg{A} \ge \deg{B} \hm{=} k$. Выберем многочлены $Q, R \in F[x]$ такие, что $A = QB + R$ и $\deg{R} < k$, и заметим, что выполнена равносильность $D\mid A, B \Leftrightarrow D\mid B, R$. Тогда, по предположению индукции, существует многочлен $C = \nd(B, R) = \nd(A, B)$ и многочлены $P', Q' \in F[x]$ такие, что $C = P'B + Q'R$. Тогда $C \hm{=} P'B + Q'R \hm{=} Q'A \hm{+} (P' - QQ')B$.
\end{proof}

\begin{corollary}
	Пусть $F$ "--- поле, $A, B \in F[x]$, и имеют место представления $A = BQ + R$, $B = Q_1R + R_1$, \dots, $R_{k - 1} = Q_{k + 1}R_{k} + 0$. Тогда наибольший общий делитель многочленов $A$ и $B$ можно вычислить следующим образом:
	\[\nd(A, B) = \nd(B, R) = \nd(R, R_1) = \dots = \nd(R_k, 0) = R_k\]
\end{corollary}

\begin{definition}
	Пусть $F$ "--- поле, $P \in F[x]$. Многочлен $P$ называется \textit{неприводимым над $F$}, если $\deg{P} > 0$ и $P$ не раскладывается в произведение двух многочленов положительной степени.
\end{definition}

\begin{example}
	Многочлен $x^2 + 1$ неприводим над $\mathbb{R}$, но приводим над $\mathbb{C}$.
\end{example}

\begin{note}
	Пусть $F$ "--- поле, $P, Q \in F[x]$. Тогда выполнены следующие свойства:
	\begin{itemize}
		\item Если $P$ неприводим, то либо $\nd(P, Q) = 1$, либо $\nd(P, Q) = P$ с точностью до ассоциированности
		\item Если $P, Q$ неприводимы и $P\mid Q$, то $P$ и $Q$ ассоциированы
	\end{itemize}
\end{note}

\begin{proposition}
	Пусть $F$ "--- поле, $Q \in F[x]$, $\deg{Q} > 0$. Тогда $Q$ раскладывается в произведение неприводимых многочленов.
\end{proposition}

\begin{proof}
	Докажем утверждение индукцией по $\deg{Q}$. База тривиальна: если $\deg{Q} = 1$, то $Q$ уже неприводим. Докажем переход. Если $Q$ неприводим, то получено требуемое, иначе --- выполнено равенство $Q = Q_1Q_2$ для некоторых $Q_1, Q_2 \in F[x]$ таких, что $0 < \deg{Q_1}, \deg{Q_2} < \deg{Q}$, тогда $Q_1$ и $Q_2$ представляются в виде произведения неприводимых по предположению индукции.
\end{proof}

\begin{proposition}
	Пусть $F$ "--- поле, $P, Q, R \in F[x]$, многочлен $P$ неприводим и выполнено $P\mid QR$. Тогда $P\mid Q$ или $P\mid R$.
\end{proposition}

\begin{proof}
	Предположим, что $P\nmid Q$. Тогда, в силу неприводимости многочлена $P$, выполнено равенство $\nd(P, Q) = 1$, поэтому существуют многочлены $K, L \in F[x]$ такие, что $KP + LQ = 1$. Умножая обе части равенства на $R$, получим, что $KPR + LQR = R$, откуда $P \mid KPR + LQR = R$.
\end{proof}

\begin{note}
	Утверждение выше легко обобщить: если $P, Q_1, \dotsc, Q_n \in F[x]$, многочлен $P$ неприводим и выполнено $P\mid Q_1\dotsm Q_n$, то существует $i \in \{1, \dots, n\}$ такое, что $P\mid Q_i$. Для доказательства достаточно провести индукцию по $n$.
\end{note}

\begin{theorem}[основная теорема арифметики для многочленов]
	Пусть $F$ "--- поле, и ${Q \in F[x] \backslash \{0\}}$. Тогда существует такой скаляр $\alpha \in F^*$ и такие неприводимые многочлены $P_1, \dots, P_k \hm{\in} F[x]$, что $Q$ можно представить в следующем виде:
	\[Q = \alpha P_1\dots P_k\]
	
	Более того, если $Q = \alpha P_1\dots P_k = \beta R_1\dots R_l$ для некоторого скаляра $\beta \in F^*$ и неприводимых многочленов $R_1, \dots, R_l \hm{\in} F[x]$, то $k = l$ и существует перестановка $\sigma \in S_k$ такая, что для каждого $i \in \{1, \dots, k\}$ многочлены $P_i$ и $R_{\sigma(i)}$ ассоциированы.
\end{theorem}

\begin{proof}~
	\begin{itemize}
		\item(Существование) Случай, когда $\deg{Q} > 0$, уже был рассмотрен. Если же $\deg{Q} = 0$, то $Q = \alpha$.
		\item(Единственность) Проведем индукцию по $k$. База, $k = 0$, тривиальна: $\deg{Q} = 0$, откуда $k = l = 0$ и $Q = \alpha = \beta$. Теперь докажем переход. Пусть $k > 0$ и выполнены равенства $Q = \alpha P_1\dots P_k = \beta R_1\dots R_l$. Тогда, поскольку $P_k\mid R_1\dots R_l$, существует $i \in \{1, \dots, l\}$ такое, что $P_k\mid R_i$, то есть многочлены $P_k$ и $R_i$ ассоциированы в силу их неприводимости: $R_i = \gamma P_k$, $\gamma \in F^*$. Пусть без ограничения общности $i = l$, тогда $\alpha P_1\dots P_{k - 1} = (\beta\gamma)Q_1\dots Q_{l - 1}$, и применимо предположение индукции.\qedhere
	\end{itemize}
\end{proof}

\begin{corollary}
	Пусть $A, B \in F[x]$, $A = \alpha P_1\dots P_k$, $B = \beta Q_1\dots Q_l$ "--- разложения многочленов $A, B$ на неприводимые сомножители, и все многочлены $P_1, \dots, P_k, Q_1, \dots, Q_l \in F[x]$ попарно неассоциированы. Тогда $\nd(A,B) = 1$.
\end{corollary}

\begin{proof}
	Пусть это не так, тогда $\nd(A, B) = C$ для некоторого $C \in F[x]$ такого, что $\deg{C} > 0$. Но тогда существует неприводимый многочлен $P \in F[x]$ такой, что $P \mid C$, откуда $P \mid A, B$, поэтому существуют индексы $i \in \{1, \dotsc, k\}$ и $j \in \{1, \dotsc, l\}$ такие, что $P_i, Q_j$ ассоциированы с $P$, --- противоречие.
\end{proof}

\begin{corollary}
	Пусть $A = \alpha P_1\dots P_k$, $B = \beta Q_1\dots Q_l$ "--- разложения многочленов $A, B \in F[x]$ на неприводимые сомножители, и существуют индексы $i \in \{1, \dotsc, k\}$ и $j \in \{1, \dotsc, l\}$ такие, что $P_i$ ассоциирован с $Q_j$. Тогда выполнено следующее равенство:
	\[\nd(A, B) = \nd\left(\frac{A}{P_i}, \frac{B}{Q_j}\right)P_i\]
\end{corollary}

\begin{definition}
	Пусть $R$ "--- целостное кольцо. Элемент $p \in R \bs \{0\}$ называется \textit{простым}, если $p$ необратим и не раскладывается в произведение двух необратимых.
\end{definition}

\begin{definition}
	Целостное кольцо $R$ называется \textit{факториальным}, если любой его необратимый элемент раскладывается в произведение простых единственным образом с точностью до перестановки и ассоциированности.
\end{definition}

\begin{definition}
	\textit{Нормой} на целостном кольце $R$ называется функция $N: R \rightarrow \mathbb{N}\cup\{0\}$ такая, что выполнены следующие условия:
	\begin{enumerate}
		\item $N(a) = 0 \Leftrightarrow a = 0$
		\item $N(a + b) \le N(a) + N(b)$
		\item $N(ab) = N(a)N(b)$
	\end{enumerate}
\end{definition}
	
\begin{definition}
	Целостное кольцо $R$ называется \textit{евклидовым} относительно нормы $N$, если для любых элементов $a \in R$, $b \in K \bs \{0\}$, существуют элементы $\exists q, r \in R$ такие, что $a = qb + r$ и $N(r) < N(b)$.
\end{definition}
	
\begin{example}
	Рассмотрим несколько примеров евклидовых колец:
	\begin{itemize}
		\item Кольцо $\mathbb{Z}$ является евклидовым относительно нормы $N$ такой, что $N(a) := |a|$ для любого $a \in \mathbb{Z}$
		\item Если $F$ "--- поле, то $F[x]$ является евклидовым относительно нормы $N$ такой, что $N(P) := 2^{\deg{P}}$ для любого $P \in F[x]$
	\end{itemize}
\end{example}
	
\begin{note}
	Рассуждения, приведенные в данном разделе, позволяют аналогичным образом доказать, что любое евклидово кольцо является факториальным.
\end{note}