%02.03.23

\begin{theorem}[критерий Коши равномерной сходимости]
    Для равномерной сходимости $\{f_{n}\}$ на $E$ необходимо и достаточно выполнения условия Коши:
    \label{cauchy-convergence}
    \[\forall \epsilon > 0 \ \exists N \in \N \ \forall n, m \geq N \ \forall x \in E \left(\left|f_{n}(x) - f_{m}(x)\right| < \epsilon\right).\]
\end{theorem}

\begin{proof}~

    ($\Rightarrow$) Пусть $\epsilon > 0$. Из условия равномерной сходимости:
    \[\exists N \in \N \ \forall n \geq N \ \forall x \in E \left(\left|f_{n}(x) - f(x)\right| < \frac{\epsilon}{2}\right).\]
    Тогда для всех $n, m \geq N$ и $x \in E$ имеем:
    \[\left|f_{n}(x) - f_{m}(x)\right| \leq |f_{n}(x) - f(x)| + |f_{m}(x) - f(x)| < \frac{\epsilon}{2} + \frac{\epsilon}{2} = \epsilon.\]
    
    ($\Leftarrow$) Пусть $\{f_{n}\}$ удовлетворяет (\ref{cauchy-convergence}). Тогда для каждого $x \in E$ числовая последовательность $\{f_{n}(x)\}$ фундаментальна и, значит, сходится. Положим $f(x) = \lim_{n \to \infty} f_{n}(x)$, $x \in E$. Пусть $\epsilon > 0$ и номер $N$ из условия (\ref{cauchy-convergence}). Зафиксируем $n \geq N$ в неравенстве и перейдем к пределу при $m \to \infty$. Получим, что $|f_{n}(x) - f(x)| \leq \epsilon$ при всех $n \geq N$ и $x \in E$. Так как $\epsilon > 0$ -- любое, то $f_{n} \rightrightarrows f$ на $E$.
\end{proof}

\begin{corollary}
    Для равномерной сходимости $\sum_{n = 1}^{+\infty}f_{n}$ на $E$ необходимо и достаточно выполнения условия Коши:
    \[\forall \epsilon > 0 \ \exists N \in \N \ \forall n, m \geq N \ \forall x \in E \left(\left|\sum_{k = m}^{n}f_{k}(x)\right| < \epsilon\right).\]
\end{corollary}

\begin{corollary}
    Пусть $E \subset \R$, $f_{n}$ непрерывна на $\overline{E}$. Если $\{f_{n}\}$ равномерно сходится на $E$, то она равномерно сходится на $\overline{E}$.
\end{corollary}

\begin{proof}
    Зафиксируем $\epsilon > 0$. По критерию Коши:
    \[\exists N \ \forall n, m \geq N \ \forall x \in E \left(|f_{n}(x) - f_{m}(x)| < \epsilon\right).\]
    Пусть $x_{0} \in \overline{E}$. Тогда $\exists \{x_{n}\} \subset E (x_{k} \to x_{0})$.
    Пользуясь $|f_{n}(x_{k}) - f_{m}(x_{k})| < \epsilon$ и непрерывностью $f_{n}, f_{m}$, при $k \rightarrow \infty$ получим:
    \[|f_{n}(x_{0}) - f_{m}(x_{0})| \leq \epsilon.\]
    Таким образом, $\{f_{n}\}$ удовлетворяет условию (\ref{cauchy-convergence}) на $\overline{E}$. По критерию Коши $\{f_{n}\}$ равномерно сходится на $\overline{E}$.
\end{proof}

Равномерная сходимость позволяет "перебрасывать"\ некоторые свойства приближающих функций на приближаемую (предельную). Приведем соответствующие теоремы для непрерывности, дифференцируемости и интегрируемости.

\begin{theorem}[о непрерывности предельной функции]
    \label{limit-fun-cont}
    Пусть $E \subset \R$ и $f_{n} \rightrightarrows f$ на $E$. Если все $f_{n}$ непрерывны в точке $a \in E$ (на $E$), то функция $f$ также непрерывна в точке $a$ (на $E$).
\end{theorem}

\begin{proof}
    Пусть $\epsilon > 0$. Из условия равномерной сходимости:
    \[\exists N \, \forall n \ge N \, \forall x \in E \ \left(|f_n(x) - f(x)| < \frac{\epsilon}{3}\right).\]
    Тогда для $x \in E$:
    \[|f(x) - f(a)| \le |f(x) - f_N(x)| + |f_N(x) - f_N(a)| + |f_N(a) - f(a)| < |f_N(x) - f_N(a)| + \frac{2}{3}\epsilon.\]
    Так как $f_N$ непрерывна в точке $a$, то
    \[\exists \delta > 0 \, \forall x \in B_\delta(a) \cap E \ \left(|f_N(x) - f_N(a)| < \frac{\epsilon}{3}\right).\]
    Следовательно, $|f(x) - f(a)| < \epsilon$ для всех $x \in B_\delta(a) \cap E$.
\end{proof}

\begin{corollary}[о непрерывности суммы ряда]
    \label{ser-sum-continuity}
    Пусть $\sum_{n = 1}^{\infty}f_{n}$ равномерно сходится на $E \subset \R$ и все $f_{n}$ непрерывны в точке $a \in E$. Тогда сумма ряда непрерывна в точке $a$ (на $E$).
\end{corollary}

\begin{note}
    Если $a$ -- предельная точка на $E$, то в условиях теоремы (\ref{limit-fun-cont})
    \[\lim_{x \to a}\lim_{n \to \infty}f_{n}(x) = \lim_{n \to \infty}\lim_{x \to a}f_{n}(x).\]
\end{note}

\begin{theorem}[об интегрируемости предельной функции]
    \label{limit-int}
    Пусть $f_{n} \rightrightarrows f$ на $[a, b]$ и $f_{n} \in \mathcal{R}[a, b] \ \forall n$. Тогда $f \in \mathcal{R}[a, b]$ и $\lim_{n \to \infty}\int_{a}^{b} f_{n}(x) dx = \int_{a}^{b} f(x) dx$.
\end{theorem}

\begin{proof}
    Зафиксируем $\epsilon > 0$. Из условия равномерной сходимости:
    \[\exists N \ \forall n \geq N \ \forall x \in [a, b] \left(|f_{n}(x) - f(x)| < \frac{\epsilon}{4(b - a)}\right).\]
    Тогда на $[a, b]$:
    \[f_{N}(x) - \frac{\epsilon}{4(b - a)} < f(x) < f_{N}(x) + \frac{\epsilon}{4(b - a)}.\]
    Поскольку $f_{N}$ интегрируема, то она ограничена на $[a, b]$, значит на $[a, b]$ ограничена $f$.

    Пусть $T$ -- произвольное разбиение $[a, b]$. Тогда для верхних сумм Дарбу имеем:
    \[S_{T}(f) = S_{T}(f - f_{N} + f_{N}) \leq S_{T}(f - f_{N}) + S_{T}(f_{N}) \leq \frac{\epsilon}{4} + S_{T}(f_{N}).\]
    (так как $\sup_I \left(g(x) + h(x)\right) \le \sup_I g(x) + \sup_I h(x)$ при $I \subset [a, b]$)\\
    Аналогично для нижних сумм Дарбу $s_{T}(f) \geq s_{T}(f_{N}) - \frac{\epsilon}{4}$.
    Так как $f_{N} \in \mathcal{R}[a, b]$, то существует $T$ -- разбиение $\left(S_T(f_N) - s_T(f_N) < \frac{\epsilon}{2}\right)$, для такого $T$ имеем 
    \[S_T(f) - s_T(f) \le S_T(f_N) - s_T(f_N) < \frac{\epsilon}{2} + \frac{\epsilon}{2} = \epsilon.\]
    По критерию Дарбу $f \in \mathcal{R}[a, b]$.
    Для $n \ge N$ имеем
    \[
        \left|\int_a^b f_n(x)\, dx - \int_a^b f(x)\, dx\right| \le \int_a^b |f_n(x) - f(x)|\, dx \le (b - a) \cdot \frac{\epsilon}{(b - a)} < \epsilon.
    \]

    Следовательно, $\int_a^b f_n(x)\, dx \rightarrow \int_a^b f(x)\, dx$.
\end{proof}

\begin{corollary}[о почленном интегрировании ряда]
    Если ряд $\sum_{n = 1}^{\infty}f_{n}$ равномерно сходится на $[a, b]$, и все функции $f_{n} \in \mathcal{R}[a, b]$, то сумма ряда интегрируема на $[a, b]$ и 
    \[\int_{a}^{b}\left(\sum_{n = 1}^{\infty} f_{n}(x)\right)dx = \sum_{n = 1}^{\infty}\left(\int_{a}^{b} f_{n}(x)dx\right).\]
\end{corollary}

\begin{note}
    В условиях теоремы (\ref{limit-int})
    \[\lim_{n \rightarrow \infty} \int_a^b f_n(x)\, dx = \int_a^b \lim_{n \rightarrow \infty} f_n(x)\, dx.\]
\end{note}

\begin{problem}
    Верно ли, что утверждение теоремы (\ref{limit-int}) верно для несобственных интегралов?
\end{problem}

\begin{theorem}[о дифференцировании предельной функции]
    \label{covergence-3.4}
    Если
    \begin{enumerate}
        \item $\exists x_{0} \in [a, b] \hookrightarrow \{f_{n}(x_{0})\}$ -- сходится.
        \item $\forall n \in \N$ функция $f_{n}: [a, b] \to \R$ дифференцируема;
        \item $f_{n}' \rightrightarrows g$ на $[a, b]$.
    \end{enumerate}
    Тогда $f_{n} \rightrightarrows f$ на $[a, b]$ и $f$ дифференцируема на $[a, b]$, причем $f' = g$ на $[a, b]$.
\end{theorem}

\begin{proof}
    Докажем, что $\{f_{n}\}$ равномерно сходится на $[a, b]$. Применяя к разности $f_{n} - f_{m}$ теорему Лагранжа о среднем, получим $(f_{n}(x) - f_{m}(x)) - (f_{n}(x_{0}) - f_{m}(x_{0})) = (f_{n}'(c) - f_{m}'(c))(x - x_{0})$ для некоторой точки $c$ между $x$ и $x_{0}$. Тогда имеет место оценка

    \[|f_{n}(x) - f_{m}(x)| \leq |f_{n}(x_{0}) - f_{m}(x_{0})| + |f_{n}'(c) - f_{m}'(c)|\cdot (b - a).\]

    Так как последовательность $\{f_{n}'\}$ удовлетворяет условию Коши равномерной сходимости на $[a, b]$, ф $\{f_{n}(x_{0})\}$ фундаментальна, то и последовательность $f_{n}$ удовлетворяет условию Коши равномерной сходимости на $[a, b]$. По теореме (\ref{cauchy-convergence}) $\{f_{n}\}$ равномерно сходится на $[a, b]$ к некоторой функции $f$.

    Докажем дифференцируемость функции $f$. Зафиксируем $x$. Рассмотрим последовательность
    \[
    \phi_{n}(t) = \begin{cases}
        \frac{f_{n}(t) - f_{n}(x)}{t - x}, \ t \neq x; \\
        f_{n}'(x), \ t = x.
    \end{cases}
    \]
    Тогда $\phi_{n} \to \phi$ на $[a, b]$, где 
    \[\phi(t) = \begin{cases}
        \frac{f(t) - f(x)}{t - x}, \ t \neq x; \\
        g(x), \ t = x.
    \end{cases}\]
    Покажем, что сходимость равномерная. При $t \neq x$ по теореме Лагранжа:
    \[\phi_{n}(t) - \phi_{m}(t) = \frac{(f_{n}(t) - f_{m}(t)) - (f_{n}(x) - f_{m}(x))}{t - x} = f_{n}'(\xi) - f_{m}'(\xi)\]
    для некоторой точки $\xi$, лежащей между $t$ и $x$. Поскольку $\{f_{n}'\}$ равномерно сходится на $[a, b]$, то $\{f_{n}'\}$ удовлетворяет условию Коши равномерной сходимости, и, значит, $\{\phi_n\}$ удовлетворяет условию Коши.
    Следовательно, $\{\phi_n\}$ равномерно сходится на $[a, b]$.
    Поскольку $f_n$ дифференцируема в точке $x$, то $\phi_n$ непрерывна в точке $x$. По теореме (\ref{limit-fun-cont}) $\phi$ непрерывна в точке $x$. Тогда $\lim_{t \rightarrow x} \phi(t) = \phi(x)$, то есть $\exists f'(x) = g(x)$.
\end{proof}