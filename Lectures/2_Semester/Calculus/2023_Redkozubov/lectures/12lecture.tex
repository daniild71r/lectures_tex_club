%16.03.23

\begin{example}
    Найти радиус сходимости $\sum_{n=1}^{+\infty} \frac{n!}{n^n}x^{2n}$.
\end{example}

\begin{solution}
    Обозначим $n$-й член ряда как $u_{n}(x)$, тогда при $z \neq 0$ имеем
    \[\frac{|u_{n+1}(x)|}{|u_{n}(x)|} = \frac{(n + 1) n^{n}}{(n + 1)^{n + 1}}(x)^{2} = (1 - \frac{1}{n})x^{2} \to \frac{x^{2}}{e}.\]

    Если $\frac{x^{2}}{e} < 1 \lra |x| < \sqrt{e}$, то ряд сходится абсолютно по признаку Даламбера (\ref{dalambert-test}).
    
    Если $\frac{x^{2}}{e} > 1 \lra |x| > \sqrt{e}$, то ряд абсолютно расходится по признаку Даламбера (\ref{dalambert-test}).

    Значит, $R = \sqrt{e}$.
\end{solution}

\begin{exercise}
    Исследовать на сходимость при $x = \pm \sqrt{e}$.
\end{exercise}

\begin{theorem}[Абель]
    \label{abel-power-series}
    Если степенной ряд (\ref{power-series}) сходится в точке $x_{1} \neq x_{0}$, то он сходится равномерно на отрезке с концами $x_{1}, x_{0}$.
\end{theorem}

\begin{proof}
    По условию ряд $\sum_{n = 0}^\infty a_n (x - x_0)^n$ сходится. Рассмотрим последовательность $\{t^{n}\}:$ она монотонна при любом $t \in [0, 1]$ и равномерно ограниченна. По признаку Абеля (\ref{abel-func-series}) ряд $\sum_{n = 0}^\infty a_n (x - x_0)^n t^{n}$ равномерно сходится на $[0, 1]$. Сделав замену $t = \frac{x - x_{0}}{x_{1} - x_{0}}$, получим, что ряд (\ref{power-series}) равномерно сходится на $\{x: x = x_{0} + t(x_{1} - x_{0})\}$.
\end{proof}

\begin{note}
    Если $x_{1} \in B_{R}(x_{0})$, то предыдущая теорема вытекает из теоремы Коши--Адамара (\ref{cauchy-hadamard}), поэтому интерес представляет случай, когда $x_{1}$ лежит на границе круга сходимости.
\end{note}

Применяя следствие о непрерывности суммы ряда (\ref{ser-sum-continuity}), получаем:

\begin{corollary}
    Сумма степенного ряда является непрерывной функцией на множестве сходимости.
\end{corollary}

\begin{problem}
    Пусть $\sum_{n=0}^{+\infty}a_{n}$, $\sum_{n=0}^{+\infty}b_{n}$ сходятся к $A$ и $B$ соответственно, а ряд $\sum_{n = 0}^{+\infty}c_{n}$, где $c_{n} = \sum_{k = 0}^{n} a_{k}b_{n-k}$, сходится к $C$.

    Покажите, что $A\cdot B = C$.
\end{problem}

\begin{lemma}
    \label{lem1-power-series}
    Если ряд (\ref{power-series}) имеет радиус сходимости $R$, то ряд $\sum_{n = 1}^{+\infty}n a_{n}(x - x_{0})^{n - 1}$ также имеет радиус сходимости $R$.
\end{lemma}

\begin{proof}
    Так как $\lim_{n \to +\infty} \sqrt[n]{n} = 1$, то последовательности $\{\sqrt[n]{|a_{n}|}\}$ и $\{\sqrt[n]{n|a_{n}|}\}$ имеют одинаковое множество частичных пределов, значит $\overline{\lim}_{n \to +\infty} \sqrt[n]{|a_{n}|}$ и $\overline{\lim}_{n \to +\infty} \sqrt[n]{n |a_{n}|}$ равны. Тогда по формуле Коши--Адамара ряды $\sum_{n = 0}^{+\infty} a_{n}(x - x_{0})^{n}$ и $\sum_{n = 1}^{+\infty}n a_{n}(x - x_{0})^{n}$ имеют одинаковые радиусы сходимости.

    Ряды $\sum_{n = 1}^{+\infty}n a_{n}(x - x_{0})^{n - 1}$ и $\sum_{n = 1}^{+\infty}n a_{n}(x - x_{0})^{n}$ отличаются при $x \neq x_{0}$ ненулевым множителем (при $x = x_{0}$ оба сходятся). Следовательно, эти ряды сходятся одновременно. Тогда, радиусы сходимости этих рядов также совпадают.
\end{proof}

\begin{theorem}
    \label{th1-power-series}
    Если $f(x) = \sum_{n = 0}^{+\infty} a_{n}(x - x_{0})^{n}$ -- сумма степенного ряда с радиусом сходимости $R > 0$, то функция $f$ бесконечно дифференцируема в $B_{R}(x_{0})$, и для всякого $n \in \N$ выполнено:
    \[f^{(m)}(x) = \sum_{n = m}^{+\infty}n(n-1)\cdot\ldots\cdot(n - m + 1) a_{n}(x - x_{0})^{n - m}.\]
\end{theorem}

\begin{proof}
    По лемме (\ref{lem1-power-series}) при дифференцировании радиус сходимости ряда не меняется, поэтому нам достаточно доказать утверждение для $m = 1$, после чего применить индукцию. Без потери общности можно также считать, что $x_{0} = 0$.

    Пусть $t \in B_{R}(0)$. Покажем, что производная $f(x) = \sum_{n = 0}^{+\infty}a_{n}x^{n}$ в точке $t$ равна $l = \sum_{n = 1}^{+\infty} n a_{n} t^{n - 1}$.

    Зафиксируем такое $r$, что $|t| < r < R$. Для $x \neq t$ с $|x| < r$ составим разность

    \[\frac{f(x) - f(t)}{x - t} - l = \sum_{n = 1}^{+\infty} a_{n} \left(\frac{x^{n} - t^{n}}{x - t} - n t^{n - 1}\right) = \sum_{n = 1}^{+\infty} a_{n} \left(x^{n - 1} + tx^{n - 2} + \ldots + t^{n - 1} - n t^{n - 1}\right).\]

    Выражение в скобках перепишем в следующем виде:

    \[(x^{n - 1} - t^{n - 1}) + t(x^{n - 2} - t^{n - 2}) + \ldots + t^{n - 2}(x - t) =\]
    \[= (x - t)\left[(x^{n - 2} + t x^{n - 3} + \ldots + t^{n - 2}) + t(x^{n - 3} + t x^{n - 4} + \ldots + t^{n - 3}) + \ldots + t^{n - 2}\right]\]
    (в квадратных скобках в первой сумме $(n - 2)$ слагаемых, во второй -- $(n - 2)$ слагаемых, и т.д.).

    Поскольку $(n - 1) + (n - 2) + \ldots + 1 = \frac{n(n - 1)}{2}$, справедлива оценка
    \[|a_{n}||x^{n - 1} + t x^{n - 2} + \ldots + t^{n - 1} - n t^{n - 1}| \leq |x - t||a_{n}|\cdot \frac{n(n - 1)}{2} r^{n - 2}.\]

    Ряд $\sum_{n = 2}^{+\infty} |a_{n}|\frac{n(n - 1)}{2} r^{n - 2}$ сходится, т.к. $r < R$, и в круге сходимости дважды продифференированный ряд сходится абсолютно. Следовательно,
    \[\left|\frac{f(x) - f(t)}{x - t} - l\right| \leq |x - t| \sum_{n = 2}^{+\infty}|a_{n}|\cdot \frac{n(n - 1)}{2} r^{n - 2} \to 0, \ x \to t.\]

    Значит, существует $\lim_{x \to t}\frac{f(x) - f(t)}{x - t} = l$.
\end{proof}

\begin{corollary}[теорема о единственности]
    Если степенные ряды $\sum_{n = 0}^{+\infty} a_{n}(x - x_{0})^{n}$ и $\sum_{n = 0}^{+\infty} b_{n}(x - x_{0})^{n}$ сходятся в круге $B_{\delta}(x_{0})$, и их суммы там совпадают, то $a_{n} = b_{n}$, $n = 0, 1, 2, \ldots$
\end{corollary}

\begin{corollary}
    \label{cor2-power-series}
    Сумма степенного ряда с радиусом сходимости $R > 0$ имеет первообразную $F(x) = C + \sum_{n = 0}^{+\infty} \frac{a_{n}}{n + 1}(x - x_{0})^{n + 1}$ при $|x - x_{0}| < R$.
\end{corollary}

\subsection{Ряды Тейлора}

\begin{definition}
    Пусть функция $f$ определена в некоторой окрестности точки $x_{0}$ и в точке $x_{0}$ имеет производные любого порядка. Тогда $\sum_{n = 0}^{+\infty} \frac{f^{(n)}(x_{0})}{n!}(x - x_{0})^{n}$ называется \textit{рядом Тейлора} функции $f$ с центром в точке $x_{0}$. Для $x_{0} = 0$ ряд называют \textit{рядом Маклорена}.
\end{definition}

Покажем, что ряд Тейлора может сходиться к сумме, отличной от $f(x_{0})$.

\begin{example}
    Рассмотрим $f: \R \to \R$, 
    \[f(x) = \begin{cases}
        0, \ x \leq 0; \\
        e^{- \frac{1}{x}}, \ x > 0.
    \end{cases}\]
    
    Существование производных любого порядка в точке $x \neq 0$ следует из теоремы о дифференцировании композиции. Более того, $f^{(n)}(x) = 0$ при $x < 0$ и $f^{(n)}(x) = p_{n}(\frac{1}{x})e^{-\frac{1}{x}}$, где $p_{n}(t)$ -- многочлен степени $2n$. последнее утверждение можно установить по индукции: $p_{0}(t) = 1$ и дифференцирование $f^{(n)}$ дает соотношение $p_{n + 1}(t) = t^{2} [p_{n}(t) - p_{n}'(t)]$.

    Индукцией по $n$ покажем, что $f^{(n)}(0) = 0$. Для $n = 0$ это верно по условию. Если предположить, что $f^{(n)}(0) = 0$, то $(f^{(n)})'_{-}(0) = 0$ и
    \[(f^{(n)})'_{+}(0) = \lim_{h \to +0} \frac{f^{(n)}(h) - f^{(n)}(0)}{h} = \lim_{h \to +0}\frac{p_{n}(\frac{1}{h})e^{-\frac{1}{h}}}{h} = \lim_{t \to +\infty} \frac{t p_{n}(t)}{e^{t}} = 0,\]
    поскольку по правилу Лопиталя $\lim_{t \to +\infty} \frac{t^{m}}{e^{t}} = 0$ для всех $m \in \N_{0}$. Это доказывает, что $f^{(n + 1)}(0) = 0$.

    Таким образом, ряд Маклорена функции $f$ нулевой, но он не сходится к $f$ ни в какой окрестности нуля.
\end{example}

\begin{problem}
    Покажите, что сумма $f(x) = \sum_{n = 0}^{+\infty} \frac{\cos(n^{2}x)}{2^{n}}$ бесконечно дифференцируема на $\R$, однако её ряд Маклорена имеет нулевой ряд сходимости.
\end{problem}