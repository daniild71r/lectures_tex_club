\begin{definition} 
    Отображение $F \colon G \to E_2$, где $G$ --- открытое множество в $E_1$, называется \textit{дифференцируемым в точке} $x \in G$, если существует линейный ограниченный оператор $L_x \colon E_1 \to E_2$ такой, что 
    \[
        \forall \eps > 0\ \exists \delta > 0 \such \forall h \in E_1,\ \|h\|_{E_1} < \delta \quad \|F(x + h) - F(x) - L_xh\|_{E_2} \leq \eps \|h\|_{E_1}
    \]
    Это эквивалентно равенству
    \[
        \|F(x+h)-F(x)-L_x h\|_{E_2}=o(\|h\|_{E_1}), h \to 0
    \]
    Выражение $L_xh$ называется \textit{сильным дифференциалом (дифференциалом Фреше) в точке $x$ по приращению $h$}, а оператор $L_x$ называется \textit{сильной производной (производной Фреше) в точке $x$}.
\end{definition}

\begin{proposition}
    Если отображение $F \colon G \to E_2$, где $G$ --- открытое множество в $E_1$, дифференцируемо в $x$, то оно непрерывно в $x$.
\end{proposition}
\begin{proof}
 	Для доказательства нужно проверить стремление нормы разности значений к нулю:
    \[
        \|F(x + h) - F(x)\|_{E_2} \xrightarrow[h \to 0]{} 0
    \]
    Воспользуемся неравенством треугольника:
    \[  
        \|F(x + h) - F(x)\|_{E_2} = \|F(x + h) - F(x) - L_xh\|_{E_2} + \|L_xh\|_{E_2}
    \]
    При этом верно 2 факта:
    \[
        \|F(x + h) - F(x) - L_xh\|_{E_2} = o(\|h\|_{E_1}) \to 0 \wedge \|L_xh\|_{E_2} \leq \|L_x\| \cdot \|h\|_{E_1} \to 0
    \]
\end{proof}

\begin{proposition}
    Производная Фреше в точке $x \in G$ отображения $F \colon G \to E_2$, где $G$ --- открытое множество в $E_1$, единственна.
\end{proposition}

\begin{proof}
    Пусть $L_x, R_x$ --- две производные Фреше отображения $F$ в точке $x$. Оценим разность между значениями на одинаковом приращении:
    \begin{multline*}
        \|L_xh - R_xh\|_{E_2} \leq \|F(x + h) - F(x) - R_xh\|_{E_2} +
        \\
        \|F(x + h) - F(x) - L_xh\|_{E_2} = o(\|h\|_{E_1}),\ \ h \to 0
    \end{multline*}
    Следовательно, имеет место предел:
    \[
        \frac{\|(L_x - R_x)h\|_{E_2}}{\|h\|_{E_1}} \xrightarrow[h \to 0]{} 0
    \]
    Распишем его по определению:
    \[
        \forall \eps > 0\ \exists \delta > 0 \such \forall h \in E_1,\ 0 < \|h\|_{E_1} < \delta \quad \frac{\|(L_x - R_x)h\|_{E_2}}{\|h\|_{E_1}} < \varepsilon
    \]
    Пусть в рамках этого предела рассматривается подходящее $h$. Тогда положим за единичный вектор $e$ отнормированный $h$:
    \[
    	e := \frac{h}{\|h\|_{E_1}},\ \|e\|_{E_1} = 1 \wedge \|(L_x - R_x)e\|_{E_2} = \frac{\|(L_x - R_x)h\|}{\|h\|_{E_1}} < \eps
    \]
    Отсюда автоматически следует, что норма разности этих операторов равна нулю, коль скоро $\eps \to 0$ и неравентсво всегда верно. В силу свойств нормы, получаем $L_x = R_x$.
\end{proof}

\begin{example}
    Пусть $E_1 = \R^n, E_2 = \R$ и $f \colon E_1 \to E_2$ --- функция многих переменных. В старом смысле $f$ была дифференцируема в точке $\vv{x}$ тогда и только тогда, когда выполнялось равенство
    \[
    	f(\vv{x} + \vv{h}) - f(\vv{x}) = (\grad f(\vv{x}), \vv{h}) + o(|\vv{h}|),\ \vv{h} \to \vv{0}
    \]
    Теперь же отдельно рассмотрим оператор $L_{\vv{x}}\vv{h} := (\grad f(\vv{x}), \vv{h})$ для нового понятия дифференцирования. Его линейность очевидна, а ограниченность следует из неравенства Коши-Буняковского-Шварца:
    \[
    	|L_{\vv{x}}\vv{h}| \le |\grad f(\vv{x})| \cdot |\vv{h}|
    \]
    Отметим, что конкретное значение этого оператора даёт \textit{дифференциал в точке $\vv{x}$ по приращению $\vv{h}$}:
    \[
    	df(\vv{x}, \vv{h}) = L_{\vv{x}}\vv{h},\ \vv{h} = (dx_1, \ldots, dx_n)^T
    \]
    Если попытаться выписать отдельно $\vv{L}_{\vv{x}}$, то этот линейный оператор состоит в скалярном умножении аргумента на вектор градиента и этот оператор называется \textit{производной в точке $\vv{x}$}:
    \[
         L_{\vv{x}} = (\grad f(\vv{x}), \cdot) = f'(\vv{x})
    \]
    Выше мы предположили что функция $f$ дифференцируема как функция многих переменных и получили что она дифференцируема как отображение одного линейного пространства в другое. Обратное тоже верно, потому что из равенства
    \[
        |f(\vv{x} + \vv{h}) - f(\vv{x}) - L_{\vv{x}}\vv{h}|  = o(|\vv{h}|),\ \vv{h} \to \vv{0}
    \]
     мы получаем, что есть какой-то вектор (мы называем его $\grad f(\vv{x})$) такой, что $L_{\vv{x}} \vv{h} \colon = (\grad f(\vv{x}), \vv{h})$.
     Следовательно, функция многих переменных дифференцируема в старом смысле тогда и только тогда, когда она дифференцируема в новом.
\end{example}

\textcolor{red}{Возможно, в примере выше всё работает не для $\R$, а произвольного ЛНП $R$.}

\begin{theorem} (Свойства сильной производной)
	Если $F \colon G \to E_2$, где $G$ --- открытое множество в $E_1$, --- линейный ограниченный оператор, то имеют место такие свойства:
	\begin{enumerate}
	    \item Если $F$ --- постоянное отображение, то его производная Фреше $=0$.

	    \item Если $F = A$ --- линейный ограниченный оператор из $E_1$ в $E_2$, то $A'=A$.
	    
	    \item (Производная композиции) Если $F$ дифференцируемо в $x \in \Omega$, $F(\Omega)$ - открытое, $G \colon F(\Omega) \to E_3$ - дифференцируема в $F(x)$, то $G \circ F$ дифференцируема в $x$, причём $(G \circ F)^\prime (x) = G^\prime(F^\prime (x)) \circ F^\prime (x)$.

	    \item (Линейность производной как функции от отображения) Если $F_1, F_2$ дифференцируемы в $x, F_1+F_2$ и $\mathcal{L} F_1$, где $\forall \mathcal{L} \in R$, дифференцируемы в $x$, причем 
	    \[
	        (F_1+F_2)^\prime (x) = F^\prime_1 (x)+F^\prime_2 (x)\\
	        (\mathcal{L} F_1)^\prime (x) = \mathcal{L} F^\prime_1(x)
	    \]
	    \begin{proof}
	        По определению дифференцируемости$\colon$ 
	        \[
	            \|(F_1+F_2)(x+h) - (F_1+F_2)(x) - (F_1^\prime(x) h + F^\prime_2(x) h \|_{E_2} \leq
	        \]
	        \[
	            \leq \|F_1 (x+h) - F_1 (x) - F^\prime_1(x)h\|_{E_2} + \|F_2(x+h) - F_2(x) - F^\prime_2(x)h\|_{E_2} =
	        \]
	        \[
	            = 2\cdot o(\|h\|_{E_1}), h \to 0
	        \]
	        Для произведения на число доказательство аналогичное.
	    \end{proof}
	\end{enumerate}
\end{theorem}

\begin{proof}~
	\begin{enumerate}
		\item Так как отображение $F$ - постоянное, то
		\[
			\|F(x + h) - F(x) - 0h\|_{E_2} = 0 = o(\|h\|_{E_1}),\ h \to 0
		\]
		
		\item Так как $A$ - линейный ограниченный оператор, то
		\[
			\|A(x + h) - A(x) - Ah\|_{E_2} = 0 = o(\|h\|_{E_1}),\ h \to 0
		\]
		
		\item Определим 2 функции:
		\begin{align*}
			&{\Phi(h) := F(x + h) - F(x) - F'(x)h}
			\\
			&{y = F(x);\ B(\eta) := G(y + \eta) - G(y) - G'(y)\eta}
		\end{align*}
		Дифференцируемость обеих функций означает, что $\Phi(h) = o(\|h\|_{E_1}),\ h \to 0$ и $B(\eta) = o(\|\eta\|_{E_2}),\ \eta \to 0$. То, что нам надо доказать, можно переформулировать таким образом (где $y = F(x)$):
		\[
			\|G \circ F(x + h) - G \circ F(x) - G'(y)F'(x)h\|_{E_3} = o(\|h\|_{E_1}),\ h \to 0
		\]
		Разумно рассмотрим $\eta = F(x + h) - F(x) = F'(x)h + \Phi(h)$. Тогда то,
		\textcolor{red}{Надо переписать}
	\end{enumerate}
\end{proof}

\begin{definition}
    Если существует $\lim_{t \to 0} \frac{F(x + th) - F(x)}{t}$ в пространстве $E_2$, то он называется \textit{слабым дифференциалом (дифференциалом Гато)} $\D F(x, h)$.
    
    Если $\D F(x, h)$ является линейным ограниченным оператором (как функция от $h$, то есть $\D F(x, h) = F^\prime_c(x) h$, то $F^\prime_c(x)$ называется \textit{слабой производной (производной Гато)} отображения $F$ в точке $x$.
\end{definition}

\begin{example}
	Рассмотрим функцию двух переменных $f$, заданную следующим образом:
	\[
		f(x, y) = \System{
			&{\frac{x^3 y}{x^4 + y^2},\ x^2 + y^2 \neq 0}
			\\
			&{0,\ x^2 + y^2 = 0}
		}
	\]
    Докажем, что данная функция имеет слабую производную:
    \[
        \liml_{t \to 0} \frac{f(t h_1, t h_2) - f(0, 0)}{t} = \liml_{t \to 0} \frac{t^4 h_1^3 h_2}{(t^4 h_1^4+t^2 h_2^2)t}=\liml_{t \to 0} \frac{t h_1^3 h_2}{h_2^2+h_1^4 t^2} = 0
    \] 
    Это верно для любых ненулевых $h_1$ и $h_2$. При этом
    \[
    	\D f((0, 0)^T, (h_1, h_2)^T) = 0 = o(h_1, h_2)^T \text{ --- тоже нулевой оператор}
    \]
    Значит, слабая производная существует и является нулевым оператором. Однако, сильной производной нет: рассмотрим вектор $(h, h^2)^T,\ |(h, h^2)^T| = \sqrt{h^2+h^4}=|h|\sqrt{1+h^2}$. Посмотрми на модуль из условия Фреше:
    \[
        |f(h, h^2) - f(0, 0) - 0(h, h^2)| = \mo{\frac{h^5}{2 h^4}} = \mo{\frac{h}{2}} \neq o(|h|\sqrt{1+h^2}),\ h \to 0
    \]
\end{example}

\begin{proposition}
    Если отображение $F \colon G \to E_2$, где $G$ --- открытое множество в $E_2$, дифференцируемо в точке $x \in G$, то оно имеет в $x$ слабую производную, причем они совпадают:
    \[
    	F'(x) = F'_c(x)
    \]
\end{proposition}
\begin{proof}
    Пусть $\Phi_x(h)$ является функцией разности приращения и дифференциала:
    \[
    	\Phi_x(h) = F(x + h) - F(x) - F'(x)h,\ \|\Phi_x(h)\|_{E_2} = o(\|h\|_{E_1}),\ h \to 0
    \]
    Теперь посчитаем предел Гато:
    \[
        \lim_{t \to 0} \frac{F(x + th) - F(x)}{t} = \liml_{t \to 0} \frac{\Phi(th) + F'(x)(th)}{t} = \lim_{t \to 0} \frac{\Phi_x(th)}{t} + F'(x)h
    \]
    Если посмотреть на предел нормы $\Phi_x(th) / t$ при $t \to 0$, то это будет ноль:
    \[
    	\|\Phi_x(th)\|_{E_2} = o(\|th\|_{E_1}),\ t \to 0 \Ra \frac{\|\Phi_x(th)\|_{E_2}}{t} = o(\|h\|_{E_1}),\ t \to 0
    \]
    Получили, что слабая производная совпадает с сильной. Обратное, согласно примеру выше, неверно.
\end{proof}

\begin{definition}
	Пусть $F \colon G \to E_2$, где $G$ --- открытое множество в $E_1$. Если отображение $F' \colon G \to \Lin(E_1, E_2)$ дифференцируемо в точке $x$, то $F$ дважды дифференцируема в точке  $x$, причем $(F')'$ называется \textit{(сильной) второй производной отображения} $F$. Обозначается как $F''_x$.
	
	Причём $F''_x$ --- линейный ограниченный оператор, то есть $F''_x \in \Lin(E_1, \Lin(E_1, E_2))$.
\end{definition}


