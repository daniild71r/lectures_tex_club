\subsection{Основные свойства интеграла Римана-Стилтьеса}

\begin{note}
	До конца параграфа зафиксируем возрастающую функцию $\alpha$, а также то, что мы рассматриваем интеграл на отрезке $[a; b]$. Напомним смежные определения:
	\[
		U(P, f, \alpha) = \suml_{k = 1}^nM_k\Delta \alpha_k
	\]
	\[
		L(P, f, \alpha) = \suml_{k = 1}^nm_k\Delta \alpha_k
	\]
	\[
		\overline{I}(f, \alpha) = \inf_P U(P, f, \alpha)
	\]
	\[
		\underline{I}(f, \alpha) = \sup_P L(P, f, \alpha)
	\]
	\[
		S(P, f, \{t_i\}, \alpha) = \suml_{i= 1}^nf(t_i)\Delta\alpha_i
	\]
	\[
		\liml_{\Delta P \to 0}S(P, f, \{t_i\}, \alpha) = \int_a^bf(x)d\alpha(x)
	\]
	Если $\alpha$ - функция ограниченной вариации, то ее можно представить в виде разности 2 неубывающих функций. $\alpha = p - q$, откуда получаем:
	\[
		\int_a^bf(x)d\alpha(x) = \int_a^bf(x)dp(x) - \int_a^bf(x)dq(x)
	\]
	Также отметим, что $f$ интегрируема по Риману-Стилтьесу, когда $f$ - непрерывна, а $\alpha$ - функция ограниченной вариации, либо $f$ - функция ограниченной вариации, а $\alpha$ - непрерывная функция ограниченной вариации. 
\end{note}

\begin{theorem}
	Пусть $f, \alpha' \in R[a; b]$. Тогда $f \in R(\alpha, [a; b])$ и при этом
	\[
		\int_a^b f(x) d(\alpha(x)) = \int_a^b f(x) \alpha'(x) dx
	\]
\end{theorem}

\begin{proof}
	Коль скоро $\alpha'$ интегрируема по Риману, то $\alpha'$ - ограниченная, а стало быть $\alpha$ - непрерывная функция ограниченной вариации, потому интеграл существует. Распишем, что означает интегральная сумма справа от равенства:
	\[
		\forall \eps > 0\ \exists \delta_1 > 0 \such \forall P, \Delta(P) < \delta_1\ \ \forall \{t_i\}\ t_i \in [x_{i - 1};x_i]\  \left|\suml_{i = 1}^n f(t_i) \alpha'(t_i) \Delta x_i - \int_a^b f(x) \alpha'(x) dx\right| < \frac{\eps}{4}
	\]
	Так как $f$ интегрируема, то она ограничена, $\exists M \such \forall x \in [a; b]\ \ |f(x)| \le M$. Запишем интеграл Римана от $\alpha'(x)$ по определению:
	\[
		\forall \eps > 0\ \exists \delta_2 > 0 \such \forall P, \Delta(P) < \delta_2\ \ \forall \{t_i\}\ \left|\suml_{i = 1}^n \alpha'(t_i)\Delta x_i - \int_a^b \alpha'(x)dx\right| < \frac{\eps}{4M}
	\]
	Выпишем аналогичную формулу для иного набора точек $\{s_i\}$, после чего сложим строки и получим:
	\[
		\forall \{t_i\}, \{s_i\} \in [x_{i - 1};x_i] \quad \suml_{i = 1}^n|\alpha'(t_i) - \alpha'(s_i)| < \frac{\eps}{2M}
	\]
	Теперь, распишем для <<хороших>> разбиений $P$ (таких, что $\Delta P < \delta = \min \{\delta_1, \delta_2\}$) их интегральную сумму:
	\[
		\forall P, \Delta P < \delta\ \forall \{t_i\} \in [x_{i - 1};x_i] \quad S(P, f, \{t_i\}, \alpha) = \suml_{i = 1}^nf(t_i)(\alpha(x_i) - \alpha(x_{i - 1}))
	\]
	По теореме Лагранжа найдем:
	\[
		\exists \{s_i\} \in [x_{i - 1} ;x_i] \quad \suml_{i = 1}^n f(t_i)(\alpha(x_i) - \alpha(x_{i - 1})) = \suml_{i = 1}^n f(t_i)\alpha'(s_i)\Delta x_i
	\]
	Добавим и вычтем $\suml_{i = 1}^n f(t_i)\alpha'(t_i)\Delta x_i$:
	\[
		S(P, f, \{t_i\}, \alpha) = \suml_{i = 1}^nf(t_i)\alpha'(t_i)\Delta x_i + \suml_{i = 1}^nf(t_i)(\alpha'(s_i) - \alpha'(t_i))\Delta x_i
	\]
	Оценим теперь модуль разности между интегралом и нашей предельной суммой:
	\begin{multline*}
		\left|S(P, f, \{t_i\}, \alpha) - \int_a^b f(x)\alpha'(x)dx \right| \leq \left|\suml_{i = 1}^n f(t_i)\alpha'(t_i)\Delta x_i - \int_a^b f(x)\alpha'(x)dx \right| +
		\\
		\suml_{i = 1}^n |f(t_i)(\alpha'(s_i) - \alpha'(t_i))|\Delta x_i < \frac{\eps}{2} + M\frac{\eps}{2M} = \eps
	\end{multline*}
\end{proof}

\begin{theorem}
	Пусть $f$ ограничена на $[a; b]$ и непрерывна в точках $\{x_1, \ldots, x_N\} \subset (a; b)$. Функция $\alpha$ постоянна между этими точками, а значит непрерывна на множестве $[a; b] \bs \{x_1, \ldots, x_N\}$. Тогда $f \in R(\alpha, [a; b])$ и
	\[
		\int_a^b f(x)d(\alpha(x)) = \suml_{k = 1}^N f(x_k) (\alpha(x_k + 0) - \alpha(x_k - 0))
	\]
\end{theorem}

\begin{proof}
	%  Сюда стоит запихнуть картинку с лекции
	Положим $P_\delta := \bigcup\limits_{i = 1}^N \{x_i \pm \delta\} \cup \{a, b\}$. Посмотрим на верхнюю сумму интеграла Римана-Стилтьеса по данному разбиению:
	
	\textcolor{red}{Тут написана жесть. Никаким образом не учитаны отрезки $[x_{k - 1} + \delta; x_k + \delta]$ и концевые. Они схлопываются в разности, но верхняя/нижняя суммы без них не корректны.}
	\[
		U(P_\delta, f, \alpha) = \suml_{k = 1}^N \sup\limits_{x \in [x_k - \delta; x_k + \delta]} f(x) (\alpha(x_k + 0) - \alpha(x_k - 0))
	\]
	Аналогично поступим с нижней суммой:
	\[
		L(P_\delta, f, \alpha) = \suml_{k = 1}^N \inf\limits_{x \in [x_k - \delta; x_k + \delta]} f(x) (\alpha(x_k + 0) - \alpha(x_k - 0))
	\]
	Посмотрим на разность этих сумм:
	\[
		U(P_\delta, f, \alpha) - L(P_\delta, f, \alpha) = \suml_{k = 1}^N (M_k(f, \delta) - m_k(f, \delta)) (\alpha(x_k + 0) - \alpha(x_k - 0))
	\]
	Так как $f$ непрерывна во всех точках $x_k$, то при стремлении $\delta \to 0$ супремумы и инфинумы тоже будут стремится к значению в этих точках $M_k(f, \delta) \to f(x_k),\ m_k(f, \delta) \to f(x_k)$. То есть их разность стремится к нулю, в свою очередь приращение $\alpha(x_k + 0) - \alpha(x_k - 0) \leq \alpha(b) - \alpha(a)$.
	Получаем, что $U(P_\delta, f, \alpha) - L(P_\delta, f, \alpha) < \eps$. Следовательно, функция $f$ и правда интегрируема по Риману-Стилтьеса.
\end{proof}

\begin{theorem}
	Пусть $f$ - непрерывна, $\alpha$ - ограниченной вариации (или же $f$ - ограниченной вариации и $\alpha$ - непрерывная функция ограниченной вариации). Тогда, если положить $v(x) := V(\alpha, [a; x])$, то
	\[
		\left|\int_a^b f(x) d(\alpha(x))\right| \le \int_a^b |f(x)| d(v(x))
	\]
\end{theorem}

\begin{proof}
	Оценим модуль произвольной интегральной суммы:
	\begin{multline*}
		|S(P, f, \{t_i\}, \alpha)| = \left|\suml_{i = 1}^n f(t_i) \Delta \alpha_i\right| \le \suml_{i = 1}^n |f(t_i)| |\alpha(x_i) - \alpha(x_{i - 1})| \le
		\\
		\suml_{i = 1}^n |f(t_i)| V(\alpha, [x_{i - 1}; x_i]) = S(P, |f|, \{t_i\}, v)
	\end{multline*}
	Так как интеграл - это предел сумм, то из данного неравенства уже следует требуемое.
\end{proof}

\begin{corollary}
	Если дополнительно известно, что $|f(x)| \le M$, то
	\[
		\left|\int_a^b f(x) d(\alpha(x))\right| \le M \cdot V(\alpha, [a; b])
	\]
\end{corollary}

\begin{theorem} (Формула интегрирования по частям для интеграла Римана-Стилтьеса)
	Если $f, \alpha$ - функции ограниченной вариации, $f$ - непрерывна на $[a; b]$, то
	\[
		\int_a^b f(x)d(\alpha(x)) = f(b)\alpha(b) - f(a)\alpha(a) - \int_a^b \alpha(x)d(f(x))
	\]
\end{theorem}

\begin{proof}
	Распишем интегральную сумму, используя перестановку Абеля:
	\[
		S(P, f, \{t_i\}, \alpha) = \suml_{i=  1}^nf(t_i)(\alpha(x_i) - \alpha(x_{i - 1})) = \alpha(b)f(t_n) - \alpha(a)f(t_1) - \suml_{i = 1}^{n - 1}\alpha(x_i)(f(t_{i + 1}) - f(t_i))
	\]
	Теперь рассмотрим разбиение $P'$, которое отличается от $P$ наличием точек ещё двух дополнительных $x_i$ у краёв и, как следствие, возможность выбрать $t_0 := a,\  t_{n + 1} := b$.
	Тогда получим, что:
	\begin{multline*}
		S(P', f, \{t_i\}, \alpha)  = \alpha(b)f(b) - \alpha(a)f(a) - \suml_{i = 0}^n \alpha(x_i)(f(t_{i + 1}) - f(t_i)) =
		\\
		\alpha(b)f(b) - \alpha(a)f(a) - S(P, \alpha, \{x_i\}, f)
	\end{multline*}
	При этом для обоих разбиений сум верно, что $\Delta P \to 0 \Lra \Delta P' \to 0$. Следовательно, требуемая формула верна.
\end{proof}

\begin{theorem} (Первая теорема о среднем для интеграла Римана-Стилтьеса)
	Если $f$ непрерывна на $[a; b]$, $\alpha$ - неубывающая функция на $[a; b]$, то
	\[
		\exists \xi \in [a; b] \such \int_a^b f(x) d(\alpha(x)) = f(\xi)(\alpha(b) - \alpha(a))
	\]
\end{theorem}

\begin{proof}
	Снова распишем сумму по разбиению:
	\[
		S(P, f, \{t_i\}, \alpha) = \suml_{i = 1}^n f(t_i) \Delta \alpha_i \le \left(\max\limits_{x \in [a; b]} f(x)\right) \cdot (\alpha(b) - \alpha(a)) = f(\xi) \cdot (\alpha(b) - \alpha(a))
	\]
\end{proof}

\begin{theorem} (Формула Бонн\'{е} для интеграла Римана-Стилтьеса)
	Пусть $f$ - монотонная функция ограниченной вариации на $[a; b]$, $\alpha$ - непрерывная на $[a; b]$. Тогда
	\[
		\exists \xi \in [a; b] \such \int_a^b f(x) d(\alpha(x)) = f(a)(\alpha(\xi) - \alpha(a)) + f(b)(\alpha(b) - \alpha(\xi))
	\]
\end{theorem}

\begin{proof}
	Применим формулу интегрирования по частям:
	\[
		\int_a^b f(x) d(\alpha(x)) = f(b)\alpha(b) - f(a)\alpha(a) - \int_a^b \alpha(x) d(f(x))
	\]
	Сделаем ещё один переход, используя первую теорему о среднем для оставшегося интеграла:
	\[
		f(b)\alpha(b) - f(a)\alpha(a) - \int_a^b \alpha(x) d(f(x)) = f(b)\alpha(b) - f(a)\alpha(a) - \alpha(\xi)(f(b) - f(a))
	\]
	Пересобрав слагаемые, получим требуемую формулу.
\end{proof}

\begin{theorem} (Формула замены переменной)
	Пусть $f, \phi$ - непрерывны на $[a; b]$, $\phi$ - строго монотонна, а $\psi = \phi^{-1}$. Тогда
	\[
		\int_a^b f(x)dx = \int_{\phi(a)}^{\phi(b)} f(\psi(t)) d\psi(t)
	\]
\end{theorem}

\begin{proof}
	Возьмём произвольное разбиение $P$ отрезка $[a; b]$. Будем считать, что $\phi$ возрастает, не умаляя общности. Тогда
	\[
		\phi(P) \colon \phi(a) < \phi(x_1) < \ldots < \phi(b)
	\]
	Обозначим $t_0 := \phi(a), t_n := \phi(b), t_i := \phi(x_i)$ и рассмотрим интегральную сумму:
	\[
		S(P, f, \{t_i\}) = \suml_{i = 1}^n f(t_i) \Delta x_i = \suml_{i = 1}^n f(\psi(\xi_i))(\psi(t_i) - \psi(t_{i - 1})) = S(\phi(P), f \circ \psi, \psi)
	\]
	Раз $f, \phi$ непрерывны, то по теореме об обратной функции $\psi$ также непрерывна. Более того, композиция $f \circ \psi$ тоже непрерывна, поэтому интеграл Римана-Стилтьеса существует и, в силу равенства интегральных сумм выше, интегралы тоже равны.
\end{proof}

\subsection{Несобственный интеграл Римана}

\begin{definition}
	Пусть $f$ интегрируема по Риману на $[a; \tilde{b}],\ \forall \tilde{b} \in [a; b)$. Тогда интеграл $\int_a^b f(x)dx$ называется \textit{несобственным интегралом Римана}
	\begin{itemize}
		\item \textit{первого рода}, если $b = +\infty$
		
		\item \textit{второго рода}, если $b < +\infty$ и $f$ не является ограниченной на $[a; b)$
	\end{itemize}
	Его понимают как $\liml_{\tilde{b} \to b-0} \int_a^{\tilde{b}} f(x)dx$ или $\liml_{\tilde{b} \to +\infty} \int_a^{\tilde{b}} f(x)dx$. Если этот предел существует и конечен, то несобственный интеграл \textit{сходится}. В противном случае, интеграл \textit{расходится}.
\end{definition}