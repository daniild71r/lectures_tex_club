\begin{theorem} (Критерий Коши сходимости несобственного интеграла Римана)
	Пусть $f \in R[a; \tilde{b}]\ \forall \tilde{b},\ a < \tilde{b} < b$. Тогда 
	\[
		\int_a^bf(x)dx \Longleftrightarrow \forall \eps > 0\ \exists B \in (a;b) \such \forall B_1, B_2,\ B < B_1 < B_2 < b\ \ \left | \int_{B_1}^{B_2}f(x)dx\right | < \eps
	\]
\end{theorem}

\begin{proof}
	$F(x) = \int_a^x f(t)dt,\ x \in [a;b)$.
	\[
		\int_a^bf(x)dx \text{ сходится } \Longleftrightarrow \exists \liml_{x \to b - 0} F(x) \in \R
	\]
	По критерию Коши для существования конечного предела:
	\[
		\forall \eps > 0\ \exists \delta > 0 \such \forall B_1, B_2,\ 0 < |B_1 - B_2| < \delta\ \ |F(B_1) - F(B_2)| < \eps
	\]
	Причем $F(B_2) - F(B_1) = \int_{B_1}^{B_2} f(x)dx$, а замена $\delta$ на $B$ не поменяет верности формулы.
\end{proof}

\begin{lemma}
	Пусть $f(x) \geq 0\ \forall x \in (a;b),\ f \in R[a; \tilde{b}]\ \forall \tilde{b},\ a < \tilde{b} < b$. Тогда
	\[
		\int_a^bf(x)dx \text{ сходится } \Longleftrightarrow F(x) = \int_a^xf(t)dt \text{ ограничена на } [a;b)		
	\]
\end{lemma}

\begin{proof}
	$f(x) \geq 0 \Ra F(x)$ монотонно неубывающая, а значит существует предел при $x \to b$. Он будет конечен лишь тогда, когда функция ограничена.
\end{proof}

\begin{theorem} (Признак сравнения)
	Пусть $f, g \in R[a; \tilde{b}],\ \forall \tilde{b},\ a < \tilde{b} < b$, а также $\forall x \in [a;b)\ 0 \leq f(x) \leq g(x)$. Тогда из сходимости $\int_a^bg(x)dx$ следует сходимость $\int_a^bf(x)dx$, а из расходимости $\int_a^bf(x)dx$ следует расходимость $\int_a^bg(x)dx$.
\end{theorem}

\begin{proof}
	$F(x) = \int_a^x f(t)dt,\ G(x) = \int_a^x g(t)dt$. Из свойств интеграла
	\[
		0 \leq F(x) \leq G(x)
	\]
	Тогда, применяя лемму выше к каждому из интегралов, утверждения теоремы становятся тривиальными.
\end{proof}

\begin{example}
	\[
		\int_0^1 \frac{dx}{x^\alpha} = \liml_{\eps \to +0} \int_\eps^1\frac{dx}{x^\alpha} 
	\]
	Интересует лишь $\alpha > 0$, ибо иначе это интеграл от полинома, который интегрируем на всех числовой плоскости. Посчитаем интеграл выражения под пределом:
	\[
		\int_\eps^1 \frac{dx}{x^\alpha} = \System{
			&{\frac{x^{-\alpha + 1}}{-\alpha + 1} \Big|_\eps^1 = \frac{1}{1 - \alpha} - \frac{\eps^{1 - \alpha}}{1 - \alpha},\ \alpha \neq 1}
			\\
			&{\ln x \Big|_\eps^1 = \ln (1 / \eps),\ \alpha = 1}
		}
	\]
	
	Получается, что интеграл сходится только при $\alpha < 1$.
\end{example}

\begin{example}
	\[
		\int_1^{+\infty} \frac{dx}{x^\alpha} = \liml_{b  \to +\infty} \int_1^b \frac{dx}{x^\alpha} 
	\]
	Снова распишем подпредельный интеграл:
	\[
		\int_1^b \frac{dx}{x^\alpha} = \System{
			&{\frac{x^{1 - \alpha}}{1 - \alpha} \Big|_1^b,\ \alpha \neq 1}
			\\
			&{\ln x \Big|_1^b,\ \alpha = 1}
		}
	\]
	Получается, что сходится при $\alpha > 1$.
\end{example}

\begin{example}
	\[
		\int_0^{+\infty}\frac{dx}{x^\alpha} = \int_0^1\frac{dx}{x_\alpha} + \int_1^{+\infty}\frac{dx}{x^\alpha}
	\]
	Расходится по прошлым 2 примерам.
\end{example}

\begin{anote}
	Рассуждение в последнем примере корректно, ибо так устроено свойство аддитивности. Если часть слева сходится, то части справа тоже обязаны и наоборот.
\end{anote}

\begin{example}
	\[
	\int_{-1}^1\frac{dx}{x} = \int_{-1}^0\frac{dx}{x} + \int_0^1\frac{dx}{x}
	\]
	Расходится, так как оба интеграла расходятся.
	Если ввести понятие \textit{главного значения интеграла (в смысле Коши)} v.p.(value principale)/p.v.(principle value), то будет иметь место следующая запись:
	\[
		v.p.\ \int_{-1}^1\frac{dx}{x} = \liml_{\eps \to 0+} \left ( \int_{-1}^{-\eps}\frac{dx}{x}  + \int_\eps^1\frac{dx}{x}\right ) = 0
	\]
	Еще одним таким же примером может служить такой интеграл:
	\[
			v.p.\ \int_{-\infty}^\infty\frac{dx}{x} = \liml_{\eps \to 0,\ b \to \infty} \left ( \int_{-b}^{-\eps}\frac{dx}{x}  + \int_\eps^b\frac{dx}{x}\right ) = 0
	\]
\end{example}

\begin{corollary}
	Если $f, g \in R[a;\tilde{b}],\ \forall \tilde{b},\ a < \tilde{b} < b$ и $f(x) \geq 0,\ g(x) \geq 0$, то
	\begin{multline*}
		\exists \liml_{x \to b - 0(+\infty)} \frac{f(x)}{g(x)} = C \in (0;+\infty) \Ra \int_a^b f(x)dx \text{ и } \int_a^b g(x)dx \text{ сходятся и расходятся} \\ \text{одновременно}
	\end{multline*}
\end{corollary}

\begin{proof}
	Если предел существует, то в некоторой окрестности $b$ верно, что
	\[
		0 < \frac{f(x)}{g(x)} \le C
	\]
	Применяя в этой окрестности признак сравнения, получим требуемое.
\end{proof}

\begin{definition}
	Пусть $f \in R[a; \tilde{b}]\ \forall \tilde{b}, a < \tilde{b} < b$. Если интеграл $\int_a^b |f(x)|dx$ сходится, то говорят, что $\int_a^b f(x)dx$ \textit{сходится абсолютно}.
\end{definition}

\begin{theorem}
	Если $\int_a^bf(x)dx$ сходится абсолютно, то он сходится.
\end{theorem}

\begin{proof}
	Так как интеграл сходится абсолютно, то по критерию Коши можно записать:
	\[
		\forall \eps > 0\ \exists B \in (a; b) \such \forall B_1, B_2, B < B_1 < B_2 < b \quad \left| \int_{B_1}^{B_2} |f(x)|dx\right| < \eps
	\]
	Заметим простое неравенство:
	\[
		\left| \int_{B_1}^{B_2} f(x)dx \right| \leq \left|\int_{B_1}^{B_2}|f(x)|dx\right| < \eps \Ra \int_a^b f(x)dx \text{ сходится}
	\]
\end{proof}

\begin{definition}
	Если $\int_a^b f(x)dx$ сходится, но не сходится абсолютно, то говорят, что он \textit{сходится относительно}.
\end{definition}

\begin{theorem}(Признак Дирихле)
	Если
	\begin{enumerate}
		\item $f \in R[a; \tilde{b}],\ \forall \tilde{b} \in (a; b)$ и $F(x) = \int_a^xf(t)dt$ ограничена на $[a;b)$.
		\item $g$ монотонна на $[a;b)$ и бесконечно мало при $x \to b - 0(+\infty)$
	\end{enumerate}
	Тогда $\int_a^b f(x)g(x)dx$ сходится.
\end{theorem}

\begin{proof}
	Рассмотрим $\forall a < B_1 < B_2 < b$. По формуле Бонне:
	\[
		\exists \xi \in (B_1; B_2) \such \int_{B_1}^{B_2} f(x)g(x)dx = g(B_1)\int_{B_1}^\xi f(x)dx + g(B_2)\int_\xi^{B_2} f(x)dx
	\]
	Из ограниченности $\exists M\ \forall x \in [a;b)\ |F(x)| \leq M$. 
	Так как $g$ бесконечно мала, то:
	\[
		\forall \eps > 0\ \exists B \in (a; b) \such \forall x \in (B; b) \quad |g(x)| < \frac{\eps}{4M}
	\]
	Теперь выберем любые $B_1$, $B_2$, для которых верно, что $B < B_1 < B_2 < b$. При этом справедливо, что:
	\[
		\left | \int_{B_1}^{B_2}f(x)g(x)dx\right | \leq |g(B_1)||F(\xi) - F(B_1)| + |g(B_2)||F(B_2) - F(\xi)| < \eps
	\]
\end{proof}

\begin{corollary} \textit{(Признак Абеля)}
	Если
	\begin{enumerate}
		\item $\int_a^b f(x)dx$ сходится.
		
		\item $g$ монотонна и ограничена на $[a;b)$.
	\end{enumerate}
	Тогда $\int_a^b f(x)g(x)dx$ сходится.
\end{corollary}

\begin{proof}
	Сходимость $\int_a^b f(x)dx$ влечет ограниченность $F(x) = \int_a^x f(t)dt$. Дополнительно определим функцию $g_1(x) = g(x) - L$, где $L = \liml_{x\to b - 0(+\infty)} g(x)$.
	Из этих двух утверждений по аддитивности Дирихле получаем:
	\[
		\int_a^b f(x)g_1(x)dx = \int_a^b f(x)g(x)dx - L\int_a^b f(x)dx
	\]
	Левая часть сходится по признаку Дирихле, второй член в правой части по положению теоремы.
\end{proof}

\begin{example}
	Рассмотрим сразу 2 интеграла (но изучать будем только первый):
	\[
		\int_1^{+\infty} \frac{\sin x}{x^\lambda},\  \int_1^{+\infty} \frac{\cos x}{x^\lambda}
	\]
	Заметим, что $|\frac{\sin x}{x^\lambda}| \leq \frac{1}{x^\lambda}$. Ранее было показано, что $\int_1^{+\infty} \frac{dx}{x^\lambda} \Longleftrightarrow \lambda > 1$ сходится абсолютно. Значит, это же верно и про исходные интегралы.
	$\\$
	Возьмем $f(x) = \sin x \Ra F(x) = -\cos x + C$ - ограниченная первообразная. $g(x) = \frac{1}{x^\lambda}$.
	Тогда при $\lambda > 0$ интегралы сходятся по признаку Дирихле.
	$\\$
	Установим, что абсолютной сходимости при $0 < \lambda < 1$ нет. Для этого оценим снова модуль:
	\[
			\left|\frac{\sin x}{x^\lambda}\right| \geq \frac{\sin^2 x}{x^\lambda} = \frac{1}{2x^\lambda} - \frac{\cos 2x}{2x^\lambda}
	\]
	Интеграл от первого слагаемого расходится при $\lambda \in (0; 1]$, а второй сходится по признаку Дирихле. Стало быть, интеграл от $\sin^2 x / x^\lambda$ тоже расходится, Откуда по признаку сравнения видим, что исходные интегралы сходятся условно при $0 < \lambda \le 1$.
	
	Покажем по критерию Коши, что интегралы расходятся при $\lambda \leq 0$. Положим $B_1 = \frac{\pi}{2} + 2\pi n$, $B_2 = \frac{3\pi}{4} + 2\pi n \Ra \forall x \in [B_1; B_2]\ \ \sin x \leq \frac{\sqrt {2}}{2}$. Следовательно
	\[
		\left| \int_{B_1}^{B_2} \frac{\sin x}{x^\lambda}dx\right| \geq \frac{\sqrt{2}}{2} \int_{B_1}^{B_2} \frac{dx}{x^\lambda} \geq \frac{\sqrt{2}}{2} B_1^{-\lambda}(B_2 - B_1) = \frac{\pi \sqrt{2}}{4}(\frac{\pi}{4})^{-\lambda}
	\]
\end{example}

\begin{theorem} (Признак Харди)
	Пусть $f$ - периодическая (с периодом $\omega$) и интегрируемая по Риману на любом отрезке $[a;b]$ функция. $g$ - монотонная, бесконечно малая при $x\to +\infty$. Тогда
	\begin{enumerate}
		\item если $\int_a^{a + \omega}f(x)dx = 0$, то $\int_a^{+\infty}f(x)g(x)dx$ сходится.
		\item если $\int_a^{a + \omega}f(x)dx \neq 0$, то $\int_a^{+\infty}f(x)g(x)dx$ и $\int_a^{+\infty}g(x)dx$ сходятся и расходятся одновременно.
	\end{enumerate}
\end{theorem}

\begin{proof}~
	\begin{enumerate}
		\item Пусть $F(x) = \int_a^x f(t)dt$ Тогда его можно записать следующим образом (при раскрытии скобок внутри мы просто ставим этот знак интеграла рядом с выражением справа): 
		\[
			F(x) = \int_a^x f(t)dt = \left|\left(\int_a^{a + \omega} + \int_{a + \omega}^{a + 2\omega} + \ldots + \int_{a + k\omega}^x\right) f(t)dt \right|
		\]
		где  $x \in [a + k\omega;a + (k + 1)\omega)$. 
		Так как все интегралы кроме правого равны 0, а также $F(x)$ ограничена на $x \in [a; a + w]$ (скажем, $\exists M \such |F(x)| \le M$), то верно следующее:
		\[
			\left|\int_{a + k\omega}^x f(t)dt\right| = \left|\int_a^{x - kw} f(t)dt \right| = |F(x - kw)| \le M
		\]
		В итоге у нас выполнены все условия признака Дирихле, и сходимость нужного интеграла доказана.
		
		\item Пусть $K = \int_a^{a + \omega} f(t)dt$, а $f_1(x) := f(x) - \frac{K}{\omega}$. По этим обозначениям $\int_a^{a + \omega}f_1(x)dx = 0$. то есть по первому пункту $\int_a^{+\infty}f_1(x)g(x)dx$ сходится. Однако, если подставить $f_1$, то:
		\[
			\int_a^{+\infty}f_1(x)g(x)dx = \int_a^{+\infty}f(x)g(x)dx - \frac{K}{\omega}\int_a^{+\infty}g(x)dx
		\]
		Так как левая часть обязательно сходится, то слагаемые в правой части обязаны сходится/расходится одновременно.
	\end{enumerate}
\end{proof}

%\begin{example}
%	\[
%		\int_0^{+\infty}\exp^{\cos x}\sin(\sin x)\frac{dx}{x}
%	\]
%\end{example}