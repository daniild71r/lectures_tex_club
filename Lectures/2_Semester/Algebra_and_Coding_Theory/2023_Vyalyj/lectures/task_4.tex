\section{№4 Гомоморфизм}

\begin{proposition}
  Гомоморфизм $G \ra H$:
    \[\forall x, y \in G \hra \phi (x \cdot y) = \phi (x) \cdot \phi (y)\]
\end{proposition}

\begin{example}
  ~
  \begin{enumerate}
    \item $\ZZ \ra \ZZ_n \quad [x]_n + [y]_n = [x + y]_n$
    \item $\ZZ \ra \ZZ_n^{*} \quad \gcdru(a, n) = 1, \quad \exp_a(x) = [a^x]_n$
    \item Возведение в степень. Пусть $A$ --- абелева группа, тогда:
    \begin{gather}
      p_k: A \ra A, \quad p_k: x \raone x^k \\
      p_k(xy) = (xy)^k = \underbrace{xyxyxy \dots xy}_k = x^k y^k = p_k(x) \cdot p_k(y)
    \end{gather}
  \end{enumerate}
\end{example}

\begin{definition}
  Ядро: $\ke(\phi) = \{ g \in G: \ \phi(g) = e_h \}$
\end{definition}

\begin{definition}
  Образ: $\im(\phi) = \{ h \in H : \ \exists g \in G \  \phi(g) = h\}$
\end{definition}

\begin{proposition}
  $\ke(\phi) < G$
\end{proposition}

\begin{proof}
  \begin{gather}
    e_G \in \ke(\phi) \\
    \phi(e_G) = e' \\
    \phi(e_G \cdot e_G) = \phi(e_G) \cdot \phi(e_g) \Ra e' = e_H \\
  \end{gather}
  \begin{itemize}
    \item $y, x \in \ke(\phi) \Ra \phi(x y) = \phi(x) \cdot \phi(y) = e_H \cdot e_H = e_H$
    \item $\phi(x^{-1}) = \phi(x)^{-1}$ \\
    $\phi(x^{-1}) \cdot \phi(x) = \phi(x \cdot x^{-1}) = \phi(e_G) = e_H$ \\
    $\phi(x) = e_H, \ \phi(x^{-1}) = \phi(x)^{-1} = e_H$
  \end{itemize}
\end{proof}

\begin{proposition}
  $\im(\phi) < H$
\end{proposition}

\begin{proof}
  \begin{gather}
    \phi(e_G) = e_H \\
    a = \phi(x), \ b = \phi(y) \\
    a = \phi(x) \hra a^{-1} = \phi(x^{-1}) \\
  \end{gather}
\end{proof}

\begin{example}
  \begin{gather}
    p_k : \ZZ_n \ra Z_n, \ p_k(x) = kx \\
    \ke (p_k) = \{ [x] : [kx] = [0]\} = \{ [x] : \ord([x]) \mid k \}
  \end{gather}
\end{example}

\begin{example}
  \begin{gather}
    \im (p_k) = \{ [y] : [y] = [kx], \ x \in \ZZ \} = \{ [id], \  0 \leq i < n / d\} \\
  \end{gather}
  \[| \im(p_k) | = \frac{n}{\gcdru(k, n)} = \frac{n}{d}\]
\end{example}

\begin{theorem}
  $|G| = |\im |(\phi)| \cdot |\ke (\phi)|$
\end{theorem}

\begin{lemma}
  $b \in \im (\phi) \hra \phi^{-1}(b) = a \cdot \ke (\phi) = \ke (\phi) \cdot a$
\end{lemma}

\begin{proof}~
  \begin{itemize}
    \item
    $\phi^{-1} (b) \supseteq a \cdot \ke (\phi)$ \\
    $x = ak, \ \phi(k) = e_H$ \\
    $\phi (x) = \phi(ak) = \phi(a) \phi(k) = b \cdot e_H = b \Ra$ \\
    $x \in \phi ^{-1} (b)$
    \item 
    $\phi^{-1} (b) \supseteq a \cdot \ke(\phi)$ \\
    $ x = ka$ \\
    Далее аналогично предыдущему пункту
    \item 
    $\phi^{-1} \subseteq a \cdot \ke (\phi)$ \\
    $\phi(x) = b, \ x = a(a^{-1} x)$ \\
    $\phi(a^{-1} x) = \phi(a^{-1}) \cdot \phi(x) = b^{-1} \cdot b = e_H$ \\
    $a^{-1} \cdot x \in \ke (\phi)$
    \item
    $\phi^{-1}(b) \subseteq \ke (\phi) \cdot a$ \\
    $\phi(x) = b$ \\
    $\phi(x a^{-1}) = \phi(x) \phi(a^{-1}) = b \cdot b^{-1} = e_H$ \\
    $x = (x a^{-1}) a = x(a^{-1}a)$
  \end{itemize}
\end{proof}

\begin{corollary}
  $|\im(\phi)| = (G : H)$ --- количество смженых классов \\
  $|G| = |H| \cdot (G : H), \quad |G| = |\ke (\phi)| \cdot (G : \ke (\phi))$
\end{corollary}

\begin{corollary}
  Для любого элемента группы образа, полный прообраз --- смежный класс, и это биекция.
\end{corollary}

\begin{proposition}
  $p$ --- простое, $a \not \equiv 0 \mod (p)$ , $a$ --- квадратичный вычет, тогда \\
  \[x^2 \equiv a \mod (p)\]
  является его решением.
\end{proposition}

\begin{lemma}[Необходимое условие]
  $a$ --- квадратичный вычет по модулю $p$ : \\
  \[a^{(p-1)/2} \equiv \mod (p).\]
\end{lemma}

\begin{definition}
  $K \vartriangleleft  G \Lra \forall g, \quad gk = gk$, где $K$ --- нормальная подгруппа группы $G$.
\end{definition}

\begin{corollary}
  $\ke (\phi) $ --- нормальная подгруппа.
\end{corollary}

\begin{example}
  \begin{gather}
    H < S_3, \quad H = \langle (12) \rangle = \{ (), (12) \} \\
    (13)H = \{ (13), (13) \circ (12) = (123) \} \\
    H(13) = \{ (13), (12) \circ (13) = (132) \}
  % $\Ra \ke (\phi) \vartriangleleft G$
  \end{gather}
\end{example}

\begin{proposition}
  $\forall K \vartriangleleft G $ --- ядро некоторого гомоморфизма.
\end{proposition}

\begin{definition}
  $K \vartriangleleft G, \quad G/R$ --- фактор-группа.
\end{definition}

\begin{proof}
  \begin{gather}
    \{ gK: g \in G \} \\
    g K = [g]_K = [g] \\
    [g_1] [g_2] = [g_1 g_2] \\
    K = [e] \text{--- класс вычетов из нейтрального элемента} \\
    [g]^{-1} = [g^{-1}] \\
  \end{gather}
\end{proof}

% дальше какая-то страшилка, см избранные в вк, я в ахуе если честно

\subsection*{Основная теорема о гомоморфизме}
\begin{theorem}[Основная теорема о гомоморфизме]
  $\im (\phi) \cong G / \ke (\phi) $, где 
  \[\phi: G \ra H \text{--- гомоморфизм}\]
  \[\Psi: G/ \ke (\phi) \ra \im (\phi) \text{--- изоморфизм}\]
  \[\Psi: [g] \raone \phi(g)\]

  Корректность: $g' = g \cdot k, \quad k \in \ke \phi$:
  \[\phi(g') = \phi(gk) = \phi(g) \phi(k) = \phi(g) \cdot e_H = \phi(g)\]

  Сохранение операции:
  \begin{gather}
    \Psi ([g_1][g_2]) = \Psi([g_1 g_2]) \\
    \phi (g_1 g_2) = \phi(g_1) \cdot \phi (g_2) = \Psi ([g_1]) \cdot \Psi([g_2]) \\
    \Psi : [g] \ra e_H = \phi(g) \Ra g \in \ke (\phi) \\
    [g] = \ke (\phi)
  \end{gather}
\end{theorem}
