\section{№2 Не абелевы группы}

\begin{reminder}
    Группа $(G, \cdot)$
    \begin{itemize}
        \item $(G1): x \cdot (y \cdot z) = (x \cdot y) \cdot z$ 
        \item $(G2): \exists c \in G \  \forall x: \ x \cdot e = e \cdot x = x$ 
        \item $(G3): \forall x \in G \  \exists x^{-1}: \ x \cdot x^{-1} = x^{-1} \cdot x = e$ 
        \item Абелевы группы: $x \cdot y = y \cdot x$
        \end{itemize}

\end{reminder}

\subsection{Группы биекций}

\begin{itemize}
    \item Тотальная функция: $f: A \ra B$
    \item Инъекция $\Lra f(x_1) = f(x_2) \Ra x_1 = x_2$
    \item Сюръекций $\Lra \forall y \in B \  \exists x \in A: \ f(x) = y$
    \item Биекция $\Lra$ инъекция и сюръекция
\end{itemize}

\subsection{Композиция функций}

\begin{definition}
    Пусть заданы функции $f, g: A \ra A$. Тогда их композицией будет:

    $(f \circ g)(x) = f(g(x))$
\end{definition}


\begin{example}
    $(S(x), +)$ --- группа.

    Где, $X$ --- множество, $S(x)$ --- биекция $X \ra X$.
\end{example}


% Пример: $(S(x),\  +)$ -- группа \\
% X -- множество \\
% S(x) -- биекция $X \rightarrow X$ \\


% Композиция \\
\begin{note}
    \[\big( (f \comp g) \comp h \big)(x) = \big( f \comp (g \comp h) \big)(x) \Ra\]
    \[\Ra \bigl( \big(g(h(x))\big) \bigl)\]
    \[\id : \ x \raone x\]
\end{note}

% $id: x \rightarrow x$ \\

Обратная функция --- просто биекция в другую сторону

\begin{example}[Не абелевой группы]
    \[f: \  x \raone x + 1\]
    \[g: \  x \raone -x\]
    \[(f \comp g)(x) = -x + 1\]
    \[(g \comp f)(x) = -x - 1\]
    Т.е. порядок имеет значение.
\end{example}
% Пример: \\
% $f: x \rightarrow x + 1$ \\
% $g: x \rightarrow -x$ \\
% \[(f \circ g)(x) = -x + 1\] 
% \[(g \circ f)(x) = -x - 1\], т.е. не равны \\

\newpage
\subsection{Перестановки}

\begin{definition}
    $X = \left \{1, 2, \dots n \right \}$ --- множество $n$ элементов.

    $S_n = S(x)$ --- группа перестановок степени $n$.
\end{definition}

\begin{enumerate}
    \item[I.] \ul{Размещение} из $n$ по $n$. 
    
    34512
    \item[II.] \ul{Табличный}
    
    \[
        \Pi = 
        \begin{pmatrix}
            1 2 3 4 5 \\
            3 4 5 1 2
        \end{pmatrix}
        \]
        
        $\Pi (1) = 3$.
        
        \[
        \sigma = 
        \begin{pmatrix}
            1 2 3 4 5 \\
            2 1 3 4 5
        \end{pmatrix}
        \]
        
        \[
        (\Pi \  \circ \  \sigma)  = 
        \begin{pmatrix}
            1 2 3 4 5 \\
            $\st{21345}$ \\
            4 3 5 1 2
        \end{pmatrix}
        \]
    \item[III.] \ul{С помощью графа}. Входящие и исходящие степени равны по 1, так как является биекцией.

    \[\Pi = (13524) = (52413)\]
    \[\sigma = (12)\] Bключая циклы длины 1 (просто их не пишем).
    
    \[(\Pi \circ \sigma) = (14)(23)\]
    
\end{enumerate}

% 1. Размещение из n по n \\
% $2451$ \\

% 2. Табличный \\
% \[
% \Pi = 
% \begin{pmatrix}
%     1 2 3 4 5 \\
%     3 4 5 1 2
% \end{pmatrix}
% \]

% $\Pi (1) = 3$

% \[
% \sigma = 
% \begin{pmatrix}
%     1 2 3 4 5 \\
%     2 1 3 4 5
% \end{pmatrix}
% \]

% \[
% (\Pi \  \circ \  \sigma)  = 
% \begin{pmatrix}
%     1 2 3 4 5 \\
%     2 1 3 4 5 \\
%     4 3 5 1 2
% \end{pmatrix}
% \], и вычеркнем центральную линию.

% 3. С помощью графа. Входящие и исходящие степени равны по 1, т.к. является биекцией.
% \[\Pi = (13524) = (52413)\]
% \[\sigma = (12), \] исключая циклы длины 1 (просто их не пишем).

% \[(\Pi \circ \sigma) = (14)(23)\]

% \subsection{Продолжение не абелевых групп}

\subsection{Общие свойства групповых операций}

\begin{enumerate}
    \item \ul{Закон сокращения}:
    \[\forall a \in G: \  x = y \Leftrightarrow ax = ay \Leftrightarrow xa = ya\]
    \[a^{-1}(ax) = (a^{-1}a)x = ex = x\]

    \item \ul{Формула обратного произведения}:
    \[(xy)^{-1} = y ^ {-1} \cdot x ^{-1}\]
    \[(xy)(y^{-1}x^{-1}) = x(y y^{-1}) x^ {-1} = xex^{-1} = x x^{-1} = e\]
    Проверка, что обратный единственный: \\
    \[C = x u = xv\] и получаем желаемое.
\end{enumerate}

% 1. Закон сокращения: \\
% \[\forall a \in G \  x = y \Leftrightarrow ax = ay \Leftrightarrow xa = ya\]
% \[a^{-1}(ax) = (a^{-1}a)x = ex = x\]

% 2. Формула обратного произведения: \\
% \[(xy)^{-1} = y ^ {-1} \cdot x ^{-1}\]
% \[(xy)(y^{-1}x^{-1}) = x(y y^{-1}) x^ {-1} = xex^{-1} = x x^{-1} = e\]
% Проверка, что обратный единственный: \\
% \[C = x u = xv,\] и получаем желаемое.

\subsection{Подгруппы}

\begin{definition}
    Подгруппа $H$ группы $G$: $H \subseteq G$ ($H < G$) (групповая операция та же самая).
\end{definition}

\begin{example}
    $3 \mathbb{Z}  = \left\{ x \mid x = 3y,\  y \in \mathbb{Z} \right\}$
\end{example}

\begin{definition}
    $GL(\mathbb{R}, n)$ --- общая группа линейных преобразований.
\end{definition}


Проверка, что элементы принадлежит группе:
\begin{itemize}
    \item[\text{$\left. 1 \right)$}] $e_G \in H: \begin{cases} e_H \cdot h = h, \\ e_G \cdot h = h \end{cases} \Ra e_H = e_G$
    \item[\text{$\left. 2 \right)$}] $x \in H \Ra x ^ {-1} \in H: \begin{cases} x_H^{-1} x = e, \\ x_G^{-1} x = e \end{cases} \Ra x_H^{-1} = x_G^{-1}$
    \item[\text{$\left. 3 \right)$}] $x, y \in H, \  x \cdot y \in H$
\end{itemize}

\begin{reminder}
    Вычеты --- $\bar{n} \ZZ \quad \left[a \right]_n = \left\{ x: x = a + z, z \in \bar n \ZZ \right\}$ ($H < G, \ g \in G$)
\end{reminder}

\begin{definition}
    ~

    Смежный (левый) класс --- $gH = \left\{x \mid x = gh, h \in H\right\}$,
    
    Смежынй (правый) класс --- $Hg = \left\{x \mid x = hg, h \in H\right\}$
\end{definition}

\begin{lemma}
    Смежные классы не пересекаются или совпадают.
\end{lemma}
\begin{proof}
    \begin{gather}
        z \in xH \  \land \  yH \quad z = x h_1 = y h_2, \quad h_1, h_2 \in H \\
        \begin{cases} x = y(h_2h_1^{-1}) \in yH, \quad xH \subseteq yH, \\
        xh_1 = y (h_2h_1^{-1}h_1) \in yH, \quad yH \subseteq xH \end{cases} \Ra xH = yH
    \end{gather}
\end{proof}

\subsection{Порядок группы}

\begin{definition}
    $|H| < \infty$ --- порядок группы.
\end{definition}

\[\Rightarrow |gH| = |H|\]
\[\alpha: H \ra gH\]
\[\alpha(h) = gH\]

\[gh_1 = gh_2 \Rightarrow h_1 = h_2\]

\begin{definition}
    $(G: H)$ --- индекс подгруппы (число смежных классов).
\end{definition}

\begin{theorem}[Лагранж]
    $|G| = |H| \cdot (G: H)$
\end{theorem}

\begin{itemize}
    \item[$1^{0}$] Порядок подгруппы делит порядок группы.
    \item[$2^{0}$] $|G| = p$ --- простое. В $G$ нет совместных подгрупп $\left\{ e \right\}, G$ --- несобственные.
    \item[$3^{0}$] $|G| = p$ --- простое, то $G$ -- абелева.
\end{itemize}

\begin{definition}
    $\langle g \rangle$ --- наименьшая подгруппа, содержащая g.
\end{definition}

\[\langle g \rangle = \left\{ e = g^0, g^1, g^2, \dots, g^{-1}, g^{-2} \right\}\]
\[\forall g: \ \langle g \rangle < G, \ |G| = p \Rightarrow \langle g \rangle = G\]

\begin{definition}
    Порядок элементов в группе --- $\ord g = |\langle g \rangle| \Leftrightarrow $ наименьшее $k: \  g^k = e$

\end{definition}

\begin{example}
    Рассмотрим группу: $(\ZZ_n, +)$
    \[\langle a \rangle = {\left [ ka \right ]}_n\]
    \[\ord a = \min k: ka \equiv 0 \mod n = \frac{\lcm (a, n)}{a} = \frac{n}{\gcdru (a, n)} \Ra\]
    \[\Ra \gcdru (a, n) \cdot \lcm (a, n) = a \cdot n\]
\end{example}


% $\langle a \rangle = {\left [ ka \right ]}_n$

% $\ord a = \min l: ka \equiv 0 \mod n = \frac{\lcm (a, n)}{a}$

$|\langle d \rangle| = \frac{n}{d}$

$|(\ZZ^{\text{*}}_n, \cdot)| = \phi (n) $ --- функция Эйлера.

\begin{theorem}[\text{Эйлер}]
    \[\gcdru(a, n) = 1 \Ra a^{\phi (n)} \equiv 1 \mod n\]
    \[k \cdot \ord a = \phi(n)\]
    \[a ^ {k \ord a} = (a ^ {\ord a})^k = 1 ^ k = 1\]
\end{theorem}

\begin{theorem}[\text{Малая теорема Ферма}]
    Если $p$ --- простое, то:
    \[a \not \equiv 0 \mod p \Ra a^{p - 1} \equiv 1 \mod p\]
\end{theorem}
