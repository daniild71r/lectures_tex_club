\subsection{Первообразные корни и индексы}

\begin{definition}
	Пусть $m \in N,\ a \in \Z \colon (a, m) = 1$. \textit{Показателем элемента} $a$ по модулю $m$ (или же \textit{порядком элемента} $a$ в группе $\Z_m$) называется минимальное положительное целое число $\delta$ такое, что
	\[
		a^{\delta} \equiv 1 \pmod m
	\]
\end{definition}

\begin{note}
	Согласно теореме Эйлера
	\[
		a^{\phi(m)} \equiv 1 \pmod m
	\]
	поэтому $\delta \le \phi(m)$ точно.
\end{note}

\begin{note}
	Отсюда и до конца параграфа мы будем сохранять введённые выше обозначения $a, m, \delta$, если не сказано обратного.
\end{note}

\begin{proposition}
	$\delta \mid \phi(m)$ всегда.
\end{proposition}

\begin{proof}
	Предположим обратное. Тогда, $\phi(m)$ делится на $\delta$ с остатком:
	\[
		\phi(m) = \delta q + r,\ 0 < r < \delta
	\]
	Но в таком случае заметим, что
	\[
		1 \equiv a^{\phi(m)} \equiv a^{\delta q + r} = a^r \cdot (a^{\delta})^q \equiv a^r
	\]
	Получили противоречие с выбором $\delta$.
\end{proof}

\begin{proposition}
	Все числа $\{1, a, a^2, \ldots, a^{\delta - 1}\}$ различны.
\end{proposition}

\begin{proof}
	Предположим, что $a^{i} \equiv a^j \pmod m,\ i \ge j$. Однако, это означает, что если домножить обе части на $a^{\delta - i}$, то получится следующее:
	\[
		1 \equiv a^\delta \equiv a^{\delta - (i - j)} \pmod m
	\]
	Такое возможно тогда и только тогда, когда $i = j$.
\end{proof}

\begin{definition}
	Назовём $a$ \textit{первообр\'{а}зным корнем} по модулю $m$, если $\delta(a) = \phi(m)$, то есть показатель элемента $a$ по модулю $m$ совпадает со значением функции Эйлера от $m$.
\end{definition}

\begin{note}
	Степени первообразного корня образуют всю приведённую систему вычетов.
\end{note}

\begin{definition}
	\textit{Дискретным логарифмом} (\textit{индексом}) числа $b$, взаимно простого с $m$, по основанию $a$ будем называть число
	\[
		\ind_a b := \log_a b := i \lra a^i \equiv b \pmod m
	\]
\end{definition}

\begin{note}
	Особенность индекса заключается в том, что на данный момент не существует алгоритма с приемлемой асимптотикой, который позволяет быстро вычислить индекс произвольного допустимого числа. На этой загвоздке основан один из методов шифрования.
\end{note}