\section{Теория вероятностей}

\begin{note}
	Теория вероятностей - это наука, возникшая из наблюдения за физическими явлениями. Она изучает \textit{случайные эксперименты}, но что это означает? Эксперимент - это физическое явление, а какие эксперименты случайны? Мы будем считать таковыми те, которые удовлетворяют следующим условиям:
	\begin{enumerate}
		\item Повторяемость - у нас есть возможность провести эксперимент заново, повторив при этом начальные условия с определенной точностью (например, мы никогда не сможем заставить молекулы воздуха в комнате оказаться ровно в тех же позициях, что были изначально. Отсюда требование по точности) 
		
		\item Отсутствие детерминистической регулярности (то есть та неточность, с которой мы допускаем повторение эксперимента, может существенно повлиять на его результат)
		
		\item Статистическая устойчивость частот. Говоря языком математики, то пусть $N_i$ - число экспериментов в серии $i$, $N_i(A)$ - количество экспериментов, в которых результатом оказалось явление $A$. Должно быть верно следующее:
		\[
			\forall i, j \in \N \quad \frac{N_i(A)}{N_i} \approx \frac{N_j(A)}{N_j} 
		\]
	\end{enumerate}
	Однако, что значит \textit{примерно} и \textit{определенная точность}? Это, увы, зависит от реальной задачи.
\end{note}

\begin{example}
	Приведём несколько некорректных экспериментов в форме вопроса:
	\begin{itemize}
		\item Какова вероятность того, что человек выйдет из дома и встретит динозавра? - у этого эксперимента есть детерминистическая регулярность, мы никогда не встретим динозавра, потому что они вымерли.
		
		\item Какова вероятность того, что Тихон сдаст экзамен по ОВиТМ на отл10? - здесь отсутствует повторяемость. Даже если брать во внимание тот факт, что у него есть 3 попытки сдать экзамен, всё равно на каждый последующий Тихон будет приходить <<другим>>, более подготовленным студентом.
	\end{itemize}
\end{example}

\begin{example}
	Бросок маленькой монеты с высоты 1 метра может рассматриваться как случайный эксперимент.
\end{example}

\begin{note}
	\textit{Математическая модель случайного эксперимента} должна переводить реальные сущности на язык математики:
	\begin{itemize}
		\item Разные результаты эксперимента $\lra \{w_1, \ldots, w_n, \ldots\} =: \Omega$ - \textit{элементарные исходы}
		
		\item Совокупность результатов экспериментов, объединенные физичечскими характеристиками, которые мы ожидаем $\lra A \subset \Omega$ - \textit{событие}
		
		\item Частота конкретного события $A \ra P(A)$ - \textit{вероятность}
	\end{itemize}
\end{note}

\begin{note}
	Вероятность - это идеализация частоты до уровня математической абстракции. Она возникает в момент создания математической модели, и на вопрос: <<А почему вероятность такая?>> - ответ создателя модели очень прост: <<Мне так захотелось>>.
	
	Строго говоря, говорить о вероятности реального события нельзя. Можно говорить только о вероятности подмножества $\Omega$, потому что вероятность - это чисто математика.
\end{note}

\begin{definition}
	Математическую модель случайного эксперимента принято описывать \textit{вероятностным пространством} $(\Omega, F, P)$, где
	\begin{itemize}
		\item $\Omega$ - это \textit{множество элементарных исходов}
		
		\item $F = \{A \such A \subseteq \Omega\}$ - \textit{множество рассматриваемых событий}, обладающее следующими свойствами:
		\begin{enumerate}
			\item $F$ замкнуто относительно операций $\cap, \cup, \bs, \tr$
			
			\item $\forall \{A_n\}_{n = 1}^\infty \subseteq F \Ra \bigcup_{n = 1}^\infty A_n \in F$
			
			\item $\Omega \in F$
		\end{enumerate}
		
		\item $P \colon F \to \R$ - \textit{вероятностная мера} (или же \textit{вероятность}), со следующими условиями:
		\begin{enumerate}
			\item $\forall A \in F \quad P(A) \in [0; 1]$
			
			\item $\forall \{A_n\}_{n = 1}^\infty \quad P(\bscup_{n = 1}^\infty A_n) = \sum_{n = 1}^\infty P(A_n)$
			
			\item $P(\Omega) = 1$
		\end{enumerate}
	\end{itemize}
\end{definition}