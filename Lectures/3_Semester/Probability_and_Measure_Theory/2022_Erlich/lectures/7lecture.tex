\begin{lemma}
	Пусть $S$ --- полукольцо, $A, A_1, \ldots, A_k \in S$, причём $\bscup_{i = 1}^k A_i \subseteq A$, тогда
	\[
		\exists A_{k + 1}, \ldots, A_n \in S \colon \bscup_{i = 1}^n A_i = A
	\]
\end{lemma}

\begin{proof}
	Проведём индукцию по $k$:
	\begin{itemize}
		\item База $k = 1$: тривиально по определению полукольца
		
		\item Переход $k > 1$: по предположению индукции
		\[
			\bscup_{i = 1}^{k - 1} A_i \subseteq A \Lora \ps{\bscup_{i = 1}^{k - 1} A_i} \sqcup \ps{\bscup_{j = 1}^q B_j} = A
		\]
		где $B_j \in S$. Понятно, что в переходе $A_k \subseteq \bscup_{j = 1}^q B_j$. По свойству полукольца, $D_j = A_k \cap B_j \in S$. В таком случае
		\[
			B_j = D_j \sqcup \ps{\bscup_{l = 1}^{a_j} C_{j, l}},\ \ C_{j, l} \in S
		\]
		Собираем всё вместе и получаем требуемое:
		\[
			A = \ps{\bscup_{i = 1}^k A_i} \sqcup \ps{\bscup_{j = 1}^q \bscup_{l = 1}^{a_j} C_{j, l}}
		\]
	\end{itemize}
\end{proof}

\begin{theorem}
	Пусть $S$ --- полукольцо, $R(S)$ --- минимальное кольцо, тогда
	\[
		R(S) = K(S) = \set{\bscup_{i = 1}^n A_i,\ A_i \in S}
	\]
\end{theorem}

\begin{proof}
	С самого начала можно заявить, что $R(S) \supseteq K(S)$. Это следует из того, что кольцо замкнуто относительно объединения. Другое вложение докажем тем фактом, что $K(S)$ --- тоже кольцо, содержащее $S$:
	\begin{enumerate}
		\item $\emptyset \in K(S)$, так как $\emptyset \in S$
		
		\item Проверим замкнутость пересечения. $\forall A, B \in K(S)$ верны следующие записи:
		\[
			A = \bscup_{i = 1}^n A_i; \quad B = \bscup_{j = 1}^k B_j
		\]
		где $A_i,\ B_j \in S$. Положив $C_{i, j} := A_i \cap B_j \in S$, имеем
		\[
			A \cap B = \bscup_{i = 1}^n \bscup_{j = 1}^k C_{i, j} \in K(S)
		\]
		Коль скоро мы записали пересечение через конечное дизъюнктное объединение.
		
		\item Осталось проверить замкнутость. Продолжая рассуждения предыдущего пункта, мы можем записать $A_i$ в следующем виде:
		\[
			A_i = \ps{\bscup_{j = 1}^k C_{i, j}} \sqcup \ps{\bscup_{s = 1}^{s_i} D_{i, s}},\ \ D_{i, s} \in S
		\]
		Аналогично с $B_j$:
		\[
			B_j = \ps{\bscup_{i = 1}^n C_{i, j}} \sqcup \ps{\bscup_{l = 1}^{l_j} E_{j, l}},\ \ E_{j, l} \in S
		\]
		С этим мы можем записать симметрическую разность так:
		\[
			A \tr B = \ps{\bscup_{i = 1}^n \bscup_{s = 1}^{s_i} D_{i, s}} \sqcup \ps{\bscup_{j = 1}^k \bscup_{l = 1}^{l_j} E_{j, l}} \in K(S)
		\]
	\end{enumerate}
	Этими тремя пунктами мы обосновали, что $K(S)$ --- полукольцо. В силу его структуры, оно также является и кольцом, это проверяется просто. Стало быть, $R(S) \subseteq K(S)$.
\end{proof}

\begin{lemma}
	Пусть $S$ --- полукольцо, а $A_1, \ldots, A_n \in S$ --- произвольный набор множеств. Тогда
	\[
		\exists B_1, \ldots, B_k \in S \such A_i = \bscup_{j \in \Delta_i} B_j
	\]
\end{lemma}

\begin{proof}
	Проведём индукцию по $n$:
	\begin{itemize}
		\item База $n = 1$: тогда $B_1 = A_1$ и всё, победа.
		
		\item Переход $n > 1$: для $A_1, \ldots, A_{n - 1}$ нашлись множества $B_1, \ldots, B_q \in S$. Возьмём $A_n$ и пересечём со всеми ними: $C_s = A_n \cap B_s \in S$. Тогда
		\[
			A_n = \ps{\bscup_{s = 1}^q C_s} \sqcup \ps{\bscup_{p = 1}^m D_p}
		\]
		В результате мы <<измельчили>> все $B_s$. Теперь они записываются следующим образом (по свойству полукольца):
		\[
			B_s = C_s \sqcup \ps{\bscup_{r = 1}^{r_s} B_{s, r}},\ \ B_{s, r} \in S
		\]
		Итого, чтобы записать набор $A_1, \ldots, A_n$, нам нужны $\{C_s\}_{i = 1}^q$, $\{B_{s, r}\}$ и $\{D_p\}_{p = 1}^m$, причём все они попарно непересекаются.
	\end{itemize}
\end{proof}

\textcolor{red}{Я остановился техать на 29:37 7й лекции}