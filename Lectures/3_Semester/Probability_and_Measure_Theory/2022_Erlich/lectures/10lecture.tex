\begin{definition}
	Лебегово продолжение меры $m$ из примера выше называется \textit{мерой Лебега-Стилтьеса}
\end{definition}

\subsection{$\sigma$-конечные меры}

\begin{definition}
	$\mu \colon N \to \R$ называется \textit{конечной мерой} на системе $N$ подмножеств $X$, если выполнены следующие условия:
	\begin{enumerate}
		\item (Конечная аддитивность) $\forall A \in N, \{A_i\}_{i = 1}^n \subseteq N$ таких, что $A = \bscup_{i = 1}^n A_i$ верно равенство $\mu(A) = \sum_{i = 1}^n \mu(A_i)$
		
		\item (Конечность значений) $\forall A \in N\ \ \mu(A) \in \lsi{0; +\infty}$
		
		\item (Конечность) $\exists \{A_i\}_{i = 1}^n \subseteq N \such \mu(A_i) < \infty,\ X = \bigcup_{i = 1}^n A_i$
	\end{enumerate}
\end{definition}

\begin{definition}
	$\mu \colon N \to \R$ называется \textit{$\sigma$-конечной мерой} на системе $N$ подмножеств $X$, если выполнены следующие условия:
	\begin{enumerate}
		\item ($\sigma$-аддитивность) $\forall A \in N, \{A_i\}_{i = 1}^\infty \subseteq N$ таких, что $A = \bscup_{i = 1}^\infty A_i$ верно равенство $\mu(A) = \sum_{i = 1}^\infty \mu(A_i)$
		
		\item (Наличие бесконечного значения) $\forall A \in N\ \ \mu(A) \in [0; +\infty]$
		
		\item ($\sigma$-конечность) $\exists \{A_i\}_{i = 1}^\infty \subseteq N \such \mu(A_i) < \infty,\ \ X = \bigcup_{i = 1}^\infty A_i$
	\end{enumerate}
\end{definition}

\begin{note}
	Если в этом параграфе где-то есть слово <<мера>> без уточнения конечности, то подразумевается, что утверждение работает для любой конечности.
\end{note}

\begin{note}
	В общем случае ($\sigma$-)конечности меры не требуется ни ($\sigma$-)конечная аддитивность, ни требований на значения меры.
\end{note}

\begin{note}
	Возникает уже известный вопрос: <<А как продлить $\sigma$-конечную меру на что-то более сложное?>> Для этого нужно хотя бы определить, от чего мы отталкиваемся:
	\begin{enumerate}
		\item $S$ - полукольцо подмножеств множества $E$
		
		\item $E \notin S$
		
		\item $\exists \{A_i\}_{i = 1}^\infty \subseteq S \such E = \bigcup_{i = 1}^\infty A_i$
		
		\item $m$ --- это $\sigma$-аддитивная мера, принимающая только конечные значения.
	\end{enumerate}
	От $S$ и $m$ мы можем спокойно перейти к $R(S)$ и $\nu$, причём $\nu$ тоже будет $\sigma$-аддитивной. Приятность состоит в том, что теперь мы можем получить $X$ при помощи дизъюнктного объединения:
	\[
		E = \bigcup_{i = 1}^\infty A_i = \bscup_{j = 1}^\infty B_j
	\]
	Естественно, $B_j \in R(S)$. Если же говорить более точно, то $B_j = A_j \bs \bigcup_{i = 1}^{j - 1} A_i$.
	
	Мы явно указали, что $E \notin S$. Однако, теперь мы можем рассмотреть кольца $R_j = R(S) \cap B_j := \{B \cap B_j \such B \in R(S)\}$, и у них уже будут единицы --- $B_j$ соответственно (то есть это алгебры). По имеющейся теории, это даёт право построить лебегово продолжение $\nu$ на подмножествах $B_j$ --- так обозначаемые $\mu_j$. Соответственно будут множества измеримых по Лебегу множеств $M_j$. Задача, которую нам предстоит решить --- это объединить полученные кусочки.
\end{note}

\begin{note}
	Далее мы продолжаем жить в обозначениях выше и с ровно тем же смыслом, если не обговорено явно иного.
\end{note}

\begin{definition}
	$A \subseteq E$ называется \textit{измеримым по Лебегу}, если $\forall j \in \N\ \ A \cap B_j \in M_j$.
\end{definition}

\begin{definition}
	Пусть $A \subseteq E$ --- измеримое по Лебегу множество. Тогда его \textit{мера Лебега} равна следующему:
	\[
		\mu(A) = \sum_{j = 1}^\infty \mu_j(A \cap B_j)
	\]
\end{definition}

\begin{note}
	Проблема --- разбиение $\bigcup_{i = 1}^\infty A_i$ необязательно единственно. Из-за этого возникает вопрос корректности происходящего, но выкладки настолько громоздки и бессмысленны, что мы принимаем корректность без доказательства.
\end{note}

\begin{note}
	Мы снова взяли какую-то белиберду и назвали её мерой, но является ли она таковой? С ходу нет. Нам нужно доказать 2 вещи:
	\begin{enumerate}
		\item Множество измеримых по Лебегу множеств $M$ является $\sigma$-алгеброй
		
		\item Полученная мера Лебега $\mu$ является $\sigma$-конечной мерой (здесь сразу вскрыты карты, что у нас будет не просто мера)
	\end{enumerate}
\end{note}

\begin{theorem}
	$M$ является $\sigma$-алгеброй.
\end{theorem}

\begin{proof}~
	\begin{enumerate}
		\item (Наличие единицы) $\forall j \in \N\ E \cap B_j = B_j \in M_j \Ra E \in M$
		
		\item (Замкнутость как кольца) Рассмотрим $A, C \in M$. При фиксированном $j \in \N$ мы знаем, что $A \cap B_j \in M_j$, $C \cap B_j \in M_j$. Отсюда сразу $(A \cap C) \cap B_j \in M_j$, а значит $A \cap C \in M$. Для симметрической разности нужно просто помнить дистрибутивность:
		\[
			(A \cap B_j) \tr (C \cap B_j) = (A \tr C) \cap B_j
		\]
		Отсюда $A \tr C \in M$
		
		\item (Замкнутость относительно счётного объединения) Аналогично предыдущему пункту, ибо каждое $M_j$ является $\sigma$-алгеброй.
	\end{enumerate}
\end{proof}

\begin{note}
	Важно понимать: первый пункт не только дал нам единицу, но и показал, что $M \neq \emptyset$ --- одно из необходимых условий.
\end{note}

\begin{theorem}
	Мера Лебега $\mu$ является $\sigma$-конечной мерой на системе $M$ подмножеств $E$.
\end{theorem}

\begin{proof}
	Посмотрим свойства прямо по порядку из определения:
	\begin{enumerate}
		\item Пусть $A = \bscup_{i = 1}^\infty A_i;\ \ A, A_i \in M$. Распишем меру $A$ по определению:
		\[
			\mu(A) = \sum_{j = 1}^\infty \mu_j(A \cap B_j) = \sum_{j = 1}^\infty \mu_j\ps{\bscup_{i = 1}^\infty (A_i \cap B_j)}
		\]
		Здесь $A_i \cap B_j \in M_j$. При этом мы знаем, что $\mu_j$ обладает $\sigma$-аддитивностью. Значит, меру можно раскрыть:
		\[
			\mu(A) = \sum_{j = 1}^\infty \mu_j\ps{\bscup_{i = 1}^\infty (A_i \cap B_j)} = \sum_{j = 1}^\infty \sum_{i = 1}^\infty \mu_j(A_i \cap B_j) = \sum_{i = 1}^\infty \sum_{j = 1}^\infty \mu_j(A_i \cap B_j) = \sum_{i = 1}^\infty \mu(A_i)
		\]
		
		\item По построению (ряд может принимать бесконечное значение)
		
		\item По построению
	\end{enumerate}
\end{proof}

\subsection*{Свойства мер}

\begin{definition}
	Мера $\mu$ с конечными значениями на кольце $R$ называется \textit{непрерывной}, если $\forall \{A_i\}_{i = 1}^\infty \subseteq R, A_i \supseteq A_{i + 1}$ при обозначении $A = \bigcap_{i = 1}^\infty A_i$ верно следующее условие:
	\[
		\mu(A) = \lim_{n \to \infty} \mu(A_n)
	\]
\end{definition}

\begin{note}
	В курсе математического анализа мы вводили предел на множествах. В частности, в определении выше можно было написать $A = \lim_{n \to \infty} A_n$ вместо пересечения всех множеств.
\end{note}

\begin{theorem}
	Мера $\mu$ с конечными значениями на кольце $R$ непрерывна тогда и только тогда, когда $\mu$ является $\sigma$-аддитивной конечной мерой.
\end{theorem}

\begin{proof}
	Проведём доказательство в 2 стороны:
	\begin{itemize}
		\item $\La$ Пусть $A = \bigcap_{i = 1}^\infty A_i$, $A_i \supseteq A_{i + 1}$ В силу сложения множеств, имеет место неравенство $m(A_i) \ge m(A_{i + 1})$, то есть последовательность мер невозрастает. При этом, она ограничена снизу нулём, то есть по теореме Вейерштрасса предел существует. Имеет место равенство:
		\[
			A_1 = \ps{\bscup_{i = 1}^\infty A_i \bs A_{i + 1}} \sqcup A
		\]
		Так как есть $\sigma$-аддитивность меры, то это из этого получаем следующее:
		\[
			m(A_1) = \sum_{i = 1}^\infty (m(A_i) - m(A_{i + 1})) + m(A)
		\]
		Ряд можно записать по определению через предел, тогда получим телескопическую сумму:
		\[
			m(A_1) = \lim_{n \to \infty} \sum_{i = 1}^n (m(A_i) - m(A_{i + 1})) + m(A) = m(A_1) - \lim_{n \to \infty} m(A_{n + 1}) + m(A)
		\]
		Сокращаем $m(A_1)$, разносим слагаемые по разные стороны и получаем требуемое.
		
		\item $\Ra$ Пусть $B = \bscup_{i = 1}^\infty B_i$. Применим непрерывность к такой последовательности:
		\begin{align*}
			&{A_1 = B}
			\\
			&{A_i = \bscup_{j = i + 1}^\infty B_j \Lora A_1 \bs A_i = \bscup_{j = 1}^i B_j}
		\end{align*}
		Тогда
		\[
			\mu(B) = \mu(B \bs \emptyset) = \lim_{n \to \infty} \mu(B \bs A_i) = \lim_{n \to \infty} \mu\ps{\bscup_{j = 1}^n B_j} = \sum_{j = 1}^\infty \mu(B_j)
		\]
	\end{itemize}
\end{proof}

\begin{note}
	Очень важно, что мы говорим про меры с конечными значениями. Банальный пример: рассмотрим последовательность правых лучей $A_n = (n; +\infty)$. Тогда $A = \lim_{n \to \infty} A_n = \emptyset$, но при этом $\mu(A_n) = +\infty$ при любом $n \in \N$.
\end{note}