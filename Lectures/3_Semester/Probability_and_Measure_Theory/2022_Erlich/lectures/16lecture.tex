\begin{theorem} (Фату)
	Пусть $\mu$ --- полная мера, а $\{f_n\}_{n = 1}^\infty$ --- последовательность неотрицательных функций, сходящихся почти всюду на $E$ к $f$. Тогда имеет место неравенство:
	\[
		\int_E f(x)d\mu \le \varliminf_{n \to \infty} \int_E f_n(x)d\mu
	\]
\end{theorem}

\begin{proof}
	Рассмотрим функции инфинумов $\phi_n(x) = \inf_{k \ge n} f_k(x)$ на $E$. Тогда понятно, что эти функции неуывают в каждой точке на $E$ и более того:
	\[
		\lim_{n \to \infty} \phi_n(x) = \varliminf_{n \to \infty} f_n(x) = f(x)
	\]
	Обозначим $E_1$ --- множество точек, где последовательность сходится. Тогда $\mu(E \bs E_1) = 0$ и можно записать следующие равенства (равенство с пределом обусловлено теоремой Леви):
	\[
		\int_E f(x)d\mu = \int_{E_1} f(x)d\mu = \lim_{n \to \infty} \int_{E_1} \phi_n(x)d\mu = \varliminf_{n \to \infty} \int_E \phi_n(x)d\mu
	\]
	Остаётся заметить, что $\forall n \in \N\ \int_E \phi_n(x)d\mu \le \int_E f_n(x)d\mu$
\end{proof}

\begin{theorem} (Лебега)
	Пусть $\mu$ --- полная мера, заданы $\{f_n\}_{n = 1}^\infty$ и $F \colon E \to \R$ на $E \in M$ со следующими свойствами:
	\begin{enumerate}
		\item $F \in L(E)$
		
		\item $F$ --- неотрицательная функция
		
		\item $\forall n \in \N\ \forall x \in E\ \ |f_n(x)| \le F(x)$
		
		\item $f_n \to f$ почти всюду на $E$
	\end{enumerate}
	Тогда имеет место равенство:
	\[
		\int_E f(x)d\mu = \lim_{n \to \infty} \int_E f_n(x)d\mu
	\]
\end{theorem}

\begin{proof}
	Коль скоро $|f_n(x)| \le F(x)$, причём последняя интегрируема по Лебегу, то и $f_n \in L(E)$. Более того, то же самое верно и про $f$ на множестве сходимости $E_1$. Так как мера полна и $\mu(E \bs E_1) = 0$ (то есть $f$, взятая на $E_1$ и дополненная какими-то значениями до $E$, будет эквивалентна $f$ на всей $E$). Пользуясь пунктом из теоремы об абсолютной непрерывности интеграла Лебега, устанавливаем, что $f \in L(E)$.
	
	Осталось установить равенство. Для этого рассмотрим такие последовательности функций:
	\begin{align*}
		&{\phi_n(x) = F(x) + f_n(x)}
		\\
		&{\psi_n(x) = F(x) - f_n(x)}
	\end{align*}
	Про них по условию мы уже можем сказать, что они неотрицательны и почти всюду есть пределы:
	\begin{align*}
		&{\lim_{n \to \infty} \phi_n(x) = F(x) + f(x)}
		\\
		&{\lim_{n \to \infty} \psi_n(x) = F(x) - f(x)}
	\end{align*}
	Применим к ним теорему Фату и получим оценки с двух сторон для равенства:
	\begin{multline*}
		\int_E F(x)d\mu + \int_E f(x)d\mu = \int_E (F(x) + f(x))d\mu \le
		\\
		\varliminf_{n \to \infty} \int_E (F(x) + f_n(x))d\mu = \int_E F(x)d\mu + \varliminf_{n \to \infty} \int_E f_n(x)d\mu
	\end{multline*}
	И второе аналогично:
	\begin{multline*}
		\int_E F(x)d\mu - \int_E f(x)d\mu = \int_E (F(x) - f(x))d\mu \le
		\\
		\varliminf_{n \to \infty} \int_E (F(x) - f_n(x))d\mu = \int_E F(x)d\mu - \varlimsup_{n \to \infty} \int_E f_n(x)d\mu
	\end{multline*}
	Итого:
	\[
		\varlimsup_{n \to \infty} \int_E f_n(x)d\mu \le \int_E f(x)d\mu \le \varliminf_{n \to \infty} \int_E f_n(x)d\mu
	\]
	Стало быть, предел существует и равенство выполнено.
\end{proof}

\begin{reminder}
	Функциональная последовательность $\{f_n\}_{n = 1}^\infty$ \textit{сходится равномерно} к $f \colon E \to \R$ на $E$, если выполнено утверждение:
	\[
		\forall \eps > 0 \exists N \in \N \such \forall n > N, x \in E\ \ |f_n(x) - f(x)| < \eps
	\]
\end{reminder}

\begin{theorem} (Егорова)
	Если $\mu(E) < \infty$ и $f_n \to f$ почти всюду на $E$, то
	\[
		\forall \eps > 0\ \exists E_\eps \subseteq E \such E_\eps \in M \wedge \mu(X \bs E_\eps) < \eps \wedge f_n \text{ сходится равномерно на } E_\eps
	\]
\end{theorem}

\begin{proof}
	По критерию сходимости почти всюду:
	\[
		\forall m \in \N\ \ \lim_{n \to \infty} \ps{\bigcup_{k = n}^\infty \set{x \in X \colon |f_k(x) - f(x)| > \frac{1}{m}}} = 0
	\]
	Отсюда получим такое утверждение:
	\begin{multline*}
		\forall \eps > 0\ \forall m \in \N\ \exists n_m \in \N \such n_m > n_{m - 1} \wedge \mu\ps{\bigcup_{k = n_m}^\infty \set{x \in X \colon |f_k(x) - f(x)| > \frac{1}{m}}} < \frac{\eps}{2^m}
	\end{multline*}
\end{proof}