\subsection{Математические модели случайного эксперимента}

\begin{itemize}
	\item Если $|\Omega| < \infty$, то такие модели называются \textit{дискретными}. Множество событий полагают равным $F = 2^\Omega$. Единственный вопрос, который остаётся открытым - Как задавать вероятность?
	\begin{enumerate}
		\item Классическая модель. В ней все вероятности элементарных событий равны:
		\[
			\Omega = \{w_i\}_{i = 1}^N; \quad \forall i \in \{1, \ldots, n\}\ \ P(\{w_i\}) = const = C
		\]
		Отсюда выводим вероятность одного исхода:
		\[
			1 = P(\Omega) = P\left(\bscup_{i = 1}^N \{w_i\}\right) = \sum_{i = 1}^N P(\{w_i\}) = C \cdot N \Longrightarrow C = \frac{1}{N} = \frac{1}{|\Omega|}
		\]
		Если у нас есть $\mathcal{A} \in F$, то вероятность такого события можно посчитать из определения:
		\[
			P(\mathcal{A}) = \sum_{w \in \mathcal{A}} P(\{w\}) = \sum_{w \in A} \frac{1}{|\Omega|} = \frac{|\mathcal{A}|}{|\Omega|}
		\]
		\begin{example}
			$n \ge 3$ незнакомых людей садятся за круглый стол. Найти вероятность того, что 2 конкретных человека окажутся рядом.
			
			Рассмотрим математические модели, которыми бы можно было описать эту задачу:
			\begin{enumerate}
				\item $w = (i_1, \ldots, i_n)$, где $i_j$ - это номер места, куда сядет $j$-й человек. Тогда $|\Omega| = n!$, а вероятность нашего события будет
				\[
					P(A) = \frac{|A|}{|\Omega|} = \frac{n \cdot 2 \cdot (n - 2)!}{n!} = \frac{2}{n - 1}
				\]
				
				\item $w = (x, y)$ - позиции, которые займут интересующие нас люди (без разбора, кто сядет первый). Тогда всего исходов $|\Omega| = C_n^2$, а вероятность нужного события будет
				\[
					P(A) = \frac{|A|}{|\Omega|} = \frac{n}{C_n^2} = \frac{n}{\frac{n(n - 1)}{2}} = \frac{2}{n - 1}
				\]
			\end{enumerate}
		\end{example}
	
		\begin{note}
			Это замечательно, когда задача сформулирована полно и разные модели дают одинаковую вероятность. Однако, бывает и иначе, но об этом будет сильно позже...
		\end{note}
	
		\begin{example}
			Есть $N$ изделий, среди которых $M$ бракованных. Найти вероятность того, что среди $n \le N$ выбранных изделий будет ровно $0 \le k \le n$ бракованных.
		
			\begin{enumerate}
				\item Положим $w = (i_1, \ldots, i_n)$, где $i_j$ - номер изделия при $j$-м вытаскивании. Тогда $|\Omega| = N^n$, а вероятность интересующего нас события будет
				\[
					P(A) = \frac{C_n^k \cdot M^k \cdot (N - M)^{n - k}}{N^n} = C_n^k \cdot p^k \cdot (1 - p)^{n - k}
				\]
				
				\item (Неклассическая модель) Теперь $w = (l_1, \ldots, l_n)$, где $l_i \in {0, 1}$ - бракован элемент или нет. Тогда $|\Omega| = 2^n$, $F = 2^\Omega$, а вероятность выбора бракованного элемента положим за $p = M / N$. Тогда
				\[
					P(\{w\}) = p^{\text{\#единиц}} \cdot (1 - p)^{\text{\#нулей}}
				\]
				где количество единиц можно выразить как $\sum_{k = 1}^n l_i$. Но очевидно ли, что $P(\Omega) = 1$? Вообще говоря, нет. Нужно это проверить:
				\[
					P(\Omega) = \sum_{w \in \Omega} p^{\sum_{i = 1}^n l_i} (1 - p)^{n - \sum_{i = 1}^n l_i} = \sum_{k = 0}^n C_n^k p^k (1 - p)^{n - k} = 1
				\]
			\end{enumerate}
		\end{example}
	
		\begin{example}
			Тихон играет с другом в монетку. $w = n$ - это число шагов, за которые выигрывает Тихон.
		\end{example}
	\end{enumerate}
\end{itemize}