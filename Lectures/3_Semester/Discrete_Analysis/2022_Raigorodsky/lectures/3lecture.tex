\subsection{Вероятности, связанные с характеристическими числами графов}

\begin{theorem}
	Почти любой граф $G$ на $n$ вершинах таков, что у него одновременно $w(G) < 2\log_2 n$ и $\alpha(G) < 2\log_2 n$
\end{theorem}

\begin{note}
	\textit{Почти любой} нужно понимать как существование следующего предела:
	\[
		\frac{\text{\#$G$ на $n$ вершинах с нужными свойствами}}{2^{C_n^2}} \xrightarrow[n \to \infty]{} 1
	\]
\end{note}

\begin{proof}
	Введём классическое вероятностное пространство. Положим $\Omega_n$ --- множество всех графов на $n$ вершинах (без изоморфизма). Тогда, хотелось бы доказать следующий факт:
	\[
		P(w(G) < 2\log_2 n) \xrightarrow[n \to \infty]{} 1
	\]
	Это эквивалентно стремлению вероятности противоположного события к нулю:
	\[
		P(w(G) \ge 2\log_2 n) \xrightarrow[n \to \infty]{} 0
	\]
	Зафиксируем некоторое $k = \floor{2\log_2 n}$ и занумеруем все наборы из $k$ вершин как $A_1, \ldots, A_{C_n^k}$. Обозначим $\mathcal{A}_i$ - событие, когда вершины $A_i$ в выпавшем графе $G$ образуют клику. $\mathcal{A} = \bigcup_{i = 1}^{C_n^k} \mathcal{A}_i$  --- событие, что будет хотя бы одна клика размера $k$. По свойству вероятностной меры
	\[
		P(\mathcal{A}) \le \sum_{i = 1}^{C_n^k} P(\mathcal{A}_i)
	\]
	Что есть $P(\mathcal{A}_i)$? Это надо поделить число графов, в которых $A_i$ образует клику, на число всех графов. То есть
	\begin{multline*}
		P(\mathcal{A}) \le \sum_{i = 1}^{C_n^k} \frac{2^{C_n^2 - C_k^2}}{2^{C_n^2}} = \sum_{i = 1}^{C_n^k} \ps{\frac{1}{2}}^{C_k^2} = C_n^k \cdot 2^{-C_k^2} \le \frac{n^k}{k!} \cdot 2^{-C_k^2} = \frac{2^{k \log_2 n - k^2 / 2 + k / 2}}{k!} \le
		\\
		\frac{1}{k!} \cdot 2^{\log^2_2 n + \log_2 n - \frac{(2\log_2 n - 1)^2}{2}} \le \frac{2^{3\log_2 n}}{k!} \le \frac{2^{1.5(k + 1)}}{k!} \to 0
	\end{multline*}
	Абсолютно аналогично можно доказать вероятность для числа независимости графа:
	\[
		P(\alpha(G) < 2\log_2 n) \xrightarrow[n \to \infty]{} 1
	\]
	Теперь поясним, почему вероятность выполнения этих условий одновременно тоже будет стремиться к единице. Пусть $\mathcal{B}$ --- это событие, что будет хотя бы одно независимое множество размера $k$. Тогда
	\[
		P(\mathcal{A} \cup \mathcal{B}) \le P(\mathcal{A}) + P(\mathcal{B}) \to 0
	\]
	Значит, $P(\overline{\mathcal{A} \cap \mathcal{B}}) \to 1$, что в точности является нужным событием.
\end{proof}

\begin{example}
	Зафиксируем $n = k^2$ и рассмотрим граф, состоящий из $k$ клик, в каждой из которых по $k$ элементов. Несложно понять, что в таком графе $w(G) = \sqrt{n} = \alpha(G)$
\end{example}

\begin{theorem} (Теорема Тур\'{а}на)
	Для любого графа $G = (V, E)$ такого, что $|V| = n$ и $\alpha(G) = \alpha$, имеет место оценка на число рёбер:
	\[
		|E| \ge n \cdot \floor{\frac{n}{\alpha}} - \alpha \frac{\floor{\frac{n}{\alpha}}\ps{\floor{\frac{n}{\alpha}} + 1}}{2}
	\]
\end{theorem}

\begin{proof}
	Пусть $A \subset V$ - это независимое подмножество вершин, реализующее число независимости (то есть $|A| = \alpha(G) = \alpha$). Тогда совершенно очевиден следующий факт:
	\[
		\forall x \in V \bs A\ \exists y \in A \such (x, y) \in E
	\]
	Следовательно, у нас уже есть не менее $n - \alpha$ рёбер. Повторим рассуждение для $V \bs A$ (в нём выделяем $A' \colon |A'| \le \alpha$ и так далее). Тогда, мы получим ещё $\ge (n - \alpha) - \alpha = n - 2\alpha$ рёбер. Продолжая спуск в разности множеств, получим следующую оценку на $|E|$:
	\[
		|E| \ge (n - \alpha) + (n - 2\alpha) + \ldots + \ps{n - \floor{\frac{n}{\alpha}}\alpha} = n \cdot \floor{\frac{n}{\alpha}} - \alpha \frac{\floor{\frac{n}{\alpha}}\ps{\floor{\frac{n}{\alpha}} + 1}}{2}
	\]
\end{proof}

\begin{note}
	<<Мощь>> теоремы заключается в том, что оценка неулучшаемая. Это показано ниже.
\end{note}

\begin{example}
	Пусть $\alpha \mid n$, тогда оценка теоремы Турана принимает вид $|E| \ge \frac{n^2}{2\alpha} - \frac{n}{2}$.
	
	\textcolor{red}{Сюда бы картиночку со второй лекции осени 2022, где-то 1:02:00}
	
	Посмотрим такой граф, для которого будет достигнуто равенство. Разобьём $n$ вершин на $\alpha$ равновеликих клик. Тогда понятно, что $\alpha(G) = \alpha$, но при этом число рёбер
	\[
		|E| = \alpha \cdot C_{n / \alpha}^2 = \alpha^2 \cdot \frac{\frac{n}{\alpha} \ps{\frac{n}{\alpha} - 1}}{2} = \frac{n^2}{2\alpha} - \frac{n}{2}
	\]
\end{example}

\subsection{Дистанционные графы}

\begin{definition}
	Граф $G = (V, E)$ называется \textit{дистанционным}, если $V \subset \R^n$, а $E = \{\{\vec{x}, \vec{y}\} \colon |\vec{x} - \vec{y}| = a\},\ a > 0$.
\end{definition}

\begin{example}
	Мы уже сталкивались с дистанционными графами в прошлом году. Примером послужит граф, к которому применялась теорема Эрдёша-Хватала: 
	\begin{itemize}
		\item \(V = \{\vec{x} \in \R^n \colon \forall i\ x_i \in \{0, 1\},\ x_1 \plusdots x_n = 3\}\)
		
		\item \(E = \{\{\vec{x}, \vec{y}\} \colon (\vec{x}, \vec{y}) = 1\}\)
	\end{itemize}
\end{example}