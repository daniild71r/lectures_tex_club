\begin{proposition}
	Имеют место следующие 2 факта про $f(n, r, s)$:
	\begin{enumerate}
		\item Если $r > \floor{n / 2}$, то $f(n, r, 1) = C_n^r$
		
		\item Если $r \le \floor{n / 2}$, то $f(n, r, 1) \ge C_{n - 1}^{r - 1}$
	\end{enumerate}
\end{proposition}

\begin{proof}~
	\begin{enumerate}
		\item Действительно, если это так, то любые $r$-элементные подмножества $\range{n}$ пересекаются. Стало быть, мы имеем право взять вообще все возможные из таких.
		
		\item В такой ситуации мы всегда можем зафиксировать какую-то вершину, скажем, 1, и просто взять все $r$-элементные подмножества, включающие её.
	\end{enumerate}
\end{proof}

\begin{theorem} (Эрдёш, Ко, Радо)
	Если $r \le \floor{n / 2}$, то $f(n, r, 1) = C_{n - 1}^{r - 1}$.
\end{theorem}

\begin{proof}
	Рассмотрим произвольный гиперграф $H = (V, E)$, удовлетворяющий условиям $f(n, r, 1)$. Пусть $\cF = \{F_1, \ldots, F_t\}$ --- множество из произвольных $t$ рёбер этого графа. Дополнительно введём $\cA = \{A_1, \ldots, A_n\}$ --- рёбра гиперграфа, которые получаются сдвигами $A_1 = \{1, \ldots, r\}$ по кругу ($A_2 = \{2, \ldots, r, r + 1\}$, $A_n = \{n, 1, \ldots, r - 1\}$)
	
	\textcolor{red}{Тут можно сделать красивый рисуночек, а пока можно посмотреть картинку на 15й лекции, 30:20}
	
	\begin{lemma}
		Верно неравенство $|\cA \cap \cF| \le r$
	\end{lemma}

	\begin{proof}
		Произведём разбор случаев:
		\begin{enumerate}
			\item $\cA \cap \cF = \emptyset$ --- сразу всё очевидно
			
			\item $|\cA \cap \cF| \ge 1$ Тогда, без ограничения общности, можно считать $A_1 \in \cF$ (иначе можно их просто перенумеровать). Какие рёбра из $\cA$ могут теоретически принадлежать $\cF$? Мы запишем их симметричными парами: $(A_2, A_{n - r + 2}), (A_3, A_{n - r + 3}), \ldots, (A_r, A_{n - r + r})$. В чём смысл объединять эти рёбра в пары? А в том, что они не могут одновременно находится в $\cF$, иначе просто не будет пересечения между ними. Стало быть, мы можем взять не более $r - 1$ множества, что вместе с $A_1$ даёт требуемую оценку.
		\end{enumerate}
	\end{proof}
	Рассмотреть множество $\cA$ мы могли не только на фиксированной последовательности вершин, но и на любой их перестановке. Для $\sigma \in S_n$ мы введём $\cA_\sigma$, которое как раз означает $\cA$, но уже на другой перестановке. Понятно, что лемма работает и для $\cA_\sigma$: $|\cA_\sigma \cap \cF| \le r$.
	
	Вся теорема сводится к подсчёту двумя способами следующей суммы:
	\[
		\sum_{i = 1}^t \sum_{\sigma \in S_n} \chi(F_i, \cA_\sigma) = \sum_{\sigma \in S_n} \sum_{i = 1}^t \chi(F_i, \cA_\sigma)
	\]
	где $\chi(F_i, \cA_\sigma)$ --- это индикатор вхождения $F_i$ в $\cA_\sigma$.
	\begin{itemize}
		\item Посмотрим на правую часть. Сумма внутри --- это ничто иное как $|\cF \cap \cA_\sigma|$, которое $\le r$. Так как всего перестановок $n!$, то мы можем через правую часть равенства оценить левую:
		\[
			\sum_{i = 1}^t \sum_{\sigma \in S_n} \chi(F_i, \cA_\sigma) = \sum_{\sigma \in S_n} \sum_{i = 1}^t \chi(F_i, \cA_\sigma) \le n! \cdot r
		\]
		
		\item Осталось разобраться с левой частью. Внутренняя сумма буквально равна числу таких перестановок, что $F_i$ лежит где-то на круге. Оно может начинаться на любой из $n$ позиций для вершин круга, при этом допускается свободно переставлять как все вершины вне $F_i$, так и все вершины отдельно внутри $F_i$. Итого $n \cdot r! \cdot (n - r)!$ перестановок. В результате получаем оценку на $t$:
		\[
			t \cdot n \cdot r! \cdot (n - r)! \le n! \cdot r \Ra t \le C_{n - 1}^{r - 1}
		\]
		Этого факта в совокупности с предыдущим утверждением достаточно для обоснования теоремы.
	\end{itemize}
\end{proof}

\begin{note}
	Снова мы доказали утверждения лишь в частном случае, когда $s = 1$. Что известно и что мы можем сказать про $f(n, r, s)$ --- когда оно, например, равно $C_n^r$?
\end{note}

\begin{proposition}
	Если $n \le 2r - s$, то $f(n, r, s) = C_n^r$
\end{proposition}

\begin{proof}
	Действительно, если взять любые $r$-элементные рёбра гиперграфа $H$, удовлетворяющего начальным требованиям $f(n, r, s)$, то они обязаны уже пересекаться по $s$ вершинам, этого нам и хотелось.
\end{proof}

\begin{proposition}
	Если $n > 2r - s$, то совершенно не факт, что $f(n, r, s) = C_{n - s}^{r - s}$
\end{proposition}

\begin{proof}
	Рассмотрим контрпример для $n = 8$, $r = 4$, $s = 2$. Тогда гипотеза бы сказала, что мы должны получить $C_6^2 = 15$, однако мы можем предъявить явный контрпример с большим числом рёбер.
	
	Разобьём множество $\{1, \ldots, 8\}$ на первые и последние 4 числа. Тогда наши первые рёбра --- это такие, где 3 элемента взяты из первой половины, а последний из второй. За счёт того, что 3-х элементные подмножества множества из 4х элементов пересекаются минимум по 2м, то никаких ограничений нам больше накладывать не нужно. Но сколько же мы рёбер уже получили? $C_4^3 \cdot 4 = 16 > 15$
\end{proof}

\subsubsection*{Историческая справка}

\begin{itemize}
	\item Конструкция, при помощи которой мы построили контрпример к гипотезе, в определённом смысле является оптимальной для нахождения рёбер в этой задаче. Интересный факт состоит в том, что всё же гипотеза выполняется при $n \ge n_0(r, s)$ --- это доказали всё те же Эрдёш-Ко-Радо в 1961г.
	
	\item Метод, которым нами была доказана теорема Эрдёша-Ко-Радо, принадлежит на самом деле научном руководителю Франкла --- венгерскому математику К\'{а}тона. Этот метод с $\cA$ появился в 1972г.
	
	\item В 1978г. уже Франкл нашёл минимально возможное $n_0(r, s)$ при $s \ge 15$, от которого уже не возникает контрпримеров. Это $n_0(r, s) = (r - s + 1)(s + 1)$
	
	\item В 1980г. Уилсон убрал требование $s \ge 15$ в результате Франкла
	
	\item В 1996г. математики Алсведе и Хачатрян установили следующую теорему:
	\begin{theorem} (Алсведе-Хачатряна, без доказательства)
		Пусть $n > 2r - s$, а $t \in \N$ такое, что
		\[
			(r - s + 1)\ps{2 + \frac{s - 1}{t + 1}} < (r - s + 1)\ps{2 + \frac{s - 1}{t}}
		\]
		Тогда верно равенство:
		\[
			f(n, r, s) = |\{F \subseteq \range{n} \colon |F| = r \wedge |F \cap \range{2t + s}| \ge t + s\}|
		\]
	\end{theorem}

	\begin{note}
		Поясним происходящее в теореме. Что за множества в равенстве? На самом деле это всё та же конструкция К\'{а}тона, просто в первую половину мы взяли $2t + s$ чисел. Понятно, что $r$-элементные рёбра $F$ как раз пересекаются минимум по $s$ числам за счёт наложенного на них условия, а оставшиеся $r - s - 1$ числа как-то добираются из второй половины.
		
		Почему такое ограничение на $t$? Нужно посмотреть на то, что мы получим при самых стандартных значениях:
		\begin{itemize}
			\item $t = 0$: \((r - s + 1)(s + 1) < +\infty\)
			
			\item $t = 1$: \((r - s + 1)(s + 3) / 2 < (r - s + 1)(s + 1)\)
		\end{itemize}
		Несложно заметить всплывшее $n_0(r, s)$ из результатов Франкла-Уилсона, а $t$ по сути просто определяет интервал для любых $r, s$.
	\end{note}
\end{itemize}

\subsection{Кнезеровские графы}

\begin{definition}
	\textit{Кнезеровским графом} называется $KG_{n, r} := G(n, r, 0) = (V, E)$, то есть:
	\begin{itemize}
		\item $V = C_{\range{n}}^r$
		
		\item $\forall A, B \in V\ \ (A, B) \in E \Lra A \cap B = \emptyset$
	\end{itemize}
\end{definition}

\begin{proposition}
	$\alpha(KG_{n, r}) = f(n, r, 1)$
\end{proposition}

\begin{proof}
	Следует из определений
\end{proof}

\begin{note}
	Кнезеровские графы возникли по имени немецкого математика Мартина Кнезера, который задался в 1950г. вопросом о хроматическом числе этого графа. Гипотеза Кнезера (которая оказалась верной и доказана Ласло Ловасом) состояла в том, что
	\[
		\chi(KG_{n, r}) = n - 2r + 2,\ n \ge 2r
	\]
\end{note}

\begin{proposition}
	Простыми рассуждениями можно утверждать, что $\ceil{n / r} \le$ \\ $\chi(KG_{n, r}) \le n - 2r + 2$.
\end{proposition}

\begin{proof}~
	\begin{itemize}
		\item Для оценки снизу мы просто можем воспользоваться связью хроматического числа с числом независимости или кликовым числом графа:
		\[
			\chi(KG_{n, r}) \ge \frac{C_n^r}{\alpha(KG_{n, r})} = \frac{C_n^r}{C_{n - 1}^{r - 1}} = \frac{n}{r} \Ra \chi(KG_{n, r}) \ge \ceil{\frac{n}{r}}
		\]
		Кликовое число даст ту же самую оценку с точностью до округления.
		
		\item Покрасим все вершины, у которых минимальный элемент равен $i < n - 2r + 2$, в соответствующий цвет. Тогда для покрашенных вершин мы получили корректную раскраску, а оставшихся чисел всего $2r - 1$. Это значит, что любые вершины, чьи множества полностью лежат в оставшихся числах, обязательно пересекаются хотя бы по 1 элементу, а потому мы объявляем их всех торжественно покрашенными в последний цвет.
	\end{itemize}
\end{proof}

\begin{example}
	Есть 2 примера, реализующих наши оценки на хроматическое число:
	\begin{enumerate}
		\item Случай, когда $r = n / 2$. Тогда $KG_{n, r}$ --- это просто паросочетание. Подстановка говорит, что он красится в 2 цвета, а это действительно так.
		
		\item Случай, когда $r = 1$. Это вообще просто полный граф на $n$ вершинах.
		
		\item $KG_{5, 3}$ --- это так называемый граф Петерсена
	\end{enumerate}
\end{example}