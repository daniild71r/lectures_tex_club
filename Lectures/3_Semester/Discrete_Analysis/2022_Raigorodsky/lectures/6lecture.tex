\subsection{Модель Эрдёша-Р\'{е}ньи случайных графов}

\begin{note}
	Также эту модель называют \textit{биномиальной}. Она одна из самых старых и наиболее изученных.
\end{note}

\begin{definition}
	\textit{Моделью Эрдёша-Р\'{е}ньи} $G(n, p)$, $p \in (0; 1)$ называется вероятностное пространство $(\Omega, F, P)$, где
	\begin{itemize}
		\item \(\Omega = \{G = (\range{n}, E)\} \Ra |\Omega| = 2^{C_n^2}\)
		
		\item \(F = 2^\Omega\)
		
		\item \(\forall G = (\range{n}, E) \in \Omega \quad P(G) := P(\{G\}) = p^{|E|} \cdot (1 - p)^{C_n^2 - |E|}\)
	\end{itemize}
\end{definition}

\begin{note}
	Понятно, что $p$ символизирует вероятность того, что в граф будет взято конкретное ребро.
\end{note}

\begin{note}
	Важно отметить, что $G(n, p)$ используется за обозначение как самой модели Эрдёша-Р\'{е}ньи, так и для обозначения случайного графа в этой модели.
\end{note}

\begin{theorem}
	Пусть в $G(n, p)$, где $p = p(n)$, причём $np(n) \to 0, n \to \infty$. Тогда
	\[
		P(\text{в $G(n, p)$ есть } \triangle) = P(X \ge 1) \xrightarrow[n \to \infty]{} 0
	\]
	где $X(G)$ --- это случайная величина, равная числу треугольников в графе $G$.
\end{theorem}

\begin{proof}
	Воспользуемся классическим приёмом теории вероятностей: разобьём нашу случайную величину на сумму индикаторов:
	\[
		X = X_1 \plusdots X_{C_n^3}
	\]
	где $X_i$ --- это индикатор того, что $i$-я тройка оказалась подграфом $G$. Тогда
	\[
		\E X = \sum_{i = 1}^{C_n^3} \E X_i = C_n^3 \cdot p^3 \sim \frac{(np)^3}{6} \xrightarrow[n \to \infty]{} 0
	\]
	По неравенству Маркова отсюда моментально следует, что и $P(X \ge 1) \to 0$ при $n \to \infty$.
\end{proof}

\begin{definition}
	Если в модели $(\Omega, F, P)$ вероятность события $A$ зависит от параметра $n \in \N$, причём
	\[
		\exists \lim_{n \to \infty} P(A) = 1
	\]
	то говорят, что \textit{событие $A$ случится асимптотически почти наверное}.
\end{definition}

\begin{theorem}
	Пусть в $G(n, p)$, где $p = p(n)$, причём $np(n) \to \infty, n \to \infty$. Тогда асимптотически почти наверное в $G(n, p)$ есть треугольники.
\end{theorem}

\begin{proof}
	Пусть $X(G) = $ число треугольников в $G$. Тогда $P(X \ge 1) = 1 - P(X \le 0) = 1 - P(X = 0)$.  Провернём такой цыганский фокус:
	\[
		P(X \ge 1) = 1 - P(X \le 0) = 1 - P(-X \ge 0) = 1 - P(\E X - X \ge \E X)
	\]
	Теперь, мы можем в 2 действия получить оценку неравенством Чебышёва:
	\[
		P(\E X - X \ge \E X) \le P(|\E X - X| \ge \E X) \le \frac{DX}{(\E X)^2} \Longrightarrow P(X \ge 1) \ge 1 - \frac{DX}{(\E X)^2}
	\]
	При этом $DX = \E X^2 - (\E X)^2$, иначе говоря
	\[
		P(X \ge 1) \ge 2 - \frac{\E X^2}{(\E X)^2}
	\]
	Распишем матожидание квадрата:
	\begin{multline*}
		\E X^2 = \E (X_1 \plusdots X_{C_n^3})^2 = \E\ps{X_1^2 \plusdots X_{C_n^3}^2 + \sum_{i \neq j} X_i X_j} =
		\\
		\underbrace{\E(X_1 + \ldots + X_{C_n^3})}_{\E X} + \sum_{i \neq j} \E(X_i \cdot X_j)
	\end{multline*}
	Чтобы посчитать сумму, нужно понять, что есть 3 возможных ситуации:
	\begin{enumerate}
		\item Тройки $i, j$ пересекаются по двум элементам. Для суммы таких $i, j$ получится выражение $(C_n^3 \cdot C_3^2 \cdot C_{n - 3}^1) \cdot p^5$
		
		\item Тройки $i, j$ пересекаются по одному элементу. Получаем $(C_n^3 \cdot C_3^1 \cdot C_{n - 3}^2) \cdot p^6$
		
		\item Тройки $i, j$ не пересекаются. Тогда $(C_n^3 \cdot C_{n - 3}^3) \cdot p^6$
	\end{enumerate}
	Теперь подставим всё в дробь:
	\begin{multline*}
		\frac{\E X^2}{(\E X)^2} = \frac{\E X + 3p^5 C_n^3 C_{n - 3}^1 + 3p^6 C_n^3 C_{n - 3}^2 + p^6 C_n^3 C_{n - 3}^3}{(C_n^3)^2 p^6} =
		\\
		\frac{1}{\E X} + \frac{3C_{n - 3}^1}{pC_n^3} + \frac{3C_{n - 3}^2}{C_n^3} + \frac{C_{n - 3}^3}{C_n^3} \xrightarrow[n \to \infty]{} 1
	\end{multline*}
	Этого нам и достаточно.
\end{proof}

\begin{theorem} (без доказательства)
	Пусть в $G(n, p)$, где $p = p(n)$, причём $np \to c > 0, n \to \infty$. Если $X(G)$ --- это число треугольников в графе, то
	\[
		P(X = 0) \xrightarrow[n \to \infty]{} e^{-\frac{c^3}6}
	\]
\end{theorem}

\begin{note}
	Можно заметить, что $\E X = C_n^3 p^3 \sim \frac{(np)^3}{6} \to \frac{c^3}{6}$, и поэтому в показателе экспоненты стоит ничто иное, как предел матожидания числа треугольников.
\end{note}