\documentclass[a4paper,12pt]{article}

\documentclass[a4paper,12pt]{article}

%%% Работа с русским языком
\usepackage{cmap}					% поиск в PDF
\usepackage{mathtext} 				% русские буквы в формулах
\usepackage[T2A]{fontenc}			% кодировка
\usepackage[utf8]{inputenc}			% кодировка исходного текста
\usepackage[russian,english]{babel}	% локализация и переносы
\usepackage{indentfirst}            % красная строка в первом абзаце
\frenchspacing                      % равные пробелы между словами и предложениями

%%% Дополнительная работа с математикой
\usepackage{amsmath,amsfonts,amssymb,amsthm,mathtools} % пакеты AMS
\usepackage{icomma}                                    % "Умная" запятая

%%% Свои символы и команды
\usepackage{centernot} % центрированное зачеркивание символа
\usepackage{stmaryrd}  % некоторые спецсимволы

\usepackage{ifthen}

\renewcommand{\epsilon}{\ensuremath{\varepsilon}}
\renewcommand{\phi}{\ensuremath{\varphi}}
\renewcommand{\kappa}{\ensuremath{\varkappa}}
\renewcommand{\le}{\ensuremath{\leqslant}}
\renewcommand{\leq}{\ensuremath{\leqslant}}
\renewcommand{\ge}{\ensuremath{\geqslant}}
\renewcommand{\geq}{\ensuremath{\geqslant}}
\renewcommand{\emptyset}{\ensuremath{\varnothing}}

\DeclareMathOperator{\sgn}{sgn}
\DeclareMathOperator{\ke}{Ker}
\DeclareMathOperator{\im}{Im}
\DeclareMathOperator{\re}{Re}

\newcommand{\N}{\mathbb{N}}
\newcommand{\Z}{\mathbb{Z}}
\newcommand{\Q}{\mathbb{Q}}
\newcommand{\R}{\mathbb{R}}
\newcommand{\Cm}{\mathbb{C}}
\newcommand{\F}{\mathbb{F}}
\newcommand{\id}{\mathrm{id}}

\newcommand{\imp}[2]{
	(#1\,\,$\ra$\,\,#2)\,\,
}
\newcommand{\System}[1]{
	\left\{\begin{aligned}#1\end{aligned}\right.
}
\newcommand{\Root}[2]{
	\left\{\!\sqrt[#1]{#2}\right\}
}

\renewcommand\labelitemi{$\triangleright$}

\let\bs\backslash
\let\Lra\Leftrightarrow
\let\lra\leftrightarrow
\let\Ra\Rightarrow
\let\ra\rightarrow
\let\La\Leftarrow
\let\la\leftarrow
\let\emb\hookrightarrow

%%% Перенос знаков в формулах (по Львовскому)
\newcommand{\hm}[1]{#1\nobreak\discretionary{}{\hbox{$\mathsurround=0pt #1$}}{}}

%%% Работа с картинками
\usepackage{graphicx}    % Для вставки рисунков
\setlength\fboxsep{3pt}  % Отступ рамки \fbox{} от рисунка
\setlength\fboxrule{1pt} % Толщина линий рамки \fbox{}
\usepackage{wrapfig}     % Обтекание рисунков текстом

%%% Работа с таблицами
\usepackage{array,tabularx,tabulary,booktabs} % Дополнительная работа с таблицами
\usepackage{longtable}                        % Длинные таблицы
\usepackage{multirow}                         % Слияние строк в таблице

%%% Теоремы
\theoremstyle{plain}
\newtheorem{theorem}[equation]{Theorem}
\newtheorem{lemma}[equation]{Lemma}
\newtheorem{proposition}[equation]{Proposition}
\newtheorem*{exercise}{Exercise}
\newtheorem*{problem}{Problem}

\theoremstyle{definition}
\newtheorem{definition}[equation]{Definition}
\newtheorem*{corollary}{Corollary}
\newtheorem*{note}{Note}
\newtheorem*{reminder}{Reminder}
\newtheorem*{example}{Example}

\theoremstyle{remark}
\newtheorem*{solution}{Solution}

%%% Оформление страницы
\usepackage{extsizes}     % Возможность сделать 14-й шрифт
\usepackage{geometry}     % Простой способ задавать поля
\usepackage{setspace}     % Интерлиньяж
\usepackage{enumitem}     % Настройка окружений itemize и enumerate
\setlist{leftmargin=25pt} % Отступы в itemize и enumerate

\geometry{top=25mm}    % Поля сверху страницы
\geometry{bottom=30mm} % Поля снизу страницы
\geometry{left=20mm}   % Поля слева страницы
\geometry{right=20mm}  % Поля справа страницы

\setlength\parindent{0pt}         % Устанавливает длину красной строки 0pt
\linespread{1.3}                  % Коэффициент межстрочного интервала
%\setlength{\parskip}{0.5em}      % Вертикальный интервал между абзацами
%\setcounter{secnumdepth}{0}      % Отключение нумерации разделов
%\setcounter{section}{-1}         % Нумерация секций с нуля
\usepackage{multicol}			  % Для текста в нескольких колонках
\usepackage{soulutf8}             % Модификаторы начертания

%%% Содержаниие
\usepackage{tocloft}
\tocloftpagestyle{main}
%\setlength{\cftsecnumwidth}{2.3em}
%\renewcommand{\cftsecdotsep}{1}
%\renewcommand{\cftsecpresnum}{\hfill}
%\renewcommand{\cftsecaftersnum}{\quad}

%%% Шаблонная информация для титульного листа
\newcommand{\CourseName}{Introduction to Calculus}
\newcommand{\FullCourseNameFirstPart}{\so{INTRODUCTION TO CALCULUS}}
\newcommand{\SemesterNumber}{I}
\newcommand{\LecturerInitials}{Nikolay Gusev}
\newcommand{\CourseDate}{fall 2022}
\newcommand{\AuthorInitials}{Danil Klishch}
\newcommand{\VKLink}{https://vk.com/dan.klishch}
\newcommand{\GithubLink}{https://github.com/daniild71r/lectures_tex_club}

%%% Колонтитулы
\usepackage{titleps}
\newpagestyle{main}{
	\setheadrule{0.4pt}
	\sethead{\CourseName}{}{\hyperlink{intro}{\;Back to table of contents}}
	\setfootrule{0.4pt}                       
	\setfoot{MIPT, \CourseDate}{}{\thepage} 
}
\pagestyle{main}

%%% Нумерация уравнений
\makeatletter
\def\eqref{\@ifstar\@eqref\@@eqref}
\def\@eqref#1{\textup{\tagform@{\ref*{#1}}}}
\def\@@eqref#1{\textup{\tagform@{\ref{#1}}}}
\makeatother                      % \eqref* без гиперссылки
\numberwithin{equation}{subsection}  % Нумерация вида (номер_секции).(номер_уравнения)
\mathtoolsset{showonlyrefs=false} % Номера только у формул с \eqref{} в тексте.

%%% Гиперссылки
\usepackage{hyperref}
\usepackage[usenames,dvipsnames,svgnames,table,rgb]{xcolor}
\hypersetup{
	unicode=true,            % русские буквы в раздела PDF
	colorlinks=true,       	 % Цветные ссылки вместо ссылок в рамках
	linkcolor=black!15!blue, % Внутренние ссылки
	citecolor=green,         % Ссылки на библиографию
	filecolor=magenta,       % Ссылки на файлы
	urlcolor=NavyBlue,       % Ссылки на URL
}

%%% Графика
\usepackage{tikz}        % Графический пакет tikz
\usepackage{tikz-cd}     % Коммутативные диаграммы
\usepackage{tkz-euclide} % Геометрия
\usepackage{stackengine} % Многострочные тексты в картинках
\usetikzlibrary{angles, babel, quotes}

%%% Мои кастомные команды
\newcommand{\df}[2]{\begin{definition}\textit{#1} -- #2\end{definition}}

\newcommand{\abs}[1]{\left\lvert #1 \right\rvert}
\newcommand{\grad}{\mathrm{grad}\:}
\newcommand{\dlta}{\text{d}}
\newcommand{\thus}{\;\Rightarrow\;}
\newcommand{\isconst}{=\text{const}}

\newcommand{\defeq}{\vcentcolon=}
\newcommand{\defev}{\stackrel{\Delta}{\Longleftrightarrow}}
\newcommand{\deriv}[3][1]{%
	\ifthenelse{#1>1}{%
		\frac{\dlta^{#1} {#2}}{\dlta {#3}^{#1}}
	}{%
		\frac{\dlta {#2}}{\dlta {#3}}
	}%
}
\newcommand{\vc}[3]{
	\ensuremath{\begin{pmatrix} #1 \\ #2 \\ #3 \end{pmatrix}}
}
\newcommand{\vb}[2]{
	\ensuremath{\begin{pmatrix} #1 \\ #2 \end{pmatrix}}
}
\newcommand{\fnis}[1]{#1{:}\;}
\newcommand{\RR}{\mathbb{R}}
\newcommand{\NN}{\mathbb{N}}
\newcommand{\sconstr}{\;\vert\;}
\newcommand{\diag}{{\rm diag}}

\newcommand{\floor}[1]{\left\lfloor#1\right\rfloor}
\newcommand{\ceil}[1]{\left\lceil#1\right\rceil}


%%% Всю шаблонную информацию можно менять тут
\newcommand{\FullCourseNameFirstPart}{\so{ДИСКРЕТНЫЙ АНАЛИЗ}}
\newcommand{\FullCourseNameSecondPart}{\so{}}
\newcommand{\SemesterNumber}{III}
\newcommand{\SchoolName}{ФПМИ}
\newcommand{\TrackName}{ПМИ}
\newcommand{\LecturerInitials}{Райгородский Андрей Михайлович}
\newcommand{\AutherInitials}{Герасимов Фёдор}
\newcommand{\VKLink}{https://vk.com/gefedya}
\newcommand{\OverleafLink}{https://www.overleaf.com/read/qfpvwgrfrfkr}
\newcommand{\GithubLink}{https://github.com/MIPT-Group/Lectures_Tex_Club}

\newcommand{\CourseName}{Краткое Название} %  Используется в преамбуле для создания названия предмета в верхнем контитуле   
\newcommand{\CourseDate}{Осень 2020}           %  Используется в преамбуле для создания даты в нижнем контитуле и в title_page

%\includeonly{lectures/lect05,lectures/lect06}  % Чтобы скомпилировать только часть лекций

\begin{document}
    \begin{titlepage}
	\clearpage\thispagestyle{empty}
	\centering
	
	\textit{Федеральное государственное автономное учреждение \\
		высшего образования}
	\vspace{0.5ex}
	
	\textbf{Московский физико-технический институт
    \\
    (национальный исследовательский университет)
    \\
     КЛУБ ТЕХА ЛЕКЦИЙ}
	\vspace{20ex}
	
	\rule{\linewidth}{0.5mm}
	{\textbf{\FullCourseNameFirstPart}}
	\\
	{\textbf{\FullCourseNameSecondPart}}
	\rule{\linewidth}{0.5mm}
	
	\SemesterNumber\ СЕМЕСТР
	\\
	Физтех-школа: \textit{\SchoolName}
	\\
	Направление: \textit{\TrackName}
	\\
	Лектор: \textit{\LecturerInitials}
	\vspace{1ex}
	
	\begin{figure}[!ht]
		\centering
		\includegraphics[width=0.6\textwidth]{logo_LTC}
	\end{figure}
\begin{flushright}
	\noindent
	Автор(ы): \href{\VKLink}{\textit{\AutherInitials}}
	\\
	\href{\OverleafLink}{\textit{Проект на overleaf}}
	\\
	\href{\GithubLink}{\textit{Проект на github}} 
\end{flushright}
	
	\vfill
	Долгопрудный, \CourseDate\ год.
	\pagebreak
	
\end{titlepage}
    \newpage
    \hypertarget{intro}{}
    \tableofcontents
    \newpage
    
    \section{3 Семестр}
    \section{Матрицы}

\subsection{Матрицы. Специальные виды матриц}

\begin{definition}
	$\textit{Матрицей m$\times$n}$ называется упорядоченный набор из $m \cdot n$ чисел, записанных в таблицу, состоящую из m строк и n столбцов.
\end{definition}



\textbf{Обозначения:}
\begin{itemize}
	\item A, B - матрицы
	\item (...), ||...|| - матрицы
	\item $a_{ij}$ - элемент матрица, расположенный в i-той строке j-того столбца
\end{itemize}


\textbf{Специальные виды матриц}
\begin{itemize}
	\item $\textit{строка}$ $-$ матрица, состоящая из 1 строки и n столбцов
	\item $\textit{столбец}$ $-$ матрица, состоящая из n строк и 1 столбца
	\item $\textit{квадратная}$ $-$ матрица, в которой количество строк равняется количеству столбцов
	\item $\textit{единичная}$ $-$ матрица, элементы главной диагонали которой являются единицами, а остальные $-$ нулями, обозначается буквой E
	\item $\textit{треугольная}$ $-$ матрица, у которой элементы над (нижняя треугольная) или под главной диагональю (верхняя треугольная) являются нулями
	\item $\textit{диагональная}$ $-$ матрица, у которой все элементы кроме элементов главной диагонали являются нулями, обозначается diag
	\item $\textit{симметрическая}$ $-$ матрица, элементы которой симметричны относительно главной диагонали
	\item $\textit{кососимметрическая}$ $-$ матрица, элементы которой симметричны относительно главной диагонали, но противоположны по знаку, элементы главной диагонали $-$ нули
	\item $\textit{нулевая}$ $-$ матрица, полностью состоящая из нулей
\end{itemize}

При этом к квадратным матрицам относятся единичные, треугольные, диагональные, симметрические и кососимметрические.



\subsection{Операции над матрицами}

\begin{enumerate}
	\item A = B, если матрицы имеют одинаковые размеры и равны поэлементно
	\item Сложение $C_{m \times n} = A_{m \times n} + B_{m \times n}$ определено для матриц одного размера, при чём $c_{ij} = a_{ij} + b_{ij}$
	\item Умножение матрицы A на число $\alpha \in \R$ $B = \alpha A, b_{ij} = \alpha a_{ij}$
	\item Транспонирование матрицы $A_{m \times n}^T = B_{n \times m}$, где $b_{ij} = a_{ji}$
	
\end{enumerate}

\textit{Свойства операций:}
\begin{itemize}
	\item A + B = B + A
	\item A + (B + C) = (A + B) + C
	\item $\alpha$(A + B) = $\alpha$A + $\alpha$B
	\item ($\alpha\beta$)A = $\alpha(\beta A)$
	\item ($\alpha + \beta$)A = $\alpha A + \beta A$
	\item $A^T = A$ для симметрической матрицы
	\item $A^T = -A$ для кососимметрической матрицы
	\item $(A^T)^T = A$
	\item $(A + B)^T = A^T + B^T$
	\item $(\alpha A)^T = \alpha A^T$
\end{itemize}


\subsection{Определитель(детерминант) матрицы}

\begin{definition}
	$\textit{Определитель(детерминант) матрицы}$ $-$ функция или числовая характеристика квадратной матрицы. Обозначается как det A, |A|.
\end{definition}
	
	Определитель n-мерной матрицы вычисляется как
	\begin{enumerate}
		\item |$a_{11}$| = $a_{11}$, при n = 1
		\item 
		$\begin{vmatrix}
			a_{11} & a_{12}\\
			a_{21} & a_{22}\\
		\end{vmatrix}$ = $a_{11}a_{22}$ - $a_{12}a_{21}$, при n = 2
		\item 
		$\begin{vmatrix}
			a_{11} & a_{12} & a_{13}\\
			a_{21} & a_{22} & a_{23}\\
			a_{31} & a_{32} & a_{33}\\
		\end{vmatrix}$ = $a_{11}
		\begin{vmatrix}
			a_{22} & a_{23}\\
			a_{32} & a_{33}\\
		\end{vmatrix}$ - $a_{12}
		\begin{vmatrix}
			a_{21} & a_{23}\\
			a_{31} & a_{33}\\
		\end{vmatrix}$ + $a_{13}
		\begin{vmatrix}
			a_{21} & a_{22}\\
			a_{31} & a_{32}\\
		\end{vmatrix}$ = $a_{11}a_{22}a_{33} + a_{12}a_{23}a_{31} + a_{13}a_{21}a_{32}$ - $a_{11}a_{23}a_{32}$ - $a_{12}a_{21}a_{33}$ - $a_{13}a_{22}a_{31}$, при n = 3
	\end{enumerate}
	
	\subsection{Решение систем линейных уравнений}
	
	\begin{definition}
		$\textit{Система линейных уравнений}$ $-$ система уравнений вида\\
		$\begin{cases}
			a_{11}x_1 + ... + a_{1n}x_n = b_1\\
			...\\
			a_{m1}x_1 + ... + a_{mn}x_n = b_m\\
		\end{cases}$
	\end{definition}
	
	A = 
	$\begin{pmatrix}
		a_{11} & ... & a_{1n}\\
		... & ... & ...\\
		a_{m1} & ... & a_{mn}\\
	\end{pmatrix}$ $-$ матрица системы\\
	\newline
	b = 
	$\begin{pmatrix}
		b_{1}\\
		...\\
		b_{m}\\
	\end{pmatrix}$ $-$ столбец свободных членов\\
	\newline
	(A | b) = 
	$\begin{pmatrix}
		a_{11} & ... & a_{1n} & | b_1\\
		... & ... & ... & | ...\\
		a_{m1} & ... & a_{mn} & | b_{m}\\
	\end{pmatrix}$ $-$ расширенная матрица системы\\
	
	\newpage
	
	$\textit{Совместная}$ система имеет хотя бы одно решение, иначе система считается $\textit{несовместной}$.
	
	Система называется $\textit{однородной}$, если 
	$\begin{pmatrix}
		b_1\\
		...\\
		b_m\\
	\end{pmatrix}$ = 
	$\begin{pmatrix}
		0\\
		...\\
		0\\
	\end{pmatrix}$, иначе $\textit{неоднородной}$.
	
	\begin{theorem}
		Однородная система всегда совместна.
	\end{theorem}
	\begin{proof}
		Если $x_1 = x_2 = ... = x_n = 0$, то система имеет решение.
	\end{proof}
	
	$\textbf{Правило Крамера (для двухмерной матрицы)}$. Система
	$\begin{cases}
		a_{11}x_1 + a_{12}x_2 = b_1\\
		a_{21}x_1 + a_{22}x_2 = b_2\\
	\end{cases}$ имеет единстывенное решение $\longleftrightarrow$ det
	$\begin{pmatrix}
		a_{11} & a_{12}\\
		a_{21} & a_{22}\\
	\end{pmatrix}$ $\ne$ 0.\\
	
	Решения могут быть найдены по $\textit{формуле Крамера}$:
	
	\begin{center}
	$	x_1 =\frac{\Delta_1}{\Delta}$, $x_2 = \frac{\Delta_2}{\Delta}$, где
	\end{center}
	$\Delta$ $-$ определитель матрицы системы;\\
	$\Delta_1$ = 
	$\begin{vmatrix}
		b_1 & a_{12}\\
		b_2 & a_{22}\\
	\end{vmatrix}$;
	$\Delta_2$ = 
	$\begin{vmatrix}
		a_{11} & b_1\\
		a_{21} & b_2\\
	\end{vmatrix}$.\\
	
	\newline
	
	\textit{Свойства детерминанта:}
	\begin{itemize}
		\item det $A^T$ = det A
		\item определитель треугольной (и диагональной) матрицы равен производонию диагональных элементов
		\item det E = 1
		\item если поменять местами две строки, то детерминант умножится на -1
		\item если в матрице есть нулевая строка, то det A = 0
	\end{itemize}
		
	\subsection{Умножение матриц}
	
	Умножение определено только для матриц с количеством столбцов в первой, равным количеству строк во второй.
	
	\begin{center}
		$\begin{pmatrix}
			a_1 & ... & a_n\\
	    \end{pmatrix}\cdot 
		\begin{pmatrix*}
			b_1\\
			...\\
			b_n\\
		\end{pmatrix*}$ = 
		$\begin{pmatrix*}
			a_1 b_1 + ... + a_n b_n\\
		\end{pmatrix*}$
	\end{center}
	
	Матрица C, являющаяся результатом умножения матрицы $A_{n \times m}$ на матрицу $B_{m \times k}$, имеет размеры $n \times k$, причём $c_{ij} = \sum_{S = 1}^{m} a_{is}b_{sj}$.\\
	
	Если AB = BA, то такие матрицы A и B называются $\textit{перестановочными}$. Так, единичная матрица является перестановочной с любой другой матрицей подходящего размера.
	
	\begin{theorem}
		Если определено А(ВС), то определено и (AB)C, а результаты этих операций равны.
	\end{theorem}
	\begin{proof}
		Пусть матрицы А, В и С имеют размеры соответственно $m \times n, n \times p$ и $p \times q$. Тогда умножение для них определено и выполняется\\
		
		$\begin{cases}
			A \cdot (B \cdot C) = A \cdot (BC)_{n \times q} = (ABC)_{m \times q}\\
			(A \cdot B) \cdot C = (AB)_{m \times p} \cdot C = (ABC)_{m \times q}\\
		\end{cases}$ $\longrightarrow$ мы доказали равенство размеров.\\
		\newline
		Докажем равенство элементов.\\
		
		\newline
		
		$(AB)_{ij} = \sum_{k = 1}^{n} a_{ik}b_{kj}$, $(AB)C_{il} = \sum_{j = 1}^{p} (AB)_{ij} \cdot c_{jl} = \sum_{l = 1}^{p}(\sum_{k = 1}^{n} a_{ik}b_{k_j}) \cdot c_{jl} = \sum_{k = 1}^{n} a_{ik} \sum_{j = 1}^{p} b_{kj}c_{jl} = \sum_{k = 1}^{n} a_{ik} \cdot (BC)_{kl} = A(BC)_{il}$
	\end{proof}

    \section{Вектор}

\subsection{Определение вектора}

\begin{definition}
	\textit{Вектор} $-$ направленный отрезок, концы которого упорядочены.
\end{definition}

Вектор принято обозначать следующим образом:

\begin{itemize}
	\item $\overline{a}, \overline{AB}$ $-$ любой вектор
	\item $\overline{0}$ $-$ нулевой вектор
\end{itemize}

\begin{definition}
	Векторы называются $\textit{коллинеарными}$, если существует прямая, которой они параллельны.
\end{definition}

\begin{definition}
	Векторы называются $\textit{компланарными}$, если существует плоскость, которой они параллельны.
\end{definition}

Нулевой вектор коллинерен и компланарен любому другому вектору.

\subsection{Операции над векторами}

\begin{definition}
	Два вектора называются $\textit{равными}$, если они
	\begin{enumerate}
		\item коллинеарны
		\item одинаково направлены
		\item имеют равные длины
	\end{enumerate}
\end{definition}

Через любую точку пространства можно провести ровно 1 вектор равный данному. Нулевые векторы равны друг другу.

\begin{definition}
	\textit{Сложение} векторов выполняется по правилу параллелограмма: чтобы получить сумму двух векторов, нужно из произвольной точки отложить 2 вектора равных данному и по строить на них параллелограмм, тогда его диагональ, исходящая из начальной точки, будет равна сумме векторов.
\end{definition}

\begin{center}
	\includegraphics[width=0.4\textwidth]{images/pravilo_parallelogramma.png}
\end{center}

\begin{definition}
	\textit{Произведение вектора на число} $\overline{b} = \lambda\overline(a)$ $-$ вектор коллинеарный данному, а его модуль равен модулю данного вектора, умноженного на число. При этом
	\begin{enumerate}
		\item $\lambda = 0 \longrightarrow \overline{b} = \overline{0}$
		\item $\lambda < 0 \longrightarrow \overline{b}$ противоположно направлен $\overline{a}$
		\item $\lambda > 0 \longrightarrow \overline{b}$ сонаправлен $\overline{a}$
	\end{enumerate}
\end{definition}

\begin{definition}
	\textit{Разность векторов} определяется через операции сложения и умножения на число $\overline{a} - \overline{b} = \overline{a} + (-1)\overline{b}$.
\end{definition}

\textit{Свойства операций над векторами:}
\begin{enumerate}
	\item $\forall\overline{a},\overline{b}\tab\overline{a} + \overline{b} = \overline{b} + \overline{a}$
	\item $\forall\overline{a},\overline{b},\overline{c}\tab\overline{a} + (\overline{b} + \overline{c}) = (\overline{a} + \overline{b}) + \overline{c}$
	\item $\forall\overline{a}\tab\overline{a} + \overline{0} = \overline{a}$
	\item $\forall\overline{a}\tab\exists(-1)\overline{a} = -\overline{a}$ $-$ противоположный : $\overline{a} + (-\overline{a}) = \overline{0}$
	\item $\forall \alpha,\beta\in\R,\forall\overline{a}\tab(\alpha\beta)\overline{a}=\alpha(\beta\overline{a})$
	\item$\forall\overline{a}\tab1\cdot\overline{a}=\overline{a}$
	\item$\forall\alpha,\beta\in\R\tab(\alpha + \beta)\overline{a} = \alpha\overline{a}+\beta\overline{a}$
	\item$\forall\alpha\in\R,\forall\overline{a},\overline{b}\tab\alpha(\overline{a}+\overline{b})=\alpha\overline{a}+\alpha\overline{b}$
\end{enumerate}
\subsection{Линейная комбинация. Линейная зависимость и независимость}

\begin{definition}
	Если существует $\overline{a_1}$, ..., $\overline{a_n}$ и $\alpha_1$, ..., $\alpha_n$, то $\overline{b}=\alpha_1\overline{a_1}+...+\alpha_n\overline{a_n}$ $-$ $\textit{линейная комбинация}$. 
\end{definition}

Линейная комбинация называется $\textit{тривиальной}$, если все коэффициенты $\alpha_1=...=\alpha_n=0$, иначе $\textit{нетривиальной}$.

\begin{definition}
	Набор векторов $\overline{a_1}$, ..., $\overline{a_n}$ называют $\textit{линейно зависимым}$, если
	\begin{center}
		$\exists\alpha_1$, ..., $\alpha_n$ : $\alpha_1^2+...+\alpha_n^2 > 0$ : $\alpha_1\overline{a_1}+...+\alpha_n\overline{a_n} = \overline{0}$
	\end{center}
\end{definition}

\begin{definition}
	Набор векторов называется $\textit{линейно независимым}$, если только тривиальная комбинация дает нулевой вектор.
\end{definition}

\textit{Свойства:}

\begin{enumerate}
	\item если в наборе есть нулевой вектор, то этот набор линейно зависим
	\item если $\exists\overline{a_1}$, ..., $\overline{a_n}$ $-$ линейно зависимая комбинация, то $\forall\overline{b_1}$, ..., $\overline{b_m}}$, то набор $\overline{a_1}$, ..., $\overline{a_n}, \overline{b_1}$, ..., $\overline{b_m}$ $-$ линейно зависмый
	\item если набор линейно независим, то его любой непустой поднабор тоже линейно независим
	\begin{proof}
		Пусть этот поднабор линейно независим, но тогда по свойству 2 весь набор линейно зависм $\longrightarrow$ противоречие.
	\end{proof}
\end{enumerate}

\begin{theorem}
	Пусть $\overline{x}$ $-$ линейная комбинация, т.е. $\overline{x}=\alpha_1\overline{a_1}+...+\alpha_n\overline{a_n}$. Тогда разложение $\overline{x}$ единсьвенное $\longleftrightarrow$ $\overline{a_1}$, ..., $\overline{a_n}$ $-$ линейно независмый набор.
\end{theorem}

\begin{proof}
	\tab
	$\longrightarrow$\\
	Пусть существует
	\begin{center}
		$\overline{x}=\alpha_1\overline{a_1}+...+\alpha_n\overline{a_n}\tab(1)$\\
		$\overline{x}=\beta_1\overline{a_1}+...+\beta_n\overline{a_n}\tab(2)$
	\end{center}
	Вычтем (1) из (2) $\longrightarrow$ существует $\alpha_i\neq\beta_i$ $\longrightarrow$ нетривиальная $\longrightarrow$ противоречие.
	
	$\tab\longleftarrow$\\
	Пусть $\overline{a_1}$, ..., $\overline{a_n}$ - линейно зависим $\longrightarrow$ $\exists$нетривиальная линейная комбинация $\gamma_1\overline{a_1}$, ..., $\gamma_n\overline{a_n} = \overline{0}$\\
	
	$\overline{x} = \overline{x} + \overline{0} = (\alpha_1 + \gamma_1)\overline{a_1}+...+(\alpha_n + \gamma_n)\overline{a_n}$ $\longrightarrow$ разложение не единственно $\longrightarrow$ противоречие.
\end{proof}

\begin{theorem}[Критерий линейной зависимости]
	Набор из k > 1 элементов линейно зависим $\longleftrightarrow$ существует вектор набора, равный линейной комбинации остальных.
\end{theorem}
\begin{proof}
	$\tab\longrightarrow$\\
	Предположим, что $\overline{a_1}$, ..., $\overline{a_k}$ $-$ линейно зависима, тогда $\exists\alpha_1$, ..., $\alpha_k$ :  $\alpha_1^2+...+\alpha_k^2$ > 0, $\alpha_1\overline{a_1}+...+\alpha_k\overline{a_k}=\overline{0}$.\\
	Пусть $\alpha_1\neq0$ $\longrightarrow$ $\overline{a_1} = -\frac{\alpha_2}{\alpha_1}\overline{a_2}-...-\frac{\alpha_k}{\alpha_1}\overline{a_2}$ $-$ вектор $\overline{a_1}$ раскладывается по остальным векторам.
	
	$\tab\longleftarrow$\\
	Пусть $\overline{a_1} = p_2\overline{a_2}+...+p_k\overline{a_k}$, тогда коэффициент при $\overline{a_1}$ = 1 $\longrightarrow$ $1\cdot\overline{a_1}+(-p_2)\overline{a_2}+...+(-p_k)\overline{a_k}=\overline{0}$ $-$ нетривиальная комбинация $\longrightarrow$ линейно зависимая.
\end{proof}

\begin{corollary}
	Рассмотрим следствия 1 - 3 без доказательства, т.к. они очевидны.\\
	\begin{enumerate}
		\item нетривиальная комбинация векторов линейно независимого набора всегда не равна $\overline{0}$
		\item система из одно вектора линейно зависима $\longleftrightarrow$ этот вектор нулевой
		\item система из двух векторов линейно зависима $\longleftrightarrow$ эти векторв коллинеарны
		\item система из трех векторов линейно зависима $\longleftrightarrow$ эти векторы компланарны
		\begin{proof}
			Пусть $\overline{a}, \overline{b}, \overline{c}$ $-$ компланарны.
			\begin{enumerate}
				\item Пусть $\overline{a}$ и $\overline{b}$ $-$ коллинеарны $\longrightarrow$ из следствия 3 они линейно зависимы, а значит весь набор линейно зависим.
				\item Пусть эти векторы неколлинеарны, тогда по правилу параллелограмма $\exists\alpha, \beta$ : $\alpha\overline{a}+\beta\overline{b}=\overline{c}$
			\end{enumerate}
		\end{proof}
		
		\item система из четырех и более векторов линейно зависима всегда
		\begin{proof} Пусть даны векторы $\overline{a}, \overline{b}, \overline{c}, \overline{d}$.
			\begin{enumerate}
				\item Если есть нулевой вектор, то набор линейно завсимый
				\item если все векторы ненулевые
				\begin{enumerate}
					\item Если $\overline{a}, \overline{b}, \overline{c}$ $-$ компланарны, то из следствия 4 вся система линейно зависима
					\item Если $\overline{a}, \overline{b}, \overline{c}$ $-$ некомпланарны достаточно доказать, что $\overline{d}$ является их линейной комбинацией.\\
					Для этого построим плоскость, проходящую через векторы Если $\overline{a}, \overline{b}$C. Проведем через $\overline{d}$ вектор, параллельный $\overline{c}$ $-$ $\overline{c_1}$.
					\begin{center}
						\includegraphics[width=0.4\textwidth]{images/5.jpeg}
					\end{center}
				\end{enumerate}
				$\overline{f}$ $-$ вектор к точке пересечения $\overline{c_1}$ с плоскостью. Т.к. Если $\overline{a}$ и $\overline{b}$ неколлинеарные, $\overline{f}$ можно разложить по ним:\\
				$\begin{cases}
					\overline{f} = \alpha\overline{a}+\beta\overline{b}\\
					\overline{c_1}=\gamma\overline{c}\\
				\end{cases}$ $\longrightarrow$ $\overline{d}=\overline{f}+\overline{c_1}= \alpha\overline{a}+\beta\overline{b}+\gamma\overline{c}$, значит $\overline{d}$ раскладывается по остальным векторам и система линейно зависимая.
			\end{enumerate}
		\end{proof}
		
	\end{enumerate}
\end{corollary}

    \section{Базис векторного пространства. Системы координат.}

\subsection{Пространства}

\begin{definition}
	Множество назовем $\textit{замкнутым}$ относительно операции, если результат применения этой операции принадлежит этому множеству.
\end{definition}

\begin{definition}
	Непустое множество, замкнутое относительно линейных операций (сложение и умножение на число), назовем $\textit{векторным пространством}$.
\end{definition}

Если одно векторное пространство является подмножеством другого, будем называть его $\textit{подпространством}$.

Само множество является своим подмножеством $\longrightarrow$ пространство является своим подпространством.


\subsection{Базис в векторном пространстве}

\begin{definition}
	$\textit{Базис в векторном пространстве}$ $-$ упорядоченный линейно независимый набор векторов, такой, что любой вектор пространства раскладывается по этим векторам.
\end{definition}

В нулевом пространстве базиса нет.

В одномерном пространстве базис $-$ любой ненулевой вектор.

В двумерном пространстве базис $-$ упорядоченная пара неколлинеарных векторов.

В трехмерном пространстве базис $-$ упорядоченная тройка некомпланарных векторов.\\

Любой вектор раскладывается по векторам базиса, и это разложение единственно.\\

Пусть существует базис ($\overline{e_1}, \overline{e_2}, \overline{e_3}$), такой что любой вектор в этом базисе $\overline{a} = x_1\overline{e_1} + x_2\overline{e_2} + x_3\overline{e_3}$. Тогда коэффициенты разложения $x_1$, $x_2$, $x_3$ определяются единственным образом и называются $\textit{координатами}$.\\

Пусть в этом же базисе существуют вектора $\overline{a}(a_1, a_2, a_3), \overline{b}(b_1, b_2, b_3), \overline{c}(c_1, c_2, c_3)$. Узнаем, является ли этот набор линейно зависимым. Для этого необходимо узнать, существует ли их нетривиальная комбинация:

\begin{center}
	$\alpha\overline{a} + \beta\overline{b} + \gamma\overline{c} = \overline{0}\tab\longleftrightarrow$\\
	
	$\alpha
	\begin{pmatrix*}
		a_1\\
		a_2\\
		a_3\\
	\end{pmatrix*} + \beta
	\begin{pmatrix*}
		b_1\\
		b_2\\
		b_3\\
	\end{pmatrix*} + \gamma
	\begin{pmatrix*}
		c_1\\
		c_2\\
		c_3\\
	\end{pmatrix*} = 
	\begin{pmatrix*}
		0\\
		0\\
		0\\
	\end{pmatrix*}$
\end{center} $\longrightarrow$ по теореме Крамера система имеет единственное решение, отличное от тривиальной комбинации, при
\begin{center}
	det = 
	$\begin{vmatrix}
		a_1 & b_1 & c_1\\
		a_2 & b_2 & c_2\\
		a_3 & b_3 & c_3\\
	\end{vmatrix} \longrightarrow$ если det = 0, система линейно зависима, иначе $-$ линейно независима.
	
\end{center}

Аналогично для плоскости.

\subsection{Декартова и некоторые другие системы координат}

\begin{wrapfigure}{l}{0.4\textwidth}
	\includegraphics[width=0.5\linewidth]{images/дск.jpeg}
\end{wrapfigure}

\begin{definition}
	\textit{Декартова система координат} $-$ совокупность точки и базиса.
\end{definition}

\tab\\ \tab\\
O $-$ начало координат\\
$\overline{e_1}, \overline{e_2}, \overline{e_3}$ $-$ базис\\
$\overline{OM}$ = $x_1\overline{e_1} + x_2\overline{e_2} + x_3\overline{e_3}$ $\longrightarrow$ $x_1, x_2, x_3$ $-$ координаты.

\tab\\ \tab\\ \tab\\ \tab\\ \tab\\

\textbf{Медианный вектор}\\

\begin{wrapfigure}{r}{0.4\textwidth}
	\includegraphics[width=0.8\linewidth]{images/медианныйвектор.jpeg}
\end{wrapfigure}

Пусть $\overline{a}, \overline{b}$ $-$ образующие векторы, $\overline{c}$ $-$ медианный. Тогда $\overline{c}$ = $\frac{\overline{a} + \overline{b}}{2}$.\\

Аналогично можно найти координаты середины отрезка, с концами в координатах ($a_1, a_2, a_3$) и ($b_1, b_2, b_3$), M = ($\frac{a_1 - b_1}{2}, \frac{a_2 - b_2}{2}, \frac{a_3 - b_3}{2}$)

\tab\\ \tab\\

\textbf{Вектор, коллинеарный биссектрисе угла}

\begin{wrapfigure}{l}{0.4\textwidth}
	\includegraphics[width=0.6\linewidth]{images/биссектриса.jpeg}
\end{wrapfigure}

\tab\\ \tab\\

Пусть $\overline{d}$ $-$ вектор, коллинеарный биссектрисе угла, построенного на векторах $\overline{a}$ и $\overline{b}$.\\
 
$\overline{d}$ = $\frac{\overline{a}}{|\overline{a}|} + \frac{\overline{b}}{|\overline{b}|}$

\tab\\ \tab\\ \tab\\ \tab\\
\textbf{Вектор, который делит прямую в соотношении m:n}

\begin{wrapfigure}{r}{0.4\textwidth}
	\includegraphics[width=0.6\linewidth]{images/соотношение.jpeg}
\end{wrapfigure}

\tab\\

$\overline{c}$ = $\frac{n}{m + n}\overline{a} + \frac{m}{m + n}\overline{b}$

\tab\\ \tab\\ \tab\\ \tab\\

\begin{definition}
	Если базисные векторы попарно ортогональны, то базис \textit{ортогональный}.
\end{definition}

\begin{definition}
	Если базисные векторы попарно ортогональны, а их длины равны 1, то базис \textit{ортонормированный} (ОНБ).
\end{definition}

\begin{definition}
	Система координат с ОНБ $-$ \textit{декартова прямоугольная система координат}.
\end{definition}

\textbf{Полярная система координат}\\
\begin{wrapfigure}{l}{0.4\textwidth}
	\includegraphics[width=0.6\linewidth]{images/полярнаяск.jpeg}
\end{wrapfigure}

Координаты задаются длиной радиус-вектора $\overline{r}$ и полярным углом $\phi$.\\

$\overline{r}(|\overline{r}|, \phi)$

\tab\\ \tab\\ \tab\\

\textbf{Цилиндрическая система координат}\\
\begin{wrapfigure}{l}{0.4\textwidth}
	\includegraphics[width=0.6\linewidth]{images/цилиндрическаяск.jpeg}
\end{wrapfigure}

Координаты задаются длиной радиус-вектора $\overline{r}$, полярным углом $\phi$ и смещением относительно оси z.\\

$\overline{r}(|\overline{r}|, \phi, h)$

\tab\\ \tab\\ \tab\\ \tab\\ \tab\\ \tab\\
\textbf{Сферическая система координат}\\
\begin{wrapfigure}{l}{0.4\textwidth}
	\includegraphics[width=0.5\linewidth]{images/сферическая.jpeg}
\end{wrapfigure}

Координаты задаются длиной радиус-вектора $\overline{r}$, азимутальным углом $\phi\in[0, 2\pi]$ и зенитным углом $\theta\in[-\frac{\pi}{2}, \frac{\pi}{2}]$.\\

$\overline{r}(|\overline{r}|, \theta, \phi)$
\tab\\ \tab\\ \tab\\ \tab\\ \tab\\
\newpage

\subsection{Переход в новую систему координат}

Пусть имеются две системы координат и вектор $\overline{a}$:\\
\tab\\
$\tab$($\overline{e_1}, \overline{e_2}, \overline{e_3})$ $-$ "старая" система координат, $\overline{a}$(x, y, z)\\
$\tab$($\overline{e_1'}, \overline{e_2'}, \overline{e_3'})$ $-$ "новая" система координат, $\overline{a}$(x', y', z')\\

Тогда базисные векторы новой системы координат выражаются следующим образом:\\
\tab\\
$\overline{e_1'}$ = $a_{11}\overline{e_1} + a_{21}\overline{e_2} + a_{31}\overline{e_3}$\\
$\overline{e_2'}$ = $a_{12}\overline{e_1} + a_{22}\overline{e_2} + a_{32}\overline{e_3}$\\
$\overline{e_3'}$ = $a_{13}\overline{e_1} + a_{23}\overline{e_2} + a_{33}\overline{e_3}$\\

В новой системе координат $\overline{a}$ = $x'\overline{e_1'} + y'\overline{e_2'} + z'\overline{e_3'}$ = $x'(a_{11}\overline{e_1} + a_{21}\overline{e_2} + a_{31}\overline{e_3}) + y'(a_{12}\overline{e_1} + a_{22}\overline{e_2} + a_{32}\overline{e_3}) + z'(a_{13}\overline{e_1} + a_{23}\overline{e_2} + a_{33}\overline{e_3})$ = $\overline{e_1}(a_{11}x' + a_{12}y' + a_{13}z') + \overline{e_2}(a_{21}x' + a_{22}y' + a_{23}z') + \overline{e_3}(a_{31}x' + a_{32}y' + a_{33}z')\tab\longrightarrow$\\
\tab\\
x = $a_{11}x' + a_{12}y' + a_{13}z'$\\
y = $a_{21}x' + a_{22}y' + a_{23}z'$\\
z = $a_{31}x' + a_{32}y' + a_{33}z'$\\

\begin{definition}
	\textit{Матрица перехода} от базиса e к базису e' имеет вид\\
	\begin{center}
		S =
		$\begin{pmatrix}
			a_{11} & a_{12} & a_{13}\\
			a_{21} & a_{22} & a_{23}\\
			a_{31} & a_{32} & a_{33}\\
		\end{pmatrix}$
	\end{center} 
\end{definition}

Столбцами S являются координаты новых базисных векторов в старом базисе.\\

\begin{center}
	$\begin{pmatrix*}
		x\\
		y\\
		z\\
	\end{pmatrix*}$ = S
	$\begin{pmatrix*}
		x'\\
		y'\\
		z'\\
	\end{pmatrix*}$, $\tab$ ($\overline{e_1'}, \overline{e_2'}, \overline{e_3'}$) = ($\overline{e_1}, \overline{e_2}, \overline{e_3}$)S, $\tab$ e' = eS
\end{center}

\textbf{Основное свойство матрицы перехода:} по теореме о линейной независимости, det S $\neq$ 0.

Рассмотрим старое начало координат О и новое О'. Тогда вектор, соединяющий их\\

\begin{wrapfigure}{l}{0.4\textwidth}
	\includegraphics[width=0.6\linewidth]{images/переход1.jpeg}
\end{wrapfigure}

$\overline{OO'}$ = $\beta_1\overline{e_1} + \beta_2\overline{e_2} + \beta_3\overline{e_3}$ \\

Пусть существует произвольная точка М, такая что $\overline{OM}$ = $\overline{OO'} + \overline{O'M}$. Тогда ее координаты\\

x = $a_{11}x' + a_{12}y' + a_{13}z' + \beta_1$\\
y = $a_{21}x' + a_{22}y' + a_{23}z' + \beta_2$\\
z = $a_{31}x' + a_{32}y' + a_{33}z' + \beta_3$\\

    \subsubsection{Последовательности де Брёйна}

\begin{Def}
	\emph{Последовательностью де Брёйна} для числа $ n $ называется последовательность $ a_1, a_2, \ldots a_N, \;a_i \in \{0, 1\} $, так что все среди подслова длины $ n $ встречаются все различные слова длины $ n $ над афлавитом $ \{0, 1\} $.
\end{Def}

Поскольку число слов длины $ n $ над афлавитом $ \{0, 1\} $ равно $ 2^n $, то $ N = 2^n + n - 1 $.

Перейдем к доказательству существования для произвольного $ n $ и построению последовательности де Брейна. Приведем два способа построения.

\begin{enumerate} \renewcommand{\theenumi}{\bfseries \arabic{enumi}}
	\item {\bfseries Граф де Брейна} Первый способ построения заключается в построении графа и нахождения в нем Эйлерова цикла.
	\begin{Def}
		\emph{Графом Де Брёйна} называется ориентированный граф на множестве слов длины $ n-1 $ над алфавитом $ \{0, 1\} $, т.ч. есть ребро из слова $ u $ в слово $ v $, когда существует слово  длины $ n $, такое что его префикс - это $ u $, а суффикс - $ v $.
	\end{Def}

	\begin{prop}
		Рассмотренный граф Эйлеров.
	\end{prop}

	\begin{proof}
		Необходимо показать, что для каждой вершины $u$, соответствующей слову $ b_1\ldots b_{n-1} $, $ indeg(u) = outdeg(u) $, откуда по критерию Эйлеровости получим, что он Эйлеров. Слово $ u $ можно дополнить справа 0 или 1, поэтому из него выходит два ребра (возможно, одно из них идет в себя). По тем же соображениям в него входит 2 ребра. Следовательно, утверждения доказано.
		%$$ \begin{cases}
		%		b_2\ldots b_{n-1} 0 \neq b_1\ldots b_{n-1} \\
		%		b_2\ldots b_{n-1} 1 \neq b_1\ldots b_{n-1}
		%\end{cases}  $$
	\end{proof}

	\begin{prop}
		Произвольный Эйлеров цикл в рассмотренном графе задает правильную последовательность, а именно, если выписать слово, соответствующее стартовой вершине, а на каждом переходе в новую вершину писать его последнюю букву, то получится последовательность де Брёйна.
	\end{prop}

	\begin{proof}
		Слово, соответствующее каждой из вершин графа, встретится в искомой последовательности. Действительно, на первом шаге мы выписали слово, соответствующее начальной вершине, а на очередном выписывании буквы мы получали префикс, равный слову, соответствующему вершине, в которую мы перешли. Поскольку длина полученной последовательности равна $ (n - 1) + 2^n = N$, то никакое слово не встретится дважды.
	\end{proof}

	\begin{note}
		\href{https://mipt.lectoriy.ru/file/synopsis/pdf/Maths-DiscreteMathem-M05-Raygor-131016.03.pdf}{Здесь} вы можете найти наглядный пример построения и обхода графа для $ n = 3 $.
	\end{note}

	\item {\bfseries Правило "0 лучше 1".} Второй алгоритм значительно проще реализуется - но сложнее доказывается.
	Вначале выставим $ n $ единиц.
\end{enumerate}

    
    %\input{preamble_and_reference/additional_info}
\end{document}