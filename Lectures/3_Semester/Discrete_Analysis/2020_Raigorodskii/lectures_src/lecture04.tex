\subsubsection{Последовательности де Брёйна}

\begin{Def}
	\emph{Последовательностью де Брёйна} для числа $ n $ называется последовательность $ a_1, a_2, \ldots a_N, \;a_i \in \{0, 1\} $, так что все среди подслова длины $ n $ встречаются все различные слова длины $ n $ над афлавитом $ \{0, 1\} $.
\end{Def}

Поскольку число слов длины $ n $ над афлавитом $ \{0, 1\} $ равно $ 2^n $, то $ N = 2^n + n - 1 $.

Перейдем к доказательству существования для произвольного $ n $ и построению последовательности де Брейна. Приведем два способа построения.

\begin{enumerate} \renewcommand{\theenumi}{\bfseries \arabic{enumi}}
	\item {\bfseries Граф де Брейна} Первый способ построения заключается в построении графа и нахождения в нем Эйлерова цикла.
	\begin{Def}
		\emph{Графом Де Брёйна} называется ориентированный граф на множестве слов длины $ n-1 $ над алфавитом $ \{0, 1\} $, т.ч. есть ребро из слова $ u $ в слово $ v $, когда существует слово  длины $ n $, такое что его префикс - это $ u $, а суффикс - $ v $.
	\end{Def}

	\begin{prop}
		Рассмотренный граф Эйлеров.
	\end{prop}

	\begin{proof}
		Необходимо показать, что для каждой вершины $u$, соответствующей слову $ b_1\ldots b_{n-1} $, $ indeg(u) = outdeg(u) $, откуда по критерию Эйлеровости получим, что он Эйлеров. Слово $ u $ можно дополнить справа 0 или 1, поэтому из него выходит два ребра (возможно, одно из них идет в себя). По тем же соображениям в него входит 2 ребра. Следовательно, утверждения доказано.
		%$$ \begin{cases}
		%		b_2\ldots b_{n-1} 0 \neq b_1\ldots b_{n-1} \\
		%		b_2\ldots b_{n-1} 1 \neq b_1\ldots b_{n-1}
		%\end{cases}  $$
	\end{proof}

	\begin{prop}
		Произвольный Эйлеров цикл в рассмотренном графе задает правильную последовательность, а именно, если выписать слово, соответствующее стартовой вершине, а на каждом переходе в новую вершину писать его последнюю букву, то получится последовательность де Брёйна.
	\end{prop}

	\begin{proof}
		Слово, соответствующее каждой из вершин графа, встретится в искомой последовательности. Действительно, на первом шаге мы выписали слово, соответствующее начальной вершине, а на очередном выписывании буквы мы получали префикс, равный слову, соответствующему вершине, в которую мы перешли. Поскольку длина полученной последовательности равна $ (n - 1) + 2^n = N$, то никакое слово не встретится дважды.
	\end{proof}

	\begin{note}
		\href{https://mipt.lectoriy.ru/file/synopsis/pdf/Maths-DiscreteMathem-M05-Raygor-131016.03.pdf}{Здесь} вы можете найти наглядный пример построения и обхода графа для $ n = 3 $.
	\end{note}

	\item {\bfseries Правило "0 лучше 1".} Второй алгоритм значительно проще реализуется - но сложнее доказывается.
	Вначале выставим $ n $ единиц.
\end{enumerate}
