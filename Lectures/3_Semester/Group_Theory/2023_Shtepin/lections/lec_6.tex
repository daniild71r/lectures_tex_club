%12.10.23

\subsection{Лемма Бернсайда}
\begin{definition}
    Пусть $I: G \to S(\Omega)$. Действие $I$ называется транзитивным, если оно имеет только одну орбиту, то есть $\forall \omega_1, \omega_2 \exists a \in G: a(\omega_1) = \omega_2$
\end{definition}

\begin{theorem}[Лемма Бернсайда]
    Пусть $G$ -- конечная группа, $\Omega$ -- конечная, $I: G \to S(\Omega)$ транзитивно и пусть для $a \in G$ $N(a) = \{ \omega \in \Omega | a(\omega) = \omega \}$ -- число неподвижных точек в $\Omega$ под действием $I_a$. Тогда $\sum\limits_{a \in G} N(a) = |G|$.
\end{theorem}

\begin{proof}
    Рассмотрим стабилизатор $\omega$: $\stationar(\omega) = \{ a \in G | a \omega = \omega \}$, тогда количество неподвижных точек будет равно $|\stationar(\omega)| = \frac{|G|}{|\Omega|}$. Назовем пару $(a, \omega) \in G \times \Omega$ допустимой, если $a(\omega) = \omega$. Путсь $\Delta$ -- множество всех допустимых пар, тогда с одной стороны
    $$|\Delta| = \sum\limits_{\omega \in \Omega} |\stationar(\omega)| = \sum\limits_{\omega \in \Omega} \frac{|G|}{|\Omega|} = \frac{|G|}{|\Omega|} \cdot |\Omega| = |G|$$
    С другой же стороны 
    $$|\Delta| = \sum\limits_{a \in G} N(a) = |G|$$
\end{proof}

\begin{corollary}[О среднем числе неподвижных точек]
    Пусть $G$ -- конечная группа и она действует на конечном множестве $\Omega$. Тогда $|\Omega/G| = \frac{1}{|G|} \sum\limits_{a \in G} N(a)$. Иначе говоря, количество орбит равно среднему чсилу неподвижных точек.
\end{corollary}

\begin{proof}
    Пусть $\Omega_1, \dots, \Omega_s$ -- все различные орбиты группы $G$ при действии на $\Omega$. \\
    $\Omega = \Omega_1 \bigsqcup \dots \bigsqcup \Omega_s$, $I: G \to S(\Omega_i)$ -- транзитивно для любого $i$. Тогда для каждой орбиты запишем $\sum\limits_{a \in G} N_i(a) = |G|$, итого
    $$\sum\limits_{i = 1}^s\sum\limits_{a \in G} N_i(a) = \sum\limits_{a \in G} N(a) = s|G| \implies s = \frac{\sum\limits_{a \in G} N(a)}{|G|}$$
\end{proof}

\section{Прямое произведение групп}
\begin{definition}
    Пусть $A, B$ -- группы относительно <<$\cdot$>>. Внешним прямым произведением групп $A$ и $B$ называется множество упорядоченных пар $(a, b), a \in A, b \in B$ с операцией <<$\cdot$>>:
    $$(a_1, b_1) \cdot (a_2, b_2) = (a_1a_2, b_1b_2)$$
\end{definition}

\begin{note}
    Покажем, что $A \times B$ -- группа: $(e_1, e_2)$ -- нейтральный элемент и $(a, b)^{-1} = (a^{-1}, b^{-1})$
\end{note}

\begin{proposition}
\label{pr6.1}
    Если $A \times B$ -- внешнее прямое произведение, то 
    \begin{enumerate}
        \item В группе $A \times B$ есть подгруппы $A \times \{ e \} \cong A, \{ e \} \times B \cong B$.
        \item $A \times \{ e \} \lhd A \times B$ и $\{ e \} \times B \lhd A \times B$.
        \item Элементы подгрупп $A \times \{ e \}$ и $\{ e \} \times B$ коммутируют.
        \item $(A \times \{ e \}) \cdot (\{ e \} \times B) = A \times B$ -- произведение подгрупп.
        \item $A \times B \cong B \times A$ (коммутативность).
        \item $(A \times B) \times C \cong A \times (B \times C)$
    \end{enumerate}
\end{proposition}

\begin{proof}
    \item Построим явно $(a, e) \to a$ -- изоморфизм. Для группы $B$ изоморфизм аналогичный.
    \item Рассмотрим $(a, b) \in A \times B$, $(a', e) \in A \times e$, тогда распишем произведение $(a, b)(a', e)(a^{-1}, b^{-1}) = (aa'a^{-1}, e) \in A \times e$.
    \item $(a, e)(e, b) = (a, b) = (e, b)(a, e)$ -- коммутативность очевидна.
    \item $(a, e)(e, b) = (a, b)$.
    \item Построим изоморфизм: $(a, b) \to (b, a)$.
    \item Опять же изоморфизм: $((a, b), c) \to (a, (b, c))$.
\end{proof}

\begin{proposition}
    Пусть $A_1 \leq A, B_1 \leq B$, тогда $A_1 \times B_1 \leq A \times B$. Более того, если эти подгруппы нормальные $A_1 \lhd A, B_1 \lhd B$, то $A_1 \times B_1 \lhd A \times B$. В этом случае имеет место изоморфизм $(A \times B)/(A_1 \times B_1) \cong (A/A_1) \times (B/B_1)$.
\end{proposition}

\begin{proof}
    Первая часть утверждения проверяется по критерию подгрупп (если элементы лежали в подгруппе, то там лежит и пара, существование обратного аналогично). \\
    Проверим $A_1 \times B_1 \lhd A \times B$. Пусть $(a', b') \in A_1 \times B_1$, $(x, y) \in A \times B$, тогда $(x, y)^{-1}(a', b')(x, y) = (x^{-1}a'z, y^{-1}b'y) \in A_1 \times B_1$. \\
    Теперь построим гомоморфизм вида $\phi: A \times B \to (A/A_1) \times (B/B_1). \phi: (a, b) \to (aA_1, bB_1)$ -- он сюръективен, так как любой паре справа можно сопоставить пару слева. Рассмотрим ядро этого гомоморфизма: $\ker \phi = A_1 \times B_1$(нормальная подгруппа по доказанному выше). Тогда по основной теореме о гомоморфизме \ref{th3.1} $(A/A_1) \times (B/B_1) \cong A \times B/ A_1 \times B_1$ и указанный изоморфизм существует.
\end{proof}

\begin{corollary}
    $A \times B/A \cong B, A \times B/B \cong A$
\end{corollary}

\begin{proof}
    Пусть $A_1 = A, B_1 = \{ e \}$, тогда $A/A_1 = \{ e \}$, $B/B_1 = B/\{ e \} \cong B$. По утверждению \ref{pr6.1} получаем $A_1 \times B_1 = A \times \{ e \} \cong A, B \cong \{ e \} \times B \cong A \times B/A$.
\end{proof}

\begin{theorem}[о разложении группы в прямое произведение нормальных подгрупп]~
\label{th6.1}
    $G$ -- группа, $A, B$ -- тривиально пересекающиеся нормальные подгруппы в $G$ $(A \cap B = \{ e \})$ и пусть $A \cdot B = G$. Тогда $G \cong A \times B$ (группа изоморфна внешнему прямому произведению подгрупп $A$ и $B$).
\end{theorem}

\begin{proof}
    \begin{enumerate}
        \item Покажем, что $\forall a \in A$ и $\forall b \in B: ab = ba$.
        Так как $(aba^{-1})b^{-1} = a(ba^{-1}b^{-1})$, где левая часть по критерию нормальности принадлежит $B$, а правая -- аналогично принадлежит $A$, то есть $aba^{-1}b^{-1} = e$ -- лежит в центре, отсюда следует коммутативность (домножение на $ba$ справа).
        \item Теперь проверим изоморфизм. Рассмотрим $\phi: A \times B \to G, \phi(a, b) = a \cdot b$. Проверим, что это гомоморфизм $\phi(a_1, b_1)\phi(a_2, b_2) = a_1b_1a_2b_2 = a_1a_2b_1b_2 = \phi (a_1a_2, b_1b_2)$. Очведино, что $\phi$ -- сюръективно, так как $\forall g \in G, g = ab, \phi(a, b) = ab = g$. Рассмотрим ядро $\ker \phi = \{ (a, b) | ab = e \}$, $a = b^{-1} \implies a \in A \cap B \implies a = e, b = e \implies \ker \phi = \{ e \}$. Значит, по основной теореме о гомоморфизмах \ref{th3.1} получаем $G/ \{ e \} \cong G \cong A \times B$.
    \end{enumerate}
\end{proof}

\begin{definition}
    Группа $G$, удовлетворяющая условиям теоремы \ref{th6.1}, называется внутренним прямым произведением подгрупп $A$ и $B$.
\end{definition}

\begin{corollary}
    Внешнее прямое произведение изоморфно внутреннему прямому произведению.
\end{corollary}

\subsection{Полупрямое произведение}

$A, B$ -- группы с операцией <<$\cdot$>>. Пусть $I: B \to \Aut A$ -- гомоморфизм (то есть $B$ действует на группу $A$ автоморфизмами, то есть $I_b \in \Aut A$).
\begin{definition}
    Полупрямым произведением групп $A$ и $B$, обозначаемым $A \rtimes_I B$, называется множество всех упорядоченных пар $(a, b) \in A \times B$ с операцией умножения:
    $$(a_1, b_1) \cdot (a_2, b_2) = (a_1 \cdot I_{b_1}(a_2), b_1b_2)$$
\end{definition}

\begin{proof}
    Докажем корректность.
    \begin{enumerate}
        \item Ассоциативность: 
        $$((a_1, b_1)(a_2, b_2))(a_3, b_3) = (a_1 I_{b_1}(a_2), b_1b_2)(a_3, b_3) = (a_1I_{b_1}(a_2)I_{b_1b_2}(a_3), b_1b_2b_3)$$
        С другой стороны: 
        \begin{equation}
            \begin{gathered}
                (a_1, b_1)((a_2, b_2)(a_3, b_3)) = (a_1, b_1)(a_2I_{b_2}(a_3), b_2b_3) = (a_1I_{b_1}(a_2I_{b_2}(a_3)), b_1b_2b_3) = \\
                = (a_1I_{b_1}(a_2)I_{b_1b_2}(a_3), b_1b_2b_3)
            \end{gathered}
        \end{equation}
        Из двух равенств вытекает ассоциативность.
        \item $(e, e)$ -- нейтральный.
        \item Обратный: 
        $(a, b)^{-1} = (I_{b^{-1}}(a^{-1}), b^{-1})$. Проверим: $$(I_{b^{-1}}(a^{-1}), b^{-1})(a, b) = (I_{b^{-1}}(a^{-1}) I_{b^{-1}}(a), e) = (e, e)$$
        Последний переход верен из следующих соображений: $\phi(a^{-1}) \phi(a) = \phi(a a^{-1}) = \phi(e) = e_1$. Проверим с другой стороны: $$(a, b)(I_{b^{-1}}(a^{-1}), b^{-1}) = (a I_b(I_{b^{-1}}(a)), e) = (a I_e(a^{-1}), e) = (e, e)$$
    \end{enumerate}
\end{proof}

\begin{note}
    Если в качестве автоморфизма взять $\id$, то получим обычное прямое произведение.
\end{note}

\begin{theorem}[о свойствах полупрямых произведений]~
    Пусть $I: B \to \Aut A, A \rtimes_I B$. Тогда:
    \begin{enumerate}
        \item $A \lhd A \rtimes_I B$ (здесь мы отождествляем $A \cong A \times \{ e \}$).
        \item $A \cdot B = A \rtimes B$ ($A \cong A \cdot \{ e \}, B \cong \{ e \} \cdot B$).
        \item Действие группы $B \times \{ e \}$ на $A \times B$ сопряжениями на $A \times \{ e \}$ совпадает с действием $I: B \to \Aut A$
        $$(a, e)^{(e, b)^{-1}} = (I_b(a), e)$$
        \item $A \rtimes B/A \cong B$.
    \end{enumerate}
\end{theorem}

\begin{proof}
    \begin{enumerate}
        \item Пусть $(a', e) \in A \times \{ e \}$. Проверим критерий нормальности: 
        \begin{equation}
        \begin{gathered}
        (a, b)^{-1}(a', e)(a, b) = (I_{b^{-1}}(a^{-1}), b^{-1})(a', e)(a, b) = (I_{b^{-1}}(a^{-1}) I_{b^{-1}}(a'), b^{-1})(a, b) = \\
        = (I_{b^{-1}}(a^{-1})I_{b^{-1}}(a')I_{b^{-1}}(a), e) = (I_{b^{-1}}(a^{-1} a' a), e) \in A \times \{ e \}
        \end{gathered}
        \end{equation}
        \item $(a, b) = (a, e) \cdot (e, b) = (a I_e(e), b)$
        \item $(e, b)(a, e)(e, b)^{-1} = (I_b(a), b)(I_{b^{-1}}(e), b^{-1}) = (I_b(a)I_b(I_{b^{-1}}(e)), bb^{-1}) = (I_b(a), e)$.
        \item $A \rtimes B/A \cong B$ -- ? \\
        Рассмотрим $\phi: A \rtimes B \to B, \phi(a, b) = b$. Проверим, что это сюръективный гомоморфизм: $\phi((a_1, b_1) \cdot (a_2, b_2)) = \phi ((a_1I_{b_1}(a_2), b_1b_2)) = b_1b_2 = \phi(a_1, b_1) \cdot \phi(a_2, b_2)$. Теперь найдем ядро: $\ker \phi = \{ (a, b)|b = e \} \cong A$. По основной теореме о гомоморфизмах \ref{th3.1}: $B \cong A \rtimes B/A$.
    \end{enumerate}
\end{proof}

\begin{theorem}
    $G$ -- группа, $A$ и $B$ -- тривиально пересекающиеся подгруппы в $G$, $A \lhd G$ и $A \cdot B = G$. Тогда $G \cong A \rtimes_c B$, где $c$ означает, что $B$ действует на группе $A$ сопряжениями: $I_b(a) = bab^{-1}$.
\end{theorem}

\begin{proof}
    Рассмотрим $\phi: A \rtimes_c B \to G, \phi(a, b) = a \cdot b \in G$. Проверим $\phi$ на гомоморфизм:
    \begin{equation}
        \begin{gathered}
            \phi((a_1, b_1) \cdot (a_2, b_2)) = \phi (a_1 I_{b_1}(a_2), b_1b_2) = \phi(a_1b_1a_2b_1^{-1}, b_1b_2) = a_1b_1a_2b_2 = \phi(a_1, b_1) \phi(a_2, b_2)
        \end{gathered}
    \end{equation}
    Также $\phi$ сюръективно, так как $A \cdot B = G$. Осталось рассмотреть ядро гомоморфизма: $\ker \phi = \{ (a, b)|ab = e \} = \{ e \}$, так как $A$ и $B$ пересекаются тривиально. Отсюда по основной теореме о гомоморфизмах \ref{th3.1} получаем желаемое.
\end{proof}