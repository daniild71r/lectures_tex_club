\section{Строение групп}

\subsection{Теоремы Силова}

\begin{definition}
	Пусть $G$ --- конечная группа, $|G| = n$, $p$ --- простой делитель числа $n$, $n = p^ks$, $(p, s) = 1$. Тогда подгруппа $H \le G$ такая, что $|H| = p^k$, называется \textit{силовской $p$-подгруппой} группы $G$.
\end{definition}

\begin{note}
	Если $G$ --- конечная группа, $|G| = n$, $t \mid n$, то необязательно в $G$ есть подгруппа порядка $t$. Например, в группе $A_4$, $|A_4| = 12$ нет подгруппы порядка 6.
\end{note}

\begin{theorem} (Теоремы Силова)
	Пусть $G$ --- конечная группа, $|G| = n$, $p$ --- простой делитель числа $n$, $n = p^ks$, $(p, s) = 1$. Обозначим через $N_p$ количество силовских $p$-подгрупп в $G$ Тогда:
	\begin{enumerate}[align=left, leftmargin=15pt]
		\item В $G$ существует силовская $p$-подгруппа, то есть $N_p > 0$
		\item[1'.] Любая $p$-подгруппа в $G$ содержится в некоторой силовской
		\item Все силовские $p$-подгруппы в $G$ сопряжены
		\item $N_p \equiv_p 1$
		\item[3'.] $N_p \mid s$
	\end{enumerate}
\end{theorem}

\begin{proof}[Доказательство 1 и 3]
	Положим $\Omega := \{M  \subset G: |M| = p^k\}$ и рассмотрим действие $G$ на $\Omega$ левыми сдвигами: $\forall g \in G: \forall M \in \Omega: g(M) = gM$. Если для некоторого $M \in \Omega$ его стабилизатор --- это $H \le G$, то $M = HM = \bigcup_{m \in M}Hm$, то есть $M$ разбивается на непересекающиеся правые смежные классы по $H$. В частности, это означает, что $|H| \mid |M| = p^k$, то есть $|H| = p^l, l \le k$. Тогда 
	\[
		|H| = p^k \Lra \exists g \in G \colon M = Hg \Lra |G(M)| = |G : H| = s
	\]
	Если же $|H| = p^l, l < k$, то $|G(M)| = |G : H| = p^{k-l}s \equiv_p 0$. Обозначим через $\Omega_1, \ldots, \Omega_r$ орбиты действия и воспользуемся формулой орбит:
	\[
		\binom{n}{p^k} = |\Omega| = \sum\limits_{i = 1}^r|\Omega_i| = \sum\limits_{i = 1}^r|G : \St(M_i)| \equiv_p N_ps
	\]
	Заметим, что полученное сравнение для $N_ps$ зависит лишь от числа $n$, но не от конкретного вида группы. Значит, мы имеем право рассмотреть конкретную группу порядка $n$, например, $\Z_n$. У неё динственная силовская $p$-подгруппа --- это $s\Z_n$, поэтому в данном случае $N_p = 1$ и $\binom{n}{p^k} \equiv_p s$. Возвращаясь к общему случаю, получаем, что
	\[
		s \equiv_p \binom{n}{p^k} \equiv_p N_ps \Lora p \mid (N_p - 1)s \Lora N_p \equiv_p 1
	\]
\end{proof}

\begin{proof}[Доказательство 1' и 2]
	Пусть $P \le G$ --- силовская $p$-подгруппа, а $Q$ --- $p$-подгруппа в $G$. Рассмотрим действие $Q$ на $G / P$ левыми сдвигами: $\forall q \in Q, gP \in G / P\ q(gP) = qgP$. Обозначим через $\Omega_1, \ldots, \Omega_r$ орбиты действия и воспользуемся формулой орбит:
	\[s = |G / P| = \sum\limits_{i = 1}^r |\Omega_i| = \sum_{i = 1}^r |Q : \St(\omega_i)|\]
	
	Поскольку правая часть --- это сумма выражений вида $p^l$, $l \in \Z$, и $p \nmid s$, то
	\[
		\exists i \in \{1, \ldots, r\} \such |Q : \St(\omega_i)| = 1 \Lra \St(w_i) = Q
	\]
	Пусть $\omega_i = gP$, тогда $QgP = gP \ra QP^{g^{-1}} = P^{g^{-1}} \ra Q \le P^{g^{-1}}$, причем подгруппа $P^{g^{-1}}$ --- силовская. Если же $Q$ --- тоже силовская $p$-подгруппа в $G$, то $|Q| = |P|$, поэтому $Q = P^{g^{-1}}$.
\end{proof}

\begin{proof}[Доказательство 3']
	Пусть $P \le G$ --- силовская $p$-подгруппа. Тогда все силовские $p$-подгруппы имеют вид $P^g$, $g \in G$. Они образуют орбиту $P^G$ при действии $G$ на множестве своих подгрупп сопряжениями. Тогда $N_p = \frac{|G|}{|N_G(P)|}$ и, поскольку $P \le N_G(P)$ (так как любой элемент $P$ при сопряжении $P$ даст просто $P$), $N_p \mid s$.
\end{proof}

\begin{note}
	Пусть $G$ --- конечная группа, $|G| = n$, $p$ --- простой делитель числа $n$, $n = p^ks$, $(p, s) = 1$. Обозначим через $N_p(l)$ число подгрупп порядка $p^l$, $l \le k$. Аналогично доказательству выше, можно показать, что $N_p(l) \equiv_p 1$.
\end{note}

\begin{proposition}
	Пусть $p < q$ --- простые числа. Тогда любая группа порядка $pq$ разрешима и, более того, она изоморфна $\Z_q \sd \Z_p$
\end{proposition}

\begin{proof}
	Пусть $|G| = pq$. По теореме Силова, в ней есть силовские $p$- и $q$-подгруппы. Обозначим таковыми $|H_p| = p$, $|H_q| = q$. В силу простоты порядка этих подгрупп, для них есть изоморфизм $H_p \cong \Z_p$ и $H_q \cong \Z_q$. Более того, по теореме Силова:
	\[
		N_q \equiv 1 \pmod q \wedge N_q \mid p < q \ra N_q = 1
	\]
	Это означает, что действие сопряжениями на $H_q$ даёт всегда её же. Стало быть, $H_q \normal G$. Осталось сказать, что $H_q \cap H_p = \{e\}$. Это так, ибо порядок пересечения должен делится на порядки этих групп. Стало быть, $H_q \sd H_p = G$. Про полупрямое произведение мы знаем, что $G / H_q \cong H_p$ и притом $H_p, H_q$ абелевы. Стало быть, $G$ разрешима.
\end{proof}

\begin{theorem} Пусть $G$ --- конечная группа. Тогда:
	\begin{enumerate}
		\item Если $P$ --- силовская $p$-подгруппа в $G$, то $P \normal G \Lra N_p = 1$
		
		\item Все силовские подгруппы нормальны в $G$ тогда и только тогда, когда их внутренее прямое произведение является самой $G$
	\end{enumerate}
\end{theorem}

\begin{proof}~
	\begin{enumerate}
		\item \begin{itemize}
			\item[$\la$] Если $N_p = 1$, то $\forall g \in G: P^g = P$, поэтому $P \normal G$.
			
			\item[$\ra$] Если $P \normal G$, то любая другая силовская $p$-подгруппа в $G$ имеет вид $P^g, g \in G$, тогда, поскольку $P^g = P$, $N_p = 1$.
		\end{itemize}
		
		\item \begin{itemize}
			\item[$\la$] Если $G = P_1 \times \ldots \times P_m$, то из определения прямого произведения, $\forall i \in \{1, \ldots, m\}\ P_i \normal G$. При этом, если посмотреть на любой $p \mid |G|$, то в силу равенства $|G| = |P_1| \cdot \ldots \cdot |P_m|$ будет существовать $P_i$ (и ровно одна для всей $G$ из-за предыдущего пункта) --- силовская $p$-подгруппа.
			
			\item[$\ra$] Проведем индукцию по $m$. База, $m = 1$, тривиальна. Пусть теперь $m > 1$. Положим $H := P_1\dots P_{m - 1} \normal G$. Тогда в $H$ есть силовские подгруппы $P_1, \ldots, P_{m - 1}$, и, более того, все они нормальны в $H$. По предположению индукции, $H = P_1 \times \dotsb \times P_{m-1}$. Поскольку $(|H|, |P_m|) = 1$, то $H \cap P_m = \{e\} \ra HP_m = G$, и $G = P_1 \times \dotsb \times P_m$.
		\end{itemize}
	\end{enumerate}
\end{proof}