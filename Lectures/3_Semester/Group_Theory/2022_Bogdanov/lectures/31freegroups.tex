\section{Задание групп}

\subsection{Свободные группы}

\begin{definition}
	Пусть $F_n$ "--- группа, $F_n = \gl f_1, \ldots, f_n\gr$. Такая группа $F_n$ называется \textit{свободной со свободными образующими} $f_1, \ldots, f_n \in F_n$, если для любой группы $G$ выполнено \textit{универсальное свойство}:
	\[
		\forall g_1, \ldots, g_n \in G\ \exists \phi \colon F_n \to G \text{ --- гомоморфизм} \such \forall i \in \{1, \ldots, n\}\ \ \phi(f_i) = g_i
	\]
\end{definition}

\begin{note}
	Если такой гомоморфизм существует, то он единственен, поскольку $F_n$ порождена элементами $f_1, \ldots, f_n$.
\end{note}

\begin{note}
	Аналогичным образом можно определить и свободные группы с бесконечным количеством свободных образующих.
\end{note}

\begin{note}
	Если $G = \trb{g_1, \ldots, g_n}$, то $G = \im\phi \cong F_n / \ke\phi$, то есть свободная группа как минимум своими факторами задаёт все группы, порождённые $n$ элементами.
\end{note}

\begin{theorem}
	Свободная группа $F_n$ со свободными образующими $f_1, \ldots, f_n$ существует.
\end{theorem}

\begin{proof}
	Заведём алфавит символов $\Sigma = \{f_1, \ldots, f_n, f_1^{-1}, \ldots, f_n^{-1}\}$. Тогда $F_n$ как множество можно описать следующим образом:
	\[
		F_n = \{w \in \Sigma^* \colon \forall i \in \{1, \ldots, n\} \text{ $w$ не содержит подслов $f_if_i^{-1}$ и $f_i^{-1}f_i$}\}
	\]
	
	Определим операцию на $F_n$ следующим образом: если $w_1, w_2 \in F_n$, то сократим взаимно обратные элементы алфавита с конца $w_1$ и начала $w_2$, получив $w_1'$ и $w_2'$, и положим $w_1 \cdot w_2 := w_1'w_2'$.
	
	Докажем, что $F_n$ "--- действительно группа:
	\begin{itemize}
		\item (Нейтральный элемент) $\exists \eps \in F_n \such \forall w \in F_n\ \ w \cdot \eps = \eps \cdot w = w$.
		\item (Обратный элемент) Пусть $w \in F_n, w = f_{i_1}^{\alpha_1}\ldots f_{i_k}^{\alpha_k}$, где $f_{i_j} \in \{f_1, \ldots, f_n\}$ и $\alpha_{i_j} \in \{\pm 1\}$. Тогда $\exists w^{-1} = f_{i_k}^{-\alpha_k} \ldots f_{i_1}^{-\alpha_1} \in F_n \such w \cdot w^{-1} = w^{-1} \cdot w = \eps$.
		\item (Ассоциативность) Докажем, что $\forall a, b, c \in F_n: (ab)c = a(bc)$ индукцией по $|b|$. 
		\begin{itemize}
			\item База $|b| = 0$: это соответствует только $b = \eps$, свойство ассоциативности тривиально.
			
			\item База $|b| = 1$: нужно сделать разбор случаев, чем заканчивается $a$ и начинается $c$.
			
			\item Переход $|b| > 1$: пусть $b = xb'$, $x \in \Sigma$. Тогда
			\[
				(ab)c = (a(xb'))c = ((ax)b')c = (ax)(b'c) = a(x(b'c)) = a((xb')c) = a(bc)
			\]
			
		\end{itemize}
	\end{itemize}
	
	Проверим теперь, что $F_n$ "--- свободная группа. Пусть $G$ "--- произвольная группа, $g_1, \ldots, g_n \in G$. Определим $\phi \colon F_n \to G$ следующим образом: $\phi(f_{i_1}^{\alpha_1} \ldots f_{i_k}^{\alpha_k}) = g_{i_1}^{\alpha_1} \ldots g_{i_k}^{\alpha_k}$. Тогда, по определению, $\forall i \in \{1, \ldots, n\}\ \phi(f_i) = g_i$. Наконец, $\phi$ "--- гомоморфизм, поскольку $\forall w_1, w_2 \in F_n$ в записях $\phi(w_1w_2)$ и $\phi(w_1)\phi(w_2)$ сокращаются одни и те же элементы.
\end{proof}

\begin{theorem}
	Пусть $F_n$ --- свободная группа со свободными образующими $f_1, \ldots, f_n$, $G_n$ --- свободная группа со свободными образующими $g_1, \ldots, g_n$. Тогда существует изоморфизм $\phi \colon F_n \to G_n$ такой, что $\forall i \in \{1, \ldots, n\}\ \phi(f_i) = g_i$.
\end{theorem}

\begin{proof}
	По определению свободной группы, существует гомоморфизм $\phi: F_n \to G_n$ такой, что $\forall i \in \{1, \ldots, n\}\ \phi(f_i) = g_i$, и, аналогично, существует гомоморфизм $\psi\ G_n \to F_n$ такой, что $\forall i \in \{1, \ldots, n\}\ \psi(g_i) = f_i$. Тогда $\psi \circ \phi = \id_{F_n}$, $\phi \circ \psi = \id_{G_n}$, поэтому эти гомоморфизмы биективны и взаимно обратны.
\end{proof}

\begin{example}
	Рассмотрим $F_1 = \{f_1^n \colon n \in \Z\}$. Легко видеть, что $F_1 \cong \Z$.
\end{example}

\begin{note}
	При $n \ge 2$ группа $F_n$ --- уже неабелева, например, потому, что $f_1f_2 \ne f_2f_1$.
\end{note}

\begin{exercise}
	Пусть $G = \SL_2(\Z[x])$. Рассмотрим следующую подгруппу в $G$:
	\[
		F = \left\gl 
		\begin{pmatrix}
			1&0
			\\
			x&1
		\end{pmatrix},
		\begin{pmatrix}
			1&x
			\\
			0&1
		\end{pmatrix} \right\gr \le G
	\]
	
	Докажите, что $F$ --- свободная группа с соответствующими свободными образующими.
\end{exercise}

\begin{definition}
	Пусть $w \in F_n$, $G$ --- группа, $g_1, \ldots, g_n \in G$. Тогда \textit{значение $w$ в группе $G$} --- это $w(g_1, \ldots, g_n) := \phi(w)$, где $\phi$ "--- гомоморфизм $F_n$ и $G$ такой, что $\forall i \in \{1, \ldots, n\}\ \phi(f_i) = g_i$.
\end{definition}