\subsection{Образующие и соотношения}

\begin{definition}
	Пусть $F_n$ "--- свободная группа, $S \subseteq F_n$, $G$ "--- группа, $g_1, \ldots, g_n \in G$, $G = \gl g_1, \ldots, g_n\gr$. Положим $K := \gl S\gr_{norm}$. Говорят, что $G$ \textit{задается образующими $g_1, \ldots, g_n$ и соотношениями $S$}, если $K = \ke\phi$, где $\phi$ "--- гомоморфизм $F_n$ и $G$ такой, что $\forall i \hm\in \{1, \ldots, n\}\ \phi(f_i) = g_i$.
	
	Обозначение "--- $G = \gl g_1, \ldots, g_n\mid S|_{f_i \mapsto g_i} = e\gr$, где $S|_{f_i \mapsto g_i}$ "--- множество слов, полученных формальной подстановкой символов $g_1, \ldots, g_n$ вместо $f_1, \ldots, f_n$ в слова из $S$.
\end{definition}

\begin{note}
	В терминах определения выше, $G = \gl g_1, \ldots, g_n\gr \hm= \im\phi \cong F_n / K$. Значит, группа $G$ задается образующими и соотношением однозначно. Кроме того, $\forall w \in S\ w(g_1, \ldots, g_n) = \phi(w) = e$, и, неформально говоря, все соотношения элементов $G$ следуют из соотношений $S$.
\end{note}

\begin{example}
	$\Z_n \cong \gl a\mid a^n\gr$, поскольку $\gl n\gr_{norm} = n\Z$ и $\Z_n \hm\cong \Z/n\Z \hm\cong F_1/n\Z$. Этот факт также можно записать в виде $\Z_n \cong \gl a\mid a^n = e\gr$.
\end{example}

\begin{note}
	Отметим, что $F_n = \gl f_1, \ldots, f_n\mid \emptyset\gr$: на элементы $F_n$ не накладывается никаких соотношений.
\end{note}

\begin{theorem} (Универсальное свойство группы, заданной образующими и соотношениями)
	Пусть $S \subseteq F_n$, $G = \gl g_1, \ldots, g_n\mid S|_{f_i \mapsto g_i}\gr$, $H$ --- группа, и $h_1, \ldots, h_n \in H$ таковы, что $\forall w \in S\ w(h_1, \ldots, h_n) \hm= e$. Тогда сущетвует $\phi \colon G \to H$ --- гомоморфизм такой, что $\forall i \in \{1, \ldots, n\}\ \phi(g_i) = h_i$.
\end{theorem}

\begin{proof}
	Будем считать, что $H = \gl h_1, \ldots, h_n \gr$ (если есть что-то лишнее, то мы об этом <<забудем>>, ведь строим просто гомоморфизм).
	
	Пусть $\psi \colon F_n \to G$, $\psi(f_i) = g_i$. В силу задания группы $G$ соотношениями, имеем $\ke\psi = K = \gl S \gr_{norm}$. В силу универсального свойства свободной группы, у нас есть гомоморфизм $\Theta \colon F_n \to H$, имеющий вид $\Theta(f_i) = h_i$. По условию
	\[
		\Theta(S) = e \ra S \le \ke\Theta =: L \normal F_n
	\]
	Стало быть, $K = \gl S \gr_{norm} \le L$. По второй теореме об изоморфизме получаем такую цепочку:
	\[
		H \cong F_n / L \cong (F_n / K) / (L / K)
	\]
	При этом, естественно $L / K \normal F_n / K$. Как и в построении этого изоморфизма, можно рассмотреть канонический эпиморфизм $\pi \colon F_n / K \to (F_n / K) / (L / K)$, заменив множество значений сразу на $F_n / L$. При этом $aK \mapsto aL$, а изоморфизмы для $G$ и $H$ утверждают, что $g_i \mapsto f_i K$ и $h_i \mapsto f_i L$, то есть сквозной гомоморфизм $\phi$ будет соблюдать равенство $\phi(g_i) = h_i$.
\end{proof}

\begin{note}
	Построенный $\phi$ единственен.
\end{note}

\begin{exercise}
	Если в $G$ выполняются соотношения из $S$, и она удовлетворяет универсальному свойству, то она и задана соотношениями из $S$.
\end{exercise}

\textcolor{red}{Далее конспект не отредактирован с 2020 года.}

\begin{example}
	Рассмотрим $G := \gl a, b\mid a^2=b^2=(ab)^2 = e\gr$. Поскольку $a^2 = b^2 = e$, то элементы группы $G$ "--- это слова из $a, b$, в которых нет подряд идущих одинаковых символов. Более того, $abab = e \ra ab \hm= ba$, поэтому $G = \{e, a, b, ab\}$ и $|G| \le 4$.
	
	Покажем, что $G \cong \Z_2\times\Z_2$. Рассмотрим в $H := \Z_2\times\Z_2$ элементы $a' = (\overline{1}, \overline{0})$, $b' = (\overline{0}, \overline{1})$. Тогда $a'^2 = b'^2 = (a'b')^2 = e$, и, по предыдущей теореме, $\exists \phi: G \to H$ "--- гомоморфизм такой, что $\phi(a) = a'$, $\phi(b) = b'$. Значит, $\im\phi = H$, тогда $|G| = 4$ и $G \cong H$.
\end{example}

\begin{note}
	Пусть $G = \gl g_1, \ldots, g_n\mid S\gr$. Чтобы выяснить, что это за группа, можно описать все различные элементы $G$. Однако этот способ нельзя алгоритмизовать, потому что в общем случае нельзя алгоритмически определить, равны ли два слова из символов $g_1, \ldots, g_n$ при заданных соотношениях $S$.
\end{note}