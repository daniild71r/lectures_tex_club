\subsection{Лемма Бернсайда}

\begin{definition}
	Действие группы $G$ на множестве $\Omega$ называется \textit{транзитивным}, если $\Omega$ является единственной орбитой действия. Иными словами, $\forall \omega_1, \omega_2 \in \Omega: \exists g \in G: g(\omega_1) = \omega_2$.
\end{definition}

\begin{example}
	$S_n$ действует на $\{1, \dotsc, n\}$ транзитивно, а $S_{n-1} \le S_n$ действует на $\{1, \dotsc, n\}$ нетранзитивно.
\end{example}

\begin{theorem}[Лемма Бернсайда]
	Пусть конечная группа $G$ действует на множестве $\Omega$ транзитивно. Для $\forall g \in G$ обозначим $F(g) := |\{\omega \hm\in \Omega: g\omega = \omega\}|$. Тогда:
	\[\sum\limits_{g \in G}F(g) = |G|\]
\end{theorem}

\begin{proof}
	Положим $S := \{(g, \omega) \in G \times \Omega \colon g\omega = \omega\}$. Посчитаем это множество двумя способами:
	\begin{itemize}
		\item С одной стороны, можно просуммировать множество пар при фиксированном $\omega$. Это будет ничто иное как $\St(\omega)$:
		\[
			|S| = \sum_{\omega \in \Omega} |\St(\omega)| = \sum_{\omega \in \Omega} \frac{|G|}{|G(\omega)|} = \sum_{\omega \in \Omega} \frac{|G|}{|\Omega|} = |G|
		\]
		
		\item С другой стороны, можно написать аналогичную сумму по преобразованиям $g \in G$:
		\[
			|S| = \sum_{g \in G} |\{\omega \in \Omega \colon g\omega = \omega\}| = \sum_{g \in G} F(g)
		\]
	\end{itemize}
\end{proof}

\begin{corollary}[Лемма Бернсайда, другая формулировка]
	Пусть конечная группа $G$ действует на множестве $\Omega$, $F(g)$ определено как в последней теореме. Тогда:
	\[
		|\Omega / G| = \frac{1}{|G|} \sum_{g \in G} F(g)
	\]
\end{corollary}

\begin{proof}
	Пусть $k := |\Omega / G|$. Представим $\Omega$ в виде $\Omega = \bigsqcup_{i = 1}^k\Omega_i$, где $\Omega_1, \dotsc, \Omega_k$ "--- орбиты действия. Тогда $\forall i \in \{1, \dotsc, k\}:$ $G$ действует на $\Omega_i$ транзитивно (значит, в частности, $\Omega$ конечно). Для $\forall g \in G$ положим $F_i(g) := |\{\omega \in \Omega_i: g\omega = \omega\}|$ и воспользуемся леммой Бернсайда:
	\[\sum\limits_{g \in G}F(g) = \sum\limits_{g \in G}\sum\limits_{i = 1}^kF_i(g) = \sum\limits_{i = 1}^k|G| = k|G| \ra k = \frac1{|G|}\sum\limits_{g \in G}F(g)\]
\end{proof}

\begin{note}
	Формально стоит требовать $|\Omega| < \infty$ в последнем следствии, но, вообще говоря, даже так равенство будет выполнено (с двух сторон просто могут быть бесконечности из-за $|\Omega| = +\infty$).
\end{note}

\begin{example}
	Рассмотрим ожерелья из $p$ бусинок ($p > 2$ "--- простое число), в которых каждая бусинка покрашена в один из $k$ цветов. Найдем количество различных ожерелий (с точностью до поворота и переворота). Пусть $\Omega$ "--- множество неподвижных ожерелий, то есть не допускающих повороты и перевороты, тогда $|\Omega| = k^p$. Группа $G = \mathcal{D}_p$ действует на $\Omega$, и искомая величина "--- это $|\Omega / G|$, поскольку элементы одной орбиты отличаются друг от друга только композицией поворотов и переворотов. Элементы $G$ имеют один из следующих видов:
	\begin{itemize}
		\item Если $g = \id$, то $F(g) = |\Omega| = k^p$
		\item Если $g$ "--- поворот на $2\pi\frac kp$, $0 < k < p$, то $F(g) = k$ в силу простоты $p$, поскольку любая фиксированная бусинка совпадает по цвету с бусинками, в которые она переходит при повороте на $2\pi\frac kp, 2\pi\frac {2k}p, \ldots, 2\pi\frac {(p-1)k}p$, и полученные таким образом бусинки образуют все ожерелье
		\item Если $g$ "--- переворот, то есть симметрия, то $F(g) = k^{\frac{p+1}2}$, поскольку ${\frac{p+1}2}$ подряд идущих бусинок, начиная с той, через которую проходит ось симметрии, однозначно задают цвета оставшихся
	\end{itemize}
	
	Применим теперь лемму Бернсайда:
	\[
		|\Omega / G| = \frac{k^p + (p-1)k + pk^{\frac{p+1}2}}{2p}
	\]
\end{example}

\begin{corollary}
	При помощи леммы Бернсайда, либо из этого примера, либо из аналогичного без отражений, можно доказать малую теорему Ферма.
\end{corollary}

\begin{definition}
	Пусть $G$ действует на множестве $\Omega$. Обозначим за $\Omega^{[k]}$ такое множество:
	\[
		\Omega^{[k]} = \{(\omega_1, \ldots, \omega_k) \in \Omega^k \colon i \neq j \ra \omega_i \neq \omega_j\}
	\]
	Действие \textit{называется $k$-транзитивным}, если выполнено условие:
	\[
		\forall (\omega_1, \ldots, \omega_k), (\delta_1, \ldots, \delta_k) \in \Omega^{[k]}\ \exists g \in G \colon (g\omega_1, \ldots, g\omega_k) = (\delta_1, \ldots, \delta_k)
	\]
	Другими словами, $G$ действует на $\Omega^{[k]}$ транзитивно.
\end{definition}

\begin{exercise}
	Пусть группа $G$ действует на множестве $\Omega$, причем действие 2\nobreakdash-транзитивно. Для $\forall g \in G$ обозначим $F(g) := |\{a \hm\in \Omega: ga = a\}|$. Докажите, что выполнено следующее равенство:
	\[\sum\limits_{g \in G}F(g)^2 = 2|G|\]
\end{exercise}