\begin{proof}
	Пусть $\phi$ --- положительная параметризация. В силу определения положительной параметризации, набор векторов $\{\pd{\phi}{u_i}\}_{i = 1}^3$ образует положительный базис в касательном пространстве $T_M$ в каждой точке многообразия $M$. Несложно понять, что границей клетки должны быть образы граней $K_\alpha^j$ стандартного куба $K$, которые притом сами являются двумерными многообразиями $M_\alpha^j = \phi(K_\alpha^j)$. Не умаляя общности, изучим клетку $M_1^2 = \phi(K_1^2)$. Её касательные пространства $T_{M_1^2}(x)$ имеют индуцированный положительный базис $\ps{\pd{\phi}{u_1}(u), \pd{\phi}{u_3}(u)}$. Зафиксируем точку $x_0 = \phi(u_0) \in M_1^2$ и последовательно докажем следующие факты:
	\begin{enumerate}
		\item Вектор $\grad \psi_2(x_0)$ ортогонален соответствующему базису в $T_{M_1^2}$. Для начала поймём, чем является скалярное произведение градиента с любым вектором $\phi'(u_0)\vv{H} \in T_{M_1^2}$. Вспомним уже возникавшее равенство:
		\[
			\psi'(x_0)(\phi'(u_0)\vv{H}) = \psi'(\phi(u_0))(\phi'(u_0)\vv{H}) = \vv{H}
		\]
		Скалярное произведение с градиентом от $\psi_2$ в точке $x$ в точности соответствует взятию первой координаты от равенства выше. Таким образом:
		\[
			(\grad \psi_2(x_0), \phi'(u_0)\vv{H}) = H^2
		\]
		где $H^2$ является соответствующей координатой вектора $\vv{H}$. Если теперь мы хотим взять произведение с базисными векторами, то $\vv{H}$ будет обладать столбцами $(1, 0, 0)^T$ и $(0, 0, 1)^T$ соответственно, в обоих случаях получим ноль.
		
		\item Вектор $\grad \psi_2(x_0)$ направлен вовне клетки $M$. В силу диффеоморфизма, необходимо и достаточно показать, что прообраз любого достаточно малого вектора, отложенного в точке на клетке $M_1^2$ в сторону от $M$, будет иметь положительное скалярное произведение с градиентом. Для этого можно рассмотреть приращение $\psi_2$:
		\[
			\psi_2(x) - \psi_2(x_0) = \tbr{\grad \psi_2(x_0), x - x_0} + o(x - x_0)
 		\]
 		Если $x - x_0$ соответствует описанному выше вектору, то $\psi_2(x) > 1$, а тогда по равенству скалярное произведение будет больше нуля, то есть градиент действительно смотрит вовне $M$.
	\end{enumerate}
	Таким образом, единичный вектор $n$, заданный формулой ниже, является внешней нормалью для $M$ на части границы, соответствующей $M_1^2$:
	\[
		n = \frac{\grad \psi_2(x)}{|\grad \psi_2(x)|}
	\]
	Если мы теперь докажем, что тройка $(n, \pd{\phi}{u_1}(u_0), \pd{\phi}{u_3}(u_0))$ является положительным базисом в $T_M$, то $n$ будет положительной нормалью клетки. Для этого выясним, как выражается основной базис (репер в точке) через этот:
	\begin{align*}
		&{\pd{\phi}{u_1}(u_0) = 0 \cdot n + 1 \cdot \pd{\phi}{u_1}(u_0) + 0 \cdot \pd{\phi}{u_3}(u_0)}
		\\
		&{\pd{\phi}{u_2}(u_0) = \frac{(\grad \psi_2(x_0), \pd{\phi}{u_2}(u_0))}{|\grad \psi_2(x_0)|^2} \cdot \grad \psi_2(x_0)} + C_1 \cdot \pd{\phi}{u_1}(u_0) + C_3 \cdot \pd{\phi}{u_3}(u_0)
		\\
		&{\pd{\phi}{u_3}(u_0) = 0 \cdot n + 0 \cdot \pd{\phi}{u_1}(u_0) + 1 \cdot \pd{\phi}{u_3}(u_0)}
	\end{align*}
	Без долгих расписываний, утверждаю, что соответствующий определитель матрицы перехода будет положителен:
	\[
		\begin{vmatrix}
			0 & \frac{1}{|\grad \psi_2(x_0)|^2} & 0
			\\
			1 & C_1 & 0
			\\
			0 & C_3 & 1
		\end{vmatrix} = \frac{1}{|\grad \psi_2(x_0)|^2} > 0
	\]
\end{proof}

\begin{theorem} (Гаусса-Остроградского для клетки)
	Если $\vv{A}$ --- гладкое векторное поле, заданное на трёхмерной клетке $M$, то имеет место равенство:
	\[
		\iint_{\vdelta M} \ntbr{\vv{A}, \vv{n}}dS = \iiint_M \Div \vv{A} d\mu(x)
	\]
	где $\vv{n}$ --- внешняя нормаль к границе $\vdelta M$.
\end{theorem}

\begin{note} (Пояснение к термину <<поток>>)
	Рассмотрим постоянное векторное поле $\vv{A}$ и ориентированный параллелограмм, натянутый на вектора $(\vv{\xi}_1, \vv{\xi_2})$. В таком случае поток $\vv{A}$ через параллелограмм равен $(\vv{A}, \vv{\xi}_1, \vv{\xi}_2)$.
\end{note}

\begin{proof}
	Всё доказательство сводится к применению теоремы Стокса-Пуанкаре для клетки, которую мы уже доказали. Пусть $\phi$ --- положительная параметризация. Достаточно заметить, что
	\[
		\ps{\vv{A}(x), \pd{\phi}{u_1}(u), \pd{\phi}{u_2}(u)} = \ps{\vv{A}, \sbr{\pd{\phi}{u_1}, \pd{\phi}{u_2}}} = \ntbr{\vv{A}, \vv{n}} \md{\sbr{\pd{\phi}{u_1}, \pd{\phi}{u_2}}} = \ntbr{\vv{A}, \vv{n}} \sqrt{EG - F^2}
	\]
	Стало быть:
	\[
		\iint_{\vdelta M} \ntbr{\vv{A}, \vv{n}}dS = \iint_K \ntbr{\vv{A}, \vv{n}}\sqrt{EG - F^2} d\mu(u)
	\]
\end{proof}

\begin{corollary} (Геометрическое определение дивергенции)
	Если $\vv{A}$ --- гладкое векторное поле в окрестности точки $\vv{r}_0$, то имеет место формула:
	\[
		\Div \vv{A}(\vv{r}_0) = \lim_{\eps \to +0} \frac{1}{V_{\text{шара}}(\eps)} \iint_{\vdelta B_\eps(\vv{r}_0)} \ntbr{\vv{A}, \vv{n}}dS
	\]
	где $V_{\text{шара}}(\eps) = \frac{4}{3}\pi\eps^3$ --- объём шара радиуса $\eps$, а $B_\eps(\vv{r_0})$ --- замкнутый шар радиуса $\eps$ с центром в точке $\vv{r}_0$.
\end{corollary}

\begin{lemma}
	Положительная ориентация границы двумерной клетки $M$ совпадает с ориентацией с помощью положительной касательной.
\end{lemma}

\begin{proof}
	\textcolor{red}{Пока что под вопросом. В курсе не вводилось понятия ориентации границы двумерной клетки.}
\end{proof}

\begin{theorem} (Формула Стокса)
	Пусть $M$ --- двумерная клетка в $\R^3$, $\vv{A}$ --- гладкое векторное поле на $M$. Тогда циркуляция поля $\vv{A}$ вдоль границы $\vdelta M$, ориентированной по правилу правого обхвата, равна потоку ротора $\vv{A}$ через $M$, то есть
	\[
		\int_{\vdelta M} \ntbr{\vv{A}, \vv{\tau}}ds = \iint_M \ntbr{\rot \vv{A}, \vv{n}}dS
	\]
\end{theorem}

\begin{proof}
	\textcolor{red}{Ожидается}
\end{proof}

\begin{corollary} (Геометрическое определение ротора)
	Пусть $\vv{A}$ --- гладкое векторное поле в окрестности точки $\vv{r}_0$. Тогда для любого единичного вектора $\vv{l} \in \R^3$ верно, что
	\[
		\ntbr{\rot \vv{A}, \vv{l}} = \lim_{\eps \to +0} \frac{1}{\pi \eps^2} \int_{\vdelta S_\eps(\vv{r}_0)} \ntbr{\vv{A}, \vv{\tau}}ds
	\]
	где $S_\eps(\vv{r}_0)$ --- круг с центром в точке $\vv{x}_0$ радиуса $\eps$, ортогональный к вектору $\vv{l}$.
\end{corollary}

\begin{proof}
	У круга в любой точке будет одинаковая положительная нормаль $\vv{l}$, поэтому
	\[
		\int_{\vdelta S_\eps(\vv{r}_0)} \ntbr{\vv{A}, \vv{\tau}}ds = \iint_{S_\eps(\vv{r}_0)} \ntbr{\rot \vv{A}, \vv{l}}dS
	\]
	Остаётся заметить, что $\ntbr{\rot \vv{A}, \vv{l}}$ является непрерывной функцией, а потому интеграл тривиально оценивается снизу и сверху:
	\[
		\min_{x \in S_\eps(\vv{r}_0)} \ntbr{\rot \vv{A}, \vv{l}} \cdot \pi\eps^2 \le \iint_{S_\eps(\vv{r}_0)} \ntbr{\rot \vv{A}, \vv{l}}dS \le \max_{x \in S_\eps(\vv{r}_0)} \ntbr{\rot \vv{A}, \vv{l}} \cdot \pi\eps^2
	\]
	Устремление $\eps \to 0$ даёт нужный результат.
\end{proof}

\begin{note}
	Следствие также показывает, что ротор инвариантен относительно замены координат.
\end{note}

\textcolor{red}{Сюда можно картиночку с кругом}

\subsection{Интегрирование дифференциальных форм на дифференцируемых многообразиях}

\begin{note}
	Теперь клетки --- это образы \underline{открытых} параметризующих кубов.
\end{note}

\begin{definition}
	Клетки $M(i)$ и $M(j)$ называются \textit{сцепленными}, если выполнены условия:
	\begin{enumerate}
		\item $M(i) \cap M(j) \neq \emptyset$
		
		\item Существует гладкий диффеоморфизм $\pi(i, j)$ класса $C^k$ такой, что он задаётся диффеоморфизмами клеток:
		\[
			\pi(i, j) = \psi(i) \circ \phi(j)
		\]
		и переводит $\psi(j)(M(i) \cap M(j))$ в $\psi(i)(M(i) \cap M(j))$.
	\end{enumerate}
	Соответствующий диффеоморфизм $\pi$ называется \textit{сцепляющим диффеоморфизмом}.
\end{definition}

\textcolor{red}{Тут должна быть картинка с двумя кубами и многообразиями}