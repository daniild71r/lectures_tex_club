\begin{lemma}
	Если $f_m \to f$ в $L_1(E)$, то $f_m \Ra f$ на $E$.
\end{lemma}

\begin{proof}
	Сходимость функций по мере на $E$, которую мы хотим доказать, означает следующее:
	\[
		\forall \eps > 0\ \lim_{m \to \infty} \mu\set{x \in E \colon |f_m(x) - f(x)| \ge \eps} = 0
	\]
	Предположим противное ($\delta$ берётся из того факта, что мера неотрицательна, но и при этом не должна быть равна нулю в предположении):
	\[
		\exists \eps > 0, \exists \delta > 0 \such \forall M \in \N\ \exists m \ge M \such \mu\set{x \in E \colon |f_m(x) - f(x)| \ge \eps} > \delta
	\]
	Тогда обозначим $e_m = \{x \in E \colon |f_m(x) - f(x)| \ge \eps\}$ и заметим цепочку неравенств:
	\[
		 \forall M \in \N\ \exists m \ge M \such \int_E |f_m(x) - f(x)|d\mu(x) \ge \int_{e_m} |f_m(x) - f(x)|d\mu(x) \ge \eps \cdot \mu(e_m) \ge \eps \delta > 0
	\]
	Это противоречит со сходимостью предела интегралов.
\end{proof}

\begin{reminder} (Теорема Рисса)
	Если $f_m \Ra f$ на измеримом множестве $E$, то существует подпоследовательность $\{f_{m_k}\}_{k = 1}^\infty$ такая, что $f_{m_k} \to f$ почти всюду на $E$.
\end{reminder}

\begin{proposition}
	Если $f \in L_1(E)$, то $f$ измерима на $E$
\end{proposition}

\begin{proof}
	\textcolor{red}{Как объяснить верность в общем случае?} Пусть $f$ --- ограниченная функция. Наша стратегия состоит в том, чтобы найти последовательность из $L_1(E)$ измеримых на $E$ функций, сходящихся к нашей (тогда $f$ тоже измерима). Суммируемость на $E$ для $f$ означает следующее:
	\[
		\forall \eps > 0\ \exists P \such U(P, f) - L(P, f) < \eps
	\]
	Будем брать $\eps = 1, \frac{1}{2}, \ldots, \frac{1}{m}, \ldots$ Этим мы найдём соответствующую последовательность разбиений $P_m$:
	\[
		P_m \colon E = \bscup_{k = 1}^{N_m} E_{k, m} \such U(P_m, f) - L(P_m, f) < \frac{1}{m}
	\]
	Положим $g_m(x) = \sum_{k = 1}^{N_m} (\inf_{x \in E_{k, m}} f(x)) \cdot \chi_{E_{k, m}}(x)$, где $\chi_{E_{m, k}}$ --- индикатор множества. Аналогично с $\sup$ определяется $h_m(x)$. Сами по себе $g_m$ и $h_m$ являются измеримыми и суммируемыми на $E$, при этом
	\begin{align*}
		&{\forall x \in E\ \forall m \in \N\ g_m(x) \le f(x) \le h_m(x)}
		\\
		&{\int_E (h_m(x) - g_m(x))d\mu(x) = U(P_m, f) - L(P_m, f) < \frac{1}{m}}
	\end{align*}
	Последнее означает, что $h_m - g_m \to 0$ в $L_1(E)$. По последней лемме, это даёт $h_m - g_m \Ra 0$ на $E$. По теореме Рисса, мы находим подпоследовательность $\{h_{m_k} - g_{m_k}\}_{k = 1}^\infty$, которая поточечно сходится к $f$ почти всюду на $E$. Следовательно, $f$ тоже измерима на $E$ (слова <<почти всюду>> убираем, так как это и без того целый класс функций).
\end{proof}

\subsection{Предельный переход в интеграле Лебега}

\begin{theorem} (Лебега о мажорируемой сходимости)
	Если есть функциональная последовательность $\{f_m\}_{m = 1}^\infty$, $f_m \colon E \to \ole{\R}$ и $F \colon E \to \ole{\R}$, при этом выполнены условия:
	\begin{enumerate}
		\item $E$ --- измеримое множество конечной меры
		
		\item $f_m \to f$ почти всюду на $E$
		
		\item $F$ --- суммируемая на $E$ функция
		
		\item $\forall m \in \N\ \ |f_m(x)| \le F(x)$ почти всюду на $E$
	\end{enumerate}
	Тогда $f(x)$ оказывается суммируемой на $E$, причём выполнено равенство:
	\[
		\int_E f(x)d\mu(x) = \lim_{m \to \infty} \int_E f_m(x)d\mu(x)
	\]
\end{theorem}

\begin{proof}
	Так как $|f_m(x)| \le F(x)$, то по признаку суммируемости все $f_m$ и $f$ суммируемы на $E$, а значит и почти всюду конечны на нём.
	
	В силу связи сходимости почти всюду со сходимостью по мере, заключаем $f_m \Ra f$ на $E$. Зафиксируем произвольное $\eps > 0$ и обозначим $E_m := \{x \in E \colon |f_m(x) - f(x)| \ge \eps\}$. Сходимость по мере утверждает, что $\mu(E_m) \to 0$. Остаётся сделать оценку на модуль разности интегралов:
	\begin{multline*}
		\mo{\int_E (f_m(x) - f(x))d\mu(x)} \le \int_E |f_m(x) - f(x)|d\mu(x) =
		\\
		\int_{E_m} |f_m(x) - f(x)|d\mu(x) + \int_{E \bs E_m} |f_m(x) - f(x)|d\mu(x) \le
		\\
		2 \int_{E_m} F(x)d\mu(x) + \eps \cdot \mu(E \bs E_m)
	\end{multline*}
	Устремим $m \to \infty$. Тогда левое слагаемое станет нулём по абсолютной непрерывности, а правое станет $\eps\mu(E)$. Устремлением $\eps \to 0$ завершаем доказательство.
\end{proof}

\begin{theorem} (Леви о монотонной сходимости)
	Если имеется функциональная последовательность $\{f_m(x)\}_{m = 1}^\infty$, $f_m \colon E \to \ole{\R}$ и некоторая функция $f \colon E \to \ole{\R}$, причём верны условия:
	\begin{enumerate}
		\item $E$ --- измеримое множество конечной меры
		
		\item $\{f_m\}_{m = 1}^\infty$ --- неубывающая последовательность
		
		\item $\forall m \in \N\ \ f_m$ неотрицательна почти всюду на $E$
		
		\item $\forall m \in \N\ \ f_m$ измерима на $E$
		
		\item $\exists \lim_{m \to \infty} f_m(x) = f(x)$ почти всюду на $E$
	\end{enumerate}
	Тогда имеет место равенство:
	\[
		\int_E f(x)d\mu(x) = \lim_{m \to \infty} \int_E f_m(x)d\mu(x)
	\]
\end{theorem}

\begin{proof}
	Из $\forall m \in \N, x \in E\ \ f_m(x) \le f_{m + 1}(x)$ следует неравенство на интегралы:
	\[
		\int_E f_m(x)d\mu(x) \le \int_E f_{m + 1}(x)d\mu(x)
	\]
	В силу свойств последовательности и наличия поточечной сходимости, $f$ неотрицательна и измерима почти всюду на $E$, значит интегрируема по Лебегу на $E$. Разберём случаи:
	\begin{enumerate}
		\item $\int_E f(x)d\mu(x)$ оказался конечен. Тогда $f$ суммируема, и утверждение теоремы следует из теоремы Лебега о мажорируемой сходимости (требование мажорирования выполнено в силу монотонности, иначе быть просто не может).
		
		\item $\int_E f(x)d\mu(x) = +\infty$ Тогда мы можем записать интеграл через предел срезок:
		\[
			\int_E f(x)d\mu(x) = +\infty = \lim_{N \to \infty} \int_E f_{[N]}(x)d\mu(x)
		\]
		В другом виде это означает, что
		\[
			\forall K \in \R_{\ge 0}\ \exists N_0 \in \N \such \forall N \ge N_0\ \ \int_E f_{[N]}(x)d\mu(x) > K
		\]
		Дальнейшая идея состоит в том, чтобы подогнать уже срезки $f_{m, [N]}$ к $f_{[N]}$ и вывести предел тривиальным образом. Зафиксируем произвольное $N \in \N$. Тогда, срезкой $f_m$ будет такая функция:
		\[
			f_{m, [N]} = \System{
				&{N, f_m(x) > N}
				\\
				&{f_m(x), f_m(x) \le N}
			}
		\]
		Заметим, что $F(x) = N$ суммируема на $E$. Значит, мы можем применить теорему Лебега о мажорируемой сходимости и здесь:
		\[
			\lim_{m \to \infty} \int_E f_{m, [N]}(x)d\mu(x) = \int_E f_{[N]}(x)d\mu(x)
		\]
		Проанализируем полученное. Если зафиксировать $K \in \R_{\ge 0}$, то для него найдётся $N_0$ с соответствующим соотношением выше на интеграл от срезки $f$. Если мы рассмотрим какой-то $N \ge N_0$, то в силу сходимости
		\[
			\exists M \in \N \such \forall m \ge M\ \ \int_E f_{m, [N]}(x)d\mu(x) > K \Lora \int_E f_m(x)d\mu(x) > K
		\]
		Иначе говоря, $\lim_{m \to \infty} \int_E f_m(x)d\mu(x) = +\infty$, что и требовалось доказать.
	\end{enumerate}
\end{proof}

\begin{theorem} (Лемма Фату)
	Если имеется функциональная последовательность $\{f_m\}_{m = 1}^\infty$, $f_m \colon E \to \ole{\R}$, $f \colon E \to \ole{\R}$ и наложены следующие условия:
	\begin{enumerate}
		\item $E$ --- измеримое множество конечной меры
		
		\item $\forall m \in \N\ \ f_m$ --- неотрицательная почти всюду на $E$
		
		\item $\forall m \in \N\ \ f_m$ --- измеримая на $E$
		
		\item $f_m \to f$ поточечно почти всюду на $E$ 
	\end{enumerate}
	В таком случае, имеет место неравенство:
	\[
		\int_E f(x)d\mu(x) \le \varliminf_{m \to \infty} \int_E f_m(x)d\mu(x)
	\]
\end{theorem}

\begin{note}
	Прелесть леммы Фату состоит в том, что мы почти ничего не потребовали от $f$.
\end{note}

\begin{proof}
	Рассмотрим функциональную последовательность $\{g_m\}_{m = 1}^\infty$, где $g_m = \inf_{k \ge m} f_k(x)$. Что мы про неё знаем?
	\begin{itemize}
		\item $g_m$ неотрицательна почти всюду на $E$
		
		\item $g_m$ измерима на $E$
		
		\item $g_m(x) \le g_{m + 1}(x)$ для $\forall x \in E$, то есть последовательность неубывающая
		
		\item $\lim_{m \to \infty} g_m(x) = \lim_{m \to \infty} \inf_{k \ge m} f_k(x) = \varliminf_{m \to \infty} f_m(x) = \lim_{m \to \infty} f_m(x) = f(x)$ --- почти всюду на $E$
	\end{itemize}
	Совокупность данных свойств позволяет применить теорему Леви, отсюда равенство:
	\[
		\int_E f(x)d\mu(x) = \lim_{m \to \infty} \int_E g_m(x)d\mu(x)
	\]
	Заметим, что $\forall k \ge m$ имеет место неравенство $\int_E g_m(x)d\mu(x) \le \int_E f_k(x)d\mu(x)$. Стало быть
	\[
		\int_E g_m(x)d\mu(x) \le \inf_{k \ge m} \int_E f_k(x)d\mu(x)
	\]
	Ну а раз так, то можно навесить предел:
	\[
		\int_E f(x)d\mu(x) = \lim_{m \to \infty} \int_E g_m(x)d\mu(x) \le \lim_{m \to \infty} \inf_{k \ge m} \int_E f_k(x)d\mu(x) = \varliminf_{m \to \infty} \int_E f_m(x)d\mu(x)
	\]
\end{proof}

\begin{proposition} (Не лекторский материал, но помогает в обосновании далее)
	Пусть $E \subseteq \R^n$, $\mu(E) = +\infty$. Тогда $\exists \{E_m\}_{m = 1}^\infty$ такая, что она обладает следующими свойствами:
	\begin{enumerate}
		\item $\forall m \in \N\ E_m \subseteq E_{m + 1}$
		
		\item $\forall m \in \N\ \mu(E_m) < +\infty$
		
		\item $\lim_{m \to \infty} E_m = E$
	\end{enumerate}
\end{proposition}

\begin{proof}
	Для начала поймём, что теорема выполнена для всеобъемлющего множества $\R^n$ --- действительно, за $E_m$ можно взять такое множество:
	\[
		E_m = \{x \in \R^n \such |x| < m\}
	\]
	где $|x|$ --- обычная евклидова длина. Теперь же, кандидатом для произвольного $E \subseteq \R^n$ будет последовательность $E'_m = E \cap E_m$. Она подходит по нескольким причинам:
	\begin{enumerate}
		\item $\forall m \in \N\ E'_m \subseteq E'_{m + 1}$ --- тривиально
		
		\item Пересечение измеримых множеств измеримо, при этом $\mu(E'_m) \le \mu(E_m) < +\infty$
		
		\item $\lim_{m \to \infty} E'_m = \lim_{m \to \infty} (E_m \cap E) = \R^n \cap E = E$
	\end{enumerate}
\end{proof}

\begin{definition}
	Пусть $E \subseteq \R^n$, $\mu(E) = +\infty$. Рассмотрим последовательность множеств $\{E_m\}_{m = 1}^\infty$ такую, что $E_m \subseteq E_{m + 1}$, $\mu(E_m) < +\infty$ и $\lim_{m \to \infty} E_m = E$.
	
	Если $f \colon E \to \ole{\R}$ неотрицательная и измерима на $E$, то она \textit{интегрируема по Лебегу на множестве бесконечной меры}, причём интеграл равен следующему пределу:
	\[
		\int_E f(x)d\mu(x) := \lim_{m \to \infty} \int_{E_m} f(x)d\mu(x)
	\]
\end{definition}

\begin{theorem} (Корректность определения интеграла Лебега по множеству бесконечной меры)
	Если задана функция $f \colon E \to \ole{\R}$ и выполнены следующие требования:
	\begin{enumerate}
		\item $E$ --- измеримое множество \textbf{бесконечной меры}
		
		\item $f$ неотрицательна и измерима на $E$
	\end{enumerate}
	Тогда интеграл Лебега существует и не зависит от выбора приближающей последовательности $\{E_k\}_{k = 1}^\infty$.
\end{theorem}

\begin{proof}
	Так как $E_{m + 1} \supseteq E_m$, то
	\[
		\int_{E_{m + 1}} f(x)d\mu(x) = \int_{E_m} f(x)d\mu(x) + \underbrace{\int_{E_{m + 1} \bs E_m} f(x)d\mu(x)}_{\ge 0}
	\]
	То есть последовательность неотрицательных интегралов неубывает. Значит, у неё есть предел и он либо конечен, либо бесконечен. Пусть $\{E'_k\}_{k = 1}^\infty$ --- это другая приближающая последовательность. Предположим, что для них получаются разные интегралы (не умаляя общности, $a > b$):
	\[
		\lim_{m \to \infty} \int_{E_m} f(x)d\mu(x) = a > b \lim_{k \to \infty} \int_{E'_k} f(x)d\mu(x)
	\]
	Возьмём $c \in \R \colon b < c < a$ и устроим разбор случаев:
	\begin{itemize}
		\item $a < +\infty$ В таком случае $\exists m \in \N$, при котором $c < \int_{E_m} f(x)d\mu(x) < +\infty$. Значит $f$ суммируема на данном $E_m$. Так как $\lim_{k \to \infty} E'_k = E$, то $\lim_{k \to \infty} (E'_k \cap E_m) = E_m$. Непрерывность меры Лебега даёт из этого $\lim_{k \to \infty} \mu(E_m \bs (E'_k \cap E_m)) = 0$. А теперь распишем интеграл по $E_m$:
		\[
			\int_{E_m} f(x)d\mu(x) = \int_{E'_k \cap E_m} f(x)d\mu(x) + \int_{E_m \bs (E'_k \cap E_m)} f(x)d\mu(x) > c
		\]
		В силу абсолютной непрерывности, правый интеграл при $k \to \infty$ обращается в ноль. Значит
		\[
			\exists K \in \N \such \forall k \ge K \quad \int_{E'_k} f(x)d\mu(x) \ge \int_{E'_k \cap E_m} f(x)d\mu(x) > c > b
		\]
		Получили противоречие со значением предела.
		
		\item $a = +\infty$ По классике, пытаемся свести к предыдущему случаю. Для этого рассмотрим срезки $f_{[N]}(x)$. Так как снова $\exists m \in \N \such \int_{E_m} f(x)d\mu(x) > c$, то теперь уже
		\[
			\exists N \in \N \such \int_{E_m} f_{[N]}(x)d\mu(x) > c
		\]
		Повторяя рассуждения предыдущего случая для срезки, получим такое неравенство:
		\[
			\int_{E'_k} f_{[N]}(x)d\mu(x) > c
		\]
		Так как $f_{[N]}(x) \le f(x)$, то имеем искомое противоречие.
	\end{itemize}
\end{proof}

\begin{note}
	После всего вышесказанного, понятие интегрируемости по Лебегу на множестве бесконечной меры можно обобщить и до любой функции, ибо $f = f^+ - f^-$.
\end{note}

\begin{note}
	Очень важно заметить, что с таким определением интегрируемости по множеству бесконечной меры, теорема Леви будет работать и в случае, когда $E$ --- измеримое множество бесконечной меры, являющееся пределом какой-то неубывающей последовательности измеримых множеств конечной меры.
\end{note}