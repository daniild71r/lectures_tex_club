\begin{corollary}
	Большинство интегральных соотношений, похожих на теорему Стокса-Пуанкаре, являются просто её следствием:
	\begin{enumerate}
		\item $m = 2$, $n = 3$. Тогда формула из теоремы Стокса-Пуанкаре называется просто формулой Стокса
		
		\item $m = 3 = n$. В этом особом случае мы говорим о 2-формах. Обозначим одну из таких $\Omega$ и пусть она имеет следующий вид:
		\[
			\Omega = P dy \wedge dz + Q dz \wedge dx + R dx \wedge dy
		\]
		Тогда дифференциал может быть лаконично записан так:
		\[
			d\Omega = \ps{\pd{P}{x} + \pd{Q}{y} + \pd{R}{z}}dx \wedge dy \wedge dz
		\]
		Если подставить эти выражения в формулу Стокса-Пуанкаре, то получим \textit{формулу Гаусса-Остроградского в терминах дифференциальных форм}:
		\[
			\int_M \ps{\pd{P}{x} + \pd{Q}{y} + \pd{R}{z}} dx \wedge dy \wedge dz = \int_{\vdelta M} P dy \wedge dz + Q dz \wedge dx + R dx \wedge dy
		\]
		
		\item $m = 2 = n$. В этом случае у нас всего лишь 1-формы. Пусть она имеет такой вид:
		\[
			\Omega = Pdx + Qdy \Lora d\Omega = \ps{\pd{Q}{x} - \pd{P}{y}} dx \wedge dy
		\]
		Если подставить эти выражения в формулу Стокса-Пуанкаре, то получится \textit{формула Грина в терминах дифференциальных форм}:
		\[
			\int_{\vdelta M} Pdx + Qdy = \int_M \ps{\pd{Q}{x} - \pd{P}{y}} dx \wedge dy
		\]
	\end{enumerate}
\end{corollary}

\begin{note}
	На время забудем о всяких многообразиях, а вернёмся к простому евклидову пространству $\R^m$ со стандартным скалярным произведением $\tbr{\cdot, \cdot}$ и ортонормированным базисом $e_0$.
	
	Если я хочу выяснить объём призмы $\Pi$, натянутой на векторы $\{\vv{H}_i\}_{i = 1}^m$, то это записывается следующим образом:
	\[
		\mu(\Pi) = \vol(\vv{H}_1, \ldots, \vv{H}_m) = |\det(H_i^j)|
	\]
	где $\vv{H}_i = H_i^j \vv{e}_0^j$ (используем соглашение Эйнштейна). С другой стороны, в курсе алгебры мы говорили о матрице Грама для системы векторов, заданной следующим образом:
	\[
		\Gamma(\vv{H}_1, \ldots, \vv{H}_m) = \Matrix{
			&\ntbr{\vv{H}_1, \vv{H}_1}& &\cdots& &\ntbr{\vv{H}_1, \vv{H}_m}
			\\
			&\quad\;\ \vdots& &\ddots& &\quad\;\ \vdots
			\\
			&\ntbr{\vv{H}_m, \vv{H}_1}& &\cdots& &\ntbr{\vv{H}_m, \vv{H}_m}
		}
	\]
	В силу того, что мы используем стандартное скалярное произведение, эта матрица является произведением матрицы столбцов системы векторов на саму себя транспонированную. Иначе говоря, имеет место равенство:
	\[
		(\det(H_i^j))^2 = \det\Gamma(\vv{H}_1, \ldots, \vv{H}_m)
	\]
	Следовательно, искомый объём призмы можно записать ещё и так:
	\[
		\mu(\Pi) = \vol{\vv{H}_1, \ldots, \vv{H}_m} = |\det(H_i^j)| = \sqrt{\det\Gamma(\vv{H}_1, \ldots, \vv{H}_m)}
	\]
\end{note}

\begin{anote}
	Неформальное определение многообразия заключается в том, что это пространство, локально схожее с евклидовым (есть скалярное произведение в каждой точке в касательном пространстве, в нашем случае). Совершенно не обязательно, что многообразие вложено в некоторое всеобъёмлющее пространство. Поэтому, сохраняя определённую общность рассуждений, мы зададим скалярное произведение в каждой точке многообразия по-отдельности.
	
	Словестное определение \textit{размерности многообразия} такое: <<Размерностью многообразия называется размерность евклидова пространства, с которым оно локально сходно>>. Самым простым примером может послужить сфера, чья поверхность локально напоминает плоскость (в физическом проявлении --- наша планета Земля).
	
	Поэтому, как я понимаю, размерность многообразия в наших ситуациях соответствует размерности касательного пространства в каждой точке.
\end{anote}

\begin{definition}
	Элементарное многообразие $M \subseteq \R^n$ размерности $m$ называется \textit{римановым}, если задана непрерывная (по $x \in M$) положительно определенная билинейная форма $\ntbr{\cdot, \cdot}_x$ на каждом касательном пространстве $T(x)$ --- \textit{риманова метрика}.
\end{definition}

\begin{definition}
	\textit{(Римановым) объёмом призмы}, натянутой на вектора $\{\vv{K}_i\}_{i = 1}^m$ пространства $T(x)$, называется следующая величина:
	\[
		\vol(\vv{K}_1, \ldots, \vv{K}_m) = \sqrt{\det\ntbr{\vv{K}_i, \vv{K}_j}_x}
	\]
\end{definition}

\begin{definition}
	\textit{Формой ориентированного объёма на римановом ориентированном многообразии} $M$ называется такая $m$-форма $V$, что
	\[
		\forall \{\vv{G}_i\}_{i = 1}^m \subset T(x)\ \ V(x)(\vv{G}_1, \ldots, \vv{G}_m) = \pm \vol(\vv{G}_1, \ldots, \vv{G}_m)
	\]
	причём знак выбирается в соответствии с ориентацией базиса $(\vv{G}_1, \ldots, \vv{G}_m)$ пространства $T(x)$.
\end{definition}

\begin{anote}
	Естественно, в определении выше мы не требуем, что $\{\vv{G}_i\}_{i = 1}^m$ является базисом в $T(x)$. Однако, если это не базис, то это линейно зависимая комбинация векторов, чей объём обязательно ноль. Для нуля, понятное дело, знак смысла не имеет.
\end{anote}

\begin{proposition}
	Если $M \subseteq \R^n$ --- риманово многообразие размерности $m$ с римановой метрикой, индуцированной стандартным скалярным произведением в $\R^n$, а также $\phi$ --- положительная параметризация, то имеет место формула:
	\[
		V(x) = \sqrt{\det\tbr{\pd{\phi}{u_i}(u), \pd{\phi}{u_j}(u)}_x} \cdot \psi^*(du^1 \wedge \ldots \wedge du^m),\ x = \phi(u)
	\]
\end{proposition}

\begin{anote}
	Отмечу явный вид вектора в определителе:
	\[
		\pd{\phi}{u_i}(u) = \ps{\pd{\phi_1}{u_i}(u), \ldots, \pd{\phi_n}{u_i}(u)}^T
	\]
	Из того, что $x = \phi(u)$ следует, что весь квадратный корень является просто функцией, зависящей только от $x$.
\end{anote}

\begin{proof}
	Заметим, что при любом фиксированном $x = \phi(u)$ линейное пространство $m$-форм над $T(x)$ одномерно. Это значит, что форма ориентированного объёма может быть выражена через базисную форму и какой-то коэффициент. Например, так:
	\[
		V(x) = \alpha(x) \cdot \psi^*(du^1 \wedge \ldots \wedge du^m)(x)
	\]
	Посмотрим на $\{\vv{H}_i\}_{i = 1}^m$ --- ортонормированный базис в пространстве $\R^m$, двойственный к $\{du^i\}_{i = 1}^m$. Тогда у нас есть репер $\{\phi'(u)\vv{H}_i\}_{i = 1}^m$, чей конкретный набор векторов в $x = \phi(u)$ образует положительный базис (в силу определения положительной параметризации). С одной стороны (по определению):
	\[
		V(x)(\phi'(u)\vv{H}_1, \ldots, \phi'(u)\vv{H}_m) = \sqrt{\det\ntbr{\phi'(u)\vv{H}_i, \phi'(u)\vv{H}_j}_x}
	\]
	где отдельно взятый вектор $\phi'(u)\vv{H}_i$ будет просто вектором $\pd{\phi}{u_i}(u)$. С другой стороны, можем подставить эти вектора в найденный вид формы $V$:
	\begin{multline*}
		V(x)(\phi'(u)\vv{H}_1, \ldots, \phi'(u)\vv{H}_m) = \alpha(x) \cdot \psi^*(du^1 \wedge \ldots \wedge du^m)(x)(\phi'(u)\vv{H}_1, \ldots, \phi'(u)\vv{H}_m) =
		\\
		\alpha(x) \cdot (du^1 \wedge \ldots \wedge du^m)(\psi'(x)(\phi'(u)\vv{H}_1), \ldots, \psi'(x)(\phi'(u)\vv{H}_m))
	\end{multline*}
	Рассмотрим отдельно взятый аргумент. По сути у нас записана сложная производная от композиции $\psi \circ \phi$, а они обратны друг к другу:
	\[
		\psi'(x)(\phi'(u)\vv{H}_i) = \psi'(\phi(u))(\phi'(u)\vv{H}_i) = \vv{H}_i
	\]
	Стало быть, так как $\{\vv{H}_i\}_{i = 1}^m$ был ортонормированным базисом, то значение формы $du^1 \wedge \ldots \wedge du^m$ на нём будет просто единицей. Таким образом, мы нашли коэффициент $\alpha(x)$:
	\[
		\alpha(x) = \sqrt{\det\tbr{\pd{\phi}{u_i}(u), \pd{\phi}{u_j}(u)}_x}
	\]
\end{proof}

\begin{note}
	Для отрицательной параметризации нужно поставить знак минус перед радикалом.
\end{note}

\begin{definition}
	\textit{Интегралом от функции $f$ по римановой ориентированной клетке} $M$ называется следующий интеграл:
	\[
		\int_M f := \int_M fV = \int_K f(\phi(u))\sqrt{\det\tbr{\pd{\phi}{u_i}(u), \pd{\phi}{u_j}(u)}_{\phi(u)}}d\mu(u)
	\]
	где $K$ --- параметризующий куб, $\phi$ --- положительная параметризация.
\end{definition}

\begin{note}
	Несмотря на то что финальная формула зависит от параметризации, понятие формы ориентированного объёма определялось независимо от неё, а потому определение интеграла по клетке инвариантно.
\end{note}

\begin{definition}
	\textit{Мерой Лебега куска клетки} $M' \subseteq M$ называется следующий интеграл:
	\[
		\int_M \chi_{M'}(x)d\mu_M(x)
	\]
	где $\chi_{M'}(x)$ --- характеристическая функция множества $M'$, определенная на клетке:
	\[
		\chi_{M'}(x) = \System{
			&{1,\ x \in M'}
			\\
			&{0,\ x \in M \bs M'}
		}
	\]
\end{definition}

\begin{definition}
	Если $m = 1$, $n = 3$, то $M$ становится одномерной клеткой $M = \{\phi(u) \in \R^n \colon u \in [0; 1]\}$. \textit{Криволинейным интегралом 1-го рода} называется следующий интеграл:
	\[
		\int_M f(x, y, z)ds := \int_M f(x, y, z)d\mu_M
	\]
\end{definition}

\begin{note}
	Продолжим существовать в рамках предыдущего определения. Рассмотрим положительную параметризацию $\phi(u) = (x(u), y(u), z(u))^T$. Выясним явную формулу в рамках этой параметризации для криволинейного интеграла первого рода. Определитель вырождается в следующее:
	\[
		\det\tbr{\frac{d\phi}{du}(u), \frac{d\phi}{du}(u)}_{\phi(u)} = (x'(u))^2 + (y'(u))^2 + (z'(u))^2
	\]
	Получается вот такая позитивная формула:
	\[
		\int_M f(x, y, z)ds = \int_0^1 f(x(u), y(u), z(u))\sqrt{(x'(u))^2 + (y'(u))^2 + (z'(u))^2}d\mu(u)
	\]
	Причём, если $f = 1$, то значение интеграла является \textit{длиной соответствующей кривой}.
\end{note}

\begin{note}
	Аналогично определяется криволинейный интеграл 2-го рода при $m = 1$, $n = 2$.
\end{note}

\begin{definition}
	Если $m = 2$, $n = 3$, то $M$ становится двумерной клеткой $M$. \textit{Поверхностным интегралом 1-го рода} называется следующий интеграл:
	\[
		\iint_M f(x, y, z)dS := \int_M f(x, y, z)d\mu_M
	\]
\end{definition}

\begin{note}
	Если $m = 2$, $n = 3$, то $M$ --- двумерная клетка. Положительную параметризацию $\phi$ запишем так:
	\[
		\phi(u, v) = (x(u, v), y(u, v), z(u, v))^T
	\]
	Тогда $g(u, v)$ будет функцией матрицы, от которой мы хотим посчитать определитель:
	\[
		g(u, v) = \Matrix{
			&\tbr{\pd{\phi}{u}(u, v), \pd{\phi}{u}(u, v)}& &\tbr{\pd{\phi}{u}(u, v), \pd{\phi}{v}(u, v)}
			\\
			&\tbr{\pd{\phi}{v}(u, v), \pd{\phi}{u}(u, v)}& &\tbr{\pd{\phi}{v}(u, v), \pd{\phi}{v}(u, v)}
		}
	\]
	Карл Гаусс ввёл следующие обозначения для частей определителя этой формулы (скобки $(u, v)$ опускаются):
	\begin{align*}
		&{E = \tbr{\pd{\phi}{u}, \pd{\phi}{u}} = \ps{\pd{x}{u}}^2 + \ps{\pd{y}{u}}^2 + \ps{\pd{z}{u}}^2}
		\\
		&{G = \tbr{\pd{\phi}{v}, \pd{\phi}{v}} = \ps{\pd{x}{v}}^2 + \ps{\pd{y}{v}}^2 + \ps{\pd{z}{v}}^2}
		\\
		&{F = \tbr{\pd{\phi}{u}, \pd{\phi}{v}} = \pd{x}{u} \cdot \pd{x}{v} + \pd{y}{u} \cdot \pd{y}{v} + \pd{z}{u} \cdot \pd{z}{v}}
	\end{align*}
	Тогда $\det g(u, v) = EG - F^2$, а соответствующий поверхностный интеграл имеет такой вид:
	\[
		\iint_M f(x, y, z)dS = \iint_K f(x(u, v), y(u, v), z(u, v))\sqrt{EG - F^2}d\mu(u, v)
	\]
	Причём, если $f = 1$, то значение интеграла является \textit{площадью соответствующей поверхности}.
\end{note}

\begin{note}
	Далее, если не оговорено обратного, мы живём в ситуации $m = 2$, $n = 3$ с двумерным многообразием $M$. При наличии параметризации $\phi$ мы используем следующие обозначения для координат:
	\[
		\phi(u, v) = (x, y, z)
	\]
\end{note}

\begin{definition}
	Пусть $(e_1(r), e_2(r))$ --- положительный базис касательного пространства $T(r)$. \textit{Положительной нормалью к поверхности в точке} $r \in M$ называется единичный вектор $n$, обладающий двумя свойствами:
	\begin{enumerate}
		\item $n \bot T(r)$
		
		\item $(n, e_1(r), e_2(r))$ --- положительный базис в $\R^3$
	\end{enumerate}
\end{definition}

\textcolor{red}{Тут должна быть картинка многообразия и этих векторочков}

\begin{proposition}
	Пусть $M$ --- ориентированная двумерная клетка, $V$ --- соответствующая форма ориентированного объёма, $(\vv{i}, \vv{j}, \vv{k})$ --- орты в $\R^3$ (ортонормированный базис) и $\vv{n}$ --- вектор положительной нормали. Тогда в каждой точке $r \in M$ верны следующие формулы:
	\begin{align*}
		&{dx \wedge dy|_M = \cos(\vv{n}, \vv{k})V}
		\\
		&{dy \wedge dz|_M = \cos(\vv{n}, \vv{i})V}
		\\
		&{dz \wedge dx|_M = \cos(\vv{n}, \vv{j})V}
	\end{align*}
\end{proposition}

\begin{proof}
	Проверим лишь первое равенство, ибо остальные делаются аналогично. Пусть $\phi$ --- положительная параметризация. Имеет место эквивалентность:
	\[
		dx \wedge dy|_M = \cos(\vv{n}, \vv{k})V \Longleftrightarrow \phi^*(dx \wedge dy|_M) = \phi^*(\cos(\vv{n}, \vv{k})V)
	\]
	Зафиксируем произвольную точку $(x_0, y_0, z_0) = \phi(u_0, v_0)$ (чтобы не запутаться в обозначениях). С одной стороны:
	\begin{multline*}
		\phi^*(dx \wedge dy|_M)(x_0, y_0, z_0) = dx(u_0, v_0) \wedge dy(u_0, v_0) = \ps{\pd{x}{u}du + \pd{x}{v}dv} \wedge \ps{\pd{y}{u}du + \pd{y}{v}dv} =
		\\
		\ps{\pd{x}{u} \cdot \pd{y}{v} - \pd{x}{v} \cdot \pd{y}{u}}du \wedge dv
	\end{multline*}
	где производные взяты в точке $(u_0, v_0)$, естественно (у $du \wedge dv$ первый аргумент, коим должна быть та же точка, по общему соглашению опускаем).
	
	В силу положительной параметризации, вектора $\pd{\phi}{u}(u_0, v_0)$ и $\pd{\phi}{v}(u_0, v_0)$ образуют положительный базис касательного пространства $T(x_0, y_0, z_0)$. Тогда положительную нормаль $\vv{n}$ в этой точке можно записать известным соотношением:
	\[
		\vv{n} = \frac{\sbr{\pd{\phi}{u}, \pd{\phi}{v}}}{\md{\sbr{\pd{\phi}{u}, \pd{\phi}{v}}}}
	\]
	С другой стороны, посчитаем косинус и выпишем явно форму ориентированного объёма:
	\begin{itemize}
		\item \[
			\cos(\vv{n}, \vv{k}) = \frac{\ntbr{\vv{n}, \vv{k}}}{\md{\sbr{\pd{\phi}{u}, \pd{\phi}{v}}} \cdot 1} = \frac{1}{\md{\sbr{\pd{\phi}{u}, \pd{\phi}{v}}}} \cdot \ps{\pd{x}{u} \cdot \pd{y}{v} - \pd{x}{v} \cdot \pd{y}{u}}
		\]
		Явно векторное произведение, как мы знаем, можно записать таким мнемоническим определителем:
		\[
			\sbr{\pd{\phi}{u}, \pd{\phi}{v}} = \Det{
				&\vv{i}& &\vv{j}& &\vv{k}
				\\
				&\pd{x}{u}& &\pd{y}{u}& &\pd{z}{u}
				\\
				&\pd{x}{v}& &\pd{y}{v}& &\pd{z}{v}
			}
		\]
		А из линейной алгебры мы знаем, что $|[\vv{a}, \vv{b}]|^2 = \ntbr{\vv{a}, \vv{a}}\ntbr{\vv{b}, \vv{b}} - \ntbr{\vv{a}, \vv{b}}^2$.
		
		\item Последнее равенство в точности совпадает с $EG - F^2$, выражающим квадрат определителя матрицы Грама для этих базисных векторов. При этом
		\[
			V = \sqrt{EG - F^2} \psi^*(du \wedge dv)
		\]
	\end{itemize}
	Осталось собрать всё вместе:
	\[
		\cos(\vv{n}, \vv{k})V = \frac{1}{\sqrt{EG - F^2}} \cdot \ps{\pd{x}{u} \cdot \pd{y}{v} - \pd{x}{v} \cdot \pd{y}{u}} \cdot \sqrt{EG - F^2} \psi^*(du \wedge dv)
	\]
	Что произойдёт при переносе $\phi^*$ с этим выражением, тривиально.
\end{proof}

\begin{note}
	Начиная отсюда $m = 1$.
\end{note}

\begin{reminder}
	Кривые изучались в конце первого семестра. Все связанные определения можно найти в соответствующем конспекте.
\end{reminder}

\begin{proposition}
	Пусть $M$ --- одномерная ориентированная клетка, $V$ --- соответствующая форма ориентированного объёма, $(\vv{i}, \vv{j}, \vv{k})$ --- орты в $\R^3$ и $\vv{\tau}$ --- вектор положительной единичной касательной к $M$. Тогда, имеют место следующие формулы:
	\begin{align*}
		&{dx|_M = \cos(\vv{\tau}, \vv{i})V}
		\\
		&{dy|_M = \cos(\vv{\tau}, \vv{j})V}
		\\
		&{dz|_M = \cos(\vv{\tau}, \vv{k})V}
	\end{align*}
\end{proposition}

\begin{proof}
	Повторим аналогичные действия, как и в предыдущем утверждении, например, для $dx|_M$. Теперь положительная параметризация $\phi$ удобно запишется так:
	\[
		\phi(t) = (x, y, z)
	\]
	Для конкретной точки $\phi(t_0) = (x_0, y_0, z_0)$ получим значение переноса формы:
	\[
		\phi^*(dx|_M)(x_0, y_0, z_0) = dx(t_0) = x'(t_0)dt
	\]
	При положительной параметризации вектор единичной касательной выражается как нормирванный положительный репер. Подойдёт такой:
	\[
		\vv{\tau}(t_0) = \frac{x'(t_0)\vv{i} + y'(t_0)\vv{j} + z'(t_0)\vv{k}}{\sqrt{(x'(t_0))^2 + (y'(t_0))^2 + (z'(t_0))^2}}
	\]
	Далее делаем те же самые действия.
\end{proof}

\begin{definition}
	Пусть $\vv{A}(r) = (P(r), Q(r), R(r))^T$ --- векторное поле, заданное на одномерной клетке $M$. Тогда \textit{криволинейным интегралом 2-го рода} называется следующий интеграл:
	\[
		\int_M \vv{A}^\# = \int_M Pdx + Qdy + Rdz
	\]
	где $dx = dx|_M$, $dy = dy|_M$ и $dz = dz|_M$.
\end{definition}

\begin{corollary} (из последнего утверждения)
	Если нужно посчитать криволинейный интеграл второго рода для векторного поля, то
	\[
		\int_M Pdx + Qdy + Rdz = \int_M (P\cos(\vv{\tau}, \vv{i}) + Q\cos(\vv{\tau}, \vv{j}) + R\cos(\vv{\tau}, \vv{k})) = \int_M \ntbr{\vv{A}, \vv{\tau}}ds
	\]
	Последнее выражение очень важно в физике, ибо является \textit{работой силы $\vv{A}$ по кривой $M$}.
\end{corollary}

\begin{definition}
	Если $\vv{A}(r)$ --- векторное поле, а $M$ --- замкнутая кривая, то криволинейный интеграл второго рода
	\[
		\int_M \ntbr{\vv{A}, \vv{\tau}}ds
	\]
	называется \textit{циркуляцией $A$ вдоль $M$}.
\end{definition}

\begin{note}
	Здесь мы снова возвращаемся к $m = 2$.
\end{note}

\begin{definition}
	Пусть $\vv{A} = (P, Q, R)^T$ --- векторное поле на двумерной клетке $M$. \textit{Поверхностным интегралом 2-го рода} называется следующий интеграл:
	\[
	\iint_M Pdydz + Qdzdx + Rdxdy := \int_M *\vv{A}^\# = \int_M Pdy \wedge dz + Q dz \wedge dx + Rdx \wedge dy
	\]
\end{definition}

\begin{corollary} (из предпоследнего утверждения)
	Если нужно посчитать поверхностный интеграл второго рода, то это можно сделать так:
	\begin{multline*}
	\iint_M Pdydz + Qdzdx + Rdxdy =
	\\
	\iint_M (P\cos(\vv{n}, \vv{i}) + Q\cos(\vv{n}, \vv{j}) + R\cos(\vv{n}, \vv{k}))dS = \iint_M \ntbr{\vv{A}, \vv{n}}dS
	\end{multline*}
\end{corollary}

\begin{definition}
	Если $\vv{A}$ --- векторное поле, а $M$ --- двумерная клетка, то поверхностный интеграл второго рода
	\[
	\iint_M \ntbr{\vv{A}, \vv{n}}dS
	\]
	называется \textit{потоком поля $\vv{A}$ через $M$}
\end{definition}

\begin{anote}
	Когда говорят об интегрировании на замкнутых объектах (замкнутая кривая, замкнутая поверхность), то используют значок $\oint$ и соответствующие с двумя/тремя интегралами.
\end{anote}

\begin{reminder}
	Нормаль, о которой мы говорили в теореме Стокса-Пуанкаре для куба, является \textit{внешней}. В более общем случае, это просто единичная нормаль, выходящая \textit{из} какого-то трехмерного объекта.
\end{reminder}

\begin{note}
	Отсюда $m = n = 3$.
\end{note}

\begin{lemma}
	Если $M$ --- трёхмерная клетка, причём граница $\vdelta M$ ориентирована по правилу выходящего вектора, то положительная нормаль клетки совпадает с внешней нормалью.
\end{lemma}

\textcolor{red}{Тут снова должна быть картинка, в этот раз с кубок и какой-то желешкой.}

\textcolor{red}{Дальнейшие рассуждения некорректны.}
