\subsection{Основные свойства интеграла Лебега}

\begin{anote}
	Если мы, например, заявляем, что $f \colon E \to \R$ --- суммируемая функция на $E$, то надо понимать, что тогда $f$ обязана своими свойствами подходить под хотя бы одно определение суммируемой функции. В частности, для всех классов функций, для которых мы вводили понятие суммирования, мы требовали измеримость и определенность на $E$. Стало быть, если говорится, что $f$ суммируема на $E$, то она по условию уже измерима и определена на этом множестве.
\end{anote}

\begin{theorem} (Линейность и монотонность интеграла Лебега)
	Имеют место 2 свойства:
	\begin{enumerate}
		\item (Линейность) Если $f_1$ и $f_2$ суммируемы на $E$, то $\forall c_1, c_2 \in \R$ функция $c_1 f_1 + c_2 f_2$ тоже суммируема на $E$, причём
		\[
			\int_E (c_1f_1(x) + c_2f_2(x))d\mu(x) = c_1\int_E f_1(x)d\mu(x) + c_2\int_E f_2(x)d\mu(x)
		\]
		
		\item (Монотонность) Если функции $f_1, f_2$ суммируемы на $E$ и $\forall x \in E\ \ f_1(x) \le f_2(x)$, то
		\[
			\int_E f_1(x)d\mu(x) \le \int_E f_2(x)d\mu(x)
		\]
	\end{enumerate}
\end{theorem}

\begin{proof}~
	\begin{enumerate}
		\item Для доказательства линейности, достаточно рассмотреть доказать 2 факта:
		\begin{itemize}
			\item Линейность при $c_1 = c_2 = 1$. Разберём случаи:
			\begin{enumerate}
				\item $f_1, f_2 \colon E \to \R$ --- ограниченные или неотрицательные и неограниченные. Тогда по эквивалентному условию суммирования
				\[
					\forall \eps > 0\ \exists P_1(E), P_2(E) \such U(P_1, f_1) - L(P_1, f_1) < \eps;\ \ U(P_2, f_2) - L(P_2, f_2) < \eps
				\]
				Положим $P := P_1 \cup P_2$ и увидим следующую цепочку неравенств:
				\[
					L(P, f_1) + L(P, f_2) \le L(P, f_1 + f_2) \le U(P, f_1 + f_2) \le U(P, f_1) + U(P, f_2)
				\]
				Отсюда понятным образом $f_1 + f_2$ тоже суммируется на $E$. Так как интегралы зажаты между соответствующими суммами (из-за суммируемости они конечны), то предельный переход даёт нужное неравенство.
				
				\item $f_1, f_2 \colon E \to \R$ --- произвольные функции. Тогда $f_1 = f_1^+ - f_1^-$, $f_2 = f_2^+ - f_2^-$ --- все эти функции суммируются на $E$. Тогда
				\[
					f_1 + f_2 = f_1^+ - f_1^- + f_2^+ - f_2^- = (f_1 + f_2)^+ - (f_1 + f_2)^-
				\]
				Левая часть равенства $(f_1 + f_2)^+ = f_1^+ + f_2^+$, суммируется на $E$, потому что это уже доказано для правой. Аналогично про $(f_1 + f_2)^-$. Стало быть и $f_1 + f_2$ суммируется на $E$. Раз так, то по определению
				\[
					\int_E (f_1 + f_2)(x)d\mu(x) = \int_E (f_1 + f_2)^+ d\mu(x) - \int_E (f_1 + f_2)^- d\mu(x)
				\]
				Более того, для интегралов справа по предыдущему пункту уже доказана дистрибутивность интеграла по сложению (это и есть то, что получается при $c_1 = c_2 = 0$), поэтому
				\begin{multline*}
					\int_E (f_1 + f_2)^+ d\mu(x) - \int_E (f_1 + f_2)^- d\mu(x) =
					\\
					\int_E f_1^+ d\mu(x) + \int_E f_2^+ d\mu(x) - \int_E f_1^- d\mu(x) - \int_E f_2^- d\mu(x) =
					\\
					\ps{\int_E f_1^+ d\mu(x) - \int_E f_1^- d\mu(x)} + \ps{\int_E f_2^+ d\mu(x) - \int_E f_2^- d\mu(x)}
				\end{multline*}
				Выражения в скобках по определению равны интегралам от $f_1$ и $f_2$ соответственно.
			\end{enumerate}
		
			\item Вынесение константы за знак интеграла. Для $c = 0$ всё тривиально верно, а для остальных значений сделаем разбор случаев:
			\begin{enumerate}
				\item $f \colon E \to \R$ --- ограниченная или неотрицательная неограниченная. Тогда заметим следующие соотношения при любом разбиении $P(E)$:
				\begin{align*}
					&{\forall c > 0\ L(P, cf) = cL(P, f);\ U(P, cf) = cU(P, f)}
					\\
					&{\forall c < 0\ L(P, cf) = cU(P, f);\ U(P, cf) = cL(P, f)}
				\end{align*}
				Отсюда по эквивалентному свойству суммирования следует суммирование $cf$ на $E$, ну и в силу соотношений между суммами Дарбу-Лебега равенство между значениями интегралов тоже тривиально.
				
				\item $f \colon E \to \R$ --- произвольная функция. Тогда мы просто замечаем, что $(cf)^+ = cf^+$ и $(cf)^- = cf^-$ для $c \in \R$.
			\end{enumerate}
		\end{itemize}
	
		\item Здесь классический приём с разбором разных $f_1$:
		\begin{itemize}
			\item Если $f_1 = 0$, то интеграл тривиально принимает значение ноль, $f_2$ оказывается неотрицательной функцией, для которой неравенство очевидно.
			
			\item Если $f_1 \neq 0$, то сведём ситуацию к первой, рассмотрев функцию $\forall x \in E\ f(x) = f_2(x) - f_1(x) \ge 0$
		\end{itemize}
	\end{enumerate}
\end{proof}

\begin{definition}
	Пусть $f \colon E \to \R$ --- неотрицательная функция. Тогда определим функцию $f_{[N]}(x) \colon E \to \R$, называемую \textit{срезкой}, следующим образом:
	\[
		f_{[N]}(x) = \System{
			&{f(x),\ f(x) \le N}
			\\
			&{N,\ f(x) > N}
		}
	\]
\end{definition}