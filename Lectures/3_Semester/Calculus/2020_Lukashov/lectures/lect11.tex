\begin{Def}
	Векторным полем на $D$ называется функция $\vec{a}: D\to\R^n$.\\ $\vec{a}=a^1\vec{e_1}+\ldots+a^n\vec{e_n}$.
	Если $(\vec{e_1}, \ldots, \vec{e_n})$ --- ортонормированный базис евклидова пространства $\R^n$, то операции соответствия векторных полей и дифференциальных форм валентности $n$ определяются так: $(\vec{a})^\sharp=a_1(x)dx^1+\ldots+a_n(x)dx^n$ и для $\Omega(x)=w_1(x)dx^1+\ldots+w_n(x)dx^n$, $(\Omega)^\flat=w^1(x)\vec{e_1}+\ldots+w^n(x)\vec{e_n}$, где $a_i(x)=a^i(x), w_i(x)=w^i(x), i=1,\ldots, n, (dx^1, \ldots, dx^n)$ --- сопряженный базис к $(\vec{e_1}, \ldots, \vec{e_n})$.
	
	Аналогично определяются операции соответствия для поливекторных полей и дифференциальных форм произвольной валентности.
\end{Def}

\subsubsection{Основные операции теории поля.}
По сути все основные операции теории поля --- это дифференцирование дифференциальных форм.

Начнем с $0$-формы $f(x)$ в $\R^n:$
\begin{align*}
	df&=\dfrac{\partial f}{\partial x^1}dx^1+\ldots+\dfrac{\partial f}{\partial x^n}dx^n \\
	(df)^\flat&= \dfrac{\partial f}{\partial x^1}e_1+\ldots+\dfrac{\partial f}{\partial x^n}e_n=\grad f
\end{align*}

$1$-форма в $\R^3$:
$$
	(\vec{a})^\sharp= a_1(x)dx^1+a_2(x)dx^2+a_3dx^3=Pdx+Qdy+Rdz
$$
\begin{multline*}
	d(\vec{a})^\sharp=dP\wedge dx+dQ\wedge dy+dR\wedge dz=
	\\
	=\left(\dfrac{\partial P}{\partial x}dx+\dfrac{\partial P}{\partial y}dy+\dfrac{\partial P}{\partial z}dz \right)\wedge dx+\\+\left(\dfrac{\partial Q}{\partial x}dx+\dfrac{\partial Q}{\partial y}dy+\dfrac{\partial Q}{\partial z}dz \right)\wedge dy+\\+\left(\dfrac{\partial R}{\partial x}dx+\dfrac{\partial R}{\partial y}dy+\dfrac{\partial R}{\partial z}dz \right)\wedge dz=
	\\
	=\left(\dfrac{\partial R}{\partial y}-\dfrac{\partial Q}{\partial z}\right)dy\wedge dz+\left(\dfrac{\partial P}{\partial z}-\dfrac{\partial R}{\partial x}\right)dz\wedge dx+\left(\dfrac{\partial Q}{\partial x}-\dfrac{\partial P}{\partial y}\right)dx\wedge dy
\end{multline*}
$$*d(\vec{a})^\sharp = \left(\dfrac{\partial R}{\partial y}-\dfrac{\partial Q}{\partial z}\right)dx+\left(\dfrac{\partial P}{\partial z}-\dfrac{\partial R}{\partial x}\right)dy+\left(\dfrac{\partial Q}{\partial x}-\dfrac{\partial P}{\partial y}\right)dz$$
$$(*d(\vec{a})^\sharp)^\flat = \left(\dfrac{\partial R}{\partial y}-\dfrac{\partial Q}{\partial z}\right)\vec{i}+\left(\dfrac{\partial P}{\partial z}-\dfrac{\partial R}{\partial x}\right)\vec{j}+\left(\dfrac{\partial Q}{\partial x}-\dfrac{\partial P}{\partial y}\right)\vec{k}:=\rot\vec{a}$$

$(n-1)$-форма в $\R^n$:

$*(\vec{a})^\sharp=*(a_1(x)dx^1+\ldots+a_n(x)dx^n)=a_1(x)dx^2\wedge\ldots\wedge dx^n+\ldots+(-1)^{n-1}a_n(x)dx^1\wedge\ldots\wedge dx^{n-1}$.
\begin{multline*}
	d(*(\vec{a})^\sharp)=da_1(x)\wedge dx^2\wedge\ldots\wedge dx^n+\ldots+(-1)^{n-1}da_n(x)\wedge dx^1\wedge\ldots\wedge dx^{n-1}=\\=\dfrac{\partial a_1(x)}{\partial x^1}dx^1\wedge\ldots\wedge dx^n+\ldots+\dfrac{\partial a_n(x)}{\partial x^n}dx^1\wedge\ldots\wedge dx^n=\left(\dfrac{\partial a_1(x)}{\partial x^1}+\ldots + \dfrac{\partial a_n(x)}{\partial x^n}\right)dx^1\wedge\ldots\wedge dx^n.
\end{multline*}
$*d(*(\vec{a})^\sharp)=\dfrac{\partial a_1(x)}{\partial x^1}+\ldots+\dfrac{\partial a_n(x)}{\partial x^n}=\di\vec{a}$ --- дивергенция.

\begin{example}
	\begin{align*}
		\rot(\grad f)&=(*(d(\grad f)^\sharp))^\flat=(*(d(df)))^\flat=0\\
		\di(\rot\vec{a})&=*(d(*(\rot\vec{a})^\sharp))=*(d(**d(\vec{a}^\sharp)))=0
	\end{align*}
\end{example}

Векторный дифференциальный оператор набла: \fbox{$\nabla:=\left(\dfrac{\partial}{\partial x},\dfrac{\partial}{\partial y}, \dfrac{\partial}{\partial z}\right)$}.

\begin{align*}
	\grad f&= \left(\dfrac{\partial f}{\partial x}, \dfrac{\partial f}{\partial y}, \dfrac{\partial f}{\partial z}\right)=\nabla f\\
	\di\vec{a}&=\dfrac{\partial P}{\partial x}+\dfrac{\partial Q}{\partial y}+\dfrac{\partial R}{\partial z}=(\nabla, \vec{a})\\
	\rot\vec{a}&=[\nabla, \vec{a}]=
	\begin{vmatrix}
		\vec{i} & \vec{j} & \vec{k}\\
		\frac{\partial}{\partial x} & \frac{\partial}{\partial y} & \frac{\partial}{\partial z}\\
		P & Q & R
	\end{vmatrix}
\end{align*}


\begin{Def}
	$p$-форма $\Omega$ называется замкнутой, если $d\Omega=0$. $p$-форма $\Omega$ называется точной, если существует $(p-1)$-форма $\Pi$, такая что $\Omega=d\Pi$.
\end{Def}

\begin{corollary}
	Каждая точная форма замкнута. 
\end{corollary}

\begin{proof}
	Пусть $\Omega$ --- точная форма. Следовательно $\Omega=d\Pi\Rightarrow d\Omega=d(d\Pi)=0\Rightarrow\Omega$ --- замкнутая.
\end{proof}

\begin{Def}
	Область $D\subset\R^n$ называется звездной, если $\exists x_0\in D$, такое, что \\$ \varphi(x, t)=x_0+(1-t)(x-x_0)$, непрерывное отображение из $D\times[0,1]$ в $D$ и такое, что $\varphi(x,0)=x\ \ \forall x\in D, \varphi(x,1)=x_0\ \ \forall x\in D.$
	
	$\varphi$ --- называется прямым стягиванием $D$ в точку $x_0$.
\end{Def}

\subsubsection{Операция замены переменных в дифференциальной форме.}
\begin{Def}
	Пусть $\Omega(x)$ --- дифференциальная $p$-форма в области $U\subset\R^n, \varphi:V\to U$ --- диффеоморфизм области $V\subset\R^n$ на $U, x=\varphi(x).
	\\ \varphi^*\Omega(y)$ --- дифференциальная $p$-форма в области $V$, определяемая на любом наборе $p$ векторов из $\R^n, b_1,\ldots, b_p$, как  $\varphi^*\Omega(y)(b_1, \ldots, b_p)=\Omega(\varphi(y))(\varphi'(y)b_1, \ldots, \varphi'(y)b_p)$, где $\varphi'(y)$ --- это матрица Якоби отображения $\varphi$.
\end{Def} 

Пусть $\Omega(x)=w(x)dx^{i_1}\wedge\ldots\wedge dx^{i_p}$.

$(\varphi^*\Omega)(y)(b_1,\ldots, b_p)=w(\varphi(y))dx^{i_1}\wedge\ldots\wedge dx^{i_p}(\varphi'(y)b_1,\ldots, \varphi'(y)b_p)=w(\varphi(y))\det((\varphi'(y)b_j)^{i_k})_{j=1,k=1}^{p,p}$

Проверим, что $(\varphi^*\Omega)(y)=w(\varphi(y))d\varphi^{i_1}(y)\wedge\ldots\wedge d\varphi^{i_p}(y)$.

Подсчитаем 
\begin{multline*}
	w(\varphi(y))d\varphi^{i_1}(y)(b_1, \ldots, b_p)=w(\varphi(y))det(d\varphi^{i_k}(y)(b_j))_{j=1,k=1}^{p,p}=\\=w(\varphi(y))\det((\varphi'(y)d_j)^{i_k})_{j=1,k=1}^{p,p}\text{, так как }\\d\varphi^{i_k}(y)(b_j) = \sum\limits_{l=1}^n\dfrac{\partial \varphi^{i_k}}{\partial y^l}(y)(b_j^l)=(\varphi'(y)b_j)^{i_k}.
\end{multline*}

Правило подсчета $\varphi^*:$ если $\Omega(x)=\sum\limits_{1\leqslant i_1<\ldots<i_p\leqslant n}w_{i_1\ldots i_p}(x)dx^{i_1}\wedge\ldots\wedge dx^{i_p}$, то\\ 
$\varphi^*\Omega(y)=\sum\limits_{1\leqslant i_1<\ldots<i_p\leqslant n}w_{i_1\ldots i_p}(\varphi(y))d\varphi^{i_1}(y)\wedge\ldots\wedge d\varphi^{i_p}(y)$.
\subsubsection{Свойства операции замены переменных.}

\begin{enumerate}
	\item $\varphi^*(\alpha\Omega+\beta\Pi)=\alpha\varphi^*\Omega+\beta\varphi^*\Pi$
	\item
	$\varphi^*(\Omega\wedge\Pi)=\varphi^*(\Omega)\wedge\varphi^*(\Pi)$
	\item
	$\varphi^*(d\Omega)=d(\varphi^*\Omega)$
	\item 
	$\varphi^*\psi^*\Omega=(\varphi\psi)^*\Omega$
\end{enumerate}

\begin{proof}\ 
	\begin{enumerate}
		\item очевидно.
		\item\begin{multline*}
			\varphi(\Omega\wedge\Pi)=\\=\varphi^*\left(\sum\limits_{1\leqslant i_1<\ldots<i_p\leqslant n}\sum\limits_{1\leqslant j_1<\ldots<j_q\leqslant n}w_{i_1\ldots i_p}(x)\varpi_{j_1\ldots j_q}(x)dx^{i_1}\wedge\ldots\wedge dx^{i_p}\wedge dx^{j_1}\wedge\ldots\wedge dx^{j_q}\right)=\\=\sum\limits_{1\leqslant i_1<\ldots<i_p\leqslant n}\sum\limits_{1\leqslant j_1<\ldots<j_q\leqslant n}w_{i_1\ldots i_p}(\varphi(y))\varpi_{j_1\ldots j_q}(\varphi(y))d\varphi^{i_1}(y)\wedge\ldots\wedge d\varphi^{i_p}(y)\wedge \\ \wedge d\varphi^{j_1}(y)\wedge\ldots\wedge d\varphi^{j_q}(y)=\varphi^*(\Omega)\wedge \varphi^*(\Pi).
		\end{multline*} 
		\item 
		\begin{multline*}
			d\Omega(x)=\sum\limits_{1\leqslant i_1<\ldots<i_p\leqslant n}dw_{i_1\ldots i_p}(x)\wedge dx^{i_1}\wedge\ldots\wedge dx^{i^p}=
			\\
			=\sum\limits_{1\leqslant i_1<\ldots<i_p\leqslant n}\sum\limits_{k=1}^n\dfrac{\partial w_{i_1\ldots i_p}}{\partial x^k}(x)dx^{k}\wedge dx^{i_1}\wedge\ldots\wedge dx^{i_p}
		\end{multline*}
		$\varphi^*d\Omega(x)=\sum\limits_{1\leqslant i_1<\ldots<i_p\leqslant n}\sum\limits_{k=1}^n\dfrac{\partial w_{i_1\ldots i_p}}{\partial x^k}(\varphi(y))d\varphi^k(y)\wedge d\varphi^{i_1}(y)\wedge\ldots\wedge d\varphi^{i_p}(y)$
		
		$\varphi^*\Omega(y) = \sum\limits_{1\leqslant i_1<\ldots<i_p\leqslant n} w_{i_1\ldots i_p}(\varphi(y))d\varphi^{i_1}(y)\wedge\ldots\wedge d\varphi^{i_p}(y)$
		\begin{multline*}
			d\varphi^*\Omega(y)=\sum\limits_{1\leqslant i_1<\ldots<i_p\leqslant n}(dw_{i_1\ldots i_p}(\varphi(y))d\varphi^{i_1}(y)\wedge\ldots\wedge d\varphi^{i_p}(y)+\\+w_{i_1\ldots i_p}(\varphi(y))\underbrace{d(d\varphi^{i_1}(y)\wedge\ldots\wedge d\varphi^{i_p})}_{=0})
		\end{multline*}
	\item $\Omega=w(x)dx^{i_1}\wedge\ldots\wedge dx^{i_p}$ \\
	$\psi^*\Omega=w(\psi(y))d\psi^{i_1}(y)\wedge\ldots\wedge d\psi^{i_p}(y)$ \\
	$\varphi^*\psi^*\Omega=w(\psi(\varphi(z)))d\psi^{i_1}(\varphi(z))\wedge\ldots\wedge d\psi^{i_p}(\varphi(z))=(\varphi\psi)^*\Omega$
	\end{enumerate}
\end{proof}

\begin{Def}
	Область $G\subset\R^n$ называется звездообразной, если она является диффеоморфным образом звездной области.
\end{Def}

Если $\Omega$ --- дифференциальная форма в $D\times[0,1]$, то $\varphi^*\Omega$--- дифференциальная форма в $D$, где $\varphi$ --- прямое стягивание звездной области $D$, то $p$-форма $\Omega$ состоит из слагаемых вида $a(x,t)dx^{i_1}\wedge\ldots\wedge dx^{i_p}$ и $b(x,t)dt\wedge dx^{i_1}\wedge\ldots\wedge dx^{i_{p-1}}$.

$\Omega(x,t_0)$ при фиксированном $t_0$ будем считать дифференциальной формой на $D, dt\equiv 0$.

\begin{lemma}
	Если $\Omega$ --- гладкая $p$-форма в $D\times[0,1]$, то $(dK\Omega+Kd\Omega)(x)=\Omega(x,1)-\Omega(x,0)$, где $K$ --- линейная операция, заданная на базисных слагаемых как $K(a(x,t)dx^{i_1}\wedge\ldots\wedge dx^{i_p})=0,K(b(x,t)dt\wedge dx^{i_1}\wedge\ldots\wedge dx^{i_{p-1}})=\left(\int\limits_{0}^1b(x,t)d\mu(t)\right)\wedge dx^{i_1}\wedge\ldots\wedge dx^{i_{p-1}}$.
\end{lemma}

\begin{proof}\ 
	\begin{enumerate}
		\item На $a(x,t)dx^{i_1}\wedge\ldots\wedge dx^{i_{p}}: K\Omega=0, \\d\Omega=\dfrac{\partial a(x,t)}{\partial t}dt\wedge dx^{i_1}\wedge\ldots\wedge dx^{i_{p}}+\sum\limits_{k=1}^n\dfrac{\partial a(x,t)}{\partial x^k}dx^k\wedge dx^{i_1}\wedge\ldots\wedge dx^{i_{p}}$;\\ $Kd\Omega=\left(\int\limits_{0}^1\dfrac{\partial a(x,t)}{\partial t}d\mu(t)\right)dx^{i_1}\wedge\ldots\wedge dx^{i_p}=(a(x, 1)-a(x,0))dx^{i_1}\wedge\ldots\wedge dx^{i_p}\Rightarrow\\ (dK\Omega+kd\Omega)(x)=\Omega(x,1)-\Omega(x,0)$.
		\item При $b(x,t)dt\wedge dx^{i_1}\wedge\ldots\wedge dx^{i_{p-1}}=\Omega:\\ K\Omega=\left(\int\limits_{0}^1b(x,t)d\mu(x)\right)\wedge dx^{i_1}\wedge\ldots\wedge dx^{i_{p-1}}=\sum\limits_{k=1}^n \int\limits_{0}^1 \dfrac{\partial b(x,t)}{\partial x^k}d\mu(t)dx^k\wedge dx^{i_1}\wedge\ldots\wedge dx^{i_{p-1}}$
		$d\Omega=db(x,t)\wedge dt\wedge dx^{i_1}\wedge\ldots\wedge dx^{i_{p-1}}=\sum\limits_{k=1}^n\dfrac{\partial b(x,t)}{\partial x^k}dx^k\wedge dx^{i_1}\wedge\ldots\wedge dx^{i_{p-1}}=\\=-\sum\limits_{k=1}^n\dfrac{\partial b(x,t)}{\partial x^k}dt\wedge dx^k\wedge dx^{i_1}\wedge\ldots\wedge dx^{i_{p-1}}$\\
		$Kd\Omega=-\sum\limits_{k=1}^n\left(\int\limits_{0}^1 \dfrac{\partial b(x,t)}{\partial x^k}d\mu(t)\right)dx^k\wedge dx^{i_1}\wedge\ldots\wedge dx^{i_{p-1}}$\\
		$dK\Omega+Kd\Omega=0=\Omega(x,1)-\Omega(x,0)$
	\end{enumerate}
\end{proof}

\begin{theorem}(Лемма Пуанкаре)
	Каждая замкнутая в звездообразной области $D$ гладкая форма точна в ней.
\end{theorem}

\begin{proof}
	Пусть $D$ --- звездная область, $\varphi$ --- прямое стягивание. Рассмотрим $\varphi^*\Omega$ в $D\times[0,1]$. $\Omega$ --- замкнута $\Rightarrow d\Omega=0\Rightarrow d\varphi^*\Omega=\varphi^*d\Omega=0\Rightarrow \varphi^*\Omega$ --- замкнута в $D\times[0,1]$. Тогда по лемме $dK\varphi^*\Omega=-Kd\varphi^*\Omega+\varphi^*\Omega(x,1)-\varphi^*\Omega(x,0)$.
	
	$\Omega=\sum\limits_{1\leqslant i_1 < \ldots < i_p \leqslant n}w_{i_1\ldots i_p}(y)dy^{i_1}\wedge\ldots\wedge dy^{i_p}$, заменим переменную $y=\varphi(x,t)$,
	
	$\varphi^*\Omega=\sum\limits_{1\leqslant i_1 < \ldots < i_p \leqslant n} w_{i_1\ldots i_p}(\varphi(x,t))d\varphi^{i_1}(x,t)\wedge\ldots\wedge d\varphi^{i_p}(x,t)$,
	
	$d\varphi^{j}(x,t)=-(x^j-x_0^j)dt+(1-t)dx^j$. Считаем $dt\equiv0$, значит все слагаемые в которых появятся $dt$ обнулятся.
	
	$\varphi^*\Omega(x,1) =0$
	
	$\varphi^*\Omega(x,0)=\sum\limits_{1\leqslant i_1 < \ldots < i_p \leqslant n}w_{i_1\ldots i_p}(x)dx^{i_1}\wedge\ldots\wedge dx^{i_p}=\Omega(x)$. 
	
	Итого $dK\varphi^*\Omega=-\Omega(x)\Rightarrow \Omega(x)=d\Pi(x)$, где $\Pi(x)=-K\varphi^*\Omega$
	
	Пусть $D=\psi(G)$, где $G$ --- звездная область, $\psi$ --- диффеоморфизм. В $D$ есть замкнута форма $\Omega$. Рассмотрим $\psi^*\Omega$ --- форма в $G$, она замкнута, так как $d\psi^*\Omega=\psi^*d\Omega=0$. $G$ --- звездная область и по уже доказанному $\Pi=-K\varphi^*\psi^*\Omega$ является первообразной, то есть $d\Pi=\psi^*\Omega$. Тогда для $(\psi^{-1})^*\Pi$ справедливо $d(\psi^{-1})^*\Pi=(\psi^{-1})^*d\Pi=(\psi^{-1})^*\psi^*\Omega=\Omega$, в предположении, что $\varphi^*\psi^*\Omega=(\varphi\psi)^*\Omega$.
\end{proof}

\begin{Def}
	Векторное поле называется потенциальным, если оно является градиетном некоторой функции(скалярного поля), которое называется его потенциалом.
	
	$\vec{a}=\grad f, \vec{a}$ --- потенциальное поле, $f$ --- потенциал ($f+C$ --- тоже потенциал).
\end{Def}

\begin{Def}
	Векторное поле называется солениодальным, если оно является ротором некоторого векторного поля, которое называется векторным потенциалом исходного поля.
	
	$\vec{a}=\rot\vec{b}, \vec{a}$ --- соленоидальное поле, $\vec{b}$ --- векторный потенциал ($\vec{b}+\grad f$ --- тоже  векторный потенциал, так как $\rot\grad f=0$).
\end{Def}

\begin{corollary}[из определения]
	Если $\vec{a}$ --- потенциальное поле, то $\rot\vec{a}=\vec{0}$. Если $\vec{a}$ --- соленоидальное поле, то $\di\vec{a}=0$.
\end{corollary}

\begin{corollary}[из леммы Пуакаре]
	Если $\rot\vec{a}=0$ в звездообразной области $D$, то гладкое векторное поле $\vec{a}$ является потенциальным. Если $\di\vec{a}=0$ в звездообразной области $D$, то гладкое векторное поле $\vec{a}$ является соленоидальным.
\end{corollary}

\subsection{Интегрирование дифференциальных форм.}

Пространство $\Lambda_n(\R^n)$ одномерно. Если $(f^1,\ldots, f^n)$ --- базис $E^*$, то\\ $\{cf^1\wedge\ldots\wedge f^n:c\in\R^n\}=\Lambda_n(\R^n)$.

Пусть $(e_1^0,\ldots,e_n^0)$ --- ортонормированный базис $E$. $(e_1,\ldots, e_n)$ --- другой базис, $e_j=t^i_je_i^0$, $T$ --- матрица перехода.

$V_{e^0}:=dx^1_0\wedge\ldots\wedge dx^n_0$ --- базисный элемент пространства $n$-форм в любой точке области $D$.

$V_e:=dx^1\wedge\ldots\wedge dx^n$.

$V_{e^0}(e_1,\ldots, e_n)=dx^1_0\wedge\ldots\wedge dx^n_0(e_1,\ldots, e_n)=\det(dx_0^i(e_j))_{i=1,j=1}^{p,p}=\det T$.

$V_e(e_1,\ldots, e_n)=dx^1\wedge\ldots\wedge dx^n(e_1,\ldots, e_n)=1$.

\begin{prop}
	$V_{e^0}=\det TV_e$.
\end{prop}

\begin{Def}
	Два базиса называются эквивалентными, если $\det T>0$, где $T$ --- матрица перехода от одного базиса к другому.
\end{Def}

\begin{prop}
	Отношение эквивалентности базисов --- отношение эквивалентности на множестве базисов. Значит все базисы разбиваются на 2 класса эквивалентности. Задается ориентация базисов.
\end{prop}

Форма $V_e$ позволяет определить ориентацию базиса. Она назывется формой ориентированного объема.

Введем обозначение: $\Pi=\{0<x^i<1:i=1,\ldots,n\}$ --- призма, натянутая на векторы $e_1, \ldots, e_n$.

\begin{prop}
	$V_{e^0}=\pm \mu(\Pi)$, где $+$ соответствует положительно определенному относительно $(e_1^0,\ldots,e^0_n)$ базису $(e_1,\ldots, e_n)$.
\end{prop}



























