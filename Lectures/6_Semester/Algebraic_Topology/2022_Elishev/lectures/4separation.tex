\section{Отделимость в топологических пространствах}

\subsection{Аксиомы отделимости}

\begin{definition}
    Топологическое пространство $X$ удовлетворяет \textit{нулевой аксиоме отделимости}, или \textit{аксиоме $T_0$}, если для любых двух различных точек $x, y \in X$ по крайней мере одна из них имеет окрестность, не содержащую другую точку. Топологические пространства, удовлетворяющие нулевой аксиоме отделимости, называются \textit{$T_0$-пространствами}.
\end{definition}

\begin{definition}
    Топологическое пространство $X$ удовлетворяет \textit{первой аксиоме отделимости}, или \textit{аксиоме $T_1$}, если для любых двух различных точек $x, y \in X$ существует окрестность $U(x)$ точки $x$, не содержащая $y$, и окрестность $U(y)$ точки $y$, не содержащая $x$. Топологические пространства, удовлетворяющие первой аксиоме отделимости, называются \textit{$T_1$-пространствами}.
\end{definition}

\begin{proposition}
    Топологическое пространство $X$ удовлетворяет первой аксиоме отделимости $\lra$ для любого $x \in X$ множество $\{x\}$ является замкнутым.
\end{proposition}

\begin{proof}~
    \begin{itemize}
        \item[$\ra$] Если $X$ "--- $T_1$-пространство, то для любого $x \in X$ множество $X \bs \{x\}$ является открытым, поэтому множество $\{x\}$ является замкнутым. 

        \item[$\la$] Для любых различных точек $x, y \in X$ множества $X \bs \{x\}$ и $X \bs \{y\}$ являются открытыми и удовлетворяют условию первой аксиомы отделимости.\qedhere
    \end{itemize}
\end{proof}

\begin{proposition}
    Пусть $X$ "--- $T_1$-пространство, $M \subset X$. Тогда любая точка прикосновения $x \in X$ множества $M$ является либо предельной точкой множества $M$, либо изолированной в $M$.
\end{proposition}

\begin{proof}
    Пусть $x \in \overline{M}$ "--- не предельная точка множества $M$, тогда существует окрестность $U(x)$ этой точки, содержащая лишь конечное число точек из множества $M$. Обозначим точки из $U(x) \cap M$, отличные от $x$, через $x_1, \dotsc, x_n$, и рассмотрим множество $U_0(x) := U(x) \bs \{x_1, \dotsc, x_n\}$. Оно является открытым и не содержит точек из $M$, отличных от $x$. Кроме того, поскольку $x$ "--- точка прикосновения множества $M$, то $U_0(x) \cap M \ne \emptyset$, поэтому $x \in M$ и множество $\{x\}$ "--- открытое в $M$, поэтому точка $x$ "--- изолированная.
\end{proof}

\begin{corollary}
    В $T_1$-пространстве $X$ замкнутые множества "--- это множества, содержащие все свои предельные точки.
\end{corollary}

\begin{note}
    Если $T_1$-пространство $X$ удовлетворяет первой аксиоме счетности, то для каждой точки $x \in \overline{M}$ можно найти сходящуюся к ней последовательность $\{x_n\} \subset M$. Если при этом $x$ -- предельная точка $M$, то можно выбрать последовательность, все точки в которой различны. 
\end{note}

\begin{definition}
    Топологическое пространство $X$ удовлетворяет \textit{второй аксиоме отделимости}, или \textit{аксиоме $T_2$}, если любые две различные точки $x, y \in X$ имеют непересекающиеся окрестности $U(x), U(y)$. Топологические пространства, удовлетворяющие второй аксиоме отделимости, называются \textit{$T_2$-пространствами}, или \textit{хаусдорфовыми пространствами}.
\end{definition}

\begin{proposition}~
    \begin{enumerate}
        \item Аксиома $T_1$ строго сильнее аксиомы $T_0$.
        \item Аксиома $T_2$ строго сильнее аксиомы $T_1$.
    \end{enumerate}
\end{proposition}

\begin{proof}~
    \begin{enumerate}
        \item Примером $T_0$-пространства, не удовлетворяющего аксиоме $T_1$, является связное двоеточие $X = \{0, 1\}$ с открытой топологией $\lbrace \emptyset, \lbrace 1\rbrace, X\}$.
        
        \item Рассмотрим множество $X := \mathbb R \cup \lbrace \xi\rbrace$, состоящее из вещественных чисел и некоторого отличного от них элемента. Объявим открытыми в $X$ множествами все открытые подмножества числовой прямой, а также все множества вида $X \backslash F$, где множество $F \subset \mathbb R$ "--- конечное. Построенная топология удовлетворяет первой аксиоме отделимости, но не является
        хаусдорфовой. Действительно, для любой точки $x \in X \bs \{\xi\}$ любые две окрестности $U(x), U(\xi)$ пересекаются, поскольку $U(\xi)$ содержит все вещественные числа за исключением, быть может, некоторого конечного множества, а $U(x)$ содержит целый интервал вещественных чисел.\qedhere
    \end{enumerate}
\end{proof}

\begin{definition}
    $T_1$-пространство $X$ называется \textit{регулярным}, если для любой точки $x \in X$ и любого не содержащего ее замкнутого множества $F \subset X$ существуют непересекающиеся окрестности $U(x), U(F)$.
\end{definition}

\begin{note}
    Всякое регулярное пространство, очевидно, является хаусдорфовым.
\end{note}

\begin{proposition}
    Аксиома регулярности строго сильнее аксиомы $T_2$.
\end{proposition}

\begin{proof}
    Рассмотрим множество $X := \mathbb R$ и зададим на нем топологию с помощью базы $\mathfrak{B}$, состоящей из всех интервалов, а также множеств, являющихся разностью некоторого интервала и множества $K := \{\frac 1n: n \in \N\}$. Топология на $X$ содержит стандартную топологию на $X$ и потому удовлетворяет аксиоме $T_2$, однако множество $K$, замкнутое по построению, невозможно отделить непересекающимися окрестностями от точки $0$.
\end{proof}

\begin{theorem}[без доказательства]\label{theoremreg}
    $T_1$-пространство $X$ регулярно $\lra$ для любой точки $x \in X$ любая ее окрестность $U(x)$ содержит замыкание некоторой окрестности $U_0(x)$ той же точки
    $x$.
\end{theorem}

\begin{definition}
    $T_1$-пространство $X$ называется \textit{нормальным}, если любые два непересекающихся замкнутых множества $F_1, F_2 \subset X$ имеют непересекающиеся окрестности $U(F_1), U(F_2)$.
\end{definition}

\begin{definition}
    Нормальное пространство $X$ называется \textit{наследственно нормальным}, если всякое его подмножество является нормальным пространством.
\end{definition}

\begin{definition}
    Свойство $\mathcal{P}$ топологических пространств называется \textit{наследственным}, если из справедливости свойства $\mc P$ для топологического пространства $X$ следует справедливость этого свойства для любого подпространства $M \subset X$.
\end{definition}

\begin{proposition}\label{t1hereditary}
    Аксиома $T_1$ является наследственным свойством.
\end{proposition}

\begin{proof}
    Тривиально.
\end{proof}

\begin{proposition}
    Регулярность является наследственным свойством.
\end{proposition}

\begin{proof}
    В силу утверждения \ref{t1hereditary}, достаточно проверить наследственность отделимости точки от замкнутого множества. Пусть $X$ "--- регулярное пространство, и  $M\subset X$. Рассмотрим произвольную точку $x \in M$ и произвольное не содержащее ее множество $C \subset M$, замкнутое в $M$. Выберем замкнутое множество $F \subset X$ такое, что $C = M \cap F$. В силу регулярности пространства $X$, существуют непересекающиеся окрестности $U(x), U(F)$ точки $x$ и множества $F$. Тогда множества $U(x)\cap M$ и $U(F)\cap M$ являются искомыми непересекающимися окрестностями точки $x$ и множества $C$ в подпространстве $M$.
\end{proof}

\begin{note}
    Наследственными являются аксиомы отделимости $T_0$, $T_1$, $T_2$, а также аксиомы счетности.
\end{note}

\begin{note}
    Нормальность не является наследственным свойством. Стандартным контрпримером является \textit{плоскость Тихонова} "--- пространство $X := [0, \omega_1]\times [0, \omega]$, где $\omega$ "--- первый бесконечный ординал, $\omega_1$ "--- первый несчетный ординал, с топологией, заданной как произведение порядковых топологий сомножителей. Пространство $X$ нормально, однако подпространство $X \backslash \lbrace(\omega_1,\omega)\rbrace$ не является нормальным. Детали этой конструкции будут изучены далее в курсе.
\end{note}

\begin{corollary}
    Аксиома нормальности строго сильнее аксиомы регулярности.
\end{corollary}

\begin{proof}
    Всякое нормальное пространство является регуярным. Примером регулярного, но не нормального пространства служит плоскость Тихонова $[0, \omega_1]\times [0, \omega]$ без точки $\lbrace(\omega_1,\omega)\rbrace$.
\end{proof}

\begin{note}
    Можно проверить, что любое метрическое пространство является нормальным, причем наследственно нормальным.
\end{note}

\begin{definition}
    Нормальное пространство $X$ называется \textit{совершенно нормальным}, если для любых двух непересекающихся замкнутых множества $F_1, F_2 \subset X$ существует непрерывная функция $f: X\rightarrow [0, 1]$ такая, что $f^{-1}(\lbrace 0\rbrace) = F_1$ и $f^{-1}(\lbrace 1\rbrace) = F_2$.
\end{definition}

\begin{theorem}[без доказательства]
    Топологическое постранство $X$ совершенно нормально $\lra$ пространство $X$ нормально и всякое его замкнутое подмножество является пересечением счетного числа открытых множеств.
\end{theorem}

\subsection{Вполне регулярные пространства}

\begin{definition}
    Пусть $X$ "--- топологическое пространство. Два непересекающихся замкнутых множества $A, B \subset X$ называются \textit{функционально отделимыми} в $X$, если существует непрерывная функция $f : X\rightarrow [0, 1]$ такая, что $f(A) = \{0\}$ и $f(B) = \{1\}$.
\end{definition}

\begin{definition}
    $T_1$-пространство $X$ называется \textit{вполне регулярным}, или \textit{пространством Тихонова}, если для любой точки $x \in X$ множество $\{x\}$ функционально отделимо в $X$ от любого не содержащего точку $x$ замкнутого множества $F \subset X$.
\end{definition}

\begin{note}
    Если замкнутые множества $A, B \subset X$ функционально отделимы, то они также обладают непересекающимися окрестностями: например, подходят прообразы открытых множеств $[0, \frac 12)$ и $(\frac12, 1]$. Значит, вполне регулярные пространства являются регулярными.
\end{note}

\begin{note}
    Построение примера регулярного простианства, не являющегося вполне регулярным, "--- это сложная задача. Это связано с тем, что свойство вполне регулярности наследственно, а также с тем обстоятельством, что добавление к регулярности некоторых других типичных свойств уже гарантирует вполне регулярность.
\end{note}

\subsection{Первая метризационная теорема Урысона}

\begin{note}
    Одной из важнейших задач общей топологии является установление достаточных условий того, чтобы топологическое пространство было метризуемым, то есть гомеоморфным некоторому метрическому пространству. В данном разделе будет доказана первая метризационная теорема Урысона, устанавливающая достаточное для метризуемости условие для пространств со счетной базой.
\end{note}

\begin{proposition}[малая лемма Урысона]
    Пусть $X$ "--- нормальное пространство, множество $A \subset X$ "--- замкнутое. Тогда для любой окрестности $U(A)$ множества $A$ существует окрестность окрестность $U_0(A)$ множества $A$ такая, что $\overline{U_0(A)} \subset U(A)$.
\end{proposition}

\begin{proof}
    В силу нормальности пространства $X$, можно выбрать непересекающиеся окрестности $U_0(A)$ и $V$ замкнутых множеств $A$ и $X \backslash U(A)$, тогда $U_0(A) \cap V = \emptyset$, откуда $\overline{U_0(A)} \subset U(A)$.
\end{proof}

\begin{proposition}[большая лемма Урысона]
    Пусть $X$ "--- нормальное пространство, $A, B \subset X$ "--- два замкнутых  непересекающихся множества. Тогда для любых двух чисел $a, b \in \R$ таких, что $a < b$, существует непрерывная функция $f: X \to \R$ такая, что $f(A) = \{a\}$, $f(B) = \{b\}$ и $f(X) \subset [a, b]$.
\end{proposition}

\begin{proof}
    Достаточно рассмотреть случай, когда множества $A, B$ непусты и выполнены равенства $a = 0$, $b = 1$. Положим $\Gamma_1 :=  X \backslash B$ и по малой лемме Урысона выберем окрестность $\Gamma_0$ множества $A$ такую, что $\overline{\Gamma_0} \subset \Gamma_1$. По индукции, построим для произвольных чисел $n \in \N$ и $p \in \{0, 1, \dotsc, 2^n\}$ окрестности $\Gamma_{\frac{p}{2^n}}$ такие, что выполнено следующее условие:
    \[\overline{\Gamma_{\frac{p}{2^n}}} \subset \Gamma_{\frac{2p+1}{2^{n+1}}}\subset \overline{\Gamma_{\frac{2p+1}{2^{n+1}}}}\subset \Gamma_{\frac{p+1}{2^n}}\]

    Таким образом, для каждого двоично-рационального числа $r \in [0, 1]$ получено открытое в $X$ множество $\Gamma_r$, и для любых двоично-рациональных чисел $r, r' \in [0, 1]$ таких, что $r < r'$, выполнено $\overline{\Gamma_r} \subset \Gamma_{r'}$. Теперь для каждого не двоично-рационального числа $t \in (0, 1)$ положим $\Gamma_t := \bigcup_{r<t}\Gamma_r$. Легко проверить, что тогда для любых $t, t' \in [0, 1]$ таких, что $t < t'$, выполнено $\overline{\Gamma_t} \subset \Gamma_{t'}$. Наконец, доопределим $\Gamma_t := \emptyset$ для $t \in (-\infty, 0)$ и $\Gamma_t := X$ для $t \in (1, +\infty)$.
    
    Теперь для каждой точки $x \in X$ положим $f(x) := \inf\{t \in \R: x \in \Gamma_t\}$ и получим функциию $f: X \to [0, 1]$, для которой выполнены равенства $f(A) = \{0\}$ и $f(B) = \{1\}$. Проверим ее непрерывность. Зафиксировуем точку $x \in X$ и произвольное $\epsilon > 0$. Рассмотрим окрестность $U(x) := \Gamma_{\tau_x + \epsilon}\backslash \overline{\Gamma_{\tau_x - \epsilon}}$, тогда для любой точки $x'\in U(x)$ выполнено следующее:
    \[f(x) - \epsilon< f(x') <f(x) +\epsilon\]

    Таким образом, получено требуемое.
\end{proof}

\begin{theorem}[первая метризационная теорема Урысона]
    Пусть $X$ "--- топологическое пространство со счетной базой. Тогда $X$ метризуемо $\lra$ $X$ нормально.
\end{theorem}

\begin{proof}
    Нетривиально только доказательство ($\la$). Покажем, что нормальное пространство $X$ гомеоморфно некоторому подмножеству метрического пространства $R^\infty$. Пусть $\{U_n\}$  "--- счетная база пространства $X$. Назовем пару $(U_i, U_k)$ \textit{канонической}, если выполнено условие $\overline{U_i}\subset U_k$. По малой лемме Урысона, для любой точки $x \in X$ и любой ее окрестности $U(x)$ существует каноническая пара $(U_i, U_k)$ такая, что $a \in U_i$ и $U_k \subset U(a)$.
    
    Обозначим множество канонических пар через $\{\pi_m\}$. По большой лемме Урысона, для каждой пары $\pi_m = (U, V)$ существует непрерывная функция $\phi_m : X \rightarrow [0, 1]$ такая, что $\varphi_m(\overline U) = \{0\}$ и $\varphi_m(X \bs V) = \{1\}$. Теперь для произвольной точки $x \in X$ определим точку $f(x) \in \R^\infty$ следующим образом:
    \[f(x) := \left(\frac{\varphi_1(x)}{2}, \frac{\varphi_2(x)}{2^2}, \dotsc\right)\]

    Легко проверить, что отображение $f$ является гомеоморфизмом пространств $X$ и $f(X)$, поэтому пространство $X$ метризуемо.
\end{proof}