\textcolor{red}{Определение квадратичной формы?}

\textcolor{red}{Дописать таблицу}

\section{Обратный оператор}

\textcolor{red}{Картинка и некоторое предисловие}

\begin{note}
	Далее мы фиксируем обозначения $E_1, E_2$ для линейных нормированных пространств.
\end{note}

\begin{definition}
	Оператор $A \colon E_1 \to E_2$ называется \textit{обратимым} на $\im A$, если
	\[
		\forall y \in \im A\ \ \exists ! x \in E_1 \such Ax = y
	\]
\end{definition}

\begin{anote}
	Фактически, оператор обратим, если он осуществляет биекцию $E_1 \to \im A \subseteq E_2$.
\end{anote}

\begin{example}
	Самый простой пример необратимого оператора --- это $A = 0$. Также подойдёт любой оператор, чьё ядро нетривиально (в силу критерия инъективности).
\end{example}

\begin{example}
	Естественно, далеко не всегда обратный оператор ограничен. Рассмотрим $E_1 = C[0; 1]$ и определим оператор $A$:
	\[
		(Af)(x) = \int_0^x f(t)dt =: g(x)
	\]
	Тогда $E_2 = \{g \in C[0; 1] \colon g(0) = 0\}$. Понятно, что $A^{-1} = \frac{d}{dx}$, но, как уже было показано в 5 семестре, этот оператор неограничен.
\end{example}

\begin{theorem} (Банаха, об обратном операторе)
	Пусть $E$ --- банахово пространство, $A \in \cL(E)$, причём $A$ биективен. Тогда $A^{-1}$ непрерывен.
\end{theorem}