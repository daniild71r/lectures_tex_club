\section{Дискретный анализ}

\subsection{Теория Рамсея}

\begin{definition}
	\textit{Числом Рамсея} $R(s, t),\ s, t \in \N$, называется минимальное число $n \in \N$ такое, что в любой раскраске рёбер $K_n$ в красный и синий цвета найдётся либо подграф $K_s$, у которого все рёбра покрашены в красный, либо подграф $K_t$, у которого все рёбра покрашены в синий.
\end{definition}

\begin{proposition}
	Имеет место эквивалентное определение:
	\[
		R(s, t) = \min\{n \in \N \colon \forall G = (V, E),\ |V| = n \wedge (\alpha(G) \ge s \vee w(G) \ge t)\}
	\]
\end{proposition}

\begin{proof}
	Скажем, что граф $G$ на $n$ вершинах --- это граф, индуцированный из раскрашенного $K_n$ путём взятия рёбер только одного цвета.
\end{proof}

\begin{proposition}
	$R(s, t) = R(t, s)$
\end{proposition}

\begin{proof}
	Действительно, если такое минимальное $n \in \N$ существует хотя бы для одного из чисел, то мы просто берём подходящий граф и инвертируем его раскраску.
\end{proof}

\begin{proposition}
	$R(1, t) = 1$
\end{proposition}

\begin{proof}
	В графе на одной вершине всегда есть независимое множество, состоящее из одной вершины.
\end{proof}

\begin{proposition}
	$R(2, t) = t$
\end{proposition}

\begin{proof}
	Нам нужно либо 1 ребро в графе, либо $t$ независимых вершин. Понятно, что $t$ тогда точно подходит, ну а меньше не получится, ибо без рёбер размер независимого множества будет меньше $t$.
\end{proof}

\begin{proposition}
	$R(3, 3) = 6$
\end{proposition}

\begin{proof}~
	\begin{itemize}
		\item[$\ge$] Простой цикл на 5 вершинах является контрпримером к меньшему значению числа Рамсея.
		
		\item[$\le$] Рассмотрим $v \in K_6$. В силу принципа Дирихле, есть либо 3 красных, либо 3 синих ребра, исходящих из неё. Посмотрим на треугольник из 3х вершин, в которые идут 3 одноцветных ребра из $v$. Если там есть хоть одно ребро того же цвета, что и у рёбер из $v$, то победа нам обеспечена. А если это не так, то этот треугольник одноцветный и этого нам тоже достаточно.
	\end{itemize}
\end{proof}

\begin{theorem} (1935г., Эрдёш, Секереш)
	Имеет место следующее соотношение между числами Рамсея:
	\[
		\forall s, t \in \N\ \ R(s, t) \le R(s - 1, t) + R(s, t - 1)
	\]
\end{theorem}

\begin{note}
	Пожалуй, это одно из важнейших утверждений в теории Рамсея. Вместе с утверждениями выше, оно обосновывает существование любого числа Рамсея.
\end{note}

\begin{proof}
	Нужно показать, что графа на $n = R(s - 1, t) + R(s, t - 1)$ вершинах хватит, чтобы найти $K_s$ или $K_t$.
	
	Итак, рассмотрим $K_n$ и вершину $v \in K_n$. Тогда, либо из неё выходит $\ge R(s - 1, t)$ красных рёбер, либо $R(s, t - 1)$ синих рёбер. Не умаляя общности, мы нашли $R(s - 1, t)$ красных рёбер. Посмотрим на вершины, которые прикреплены к этим концам. В силу определения числа Рамсея, там либо найдётся синий $K_t$, либо красный $K_{s - 1}$, но тогда мы можем добавить к нему вершину $v$ и получить $K_s$.
\end{proof}

\begin{proposition}
	Имеет место следующая оценка на $R(s, t)$: \(R(s, t) \le C_{s + t - 2}^{s - 1}\)
\end{proposition}

\begin{proof}
	Найденное нами неравенство очень напоминает $C_n^k = C_{n - 1}^{k - 1} + C_{n - 1}^k$.
	
	Как вывести, если забыл формулу? Нужно по виду исходного неравенства понять, что нам нужен общий знаменатель в биномиальном коэфициенте. Его добьёмся лишь тем, что поставим туда какую-то сумму от двух аргументов. Второй момент заключается в том, что нам нужна одинаковая оценка при симметричных параметрах Рамсея. Стало быть, стоит попробовать красивое $C_{(s - 1) + (t - 1)}^{s - 1}$
	
	Доказывается верность оценки по индукции.
\end{proof}

\begin{reminder}
	В прошлом семестре был установлен следующий факт: для модели $G(n, 1/2)$ асимптотически почти наверное $\alpha(G), \omega(G) < 2\log_2 n$
\end{reminder}

\begin{corollary}
	Для всех $n \ge n_0$ верно, что \(R(2\log_2 n, 2\log_2 n) > n\)
\end{corollary}

\begin{proof}
	Положительная вероятность означает, что как минимум 1 граф со свойствами из следствия при данном $n$ существует.
\end{proof}

\begin{theorem}
	Пусть $s \ge 4$. Тогда $R(s, s) > \floor{2^{s / 2}}$
\end{theorem}

\begin{proof}
	В теории Рамсея подавляющее число оценок снизу --- это вероятностный метод. Здесь мы тоже его применим: рассмотрим модель $G(n, 1/2)$ и найдём такое $n(s)$, что вероятность случайного графа иметь $s$-клику или антиклику строго меньше единицы.
	
	Перечислим все $s$-элементные подмножества вершин: $A_1, \ldots, A_{C_n^s}$. Введём соответствующее событие $\cA_i$, что в случайном графе $G$ подмножество $A_i$ является либо кликой, либо независимым множеством. Тогда
	\[
		P(\cA_i) = 2 \cdot 2^{-C_s^2} = 2^{1 - C_s^2}
	\]
	Посчитаем вероятность наличия $s$-элементной клики либо независимого множества:
	\[
		P\ps{\bigcup_{i = 1}^{C_n^s} \cA_i} \le \sum_{i = 1}^{C_n^s} P(\cA_i) = C_n^s \cdot 2^{1 - C_s^2}
	\]
	Осталось проверить, что $n(s) = \floor{2^{s / 2}}$ подходит:
	\[
		C_n^s \cdot 2^{1 - C_s^2} \le \frac{n^s}{s!} \cdot 2^{1 - \frac{s^2}{2} + \frac{s}{2}} \le \frac{2^{s^2 / 2}}{s!} \cdot 2^{1 - s^2 / 2 + s / 2} = \frac{2^{1 + s / 2}}{s!}
	\]
	Можно по индукции установить, что оставшаяся величина $\le 1 / 3$, чего нам и достаточно.
\end{proof}

\begin{corollary}
	Если числа $n, s$ таковы, что $C_n^s \cdot 2^{1 - C_s^2} < 1$, то $R(s, s) > n$
\end{corollary}

\begin{corollary}
	Всегда верно, что $R(s, s) \ge (1 + o(1))\frac{s}{e\sqrt{2}} \cdot 2^{s / 2}$.
\end{corollary}

\begin{proof}
	Доказательство состоит в том, чтобы проверить асимптотическое выполнение неравенства, сформулированного в предыдущем следствии при $n = (1 + o(1))\frac{s}{e\sqrt{2}} \cdot 2^{s / 2}$. Итак (подставили и $n$, и формулу Стирлинга):
	\[
		\frac{n^s}{s!} \cdot 2^{1 - s^2 / 2 + s / 2} \le (1 + o(1))^s \frac{s^s \cdot 2^{s^2 / 2}}{e^s \cdot 2^{s / 2}} \cdot \frac{2^{1 - s^2 / 2 + s / 2}}{(1 + o(1))\sqrt{2\pi s} (s / e)^s} = \frac{2(1 + o(1))^s}{(1 + o(1))\sqrt{2\pi s}}
	\]
	Важно: мы сейчас просто сказали, что в нашей оценке есть $o(1)$, но мы вольны его выбрать сами! (Это же неверно, если мы говорим про формулу Стирлинга, там функция конкретная). Осталось подобрать такую функцию, что она в степени $s$ проиграет корню в знаменателе (и вся дробь устремится к нулю). Например, можно взять $f(s) = -1 / s$, но просто так она не подойдёт для всех $s$. Поступим просто - возьмём $s_0$, начиная с которого оценка верна, а для меньших положим $f(s) = -1$.
\end{proof}

\begin{theorem}
	$\forall n, s \in \N\ \ R(s, s) > n - C_n^s \cdot 2^{1 - C_s^2}$
\end{theorem}

\begin{proof}
	Продолжаем жить в модели $G(n, 1 / 2)$. Если $X(G)$ --- это число клик и антиклик размера $s$ в случайном графе $G$, то $\E X$ нам уже известно:
	\[
		\E X = C_n^s \cdot 2^{1 - C_s^2}
	\]
	Тогда, существует хотя бы 1 граф $G$ такой, что $X(G) \le \E X$. Возьмём и удалим все вершины, которые принадлежат этим кликам и антикликам. Тогда получится граф $G'$, в котором $|V(G')| \ge n - C_n^s \cdot 2^{1 - C_s^2}$ и нет клик и антиклик на $s$ вершинах, что и требовалось доказать.
\end{proof}

\begin{corollary}
	\(R(s, s) \ge (1 + o(1))\frac{s}{e} \cdot 2^{s / 2}\)
\end{corollary}

\begin{proof}
	Идея воспользоваться тем, что мы уже показали для произвольных $n, s$, просто подобрав $n(s)$. На самом деле $n = \ceil{\frac{s}{e} \cdot 2^{s / 2}}$, но это можно было понять и иначе, если оценить вычитаемое в теореме выше (через оценку на биномиальный коэффициент, подстановки туда формулы Стирлинга, ну как обычно). Итак, проверим нашу гипотезу:
	\begin{multline*}
		n - C_n^s \cdot 2^{1 - s^2 + s / 2} \ge \frac{s}{e} \cdot 2^{s / 2} - \frac{s^s \cdot 2^{s^2 / 2}}{e^s \cdot (s / e)^s \sqrt{2\pi s} (1 + o(1))} \cdot 2^{1 - s^2 + s / 2} =
		\\
		\frac{s}{e} \cdot 2^{s / 2} - \frac{2^{1 + s / 2}}{\sqrt{2\pi s}(1 + o(1))} = \frac{s}{e} \cdot 2^{s / 2}(1 + o(1))
	\end{multline*}
\end{proof}