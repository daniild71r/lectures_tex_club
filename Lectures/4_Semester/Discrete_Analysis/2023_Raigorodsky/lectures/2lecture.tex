\subsection{Локальная Лемма Ловаса (ЛЛЛ)}

\begin{note}
	Когда мы говорили о теории Рамсея, нам нужно было доказывать существование определённых графов, и делали мы это при помощи  вероятностного метода:
	\[
		P\ps{\bigcup_{i = 1}^n \cA_i} \le \sum_{i = 1}^n P(\cA_i) < 1 \Lra P\ps{\overline{\bigcup_{i = 1}^n \cA_i}} = P\ps{\bigcap_{i = 1}^n \overline{\cA_i}} > 0
	\]
	Локальная Лемма Ловаса --- это такое утверждение, которое позволяет использовать в каком-то смысле формулу включений и исключений (тем самым получаем более хорошую оценку на вероятность объединения)
\end{note}

\begin{reminder}
	Говорят, что $A$ независимо от множество $\{B_i\}_{i = 1}^k$, если выполнено утверждение:
	\[
		\forall I \subseteq \range{k}\ \ P\ps{A \cap \bigcap_{j \in I} B_j} = P(A) \cdot P\ps{\bigcap_{j \in I} B_j}
	\]
\end{reminder}

\begin{theorem} (ЛЛЛ, симметричный случай)
	Пусть $\{A_i\}_{i = 1}^n$ --- события вероятностного пространства $(\Omega, \cF, P)$, причём выполнены условия:
	\begin{itemize}
		\item $\forall i \in \range{n}\ P(A_i) \le p$
		
		\item $\forall i \in \range{n}\ A_i$ не зависит от совокупности всех остальных событий кроме, может быть, $d$ штук
		
		\item Имеет место неравенство $ep(d + 1) \le 1$
	\end{itemize}
	Тогда $P\ps{\bigcap_{i = 1}^n \overline{A_i}} > 0$
\end{theorem}

\begin{theorem}
	При помощи ЛЛЛ сейчас достигнута лучшая оценка снизу для числа Рамсея:
	\[
		R(s, s) \ge (1 + o(1)) \frac{s\sqrt{2}}{e} \cdot 2^{s / 2}
	\]
\end{theorem}

\begin{proof}
	Работаем по старой схеме, в модели $G(n, 1 / 2)$. $\cA_i$ --- это всё то же событие, что $A_i$ является кликой или антикликой размера $s$. Но теперь мы зададимся вопросом: <<А от каких других $A_j$ может зависеть $A_i$?>> Ответ прост: от всех тех $A_j$, которые пересекаются с нашим как минимум по двум вершинам. Тогда $d$ из ЛЛЛ имеет вид ($k$ --- число общих вершин из $s$ тех, что закреплены за $A_i$):
	\[
		d = \sum_{k = 2}^{s - 1} C_s^k \cdot C_{n - s}^{s - k}
	\]
	Анализ такого выражения слишком сложный, поэтому мы обойдёмся более консервативной оценкой: возьмём просто любые 2 вершины из $A_i$ и посчитаем все $A_j$, которые их содержат. Общее число не будет оптимальным, но будет хорошей оценкой:
	\[
		d \le C_s^2 \cdot C_{n - 2}^{s - 2}
	\]
	Теперь мы говорим, что $n(s)$ равно оценке, указанной в теореме ($o(1)$ пока само по себе не фиксировано). Нам достаточно проверить следующее неравенство (которое является оценкой сверху того, что требует Локальная Лемма Ловаса):
	\[
		3 \cdot 2^{1 - C_s^2} \cdot C_s^2 \cdot C_{n - 2}^{s - 2} \le 1
	\]
	Посмотрим на асимптотику выражения слева (теперь $s = o(n)$, поэтому переход для последнего биномиального коэффициента корректный):
	\begin{multline*}
		3 \cdot 2^{1 - C_s^2} \cdot C_s^2 \cdot C_{n - 2}^{s - 2} \sim
		\\
		3 \cdot 2^{1 - s^2 / 2 + s / 2} \cdot \frac{s^2}{2} \cdot \frac{(n - 2)^{s - 2}}{(s - 2)!} \sim
		\\
		3 \cdot 2^{1 - s^2 / 2 + s / 2} \cdot \frac{s^2}{2} \cdot \frac{n^{s - 2}}{s! / s^2} \sim
		\\
		3 \cdot 2^{1 - s^2 / 2 + s / 2} \cdot \frac{s^4}{2} \cdot \frac{(1 + o(1))^{s - 2}}{\sqrt{2\pi s} (s / e)^s (1 + o(1))} \cdot 2^{\frac{s - 2}{2}} \cdot s^{s - 2} \cdot 2^{s^2 / 2 - s} \cdot e^{2 - s} =
		\\
		\frac{3}{2} \frac{e^2 \cdot s^2}{(1 + o(1))\sqrt{2\pi s}} (1 + o(1))^{s - 2}
	\end{multline*}
	Достаточно взять $f(s) = 1 - 1 / s$, чтобы со степенью $s - 2$ получилась экспонента с отрицательным показателем и данное выражение устремилось к нулю (значит, с какого-то момента мы можем использовать ЛЛЛ и утверждение доказано).
\end{proof}

\begin{corollary}
	Если $n, s$ таковы, что выполнено неравенство
	\[
		e \cdot \ps{2^{1 - C_s^2}} \cdot \ps{\sum_{k = 2}^{s - 1} C_s^k \cdot C_{n - s}^{s - k} + 1} \le 1
	\]
	то $R(s, s) \ge n$
\end{corollary}

\begin{definition}
	Орграф $G = \{A_1, \ldots, A_n, E\}$, где $A_i$ --- это события из вероятностного пространства $(\Omega, \cF, P)$, называется \textit{орграфом зависимостей}, если верно утверждение:
	\[
		\forall i \in \range{n}\ \ A_i \text{ не зависит в совокупности от всех $A_j$, для которых } (A_i, A_j) \notin E
	\]
\end{definition}

\begin{note}
	Этот граф нужен для формулировки несимметричного случая ЛЛЛ. Из определения можно задаться вопросом: <А какие графы вообще могут быть графом зависимостей?>> Например, это точно верно граф без рёбер, или про цикл на 3х вершинах (причём это пример, когда $A_1, A_2, A_3$ попарно независимы, но зависимы в совокупности).
\end{note}

\begin{theorem} (Общий, несимметричный случай ЛЛЛ)
	Пусть $A_1, \ldots, A_n$ --- события из вероятностного пространства $(\Omega, \cF, P)$. Пусть $G = (V, E)$ --- такой орграф зависимостей, что
	\[
		\exists x_1, \ldots, x_n \such \forall i \in \range{n}\ P(A_i) \le x_i \cdot \prod_{j \colon (A_i, A_j) \in E} (1 - x_j)
	\]
	Тогда $P\ps{\bigcap_{i = 1}^n \overline{A_i}} \ge \prod_{i = 1}^n (1 - x_i) > 0$
\end{theorem}

\begin{proof} (Вывод симметричного случая ЛЛЛ из общего)
	Для начала, придётся сделать смешной разбор случаев:
	\begin{enumerate}
		\item $d = 0 \Ra P\ps{\bigcap_{i = 1}^n \overline{A_i}} = \prod_{i = 1}^n P(\overline{A_i}) \ge (1 - p)^n$
		
		\item $d \ge 1$ Прежде всего, нужно взять граф зависимостей. Поступим просто: зная не более чем $d$ возможных $A_j$, от которых зависит $A_i$, пустим к ним рёбра $(A_i, A_j)$. Тогда $\forall i \in \range{n} \deg A_i \le d$. В силу симметрии положим все $x_i = 1 / (d + 1)$. Остаётся проверить неравенство, которое требует общий случай:
		\[
			P(A_i) \le \frac{1}{d + 1} \prod \ps{1 - \frac{1}{d + 1}}
		\]
		К сожалению, в произведении у нас не более $d$ чисел, которые меньше единицы, поэтому просто продолжить неравенство не получится. Но что если мы просто покажем, что
		\[
			P(A_i) \le \frac{1}{d + 1} \cdot \ps{1 - \frac{1}{d + 1}}^d
		\]
		Этого будет точно достаточно. Осталось заметить, что мы уже имеем это требование из условия симметричного случая:
		\[
			ep(d + 1) \le 1 \Lra p \le \frac{1}{d + 1} \cdot e^{-1} \Ra p \le \frac{1}{d + 1} \ps{1 - \frac{1}{d + 1}}^d
		\]
	\end{enumerate}
\end{proof}

\begin{proof} (общего случая ЛЛЛ)
	\textcolor{red}{Это вопрос на отл, поэтому только после сессии}
\end{proof}