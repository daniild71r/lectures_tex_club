\subsection{Двудольные числа Рамсея}

\begin{definition}
	\textit{Двудольным числом Рамсея} $b(k, k)$ называется минимальное число $n \in \N$ такое, что при любой раскраске рёбер двудольного графа $K_{n, n}$ в красный и синий цвета найдётся одноцветный $K_{k, k}$
\end{definition}

\begin{exercise}
	Ровно теми же методами, которыми получалась оценка снизу для простого числа Рамсея, можно получить такой результат для двудольного:
	\[
		b(k, k) \ge ck2^{k / 2},\ c = const
	\]
\end{exercise}

\begin{theorem} (Конлон)
	Имеет место оценка сверху: \(\forall \eps > 0\ b(k, k) \le (1 + \eps)k \cdot 2^k\)
\end{theorem}

\begin{lemma}
	Рассмотрим произвольный $G \subseteq K_{m, n}$ плотности не ниже, чем $p \in [0; 1]$ (то есть $|E(G)| / (mn) \ge p$). Если числа $m, n, r, s$ и $p \in [0; 1]$ таковы, что
	\[
		nC_{mp}^r > (s - 1)C_m^r
	\]
	Тогда $\exists K_{r, s} \subseteq G$ ($r$ из доли размера $m$, $s$ из доли размера $n$, естественно)
\end{lemma}

\begin{proof}
	Предположим противное: в $G$ нет $K_{r, s}$. Тогда, посчитаем число $K_{r, 1}$ в $G$ двумя способами:
	\begin{itemize}
		\item С одной стороны, любое $r$-элементное подмножество $m$ может быть долей $K_{r, 1}$. Однако, выбрать последнюю вершину в другой доле мы можем не более $s - 1$ раз (иначе возникнет $K_{r, s}$). Таким образом, имеем $C_m^r (s - 1)$ возможных $K_{r, 1}$
		
		\item С другой стороны, пусть $d_1, \ldots, d_n$ --- это степени вершин в $G$, которые находятся в доле $K_{m, n}$ размера $n$ (если в $G$ меньше элементов в этой доле, то некоторые $d_i$ зануляются). Несложно понять, что тогда наша величина есть просто $C_{d_1}^r \plusdots C_{d_n}^r$. Причём известно, что биномиальные коэффициенты выпуклы, то есть
		\[
			C_{d_1}^r \plusdots C_{d_n}^r \ge n C_{\frac{d_1 \plusdots d_n}{n}}^r
		\]
		Сумма степеней вершин в точности равна числу рёбер в $G$, поэтому можно переписать с учётом условия это неравенство следующим образом:
		\[
			C_{d_1}^r \plusdots C_{d_n}^r \ge n C_{mp}^r
		\]
	\end{itemize}
	Получили противоречие с условием
\end{proof}

\begin{proof} (оценки Конлона)
	Зафиксируем $\eps > 0$ и $n = (1 + \eps)k \cdot 2^k$. Нужно показать, что любая раскраска этого графа приводит к существование одноцветного $K_{k, k}$. Для этого этого рассмотрим граф $G \subseteq K_{n, n}$, в котором находятся только красные рёбра. Без ограничения общности, можем сказать, что $p = 1 / 2$. Остаётся проверить, что при всех больших $k$ выполнено неравенство:
	\[
		nC_{n / 2}^k > (k - 1)C_n^k
	\]
	Хочется работать с этими выражениями асимптотически. Для этого заметим, что $k \sim \log_2 n$, поэтому мы имеем право пользоваться асимптотикой $C_n^k$:
	\[
		nC_{n / 2}^k \sim (1 + \eps)k2^k \cdot \frac{(n / 2)^k}{k!}(1 + o(1)) > k \frac{n^k}{k!}(1 + o(1)) \sim (k - 1)C_n^k
	\]
	Всё свелось к проверке следующего утверждения:
	\[
		(1 + \eps)(1 + o(1)) > (1 + o(1))
	\]
	Очевидно, что за счёт существование пределов обеих частей и уже их соотношения, это действительно так
\end{proof}

\begin{theorem}
	Имеет место оценка сверху: \(b(k, k) \le (1 + \eps)(\log_2 k) \cdot 2^{k + 1}\)
\end{theorem}

\begin{proof}
	\textcolor{red}{Спасибо, но это вообще на отл10}
\end{proof}

