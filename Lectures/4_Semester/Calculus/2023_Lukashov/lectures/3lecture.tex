\subsection{Пространства $L_p$}

\begin{note}
	В этой главе мы закрепляем за $E$ --- измеримое множество, а любая функция $f \colon E \to \sxR$ автоматически считается измеримой.
\end{note}

\begin{definition}
	Для функции $f \colon E \to \sxR$ назовём \textit{$p$-нормой} $\|f\|_p$ значение следующего выражения:
	\[
		\|f\|_p := \ps{\int_E |f(x)|^pd\mu(x)}^{1/p}
	\]
	где $p \in (0; +\infty)$. При этом мы разрешаем $p$-норме принимать бесконечное значение.
\end{definition}

\begin{reminder} (Аксиомы нормы)
	Пусть $V$ --- линейное пространство над $\R$. Тогда на нём определена \textit{норма} $\|\cdot\| \colon V \to \R$, если этот оператор удовлетворяет следующим требованиям:
	\begin{enumerate}
		\item (Невырожденность) \(\forall v \in V\ \ \|v\| = 0 \Lra v = 0\)
		
		\item (Неравенство треугольника) \(\forall v, u \in V\ \ \|v + u\| \le \|v\| + \|u\|\)
		
		\item (Линейность) \(\forall \alpha \in \R, v \in V\ \ \|\alpha v\| = |\alpha| \cdot \|v\|\)
	\end{enumerate}
	Линейное пространство с нормой называется \textit{линейно нормированным пространством}.
\end{reminder}

\begin{note}
	Наша цель --- доказать, что при $p \in \lsi{1; +\infty}$ $p$-норма действительно является нормой для пространства классов эквивалентных функций.
\end{note}

\begin{note}
	Если $p \in (0; 1)$, то $\|f\|_p$ не удовлетворяет неравенству треугольника.
	
	Рассмотрим $f = \chi_{[0; 1/2]}$ и $g = \chi_{[1/2; 1]}$. Тогда $\|f\|_p = \|g\|_p = (1/2)^{1/p}$, при этом $\|f + g\|_p = 1$. Осталось посмотреть, при каких $p$ не выполнено неравенство треугольника:
	\[
		1 > (1/2)^{1/p} + (1/2)^{1/p};\ \ \ps{\frac{1}{2}}^{1/p} < \frac{1}{2} \Lra \frac{1}{p} > 1 \Lra p < 1 
	\]
\end{note}

\begin{proposition} (Неравенство Юнга)
	Пусть $p > 1$, $q > 1$ и $1 / p + 1 / q = 1$. Тогда выполнено неравенство:
	\[
		\forall a, b \ge 0\ \ a \cdot b \le \frac{a^p}{p} + \frac{b^q}{q}
	\]
\end{proposition}

\begin{proof}~
	\begin{itemize}
		\item Геометрическое доказательство. Нарисуем график $y = x^{p - 1}$, отметим на оси $x$ точку $x = a$, на оси $y$ точку $y = b$. Тогда верно неравенство \(ab \le S_1 + S_2\), где $S_1$ --- площадь под графиком до $a$, $S_2$ --- площадь над графиком до $b$. С одной стороны, $S_1 = \int_0^a x^{p - 1}dx = x^p / p$, а с другой стороны, $x = y^{1 / (p - 1)} = y^{q - 1}$, то есть $S_2 = \int_0^b y^{q - 1}dy$, что и требовалось доказать.
		
		\item Доказательство через выпуклость $\ln x$: неравенство тривиально выполнено для $a = 0$ или $b = 0$. Иначе, нам известно с первого курса, что функция натурального логарифма является выпуклой вверх. Это значит верность следующего утверждения:
		\[
			\forall a, b > 0\ \forall \alpha, \beta \ge 0, \alpha + \beta = 1\ \ \ \ln(\alpha a + \beta b) \ge \alpha\ln(a) + \beta\ln(b)
		\]
		В частности, можно рассмотреть $\alpha = 1 / p$ и $\beta = 1 / q$ и пропотенцировать это неравенство:
		\[
			\exp(\ln(a / p + b / q)) \ge \exp(\ln(a) / p + \ln(b) / q);\ \ \frac{a}{p} + \frac{b}{q} \ge a^{1/p} \cdot b^{1/q}
		\]
		Полученное неравенство эквивалентно неравенству Юнга (коль скоро отображение $x \mapsto x^t$ монотонно возрастает для $t > 1$, поэтому заменами $a \mapsto a^p$ и $b \mapsto b^q$ получаем требуемое).
	\end{itemize}
\end{proof}

\begin{theorem} (Неравенство Гёльдера)
	Для произвольных $f, g \colon E \to \sxR$ верно следующее утверждение:
	\[
		\forall p > 1, q := \frac{p}{p - 1} \quad \|f \cdot g\|_1 \le \|f\|_p \cdot \|g\|_q
	\]
\end{theorem}

\begin{proof}
	Произведём разбор случаев:
	\begin{itemize}
		\item Основной случай $\|f\|_p, \|g\|_q \in (0; +\infty)$. Докажем, что $\|\tilde{f} \cdot \tilde{g}\|_1 \le 1$, где $\tilde{f} = f / \|f\|_p$, $\tilde{g} = g / \|g\|_q$. Для этого распишем эту норму по определению (и воспользуемся неравенством Юнга):
		\[
			\|\tilde{f} \cdot \tilde{g}\|_1 = \int_E |\tilde{f}| \cdot |\tilde{g}|d\mu(x) \le \int_E \ps{\frac{|\tilde{f}|^p}{p} + \frac{|\tilde{g}|^q}{q}}d\mu(x) = \frac{1}{p} \int_E |\tilde{f}|^pd\mu(x) + \frac{1}{q} \int_E |\tilde{g}|^qd\mu(x) = 1
		\]
		Интегралы перед последним равенством равны единице. Почему? Распишем один из них подробнее:
		\[
			\int_E |\tilde{f}|^pd\mu(x) = \int_E (|f| / \|f\|_p)^pd\mu(x) = \frac{\|f\|_p^p}{\|f\|_p^p} = 1
		\]
		
		\item Оставшийся случай $\|f\|_p = +\infty$ или $\|g\|_q = +\infty$. В таком случае неравенство тривиально верно.
	\end{itemize}
\end{proof}

\begin{theorem} (Неравенство Минковского)
	Пусть $p \in \lsi{1; +\infty}$. Тогда $p$-норма обладает неравенством треугольника:
	\[
		\forall f, g \colon E \to \sxR\ \ \|f + g\|_p \le \|f\|_p + \|g\|_p
	\]
\end{theorem}

\begin{proof}
	Случай, когда $\|f\|_p$ или $\|g\|_p$ равны $\infty$ тривиально верен, а потому нам надо обосновать утверждение лишь при $\|f\|_p, \|g\|_p < \infty$. Проведём доказательство в 2 стадии:
	\begin{enumerate}
		\item $p = 1$. Тогда есть такая цепочка:
		\[
			\|f + g\|_1 = \int_E |f + g|d\mu(x) \le \int_E (|f| + |g|)d\mu(x) = \underbrace{\int_E |f|d\mu(x)}_{\|f\|_1} + \underbrace{\int_E |g|d\mu(x)}_{\|g\|_1}
		\]
		
		\item $p > 1$. Для начала докажем, что величина $\|f + g\|_p$ будет тоже конечной. Коль скоро отображение $x \mapsto x^p$ является выпуклым вверх при $x \ge 0$, то мы имеем право записать следующее неравенство:
		\[
			|f + g|^p \le (|f| + |g|)^p  = \ps{2 \cdot \frac{|f| + |g|}{2}}^p \le 2^p \ps{\frac{|f|^p}{2} + \frac{|g|^p}{2}}
		\]
		Стало быть, норма оценится следующим образом:
		\[
			\|f + g\|_p^p = \int_E |f + g|^pd\mu(x) \le 2^{p - 1} \int_E (|f|^p + |g|^p)d\mu(x) = 2^{p - 1}(\|f\|_p^p + \|g\|_p^p) < \infty
		\]
		Теперь проводить оценку нормы корректно:
		\[
			\int_E |f + g|^pd\mu(x) = \int_E |f + g| \cdot |f + g|^{p - 1}d\mu(x) \le \int_E |f| \cdot |f + g|^{p - 1}d\mu(x) + \int_E |g| \cdot |f + g|^{p - 1}d\mu(x)
		\]
		По неравенству Гёльдера каждое слагаемое можно оценить таким образом:
		\begin{multline*}
			\int_E |f| \cdot |f + g|^{p - 1}d\mu(x) \le 
			\\
			\ps{\int_E |f|^pd\mu(x)}^{1 / p} \cdot \ps{\int_E |f + g|^{q(p - 1)}d\mu(x)}^{1 / q} =
			\\
			\|f\|_p (\|f + g\|_p^p)^{1 / q} = \|f\|_p \cdot \|f + g\|_p^{p - 1}
		\end{multline*}
		Аналогично повторяем для второго и соединяем их:
		\[
			\|f + g\|_p^p \le \|f\|_p \cdot \|f + g\|_p^{p - 1} + \|g\|_p \cdot \|f + g\|_p^{p - 1}
		\]
		Если $\|f + g\|_p = 0$, то исходное неравенство тривиально верно. Иначе мы сокращаем на норму в полученном неравенстве и получаем требуемое.
	\end{enumerate}
\end{proof}

\begin{definition}
	\textit{Линейным пространством $\mathcal{L}_p(E)$} называется пространство измеримых функций $f \colon E \to \ole{\R}$, для которых $\|f\|_p < \infty$.
\end{definition}

\begin{definition}
	\textit{Линейным пространство $L_p(E)$} называется факторпространство \\ $\mathcal{L}_P(E) / \sim$, где $f \sim g$ тогда и только тогда, когда $f = g$ почти всюду на $E$.
\end{definition}

\begin{corollary}
	$L_p(E)$ является линейным нормированным пространством, где норма есть уже определённая ранее $p$-норма.
\end{corollary}

\begin{reminder}
	Метрическое пространство называется \textit{полным}, если любая фундаментальная последовательность в нём является сходящейся.
\end{reminder}

\begin{theorem}
	При $p \ge 1$ пространство $L_p(E)$ является полным.
\end{theorem}

\begin{exercise}
	Если числовая последовательность $\{a_n\}_{n = 1}^\infty \subset \R$ фундаментальна, то из неё можно выделить подпоследовательность $\{a_{n_k}\}_{k = 1}^\infty$ такую, что выполнено утверждение:
	\[
		\forall k \in \N\ \ |a_{n_{k + 1}} - a_{n_k}| < 2^{-k}
	\] 
\end{exercise}

\begin{proof} (теоремы)
	Рассуждения из упражнения верны и в случае любого линейно-нормированного пространства над $\R$ (вместо модуля, естественно, ставится норма). Итак, рассмотрим произвольную фундаментальную последовательность $\{f_n\}_{n = 1}^\infty \subseteq L_p(E)$. Тогда, существует подпоследовательность $\{f_{n_k}\}_{k = 1}^\infty \subseteq L_p(E)$ такая, что
	\[
		\forall k \in \N\ \ \|f_{n_{k + 1}} - f_{n_k}\|_p < 2^{-k}
	\]
	Покажем, что эта последовательность сходится почти всюду на $E$. Для этого запишем $f_{n_k}$ следующим образом:
	\[
		f_{n_k} = f_{n_1} + \sum_{i = 1}^k (f_{n_{i + 1}} - f_{n_i}) \Lora |f_{n_k}| \le |f_{n_1}| + h,\ \ h := \sum_{i = 1}^\infty |f_{n_{i + 1}} - f_{n_i}|
	\]
	Покажем, что почти всюду на $E$ величина $h$ будет конечна. Этого достаточно для сходимости почти всюду на $E$ (заметьте, $h$ не зависит от $k$). Исходя из определения ряда, введём следующие функции частичных сумм:
	\[
		h_N := \sum_{i = 1}^N |f_{n_{i + 1}} - f_{n_i}|
	\]
	Понятно, что эта последовательность является возрастающей, причём она сходится к $h$. По теореме Леви имеем право записать такое равенство:
	\[
		\int_E |h(x)|^pd\mu(x) = \int_E h^p(x)d\mu(x) = \lim_{N \to \infty} \int_E h_N^p(x)d\mu(x)
	\]
	Остаётся увидеть, что интеграл под пределом ограничен:
	\[
		\ps{\int_E h_N^p(x)d\mu(x)}^{1 / p} = \|h_N\|_p = \Bigg\|\sum_{i = 1}^N |f_{n_{i + 1}} - f_{n_i}|\Bigg\|_p \le \sum_{i = 1}^N \|f_{n_{i + 1}} - f_{n_i}\| \le \sum_{i = 1}^N 2^{-i} \le 1
	\]
	Коль скоро это верно для любого $N \in \N$, то и предел интегралов ограничен сверху той же единицей. Стало быть, $\int_E |h(x)|^pd\mu(x) \le 1 < \infty$, а это возможно лишь тогда, когда $h$ конечна почти всюду на $E$, что и требовалось. Обозначим предел подпоследовательности как функцию $f$. Покажем, что помимо поточечной сходимости почти всюду, у нас ещё есть сходимость по норме, то есть
	\[
		\|f_{n_k} - f\|_p \xrightarrow[k \to \infty]{} 0
	\]
	Для этого придётся поработать с подыинтегральной функцией нормы (то есть модулем разности $|f_{n_k} - f|$). Воспользуемся знанием про $h$:
	\[
		|f_{n_k}(x) - f(x)| \le |f_{n_k}(x)| + |f(x)| \le 2(|f_{n_1}(x)| + h) =: \phi(x)
	\]
	Коль скоро $f_{n_1}, h \in L_p(E)$, то и $\phi \in L_p(E)$. При этом $\int_E \phi^p(x)d\mu(x) < \infty$ (как раз за счёт определения нормы, её конечности и неотрицательности $\phi$). Это подводит нас к применению предельной теоремы Лебега, ибо соблюлись её условия:
	\begin{enumerate}
		\item $|f_{n_k} - f|^p \le \phi^p$
		
		\item $\int_E \phi^p(x)d\mu(x) < \infty$
		
		\item $f_{n_k} \to f$ почти всюду на $E$
	\end{enumerate}
	Отсюда получаем равенство по этой теореме:
	\[
		\lim_{k \to \infty} \int_E |f_{n_k}(x) - f(x)|^pd\mu(x) = \int_E \lim_{k \to \infty} |f_{n_k}(x) - f(x)|^pd\mu(x) = 0 \Lora \lim_{k \to \infty} \|f_{n_k} - f\|_p = 0
	\]
	Осталось доказать основное утверждение. Мы будем делать ставку на то, что вся последовательность сходится к $f$ в $L_p$. Это действительно так, ибо верно следующее:
	\[
		\exists N \in \N \such \forall n > N\ \ \|f_n - f\|_p \le \|f_n - f_{n_k}\|_p + \|f_{n_k} - f\|_p < \eps
	\]
	Если зафиксировать $\eps > 0$, то мы можем подобрать $N_1$ из фундаментальности так, чтобы левое слагаемое стало меньше $\eps / 2$. Для второго слагаемого найдём $N_2$ из уже доказанной сходимости, и останется лишь положить $N = \max\{N_1, N_2\}$.
\end{proof}

\begin{definition} (Не было у лектора)
	Линейно нормированное пространство $X$, обладающее полнотой по индуцированной метрике, называется \textit{банаховым}.
\end{definition}

\begin{corollary}
	$L_p$ является банаховым пространством.
\end{corollary}

\begin{definition}
	Пусть есть $f \colon E \to \sxR$, тогда её \textit{бесконечной нормой} $\|f\|_\infty$ назовём следующий инфинум:
	\[
		\|f\|_\infty := \inf_{S \colon \mu(S) = 0} \sup_{x \in E \bs S} |f(x)| = \min_{S \colon \mu(S) = 0} \sup_{x \in E \bs S} |f(x)| = \sup \{C \colon \mu(|f| > C) > 0\}
	\]
\end{definition}

\begin{definition}
	\textit{Линейным пространством $\cL_\infty$} называется следующее пространство функций:
	\[
		\cL_\infty := \{f \colon E \to \sxR, \|f\|_\infty < \infty\}
	\]
\end{definition}

\begin{note}
	Старым способом определяется $L_\infty := \cL_\infty / \sim$ и оно аналогично будет линейно нормированным пространством.
\end{note}

\begin{theorem} (Без доказательства)
	Линейно-нормированное пространство $L_\infty$ полно.
\end{theorem}

\begin{proof}
	\textcolor{red}{Это следует из того, что сходимость по норме $\|\cdot\|_\infty$ очень похожа на равномерную сходимость функций.}
\end{proof}

\begin{exercise}
	Пусть $p, q \in [1; +\infty]$ и $p < q$. Имеет место 2 утверждения:
	\begin{enumerate}
		\item Если $\mu(E) < \infty$, то верно вложение $L_q(E) \subset L_p(E)$, причём $\exists C > 0$ такое, что для любой функции из $L_p(E)$ верно неравенство $\|f\|_p \le C\|f\|_q$. \textit{Указание: используйте неравенство Гёльдера}
		
		\item Если $\mu(E) = \infty$, то $L_p \not\subset L_q$ и $L_q \not\subset L_p$ в общем случае.
	\end{enumerate}
\end{exercise}