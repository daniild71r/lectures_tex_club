\begin{definition}
	Если функция $F \colon E \to \sxR$ лежит в $L_{2\pi}$, то мы называем \textit{интегральным модулем непрерывости функции $F$} значение следующей функции:
	\[
		\omega_1(\delta, F) = \sup_{0 \le h \le \delta} \int_{[-\pi; \pi]} |F(t + h) - F(t)|d\mu(t)
	\]
\end{definition}

\begin{lemma} (У лектора этот было объединено с леммой \ref{average_continuous})
	Для $f \in L_{2\pi}$ имеет место следующий предел:
	\[
		\lim_{\delta \to 0+} \omega_1(\delta, f) = 0
	\]
\end{lemma}

\begin{proof}
	В лемме \ref{average_continuous} было установлено, что
	\[
		\forall a \le b, f \in L_1[a; b]\ \ \exists \lim_{t \to 0+} \int_{[a; b - t]} |f(x + t) - f(x)|d\mu(x) = 0
	\]
	В частности, это верно для $[a; b] = [-\pi; \pi]$. Так как при $\delta \to 0+$ верно, что $h \to 0+$, то и $\omega_1(\delta, f) \to 0$.
\end{proof}

\begin{lemma}
	Пусть $f \in L_{2\pi}$, а $g$ --- $2\pi$-периодическая и ограниченная функция. Тогда коэффициенты Фурье функции $\chi(t) = f(x + t)g(t)$ стремятся к нулю равномерно по $x$.
\end{lemma}

\begin{note}
	Аналогичное утверждение верно для $\kappa(t) = f(x - t)g(t)$
\end{note}

\begin{proof}
	По основной теореме об интеграле Лебега, функция $g$ является суммируемой на $[-\pi; \pi]$. По признаку суммируемости будет суммируема и $\chi$. Распишем стандартным образом коэффициенты ряда Фурье для $\chi$ (для $b_n$ аналогично):
	\begin{multline*}
		a_n(\chi) = \frac{1}{\pi} \int_{[-\pi; \pi]} f(x + t)g(t)\cos(nt)d\mu(t) = \{\text{замена } u = t - \pi / n\} =
		\\
		-\frac{1}{\pi} \int_{[-\pi; \pi]} f\ps{x + u + \frac{\pi}{n}}g\ps{u + \frac{\pi}{n}}\cos(nu)d\mu(u) =
		\\
		-\frac{1}{2\pi} \int_{[-\pi; \pi]} \Bigg(f\ps{x + t + \frac{\pi}{n}}g\ps{t + \frac{\pi}{n}} - f(x + t)g(t)\Bigg)\cos(nt)d\mu(t) = I_1(x) + I_2(x)
	\end{multline*}
	где $I_{1, 2}$ --- это следующие интегралы (добавили промежуточное слагаемое и разбили последний интеграл выше, полученный как среднее арифметическое первых двух, на два):
	\begin{align*}
		&{I_1(x) = -\frac{1}{2\pi}\int_{[-\pi; \pi]} \ps{f\ps{x + t + \frac{\pi}{n}} - f(x + t)}g\ps{t + \frac{\pi}{n}}\cos(nt)d\mu(t)}
		\\
		&{I_2(x) = -\frac{1}{2\pi}\int_{[-\pi; \pi]} \ps{g\ps{t + \frac{\pi}{n}} - g(t)}f(x + t)\cos(nt)d\mu(t)}
	\end{align*}
	Коль скоро $g$ ограничена, то пусть $M > 0$ таково, что $\forall x \in [-\pi; \pi]\ |g(x)| \le M$. Покажем, что $I_1$ равномерно сходится к нулю:
	\[
		|I_1| \le \frac{M}{2\pi} \int_{[-\pi; \pi]} \md{f\ps{x + t + \frac{\pi}{n}} - f(x + t)}d\mu(t) \le \frac{M}{2\pi} \omega_1\ps{\frac{\pi}{n}, f} \to 0,\ n \to \infty
	\]
	Так как мы получили оценку на интеграл вида константы на функцию, не зависящую от $x$, а только от $n$, то $I_1(x) \rra 0$ --- сходится к нулю равномерно. Теперь займёмся $I_2$: по лемме о всюду плотности множества финитных непрерывных функций, если зафиксировать $\eps > 0$, то мы можем разложить $f$ следующим образом:
	\[
		f = f_1 + f_2,\ f_1 \in C[-\pi; \pi],\ \int_{[-\pi; \pi]} |f_2(t)|d\mu(t) < \eps
	\]
	Так как $f_1$ непрерывна на отрезке, то по теореме Вейештрасса она ограничена. Скажем, что $|f_1| \le B$. Соответствующим образом мы можем теперь разбить интеграл $I_2 = I_{2, 1} + I_{2, 2}$ и разобраться с каждым по отдельности:
	\begin{align*}
		&{|I_{2, 1}| \le \frac{B}{2\pi} \omega_1\ps{\frac{\pi}{n}, g} \to 0,\ n \to \infty}
		\\
		&{|I_{2, 2}| \le \frac{M}{\pi}\eps}
	\end{align*}
	Обе оценки стремятся к нулю при $\eps \to 0$, $n \to \infty$ и не зависят от $x$, чего достаточно для завершения доказательства.
\end{proof}

\begin{theorem} (Принцип локализации Римана-Лебега)
	Пусть $f \in L_{2\pi}$ такова, что $f = 0$ на некотором интервале $I$. Тогда $\forall x \in I$ тригонометрический ряд Фурье функции $f$ сходится к нулю в точке $x$, причём равномерно на любом отрезке $S \subset I$.
\end{theorem}

\begin{proof}
	Выберем произвольный отрезок $S \subset I$. Так как $I$ --- это интервал, то есть открытое множество, то верно утверждение:
	\[
		\exists \eta > 0 \such \forall x \in S\ \forall h, |h| < \eta\ \ x + h \in S
	\]
	Зафиксируем такое $\eta$ и введём хитрый индикатор:
	\[
		\lambda(t) = \System{
			&{0,\ |t| < \eta}
			\\
			&{1,\ \eta \le |t| \le \pi}
		}
		= \chi_{[-\pi; \pi] \bs (-\eta; \eta)}
	\]
	Используя лемму о виде частичной суммы ряда Фурье с ядром Дирихле, запишем такую цепочку равенств:
	\begin{multline*}
		S_n(f, x) = \frac{1}{\pi} \int_{[-\pi; \pi]} f(x + t)D_n(t)d\mu(t) = \frac{1}{\pi} \int_{[-\pi; \pi]} f(x + t)\lambda(t)D_n(t)d\mu(t) =
		\\
		\frac{1}{2\pi} \int_{[-\pi; \pi]} f(x + t)\lambda(t) \frac{\sin(nt)\cos(t / 2)}{\sin(t / 2)}d\mu(t) + \frac{1}{2\pi} \int_{[-\pi; \pi]} f(x + t)\lambda(t)\cos(nt)d\mu(t)
	\end{multline*}
	\textcolor{red}{Ничего не понял, но очень интересно. Дописать}
\end{proof}

\begin{corollary}
	Если $f_1 = f_2$ на интервале $I$, причём $f_1, f_2 \in L_{2\pi}$, то тригонометрические ряды Фурье функций $f_1$ и $f_2$ равномерно сходятся на любом отрезке $S \subset I$. В частности, эти ряды равномерно сходятся $\forall x \in I$.
\end{corollary}

\begin{lemma}
	Пусть $f \in L_{2\pi}$, а $\phi_x(t, s)$ представляет собой следующую функцию:
	\[
		\phi_x(t, s) = \frac{f(x + t) + f(x - t) - 2s}{t}
	\]
	Тогда для сходимости тригонометрического ряда Фурье функции $f$ к $s$ в точке $x$ необходимо и достаточно, чтобы выполнялось утверждение:
	\[
		\exists \delta \in (0; \pi) \such \lim_{n \to \infty} \int_{[0; \delta]} \phi_x(t, s)\sin(nt)d\mu(t) = 0
	\]
	Если, кроме того, $f \in C(a; b)$, то для равномерной сходимости тригонометрического ряда Фурье функции $f$ к $f$ на $[a + \eps; b - \eps]$ для $\forall \eps \in (0; b - a)$ необходимо и достаточно, чтобы выполнялось утверждение:
	\[
		\exists \delta \in (0; \pi) \such \int_{[0; \delta]} \phi_x(t, s)\sin(nt)d\mu(t) \rra 0,\ n \to \infty
	\]
\end{lemma}