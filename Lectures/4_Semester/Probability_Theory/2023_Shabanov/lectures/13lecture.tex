\begin{corollary}
    В условиях ЦПТ верна ещё такая сходимость:
    \[
        \forall x \in \R \ \ F_n(x) := P \ps{\frac{S_n - \E S_n}{\sqrt{DS_n}} \le x} \xrightarrow[n \to \infty]{} \Phi(x) := \int_{-\infty}^{x} \frac{1}{\sqrt{2\pi}} e^{-y^2/2} dy
    \]
    Иными словами, можем записать сходимость в виде сходимости функций распределения.
\end{corollary}

\begin{proof}
    По Центральной Предельной Теореме имеем сходимость:
    \[
        \frac{S_n - \E S_n}{\sqrt{DS_n}} \xrightarrow{d} N(0, 1)
    \]
    $\Phi$ --- это функция распределения $N(0, 1)$. Из определения, у $\Phi$ есть плотность, то есть $\Phi$ абсолютно непрерывна (и просто непрерывна, в частности). Тогда, по теореме об эквивалентности сходимостей:
    \[
    	\frac{S_n - \E S_n}{\sqrt{DS_n}} \xrightarrow{d} N(0, 1) \Lra \Big(\forall x \in C(\Phi)\ \lim_{n \to \infty} F_n(x) = \Phi(x)\Big)
    \]
    В силу вышесказанного, $C(\Phi) = \R$, то есть мы показали требуемое.
\end{proof}

\begin{corollary}
    В условиях ЦПТ
    \[
        \sqrt{n} \ps{\frac{S_n}{n} - a} \xrightarrow{d} N(0, \sigma^2)
    \]
\end{corollary}

\begin{proof}
	Будем использовать то же обозначение $\eta_n = (\xi_n - a) / \sigma$. Тогда мы знаем, что сходимость по распределнию устроена так:
	\[
		\eta_n \xrightarrow{d} \eta \Lra \forall f \colon \R \to \R \text{ --- непрерывная ограниченная }\E f(\eta_n) \xrightarrow[n \to \infty]{} \E f(\eta)
	\]
	Так как для $\eta_n$ и $c\eta_n, c \in \R$ множества композиций с непрерывными ограниченными функциями одинаковы, то мы можем модифицировать сходимость из ЦПТ на любую наперёд заданную константу:
	\[
		\sqrt{n}\ps{\frac{S_n}{n} - a} = \frac{S_n - na}{\sqrt{n}} = \frac{S_n - an}{\sigma\sqrt{n}}\sigma \xrightarrow{d} \sigma N(0, 1) = N(0, \sigma^2)
	\]
\end{proof}

\begin{note}
    Такая формулировка осмыслена и при $\sigma = 0$: действительно, тогда $\xi_n$ --- константы, и в сходимости 
    \[
        \sqrt{n} \ps{\frac{S_n}{n} - a} \xrightarrow{d} N(0, \sigma^2)
    \]
    слева и справа стоят тождественные нули.
    
    Это будет полезно при обобщении на многомерный случай: в такой ЦПТ в роли $\sigma$ будет выступать матрица, которая может быть вырожденной и при этом не обязательно нулевой, тогда обратить её (<<делить на неё>>) мы не сможем, но сможем использовать в такой форме записи.
\end{note}

\begin{note}
    ЦПТ позволяет в определённом смысле оценить скорость сходимости УЗБЧ. Так, для УЗБЧ в форме Колмогорова нам были нужны независимые одинаково распределённые случайные величины, у которых конечное матожидание, а в ЦПТ к этому добавляется конечность дисперсии. Перепишем ЦПТ в терминах сходимости функции распределения:
    \[
    	\forall x \in \R\ \ P\ps{\sqrt{n}\ps{\frac{S_n}{n} - a} \le x} \xrightarrow[n \to \infty]{} P(\xi \le x),\ \xi \sim N(0, \sigma^2)
    \]
    Так как нормальное распределение абсолютно непрерывно, то можем выбрать такие $u' \le u$, что $P(u' \le |\xi| \le u) = 0.99$. Если применить эти числа в ЦПТ, то получится такой факт:
    \[
    	P\ps{\frac{u'}{\sqrt{n}} \le \md{\frac{S_n}{n} - a} \le \frac{u}{\sqrt{n}}} \xrightarrow[n \to \infty]{} 0.99
    \]
    Иными словами, есть такой номер $N_0 \in \N$, что, начиная с него, с огромной вероятностью будет выполнена оценка:
    \[
    	\md{\frac{S_n}{n} - a} = O\ps{\frac{1}{\sqrt{n}}}
    \]
    При этом данная оценка неулучшаема.
\end{note}

\begin{note}
    Теперь хотим оценить скорость сходимости в самой ЦПТ.
\end{note}

\begin{theorem} (Берри-Эссеена, без доказательства)
    Пусть $\{\xi_n\}_{n = 1}^\infty$ --- независимые одинаково распределённые случайные величины, причём $\E|\xi_1-\E\xi_1|^3 < +\infty$ и пусть $D\xi_1 \neq 0$. Обозначим $S_n = \xi_1 \plusdots \xi_n$ и $T_n = \frac{S_n - \E S_n}{\sqrt{DS_n}}$, $F_n$ --- функция распределения $T_n$, $\Phi$ --- функция распределения $N(0, 1)$. Тогда, имеет место следующее неравенство:
    \[
        \sup_{x \in \R} |F_n(x)-\Phi(x)| \le \frac{c \ \E|\xi_1-\E\xi_1|^3}{\sigma^3 \sqrt{n}}
    \]
    где $c > 0$ --- абсолютная константа
\end{theorem}

\begin{note}
    Здесь уже абсолютно явная оценка сходимости $O(1 / \sqrt{n})$. Про константу $c$ можно сказать следующее:
    \begin{align*}
    & c \le 0,478\ldots < \frac{1}{2}
    \\
    & c \ge \frac{1}{\sqrt{2\pi}} = 0,309\ldots
    \end{align*}
\end{note}

\begin{example}
    Складываются $10^4$ чисел, каждое из которых было вычислено с точностью $10^{-6}$. В каких пределах с вероятностью $0,98$ лежит суммарная ошибка, если считать ошибки независимыми?
\end{example}

\begin{anote}
	Хоть это нигде и не оговаривалось, но техника, которую мы развивали все эти страницы, теперь позволяет получить ответ на эту задачу для абсолютно любого вероятностного пространства $(\Omega, \F, P)$, которое способно реализовать всё условие задачи.
\end{anote}

\begin{solution}
    Пусть $\xi_1, \ldots, \xi_n \sim U[-10^{-6}, 10^{-6}]$, $n = 10^4$ --- независимые случайные величины. Посчитаем моменты различных порядков:
    \begin{align*}
        &{\E\xi_1 = \ldots = \E\xi_n = 0}
        \\
        &{D\xi_1 = \ldots = D\xi_n = \E|\xi_1 - \E\xi_1|^2 = \E\xi_1^2 = \int_{\R} x^2 \chi[-10^{-6}, 10^{-6}] \frac{1}{2 \cdot 10^{-6}}dx = \frac{10^{-12}}{3}}
        \\
        &{\E|\xi_1 - \E\xi_1|^3=\E|\xi_1|^3 =  \int_{\R} |x|^3 \chi[-10^{-6}, 10^{-6}] \frac{1}{2 \cdot 10^{-6}} dx = \frac{10^{-18}}{4}}
    \end{align*}

    Обозначим $S_n = \xi_1 \plusdots \xi_n$, Ф --- функция распределения $\xi \sim N(0, 1)$. Выполнены условия теоремы Берри-Эссеена, тогда:
    \[
        \forall x \in \R \ \ \md{P \ps{\frac{S_n-\E S_n}{\sqrt{DS_n}} \le x} - \Phi(x)} \le \frac{c \ \E|\xi_1-\E\xi_1|^3}{(D\xi_1)^{3/2} \sqrt{n}}
    \]

    Обозначим $\sigma = D\xi_1$ и оценим модуль отклонения вероятности из ЦПТ с вероятностью по нормальному распределению:
    \[
        \forall x \in \R\ \ \md{P\ps{\frac{S_n}{\sigma \sqrt{n}} \le x} - \Phi(x)} \le \frac{\frac{1}{2} \cdot \frac{10^{-18}}{4}}{(\frac{10^{-12}}{3})^{3/2}\sqrt{10^4}} = \frac{\frac{1}{2} \cdot \frac{10^{-18}}{4}}{\frac{10^{-18}}{3 \sqrt{3}}\sqrt{10^4}} = \frac{3\sqrt{3}}{8 \cdot 100}
    \]

    Теперь мы можем дать оценку на просто вероятность среднего оказаться в отрезке $[-x; x]$:
    \begin{multline*}
        P\ps{\md{\frac{S_n}{\sigma \sqrt{n}}} \le x} = \md{P\ps{\md{\frac{S_n}{\sigma \sqrt{n}}} \le x}} = \md{P\ps{\frac{S_n}{\sigma \sqrt{n}} \le x} - P\ps{\frac{S_n}{\sigma \sqrt{n}} < -x}} =
        \\
        = \md{P \ps{\frac{S_n}{\sigma \sqrt{n}} \le x} - P \ps{\frac{S_n}{\sigma \sqrt{n}} < -x} + \Phi(x) - \Phi(x) + P(\xi < -x) - \Phi(-x)} =
        \\
        = \md{\Phi(x) - \Phi(-x) + P\ps{\frac{S_n}{\sigma \sqrt{n}} \le x} - \Phi(x) - P\ps{\frac{S_n}{\sigma \sqrt{n}} < -x} + P(\xi < -x)} \ge
        \\
        |\Phi(x) - \Phi(-x)| - \md{P\ps{\frac{S_n}{\sigma \sqrt{n}} \le x} - \Phi(x)} - \md{P\ps{\frac{S_n}{\sigma \sqrt{n}} < -x} - P(\xi < -x)} \ge
        \\
        |\Phi(x) - \Phi(-x)| - \frac{3\sqrt{3}}{8 \cdot 100} - \frac{3\sqrt{3}}{8 \cdot 100} = \Phi(x) - \Phi(-x) - \frac{3\sqrt{3}}{4 \cdot 100}
    \end{multline*}

    Для функции распределения стандартного нормального распределения можем подобрать параметры:
    \[
        x = 2,807,\ \ \Phi(x) - \Phi(-x) \ge 0,995
    \]

    С учётом этого получим:
    \begin{align*}
        & P \ps{\md{\frac{S_n}{\sigma \sqrt{n}}} \le 2,807} \ge 0,995 - \frac{3\sqrt{3}}{4 \cdot 100}
        \\
        & P \ps{|S_n| \le \sqrt{\frac{10^{-12}}{3}} \sqrt{10^4} \cdot 2,807} \ge 0,98
        \\
        & P(|S_n| \le 1,7 \cdot 10^{-4}) \ge 0,98
    \end{align*}
\end{solution}

\section{Сходимости случайных векторов}

\begin{note}
	Далее мы резервируем обозначение $(\Omega, \F, P)$ под вероятностное пространство.
\end{note}

\begin{definition}
	Пусть $\{\xi_n\}_{n = 1}^\infty$, $\xi$ --- случайные векторы из $\R^m$. Тогда \textit{$\xi_n$ сходится к $\xi$ с вероятностью 1 ($P$-почти наверное)}, если выполнено равенство:
	\[
		P\ps{\lim_{n \to \infty} \xi_n = \xi} = 1
	\]
	Обозначается как $\xi_n \xrightarrow{P \text{ п.н.}} \xi$
\end{definition}

\begin{definition}
	Пусть $\{\xi_n\}_{n = 1}^\infty$, $\xi$ --- случайные векторы из $\R^m$. Тогда \textit{$\xi_n$ сходится к $\xi$ по вероятности}, если выполнено утверждение:
	\[
		\forall \eps > 0\ \ P(\|\xi_n - \xi\|_2 \ge \eps) \xrightarrow[n \to \infty]{} 0, \text{ где } \|x\|_2 = \sqrt{x_1^2 + \ldots + x_m^2}
	\]
	Обозначается как $\xi_n \xrightarrow{P \text{ п.н.}} \xi$
\end{definition}

\begin{definition}
	Пусть $\{\xi_n\}_{n = 1}^\infty$, $\xi$ --- случайные векторы из $\R^m$. Тогда \textit{$\xi_n$ сходится к $\xi$ по распределению}, если выполнено утверждение:
	\[
		\forall f \colon \R^m \to \R \text{ --- непрерывная ограниченная }\ \E f(\xi_n) \xrightarrow[n \to \infty]{} \E f(\xi)
	\]
	Обозначается как $\xi_n \xrightarrow{d} \xi$
\end{definition}

\begin{exercise}
    Пусть $\xi_n = (\xi_{1, n}, \ldots, \xi_{m, n}),\ n \in \N,\ \ \xi = (\xi_1, \ldots, \xi_m)$ --- случайные векторы. Тогда
    \begin{enumerate}
        \item $\xi_n \xrightarrow{P\text{ п.н.}} \xi \Lolra \Big(\forall i \in \range{1}{m}\ \xi_{i, n} \xrightarrow{P\text{ п.н.}} \xi_i\Big)$

        \item $\xi_n \xrightarrow{P} \xi \Lolra \Big(\forall i \in \range{1}{m}\ \xi_{i, n} \xrightarrow{P} \xi_i\Big)$

        \item $\xi_n \xrightarrow{d} \xi \Lolra \Big(\forall i \in \range{1}{m}\ \xi_{i, n} \xrightarrow{d} \xi_i\Big)$
    \end{enumerate}
\end{exercise}

\begin{theorem} (без доказательства)
    Пусть ${\xi_n}_{n = 1}^\infty$, $\xi$ --- случайные векторы в $\R^m$. Тогда эквивалентны следующие утверждения:
    \begin{enumerate}
        \item $\xi_n \xrightarrow{d} \xi$
        
        \item $\forall x \in C(F_\xi)\ F_{\xi_n}(x) \xrightarrow[n \to \infty]{} F_{\xi}(x)$
    \end{enumerate}
\end{theorem}

\begin{note}
	Просто по определению и по теореме о замене переменной в интеграле Лебега
	\[
		\xi_n \xrightarrow{d} \xi \Leftrightarrow P_{\xi_n} \xrightarrow{w} P_{\xi},
	\]
	где справа стоят соответствующие распределения случайных векторов. По теореме Александрова
	\[
		P_{\xi_n} \xrightarrow{w} P_{\xi} \Leftrightarrow P_{\xi_n} \Ra P_{\xi}.
	\]
	Это начало доказательства теоремы выше.
\end{note}

\begin{lemma} (О взаимоотношении видов сходимостей)
    Пусть $\{\xi_n\}_{n = 1}^\infty$, $\xi$ --- случайные векторы в $\R^m$. Тогда:
    \begin{enumerate}
        \item $\xi_n \xrightarrow{P\text{ п.н.}} \xi \Ra \xi_n \xrightarrow{P} \xi$
        
        \item $\xi_n \xrightarrow{P} \xi \Ra \xi_n \xrightarrow{d} \xi$
    \end{enumerate}
\end{lemma}

\begin{proof}~
    \begin{enumerate}
        \item С учётом упражнения достаточно доказать в одномерном случае, а это уже делали.
        \item Доказательство в точности повторяет доказательство одномерного случая, в виду того, что оно достаточно громоздкое, не будем проделывать ещё раз.
    \end{enumerate}
\end{proof}

\begin{theorem} (О наследовании сходимости)
    Пусть $\{\xi_n\}_{n = 1}^\infty$, $\xi$ --- случайные векторы в $\R^m$. Пусть $h \colon \R^m \to \R^k$ --- непрерывна почти всюду относительно распределения случайного вектора $\xi$ (то есть $\exists B \in \B(\R^m) \colon P_\xi(B) = 1 \wedge h$ непрерывна на $B$). Тогда:
    \begin{enumerate}
        \item $\xi_n \xrightarrow{P\text{ п.н.}} \xi \Ra h(\xi_n) \xrightarrow{P\text{ п.н.}} h(\xi)$
        \item $\xi_n \xrightarrow{P} \xi \Ra h(\xi_n) \xrightarrow{P} h(\xi)$
        \item $\xi_n \xrightarrow{d} \xi \Ra h(\xi_n) \xrightarrow{d} h(\xi)$
    \end{enumerate}
\end{theorem}

\begin{proof}~
    \begin{enumerate}
        \item Заметим, что так как $h$ непрерывна на $B$, то:
        \[
            \xi_n(\omega) \to \xi(\omega) \wedge \xi(\omega) \in B \Ra h(\xi_n(\omega)) \to h(\xi(\omega))
        \]
        Отсюда получаем:
        \[
            P(h(\xi_n) \to h(\xi)) \ge P(\xi_n \to \xi \wedge \xi \in B) = 1
        \]
        Последнее верно, ибо $P(\xi_n \to \xi) = P(\xi \in B) = 1$, а мы знаем формулу $P(A \cup B) + P(A \cap B) = P(A) + P(B)$, для любых событий $A, B$.

        \item Хотим доказать, что $h(\xi_n) \xrightarrow{P} h(\xi)$, то есть:
        \[
            \forall \eps > 0 \ \ P(||h(\xi_n) - h(\xi)||_2 \ge \eps) \xrightarrow[n \to \infty]{} 0
        \]
        Предположим противное. Тогда
        \[
            \exists \eps > 0 \ \ \exists \delta > 0 \ \ \exists \{n_k\}_{k = 1}^\infty \subseteq \N \such \forall k \in \N \ \ P(||h(\xi_{n_k}) - h(\xi)||_2 \ge \eps) \ge \delta
        \]
        Но по одному из результатов главы про сходимость случайных величин:
        \[
            \xi_{n_k} \xrightarrow{P} \xi \Ra \exists \{n_{k_l}\}_{l = 1}^\infty \subseteq \{n_k\}_{k = 1}^\infty \ \ \xi_{n_{k_l}} \xrightarrow{P\text{ п.н.}} \xi
        \]
        Отметим, что там результат был доказан для одномерного случая. Но так как сходимости почти наверное и по вероятности эквивалентны таким же покоординатным сходимостям, мы можем постепенно выбирать подпоследовательность: сначала сходящуюся почти наверное по первой координате, затем из неё сходящуюся почти наверное по первой и второй координате, и так далее. Теперь, согласно первому пункту и лемме о взаимоотношении видов сходимостей:
        \[
            \xi_{n_{k_l}} \xrightarrow{P\text{ п.н.}} \xi \Ra h(\xi_{n_{k_l}}) \xrightarrow{P\text{ п.н.}} h(\xi) \Ra h(\xi_{n_{k_l}}) \xrightarrow{P} h(\xi)
        \]
        Тогда получаем, что:
        \[
            0 < \delta \le P(||h(\xi_{n_{k_l}}) - h(\xi)||_2 \ge \eps) \xrightarrow[l \to \infty]{} 0 \text{ --- противоречие}
        \]

        \item Обозначим $Q_n$ --- распределение случайного вектора $h(\xi_n)$, $Q$ --- распределение случайного вектора $h(\xi)$. Хотим доказать, что $h(\xi_n) \xrightarrow{d} h(\xi)$. Как поняли в одном из замечаний выше, для этого нам достаточно доказать, что $Q_n \xrightarrow{w} Q$. По теореме Александрова достаточно проверить, что:
        $\varlimsup_{n \to \infty} Q_n(F) \le Q(F)$ для любого замкнутого $F \subseteq \R^m$. Также обозначим $P_n$ --- распределение случайного вектора $\xi_n$, $P_{\xi}$ --- распределение случайного вектора $\xi$. Знаем, что $\xi_n \xrightarrow{d} \xi$, по тому же замечанию и по теореме Александрова из этого следует, что:
        \[
            \forall F \subseteq \R^m \text{ --- замкнутое }\ \varlimsup_{n \to \infty} P_n(F) \le P_{\xi}(F)
        \]
        Далее будем обозначать $[A]$ --- замыкание множества $A$. Тогда получим:
        \begin{multline*}
            \varlimsup_{n \to \infty} Q_n(F) = \varlimsup_{n \to \infty} P(h(\xi_n) \in F) = \varlimsup_{n \to \infty} P(\xi_n \in h^{-1}(F)) =
            \\
            \varlimsup_{n \to \infty} P_n(h^{-1}(F)) \le \varlimsup_{n \to \infty} P_n([h^{-1}(F)]) \le P_{\xi}([h^{-1}(F)]) = P(\xi \in [h^{-1}(F)])
        \end{multline*}
        Теперь утверждается, что $[h^{-1}(F)] \subseteq h^{-1}(F) \cup (\R^m \setminus B)$ для того самого множества $B$ из условия теоремы. Действительно:
        \[
        	x \in [h^{-1}(F)] \Lra \exists \{x_n\}_{n = 1}^\infty \subseteq h^{-1}(F) \colon x = \lim_{n \to \infty} x_n
        \]
        Остаётся 2 варианта: либо $x \in \R^m \bs B$, либо $x \in B$. Во втором случае следующие 3 факта полностью обосновывают принадлежность к $h^{-1}(F)$:
        \begin{enumerate}
        	\item $h$ непрерывна на $B$, то есть $h(x) = \lim_{n \to \infty} h(x_n)$
        	
        	\item $\forall n \in \N\ x_n \in h^{-1}(F) \Lora \forall n \in \N\ h(x_n) \in F$
        	
        	\item $F$ --- замкнутое множество. Соединяя вышенаписанные факты, получаем требуемое: $h(x) = \lim_{n \to \infty} h(x_n) \in F$
        \end{enumerate}
		Значит, мы можем продолжить цепочку неравенств с верхним пределом так:
		\begin{multline*}
			\varlimsup_{n \to \infty} Q_n(F) \le P(\xi \in [h^{-1}(F)]) \le
			\\
			P(\xi \in h^{-1}(F)) + P(\xi \in \R^m \bs B) = P(h(\xi) \in F) + 0 = Q(F)
		\end{multline*}
        Это ровно то, что нам нужно было проверить.
    \end{enumerate}
\end{proof}

\begin{lemma}
    Пусть $\xi_n$ --- случайные величины, $c = const$, $c \in \R$. Тогда следующие утверждения эквивалентны:
    \begin{enumerate}
        \item $\xi_n \xrightarrow{P} c$
        \item $\xi_n \xrightarrow{d} c$
    \end{enumerate}
\end{lemma}

\begin{proof}~
    \begin{itemize}
        \item[$1 \Ra 2$] Уже доказано в более общем случае.

        \item[$2 \Ra 1$] Запишем сходимость по распределению в терминах функций распределения:
        \[
            \xi_n \xrightarrow{d} c \Lra \Big(\forall x \in C(F_c)\ \lim_{n \to \infty} F_{\xi_n}(x) = F_c(x)\Big)
        \]
		С учётом определения $F_c$ для константы
        \[
            F_c(x) = \System{
                        & 1,\ x \ge c,
                        \\
                        & 0,\ x < c
                    }
        \]
        Заключаем, что $C(F_c) = \R \bs \{c\}$. А показать нам надо следующее:
        \[
            \xi_n \xrightarrow{P} c \Lra \Big(\forall \eps > 0\ \ P(|\xi_n - c| \ge \eps) \xrightarrow[n \to \infty]{} 0\Big)
        \]
		Зафикируем $\eps > 0$ и выразим вероятность выше через функции распределения:
        \begin{multline*}
            P(|\xi_n - c| \ge \eps) = P(\xi_n - c \ge \eps) + P(\xi_n - c \le -\eps) =
            \\
            1 - P(\xi_n < c + \eps) + P(\xi_n \le c - \eps) \le 1 - P \ps{\xi_n \le c + \frac{\eps}{2}} + P(\xi_n \le c - \eps) =
            \\
            1 - F_{\xi_n} \ps{c + \frac{\eps}{2}} + F_{\xi_n}(c - \eps) \xrightarrow[n \to \infty]{} 1 - F_c \ps{c + \frac{\eps}{2}} + F_c(c - \eps) = 1 - 1 + 0 = 0
        \end{multline*}
    \end{itemize}
\end{proof}

\begin{theorem} (Лемма Слуцкого)
    Пусть $\xi_n \xrightarrow{d} \xi$, $\eta_n \xrightarrow{d} c = const$ --- случайные величины. Тогда
    \begin{enumerate}
        \item $\xi_n + \eta_n \xrightarrow{d} \xi + c$
        
        \item $\xi_n \cdot \eta_n \xrightarrow{d} \xi \cdot c$
    \end{enumerate}
\end{theorem}

\begin{note}
    Если бы была дана сходимость случайных векторов $(\xi_n, \eta_n) \xrightarrow{d} (\xi, c)$, то утверждение леммы Слуцкого мгновенно бы следовало из теоремы о наследовании сходимости.
\end{note}

\begin{proof}~
    \begin{enumerate}
        \item Необходимо и достаточно доказать сходимость функций распределения:
        \[
        	\forall x \in C(F_{\xi + c})\ \ \lim_{n \to \infty} F_{\xi_n + \eta_n}(x) = F_{\xi + c}(x)
        \]

		Основная идея доказательства состоит в том, чтобы хорошо оценить верхние и нижние пределы для $F_{\xi_n + \eta_n}(x)$ (не без помощи леммы выше). Пусть $x \in C(F_{\xi + c})$. Тогда просто по определению $x - c$ --- точка непрерывности $F_{\xi}$:
		\[
			F_{\xi + c}(x) = P(\xi + c \le x) = P(\xi \le x - c) = F_\xi(x - c)
		\]
		В силу той же непрерывности, мы можем выбрать сколь угодно малое $\eps > 0$, так, что $x - c \pm \eps$ --- точки непрерывности функции $F_\xi$. Действительно, можем так сделать, так как $F_\xi$ монотонна, имеет не более, чем счётное множество точек разрыва. А если бы не могли так выбрать, получили бы, что в какой-то окрестности точки $x - c$ из любой пары точек $x - c \pm \eps$ хотя бы одна является точкой разрыва, то есть в этой окрестности мощность множества точек разрыва не меньше мощности множества точек непрерывности, то есть множество точек разрыва континуально. Итак, зафиксируем нужное $\eps > 0$.
		\begin{itemize}
			\item[$\varlimsup$] Сделаем оценку сверху на $F_{\xi_n + \eta_n}(x)$ следующим образом:
			\begin{multline*}
				F_{\xi_n + \eta_n}(x) = P(\xi_n + \eta_n \le x) =
				\\
				P(\xi_n + \eta_n \le x \wedge c - \eta_n \ge \eps) + P(\xi_n + \eta_n \le x \wedge c - \eta_n < \eps) \le
				\\
				P(c - \eta_n \ge \eps) + P(\xi_n + c < x + \eps) \le
				\\
				P(|c - \eta_n| \ge \eps) + P(\xi_n \le x - c + \eps)
			\end{multline*}
			По доказанной лемме, $\eta_n \xrightarrow{d} c \Ra \eta_n \xrightarrow{P} c$. Это даёт следующий предел:
			\[
				\forall \eps > 0\ \ P(|c - \eta_n| \ge \eps) = P(|\eta_n - c| \ge \eps) \xrightarrow[n \to \infty]{} 0
			\]
			Также вспомним, что $x - c + \eps \in C(F_\xi)$ и $\xi_n \xrightarrow{d} \xi$ по условию:
			\[
				\lim_{n \to \infty} P(\xi_n \le x - c + \eps) = \lim_{n \to \infty} F_{\xi_n}(x - c + \eps) = F_\xi(x - c + \eps)
			\]
			Теперь мы можем навесить верхний предел на исходное неравенство и получить желаемую оценку:
			\begin{multline*}
				\varlimsup_{n \to \infty} F_{\xi_n + \eta_n}(x) \le \varlimsup_{n \to \infty} P(|c - \eta_n| \ge \eps) + \varlimsup_{n \to \infty} P(\xi_n \le x - c + \eps) =
				\\
				\lim_{n \to \infty} P(|c - \eta_n| \ge \eps) + \lim_{n \to \infty} P(\xi_n \le x - c + \eps) = F_\xi(x - c + \eps)
			\end{multline*}
			
			\item[$\varliminf$] Чтобы получить оценку снизу на $F_{\xi_n + \eta_n}(x)$, получим оценку сверху на $1 - F_{\xi_n + \eta_n}(x)$:
			\begin{multline*}
				1 - F_{\xi_n + \eta_n}(x) = P(\xi_n + \eta_n > x) =
				\\
				P(\xi_n + \eta_n > x \wedge c - \eta_n \le -\eps) + P(\xi_n + \eta_n > x \wedge c - \eta_n > -\eps) \le
				\\
				P(c - \eta_n \le -\eps) + P(\xi_n + c > x - \eps) \le P(|c - \eta_n| \ge \eps) + P(\xi_n > x - c - \eps)
			\end{multline*}
			Из этого получаем неравенство для только $F_{\xi_n + \eta_n}(x)$:
			\[
				F_{\xi_n + \eta_n}(x) \ge 1 - P(|c - \eta_n| \ge \eps) - P(\xi_n > x - c - \eps) = P(\xi_n \le x - c - \eps) - P(|c - \eta_n| \ge \eps)
			\]
			Теперь нужно воспользоваться всё тем же фактом $\eta_n \xrightarrow{d} c \Ra \eta_n \xrightarrow{P} c$ и непрерывностью $F_\xi$ в точке $x - c - \eps$:
			\[
				\lim_{n \to \infty} P(\xi_n \le x - c - \eps) = \lim_{n \to \infty} F_\xi(x - c - \eps) = F_\xi(x - c - \eps)
			\]
			Навешиваем нижние пределы и получаем последнюю оценку:
			\begin{multline*}
				\varliminf_{n \to \infty} F_{\xi_n + \eta_n}(x) \le \varliminf_{n \to \infty} P(\xi_n \le x - c - \eps) - \varliminf_{n \to \infty} P(|c - \eta_n| \ge \eps) =
				\\
				\lim_{n \to \infty} P(\xi_n \le x - c - \eps) - \lim_{n \to \infty} P(|c - \eta_n| \ge \eps) = F_\xi(x - c - \eps)
			\end{multline*}
		\end{itemize}
		В итоге мы получили такое утверждение:
		\[
			\forall \eps > 0\ \ F_\xi(x - c - \eps) \le \varliminf_{n \to \infty} F_{\xi_n + \eta_n}(x) \le \varlimsup_{n \to \infty} F_{\xi_n + \eta_n}(x) \le F_\xi(x - c + \eps)
		\]
		Так как $x - c \in C(F_\xi)$, то можем устремить $\eps$ к нулю и получить равенство. Стало быть:
		\[
			\lim_{n \to \infty} F_{\xi_n + \eta_n}(x) = F_\xi(x - c) = F_{\xi + c}(x)
		\]

        \item Доказательство второй части теоремы очень похоже на доказательство первой части. Вместо $x - c$ возникнет $x/c$, сложение заменится на умножение. Нужно только аккуратно разобрать случаи, где возникает деление на ноль. Тем не менее, эту часть оставляем без доказательства.
    \end{enumerate}
\end{proof}

\begin{theorem} (Обобщение леммы Слуцкого, без доказательства)
    Пусть $\xi_n \xrightarrow{d} \xi,\ \eta_n \xrightarrow{d} c = const$ --- случайные величины. Тогда $(\xi_n, \eta_n) \xrightarrow{d} (\xi, c)$.
\end{theorem}

\begin{example}
    Пусть известно, что $X_1, \ldots, X_n$ --- независимые одинаково распределённые случайные величины, причём $X_1 \sim Bin(1, p) = Bern(p)$, но мы не знаем $p$. Хотим как раз оценить $p$. Покажем, как это можно организовать, если мы знаем просто значения $X_1, \ldots, X_n$ на каких-то хороших исходах. Обозначим среднее арифметическое $n$ величин за $\ol{X}$ 
    \[
        \ol{X} := \frac{X_1 + \ldots + X_n}{n}
    \]
    За счёт известного распределения, мы знаем матожидание и дисперсию: $\E X_1 = p$, $DX_1 = p(1 - p)$. Применим ЦПТ:
    \[
        \frac{n\ol{X}-np}{\sqrt{np(1 - p)}} = \frac{\sqrt{n}(\ol{X} - p)}{\sqrt{p(1 - p)}} \xrightarrow[n \to \infty]{d} N(0, 1)
    \]
    Хочется заменить знаменатель последней дроби на что-то, зависящее от $X_1, \ldots, X_n$, чтобы сходимость при этом сохранилась (тогда мы можем перейти к вероятностям и выразить границы оценки $p$). Итак, по УЗБЧ в форме Колмогорова:
    \[
        \ol{X} \xrightarrow{P\text{ п.н.}} p
    \]
    По теореме о наследовании сходимости:
    \[
        \sqrt{\ol{X}(1 - \ol{X})} \xrightarrow{P\text{ п.н.}} \sqrt{p(1 - p)}
    \]
    По этой же теореме, так как $p=const$:
    \[
        \frac{\sqrt{p(1 - p)}}{\sqrt{\ol{X}(1 - \ol{X})}} \xrightarrow{P\text{ п.н.}} 1
    \]

    Сходимость почти наверное влечёт сходимость по распределению, поэтому, применяя лемму Слуцкого для произведения, имеем уже такую сходимость:
    \[
        \frac{\sqrt{n}(\ol{X} - p)}{\sqrt{\ol{X}(1 - \ol{X})}} = \frac{\sqrt{n}(\ol{X} - p)}{\sqrt{p(1 - p)}} \frac{\sqrt{p(1 - p)}}{\sqrt{\ol{X}(1 - \ol{X})}} \xrightarrow{d} N(0, 1) \cdot 1 = N(0, 1)
    \]

    Нормальное распределение абсолютно непрерывно, поэтому, в терминах функций распределения:
    \[
        \forall u \in \R\ \ P \ps{\frac{\sqrt{n}(\ol{X} - p)}{\sqrt{\ol{X}(1 - \ol{X})}} \le u} \xrightarrow[n \to \infty]{} \Phi(u)
    \]
    Здесь $\Phi$ --- функция распределения стандартного нормального распределения.

    Далее, применяя оценку из примера к теореме Берри-Эссеена, получим:
    \[
        P \ps{\md{\frac{\sqrt{n}(\ol{X} - p)}{\sqrt{\ol{X}(1 - \ol{X})}}} \le 2,807} \xrightarrow[n \to \infty]{} \Phi(2,807) -  \Phi(-2,807) \ge 0,995
    \]

    Таким образом, с вероятностью, стремящейся к чему-то большему, чем $0,995$:
    \[
        \ol{X} - \frac{2.807}{\sqrt{n}} \sqrt{\ol{X}(1 - \ol{X})} \le p \le \ol{X} + \frac{2.807}{\sqrt{n}} \sqrt{\ol{X}(1 - \ol{X})}
    \]
\end{example}