\subsection{Классификация вероятностных мер}

\subsubsection{Дискретные вероятностные меры}

\begin{definition}
	Пусть $P$ --- вероятностная мера на $(\R, \B(\R))$. Она называется \textit{дискретной}, если выполнено условие:
	\[
		\exists X \subseteq \R \text{ --- не более чем счётное множество} \such P(\R \bs X) = 0 \wedge \forall x \in X\ P(\{x\}) > 0
	\]
	При этом говорят, что \textit{вероятностная мера $P$ сосредоточена на $X$}.
\end{definition}

\begin{definition}
	Пусть вероятностная мера $P$ сосредоточена на $X = \{x_k\}_{k = 1}^\infty \subset \R$. Обозначим $p_k = P(\{x_k\})$, тогда \textit{набор $(p_1, p_2, \ldots)$ образует распределение вероятностей на $X$}.  
\end{definition}