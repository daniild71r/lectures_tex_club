\begin{lemma} (Свойства характеристических функций)
	\begin{enumerate}
		\item Если $\phi(t)$ --- характеристическая функция случайной величины, то \(\forall t \in \R\ \ |\phi(t)| \le \phi(0) = 1\)
		
		\item Если $\phi_\xi(t)$ --- характеристическая функция случайной величины $\xi$, а $\eta = a\xi + b$ для некоторых $a, b \in \R$, то верно равенство \(\phi_\eta(t) = e^{itb}\phi_\xi(at)\)
		
		\item Если $\phi$ --- характеристическая функция случайной величины $\xi$, то $\phi$ равномерно непрерывна на $\R$
	\end{enumerate}
\end{lemma}

\begin{proof}
	\begin{enumerate}
		\item В тупую оценим модуль:
		\[
			|\phi(t)| = |\E e^{it\xi}| \le \E|e^{it\xi}| = \E 1 = 1 = \phi(0)
		\]
		
		\item Просто пользуемся определением:
		\[
			\phi_\eta(t) = \E e^{it\eta} = \E e^{it(a\xi + b)} = \E e^{it(a\xi)} \cdot e^{itb} = e^{itb} \cdot \phi_\xi(at)
		\]
		
		\item Рассмотрим разность $\phi$ в точке $t$ при приращении $h$.
	\end{enumerate}
\end{proof}