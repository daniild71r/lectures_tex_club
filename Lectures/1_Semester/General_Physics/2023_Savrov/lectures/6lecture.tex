\section{Момент инерции}
Изучим частный случай движения, когда тело поворачивается вокруг неподвижной оси.
    Пусть оно двигается с угловой скоростью $\omega$, рассмотрим произвольную точку.
    Запишем для неё момент импульса
    \[\Vec{L_i} = [\Vec{r_i} * \Vec{p_i}]\]
    Рассмотрим проекцию на ось z.
    \[L_{iz} = r_{iz}p_i = m_ir_{iz}v_i = m_i \omega r_{iz}^2 = I_i * \omega\]
    Тогда для всего тела
    \[L_z = \sum_i L_{iz} = \omega \sum_i I_i = I\omega\]
    Тогда рассмотрим 2ЗН через L (оттуда найдём через I)
    \[\frac{dL}{dt} = M_z => \frac{d}{dt}(I\omega) = M_z\]
    Запишем кинетическую энергию через момент импульса и инерции
    \[K \frac{1}{2}\sum_im_i\Vec{v_i}^2 = \frac{1}{2}\sum_i m_i \omega^2 r_{iz}^2 = \frac{1}{2}I\omega^2 = \frac{L_z^2}{2I}\]
    Докажем несколько полезных теорем.
    \newline 1) Теорема Гюйгенса-Штейнера.
    Пусть есть ось вращения, проходящая через ц.м. тела и есть ось, параллельная ей.
    Запишем радиус вектор для параллельной оси
    \[r_i = a + r_c\]
    Тут $a$ - расстояние от оси, содержащей ц.м. до параллельной, $r_c$ - радиус-вектор относительно ц.м.
    Запишем момент инерции относительно паралелльной оси.
    \[I = \sum_i m_i \Vec{r_i} = \sum_i m_i \Vec{r_{ci}} + 2\Vec{a} \sum_i m_i \Vec{r_{ci} + a^2\sum_i m_i = I_c + Ma^2}\]
    Далее рассмотрим неравенство треугольника.
    Введём СК (x, y, z).
    Запишем момент инерции точки $m_i$, относительно трёх осей.
    \[I_z = \sum_i m_i (x_i^2 + y_i^2)\]
    \[I_y = \sum_i m_i (x_i^2 + z_i^2)\]
    \[I_x = \sum_i m_i (z_i^2 + y_i^2)\]
    Сложим
    Получим, что:
    \[I_x + I_y + I_z = 2\sum_i m_i (x_i^2 + y_i^2 + z_i^2) = 2\theta\]
    Заметим, что от поворота этих осей эта величина не зависит (то есть если 3 оси перпендикулярны, то ничего не меняется при повороте этих трёх осей (инвариантность относительно поворота)).
    Далее
    \[\theta >= I_z => I_x + I_y + I_z >= 2 I_z => I_x + I_y >= I_z\]
    Получили неравенство треугольника, посчитаем с его помощью момент инерции тонкой пластинки.
    ось вращения проходит через её центр.
    Пусть эта ось z:
    \[I_z = \sum_i m_i (x_i^2 + y_i^2) = \theta\]
    Тогда неравенство треугольник превращается в равенство.
    \[I_x + I_y = I_z\]
    Посчитаем момент инерции стержня, относительно конца.
    \[I_A = \int_0^l x^2dm = \frac{m}{l}\int_0^l x^2dx = \frac{1}{3}ml^2\]
    Тогда относительно Ц.М. через Т. Гюйгенса-Штейнера
    \[I_c = \frac{1}{12}ml^2\]
    Момент инерции диска, относительно центра.
    \[I_z = \int_0^R\rho^2dm = \frac{m}{\pi R^2} \int_0^R \rho^2 2 \pi \rho d\rho = \frac{1}{2}mR^2\]
    Тогда относительно диаметра:
    \[I_x = \frac{1}{2}I_z = \frac{1}{4}mR^2\]
    Толстый стержень, относительно конца:
    \[dI = \frac{1}{4}dmR^2 + dmx^2\]
    \[I = \int_0^ldI = \frac{1}{4}mR^2 + \frac{1}{3}ml^2\]
    Найдём для сферического тела, заметим, что (так как это сфера, то для любой точки верно):
    \[I_z = I_x = I_y = I => 3I = 2\theta\]
    \[I = \frac{2}{3}MR^2\]
    Найдём момент инерции для сплошного шара.
    \[I = \frac{2}{3}\int_0^Rr^2dm = \frac{2}{3}\frac{M}{\frac{4\pi}{3}R^3}\int_0^R r^2 4 \pi r^2 dr = \frac{2M}{R^4}\int_0^R r^4 dr = \frac{2}{5}MR^2\]
    Рассмотрим движение колеса по наклонной поверхности с трением.
    \[I_A \dot \omega = M_A => I_A \frac{a}{r} = mgrsin\alpha\]
    \[a = \frac{mgr^2}{I_A}sin\alpha = \frac{mgr^2sin\alpha}{I_C + mr^2} = \frac{gsin\alpha}{1 + \frac{I_C}{mr^2}}\]
    \[I_C \frac{a}{r} = F_{tr}r\]
    \[ma = mgsin\alpha - F_{tr}\]
    \[F_{tr} <= kN\]
    Лекция 7
    2 ЗН:
    \[\frac{\Vec{dp}}{dt} = \Vec{F}\]
    Изменение импульса:
    \[\frac{d\Vec{L}}{dt} = \Vec{M}|\]
    \[\Vec{\omega} = \frac{d\Vec{\phi}}{dt} => \Vec{v} = [\Vec{\omega} * \Vec{r}]\]
    Пусть есть радиус-вектор $\Vec{r_A}$ точки А, относительно O.
    Тогда скорость можно представить как
    \[\Vec{v_A} = \Vec{v_0} + [\Vec{\omega} * \Vec{r_A}]\]
    Если рассмотреть скорость точки относительно O':
    \[\Vec{v_A} = \Vec{v_0'} + [\Vec{\omega'} * \Vec{r_A'}\], тогда из этих двух равеств, учитывая, что:
    \[\Vec{V_0} = \Vec{V_0'} + [\Vec{\omega'} * \Vec{R}]\]
    Получаем, что:
    \[\omega = \omega'\]
    То есть векторы угловой скорости не меняется при параллельном переносе.
    Что будет если проинтегрировать, получим ли мы угол поворота?
    При плоском движении - да, при общем - нет.
    Повороты являются элементами некоммутативной алгебры.
    Запишем момент импульса, относительно неподвижной точки, тогда:
    \[\Vec{L} = m[\Vec{r} * \Vec{v}]\]
    Так как точка неподвижная, то можем воспользоваться тем, что:
    \[\Vec{L} = m[\Vec{r} * [\Vec{\omega} * \Vec{r}] = m\Vec{\omega} (\Vec{r} * \Vec{r}) - m\Vec{r}(\Vec{r} * \Vec{\omega} = mr^2\Vec{\omega} - m\Vec{r}(\Vec{r} * \Vec{w}) => \Vec{L} = I * \Vec{\omega}\]
    Распишем на ось X:
    \[L_x = m(x^2 + y^2 + z^2)\omega_x - mx(x\omega_x + y\omega_y + z\omega_z)\]
    \[L_x = m[(y^2 + z^2)\omega_x - xy\omega_y - xz\omega_z]\]
    Для других осей, очевидно, аналогично.
    \[L_i = \sum_k I_{ik}\omega_k\]
    Кинетическая энергия вращаюшегося тела:
    \[K = \frac{1}{2}\int \Vec{v}^2dm = \frac{1}{2}\int [\Vec{\omega} * \Vec{r}] d\Vec{p}\]
    Получили смешанное произведение, его можно перезаписать:
    \[K = \frac{1}{2}\int [\Vec{r} * d\Vec{p}]\Vec{\omega} = \frac{1}{2}\Vec{L}\Vec{\omega} = \frac{1}{2}\sum_i L_i\omega_i = \frac{1}{2}\sum_{ik}I_{ik}\omega_i \omega_k\]
    В осях x, y, z запись упрощается:
    \[K = \frac{1}{2}I_x\omega_x^2 + \frac{1}{2}I_y\omega_y^2 + \frac{1}{2}I_z\omega_z^2 = \frac{L_x^2}{2I_x} + \frac{L_y^2}{2I_y} + \frac{L_z^2}{2I_z}\]
    При условии:
    \[L^2 = L_x^2 + L_y^2 + L_z^2\]
    Если нарисовать в СК (Lx, Ly, Lz), то первое уравнение задаёт эллипсоид, а второе - сферу.
    \[I_x < I_y < I_z\]
    Домножением уравнения на $2I_x, 2I_z$
    \[2KI_x - L^2 = \frac{I_x - I_y}{I_y}L_y^2 + \frac{I_x - I_z}{I_z}L_z^2 < 0\]
    \[2KI_z - L^2 = \frac{I_z - I_x}{I_x}L_x^2 + \frac{I_z - I_y}{I_y}L_y^2 > 0\]
    Из этого следует:
    \[2KI_x < L^2 < 2KI_z\]
    То есть сфера пересечёт эллипсоид.
    Можно записать следующее тождество:
    \[L_x^2\frac{L^2 - 2KI_x}{2KI_x} + L_y^2\frac{L^2 - 2KI_y}{2KI_y} + L_z^2\frac{L^2 - 2KI_z}{2KI_z} = 0\]
    \[L^2 = L_x^2\]