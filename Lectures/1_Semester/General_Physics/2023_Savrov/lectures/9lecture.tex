\section{Динамика в СТО}
2 закон Ньютона не является ковариантным, то есть неверным в ЛСО.
Запишем импульс по новому:
\[\Vec{p} = m\Vec{v}f(\Vec{v}/c)\]
Эта $f(\Vec{v}/c)$ - это поправка, она должна быть такой, что $f(0) = 1$, то есть при скоростях сильно меньше скорости света - верен обычный импульс.
Рассмотрим следующий мысленный эксперимент: пусть у нас есть 2 шарика - A и B. Тогда шарик A движется горизонтально вниз со скростью $v$, у шарика Б 2 составляющие скорости - горизонтальная вправо $u$, вертикальная вверх $v_1$, тогда пусть после взаимодействия, все компоненты скоростей поменяют направление, но не модуль (упругуе взаимодействие).
Тогда запишем сохранение импульса:
\[vf(\frac{v}{c}) = v_1f(\frac{1}{c}\sqrt{u^2+v_1^2})\]
Перейдём в систему координат, которая движется со скоростью $u$ горзизонтально вправо относительно ЛСО.
Тогда запишем скорости в этой системе координат:
\[v_x' = \frac{v_x - u}{1 - \frac{v_xu}{c^2}}\]
\[v_y' = \frac{v_y}{\gamma(1 - \frac{v_xu}{c^2})}\]
Тогда у нас у шарика А - горизонтальная влево $u$, вертикальная вниз $\frac{v}{\gamma}$
У B только вертикальная вверх $\frac{v_1}{\gamma(1 - \frac{u^2}{c^2}}$
После взаимодействия вертикальные компоненты поменяют направление, тогда запишем уже ЗСИ в этой штрихованной системе координат:
\[\frac{v}{\gamma}f(\frac{1}{c}\sqrt{u^2 + \frac{v^2}{\gamma^2}} = v_1\gamma f(\frac{v_1\gamma}{c}\]
Заметим, что подстановка $v = v_1\gamma$ превращает второе уравнение в первое, подставим и сократим:
\[f(\frac{v}{c}) = \frac{1}{\gamma} f (\frac{1}{c}\sqrt{u^2 + \frac{v^2}{\gamma^2}})\]
\[1 = \frac{1}{\gamma} f (\frac{u}{c}) => f(\frac{u}{c}) = \gamma = \frac{1}{\sqrt{1 - \frac{u^2}{c^2}}}\]
Проверим, что при данной поправке f у нас справедливы уравнения:
\[\frac{1}{\sqrt{1 - \frac{u^2}{c^2}}} * \frac{1}{\sqrt{1 - \frac{v^2}{c^2}}} = \frac{1}{\sqrt{1 - \frac{u^2 + v^2(1 - \frac{u^2}{c^2})}{c^2}}}\]
Верно, то есть мы получили, что при данной поправке у нас справедлив ЗСИ при скоростях, близких к скорости света. Итого получаем:
\[\Vec{p} = \gamma m \Vec{v} = \frac{m\Vec{v}}{\sqrt{1 - \frac{v^2}{c^2}}}\]
Изучим теперь вопрос энергии:
\[K_2 - K_1 = \int_1^2 \Vec{F}d\Vec{r} = \int_1^2 \frac{d\Vec{p}}{dt}\Vec{v}dt = m\int_1^2 \Vec{v}d\Vec{p} = m \int_1^2 \Vec{v}d(\gamma \Vec{v})\]
Сделаем некоторые замечания, чтоб этот интеграл было возможно взять:
\[\Vec{v}d\Vec{v} = vdv\]
\[d\gamma = d(1 - \frac{v^2}{c^2})^{-\frac{1}{2}} = \gamma^3 \frac{vdv}{c^2}\]
Далее, упрощаем интеграл:
\[\Vec{v}d(\gamma * \Vec{v}) = \Vec{v}(d\gamma \Vec{v} + \gamma d\Vec{v}) = v^2d\gamma + \gamma v dv = v^2d\gamma + \frac{c^2}{\gamma^2} d\gamma = (v^2 + c^2 - v^2)d\gamma = c^2d\gamma\]
Тогда:
\[K_2 - K_1 = mc^2 \int_1^2 d\gamma = mc^2(\gamma_2 - \gamma_1)\]
Тогда получаем, что:
\[K = mc^2\gamma + const = \frac{mc^2}{\sqrt{1 - \frac{v^2}{c^2}}} + const = mc^2 + \frac{mv^2}{2} + o(\frac{v^2}{c^2}) + const\]
\[K = \frac{mc^2}{\sqrt{1 - \frac{v^2}{c^2}}}\]
\[E = K + mc^2 => mc^2\]
Далее Эйнштейн понял, что энергия переход в массу, а массу в энергию. Если у вас есть энергия, то у вас есть инерция.
Когда происходит химическая реакция, то выделяется или поглощается энергия, тогда масса либо уменьшается, либо увеличивается, но мы не можем эту заметить, так как очень малое изменение.
Итого получили следующие соотношения:
\[E = \gamma mc^2, \ \Vec{p} = \gamma m\Vec{v}\]
Запишем следующее соотношение:
\[E^2 - \Vec{p}^2c^2 = (mc^2)^2\gamma^2 - (mc)^2\gamma^2v^2 = (mc^2)^2\]
То есть эта величина - инвариант.
Пусть у нас имеется мишень - водород. В него летит протон.
Может образоваться:
\[p + p => p + p + p + \Tilde{p}\]
Запишем закон сохранения энергии в ЛСО:
\[E + mc^2 = 4mc^2 + K\]
У этих 4 частиц будет кинетическая энергия, так как изначально у летящего протона был какой-то импульс.
В системе Ц.М.:
\[(E + mc^2)^2 - \Vec{p}^2c^2 = (4mc^2)^2\]
\[E = 7mc^2 => K_p = 6mc^2\]
\[S^2 = c^2(t_2 - t_1)^2 - (\Vec{x_2} - \Vec{x_1})^2 = inv\]
Рассмотрим теорию межзвёздных перелётов, тогда запишим уравнение Мещерского, но учтём релятивистские эффекты.
\[\frac{m}{m_0} = e^{-\frac{v}{u}}\]
\[\frac{m}{m_0} = (\frac{1 - \frac{v}{c}}{1 + \frac{v}{c}})^{\frac{c}{2u}}\]
В пределе, мы стремимся к обычному уравнению Мещерского.
Получаем, что чтобы получить хотя бы половины скорости света, должно быть, чтоб:
\[\frac{c}{2u} = 10^5\]
Что очень много, если мы хотим именно разгонять корабль с помощью химической реакции.
Рассмотрим фотонный звездолёт:
\[d(\gamma mc^2) = - dE = - cdp_{f} = - cd(\gamma mv)\]
\[d(\gamma mc + \gamma mv) = 0\]
То есть:
\[m\gamma(1 + \frac{v}{c}) = const = m_0\]
\[\frac{m}{m_0} = \sqrt{\frac{1 - \frac{v}{c}}{1 + \frac{v}{c}}}\]
Итого, если положить, что $u = c$, то уже требуется реальная достижимая масса.
В чём проблема? Мы хотим получить чистую энергию, а это возможно сделать с помощью антиматерии. 
\[\frac{d}{dt}(\gamma v) = a\]
Связь времени для космонавтов:
\[t = \frac{c}{a}sh\frac{a\tau}{c}\]
\[x = \frac{c^2}{a}(ch \frac{a\tau}{c} - 1)\]
Но опять же проблема в том, что $\frac{m}{m_0} = e^{-\frac{a\tau}{c}}$
Тело, брошенное в гравитационном поле движется по минимально возможной траектории в неевклидовой геометрии.