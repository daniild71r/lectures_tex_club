\section{Гравитация и законы Кеплера}
Первый закон Кеплера - планеты вращаются вокруг солнца по эллипсу, в одном из фокусов находится Солнце.
\newline Закон всемирного тяготения.
\[\Vec{F} = G\frac{Mm}{|\Vec{r}|^2}\Vec{r}\]
Мы интересуемся движением планет вокруг Солнца, поэтому
\[\frac{m}{M} \rightarrow 0, \mu \rightarrow m\]
(масса Солнца много больше массы планет)
\newline
В силу этого можно сказать, что
\[\Vec{\dot L} = [\Vec{r} x \Vec{F}] = \Vec{0}\]
Рассмотрим движение:
\[v_r = \frac{\Delta r}{\Delta t} = \dot \Vec{r}\]
\[v_{\phi} = r \frac{\Delta \phi}{\Delta t} = r \dot \phi\]
Из этого получаем, что
\[\Vec{v}^2 = v_r^2 + v_{\phi}^2 = \dot r^2 + r^2 * \dot \phi^2\]
Учитывая, что $v_r$ - не даёт вклада в изменение момента импульса, так как векторное произведение равно 0.
Распишем L
\[L = mv_{\phi}r = mr^2\dot \phi = const\]
Это второй закон Кеплера, то есть секториальная скорость одинакова.
\newline Исследуем траекторию движения тел:
\[\int_1^2 \Vec{F}d\Vec{r} = -(u_2 - u_1) => u = -\frac{GMm}{r}\]
\[E = \frac{m\Vec{v}^2}{2} - G\frac{Mm}{r} = \frac{m\dot \Vec{r}^2}{2} + \frac{mr^2\dot \phi^2}{2} - G\frac{Mm}{r}\]
\[= \frac{m\dot \Vec{r}^2}{2} + \frac{L^2}{2mr^2} - G\frac{Mm}{r}\]
Отсюда, если последнюю разность обозначить $u_{eff}$, то заметим, что если она больше 0, то у нас движение неограниченное (инфинитное), иначе финитное (ограничное).
Запишем уравнения движения и решим их.
\begin{equation*}
    \frac{d\phi}{dt} = \frac{L}{mr^2} \\
    \\
    \frac{dr}{dt} = \sqrt{\frac{2}{m}(E - u_{eff})}
\end{equation*}
После преобразований и интегрирования получаем, что:
\[\frac{p}{r} = 1 +ecos\phi\]
\[p = \frac{L^2}{GMm^2}, e = \sqrt{1 + \frac{2EL^2}{(GM)^2m^3}}\]
Где e - эксцентриситет.
\newline Получили уравнения конических сечений (параболы, эллипсы и гиперболы).
Эксцентриситет определяет само сечение.
e = 0 - окружность
e < 1 - эллипс
e = 1 - парабола
e > 1 - гипербола
При движении по эллипсу эксцентриситет равен:
\[e = \sqrt{1 - \frac{b^2}{a^2}}\]
Выведем третий закон Кеплера (рассматриваем движение по эллипсу, запишем уравнение движения).
\[\frac{L}{2m}dt = \frac{1}{2}r^2d\phi\]
Проинтегрируем:
\[\frac{L}{2m}T = \pi ab\]
Из этого получаем:
\[T = 2\pi \frac{a^{\frac{3}{2}}}{\sqrt{GM}} => \frac{T^2}{a^3} = const = \frac{4\pi^2}{GM}\]
Теорема Гаусса:
\[\Vec{g} = \frac{\Vec{F_{\tau}}}{m}\]
\[\Vec{F_t = \sum_i \Vec{f_i} = - Gm \sum_i \frac{m_i\Vec{r_i}}{|\Vec{r_i}|}}\]
$f_i$ - сила тяжести, действующая на отедельно взятые точки (то есть мы как бы разбили силу тяжести на сумму множества).
Рассмотрим точку массой M и окружим произвольно взятой поверхностью.
Тогда рассмотрим малый телесный угол $d\Omega$, который как бы окружает это тело.
Пусть есть $\Vec{g}$ - который направлен от кусочка поверхности угла $d\Omega$ к точке массой M, $d\Vec{S}$ - вектор нормали к кусочку.
Запишем скалярное произведение, угол $\theta$ между нормалью и g.
\[(\Vec{g}, d\Vec{S}) = gdScos\theta\ = -gdScos(\pi - \theta)= -gdS_{perp} = - \frac{GM}{r^2}r^2d\Omega = -GMd\Omega\]
\[\oint \Vec{g}d\Vec{S} = 4\pi GM\]
Это поток вектора g. И есть формула теоремы Гаусса.
\[\Vec{g_1}d\Vec{S_1} + \Vec{g_2}d\Vec{S_2} = \Vec{0}\]
Найдём g в какой-то точке. Пусть g находится на расстоянии r от поверхности Земли, тогда у нас g находится на какой-то сфере радиуса r, можем вычислить замкнутый интеграл в т.Гаусса, получим
\[g * 4\pi r^2 = -4\pi GM => g = -\frac{GM}{r^2}\]
Дана сфера, с массой в центре m, есть точка M вне этой сферы.
\[\rho^2 = R^2 + r^2 - 2Rrcos\theta\]
\[dm = \frac{m}{4\pi r^2}2\pi r sin\theta rd\theta = \frac{m}{2}sin\theta d\theta\]
\[du = - \frac{Gmdm}{\rho} = - \frac{GMm}{2Rr}d\rho\]
\[u = \int_{\rho_{min}}^{\rho_{max}}du = -\frac{GMm}{2Rr}2r = - \frac{GMm}{R}\]
Для двух сферических тел, с массами в центре $m_1, m_2$, где плостность распределена как-то $\rho(r)$, получаем, что:
\[F = -\frac{Gm_1m_2}{R^2}\]