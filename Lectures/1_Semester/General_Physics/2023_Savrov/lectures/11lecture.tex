\section{Неинерциальные системы отсчёта}
Запишем уравнение вынужденных колебаний:
\[m\dot \dot x = -kx - \beta \dot x + fcos\omega t\]
\[2\gamma = \frac{\beta}{m}\]
\[\omega_0^2 = \frac{k}{m}\]
Разделим уравнение вынужденных колебаний на массу и решим это уравнение в двух случаях:
\[\omega = 0 => x = a_0 = \frac{f}{m\omega^2}\]
\[\omega \ne 0 => x = Acos(\omega t + \delta)\]
\[(-\omega^2 + \omega_0^2)Acos(\omega t +\delta) - 2\gamma \omega A sin(\omega t + \delta) = \frac{f}{m}cos\omega t\]
\[(-\omega^2 + \omega_0^2)A(cos\omega tcos\delta - sin \omega t sin \delta) - 2\gamma \omega A(sin \omega t cos \delta + cos \omega t sin \delta) = \frac{f}{m}cos\omega t\]
Что стоит справа при косинусе:
\[cos \omega t : (-\omega^2 + \omega_0^2)Acos\delta - 2\gamma \omega A sin\delta = \frac{f}{m}\]
\[sin \omega t: -(-\omega^2 + \omega_0^2)Asin\delta - 2\gamma \omega A cos\delta = 0\]
A != 0, иначе случай тривиален:
из второго уравнения, делением на $cos \delta$
\[tg \delta = \frac{2\gamma \omega}{\omega^2 - \omega_0^2}\]
Как выглядят наши уравнения:
\[acos\delta - bsin\delta = c\]
\[asin\delta + bcos\delta = 0\]
Отсюда, если возвести оба уравнения в квадрат и сложив, мы получим:
\[a^2 + b^2 = c^2\]
Тогда применяя это к нашим уравнениям:
\[A = \frac{\frac{f}{m}}{\sqrt{(\omega^2 - \omega_0^2)^2 + (2\gamma \omega)^2}}\]
Получили амплитуду вынужденных колебаний, выведем, как она зависит от вынуждающей силы (то есть найдём амплитудно-частотную характеристику).
Когда частота равна = 0 (то есть нет вынуждающей силы), то амплитуда равна a0, тогда сначала наша функция будет возрастать до какой-то частоты $\omega_m$, а потом убывать.
Как можно найти это значение? Просто приравняв производную нулю:
\[\frac{dA}{d\omega} = 0 => \omega_m = \sqrt{\omega_0^2 - 2\gamma^2}\]
Будем исследовать наиболее существенный случай (то есть когда $\gamma << \omega_0$ (когда затухание много меньше чем собственная частота)
То есть в целом мы можем считать, что максимум, достигается на собственной частоте.
Чему же равна максимальная амплитуда?
\[\omega_m = \omega_0\]
\[a_m = \frac{f}{2m\gamma\omega_0} = \frac{a_0\omega_0}{2\gamma} = a_0\frac{\pi}{\gamma T_0}\]
При этом $\gamma T_0$ - логарифмический декремент затухания, тогда это можно записать:
\[a_m = a_0\frac{\pi}{d}\]
При этом величина $\frac{\pi}{d}$ - это добротность, тогда отсюда:
\[a_m = a_0Q\]
Какой ещё смысл можно придать добротности? Найдём частоты, при которых амплитуда меньше максимальной в $\sqrt{2}$ раз. Почему так? Потому что энергия колебаний пропорциональная квадрату отклонения, подставим это значение в формулу для амплитуды и учтём, что затухания много меньше $\omega_0$, тогда получаем:
\[\omega^2 = \omega_0^2 +- 2\gamma\omega_0 = (\omega_0 +- \gamma)^2\]
Тогда найдём разность этих частот (при которых реализуется частота $\frac{a_m}{\sqrt{2}}$)
\[\Delta\omega = \omega_2 - \omega_1 = 2\gamma = \frac{a_0\omega_0}{a_m} = \frac{\omega_0}{Q}\]
Вот мы выразили $tg \delta$ чуть раньше, построим фазово-частотную характеристику (зависимость $\delta$ от частоты).
$\delta$ меняется от $\frac{\pi}{2}$ до $-\frac{\pi}{2}$
Проинтегрируем уравнение вынужденных колебаний:
\[\dot x = \frac{f}{2m\gamma}cos\omega t\]
\[x = \frac{f}{2m\gamma\omega} sin \omega t\]
Внешняя сила работает против силы трения в резонансе.
Обсудим волновое уравнение для волны, распространяющейся в одном измерении:
s(x, t) - координата и время:
\[\frac{\delta^2S}{\delta t^2} - c^2\frac{\delta^2S}{\delta x^2} = 0\]
\[s(xt) = f(x - ct) + g(x + ct)\]
Вторая функция описывает распространение волны справа налево, первая слева направо.
Анализ столкновения двух волн:
1) Если жёстко закрепрелена струна (в этом случае волна переворчивается)
\[s(xt) = f(x - ct) - f(-x-ct) = 0\]
Во втором случае - фаза не меняется при отражении:
\[\frac{\delta S}{\delta x} = 0\]
\[\frac{\delta S}{\delta x} = f'(x - ct) - f(-x-ct) = 0\]
Рассмотрим движение волны:
\[s(xt) = Acos(\omega t - kx + \delta)\]
\[x = \frac{\omega}{k}t => \frac{\omega}{k} = c, / \omega = \frac{2\pi}{T}, / k = \frac{2\pi}{\lambda}\]
Есть 2 скорости - фазовая и групповая.
k - волновое число, $\lambda$ - длина волны.
Стоячая волна - это суперпозиция двух волн, бегущих навстречу:
\[s(xt) = Acos(\omega t - kx) + Acos(\omega t + kx) = 2Acoskxcos\omega t\]