\subsection{Формула включений и исключений.}

\begin{definition}
    \textit{Характеристической функцией} множества $A$ объемлющего множества $U$ называется функция $\chi_{A}: U \rightarrow \{0, 1\}.$ Причем $\chi_{A}(x) = 1 \lra x \in A.$
\end{definition}

\begin{theorem}
    Пусть даны множества $A_1, A_2, \ldots, A_n \subseteq U.$
    Тогда $$|\overline{A_1} \cap \overline{A_2} \ldots \cap \overline{A_n}| =  |U| - |A_1|- \ldots - |A_n| + |A_1 \cap A_2| + \ldots + (-1)^n |A_1 \cap A_2 \ldots A_n|.$$
\end{theorem}

\begin{proof}
Докажем утверждение с помощью индикаторной функции. Понятно, что:
$$\chi_{\overline{A}} = 1 - \chi_A$$
$$\chi_{A\cap B} = \chi_A \cdot \chi_B$$
$$|A| = \sum_{i \in U} \chi_A(i)$$
$$\chi_{|\overline{A_1} \cap \overline{A_2} \ldots \cap \overline{A_n}|} = \chi_{\overline{A_1}} \cdot \ldots  \cdot \chi_{\overline{A_n}} = (1 - \chi_{A_1}) \cdot (1 - \chi_{A_2}) \ldots (1 - \chi_{A_n}) = $$
$$= 1 - \chi_{A_1} - \chi_{A_2} - \ldots + (-1)^n \chi_{A_1 \cap A_2 \ldots A_{n}}$$
Тогда просуммируем эту функцию по всем элементам из $U,$ отсюда получим требуемое.
\end{proof}

\begin{example}
    Найти количество счастливых билетов, то есть билетов, у которых сумма первых трех цифр равна сумме последних трех.
\end{example}

\begin{solution}
    Пусть у нас есть билет $\overline{abcdef}.$ Тогда сопоставим ему число $\overline{abc(9-d)(9-e)(9-f)}.$ Тогда его сумма цифр равна $27.$ То есть задача сводится к поиску решений системы:
    $$\left\{\begin{array}{l}
    a + b + c + \overline{d} + \overline{e} + \overline{f} = 27\\
    0\leq a, b, c, \overline{d}, \overline{e}, \overline{f} \leq 9.
    \end{array}\right.$$
    Решений в неотрицательных целых числах $x_1 + x_2 + x_3 + x_4 + x_5 + x_6 = 27$ всего $\binom{32}{5}.$ Посчитаем число решений таких, что у нас ровно какая-то переменная хотя бы $10.$ Тогда уменьшим эту переменную на $10$ получим другое уравнение в неотрицательных целых числах:
    $$y_1 + y_2 + y_3 + y_4 + y_5 + y_6 = 17.$$ Если ровно две переменные хотя бы $10,$ то аналогично уравнение сводится к следующему в целых неотрицательных числах:
    $$y_1 + y_2 + y_3 + y_4 + y_5 + y_6 = 7.$$
    Больше двух переменных, которые хотя бы $10$ быть не может, так как $27 < 30.$ Тогда по формуле включений и исключений получим ответ.
\end{solution}

\begin{example}
    Пусть есть множества $A, B; |A| = n, |B| = k.$ Посчитаем количество функций $f: A \rightarrow B,$ таких что $f$ -- отображение, инъекция, сюръекция, биекция
\end{example}
\begin{solution}
    \underline{Отображения.} Для каждого элемента $A$ есть ровно $k$ вариантов. Тогда по правилу произведения $k^n$ отображений.
    \\
    \underline{Частичных функций.} Тогда аналогично получим $(k + 1) ^ n.$
    \\
    \underline{Биекций.} Значит, $k = n.$ Тогда по правилу произведения будет ровно $n!$ биекций.
    \\
    \underline{Инъекций.} Значит, $n \leq k.$ Пронумеруем элементы каждого из множеств. Тогда таких функций ровно $A_{k}^{n}.$
    \\
    \underline{Сюръекций.} Пронумеруем элементы каждого из множеств. Пусть множество $D_i = \{ f: A\rightarrow B |f^{-1}(i) = \varnothing\}.$ Тогда искомое число функций равно $|\overline{D_1} \cap \overline{D_2} \ldots \cap \overline{D_n}|.$ А это вычислим по формуле включений и исключений.
\end{solution}