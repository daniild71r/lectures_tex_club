% 27.09.2023

\section{Теория графов}

Будем считать, что $ \binom{V}{2} = \{ \{a, b\} \ | \ a, b \in V, \ a \neq b \}$

\begin{definition}
    \textif{Обыкновенный (простой) граф} -- это упорядоченная пара: $(V, E),$ где $V $ -- множество вершин, $E$ -- множество рёбер, причем $E \subseteq  \binom{V}{2}$ 
\end{definition}

Если не оговорено противного, то под графом мы понимаем простой неориентированный граф. Простой означает, что в графе нет петель и кратных рёбер, а неориентированный, что рёбра графа не имеют направлений. В случае если заменить в графе линии на стрелки, т. е. ребро — упорядоченная пара вершин, то получится ориентированный граф. Заметим, что множество вершин графа может вообще говоря быть бесконечным или пустым, но если не оговорено противного, то мы считаем, что множество вершин конечно и непусто.

\begin{definition}
Зафиксируем граф $G(V, E).$ Вершины $u$ и $v$ называются смежными или соседями, если они образуют ребро: ${u, v} \in E.$ Рёбра $e$ и $f$ называются \textit{смежными,}
если они имеют общую вершину: $e \ \cap \ f ̸ \varnothing.$ Вершина $u$ инцидента ребру $e,$ если
$u \in e.$ Вершины $u$ и $v,$ \textit{инцидентные ребру} $e,$ называются его концами; говорят,
что $e$ \textit{соединяет} $u$ и $v.$ Рёбра часто записывают сокращённо: $uv$ вместо $\{u, v \}.$ 
\end{definition}

\begin{definition}
    Зафиксируем два графа $G = (V, E), \ H = (W, F):$
    \begin{enumerate}
        \item $G \cup H = (V \cup W, E \cup F)$ \\
        \item $G \cap H = (V \cap W, E \cap F)$ \\
        \item $\overline{G} = (V,  \binom{V}{2} \backslash E)$
    \end{enumerate}
\end{definition}

\begin{definition}
    \textit{Граф-пусть из $n$ ребер} $(n \geq 0)$ -- это граф, который состоит из вершин ${v_0, v_1, \dots , v_n}$ и рёбер ${v_i, v_{i+1}}.$

    \textit{Граф-цикл из $n$ ребер} $(n \geq 3)$ -- это граф, который состоит из вершин ${v_0, v_1, \dots , v_n}$ и рёбер $\{v_i, v_{i+1}\},$ а также из ребра $\{v_n, v_1 \}$

    \textit{Полный граф на $n$ вершинах} -- это граф, у которого $E = \binom{V}{2}$

  \textit{Подграфом графа $G = (V, E)$} называется граф $G',$ если $V' \in V, \ E' \in E.$ Причем, если $V' \in V, \ E' = E \cap  \binom{V}{2},$ то он называется \textbf{индуцированным.}
\end{definition}

\begin{definition}
    Назовем графы $G$ и $H$ \textit{изоморфными} и будем записать так: $G = (V, E) \cong H = (W, F),$
    если существует биекция $f: V \rightarrow W,$ для которой $\{u, v \} \Longleftrightarrow \{ f(u), f(v) \}$
\end{definition}

\begin{definition}
    Путь длины $n$ в графе $G$ это подграф, который изоморфен $P_{n}.$
\end{definition}

\begin{definition}
    \textit{Степень вершины $d(v) -$} количество ребер с концом в $v.$
\end{definition}

\begin{lemma}
    Сумма степеней всех вершин графа равна удвоенному числ его ребер.
\end{lemma}
\begin{proof}
    Посчитаем $|\{(v, e) \ | \ e \text{ инцидентно } v \} | = 2 \cdot |E|.$
\end{proof}

\begin{definition}
    \textif{Кликой на $n$ вершинах} называется подграф, изоморфный полному графу на $n$ вершинах.

    \textif{Анти-клика} или независимое множество на $n$ вершинах -- это индуцированный подграф, не имеющий ребер.
\end{definition}

\begin{definition}
    Граф $G$ называется \textif{связным,} если $\forall \ u, \ v \in V $ существует путь между $u$ и $v.$ 
\end{definition}

\begin{definition}
    \textif{Компонентой связности} называется максимальный по включению связный подграф, то есть не существует связный подграф $H: G' \subset H \subseteq G.$
\end{definition}

\begin{lemma}
    Пусть $G(V, E)$ -- связный граф и ребро $e$ лежит на цикле.  Тогда граф $G' = (V, E \backslash {e})$ связный.
\end{lemma}
\begin{proof}
Перед доказательством введём вспомогательные обозначения. Пусть $P$ и $Q$ -- пути в графе $G, x, y$ — вершины, лежащие на пути $P,$ а $y$ и $z$ -- вершины, лежащие на пути $Q.$ Обозначим через $xPy$ -- подпуть пути $P,$ начинающийся с вершины $x$ и заканчивающийся в вершине $y;$ считаем, что вершины $p_0, p_1, \dots , p_n$ пути $P$ упорядочены так, что $x = p_i, y = p_j, i < j$ и соответственно $xPy = p_iPp_j$ -- путь на вершинах $p_i, \dots , p_j.$ Если в результате объединения путей $xPy$ и $yQz$ получится путь, то мы обозначаем этот путь через $xPyQz.$ Это обозначение переносится и на объединение нескольких путей, а если пути $P$ и $Q$ имеют единственную общую вершину -- общий конец, то путь, получившийся их объединением обозначим через $P Q.$

Пусть ребро $e$ лежит в подграфе-цикле $C.$ Обозначим через 
$Q \subseteq C$ подграф-путь, получающийся из $C$ удалением ребра $e$ (с сохранением его концов). Зафиксируем все пути между всеми парами вершин перед удалением $e$ и рассмотрим путь $P$ с началом в вершине $w$ и концом в вершине $z.$ Если ребро $e$ не лежит на пути $P,$ то после его удаления этот путь не пострадает. Если же $e$ лежит на пути, то превратим этот путь в другой путь с помощью пути $Q.$ Упорядочим вершины $P;$ пусть вершина $x$ -- первая общая вершина путей $P$ и $Q$ (ближайшая к $w,$ возможно сама $w),$ а $y$ -- последняя общая вершина путей $P$ и $Q$ (ближайшая к $z,$ быть может сама $z).$ Вершины $x$ и $y$ определены, потому что пути $P$ и $Q$ имеют хотя бы две общие вершины -- концы ребра $e.$ Докажем, что $wPxQyPz$ -- путь, соединяющий вершины $w$ и $z,$ и не проходящий через ребро $e.$ Действительно, пути $wPx$ и $xQy$ не имеют общих вершин, кроме $x,$ поскольку иначе в пути $P$ нашлась бы вершина ближе к $w,$ чем $x,$ которая была бы общая с путём $Q,$ что противоречит выбору $x;$ симметрично пути $xQy$ и $yPz$ не имеют общих вершин, кроме $y$ (иначе нашлась бы общая вершина ближе к $z,$ чем $y);$ пути $wPx$ и $yPz$ не имеют общих вершин, поскольку это непересекающиеся подпути пути $P.$

Итак, мы доказали, что после удаления ребра $e$ в графе по-прежнему останутся пути между всеми парами вершин, т. е. граф останется связным.
\end{proof}
