%09.03.23

\section{Отношения.}

\begin{definition}
    \textit{Отношение} -- произвольное подмножество декартово произведения непустых множество $A, B.$ То есть это  множество $R \subseteq A \times B.$ При этом $$(a, b) \in R \Longleftrightarrow a R b.$$
    Также 
    $$\forall a \in A: R(a) = \{ b \in B| (a, b) \in R\}.$$
\end{definition}

\begin{definition}
    Отношение $R$ \textit{функционально}, если $R$ -- функция, то есть $$\forall a \in A \ |R(a)| \leq 1.$$ 
\end{definition}

\begin{definition}
    Отношение $R$ \textit{(лево) тотальное}, если $R$ -- функция, то есть $$\forall a \in A \ |R(a)| \geq 1.$$ 
    Отношение $R$ \textit{право-тотальное}, если $R$ -- функция, то есть $$\forall b \in B \  \exists a \in A: (a, b) \in R.$$ 
    Отношение $R$ \textit{инъективное}, если при $a' \neq a \ \  R(a) \cap R(a') = \varnothing.$ 
\end{definition}

\subsection{Бинарное отношение.}

\begin{definition}
    Подмножеством декартового произведения $A \times A$ называется \textit{бинарным отношением.}
\end{definition}

У такого отношения есть могут быть следующие свойства:
\begin{enumerate}
    \item Рефлексивность: $\forall a \in A \ a R a$
    \item Симметричность: если $a R b,$ то $b R a$
    \item Транзитивность: если $a R b$ и $b R c,$  то $a R c.$
\end{enumerate}

\begin{definition}
    Если бинарное отношение обладает этими свойствами, то его называют \textit{отношением эквивалентности.}
\end{definition}

\begin{example}
    Равенство треугольников -- является отношением эквивалентности.
\end{example}

\begin{example}
    Компонент связности в графе. Пусть $v, w \in V$ эквивалентны, если $\exists$ путь из $v$ в $w.$
\end{example}


\subsection{Класс эквивалентности.}

\begin{definition}
    Пусть задано множество $A,$  $R-$ отношение эквивалентности на $A.$ Тогда 
    $$R_{a} = [a]_{R} = [a] = \{ b \in A | \  a R b\}.$$
\end{definition}

\begin{proposition}
    Если $b, c \in [a],$ то $bRc.$
\end{proposition}

\begin{proof}
$b, c \in [a] \Longrightarrow aRb, aRc \Longrightarrow bRc.$
\end{proof}

\begin{proposition}
    $a, b \in A \Longrightarrow [a] \cup [b] = \varnothing \vee [a] = [b].$
\end{proposition}

\begin{proof}
    Пусть пересечение не пусто, есть общий элемент $x.$ Тогда 
    $aRx, \ bRx \Longrightarrow aRx, \ xRb \Longrightarrow a R b \Longrightarrow [b] \subseteq [a].$
    Аналогично доказывается другое дополнение.
\end{proof}

\begin{definition}
    \textit{Разбиение} множества $A$ -- это набор непересекающихся подмножеств $f \subseteq 2^A$ такой, что 
    $$A = \underset{S \in f}{\cup} S.$$
\end{definition}

\begin{theorem}
    Пусть $A, R-$ отношение эквивалентности. Тогда $A$ разбивается на классы эквивалентности по отношению $R.$
\end{theorem}

\begin{proof}
    Это вытекает из утверждения $6.1.$
\end{proof}

\begin{proposition}
    Есть биекция между отношением эквивалентности на $A$ и разбиением $A$ на подмножества.
\end{proposition}

\subsection{Операции с отношениями.}

Можно брать объединение отношений, пересечение, дополнение.

\begin{definition}
    $$R^{-1} \subseteq B \times A, (b, a) \in R^{-1} \Longleftrightarrow (a, b) \in R.$$
\end{definition}