% TODO format
\subsection{Монотонные последовательности}
\begin{definition}
    Последовательность $ \{a_n\} $ называется не строго возрастающей(строго возрастающей), если \begin{equation}
        \forall n \in \N \ (a_n \underset{ <}{ \leq} a_{n + 1})
    \end{equation}
    Последовательность $ \{a_n\} $ называется не строго убывающей (строго убывающей), если $ \{- a_n\} $ не строго возрастающая(строго возрастающая).
\end{definition}
\begin{note}
    Если $ \forall  n \in \N\ (a_n \leq  a_{n + 1}) $, то по индуции доказывается, что $ \forall  n, m \in \N\ (n < m \implies  a_n \leq  a_m) $.
\end{note}
\begin{theorem}[О пределе монотонной последовательности --- теорема Вейштрасса] \label{Weierstrass_theorem}
    Если $ \{a_n\} $ не строго возрастает, то существует \[ \lim_{n \to \infty}a_n = \sup\{a_n\} \]
    Для не строго убывающей последовательности \[ \lim_{n \to \infty}a_n = \inf\{a_n\} \]
\end{theorem}
\begin{proof}
    Пусть $ \{a_n\} $ не строго возрастает. Если $ \{a_n\} $ ограничена, то $ c = \sup{a_n} \in  \R $. Тогда по определению $ \sup $: \begin{gather} \begin{cases}
        \forall n \in \N\:  (a_n \leq\, c) \\
        \exists N \ (a_N > c - \varepsilon)
    \end{cases} \implies a_n \geq\, a_N, \forall n \geq N
    \end{gather}
    Тогда при $ n \geq  N $ имеем $ \forall \varepsilon > 0 $: \begin{gather}
        c - \epsilon < a_N \leq\, a_n \leq x \leq\, c + \varepsilon \\
        |a_n - c | < \varepsilon \implies c = \lim_{n \to \infty} a_n
    \end{gather}
    Пусть $ a_n $ неограниченна сверху, $ \sup\{a_n\} = + \infty $. Зафиксируем $ \varepsilon > 0 $ \begin{equation}
        \exists N \ \left(a_N > \frac{1}{\varepsilon}\right)
    \end{equation} В силу возрастания $ a_n \geq a_N $ при всех $ n \geq  N $ и значит, $ a_n > \dfrac{1}{\varepsilon}  \implies  \lim a_n = + \infty$
    Аналогично доказывается для не строго убывающей.
\end{proof}
