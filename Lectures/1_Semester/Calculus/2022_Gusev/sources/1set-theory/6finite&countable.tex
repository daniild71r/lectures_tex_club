\subsection{Конечные и счётные множества}
\begin{proposition}
   Отображение $ f: X \to Y $ является биекцией $ \Leftrightarrow f $ является одновременно инъекцией и сюрьекцией 
\end{proposition}
\begin{definition}
    Множества $ X \text{ и } Y $ равномощны $ \Leftrightarrow \exists \text{ биекция } f: X \to Y $ 
\end{definition}
\begin{example}
    $ \N \text{ и } \Z $ равномощны 
\end{example} \begin{proof}
    
\end{proof}
\begin{definition}
    Множество $ X $ называется конечным, если $ \exists n \in \N\, X $ равномощно множеству $ \overline{1,n}\defeq \{1, 2, \dots , n\} $
    В противном случае $ X $ называется бесконечным.
\end{definition}
\begin{proposition}
    Пусть $ m, n \in \N $ тогда $ \overline{1,n} \text{ и } \overline{1,m} $ равномощны $ \Leftrightarrow  m = n$ (доказывается по индукции по $ \min(m, n) $) 
\end{proposition}
\begin{proposition}
    Если $ X $ конечно, то $ \exists ! n \in \N$:
    $ X $ равномощно $ \overline{1,n} $. Пишут $ |X| = n $
\end{proposition}
\begin{proposition}
    Если $ X $ -- конечное множество и $ Y \sub X $, то $ Y $ -- тоже конечное множество (Достаточно доказать, что если $ Y \sub \overline{1,n} $, то $ Y $ конечно)
\end{proposition}
\begin{lemma}
    Если $ A \sub \N $, то в $ A $ имеется минимальный элемент. Если $ A $ -- ограничено, то есть и максимальный элемент
\end{lemma}
\begin{proof}
    
\end{proof}
\begin{definition}
    Если $ X \text{ равномощно } \N $, то оно называется счётным.
\end{definition}
\begin{proposition}
    Если $ X \sub \N $ бесконечно, то $ X $счётно.
\end{proposition}\begin{proof}
    Построим биекциию $ f: \N \to X $ \begin{gather}
        f(1)\defeq \min\left(X\right) \\
        f(2) \defeq \min\left(X\backslash f(1)\right) \\
        \dots 
    \end{gather}
    Тогда $ f $ -- инъекция
\end{proof}
\begin{proposition}
    Если $ n < m $, то $ f(n) \neq f(m) $, то $ f(m) $ по построению принадлежит множеству, в которое не входит $ f(n) $
\end{proposition}
Кроме того, $ f $ -- сюрьекция.
Пусть $ a \in X $, $ \overline{1,a} $ -- конечно. $ X \cap \overline{1, a} $ -- тоже конечное в силу (утв. 7) TODO $ \implies \exists k \in  \N: |X \cap \overline{1, a}| = k \implies  f(k) = a $
\begin{proposition}
    $ \forall \text{ бесконечные подмножества счётного Множества счётны } $
\end{proposition}
\begin{proposition}
    $ \N^2 $ счётно $ \N^2 = \{(m, n)|\ m, n \in \N\} $
\end{proposition}
Точка $ (m, n) $ находится на диагонали $ d = m + n - 1 $ \begin{gather}
    1 + \dots\ + d - 1 = \dfrac{(1 + d - 1)(d - 1)}{2} = \dfrac{d(d - 1)}{2} \\
    f(m, n)\defeq \dfrac{(m + n - 1)(m + n - 2)}{2} + m
\end{gather}
По построению $ f $ является биекцией.
\begin{note}
    \begin{gather}
        \Z  \left\{m - n \sconstr m, n \in \N \right\} \\
        \Q \defeq \left\{ \frac{m}{n} \sconstr m \in\, \Z, n \in \N \right\} \\
        \I \defeq \R \backslash \Q
    \end{gather}
\end{note}
\begin{proposition}
    Пусть $ f : X \to Y $ -- инъекция. Обозначим $ \Z\defeq \{f(x) | x \in X\}$. Тогда $ f: X \to \Z $ -- биекция
\end{proposition}
\begin{proposition}
    \begin{enumerate}
        \item $ \Q $ -- счётно.
        \begin{proof}
            \begin{gather}
                \forall r \in \Q_+ \exists\, \text{ несократимая дробь } \frac{p}{q} \\ 
                f: r \to (p, q) \text{ -- инъекция } \implies \Q_ + \text{ равномощно некоторому множеству  } \N^2, \\
                Q_ + \text{ -- бесконечно } \Rightarrow \Q_ +  \text{счётно} \text{ (как бесконечное подмножество счётного множества) }
            \end{gather}
        \end{proof}
    \end{enumerate}
\end{proposition}
\begin{proposition}
    Множество $ \R $ несчетно.
\end{proposition} \begin{proof}
    От противного. Пусть $ \exists  $ биекция $ f: \N \to \R $
    По теореме Кантора о вложенных отрезках \begin{gather}
        \exists c \in \R: \ c \in \cap_{n = 1}^ \infty[a_n, b_n] \implies \exists\, k \in\, \N: c = f(k)
    \end{gather}
    Но $ f(k) \nexists [a_n, b_k] \implies c \ni \cap_{n  = 1}^ \infty [a_n, b_n] $
    Пусть $ a_1 < b_1 $ и $ f(1) \ni [a_1, b_1] $
    Пусть $ [a_1, b_1] > [a_2, b_2] > \dots > [a_n, b_n] $ и \[\begin{cases}
        f(k) \ni [a_k, b_k] \\
        a_k < b_k
    \end{cases} \implies \exists a_{n + 1} \]
\end{proof}
\begin{definition}
    Последовательность $ (x_k) $ называется подпоследовательности последовательности $ (y_n) $,
    если $ \exists  $ строго возрастающая последовательность $ (n_k) $ натуральных чисел такой, что $ x_k = y_{n_k} $
\end{definition}
\begin{example}
    $ y_n = \dfrac{( - 1)^ n}{n}, x_k = \frac{1}{2k} $ -- последовательность $ (y_n) $
\end{example}
\begin{proposition}
    Если последовательность $ (x_n) $ имеет предел $ A \in  \overline{ \R} $, то $ \forall $ подпоследовательности $ (x_{n_k}) $ выполнено $ \lim_{k \to  \infty} x_{n_k} = A $.
\end{proposition}
\begin{proof}
   TODO 
\end{proof}
\begin{theorem}[Больцано - Вейерштасса]
    Если $ (x_n) $ ограничена, то $ \exists  $ подпоследовательность $ (x_{n_k}) $ имеется конечный предел.
\end{theorem}
\begin{proof}
    Если последовательность ограничена, то $ \exists a_0, b_0 \forall n \in \N: a_0 \leq  x_n \leq  b_0 $. Разделим этот отрезок пополам, хотя бы в одном из получившихся отрезков
    бесконечное число элементов, поскольку в последовательности бесконечное число элементов. Перейдем в него, назовём $ [a_1, b_1] $ и повторим для него тоже самое, тогда на $ k $-ом шаге:
    \begin{gather}
        b_k - a_k = 2^{ - k}(b - a)
    \end{gather}
    По теореме Кантора $ \exists ! c \in \cap_{n = 1}^\infty [a_n, b_n] $. Выберем $ n: x_{n_1}  \in [a_1, b_1], \dots , n_k : x_{n_k} \in [a_k, b_k]$, причём $ n_k < n_{k + 1} \forall k $.
    Тогда $ \sconstr x_{n_k} - c \sconstr \leq  2^{ - k}(b - a) \to  0 \implies  x_{n_k} \to c $
\end{proof}