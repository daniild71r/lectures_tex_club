\subsection{Расширенная числовая прямая}
\textbf{Внимание}: синтетический подраздел. Это означает, что если следовать хронологическому порядку изложения материала на лекциях, текст, написанный далее до конца подраздела, должен располагаться в другом месте, однако автор конспекта посчитал логичным расположить его именно тут.

\df{Расширенная числовая прямая $\RR$}{множество вещественных чисел, дополненное двумя бесконечностями: $\pm\infty$. Для $\forall x \in \R$ доопределены следующие операции:
    \begin{enumerate}
        \item $- \infty < x < + \infty$
        \item $ x + ( - \infty) = x - ( - \infty) = + \infty $
        \item $ x - ( + \infty) = x + ( - \infty) = - \infty $
        \item $ \dfrac{x}{ + \infty} = \dfrac{x}{ + \infty} = 0 $
        \item $ x \cdot \pm \infty = \pm \infty,\ x > 0 $
        \item $ x \cdot \pm \infty = \mp\, \infty,\ x < 0 $
        \item $ + \infty + ( + \infty) = + \infty $
        \item $ - \infty + ( - \infty) = - \infty $
        \item $ + \infty \cdot ( + \infty) = - \infty \cdot ( - \infty) = + \infty $
        \item $ + \infty \cdot ( - \infty) = - \infty \cdot ( + \infty) = - \infty $
   \end{enumerate}
}

\begin{note}
    Остаётся недопустимым выполнять
    \begin{gather*}
        + \infty - ( + \infty), \ - \infty - ( - \infty) \\
        - \infty + ( + \infty), \  + \infty + ( - \infty)
    \end{gather*}
\end{note}

\begin{equation}
    \begin{gathered}
        \forall a \in \R\ U_\epsilon(a) \defeq (a - \epsilon;\ a + \epsilon) \\
        U_\epsilon( + \infty) \defeq \left(\frac{1}{\epsilon}; + \infty\right) \\
        U_\epsilon( -\infty) = \left( - \infty; - \frac{1}{\epsilon}\right) \\
        U_\epsilon(\infty) = U_\varepsilon(+\infty) \cup U_\varepsilon(-\infty)
    \end{gathered}
\end{equation}

$U_\varepsilon(a)$ называют $\varepsilon$-окрестностью числа $a$.

\begin{equation}
    \begin{gathered}
        \forall a, b \in \R\ \sep(a, b) = \frac{\abs{a - b}}{2} \\
        \forall a \in \R\ \sep(-\infty, a) = \sep(a, +\infty) = \frac{-\abs{a} + \sqrt{a^2+4}}{2} \\
        \sep(-\infty, +\infty) = 1
    \end{gathered}
\end{equation}
\textbf{Внимание}: такая функция на лекциях не была определена.

\begin{theorem}[Очевидный разбор случаев]
    $ \forall a, b \in \RR\ U_{\sep(a, b)}(a) \cap U_{\sep(a, b)}(b) = \varnothing$
\end{theorem}