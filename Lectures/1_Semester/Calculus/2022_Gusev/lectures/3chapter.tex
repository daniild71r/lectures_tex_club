\section{Верхняя и нижняя грани}
\subsection{Определение}
\begin{definition}
    Пусть $ A \in \R $. Число $ m \in  \R $ называется верхней гранью множества $ A $,
    если $ \forall a \in A \ a \leq  m $ ( $ m $ -- мажоранта)
\end{definition}
Аналогично определяется нижняя грань.
\begin{definition}
    Если у множества $ A $ существует верхняя грань, то $ A $ называется огрнаниченным сверху.
    Если у множества $ A $ существует нижняя грань, то $ A $ называется огрнаниченным снизу.
\end{definition}
\begin{definition}
    Если у $ A   \nexists  $ верхней грани, то оно не огрнаниченное сверху.
    Аналогично с нижней гранью.
\end{definition}
\begin{definition}
    Если у $ A \exists  $ верхняя и нижняя грань, то оно называется ограниченным.
    Если у $ A \nexists  $ верхняя и нижняя грань, то оно называется ограниченным.
\end{definition}
\begin{example}
   $ \R $ неограничено сверху. Иначе $ \exists m \in \R:\ \forall x \in \R\ x \leq  m $: \begin{gather}
    m + 1 \leq\, m \\
    1 \leq\, 0 \text{ -- противоречие }
   \end{gather}
\end{example}
\subsection{Точные верхние грани}
\begin{definition}
    Пусть $ A \subset \R $ непусто. Если $ \exists s \in  \R \: s$ -- верхняя грань $ A $ и
    $ \forall m \in  \R:  m < s \ m $ -- не верхняя грань $ A $, то $ s $ называется точной верхней гранью.
    Если $ m $ точная верхняя грань $ A $, то $ \forall s > m \:  s$ -- верхняя грань, то \begin{gather}
        \forall a \in A \ a \leq\, m < s
    \end{gather} 
\end{definition}
\begin{theorem}[О точной верхней грани]\label{theorem_about_supremum}
    Если $ A \subset \R $ непусто и огрнаничено сверху, то у $ A \exists ! $ точная верхняя грань $ s \in \R $
\end{theorem}
\begin{proof}
    \begin{equation}
        B : = \{b \in \R| b \text{ --  верхняя грань  }A\}
   \end{equation}
   По построенпию $ A $ лежит слева от $ B $( $ \forall a \in  A\ \forall b \in  B \ a  \leq  b $)
   по аксиоме непрерывности $ \exists  s \in \R: s $ разделяет $ A \text{  и } B( \forall a \in A, b \in B\; a \leq s \leq b) \implies  s \in B $.
   Если $\exists m\ m < s $ и $ m \in B $, то $ s \leq  m $ -- противоречие. \newline
   Единственность. Пусть $ s' $ -- другая точная верхняя грань $ A \implies s' \in B \implies  s \leq  s'$
   Но по определению точной верхней грани множества $ A $ выполнено условие, что $ \forall x \leq s \ s$ не может быть точной верхней гранью --- противоречие. 
\end{proof}
Точная верхняя грань множества $ A $ обозначается $ \sup A $(супремум $ A $)
Аналогично для любого непустого огрнаниченного снизу множества $ A\subset \R $,
 $ \exists !  $ точная нижняя грань и обозначается $ \inf A $(инфинум $ A $)
 \begin{note}
    Если $ A $ не огрнаничено сверху, то $ \sup A: =+ \infty $.
    Если $ A $ не огрнаничено снизу, то $ \inf A : = - \infty $
 \end{note}
 \subsection{Принцип Архимеда}
 \begin{theorem}[Принцип Архимеда]\label{Archemedian_principle}
   \begin{gather}
    \forall x \in \R\ \exists n \in\, \N, n > x
   \end{gather}
 \end{theorem}\begin{proof}
    Иначе $ \exists n \in \R \ (\forall  n \in \N) \ n \leq  x \implies \N $ огрнаниченно сверху $ \implies  $ по 
    \ref{theorem_about_supremum} $ \exists ! s = \sup \N  \implies \forall n \in \N\ n \leq  s $ и $ s - 1 $ --- не верхняя грань $ \N \implies  \exists m \in  \N: m > s - 1 \implies m + 1 > s \implies  (m + 1 \notin \N)$,
    противоречие с индуктивностью $ \N $. 
 \end{proof}
\begin{note}
    \begin{gather}
        \lnot (\forall x \ P(x)) \longleftrightarrow (\exists x \ \lnot P(x)) \\
        \lnot (\exists P(x)) \longleftrightarrow (\forall x \ \lnot P(x))
    \end{gather}
    При отрицании кванторы меняются на противоположные.
\end{note}
\begin{note}
    Если $ A \subset \R $ непусто и огрнаничено сверху, то $ \sup A $ может как принадлежать $ A $,
    так и не принадлежать $ A $.
\end{note}
\begin{example}
    \begin{gather}
        A = \{x \in \R|\ x \leq\, 1\} \\
        \sup A = 1 \in A \\
        A = \{x \in \R | x < 1\} \\
        \sup A = 1 \notin A
    \end{gather}
\end{example}
\begin{exercise}
    Пусть $ B $ -- множество из доказательсва теоремы о точной верхней грани. Доказать, что $ \sup A = \inf B $
\end{exercise} \begin{proof}
    По построенпию $ A $ лежит слева от $ B $( $ \forall a \in  A\ \forall b \in  B \ a  \leq  b $)
   по аксиоме непрерывности $ \exists  s \in \R: s $ разделяет $ A \text{  и } B( \forall a \in A, b \in B\; a \leq s \leq b) \implies s = \sup A $. Кроме того $ \forall b \in B s \leq  B \implies  s = \inf B \implies \sup A = \inf B $.
\end{proof}
\section{Предел числовой последовательности}
\begin{definition}
    Пусть $ x $ -- формула вида $ x: \N \to \R $
    Тогда $ x $ называется числовой последовательностью и ( $ \forall n \in  \N \: x_n : = x(n) $) называется $ n $ - м членом последовательности.
\end{definition}
Пусть $ (x_n) $ -- числовая последовательность $ (x_n)_{n = 1}^\infty $
\begin{definition}
    Число $ a \in  \R $ называется пределом последовательности $ (x_n) $, если $ \forall \varepsilon > 0 $ \begin{gather}
        \exists N \in \N \: \forall n \geq N \: |x_n\, - a| < \varepsilon
    \end{gather}
    В этом случае пишут $ a = : \lim_{n \to \infty} x_n $
    Если $ a = 0 $, то $ (x_n) $ называется бесконечно малой.
\end{definition}
\begin{definition} $ \epsilon $-окрестность $ U $:
    \begin{gather}
        U_\epsilon(a): =(a - \epsilon; a + \epsilon); \quad U_\epsilon( + \infty): = \left(\frac{1}{\epsilon}; + \infty\right) \quad U_\epsilon( -\infty) = \left( - \infty; - \frac{1}{\epsilon}\right)
    \end{gather}
\end{definition}
\begin{note}
    \begin{equation}
        (a, b): = \{x \in \R | \ a < x < b\}
   \end{equation}
   
\end{note}
\begin{theorem}[Единственность предела]
    Если $ x_n \to a, n \to \infty $ и $ x_n \to b, n \to \infty $, то $ a = b $
\end{theorem} \begin{proof}
    Пусть $ a \neq  b $. Возьмём $ \varepsilon > 0 $ такое, что \begin{gather}
        U_\varepsilon(a) \cap U_\varepsilon(b) = \emptyset \\
        \text{ достаточно взять } \varepsilon < \frac{b - a}{2}  \\
        a + \varepsilon < a + \frac{b - a}{2} = \frac{a + b}{2} \\
        b - \varepsilon > b - \frac{b - a}{2} = \frac{a + b}{2}
    \end{gather}
    \begin{multline}
        \begin{rcases}
            \exists N_1 : \forall n \ge N_1 \; x_n \in U_\varepsilon(a) \\
            \exists N_2 : \forall n \ge N_2 \; x_n \in U_\varepsilon(b)
        \end{rcases}
        \thus \forall n \ge \max(N_1, N_2) \; x_n \in U_\varepsilon(a) \land x_n \in U_\varepsilon(b)
        \thus \\ x_n \in U_\varepsilon(a) \cap U_\varepsilon(b)
        \thus x_n \in \varnothing \text{ -- противоречие}
    \end{multline}
\end{proof}
\begin{example} Доказать
    \[ \lim_{n \to \infty} \dfrac{1}{n} = 0 \]
\end{example} \begin{proof}
    Пусть $ \varepsilon = \frac{1}{2} $.
    Нужно найти $ N \in \N: $ $ n \geq N $  $ \frac{1}{n} \in U_\varepsilon(0) $
    По принципу Архимеда \begin{gather}
        \exists N \in \N\, N > \frac{1}{2} \implies \forall n \in \N: n \geq\, N \text{ выполнено } n \geq  N > \frac{1}{\varepsilon} \implies  \frac{1}{n} < \varepsilon \implies\, \lim_{n \to \infty}\frac{1}{n} = 0
    \end{gather}
\end{proof}
\begin{exercise}
    Доказать, что $ \lim_{n \to +\infty}n = + \infty $.
\end{exercise} \begin{proof}
    Рассмотрим $ \frac{1}{\epsilon} $. По принципу Архимеда(\ref{Archemedian_principle}) \[ \exists N \in \N: N > \frac{1}{\epsilon} \implies N \in U_\epsilon( + \infty) \implies  \lim_{n \to \infty}n = + \infty \]
\end{proof}