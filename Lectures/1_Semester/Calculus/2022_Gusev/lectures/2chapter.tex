\section{Вещественные числа}
\subsection{Аксиомы действительных чисел}
\begin{definition}
    Множество $\R$, на котором введены операции \begin{gather}
        + : \R^2 \to \R \\ \cdot : \R^2 \to \R \\ \text{ и отношение } \leq \supset \R^2
    \end{gather} такие, что $ a, b \in \R $ \begin{gather}
        +(a, b) = a + b \\
        \cdot (a, b) = a \cdot b 
    \end{gather} выполнены следующие аксиомы:
    \begin{enumerate}
        \item $\exists 0 \in \R : \forall x \in \R \; 0 + x = x + 0 = x$ 
    \item $\forall x \in \R \; \exists (-x) \in \R : x + (-x) = (-x) + x = 0$ 
    \item $\forall x, y, z \in \R \; x + (y + z) = (x + y) = z$ % ассоциативность 
    \item $\forall x, y \in \R \; x + y = y + x$ % коммутативность 
    \item $\exists 1 \in \R \setminus \{0\} : \forall x \in \R \; 1 \cdot x = x \cdot 1 = 1$
    \item $\forall x \in \R \setminus \{0\} \; \exists x^{-1} \in \R : x \cdot x^{-1} = x^{-1} x = 1$
    \item $\forall x, y, z \in \R \; x \cdot (y \cdot z) = (x \cdot y) \cdot z$
    \item $\forall x, y \in \R \; x \cdot y = y \cdot x$
    \item $\forall x, y, z \in \R \; (x + y) \cdot z = x \cdot z + y \cdot z$ % дистрибутивность
    \item $\forall x \in \R \; x \le x$
    \item $\forall x, y \in \R \; (x \le y) \land (y \le x) \Rightarrow (x = y)$
    \item $\forall x, y, z \in \R \; (x \le y) \land (y \le z) \Rightarrow (x \le z)$
    \item $\forall x, y \in \R \; (x \le y) \lor (y \le x)$
    \item $\forall x, y, z \in \R \; (x \le y) \Rightarrow (x + z \le y + z)$
    \item $\forall x, y \in \R \; (0 \le x) \land (0 \le y) \Rightarrow (0 \le x \cdot y)$
    \item $\forall X, Y \subseteq \R : (\forall x \in X \; \forall y \in Y \; x \le y) \; \exists c \in \R : (\forall x \in X \; \forall y \in Y \; x \le c \le y)$
    \end{enumerate} называется множеством действительных чисел. 
\end{definition}
\begin{note}
    \begin{gather}
        x \leq y \longleftrightarrow (x, y) \in \leq 
        x < y \defev (x \leq y) \lor (x \neq y)
    \end{gather}
\end{note}
\begin{exercise}
    Привести множество $ \leq \subset \{0, 1\}^2 $ задающее стандартное отношение
\end{exercise}
\subsection{Следствия из аксиом}
    \begin{theorem}
        $ \forall a, b \in  \R $ уравнение $ a + x  = b$ имеет единственное решение $ x \in \R: x = ( - a) + b $
    \end{theorem}
    \begin{theorem}[Единственность нуля] \end{theorem}
    \begin{proof}
        Пусть  $ 0_1 $и $ 0_2 $ -- нейтральные элементы, тогда \begin{gather}
            0_1 + 0_2 = 0_2 \text{ так как } 0_1 \text{ -- нейтральное } \\
            0_1 + 0_2 = 0_1 \text{ так как } 0_2 \text{ -- нейтральное }
        \end{gather} $ \implies 0_1 = 0_2 $
    \end{proof}
        
    \begin{theorem}[Единственность противоположного элемента]
        $\exists! (-x): x + (-x) = 0$.
    \end{theorem} \begin{proof}
        Пусть $ a $ и $ b $ $ \in \R $ и являются противоположными к
    $ x \in \R $:
    \begin{gather}
        a + x =  x + a = 0 \\ b + x =  x + b = 0 \\
       x + a =  x + b = 0 \\ a + x + a =  a + x + b \\ 0 + a =  0 + b \\
       a = b
    \end{gather} 
    \end{proof}
      
    \begin{theorem}
        $ \forall x = \R \ 0 \cdot  x = 0 $ 
    \end{theorem} \begin{proof}
        \begin{gather}
            0 \cdot x = (0 + 0) \cdot x = 0 \cdot x + 0 \cdot x \\
            ( -(0 \cdot x)) + 0 \cdot x = ( -(0 \cdot x) + 0 \cdot x) + 0 \cdot x \\
            0 = 0 + 0 \cdot x \\
            0 = 0 \cdot x
        \end{gather}
    \end{proof} 
    \begin{exercise}
        $ - x = ( - 1)x $
    \end{exercise} \begin{proof}
        \begin{gather}
            0 = 0 \\
        x + (-x) = x \cdot 0 \\
        x + (-x) = x \cdot (1 + (-1)) \\
        x + (-x) = x + (-1) \cdot x \\
        (-x) = (-1) \cdot x
        \end{gather}
    \end{proof}
    \begin{exercise}
        $ 0 < 1 $
    \end{exercise} \begin{proof}
        Предположим $ 1 \leq 0 $, тогда по аксиоме 14 $ 1 + (-1) \leq 0 + (-1) $, то есть $ 0 \leq - 1 $.
        Тогда $ 0 \leq ( - 1)^2 = 1 $, получили $ 0 \leq  1 $ --- противоречие.
    \end{proof}
\section{Натуральные числа и математическая индукция}
\subsection{Индуктивное множество}
\begin{definition} \label{definition_F}
    Пусть $ F $ -- произвольное семейство множеств. Тогда \begin{gather}
        \cap F : = \{x|\ \forall A \in F\ x \in A\} \\
        \cup F : = \{x |\ \exists A \in F\ x \in A\}
    \end{gather} 
\end{definition}
\begin{example}
    $ F = \{\{2, 3\}, \{3, 4\}\} $: \begin{gather}
        \cap F = \{2, 3\}\cap\{3, 4\} = \{3\} \\
        \cup F = \{2, 3\}\cup \{3, 4\} = \{2, 3, 4\} 
    \end{gather}
\end{example}
\begin{definition}
    $ X \subset \R $ называется индуктивным, если \begin{enumerate}
        \item $ 1 \in  X $ 
        \item $ \forall  x \in X $ выполнено $ x + 1 \in X $
    \end{enumerate}
\end{definition}
\begin{note}
    $ a \in X \ a $ является элементом множества $ X $
 $ A \subset X \ A $ является подмножеством множества $ X\ (\forall a \in  A a \in  X) $
\end{note}
 \subsection{Натуральные числа}
 \begin{definition}
    $ F: = \{A \subset \R |\ A \text{ -- индуктивно }\} $
    $ \N: = \cup F $ называется множеством натуральных чисел.
 \end{definition}
 \begin{proposition}
    Пусть $ g $ -- семейство подмножеств $ \R $ и $ \forall  A \in g \ A $ -- индуктивно
    Тогда $ \cap g $ -- тоже индуктивно \label{definition_inductive}
 \end{proposition}\begin{proof}
    $ M : = \cap g$. $ \forall A \in  g \ A \text{ -- индуктивно } \implies 1 \in A $
    \begin{equation}
        x \in M \implies \forall A \in g \; x \in A \implies \forall X \in g \; (x + 1) \in A \implies (x + 1) \in M
   \end{equation}
 \end{proof}
 \begin{proposition}
    Пусть $ x \in  M \ x + 1 \in^? M $
 \end{proposition}
\begin{proof}
    $\forall A \in g$ выполнено $ x \in A \implies x + 1 \in A $
    Это верно $ \forall A \in g \implies x + 1 \in M $
\end{proof}
\begin{proposition}
    Множество $ \N $ единственно. 
\end{proposition}
\begin{proof}  
    \phantom \\
    \begin{enumerate}
        \item $\N$ индуктивно по утвкрждению \ref{definition_inductive}
        \item $ \forall A \in  F\ \N \subset A $ -- Следствие из Определения
    \end{enumerate}
\end{proof}

\begin{theorem}[Принцип математической индукции] \label{math_induction}
    Если $ M \subset \N, 1 \in M $ и $ \forall n \in M \; n + 1 $, то $ M = \N $
\end{theorem} \begin{proof}
    $ M $ удовлетворяет определению индуктивого подмножества $ \R $, значит $ N \sub M $, но $ M \sub N \implies M = \N$.
\end{proof}
\begin{theorem}
    $ \forall x \in \N \; x \ge 1 $
\end{theorem}
\begin{proof}
    Пусть $\exists x \in N : x < 1$. Выберем произвольный набор $A \in F$(определение \ref{definition_F}) и построим $B \defeq \{b \sconstr b \in A, b > x\}$. $1 \in B$ потому что $1 \in A$ и $x < 1$, $\forall c \in B \; (1 + c) \in A$ и так как $(1 + c) > (1 + x) > x$, $(1 + c) \in B$ $\thus B \in F$. $x \notin B \thus x \notin \cap F \implies  x \notin \N$ --- противоречие.
\end{proof}
\begin{theorem}
    $ \forall m, n \in \N \; m + n \in \N,\ m \cdot n \in \N$
\end{theorem}
\begin{proof}
    Построим $M_1 \defeq \{x \sconstr x \in \N,\; (m + x) \in \N\}$. $M_1$ удовлетворяет критериям теоремы \ref{math_induction} $\thus M_1 = \N \thus n \in M_1 \thus m + n \in \N$.

    Построим $M_2 \defeq \{x \sconstr x \in \N,\; m \cdot x \in \N\}$. Если $x \in M_2$, то $m \cdot (x + 1) = m \cdot x + m \in \N$ (по 1 части утвержения,так $x \in M_2 \Rightarrow m \cdot x \in \N$, $m \in \N$). Поэтому, $M_2$ удовлетворяет критериям теоремы \ref{math_induction} $\thus M_2 = \N \thus n \in M_2 \thus m \cdot n \in \N$.
\end{proof}