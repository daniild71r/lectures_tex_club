\begin{definition}
	\textit{Метрическим} пространством $(X, \rho)$ называется множество
	$X$ с определенной функцией $\rho(\cdot, \cdot):
	X \times X \ra \R$ (метрика),
	обладающей свойствами:
	\begin{enumerate}
		\item $(\forall x, y \in X)\ \ \rho(x, y) \ge 0$,
			причём $\rho(x, y) = 0 \lra x = y$
		
		\item $(\forall x, y \in X)\ \ \rho(x, y) =
			\rho(y, x)$
		
		\item $(\forall x, y, z \in X)\ \ \rho(x, y)
			\le \rho(x, z) + \rho(z, y)\ $ \textit{(неравенство треугольника)}
	\end{enumerate}
\end{definition}

\begin{lemma}
	Любое линейное нормированное пространство является
	метрическим пространством с метрикой, индуцированной
	нормой по правилу:
	\[
		\rho(x, y) := ||x - y||
	\]
\end{lemma}

\begin{proof}
	Доказательство сводится к проверке свойств:
	\begin{enumerate}
		\item $(\forall x \in X)\ \ \|x\| \ge 0
			\Ra (\forall x, y \in X)\ 
			\rho(x, y) = \|x - y\| \ge 0$
		
		\item
		$
			(\forall x, y \in X)\ 
			\rho(y, x) = ||y - x|| = ||(-1) \cdot (x - y)||
			= |-1| \cdot ||x - y|| = ||x - y|| = \rho(x, y)
		$
		
		\item 
		$
			(\forall x, y, z \in X)\ 
			\rho(x, y) = ||x - y|| = ||(x - z) + (z - y)||
			\le ||x - z|| + ||z - y|| = \rho(x, z) + \rho(z, y)
		$
	\end{enumerate}
\end{proof}

\begin{corollary}
	$\R^n$ --- метрическое пространство с метрикой
	\[
		\rho(\vec{x}, \vec{y}) = \sqrt{\sum_{i = 1}^{n}
		(x_i - y_i)^2} = \|\vec{x} - \vec{y}\|
	\]
	$\Cm^n$ также, но есть отличие:
	\[
		\rho(\vec{x}, \vec{y}) = \sqrt{\sum_{i = 1}^{n}
		|x_i - y_i|^2} = \|\vec{x} - \vec{y}\|
	\]
\end{corollary}

\begin{proposition}
	Любое множество является метрическим пространством
\end{proposition}

\begin{proof}
	Пусть $X$ - произвольное множество.
	Тогда можно определить метрику как
	\[
		\rho(x, y) = \System{
			&{1,\ x \neq y}
			\\
			&{0,\ x = y}
		}
	\]
\end{proof}

\begin{note}
	Дальнейшие определения даны для метрического
	пространства $X$. В качестве примера удобно
	брать $X = \R^2$.
\end{note}

\begin{definition}
	\textit{Открытым шаром} с центром в точке
	$x_0 \in X$ радиусом $\eps > 0$
	(или же $\eps$-окрестностью точки $x_0$)
	называется множество
	\[
		U_\eps(x_0) = \{x \in X \such \rho(x, x_0) < \eps\}
	\]
\end{definition}

\begin{definition}
	Точка $x_0$ называется \textit{внутренней точкой}
	множества $A \subset X$, если она принадлежит
	$A$ вместе с некоторой своей $\eps$-окрестностью:
	\[
		(\exists \eps > 0)\ \ U_\eps(x_0) \subset A
	\]
\end{definition}

\begin{definition}
	\textit{Внутренностью} множества $A$ называется
	множество всех внутренних точек множества $A$.
	Обозначается как
	\[
		\Int A,\ \mc{A}
	\]
\end{definition}

\begin{definition}
	Точка $x_0$ называется \textit{точкой прикосновения}
	множества $A \subset X$, если любая $\eps$-окрестность
	пересекается с $A$:
	\[
		(\forall \eps > 0)\ \ U_\eps(x_0) \cap A \neq \emptyset
	\]
\end{definition}

\begin{definition}
	Множество всех точек прикосновения множества
	$A \subset X$ называется его \textit{замыканием} и
	обозначается как
	\[
		\cl A,\ \bar{A}
	\]
\end{definition}

\begin{definition}~

	\begin{itemize}
		\item Множество $A \subset X$ называется
			\textit{открытым}, если все его точки - внутренние,
			то есть $A \subset \Int A$.
			Естественно, $\emptyset$ - открытое множество.

		\item Множество $A \subset X$ называется
			\textit{замкнутым}, если оно содержит все свои
			точки прикосновения, то есть $A \supset \cl A$.
			Естественно, $\emptyset$ - замкнутое множество
	\end{itemize}
\end{definition}

\begin{lemma}
	Для любого $A \subset X$ верно, что
	\[
		\Int A \subset A \subset \cl A
	\]
\end{lemma}

\begin{proof}
	Первое включение очевидно, потому что любая
	внутренняя точка принадлежит $A$, так как из
	определения: 
	$(\exists \eps > 0)\ \ U_\eps(x_0) \subset A \Ra
	x_0 \in A$
	
	Второе включение следует из того, что если
	$x_0 \in A$, то
	\[
		(\forall \eps > 0)\ \ U_\eps(x_0) \cap
		A \supset \{x_0\} \neq \emptyset
	\]
	Значит, любая точка $A$ также лежит и в замыкании $A$.
\end{proof}

\begin{corollary} \label{defEqualLemma}
	Из определений и леммы выше сразу следует:
	\begin{itemize}
		\item $A$ - открытое множество $\lra \Int A = A$
	
		\item $A$ - замкнутое множество $\lra \cl A = A$ 
	\end{itemize}
	При этом не бывает открытых и замкнутых одновременно множеств
\end{corollary}

\begin{lemma} \label{includeLemma}
	$(\forall A_1, A_2)\ A_1 \subset A_2 \subset X$ верно, что
	\begin{align*}
		&\Int A_1 \subset \Int A_2
		\\
		&\cl A_1 \subset \cl A_2
	\end{align*}
\end{lemma}

\begin{proof}~
	\begin{itemize}
		\item Вначале разберёмся с внутренностями
		множеств. Пусть $x_0 \in \Int A_1$. Тогда
		\[
			(\exists \eps > 0)\ \ U_\eps(x_0)
			\subset A_1 \subset A_2
		\]
		Значит, по определению $x_0 \in \Int A_2$ тоже.
		
		\item Теперь посмотрим на замыкание.
		Пусть $x_0 \in \cl A_1$. Тогда
		\[
			(\forall \eps > 0)\ \ U_\eps(x_0) \cap
			A_1 \neq \emptyset
		\]
		Мы знаем, что $A_1 \subset A_2$, значит:
		\[
			(\forall \eps > 0)\ \ U_\eps(x_0) \cap
			A_2 \neq \emptyset
		\]
		Следовательно, $x_0 \in \cl A_2$.
	\end{itemize}
\end{proof}

\begin{lemma} (Открытость открытого шара) Любой открытый
	шар является открытым множеством, то есть:
	$(\forall x_0 \in X)(\forall \eps > 0)\ U_\eps(x_0)$
	--- открытое множество.
\end{lemma}

\begin{idea}
	Докажем по определению, то есть покажем, что любая точка
	из открытого шара является внутренней точкой для него.
\end{idea}

\begin{proof}
	Рассмотрим $\forall x_1 \in U_\eps(x_0)$. Тогда
	\[
		r := \rho(x_0, x_1) < \eps
	\]
	Рассмотрим шар с центром в точке $x_1$ и
	радиусом $\eps - r$. Выберем
	$\forall x_2 \in U_{\eps - r}(x_1)$. Тогда
	\[
		\rho(x_1, x_2) < \eps - r
	\]
	Отсюда следует
	\[
		\rho(x_0, x_2) \le \rho(x_0, x_1) +
		\rho(x_1, x_2) < r + \eps - r = \eps
	\]
	То есть $U_{\eps - r}(x_1)
	\subset U_\eps(x_0)$. Значит $x_1$ - внутренняя точка.
	А раз мы выбирали $x_1$ как произвольную точку данного шара,
	то и любая точка этого открытого шара есть его внутренняя точка.
	Значит, по определению данный открытый шар является открытым множеством.
\end{proof}

\begin{definition}
	\textit{Замкнутым шаром} с центром в точке
	$x_0 \in X$ и радиусом $\eps > 0$ называется
	\[
		\bar{B}_\eps(x_0) :=
		\{x \in X \such \rho(x_0, x) \le \eps\}
	\] 
\end{definition}

\begin{lemma} (Замкнутость замкнутого шара) Любой замкнутый
	шар является замкнутым множеством, то есть:
	$(\forall x_0 \in X)(\forall \eps > 0)\ \bar{B}_\eps(x_0)$
	--- замкнутое множество.
\end{lemma}

\begin{idea}
	Докажем по определению. Покажем, что любая точка
	прикосновения лежит в замкнутом шаре.
\end{idea}

\begin{proof}
	Пусть $x$ - точка прикосновения для
	$\bar{B}_\eps(x_0)$. Это означает, что
	\[
		(\forall \eta > 0)\ \ U_\eta(x) \cap
		\bar{B}_\eps(x_0) \neq \emptyset \Ra
		\exists x_1 \in U_\eta(x) \cap \bar{B}_\eps(x_0)
	\]
	Оценим расстояние от $x_0$ до $x$:
	\[
		\rho(x_0, x) \le \rho(x_0, x_1) + \rho(x_1, x) <
		\eps + \eta
	\]
	То есть получили утверждение
	\[
		(\forall \eta > 0)\ \rho(x_0, x) < \eps + \eta
	\]
	Из этого следует, что
	\[
		\rho(x_0, x) \le \eps
	\]
	Следовательно, $x$ --- точка замкнутого шара.
\end{proof}

\begin{theorem}
	$(\forall A \subset X)$
	\begin{enumerate}
		\item $\Int A$ --- открыто. То есть
			внутренность любого множества открыта.
		\item $\cl A$ --- замкнуто. То есть
			замыкание любого множества замкнуто.
	\end{enumerate}
\end{theorem}

\begin{idea}
	Построим доказательство исходя из определения. В первом
	случае возьмем произвольную точку из внутренности $A$ и докажем, что
	она внутренняя для внутренности $A$. Во втором случае - наоборот,
	докажем, что любая точка прикосновения для замыкания $A$
	также лежит в этом замыкании.
\end{idea}

\begin{proof}~
	\begin{enumerate}
		\item Положим $G := \Int A$. Выберем $\forall x_0
			\in G$. Раз точка лежит в данном множестве, то
			\[
				(\exists \eps > 0)\ \ U_\eps(x_0) \subset A
			\]
			По лемме \ref{includeLemma} из этого следует
			\[
				\Int U_\eps(x_0) \subset \Int A
			\]
			Так как открытый шар является открытым
			множеством, то $U_\eps(x_0) = \Int U_\eps(x_0)$,
			тогда получаем вложение
			\[
				U_\eps(x_0) \subset G
			\]
			То есть $x_0$ - внутренняя точка $G$ по определению.
			Значит $G$ - открытое множество.
	
		\item Положим $K := \cl A$. Пусть $x_0$ --- точка
			прикосновения множества $K$. Это означает
			\[
				(\eps_0 = \frac{\eps}{2})\ \ 
				U_{\eps / 2}(x_0) \cap K \neq 0
			\]
			Обозначим за $x_1 \in U_{\eps / 2}(x_0) \cap K$. Тогда
			$x_1 \in K$, а так как $K$ --- замыкание $A$, то по
			определению $x_1$ --- точка прикосновения множества $A$.
			То есть
			\[
				U_{\eps / 2}(x_1) \cap A \neq \emptyset
			\]
			Теперь выберем $(\eps_1 = \frac{\eps}{2}) \Ra
			x_2 \in U_{\eps / 2}(x_1) \cap A$ и
			оценим расстояние между ней и $x_0$:
			\[
				\rho(x_0, x_2) \le \rho(x_0, x_1) +
				\rho(x_1, x_2) < \frac{\eps}{2} + \frac{\eps}{2}
				= \eps
			\]
			Следовательно мы построили:
			\[
				(\forall \eps > 0)\ \ U_{\eps}(x_0) \cap A \supset
				\{x_2\} \neq \emptyset
			\]
			Значит $x_0$ - точка прикосновения множества $A$ $\lra x_0 \in K$.
	\end{enumerate}
\end{proof}

\begin{example}
	Порой геометрическая интерпретация данной модели
	обманывает, ибо, например, здесь шар с большим
	радиусом может оказаться вложенным в шар с меньшим радиусом.
	Рассмотрим метрическое пространство $X = (-1; 1)$:
	\begin{align*}
		&{U_{1}(0) = (-1; 1)}
		\\
		&{U_{5/4}(0.5) = \left(-\frac{3}{4}; 1\right)
		\Ra U_{5/4}(1/2) \subsetneq U_{1}(0)}
	\end{align*}
\end{example}

\begin{lemma}
	$(\forall A \subset X)$
	\begin{itemize}
		\item $X \bs \Int A = \cl(X \bs A)$
		
		\item $X \bs \cl A = \Int(X \bs A)$
	\end{itemize}
\end{lemma}

\begin{idea}
	Совершаем равносильные переходы, рассматривая точки
	из данных множеств и используя
	отрицания определений внутренней точки и точки прикосновения.
\end{idea}

\begin{proof}~
	\begin{itemize}
		\item Используем определение внутренности $A$.
			Тогда $x \in X \bs \Int A \lra x$ не
			лежит в $\Int A$, а значит для любого открытого
			шара есть точка, не лежащая в $A$, то есть
			\[
				x \in X \bs \Int A \lra
				(\forall \eps > 0)\ U_\eps(x) \cap
				(X \bs A) \neq \emptyset \lra x \in \cl(X \bs A)
			\]
			
		\item Здесь по аналогии:
			\[
				x \in X \bs \cl A \lra
				(\exists \eps > 0)\ U_\eps(x) \cap A
				= \emptyset \lra
				(\exists \eps > 0)\ U_\eps(x) \subset (X
				\bs A) \lra x \in \Int (X \bs A)
			\]
	\end{itemize}
\end{proof}

\begin{theorem} \label{additionInverse}
	$(\forall A \subset X)$
	\begin{itemize}
		\item $A$ --- открыто $\lra X \bs A$ --- замкнуто
		\item $A$ --- замкнуто $\lra X \bs A$ --- открыто
	\end{itemize}
\end{theorem}

\begin{idea}
	Используем критерий (\ref{defEqualLemma}) замкнутого
	и открытого множеств и только что доказанную лемму.
\end{idea}

\begin{proof}
	По следствию \ref{defEqualLemma} $\Ra A$ --- открыто $\lra
	\Int A = A$.
	
	По последней лемме $\Ra X \bs A = X \bs \Int A
	= \cl (X \bs A)$, то есть по этому же следствию \ref{defEqualLemma}
	$X \bs A$ --- замкнуто. Второе утверждение доказывается аналогично.
\end{proof}

\begin{definition}
	Внутренняя точка дополнения множества $A \subset X$
	называется \textit{внешней точкой}.
\end{definition}

\begin{definition}
	Точка $x$ называется \textit{граничной точкой} множества
	$A \subset X$, если
	\[
		(\forall \eps > 0)
		\System{
			&U_\eps(x) \cap A \neq \emptyset \\
			&U_\eps(x) \cap (X \bs A) \neq \emptyset
		}
	\]
	Множество всех граничных точек $A$ называется
	\textit{границей} $A,\ \ \vdelta A$
\end{definition}

\begin{lemma}
	$(\forall A \subset X)\ \ \vdelta A = \cl A \bs \Int A$
\end{lemma}

\begin{idea}
	Действуем по определению, покажем, что первая строчка отвечает
	за принадлежность замыканию, а вторая - за отрицание
	принадлежности внутренности.
\end{idea}

\begin{proof}
	\[
	x \in \vdelta A \lra (\forall \eps > 0)
	\System{
		&U_\eps(x) \cap A \neq \emptyset \lra x \in \cl(A)\\
		&U_\eps(x) \cap (X \bs A) \neq \emptyset \lra
		(\forall \eps > 0)\ U_\eps(x) \not\subset A \lra x \notin \Int A
	}
	\]
	Это значит, что
	\[
		x \in \vdelta A \lra x \in \cl A \bs \Int A
	\]
\end{proof}

\begin{theorem} (Основное свойство совокупности открытых множеств) \label{mainProp}
	Пусть $(X, \rho)$ - метрическое пространство. Тогда
	\begin{enumerate}
		\item $\emptyset,\ X$ --- открытые множества
		
		\item $(\forall G_1, G_2\ \text{- открытые})\ \ 
			G_1 \cap G_2 $ --- открытое
		
		\item $(\forall \{G_\alpha\}_{\alpha \in A}\ \text{- открытые})
			\ \bigcup\limits_{\alpha \in A}
			G_\alpha$ - открытое, где $A$ --- некоторое множество
			индексов
	\end{enumerate}
\end{theorem}

\begin{proof}~
	\begin{enumerate}
		\item $\emptyset$ --- замкнутое и открытое множество.
			Следовательно, по теореме \ref{additionInverse}
			$\Ra X \bs \emptyset = X$ --- открытое множество.
		
		\item Рассмотрим $\forall x \in G_1 \cap G_2$. Из выбора следует
			\begin{align*}
				&(\exists \eps_1)\ \ U_{\eps_1}(x) \subset G_1
				\\
				&(\exists \eps_2)\ \ U_{\eps_2}(x) \subset G_2 
			\end{align*}
			Следовательно, $(\exists \eps_0 = \min(\eps_1, \eps_2))\ 
			U_{\eps_0}(x) \subset G_1 \cap G_2$. То есть $G_1 \cap G_2$ - открыто
		
		\item $x \in \bigcup\limits_{\alpha \in A} G_\alpha \Ra
			(\exists \alpha_0 \in A)\ \ x \in G_{\alpha_0}$. Значит,
			так как $G_{\alpha_0}$ --- открытое, то
			\[
				(\exists \eps > 0)\ \ U_\eps(x) \subset G_{\alpha_0}
				\subset \bigcup\limits_{\alpha \in A} G_{\alpha}
			\]
			То есть всё объединение открытое по определению
	\end{enumerate}
\end{proof}