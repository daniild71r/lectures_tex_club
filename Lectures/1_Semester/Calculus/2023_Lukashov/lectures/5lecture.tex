\subsection{Комплексные числа}

\begin{definition}
    \textit{Множеством комплексных чисел} называют множество $\Cm = \R^2$.
\end{definition}

\subsubsection*{Сложение}

\begin{definition}
    \textit{Суммой} двух комплексных чисел $(a, b)$ и $(c, d)$ называют число
    \[
        (a, b) + (c, d) := (a + c, b + d)
    \]
\end{definition}

\subsubsection*{Умножение}

\begin{definition}
    \textit{Произведением} двух комплексных чисел $(a, b)$ и $(c, d)$ называют число
    \[
        (a, b) \cdot (c, d) := (ac - bd, ad + bc)
    \]
\end{definition}

\subsubsection*{Мнимая единица}

\begin{definition}
    \textit{Мнимой единицей} $i$ называют комплексное число $(0, 1)$, которое из определения выше имеет свойство:
    \[
        i^2 := -1
    \]
\end{definition}

\begin{proposition}
    Множество $\R$ вложено в множество $\Cm$.
\end{proposition}

\begin{proof}
    Действительно, если $a \in R$, то $a = (a, 0)$. Несложно проверить, что все операции будут точно такими же, как и с обычными действительными числами.
\end{proof}

\subsubsection*{Алгебраическая форма комплексного числа}

\begin{definition}
    Заметим, что $(a, b) = a \cdot (1, 0) + b \cdot (0, 1)$. То есть:
    $$
        (a, b) = a + bi
    $$
    Запись числа $z$ в виде $a + bi$ называется \textit{алгебраической формой комплексного числа}
\end{definition}

\begin{center}
	\scalebox{1}{
		\begin{tikzpicture}
			\clip (-1.4, -1.3) rectangle (3.4, 3.3);
			\draw [->] (-1, 0) -- (3, 0) node [above, black] {$\re z$};
			\draw [->] (0, -1) -- (0, 3) node [right, black] {$\im z$};
			
			\draw [line width = 1pt, black!15!blue] (1.4,3pt) -- (1.4,-3pt) node [below, black] {$1$};
			\draw [line width = 1pt, black!15!blue] (3pt,1.4) -- (-3pt,1.4) node [left, black] {$i$};
			
			\draw [->, black!15!blue] (0, 0) -- (2, 1.5) node [black, above right, scale = 1.2] {$z$};
			\node[draw, circle, inner sep=1.4pt, fill, black!15!blue] at (2.06, 1.54) {};
			
			\coordinate (a1) at (2, 1.5);
			\coordinate (b) at (0, 0);
			\coordinate (c) at (1, 0);
			
			\pic [draw, ->] {angle = c--b--a1};
			\node [] at (0.75, 0.25) {$\phi$};
		\end{tikzpicture}
	}
\end{center}

\begin{definition}
    \textit{Модулем} комплексного числа нызывают число:
    $$
        |z| := \sqrt{a^2 + b^2}
    $$
\end{definition}

\begin{definition}
    \textit{Реальной частью} комплексного числа называют число $a$ в его алгебраической форме:
    $$
        \re(a + bi) := a
    $$
\end{definition}

\begin{definition}
    \textit{Мнимой частью} комплексного числа называют число $b$ в его алгебраической форме:
    $$
        \im(a + bi) := b
    $$
\end{definition}

\subsubsection*{Неравенство треугольника}

\begin{center}
	\scalebox{1}{
		\begin{tikzpicture}
			\clip (-1.4, -1.3) rectangle (3.4, 3.3);
			\draw [->] (-1, 0) -- (3, 0) node [above, black] {$\re z$};
			\draw [->] (0, -1) -- (0, 3) node [right, black] {$\im z$};
			
			\draw [line width = 1pt, black!15!blue] (1,3pt) -- (1,-3pt) node [below, black] {$1$};
			\draw [line width = 1pt, black!15!blue] (3pt,1) -- (-3pt,1) node [left, black] {$i$};
			
			\draw [->, black!15!blue] (0, 0) -- (1.8, 0.8) node [black, below, scale = 1] {};
			\node[draw, circle, inner sep=1pt, fill, black!15!blue] at (1.85, 0.82) {};
			\node [] at (1.25, 0.3) {$z_1$};
			
			\draw [->, black!15!blue] (1.85, 0.82) -- (2.3, 2.6) node [black, below, scale = 1] {};
			\node[draw, circle, inner sep=1pt, fill, black!15!blue] at (2.34, 2.63) {};
			\node [] at (2.4, 2) {$z_2$};
			
			\draw [->, black!15!blue] (0.0, 0.0) -- (2.3, 2.6) node [black, below, scale = 1] {};
			\node [] at (0.95, 2.05) {$z_1 + z_2$};
		\end{tikzpicture}
	}
\end{center}

Геометрически очевидны следующие неравенства:
\begin{align*}
    &|z_1 + z_2| \le |z_1| + |z_2| \\
    &|z_1 - z_2| \ge ||z_1| - |z_2||
\end{align*}

\subsubsection*{Деление комплексных чисел}

\begin{definition}
    Комплексное число $z_3$ называется \textit{частным} от деления числа $z_1$ на число $z_2$, если верно равенство:
    $$
        z_2 \cdot z_3 = z_1 \lra z_3 := \frac{z_1}{z_2} := z_1 / z_2
    $$
\end{definition}

\begin{corollary}
    Выведем действительную и мнимую часть частного, если $z_1 = a + bi$, а $z_2 = c + di$. При этом обозначим $z_3 = x + yi$:
    \begin{align*}
        &(c + di) \cdot (x + yi) = a + bi
        \\
        &cx - dy + (cy + dx)i = a + bi
        \\
        &\Ra \System{a = cx - dy \\ b = cy + dx}
        \Ra \System{x = \frac{\dse ac + bd}{\dse c^2 + d^2} \\ y = \frac{\dse bc - ad}{\dse c^2 + d^2}}
    \end{align*}
\end{corollary}

\subsubsection*{Комплексно сопряженное число}

\begin{definition}
    Число $\bar{z}$ называется \textit{комплексно сопряжённым} к числу $z$, если
    $$
        z = a + bi \Ra \bar{z} = a - bi
    $$
\end{definition}

\begin{proposition}
    Произведение комплексного числа $z$ на своё сопряженное является квадратом модуля
\end{proposition}

\begin{proof}
    Пусть $z = a + bi$. Тогда:
    $$
        z \cdot \bar{z} = (a + bi) \cdot (a - bi) = a^2 + b^2 = |z|^2
    $$
\end{proof}

\subsubsection*{Аргумент комплексного числа}

\begin{definition}
    \textit{Аргументом} комплексного числа $z = a + bi$ называется угол $\phi$, отсчитываемый от положительного направления оси $\re$, с точностью до $2\pi k$, $k \in \Z$
    $$
        \arg z = \phi + 2\pi k, k \in \Z
    $$
    Угол считается положительным, если отсчитывается против часовой стрелки, и отрицательным, если наоборот.
\end{definition}

\begin{note}
    Аргумент определен только для комплексного числа, не равного нулю.
\end{note}

\subsubsection*{Комплексное число в полярной записи}

\begin{definition}
    Несложно заметить, что
    \begin{align*}
        a = |z| \cdot \cos \phi \\
        b = |z| \cdot \sin \phi
    \end{align*}
    $$
        \Ra z = |z|(\cos \phi + i \sin \phi)
    $$
\end{definition}

\subsubsection*{Умножение чисел в полярных координатах}

Пусть есть 2 комплексных числа $z_1$ и $z_2$:

\begin{align*}
    z_1 = |z_1|(\cos\phi + i \sin\phi)
    \\
    z_2 = |z_2|(\cos\psi + i \sin\psi)
\end{align*}

Найдём их произведение в виде комплексного числа, записанного в полярных координатах:
\begin{multline}
    z_1 \cdot z_2 = |z_1||z_2|(\cos\phi + i \sin\phi)(\cos\psi + i \sin\psi) = \\
    |z_1||z_2|(\cos\phi \cdot \cos\psi - \sin\phi \cdot \sin\psi + i(\sin\phi \cdot \cos\psi + \sin\psi \cdot \cos\phi)) = \\
    |z_1||z_2|(\cos(\phi + \psi) + i \sin(\phi + \psi))
\end{multline}

Таким образом,
$$
    \System{
    &\arg(z \cdot w) = \arg(z) + \arg(w)
    \\
    &|z_1 \cdot z_2| = |z_1| \cdot |z_2|
    }
$$

\subsubsection*{Показательная форма комплексного числа}

\begin{definition}
    Комплексное число можно записать как степень по натуральному основанию
    $$
        \cos \phi + i \sin \phi = e^{i \phi}
    $$
    Также это выражение называется \textit{формулой Эйлера}. С её помощью, комплексное число можно записать в \textit{показательной форме}.
    $$
        z = |z| \cdot e^{i \phi}
    $$
\end{definition}

\begin{note}
    Сейчас формулу Эйлера нужно принять <<на веру>>. В будущем её можно и нужно доказать.
\end{note}

\subsubsection*{Комплексное расширение тригонометрических функций}

Имея на руках формулу Эйлера, можно вывести интересные выражения для тригонометрических функций:
\begin{align*}
    e^{i\phi} &= \cos\phi + i \sin\phi
    \\
    e^{-i\phi} &= \cos\phi - i \sin\phi
    \\
    \Ra \cos\phi &= \frac{\dse e^{i\phi} + e^{-i\phi}}{2}
    \\
    \sin\phi &= \frac{\dse e^{i\phi} - e^{-i\phi}}{2}
    \\
    \tg\phi &= \frac{\sin\phi}{\cos\phi}
    \\
    \ctg\phi &= \frac{\cos\phi}{\sin\phi}
\end{align*}

\subsubsection*{Формула Муавра}

\begin{definition}
    \textit{Формулой Муавра} называется выражение:
    
    \[
        (\cos \phi + i \sin \phi)^n = \cos n\phi + i \sin n\phi,\ n \in \Z
    \]
\end{definition}

С помощью формулы Муавра можно находить натуральную степень комплексного числа:

\[
    z^n = |z|^n (\cos n\phi + i \sin n\phi) = (|z|(\cos \phi + i \sin \phi))^n
\]

\subsubsection*{Натуральный корень из комплексного числа}

Решим уравнение $z^n = w$

\begin{enumerate}
    \item $w = 0 \Ra z = 0$
    \item \begin{align*}
            &w \neq 0 \Ra w = |w|(\cos \psi + i \sin \psi)
            \\
            &z = |z|(\cos \phi + i \sin \phi)
            \\
            &z^n = |z|^n(\cos n\phi + i \sin n\phi)
            \\
            &\Ra |z| = \sqrt[n]{|w|},\ n\phi = \psi + 2\pi k,\ k \in \Z
            \\
            &\phi = \frac{\psi + 2\pi k}{n},\ k = \{0, 1, \dots n - 1\}
        \end{align*}
\end{enumerate}

\section{Пределы}

\subsection{Дополнительные свойства действительных чисел}

\subsubsection*{Плотность множества рациональных чисел в множестве действительных}

\begin{proposition}
    Между любыми двумя неравными действительными числами найдётся рациональное.
    \[
    	(\forall a, b \in \R\ |\ a < b)(\exists c \in \Q)\ |\ a < c < b
    \]
\end{proposition}

\begin{proof}
    По определению действительных чисел
    $$
    a := \{[p_n; q_n]_\Q\}_{n = 1}^\infty,\ b := \{[r_n; s_n]_\Q\}_{n = 1}^\infty
    $$
    $r_1 - q_1 > 0;\ r_1 > q_1 \Ra c:= \frac{r_1 + q_1}{2} \Ra a \le
    q_n \le q_1 < c < r_1 \le r_n \le b$
\end{proof}

\subsubsection*{Равномощность}

\begin{definition}
    Множества $A$ и $B$ называются \textit{равномощными},
    если существует биекция из $A$ в $B$. Обозначается как
    $A \simeq B$
\end{definition}

\subsubsection*{Счётность}

\begin{definition}
    Множство $A$ называется \textit{счётным}, если оно равномощно $\N$
\end{definition}

\subsubsection*{Теорема Кантора}

\begin{proposition}
    $\Q$ счётно, $R$ - несчётно
\end{proposition}

\begin{proof}
    По определению рационального числа, $(\forall r \in \Q)\ r = \frac{m}{n}, m \in \Z, n \in \N$. То есть число полностью задаётся парой $(m, n)$. Отсюда
    \[
    	\Q \subset \Z \times \N
    \]
    Построим таблицу, где номер столбца будет обозначать числитель,
    а номер строки --- знаменатель рационального числа. 

    \begin{tabular}{|c|c|c|c|c|}
        \hline
                     & 0 & 1 & -1 & 2\\
        \hline
        1 &        1 & 2 & 3 & 5\\
        \hline
        2 & $\times$ & 4 & 6 & \\
        \hline
        3 & $\times$ & 7 & & \\
        \hline
        4 & $\times$ & & & \\
        \hline
    \end{tabular}

    Пронумеруем все клетки по диагонали (как в таблице). Так мы построили
    биекцию между множеством $\Q$ и множеством $\N$. Значит, по определению
    $\Q$ является счётным.

    При помощи функций несложно показать, что $\R \simeq [0; 1)$. Предположим, что $[0; 1)$ - счётно, то есть $[0, 1) = \{x_1, x_2, \dots\}$
    Запишем каждое число в виде десятичной дроби:
    \begin{align*}
		&{x_1 = 0, \alpha_{11} \alpha_{12} \alpha_{13}}
        \\
        &{x_2 = 0, \alpha_{21} \alpha_{22} \alpha_{23}}
        \\
        &{x_3 = 0, \alpha_{31} \alpha_{32} \alpha_{33}}
        \\
        &\vdots
    \end{align*}
    Рассмотрим число $\gamma = 0,\alpha_{11}\alpha_{22}\alpha_{33}\dots$. Сдвинем циклически на 1 назад каждую цифру числа (т.е. $\alpha'_{ii} = \alpha_{ii} - 1$ если $> 0$, иначе $\alpha'_{ii} = 9$) и посмотрим на число $\gamma'$
    \[
    	\gamma' = 0,\alpha'_{11}\alpha'_{22}\alpha'_{33}\dots
    \]
    Утверждение - данное число никогда не встречалось в таблице.
    Действительно, для любого $x_m,\ m \in \N$ они будут различны 
    в $\alpha_{mm}$ знаке. То есть предположение неверно и
    $\R \gtrsim \N$
\end{proof}

\begin{definition}
    Множество, равномощное $\R$, называется множеством мощности
    континуума.
\end{definition}

\begin{definition}
    Если $A \subset \R$ --- ограниченное сверху множество, то число

    $M \in \R\ |\ (\forall a \in A)\ a \le M$ называется \textit{верхней гранью} множества $A$.
    Наименьшая из верхних граней называется 
    \textit{точной верхней гранью}, обозначается как $\sup A$
    (supremum)
\end{definition}

\begin{definition}
    Если $A \subset \R$ --- ограниченное снизу множество, то число
    
    $m \in \R\ |\ (\forall a \in A)\ a \ge m$ называется \textit{нижней гранью} множества $A$.
    Наибольшая из нижних граней называется \textit{точной нижней гранью},
    обозначается как $\inf A$ (infinum)
\end{definition}

\begin{definition}
    $c = \sup E \overset{def}{\lra} 
    \System{
    &(\forall x \in E)\ x \le c 
    \\ 
    &(\forall \eps > 0)(\exists x \in E)\ c - \eps < x}
    $

    $\sup E = \max E \lra c \in E$
\end{definition}

\begin{theorem} (О существовании точной верхней (нижней) грани).
    Любое непустое ограниченное сверху (снизу) множество действительных чисел имеет точную верхнюю (нижнюю) грань.
\end{theorem}

\begin{proof}
    Пусть $E \subset \R$ - ограниченное сверху множество. Обозначим через
    $B$ множество всех верхних граней множества $E$ ($B \neq \emptyset$).
    Тогда $A := \R \backslash B$.

    Множество $E$ - непустое $\lra \exists x \in E$. Это значит, что
    \[
    	(\forall a \in \R,\ a < x) \Ra a \in A
    \]
    То есть и $A$ - непустое множество. При этом
    \[
    	(\forall b \in B) (\forall l \in \R \such l \ge b) \Ra l \in B
    \]
    В итоге имеем, что
    \[
    	A, B \subset \R;\ A \cap B = \emptyset;\ (\forall a \in A)
        (\forall b \in B)\ a < b
    \]
    Значит, по свойству полноты множества $\R$:
    \[
    	(\exists c \in \R)(\forall a \in A, b \in B)\ a \le c \le b
    \]
    Докажем, что $c = \sup E$:
    
    Разберёмся с первой частью ($c \in B$): предположим обратное. Тогда
    \[
        c \notin B \Ra (\exists x \in E)\ x > c
    \]
    Рассмотрим число $\frac{c + x}{2}$:
    \[
        c < \frac{c + x}{2} < x
    \]
    Так как $\frac{c + x}{2} < x$, то $\frac{c + x}{2} \in A \Ra$ $c < a$, что противоречит его определению ($c \ge a$).
    
    Утверждение, что $c$ --- наименьший элемент множества $B$
    доказывается также, от противного: пусть $(\exists c' \in B)
    \ c' < c$. Тогда $(\exists b = c' \in B)\ b < c$, противоречие ($c \le b$).
\end{proof}