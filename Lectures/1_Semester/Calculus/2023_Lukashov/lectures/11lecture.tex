\begin{theorem} (Первая теорема Вейерштрасса о непрерывных на отрезке функциях)
	Если $f$ непрерывна на $[a; b]$, то она ограничена на $[a; b]$
\end{theorem}

\begin{proof}
	Докажем от противного. Пусть $f$ - неограничена сверху (снизу аналогично). Это означает
	\[
		(\forall n \in \N)(\exists x_n \in [a; b])\ f(x_n) > n
	\]
	
	Получим $\{x_n\}_{n = 1}^\infty \subset [a; b],\ a \le x_n \le b$. По теореме Больцано-Вейерштрасса
	\[
		\exists \{x_{n_k}\}_{k = 1}^\infty,\ \liml_{k \to \infty} x_{n_k} = x_0
	\]
	\[
		a \le x_{n_k} \le b \Ra a \le x_0 \le b
		\Ra \liml_{k \to \infty} f(x_{n_k}) = f(x_0)
	\]

	Значит, $f(x_{n_k})$ ограничена по свойству сходящейся последовательности.
	
	При этом $f(x_{n_k}) > n_k \Ra \{f(x_{n_k})\}_{k = 1}^{\infty}$ неограничена.
	Противоречие.
\end{proof}

\begin{theorem} (Вторая теорема Вейерштрасса о непрерывных на отрезке функциях)

	Если $f$ непрерывна на $[a; b]$, то она достигает на $[a; b]$
	своих точных верхней и нижней граней. То есть
	\[
		(\exists x', x'' \in [a; b])\ f(x') = \inf\limits_{x \in [a; b]} f(x),\ f(x'') = \sup\limits_{x \in [a; b]} f(x)
	\]
\end{theorem}

\begin{proof}
	По определению минимума
	\[
		m := \inf\limits_{x \in [a; b]} f(x) \Ra (\forall \eps > 0)
		(\exists x \in [a; b])\ m \le f(x) < m + \eps
	\]
	Построим подпоследовательность через выбор $\eps$:
	\begin{align*}
		&\eps := 1 & &m \le f(x_1) < m + 1
		\\
		&\eps := 1/2 & &m \le f(x_2) < m + 1/2
		\\
		&\dots & &\dots
		\\
		&\eps := 1/n & &m \le f(x_n) < m + 1/n
		\\
		&\dots & &\dots
	\end{align*}

	Получили ограниченную последовательность $\{x_n\}_{n = 1}^\infty$.
	По теореме Больцано-Вейерштрасса:
	\[
		\exists \{x_{n_k}\}_{k = 1}^\infty \subset [a; b] :
		\liml_{k \to \infty} x_{n_k} = x_0 \in [a; b] \Ra \liml_{k \to \infty}
		f(x_{n_k}) = f(x_0)
	\]
	Так как	$(\forall n \in \N)\ m \le f(x_n) < m + \frac{1}{n} \Ra$

	\[
	(\forall k \in \N)\ m \le f(x_{n_k}) < m + \frac{1}{n_k}
	\Ra \liml_{k \to \infty} f(x_{n_k}) = m \Ra f(x_0) = m
	\]
\end{proof}

\begin{theorem} (Больцано-Коши о промежуточных значениях) \\
	Если $f$ непрерывна на $[a; b]$ и $y_1, y_2$ - два её
	произвольных значения:
	\[
	(\exists x_1, x_2\ a \le x_1 < x_2 \le b)\ \{f(x_1),\ f(x_2)
	\} = \{y_1,\ y_2\} \Ra (\forall \gamma \in (y_1, y_2))
	(\exists c \in (x_1, x_2))\ f(c) = \gamma
	\]
\end{theorem}

\begin{proof}.

\begin{enumerate}
	\item 
	Обратим внимание, что порядок $x_1, x_2$ и $y_1, y_2$ не связан
	(возможно, что $f(x_1) = y_2$).
	Для упрощения доказательства сначала рассмотрим частный случай:
	$\gamma = 0$, т.к. $\gamma \in (y_1, y_2) \Ra y_1 < \gamma
	< y_2 \Ra f(x_1) \cdot f(x_2) < 0$

	Обозначим за $[a_1, b_1] := [x_1, x_2]$. Поделим $[a_1, b_1]$
	пополам и обозначим через $[a_2, b_2]$ ту половину, на которой
	$f(b_2) \cdot f(a_2) < 0$. 
	
	Если $f\left(\frac{a_1 + b_1}{2}\right)
	= 0$, то все доказано. Продолжаем этот процесс. Тогда он либо
	оборвется, т.е. ч.т.д., либо получим систему стягивающихся
	отрезков

	\[
	\{[a_n, b_n]\}_{n = 1}^{\infty}\ 
	(b_n - a_n = \frac{x_2 - x_1}{2^{n - 1}})
	\Ra (\exists c \in \R)(\forall n \in \N)(c \in [a_n, b_n])
	\Ra 
	\]

	\[
	\liml_{n \to \infty} a_n = \liml_{n \to \infty} b_n
	= c
	\]

`	Из определения непрерывности по Гейне следует:
	\[
	\liml_{n \to \infty} a_n = \liml_{n \to \infty} b_n
	= c \Ra \liml_{n \to \infty} f(a_n) = f(c) =
	\liml_{n \to \infty} f(b_n) 
	\]

	Из свойства с неравенствами пределов следует:

	\[
	\liml_{n \to \infty} \underbrace{f(a_n) \cdot f(b_n)}_{< 0}
	= f^2 (c) \le 0 \Ra f(c) = 0
	\]
	
	\item
	Теперь докажем $(\forall \gamma \in (y_1, y_2))$, т.е. общий случай:

	Рассмотрим $F(x) = f(x) - \gamma$
	
	$F$ непрерывна, значит,
	по доказанному $(\exists c \in (x_1, x_2))\ F(c) = 0 \lra
	f(c) = \gamma$
\end{enumerate}
\end{proof}

\subsubsection*{Промежутки}

\begin{definition}
	Множество $I \subset \bar{\R}$ называется \textit{промежутком},
	если $(\forall x_1 < x_2)\ \{x_1, x_2\} \subset I \Ra
	[x_1; x_2] \subset I$. Если таких $x_1, x_2$ не существует
	(то есть $I = \emptyset$ либо точка), то $I$ называется
	\textit{вырожденным} промежутком
\end{definition}

\begin{lemma}
	$I$ - невырожденный промежуток $\lra$
	$\left(\exists a < b,\ \{a, b\} \subset \R\right)$
	такие, что
	\[
		I = \left[
		\begin{aligned}
			&(-\infty; +\infty)
			\\
			&(-\infty; a)
			\\
			&(b; +\infty)
			\\
			&(-\infty; a]
			\\
			&[b; +\infty)
			\\
			&[a; b]
			\\
			&(a; b]
			\\
			&[a; b)
			\\
			&(a; b)
		\end{aligned}
		\right.
	\]
\end{lemma}

\begin{proof}
	Здесь приведено доказательство двух возможных случаев. Остальные - аналогично.
	
	Пусть $I$ - невырожденный промежуток, который неограничен сверху и ограничен снизу. Тогда $\exists a := \inf I$. Сам по себе $I$ может оказаться каким угодно, но точно верно, что (так как $I \subset \bar{\R}$)
	\[
		I \subset [a; +\infty)
	\]
	Выберем $\forall x_0 \in (a; +\infty)$. Из этого следует, что
	\begin{align*}
		\exists x_2 \in I \cap (x_0; +\infty) \text{, иначе } I \text{ ограничено сверху}
		\\
		\exists x_1 \in I \cap (a; x_0) \text{, иначе } \inf I \text{ определен неверно}
	\end{align*}
	А значит и $[x_1; x_2] \subset I$, то есть $x_0 \in I$. Следовательно,
	$(a;+\infty) \subset I$. Ну а из этого уже либо $I = (a; +\infty)$, либо $I = [a; +\infty)$, в зависимости от достижимости инфинума.
\end{proof}

\begin{lemma}
	Если $f$ непрерывна на промежутке $I$, то её множество
	значений $f(I) = \{f(x) : x \in I\}$ - промежуток (возможно
	вырожденный)
\end{lemma}

\begin{proof}
	Будем считать, что $f$ - непостоянна.
	(Иной случай тривиален)
	
	Рассмотрим $(\forall y_1 < y_2,\ \{y_1, y_2\}
	\subset f(I))$.
	Это значит, что
	\[
		(\exists x_1 < x_2,\ \{x_1, x_2\} \subset I)
		\ \{f(x_1), f(x_2)\} = \{y_1, y_2\}\ (\times)
	\]
	Так как мы работаем с промежутком, то $[x_1; x_2]
	\subset I$, $f$ непрерывна на $[x_1; x_2] \Ra
	(\forall c \in (y_1; y_2))(\exists d \in (x_1; x_2))
	\ f(d) = c \Ra (y_1, y_2) \subset f(I)\ (*)$ по теореме
	Больцано-Коши. А это означает, что
	\[
		(\times)\wedge(*) \Ra [y_1; y_2] \subset f(I)
	\]
\end{proof}

\begin{example}
	\[
		f(x) = \System{
			&x,\ x \in \Q \\
			&-x,\ x \in R \bs \Q
		}
		\ \ (\forall a \in \R)\ f([-a, a]) = [-a, a]
	\]
\end{example}

\begin{note}
	Существуют такие функции, что $(\forall (a, b))\ f((a, b)) = \R$
\end{note}

\begin{lemma} \label{for_back}
	Пусть $f$ - монотонна и непостоянна на промежутке $I$. Тогда $f$
	непрерывна на $I$ тогда и только тогда, когда $f(I)$ -
	промежуток.
\end{lemma}

\begin{proof}
	Нужно доказать только достаточность, остальное следует
	из предыдущей леммы.
	
	Пусть $f(I)$ - промежуток. Пойдем от противного. Предположим, что $f$ -
	разрывная, $x_0$ - не концевая точка $I$ и $f$ имеет точку
	разрыва в $x_0$. Будем считать, что $f$ - невозрастающая на $I$.
	Тогда
	\[
		f(x_0 - 0) > f(x_0 + 0)
	\]
	Рассмотрим $(\forall x < x_0,\ x \in I) \Ra f(x) \ge f(x_0 - 0)$.
	Аналогично $(\forall x > x_0,\ x \in I) \Ra f(x) \le f(x_0 + 0)$.
	Отсюда следует, что
	\[
		f(I) \subset (-\infty; f(x_0 + 0)]
		\cup \{f(x_0)\} \cup [f(x_0 - 0); +\infty) \Ra
	\]
	\[
		f(x_0 - 0) \ge f(x_0) \ge f(x_0 + 0)
	\]
	Причем хотя бы одно неравенство строгое (для существования
	разрыва). Значит хотя бы один из интервалов
	$(f(x_0 + 0), f(x_0)), (f(x_0), f(x_0 - 0))$ не вырожден.
	Пусть это будет $(f(x_0), f(x_0 - 0))$. Тогда $y_1 
	= f(x_0),\ y_2 \in [f(x_0 - 0), +\infty)$ дают противоречие
	с $f(I)$ - промежуток, т.к. $(f(x_0), f(x_0 - 0)) \subset 
	[y_1, y_2] \subset I$, но $(f(x_0), f(x_0 - 0)) \cap I
	= \emptyset$.
	
	Пусть теперь $x_0$ - концевая точка, например,
	$x_0 = \inf I$. Раз $f$ - невозрастающая, то
	\[
		\exists f(x_0 - 0) > f(x_0)
	\]
	Получаем $f(I) \subset \{f(x_0)\} \cup [f(x_0 - 0); +\infty)$. Взяв $y_1$ и $y_2$ из разных частей, снова получим противоречие.
\end{proof}

\begin{definition}
	Если $f$ инъективно на $X$, то на $f(X)$ определено
	обратное отображение $f^{-1}$ так, что $(\forall x \in X)
	\ f^{-1} (f(x)) = x,\ (\forall y \in f(X))\ f(f^{-1} (y)) = y$
\end{definition}

\begin{theorem} \label{inverse_function} (Теорема об обратной функции) 
	Если $f$ непрерывна и строго монотонна на промежутке $I$,
	то на промежутке $f(I)$ определена обратная функция
	$f^{-1}$,  непрерывная на $f(I)$ и строго монотонная в
	том же смысле, что и $f$.
\end{theorem}

\begin{proof}
	Будем рассматривать такую $f$, что $(\forall x_1 < x_2,\  
	x_1, x_2 \in I) \Ra f(x_1) > f(x_2)$. Положим
	\begin{align*}
		y_1 := f(x_1)
		\\
		y_2 := f(x_2)
	\end{align*}
	То есть
	\begin{align*}
		f^{-1}(y_1) := x_1
		\\
		f^{-1}(y_2) := x_2
	\end{align*}
	
	$f^{-1}(y_2) > f^{-1}(y_1)$, то есть $f^{-1}$ монотонно убывает.
	
	По лемме \ref{for_back} $f(I)$ - промежуток.
	А значит, $f^{-1}$ определена на промежутке и при
	этом строго монотонна. Следовательно, по последней лемме
	$f^{-1}$ - непрерывна на $f(I)$.
\end{proof}

\subsection{Непрерывность элементарных функций}

\begin{enumerate}
	\item $y = x^n,\ n \in \N,\ n$ - нечётное
	
	Возрастает на $(-\infty; +\infty)$, непрерывна по 3й лемме
	
	Обратная: $f^{-1}(y) := \sqrt[n]{x}$
	
	\item $y = x^n,\ n \in \N,\ n$ - чётное
	
	Возрастает на $[0; +\infty)$, непрерывна по 3й лемме
	
	$f^{-1}(y) = \sqrt[n]{x}$
	
	\item $y = x^r,\ r \in \Q$
	
	Определена и непрерывна на $(0; +\infty)$
\end{enumerate}

\subsubsection*{Тригонометрические функции}

\begin{lemma} \label{for_trig}
	$\forall x \in (0; \frac{\pi}{2}) \Ra \sin x < x < \tg x$
\end{lemma}

\begin{center}
	\begin{tikzpicture}[scale=1]
		% Axis
		\coordinate (y) at (0,3);
		\coordinate (x) at (3.6,0);
		\draw[<->] (y) -- (0,0) --  (x);
		\draw (-3,0) -- (0,0) --  (0,-3);
		
		\path
		coordinate (c1) at +(2.5, 3)
		coordinate (c2) at +(2.5, -3)
		coordinate (top) at (4.8,3.6);
		
		\draw (c1) -- (c2);
		\draw (2.1, 1.35) -- (2.1, 0) node[below] {$A$};
		\draw (0,0) node[above left] {$O$} -- (2.5, 1.6) node[above right] {$B$};
		
		\filldraw[black] (2.5, 0) circle (1.2pt) node[below right] {$C$};
		\filldraw[black] (2.5, 1.6) circle (1.2pt);
		\filldraw[black] (0, 0) circle (1.2pt);
		\filldraw[black] (2.1, 0) circle (1.2pt);
		\filldraw[black] (2.1, 1.35) circle (1.2pt) node[yshift=10, xshift=3] {$M$};
		\draw[black] circle(2.5);
		
		\coordinate (a) at (1, 0);
		\coordinate (z) at (0, 0);
		\coordinate (m) at (2.1, 1.35);
		\draw (1, 0.3) node {$x$};
		\pic [draw, ->, angle radius = 0.7cm] {angle = a--z--m};
	\end{tikzpicture}
\end{center}

\begin{proof}
	Рассмотрим рисунок, на котором $x \in (0; \frac{\pi}{2})$. Согласно обозначениям, $\sin x = MA,\ \tg x = BC$. При этом несложно увидеть, что
	\[
		S_{\triangle OMC} < S_{OMC} < S_{\triangle OBC}
	\]
	где $S_{\triangle OMC} = \frac{\sin x}{2}$, $S_{OMC} = \frac{x}{2}$, $S_{\triangle OBC} = \frac{\tg x}{2}$. То есть
	\[
		\frac{\sin x}{2} < \frac{x}{2} < \frac{\tg x}{2} \Ra \sin x < x < \tg x
	\]
\end{proof}