\section{О числах}

\subsection{Натуральные числа}

\begin{definition}
    Неопределяемые понятия:
    \begin{itemize}
        \item "1"
        \item $Sc(\ )$ - следующий элемент
            (от английского слова \textit{successor} - преемник)
    \end{itemize}
    
    Натуральные числа - система $\N$, удовлетворяющая
    следующим аксиомам:
\end{definition}

\subsubsection*{Аксиомы Пеано}

\begin{definition}
    Аксиомы Пеано - это набор бездоказательных
    высказываний, позволяющих на своей основе построить
    всю систему натуральных чисел (т.е. определить все
    элементы, отношения и операции).
\end{definition}

\begin{enumerate}
    \item $\mathbf{1}$ есть натуральное число, то есть
        $\mathbf{1} \in \N$
    \item $(\forall n \in \N)\ \exists!\ Sc(n) \in \N$ - для
        любого числа существует единственное, следующее за ним
    \item ($\forall n \in \N)\ 1 \neq Sc(n)$ - число $1$ не
        является чьим-либо преемником
    \item $(\forall n \in \N)(\forall m \in \N)\ 
        Sc(n) = Sc(m) \Ra n = m$ - то есть $Sc$ инъективна
    \item (Аксиома индукции) $(\forall A
        \subset \N)\ \such 
        \System{
            &(1 \in A)
            \\
            &(\forall n \in A)\ Sc(n) \in A
        }
        \Ra A = \N$
\end{enumerate}

\begin{note}
    Использование аксиомы индукции (метод мат. индукции):

    Чтобы доказать, что $P(n)$ справедливо $(\forall n \in \N)$,
    достаточно проверить два факта:
    \begin{enumerate}
        \item $P(1)$ --- истинно 
        \item $P(n)$ --- истинно $\Ra \ P(Sc(n))$ --- истинно 
    \end{enumerate}
\end{note}

\begin{example}
    Понятно, что данным аксиомам могут соответствовать
    разные модели. Одной из таких является модель Фреге-Рассела:
    
    \[1 := \{\emptyset\}\]
    \[Sc(n) := n \cup \{n\}\]
    То есть:
    $2 = \{\emptyset, \{\emptyset\}\}$,
    $\ 3 = \{\emptyset, \{\emptyset\}, \{\emptyset,\{\emptyset\}\}\}$
\end{example}

\subsubsection*{Сложение}

Чтобы определить операцию сложения, нам необходимо
ввести 2 аксиомы:
\begin{enumerate}
    \item $(\forall m \in \N)\ m + 1 := Sc(m)$
    \item $(\forall n,\ m \in \N)\ m + Sc(n) := Sc(m + n)$
    \begin{note}
        Эта аксиома не интуитивна (ведь мы не определили
        $m + n$), но верна за счёт первой.
        
        База $n = 1$: $m + Sc(1) := Sc(m + 1)$ - всё верно и определено.

        Предположение индукции: $m + Sc(n) = Sc(m + n)$ --- верно.
        Тогда поймём, что и
        $m + Sc(Sc(n)) = Sc(m + Sc(n))$ - также определено и
        верно в силу предыдущего шага
        $\Ra$ аксиома верна и для любых натуральных $m$ и $n$.
        (Несмотря на наличие "доказательной"\ 
        части, данное выражение остаётся аксиомой, иначе базу
        не обосновать)
    \end{note}
\end{enumerate}

Из этих двух аксиом следует, что операция сложения
существует и единственна.

\subsubsection*{Умножение}

Чтобы определить операцию умножения, нам необходимо
ввести 2 аксиомы:
\begin{enumerate}
    \item $(\forall m \in \N)\ m \cdot 1 := m$
    \item $(\forall m \in \N)\ m \cdot Sc(n) := m \cdot n + m$
\end{enumerate}

Из этих двух аксиом следует, что операция сложения существует и единственна.

\begin{example}
    Доказать: $2 \cdot 2 = 4$
    
    Что такое $2$? По определению, $2 := Sc(1)$.
    Аналогично $3 := Sc(2),\ 4 := Sc(3)$. Тогда:
    
    $2 \cdot 2 = 2 \cdot Sc(1) = 2 \cdot 1 + 2 = 2 + Sc(1)
    = Sc(2 + 1) = Sc(Sc(2)) = Sc(3) = 4$, что и требовалось доказать.
\end{example}

\subsubsection*{Отношение строгого порядка на множестве
натуральных чисел (не материал лектора)}

\begin{definition}
    Отношение строгого порядка $<$ на множестве $\N$ определяется как
    \[
        (m < n) \overset{def}{\lra}\ (\exists p \in \N)\ m + p = n
    \]
    Для $>$ аналогично.
\end{definition}

\subsubsection*{Отношение порядка на множестве натуральных чисел}

\begin{definition}
	Отношение порядка $\le$ на множестве $\N$ определяется как

	\[
        (\forall n,\ m \in \N)\ m \le n \overset{def}{\lra}\ 
        (m = n) \vee ((\exists p \in \N)\ n = p + m)
    \]

	Для $\ge$ аналогично.
\end{definition}

\begin{anote}
    Отношение порядка также можно задать следующим образом:
    \[
    	(a \le b) := \big((a = b) \vee (a < b)\big)
    \]
\end{anote}

Из определения становится возможным доказать теорему,
о том, что отношения $\le$ и $\ge$ являются отношениями порядка.
(Для этого необходимо доказать 3 свойства).

\subsubsection*{Свойства операций и отношений на натуральных числах}

\begin{theorem}
    $(\forall m, n, p \in \N)$ верно следующее:

    \begin{itemize}
        \item Сложение
        \begin{itemize}
            \item[I-а).] $m + n = n + m$ (коммутативность)
            \item[I-б).] $(m + n) + p = m + (n + p)$ (ассоциативность)
        \end{itemize}
        \item Умножение
        \begin{itemize}
            \item[II-а).] $m \cdot n = n \cdot m$ (коммутативность)
            \item[II-б).] $(m \cdot n) \cdot p = m \cdot (n \cdot p)$ (ассоциативность)
            \item[II-в).] $m \cdot 1 = 1 \cdot m = m$ (единичный элемент)
        \end{itemize}
        \item Отношение порядка
        \begin{itemize}
            \item[III-а).] $m \le m$ (рефлексивность)
            \item[III-б).] $(m \le n) \wedge (n \le m) \Ra (m = n)$ (антисимметричность)
            \item[III-в).] $(m \le n) \wedge (n \le p) \Ra (m \le p)$ (транзитивность)
            \item[III-г).] $(m \le n) \vee (n \le m)$ - всегда истинное
                выражение (множество $\N$ линейно упорядочено)
        \end{itemize}
        \item Дистрибутивность:

        \begin{itemize}
            \item[I-II).] $m \cdot (n + p) = m \cdot n + m \cdot p$ (сложения и умножения)
            \item[I-III).] $(m \le n) \Ra \ m + p \le n + p$
            \item[II-III).] $(m \le n) \Ra \ m \cdot p \le n \cdot p$
        \end{itemize}
        \item Свойство Архимеда
        \[
            (\forall m \in \N)(\forall p \in \N)(\exists n \in \N)\ 
            m \cdot n > p
        \]
    \end{itemize}
\end{theorem}


\subsection{Целые числа}

\begin{definition}
    \textit{Множеством целых чисел} называется множество
    
    \[\Z := \N \cup \{0\} \cup \{-n\ :\ n \in \N\}\]
\end{definition}

Из определения сразу следует, что $\N \subset \Z$

\subsubsection*{Сложение}

Рассмотрим $\forall m, n \in \N$ и доопределим операции:

\begin{enumerate}
    \item $m + n \Ra$ сложение происходит также, как и с натуральными числами.
    \item $(-m) + (-n) := -(m + n)$
    \item $m + (-n) := \System{&{p \in \N\ :\ n + p = m, \text{ если } m > n} \\ 
                              &{0, \text{ если } m = n} \\ 
                              &{-p,\ p \in \N\ :\ m + p = n, \text{ если } m < n}}$
\end{enumerate}

\subsubsection*{Умножение}

Рассмотрим $\forall m, n \in \N$:

\begin{enumerate}
    \item $m \cdot n \Ra$ умножение происходит также, как и с натуральными числами.
    \item $(-m) \cdot (-n) := m \cdot n$
    \item $m \cdot (-n) := -(m \cdot n)$
    \item $m \cdot 0 := 0$
\end{enumerate}

\subsubsection*{Свойства операций и отношений на множестве целых чисел}

\begin{theorem}~

    \begin{itemize}
        \item Сложение
        \begin{itemize}
            \item[I).] Все свойства сложения натуральных чисел
                верны и для целых.
            \item[I-в).] $m + 0 := m$ (существование нейтрального
                к сложению числа)
            \item[I-г).] $(\forall m \in \Z)(\exists (-m) \in \Z)
                \ \ m + (-m) = 0$
                (существование обратного числа по сложению)
        \end{itemize}
        \item Умножение
        \begin{itemize}
            \item[II).] Все свойства умножения натуральных чисел
                верны и для целых.
        \end{itemize}
        \item Отношение порядка
        \begin{itemize}
            \item[II-III).] $(\forall m \in \Z) (\forall n \in \Z)
                (\forall p \in \N)\ (m \le n) \Ra
                (m \cdot p \le n \cdot p)$
            \item[IV).] $(\forall m_1 \in \Z)
                (\forall m_2 \in \Z \bs \{0\})
                (\exists n \in \Z)\ n \cdot m_2
                \ge m_1$ (свойство Архимеда)
        \end{itemize}
    \end{itemize}
\end{theorem}

\subsection{Рациональные числа}

\begin{definition}
    \textit{Рациональным числом} называется число вида
    $\frac{m}{n}$, где $m \in \Z, n \in \N$. Его можно
    однозначно задать с помощью упорядоченная пара $(m, n)$.
    Множество рациональных чисел обозначают $\Q$.
\end{definition}

\subsubsection*{Отношение эквивалентности на
множестве рациональных чисел}

\begin{definition}
    Введем отношение: $(m, n)R(p, q) \overset{def}{\lra}
    (m \cdot q = n \cdot p)$
\end{definition}

\begin{proposition}
    Отношение $R$ является отношением эквивалентности
    на множестве рациональных чисел
\end{proposition}

\begin{proof}
    Для доказательства необходимо проверить, что
    выполнены все свойства отношения эквивалентности:
    \begin{enumerate}
        \item Рефлексивность: $(m, n)R(m, n) \overset{def}{\lra}
            (m \cdot n = n \cdot m)$ - верно.
        \item Симметричность: $(m, n)R(p, q) \Ra
            (p, q)R(m, n)$
            
            $(p, q)R(m, n) \lra
            (p \cdot n = q \cdot m) \lra (m \cdot q = n \cdot p)
            \lra (m, n)R(p, q)$.
        \item Транзитивность: 
        $(m, n)R(p, q) \wedge (p, q)R(r, s) \Ra (m, n)R(r, s)$.
        
        Опуская формальности, имеем $2$ равенства: $mq = np$ и $ps = qr$.
        Домножим первое на $s$, а второе на
        $n$: $mqs = nps = psn = qrn \Ra mqs =
        qrn \Ra ms = rn = nr$ - верно.
    \end{enumerate}
    все $3$ свойства верны, а значит $R$ является
    отношением эквивалентности, что и требовалось доказать.
\end{proof}

\subsubsection*{Положительное рациональное число}

\begin{definition}
    \textit{Положительным рациональным числом} называется
    класс эквивалентности в $\N^2$ по отношению $R$
    на множестве $\Q$.
\end{definition}
    
Множество всех положительных чисел обозначается $\Q_+$
    
Тогда множество рациональных чисел задается так:
\[\Q := \Q_+ \cup \{0\} \cup \{-r\ :\ r \in \Q_+\}\]
    
Отсюда также следует, что $\N \subset \Z \subset \Q$ 

\begin{anote}
    Рациональное число определяется как класс
    эквивалентности из-за того факта, что
    $\frac{1}{2} = \frac{2}{4}$ и так далее.
    Определение положительного рационального числа
    через $N^2$ справедливо, так как
    $\N \subset\Z \Ra \N^2 \subset \Z \times \N$
\end{anote}

\subsubsection*{Отношение строгого порядка на множестве
рациональных чисел (не материал лектора)}

\begin{definition}
    Отношение строгого порядка $<$ ($>$) на
    множестве $\Q$ задаётся как
    \[
        \left(\frac{p}{q} < \frac{m}{n}\right)
        := (p \cdot n < m \cdot q)
    \]
\end{definition}

\begin{anote}
    В данном определении всё верно, так как
    $q, n \in \N$ по определению рациональных чисел.
\end{anote}

\subsubsection*{Отношение порядка на множестве
рациональных чисел (не материал лектора)}

\begin{definition}
    Отношение порядка $\le$ ($\ge$) на множестве
    $\Q$ задаётся как и на предыдущих:
    \[
        (a \le b) := \big((a = b) \vee (a < b)\big)
    \]
\end{definition}

\subsubsection*{Операции на множестве рациональных чисел}

Обозначим фигурными скобками $r = \{(m, n)\}$ класс эквивалентности
упорядоченных пар чисел $m, n$ (то есть рациональное число $r$).
Тогда если $r, s \in \Q_+ \ \ (m, n) \in r;\ (p, q) \in s$, то
сложение и умножение определяются так:
\[
    r + s = \{(m, n)\} + \{(p, q)\} :=
    \{(m \cdot q + n \cdot p,\ n \cdot q)\}
\]
\[
    r \cdot s = \{(m, n)\} \cdot \{(p, q)\} :=
    \{(m \cdot p,\ n \cdot q)\}
\]

\subsubsection*{Свойства операций и отношений на
множестве рациональных чисел}
\begin{theorem}~

    \begin{itemize}
        \item Сложение
        \begin{itemize}
            \item[I.)] Все свойства сложения целых чисел
                верны и для рациональных.
        \end{itemize}
        \item Умножение
        \begin{itemize}
            \item[II.)] Все свойства умножения целых чисел
                верны и для рациональных.
            \item[II-г.)] $(\forall p \in \Q)(p \neq 0)\ \exists
                p^{-1} \in \Q :\ p \cdot p^{-1} = 1$
                (существование обратного элемента по умножению)
        \end{itemize}
        \item Отношение порядка
        \begin{itemize}
            \item[III.)] Все свойства отношения порядка целых
                чисел верны и для отношения порядка рациональных
                чисел (хоть определение и отличается от того,
                что используется на целых числах).
        \end{itemize}
    \end{itemize}
\end{theorem}