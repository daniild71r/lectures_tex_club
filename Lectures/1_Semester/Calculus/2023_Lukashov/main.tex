\documentclass[a4paper,12pt]{article}

%%% Работа с русским языком
\usepackage{cmap}					% поиск в PDF
\usepackage{mathtext} 				% русские буквы в формулах
\usepackage[T2A]{fontenc}			% кодировка
\usepackage[utf8]{inputenc}			% кодировка исходного текста
\usepackage[russian,english]{babel}	% локализация и переносы
\usepackage{indentfirst}            % красная строка в первом абзаце
\frenchspacing                      % равные пробелы между словами и предложениями

%%% Дополнительная работа с математикой
\usepackage{amsmath,amsfonts,amssymb,amsthm,mathtools} % пакеты AMS
\usepackage{icomma}                                    % "Умная" запятая

%%% Свои символы и команды
\usepackage{centernot} % центрированное зачеркивание символа
\usepackage{stmaryrd}  % некоторые спецсимволы

\usepackage{ifthen}

\renewcommand{\epsilon}{\ensuremath{\varepsilon}}
\renewcommand{\phi}{\ensuremath{\varphi}}
\renewcommand{\kappa}{\ensuremath{\varkappa}}
\renewcommand{\le}{\ensuremath{\leqslant}}
\renewcommand{\leq}{\ensuremath{\leqslant}}
\renewcommand{\ge}{\ensuremath{\geqslant}}
\renewcommand{\geq}{\ensuremath{\geqslant}}
\renewcommand{\emptyset}{\ensuremath{\varnothing}}

\DeclareMathOperator{\sgn}{sgn}
\DeclareMathOperator{\ke}{Ker}
\DeclareMathOperator{\im}{Im}
\DeclareMathOperator{\re}{Re}

\newcommand{\N}{\mathbb{N}}
\newcommand{\Z}{\mathbb{Z}}
\newcommand{\Q}{\mathbb{Q}}
\newcommand{\R}{\mathbb{R}}
\newcommand{\Cm}{\mathbb{C}}
\newcommand{\F}{\mathbb{F}}
\newcommand{\id}{\mathrm{id}}

\newcommand{\imp}[2]{
	(#1\,\,$\ra$\,\,#2)\,\,
}
\newcommand{\System}[1]{
	\left\{\begin{aligned}#1\end{aligned}\right.
}
\newcommand{\Root}[2]{
	\left\{\!\sqrt[#1]{#2}\right\}
}

\renewcommand\labelitemi{$\triangleright$}

\let\bs\backslash
\let\Lra\Leftrightarrow
\let\lra\leftrightarrow
\let\Ra\Rightarrow
\let\ra\rightarrow
\let\La\Leftarrow
\let\la\leftarrow
\let\emb\hookrightarrow

%%% Перенос знаков в формулах (по Львовскому)
\newcommand{\hm}[1]{#1\nobreak\discretionary{}{\hbox{$\mathsurround=0pt #1$}}{}}

%%% Работа с картинками
\usepackage{graphicx}    % Для вставки рисунков
\setlength\fboxsep{3pt}  % Отступ рамки \fbox{} от рисунка
\setlength\fboxrule{1pt} % Толщина линий рамки \fbox{}
\usepackage{wrapfig}     % Обтекание рисунков текстом

%%% Работа с таблицами
\usepackage{array,tabularx,tabulary,booktabs} % Дополнительная работа с таблицами
\usepackage{longtable}                        % Длинные таблицы
\usepackage{multirow}                         % Слияние строк в таблице

%%% Теоремы
\theoremstyle{plain}
\newtheorem{theorem}[equation]{Theorem}
\newtheorem{lemma}[equation]{Lemma}
\newtheorem{proposition}[equation]{Proposition}
\newtheorem*{exercise}{Exercise}
\newtheorem*{problem}{Problem}

\theoremstyle{definition}
\newtheorem{definition}[equation]{Definition}
\newtheorem*{corollary}{Corollary}
\newtheorem*{note}{Note}
\newtheorem*{reminder}{Reminder}
\newtheorem*{example}{Example}

\theoremstyle{remark}
\newtheorem*{solution}{Solution}

%%% Оформление страницы
\usepackage{extsizes}     % Возможность сделать 14-й шрифт
\usepackage{geometry}     % Простой способ задавать поля
\usepackage{setspace}     % Интерлиньяж
\usepackage{enumitem}     % Настройка окружений itemize и enumerate
\setlist{leftmargin=25pt} % Отступы в itemize и enumerate

\geometry{top=25mm}    % Поля сверху страницы
\geometry{bottom=30mm} % Поля снизу страницы
\geometry{left=20mm}   % Поля слева страницы
\geometry{right=20mm}  % Поля справа страницы

\setlength\parindent{0pt}         % Устанавливает длину красной строки 0pt
\linespread{1.3}                  % Коэффициент межстрочного интервала
%\setlength{\parskip}{0.5em}      % Вертикальный интервал между абзацами
%\setcounter{secnumdepth}{0}      % Отключение нумерации разделов
%\setcounter{section}{-1}         % Нумерация секций с нуля
\usepackage{multicol}			  % Для текста в нескольких колонках
\usepackage{soulutf8}             % Модификаторы начертания

%%% Содержаниие
\usepackage{tocloft}
\tocloftpagestyle{main}
%\setlength{\cftsecnumwidth}{2.3em}
%\renewcommand{\cftsecdotsep}{1}
%\renewcommand{\cftsecpresnum}{\hfill}
%\renewcommand{\cftsecaftersnum}{\quad}

%%% Шаблонная информация для титульного листа
\newcommand{\CourseName}{Introduction to Calculus}
\newcommand{\FullCourseNameFirstPart}{\so{INTRODUCTION TO CALCULUS}}
\newcommand{\SemesterNumber}{I}
\newcommand{\LecturerInitials}{Nikolay Gusev}
\newcommand{\CourseDate}{fall 2022}
\newcommand{\AuthorInitials}{Danil Klishch}
\newcommand{\VKLink}{https://vk.com/dan.klishch}
\newcommand{\GithubLink}{https://github.com/daniild71r/lectures_tex_club}

%%% Колонтитулы
\usepackage{titleps}
\newpagestyle{main}{
	\setheadrule{0.4pt}
	\sethead{\CourseName}{}{\hyperlink{intro}{\;Back to table of contents}}
	\setfootrule{0.4pt}                       
	\setfoot{MIPT, \CourseDate}{}{\thepage} 
}
\pagestyle{main}

%%% Нумерация уравнений
\makeatletter
\def\eqref{\@ifstar\@eqref\@@eqref}
\def\@eqref#1{\textup{\tagform@{\ref*{#1}}}}
\def\@@eqref#1{\textup{\tagform@{\ref{#1}}}}
\makeatother                      % \eqref* без гиперссылки
\numberwithin{equation}{subsection}  % Нумерация вида (номер_секции).(номер_уравнения)
\mathtoolsset{showonlyrefs=false} % Номера только у формул с \eqref{} в тексте.

%%% Гиперссылки
\usepackage{hyperref}
\usepackage[usenames,dvipsnames,svgnames,table,rgb]{xcolor}
\hypersetup{
	unicode=true,            % русские буквы в раздела PDF
	colorlinks=true,       	 % Цветные ссылки вместо ссылок в рамках
	linkcolor=black!15!blue, % Внутренние ссылки
	citecolor=green,         % Ссылки на библиографию
	filecolor=magenta,       % Ссылки на файлы
	urlcolor=NavyBlue,       % Ссылки на URL
}

%%% Графика
\usepackage{tikz}        % Графический пакет tikz
\usepackage{tikz-cd}     % Коммутативные диаграммы
\usepackage{tkz-euclide} % Геометрия
\usepackage{stackengine} % Многострочные тексты в картинках
\usetikzlibrary{angles, babel, quotes}

%%% Мои кастомные команды
\newcommand{\df}[2]{\begin{definition}\textit{#1} -- #2\end{definition}}

\newcommand{\abs}[1]{\left\lvert #1 \right\rvert}
\newcommand{\grad}{\mathrm{grad}\:}
\newcommand{\dlta}{\text{d}}
\newcommand{\thus}{\;\Rightarrow\;}
\newcommand{\isconst}{=\text{const}}

\newcommand{\defeq}{\vcentcolon=}
\newcommand{\defev}{\stackrel{\Delta}{\Longleftrightarrow}}
\newcommand{\deriv}[3][1]{%
	\ifthenelse{#1>1}{%
		\frac{\dlta^{#1} {#2}}{\dlta {#3}^{#1}}
	}{%
		\frac{\dlta {#2}}{\dlta {#3}}
	}%
}
\newcommand{\vc}[3]{
	\ensuremath{\begin{pmatrix} #1 \\ #2 \\ #3 \end{pmatrix}}
}
\newcommand{\vb}[2]{
	\ensuremath{\begin{pmatrix} #1 \\ #2 \end{pmatrix}}
}
\newcommand{\fnis}[1]{#1{:}\;}
\newcommand{\RR}{\mathbb{R}}
\newcommand{\NN}{\mathbb{N}}
\newcommand{\sconstr}{\;\vert\;}
\newcommand{\diag}{{\rm diag}}

\newcommand{\floor}[1]{\left\lfloor#1\right\rfloor}
\newcommand{\ceil}[1]{\left\lceil#1\right\rceil}


%\includeonly{lectures/lect05,lectures/lect06}  % Скомпилировать только часть лекций

\begin{document}
    \begin{titlepage}
	\clearpage\thispagestyle{empty}
	\centering
	
	\textit{Федеральное государственное автономное учреждение \\
		высшего образования}
	\vspace{0.5ex}
	
	\textbf{Московский физико-технический институт
    \\
    (национальный исследовательский университет)
    \\
     КЛУБ ТЕХА ЛЕКЦИЙ}
	\vspace{20ex}
	
	\rule{\linewidth}{0.5mm}
	{\textbf{\FullCourseNameFirstPart}}
	\\
	{\textbf{\FullCourseNameSecondPart}}
	\rule{\linewidth}{0.5mm}
	
	\SemesterNumber\ СЕМЕСТР
	\\
	Физтех-школа: \textit{\SchoolName}
	\\
	Направление: \textit{\TrackName}
	\\
	Лектор: \textit{\LecturerInitials}
	\vspace{1ex}
	
	\begin{figure}[!ht]
		\centering
		\includegraphics[width=0.6\textwidth]{logo_LTC}
	\end{figure}
\begin{flushright}
	\noindent
	Автор(ы): \href{\VKLink}{\textit{\AutherInitials}}
	\\
	\href{\OverleafLink}{\textit{Проект на overleaf}}
	\\
	\href{\GithubLink}{\textit{Проект на github}} 
\end{flushright}
	
	\vfill
	Долгопрудный, \CourseDate\ год.
	\pagebreak
	
\end{titlepage}
    \newpage
    \hypertarget{intro}{}
    \tableofcontents
    
    \linespread{1}
    \selectfont
    
    \newpage

    % 2023 reTeXed Lectures:
    % 01.02.23

\subsection{Преобразование Абеля}
\begin{definition}
    Пусть $\{a_n\}$, $\{b_n\}$ --- (комлексные) последовательности, $m \in \N$, и пусть $A_n = \sum_{k=1}^n a_k$ для всех $n \in \N$. Тогда $a_k = A_k - A_{k-1}$ ($A_0 = 0$), и, значит,
    \[
        \sum_{k=m}^{n} a_k b_k = \sum_{k=m}^n (A_k - A_{k - 1})b_k = \sum_{k=m}^n A_k b_k - \sum_{k = m - 1}^{n - 1} A_k b_{k + 1}.
    \]

    Справедливо \emph{преобразование Абеля}:
    \[
        \sum_{k=m}^n a_k b_k = A_n b_n - A_{m - 1} b_m - \sum_{k = m}^{n - 1} A_k (b_{k + 1} - b_k).
    \]
\end{definition}

\begin{lemma}[Абель]
    Пусть $\{a_n\}$ --- (комплексная) последовательность, $\{b_n\}$ --- монотонная последовательность, и пусть $\forall k \ |A_k| \le M$. Тогда:
    \[
        \left|\sum_{k=m}^n a_k b_k \right| \le 2M(|b_m| + |b_n|).
    \]

    \begin{proof}
        По монотонности $\{b_n\}$ знаки $b_{k + 1} - b_k$ сохраняются, поэтому:
        \footnote{Сумма телескопируется.}
        \[
            \left| \sum_{k=m}^n a_k b_k \right| \le M \left(|b_n| + |b_m| + \left| \sum_{k = m}^{n - 1} (b_{k+1} - b_k) \right| \right) = M \left(|b_n| + |b_m| + |b_n - b_m| \right).
        \]
    \end{proof}
\end{lemma}

\begin{note}
    Пусть $\{b_n\}$ нестрого убывает и неотрицательна, $\widetilde{M} \le A_k \le M$, тогда при $m = 1$ неравенство можно усилить:
    \[
        \widetilde{M} b_1 \le \sum_{k=1}^n a_k b_k \le M b_1.
    \]
\end{note}

\label{abel-lemma}
\begin{lemma}[Абель]
    Пусть $f \in \mathcal{R}[a, b]$, $g$ монотонна на $[a, b]$, и пусть $\forall x \in [a, b] \ \left|\int_a^x f(t) dt\right| \le M$. Тогда:
    \[
        \left|\int_a^b f(x) g(x) dx\right| \le 2M(|g(a)| + |g(b)|).
    \]

    \begin{proof}
        Зафиксируем $\varepsilon > 0$. Положим $I = \int_a^b f(x) dx$. Тогда $\exists \delta > 0 \ \forall (T, \xi) \ (|T| < \delta \rightarrow |\sigma_T (f, \xi) - I| < \frac{\varepsilon}{2})$.

        Выберем одно такое разбиение $T = \{x_i\}_{i=0}^{n}$.

        Пусть $T_k = \{x_i\}_{i=0}^k$ --- соответствующее разбиение $[x_0, x_k], k = 1, \ldots, n$. Числа $\sigma_{T_k} (f, \xi_k)$ и $\int_{x_0}^{x_k} f(x) dx$ лежат\footnote{По критерию Дарбу.} на отрезке $[s_{T_k}(f), S_{T_k}(f)]$, и верно $S_{T_k}(f) - s_{T_k}(f) \le S_T(f) - s_T(f)$.

        \[
            I - \frac{\varepsilon}{2} < \sigma_T (f, \xi) < I + \frac{\varepsilon}{2} ,
        \]

        \[
            I - \frac{\varepsilon}{2} \le s_T(f) \le S_T(f) \le I + \frac{\varepsilon}{2},
        \]

        \[
            \left|\sigma_{T_k} (f, \xi_k) - \int_{x_0}^{x_k} f(x) dx\right| \le \varepsilon.
        \]

        Положим $A_k = \sum_{i = 1}^k f(c_i) \Delta x_i.$ Тогда $A_k = \sigma_{T_k}(f, \xi_k)$ и, значит, из последнего неравенства $|A_k| \le M + \varepsilon$.
        Применим лемму \ref{abel-lemma} для $a_k = f(c_k) \Delta x_k, b_k = g(c_k)$, получим
        \[
            \left|\sum_{k=1}^n f(c_k) g(c_k) \Delta x_k\right| \le 2(M + \varepsilon)(|g(c_1)| + |g(c_n)|).
        \]

        Неравенство верно для любого набора отмеченных точек, в том числе и $c_1 = a$, $c_n = b$.
    \end{proof}
\end{lemma}

\begin{note}
    Предельным переходом по мелкости разбиения в случае $c_1 = a, c_n = b$ получим оценку:

    \[
        \left|\int_a^b f(x) g(x) dx\right| \le 2(M + \varepsilon)\left(|g(a)| + |g(b)|\right).
    \]

    Перейдём к $\varepsilon \rightarrow 0$:
    \[
        \left|\int_a^b f(x) g(x) dx\right| \le 2M\left(|g(a)| + |g(b)|\right).
    \]
\end{note}


\begin{problem}[формула Бонне]
    Пусть $f \in R[a, b]$, $g$ нестрого убывает и неотрицательна на $[a, b]$. Доказать, что $\exists c \in [a, b]$, такое, что выполняется

    \[
        \int_a^b f(x) g(x) dx = g(a) \int_a^c f(x) dx.
    \]
\end{problem}

Выбором $a$ из формулы Бонне можно получить \emph{вторую интегральную теорему о среднем}.

\section{Несобственный интеграл Римана}
\subsection{Основные понятия}

\begin{definition}
    Функция $f$ называется \emph{локально интегрируемой по Риману на промежутке $I$}, если $\forall [a, c] \subset I \hookrightarrow f \in \mathcal{R}[a, c]$.

    \example{Всякая непрерывная на промежутке функция локально интегрируема на этом промежутке.}
\end{definition}

\begin{definition}
    Пусть $-\infty < a < b \le +\infty$, и $f$ локально интегрируема на $[a, b)$. Предел
    \[
        \int_a^b f(x) dx \coloneqq \lim_{c \rightarrow b-0} \int_a^c f(x) dx
    \]
    называется \emph{несобственным интегралом (Римана) от $f$ на $[a, b)$}.

    Если предел существует и конечен, то интеграл $\int_a^b f(x) dx$ называется \emph{сходящимся}, иначе --- \emph{расходящимся}.
\end{definition}

\begin{note}
    Пусть $b \in \R$, функция $f$ локально интегрируема и \emph{ограничена} на $[a, b)$. Тогда по свойствам определённого интеграла $f \in \mathcal{R}[a, b]$ (при любом доопределении в точке $b$). В силу непрерывности интеграла с переменным верхним пределом:
    \[
        \lim_{c \rightarrow b - 0} \int_a^c f(x) dx = \int_a^b f(x) dx.
    \]

    Следовательно, несобственный интеграл совпадает с определённым интегралом.

    Поэтому новая ситуация может возникать лишь если:
    \begin{itemize}
        \item $b = +\infty$,
        \item $b \in \R$, $f$ неограничена на $[a, b)$.
    \end{itemize}

    Аналогично определяется несобственный интеграл от $f$ по $(a, b]$, $-\infty \le a < b < +\infty$.
\end{note}

Некоторые свойства переносятся предельным переходом из аналогичных свойств определённого интеграла:

\begin{property}[принцип локализации]
    Пусть $f$ локально интегрируема на $[a, b)$, $a^* \in (a, b)$. Тогда интегралы $\int_{a^*}^b f(x) dx$ и $\int_a^b f(x) dx$ сходятся или расходятся одновременно, и если сходятся, то
    \label{eqn-2-1}
    \begin{equation}
        \int_a^b f(x) dx = \int_a^{a^*} f(x) dx + \int_{a^*}^b f(x) dx
    \end{equation}

    \begin{proof}
        Если $c \in (a, b)$, то по свойству аддитивности определённого интеграла верно:
        \[
            \int_a^c f(x) dx = \int_a^{a^*} f(x) dx + \int_{a^*}^c f(x) dx.
        \]

        Поэтому пределы $\lim_{c \rightarrow b - 0} \int_{a^*}^{c} f(x) dx = \int_{a^*}^b f(x) dx$ и $\lim_{c \rightarrow b - 0} \int_a^c f(x) dx = \int_a^b f(x) dx$ существуют (конечны) одновременно. Равенство \ref{eqn-2-1} получается из равенства для определённых интегралов переходом к пределу $c \rightarrow b - 0$.
    \end{proof}
\end{property}

Следующие три свойства доказываются аналогично.

\begin{property}[линейность] 
    Пусть несобственные интегралы $\int_a^b f(x) dx$ и $\int_a^b g(x) dx$ сходятся, и $\alpha, \beta \in \R$. Тогда сходится интеграл $\int_a^b \left(\alpha f(x) + \beta g(x)\right) dx$ и
    \[
        \int_a^b \left(\alpha f(x) + \beta g(x)\right) dx = \alpha \int_a^b f(x) dx + \beta \int_a^b g(x) dx.
    \]
\end{property}

\begin{property}[формула Ньютона-Лейбница]
    Пусть $f$ локально интегрируема на $[a, b)$ и $F$ --- первообразная $f$ на $[a, b)$. Тогда
    \[
        \int_a^b f(x) dx = F(b - 0) - F(a) = F \vert_a^{b - 0}.
    \]
\end{property}

\begin{property}[интегрирование по частям]
    Пусть $F, G$ дифференцируемы, а их производные $f, g$ локально интегрируемы на $[a, b)$. Тогда
    \[
        \int^b_a F(x) g(x) dx = F(x)G(x)\vert^{b - 0}_a - \int_a^b G(x) f(x) dx.
    \]

    Существование двух из трёх конечных пределов влечёт существование третьего и выполнение равенства.
\end{property} %Предварительные сведения
    \subsubsection*{Связь метрического пространства с топологическим}

\begin{theorem} (Двойственность открытых и замкнутых множеств)
	Пусть $(X, \rho)$ --- метрическое пространство. Тогда подмножество $G \subseteq X$ является открытым тогда и только тогда, когда $F := X \bs G$ является замкнутым
\end{theorem}

\begin{proof}~
	\begin{itemize}
		\item[$\Ra$] Рассмотрим $x \in \cl F$. Если так оказалось, что $x \in G \cap \cl F$, то существует шар $B(x, r) \subseteq G$, и, следовательно, это противоречит исходному факту о принадлежности $x$ замыканию $F$. Таким образом, остаётся лишь вариант $x \in F \cap \cl F$, то есть $F = \cl F$.
		
		\item[$\La$] Рассмотрим $x \in G$. Это же означает, что $x \notin F = \cl F$, то есть $x$ --- не точка прикосновения $F$:
		\[
			\exists r > 0 \such B(x, r) \cap F = \emptyset
		\]
		Так как $F = X \bs G$, то $B(x, r) \subseteq G$, а значит $x \in \Int G$, что и требовалось показать.
	\end{itemize}
\end{proof}

\begin{theorem}
	Путь $(X, \rho)$ --- метрическое пространство, $\{G_\alpha\}_{\alpha \in \gA}$ --- семейство открытых множеств, а $\{F_\beta\}_{\beta \in \gM}$ --- семейство замкнутых множеств. Тогда выполнены следующие утверждения:
	\begin{enumerate}
		\item $\forall \gN \subseteq \gA\ \bigcup_{\alpha \in \gN} G_\alpha$ --- открытое множество
		
		\item $\forall \gZ \subseteq \gA, |\gZ| < \infty\ \bigcap_{\alpha \in \gZ} G_\alpha$ --- открытое множество
		
		\item $\forall \gN \subseteq \gM\ \bigcap_{\beta \in \gN} F_\beta$ --- замкнутое множество
		
		\item $\forall \gZ \subseteq \gM, |\gZ| < \infty\ \bigcup_{\beta \in \gZ} F_\beta$ --- замкнутое множество
	\end{enumerate}
\end{theorem}

\begin{proof}
	Достаточно доказать утверждения для открытых множеств, потому что дальше мы используем правила де Моргана и получаем оставшиеся утверждения просто через дополнения. Итак, приступим:
	\begin{enumerate}
		\item Рассмотрим $x \in \bigcup_{\alpha \in \gN} G_\alpha$. Тогда должен существовать $\alpha_0 \in \gN$, при котором этот $x$ был включен в объедиение. Стало быть, $x \in G_{\alpha_0}$, а раз $G_{\alpha_0}$ --- открытое множество, то $\exists B(x, r) \subseteq G_{\alpha_0} \subseteq \bigcup_{\alpha \in \gN} G_\alpha$, что и требовалось показать.
		
		\item Пусть $x \in \bigcap_{\alpha \in \gZ} G_\alpha$. Тогда этот $x$ был включен в каждое $G_\alpha$, причём в каждом из них есть свой радиус шара $r_\alpha > 0$ с центром в $x$, что $B(x, r_\alpha) \subseteq G_\alpha$. Так как мы рассмотрели конечное пересечение открытых множеств, то, взяв $r = \min_{\alpha \in \gZ} r_\alpha$, мы нашли радиус шара, включённого во все $G_\alpha$, а значит и в их пересечение.
	\end{enumerate}
\end{proof}

\begin{corollary}
	Любое метрическое пространство является топологическим, чья топология состоит из всех открытых множеств (по независимому определению конкретно в метрических пространствах).
\end{corollary}

\begin{example}
	Покажем, что существует счётное объединение замкнутых множеств, которое не является само по себе замкнутым. Достаточно рассмотреть числовую прямую и последовательность вложенных расширяющихся отрезков, которая пытается достичь другого отрезка. Например, зададим такую последовательность так:
	\[
		\forall n \in \N\ \ [a_n; b_n] = \sbr{1 + \frac{1}{n}; 4 - \frac{1}{n}}
	\]
	Казалось бы, это последовательность должна сходиться к $[1; 4]$, но на деле предел --- это интервал $(1; 4)$! Он тривиально не является замкнутым множеством за счёт своих крайних точек.
\end{example}

\begin{exercise}
	Топологическое проcтранство основных функций $D$ не является метризуемым
\end{exercise}

\begin{lemma}
	Пусть $(X, \rho)$ --- метрическое пространство, $M_{1, 2} \subseteq X$. Если $M_1 \subseteq M_2$, то и $\Int M_1 \subseteq \Int M_2$.
\end{lemma}

\begin{proof}
	Факт, на самом деле, очевидный. Если $x \in \Int M_1$, то какой-то шар с центром в этой точке включен в $M_1 \subseteq M_2$, то есть $x \in \Int M_2$.
\end{proof}

\begin{lemma}
	Пусть $(X, \rho)$ --- метрическое пространство, $M_{1, 2} \subseteq X$. Если $M_1 \subseteq M_2$, то и $\cl M_1 \subseteq \cl M_2$.
\end{lemma}

\begin{proof}
	Совсем тривиально, что точка прикосновения $M_1$ является точкой прикосновения $M_2 \supseteq M_1$.
\end{proof}

\begin{theorem}
	Пусть $(X, \rho)$ --- метрическое пространство. Тогда:
	\begin{enumerate}
		\item Открытый шар $B(x, r)$ является открытым множеством
		
		\item Для любого множества $M \subseteq X$ верно, что $\Int M$ --- открытое множество, причём наибольшее из тех, что вложены в $M$
		
		\item Для любого множества $M \subseteq X$ верно, что $\cl M$ --- замкнутое множество, причём наименьшее из тех, что содержат $M$.
		
		\item Замкнутый шар $\ole{B}(x, r)$ является замкнутым множеством
	\end{enumerate}
\end{theorem}

\begin{proof}~
	\begin{enumerate}
		\item Рассуждение довольно геометрическое. Если $y \in B(x, r)$, то по определению $y$ не лежит на границе шара, значит есть некоторое расстояние до неё и мы можем найти достаточно малый шарик. Если формально, то $\rho(x, y) < r$ до границы у нас расстояние $l = r - \rho(x, y)$. Значит, можно рассмотреть шар $B(y, l / 2)$. Покажем, что любая точка в нём действительно остаётся в шаре $B(x, r)$:
		\begin{multline*}
			\forall z \in B(y, l / 2)\ \rho(z, y) < \frac{l}{2} \Ra
			\\
			\rho(x, z) \le \rho(x, y) + \rho(y, z) < \rho(x, y) + \frac{r - \rho(x, y)}{2} = \frac{r + \rho(x, y)}{2} < r
		\end{multline*}
		
		\item Настало время воспользоваться леммой для внутренностей. Если $x \in \Int M$, то по определению $\exists r > 0 \such B(x, r) \subseteq M$. Применив к этим множествам лемму, имеем $B(x, r) = \Int B(x, r) \subseteq \Int M$, что и требовалось показать.
		
		\item Заметим тривиальную \textit{микродвойственность}: точка $x \in X$ не лежит в замыкании $M$ тогда и только тогда, когда она лежит во внутренности $X \bs M$. Этот факт позволяет заявить, что $\cl M \sqcup \Int (X \bs M) = X$, а так как $\Int (X \bs M)$ по доказанному является открытым множеством, то в силу теоремы о двойственности мы тривиально имеем требуемое.
		
		\item Тривиально. Для точек на расстоянии $< r$ мы уже всё знаем по первому пункту, а для тех, у кого расстояние $= r$, достаточно сказать, что всегда есть точки на пересечении отрезка от $x$ до $y$ и шара с центром в рассматриваемой точке.
	\end{enumerate}
\end{proof}

\begin{proposition}
	Пусть $(X, \rho)$ --- метрическое пространство. Всегда верно, что \\ $\ole{B(x, r)} \subseteq \ole{B}(x, r)$, но равенство выполнено не всегда.
\end{proposition}

\begin{proof}
	Для доказательства вложения достаточно заметить, что точки на расстоянии $l > r$ не могут быть точками прикосновения $B(x, r)$, ибо можно рассмотреть шар радиуса $(l - r) / 2$, который не будет пересекать $B(x, r)$, а значит остаётся лишь вариант $\le r$, который по определению соответствует $\ole{B}(x, r)$.
	
	\textcolor{red}{Контрпример нужно искать среди дискретных метрик, возможно на манхэттенской...}
\end{proof}

\begin{definition}
	Пусть $X, Y$ --- топологические пространства, $f \colon X \to Y$. Тогда $f$ называется \textit{непрерывной в точке $x_0 \in X$}, если выполнено утверждение:
	\[
		\forall\ V(f(x_0)) \in \Tau_Y\ \exists U(x_0) \in \Tau_X \such f(U(x_0)) \subseteq V
	\]
\end{definition}

\begin{definition}
	Пусть $X, Y$ --- топологические пространства, $f \colon X \to Y$ называется \textit{непрерывной}, если для любой точки $x \in X$ верно, что $f$ непрерывна в точке $x$.
\end{definition}

\begin{theorem}
	Пусть $X, Y$ --- топологическое пространство, $f \colon X \to Y$. Тогда следующие свойства функции эквивалентны:
	\begin{enumerate}
		\item $f$ непрерывна
		
		\item $\forall G \in \Tau_Y\ f^{-1}(G) \in \Tau_X$
		
		\item $\forall F \subseteq Y$ --- замкнутое множество, тогда $f^{-1}(F) \subseteq X$ --- тоже замкнутое
	\end{enumerate}
\end{theorem}

\begin{proof}
	\item $2 \Ra 1$ Тривиально
	
	\item $1 \Ra 2$ Рассмотрим $G \in \Tau_Y$ и соответствующий прообраз $f^{-1}(G)$. По определению это такое подмножество $X$:
	\[
		f^{-1}(G) = \{x \in X \colon f(x) \in G\}
	\]
	Покажем, что у каждой точки $x \in f^{-1}(G)$ есть некоторая окрестность $U(x) \in \Tau_X$, которая находится тоже в $f^{-1}(G)$. Действительно, если рассмотреть $f(x) \in G$, то по тому, что $G$ является окрестностью $f(x)$ и $f$ непрерывна, должна существовать окрестность $U(x) \in \Tau_X$ такая, что $f(U(x)) \subseteq G$, а это эквивалентно как раз требованию $U(x) \subseteq f^{-1}(G)$. Осталось заметить, что прообраз $G$ можно представить через объединение таких окрестностей:
	\[
		\ps{f^{-1}(G) = \bigcup_{x \in f^{-1}(G)} U(x),\ U(x) \in \Tau_X} \Ra f^{-1}(G) \in \Tau_X
	\]
	
	\item $2 \Lra 3$ Прообраз сохраняет все стандартные операции над множествами, поэтому эквивалентность тривиальна за счёт двойственности открытых и замкнутых множеств.
\end{proof} %Глава2. Натуральные, целые, рациональные
    \subsection{Классификация вероятностных мер}

\subsubsection{Дискретные вероятностные меры}

\begin{definition}
	Пусть $P$ --- вероятностная мера на $(\R, \B(\R))$. Она называется \textit{дискретной}, если выполнено условие:
	\[
		\exists X \subseteq \R \text{ --- не более чем счётное множество} \such P(\R \bs X) = 0 \wedge \forall x \in X\ P(\{x\}) > 0
	\]
	При этом говорят, что \textit{вероятностная мера $P$ сосредоточена на $X$}.
\end{definition}

\begin{definition}
	Пусть вероятностная мера $P$ сосредоточена на $X = \{x_k\}_{k = 1}^\infty \subset \R$. Обозначим $p_k = P(\{x_k\})$, тогда \textit{набор $(p_1, p_2, \ldots)$ образует распределение вероятностей на $X$}.  
\end{definition} %Рациональные. Действительные - сложение
    \begin{theorem}
	Пусть $H$ --- это $n$-однородный гиперграф, причём $|E(H)| < 2^{n - 1}$. Тогда $\chi(H) = 2$
\end{theorem}

\begin{proof}
	Рассмотрим случайную раскраску вершин нашего гиперграфа $H$ в красный и синий цвета (выбор цвета происходит с вероятностью 0.5). Сопоставим каждому ребру $e \in E$ событие $A_e$, что $e$ --- одноцветное множество. Тогда
	\[
		P(A_e) = 2^{1 - n}
	\]
	Сделаем тривиальную оценку на то, что ни одно ребро не является одноцветным (это и будет доказательством):
	\[
		P\ps{\bigcup_{e \in E} A_e} \le \sum_{e \in E} P(A_e) < 2^{n - 1} \cdot 2^{1 - n} = 1 \Ra P\ps{\bigcap_{e \in E} \overline{A_e}} > 0
	\]
\end{proof}

\begin{theorem}
	Пусть $H$ --- это $n$-однородный гиперграф, где степень каждой вершины не превосходит $n$. Если $n \ge 9$, то $\chi(H) = 2$
\end{theorem}

\begin{proof}
	Вводим всё то же вероятностное пространства и события $A_e$, как и в предыдущей теореме. Для доказательства этой теоремы мы применим симметричный случай ЛЛЛ, поэтому нужно ответить на вопрос: <<От каких $A_{e'}$ не зависит $A_e$?>> Для кого-то удивительно, но события, пересекающиеся по одной вершине, будут всё ещё независимы. Но если, вдруг, у нас есть 3 события, которые пересекаются попарно по 1 вершине, то в совокупности они зависимы. Самая простая оценка --- это всё же оценить число рёбер, которые пересекают наше хотя бы по одной вершине:
	\[
		d \le n \cdot \underbrace{(n - 1)}_{\forall v \in V\ \deg v \le n - 1}
	\]
	Осталось проверить неравенство $e \cdot (n(n - 1) + 1) \cdot 2^{1 - n} \le 1$. Понятно, что левая часть стремится к нулю, а требование $n \ge 9$ обеспечило нам уже верность этого неравенства.
\end{proof}

\begin{theorem}
	Пусть $H$ --- это $n$-однородный гипеграф. Тогда верна следующая оценка снизу на хроматическое число:
	\[
		\chi(H) \le \sqrt[n - 1]{e(n(\Delta - 1) + 1)}
	\]
	где $\Delta$ --- наибольшая степень вершины в $H$.
\end{theorem}

\begin{proof}
	Результат, который указан в теореме, получается только через ЛЛЛ, по крайней мере сегодня.
	
	Снова рассмотрим вероятностное пространство со случайными раскрасками графа в $r$ цветов. Тогда $A_e$ это всё то же событие, что вершины $e$ покрашены в одинаковый цвет, а вероятность этого события уже $P(A_e) = r (1 / r)^n = r^{1 - n}$. Оценку на число зависимых с каким-то выбранным событий можно записать как $d \le n \cdot (\Delta - 1)$. Снова остаётся разобраться с неравенством из симметричного случая ЛЛЛ:
	\[
		e(n(\Delta - 1) + 1) \cdot r^{1 - n} \le 1 \Lra e(n(\Delta - 1) + 1) \le r^{n - 1} \Lra r \ge \sqrt[n - 1]{e(n(\Delta - 1) + 1)}
	\]
	То есть достаточно взять за $r$ округлённую вверх целую часть от правого выражения, чтобы найти хорошую раскраску.
\end{proof}

\begin{note}
	Можно сравнить полученный результат в случае обыкновенных графов (то есть 2-однородных гиперграфов) с тривиальной оценкой: $\chi(G) \le \Delta(G) + 1$. Тогда продвинутый результат говорит, что $\chi(G) \le \sqrt{2e\Delta - 2e + 1}$, что для графа с $\Delta(G) = 3$ даст улучшение в порядка 1.5 раза.
\end{note}

\subsection{Конструктивные оценки снизу числа Рамсея}

\begin{note}
	Большая часть оценок, которая была нами получена, сделана через вероятностный метод. Он позволяет установить существование объекта, но не даёт конкретных примеров. Естественно, профессиональные математики это тоже заметили и стали искать конструкции. Об этих результатах мы говорим далее.
\end{note}

\begin{example}
	Можно вспомнить граф, который возникает как предельный случай теоремы Турана. Если взять $s - 1$ компоненту, каждая из которых является $K_{s - 1}$, то получится граф, в котором $\alpha(G) = s - 1$ и $\omega(G) = s - 1$. Стало быть, $R(s, s) > (s - 1)^2$ (сама оценка глобально смешна, но важна сама конструкция и какую оценку она даёт)
\end{example}

\begin{example}
	Рассмотрим $G(n, 3, 1)$. Тогда мы уже знаем, что $\alpha(G(n, 3, 1)) \le n$, а что можно сказать про $\omega(G(n, 3, 1))$? Его значение можно найти точно:
	\[
		\omega(G) = \floor{\frac{n - 1}{2}} (< n \text{ при } n \ge 8)
	\]
	Как это получается? Мы берём любую из вершин, а оставшиеся 2 это 2 последовательных числа из тех $n - 1$, что остались.
	Как известно, в таком графе $C_n^3 \sim n^3 / 6$ вершин. Если ввести обозначение $s = n + 1$, то получаем конструкцию, доказывающую оценку $R(s, s) > (1 + o(1))\frac{s^3}{6}$
\end{example}

\textcolor{red}{Дальше начинается конструктивная теорема Франкла-Уилсона для чисел Рамсея. В нашей программе это снова вопрос на отл, поэтому отсутствие конспекта понятно} %Действительные добиваются
    
    % 2023 reTeXed - partially:
    \subsection{Двудольные числа Рамсея}

\begin{definition}
	\textit{Двудольным числом Рамсея} $b(k, k)$ называется минимальное число $n \in \N$ такое, что при любой раскраске рёбер двудольного графа $K_{n, n}$ в красный и синий цвета найдётся одноцветный $K_{k, k}$
\end{definition}

\begin{exercise}
	Ровно теми же методами, которыми получалась оценка снизу для простого числа Рамсея, можно получить такой результат для двудольного:
	\[
		b(k, k) \ge ck2^{k / 2},\ c = const
	\]
\end{exercise}

\begin{theorem} (Конлон)
	Имеет место оценка сверху: \(\forall \eps > 0\ b(k, k) \le (1 + \eps)k \cdot 2^k\)
\end{theorem}

\begin{lemma}
	Рассмотрим произвольный $G \subseteq K_{m, n}$ плотности не ниже, чем $p \in [0; 1]$ (то есть $|E(G)| / (mn) \ge p$). Если числа $m, n, r, s$ и $p \in [0; 1]$ таковы, что
	\[
		nC_{mp}^r > (s - 1)C_m^r
	\]
	Тогда $\exists K_{r, s} \subseteq G$ ($r$ из доли размера $m$, $s$ из доли размера $n$, естественно)
\end{lemma}

\begin{proof}
	Предположим противное: в $G$ нет $K_{r, s}$. Тогда, посчитаем число $K_{r, 1}$ в $G$ двумя способами:
	\begin{itemize}
		\item С одной стороны, любое $r$-элементное подмножество $m$ может быть долей $K_{r, 1}$. Однако, выбрать последнюю вершину в другой доле мы можем не более $s - 1$ раз (иначе возникнет $K_{r, s}$). Таким образом, имеем $C_m^r (s - 1)$ возможных $K_{r, 1}$
		
		\item С другой стороны, пусть $d_1, \ldots, d_n$ --- это степени вершин в $G$, которые находятся в доле $K_{m, n}$ размера $n$ (если в $G$ меньше элементов в этой доле, то некоторые $d_i$ зануляются). Несложно понять, что тогда наша величина есть просто $C_{d_1}^r \plusdots C_{d_n}^r$. Причём известно, что биномиальные коэффициенты выпуклы, то есть
		\[
			C_{d_1}^r \plusdots C_{d_n}^r \ge n C_{\frac{d_1 \plusdots d_n}{n}}^r
		\]
		Сумма степеней вершин в точности равна числу рёбер в $G$, поэтому можно переписать с учётом условия это неравенство следующим образом:
		\[
			C_{d_1}^r \plusdots C_{d_n}^r \ge n C_{mp}^r
		\]
	\end{itemize}
	Получили противоречие с условием
\end{proof}

\begin{proof} (оценки Конлона)
	Зафиксируем $\eps > 0$ и $n = (1 + \eps)k \cdot 2^k$. Нужно показать, что любая раскраска этого графа приводит к существование одноцветного $K_{k, k}$. Для этого этого рассмотрим граф $G \subseteq K_{n, n}$, в котором находятся только красные рёбра. Без ограничения общности, можем сказать, что $p = 1 / 2$. Остаётся проверить, что при всех больших $k$ выполнено неравенство:
	\[
		nC_{n / 2}^k > (k - 1)C_n^k
	\]
	Хочется работать с этими выражениями асимптотически. Для этого заметим, что $k \sim \log_2 n$, поэтому мы имеем право пользоваться асимптотикой $C_n^k$:
	\[
		nC_{n / 2}^k \sim (1 + \eps)k2^k \cdot \frac{(n / 2)^k}{k!}(1 + o(1)) > k \frac{n^k}{k!}(1 + o(1)) \sim (k - 1)C_n^k
	\]
	Всё свелось к проверке следующего утверждения:
	\[
		(1 + \eps)(1 + o(1)) > (1 + o(1))
	\]
	Очевидно, что за счёт существование пределов обеих частей и уже их соотношения, это действительно так
\end{proof}

\begin{theorem}
	Имеет место оценка сверху: \(b(k, k) \le (1 + \eps)(\log_2 k) \cdot 2^{k + 1}\)
\end{theorem}

\begin{proof}
	\textcolor{red}{Спасибо, но это вообще на отл10}
\end{proof}

 %Комплексные. Предел
    %Комплексные числа не трогали с 2021

    % 2023 reTeXed - full:
    %16.02.23

\begin{theorem}[признак Гаусса]
    Пусть $a_{n} > 0$ для всех $n \in \N$ и существуют такие $s > 1$ и ограниченная последовательность $\{\alpha_{n}\}$, что для всех $n$ выполнено
    \[\frac{a_{n+1}}{a_{n}} = 1 - \frac{A}{n} + \frac{\alpha_{n}}{n^{s}}.\]
    Тогда ряд $\sum_{n = 1}^{+\infty} a_{n}$ сходится при $A > 1$ и расходится иначе. 
\end{theorem}

\begin{proof}
    При $n > 1$ имеем
    \[a_{n} = a_{1}\cdot \frac{a_{2}}{a_{1}}\cdot \ldots \cdot \frac{a_{n}}{a_{n - 1}} = a_{1}\cdot\prod_{k = 1}^{n - 1}\left(1 - \frac{A}{k} + \frac{\alpha_{k}}{k^{s}}\right) = a_{1}\cdot \exp\left(\sum_{k = 1}^{n-1} \ln(1 - \frac{A}{k} + \frac{\alpha_{k}}{k^{s}})\right).\]
    
    Так как $\ln(1 + t) = t + O(t^{2})$, $t \to 0$, имеем
    \[a_{n} = a_{1}\cdot \exp\left(\sum_{k = 1}^{n - 1}\left(-\frac{A}{k} + \frac{\alpha_{k}}{k^{s}} + O\left(\frac{1}{k^{2}}\right)\right)\right).\]
    
    Воспользуемся равенством $\sum_{k = 1}^{n - 1} \frac{1}{k} = \ln n + \gamma + o(1)$ и сходимостью рядов $\sum_{k = 1}^{+\infty}\frac{1}{k^{p}}$ при $p > 1$. Тогда
    \[a_{n} = a_{1}\cdot \exp\left(-A \ln n + O(1)\right) = a_{1} \frac{e^{O(1)}}{n^{A}}.\]
    
    Теперь утверждение следует по признаку сравнения с рядом $\sum_{n = 1}^{+\infty}\frac{1}{n^{A}}$.
\end{proof}

\subsection{Ряды с произвольными членами}

Вернемся к изучению рядов с произвольными (в общем случае комплексными) членами.

\begin{definition}
    Ряд $\sum_{n = 1}^{+\infty}a_{n}$ называется \textit{абсолютно сходящимся}, если сходится ряд $\sum_{n = 1}^{+\infty}|a_{n}|$.

    Если ряд $\sum_{n = 1}^{+\infty}a_{n}$ сходится, но не сходится абсолютно, то он называется \textit{условно сходящимся}. 
\end{definition}

\begin{corollary}
    Абсолютно сходящийся ряд сходится.
\end{corollary}

\begin{proof}
    Для любых $m, n \in \N, m \leq n$,
    \[\left|\sum_{k = m}^{n} a_{k}\right| \leq \sum_{k = m}^{n}|a_{k}|.\]
    Поэтому, если ряд $\sum_{n = 1}^{+\infty} |a_{n}|$ удовлетворяет условию Коши, то условию Коши удовлетворяет ряд $\sum_{n = 1}^{+\infty}a_{n}$. 
\end{proof}

\begin{note}
    Если ряд $\sum_{n = 1}^{+\infty}a_{n}$ сходится абсолютно, то
    \[\left|\sum_{k = 1}^{+\infty} a_{k}\right| \leq \sum_{k = 1}^{+\infty}|a_{k}|.\]
\end{note}

\begin{lemma}
    \begin{enumerate}
        \item Если $\sum_{n = 1}^{+\infty} b_{n}$ сходится, то $\sum_{n = 1}^{+\infty} (a_{n} + b_{n})$ и $\sum_{n = 1}^{+\infty} a_{n}$ сходятся или расходятся одновременно.
        \item Если $\sum_{n = 1}^{+\infty} b_{n}$ абсолютно сходится, то $\sum_{n = 1}^{+\infty} (a_{n} + b_{n})$ и $\sum_{n = 1}^{+\infty} a_{n}$ либо одновременно расходятся, либо одновременно сходятся условно, либо одновременно сходятся абсолютно.
    \end{enumerate}
\end{lemma}

\begin{proof}~

    \begin{enumerate}
        \item Следует из свойства линейности. Для всех $n \in \N$ верно
        \[|a_{n} + b_{n}| \leq |a_{n}| + |b_{n}|, \ |a_{n}| \leq |a_{n} + b_{n}| + |b_{n}|.\]
        Следовательно, по признаку сравнения ряды $\sum_{n = 1}^{+\infty} (a_{n} + b_{n})$ и $\sum_{n = 1}^{+\infty} a_{n}$ одновременно абсолютно сходятся.
        
        \item Вытекает из пункта 1.
    \end{enumerate}
    
\end{proof}

\begin{theorem}[признак Дирихле]
    Пусть $\{a_{n}\}$ -- комплексная последовательность, $\{b_{n}\}$ -- действительная последовательность, причем
    \begin{enumerate}
        \item Последовательность $A_{N} = \sum_{n = 1}^{N} a_{n}$ ограничена,
        \item $\{b_{n}\}$ монотонна,
        \item $\lim_{n \to +\infty} b_{n} = 0$.
    \end{enumerate}
    Тогда ряд $\sum_{n = 1}^{+\infty} a_{n}b_{n}$ сходится.
\end{theorem}

\begin{problem}
    Доказать признак Дирихле.
\end{problem}

В следующих параграфах признак будет доказан в общности. Это относится и к следующему утверждению, которое можно сформулировать как следствие признака Дирихле.

\begin{theorem}[признак Абеля]
    Пусть $\{a_{n}\}$ -- комплексная последовательность, $\{b_{n}\}$ -- действительная последовательность, причем
    \begin{enumerate}
        \item Ряд $\sum_{n = 1}^{+\infty} a_{n}$ сходится,
        \item $\{b_{n}\}$ монотонна,
        \item $\{b_{n}\}$ ограничена.
    \end{enumerate}
    Тогда ряд $\sum_{n = 1}^{+\infty} a_{n}b_{n}$ сходится.
\end{theorem}

Полезно для практики выделить частные случаи признака Дирихле, в которых ограниченность частичных сумм (условие 1) выполняется автоматически.

\begin{corollary}[признак Лейбница]
    Пусть последовательность $\{\alpha_{n}\}$ монотонна и $\alpha_{n} \to 0$. Тогда ряд $\sum_{n = 1}^{+\infty} (-1)^{n - 1} \alpha_{n}$ сходится, причем
    \[|S - S_{n}| \leq |\alpha_{n + 1}|.\]
\end{corollary}

\begin{proof}
    Сходимость вытекает из признака Дирихле. Докажем ее прямо. Пусть для определенности $\{\alpha_{n}\}$ нестрого убывает, и, значит, все $\{\alpha_{n}\} \geq 0$.
    
    $S_{2n + 2} - S_{2n} = \alpha_{2n + 1} - \alpha_{2n + 2} \geq 0 \Rightarrow \{S_{2n}\}$ нестрого возрастает.
    
    $S_{2n + 1} - S_{2n - 1} = -\alpha_{2n} + \alpha_{2n + 1} \leq 0 \Rightarrow \{S_{2n - 1}\}$ нестрого убывает.
    
    Кроме того, $S_{2n} - S_{2n - 1} = - \alpha_{2n} \leq 0$. Поэтому для любых $m, n \in \N$ имеем
    \[S_{2n} \leq S_{2k} \leq S_{2k - 1} \leq S_{2m - 1},\]
    
    где $k = \max\{m, n\}$. Следовательно, последовательности $\{S_{2n}\}$ и $\{S_{2n - 1}\}$ сходятся, $S_{2n} \to S'$, $S_{2n - 1} \to S''$, и, в частности,
    \[S_{2n} \leq S' \leq S'' \leq S_{2n - 1}.\]
    
    Поскольку $S_{2n} - S_{2n - 1} = -\alpha_{2n} \to 0$, то $S' = S'' = S$.
\end{proof}

\begin{corollary}
    Пусть $\{\alpha_{n}\}$ монотонна и $\alpha_{n} \to 0, \ x \neq 2\pi m, \ m \in \Z$. Тогда ряды
    $\sum_{n = 1}^{+ \infty} \alpha_{n}\cos(nx)$ и $\sum_{n = 1}^{+ \infty} \alpha_{n}\sin(nx)$ сходятся.
\end{corollary}

\begin{proof}
    Положим $s_{N} = \sum_{n = 1}^{N} e^{inx}$. По формуле суммы геометрической прогрессии с $q = e^{ix}$ имеем 
    \[s_{N} = \frac{e^{ix}(1 - e^{iNx})}{1 - e^{ix}}.\]
    
    Поэтому, так как $|e^{ikx}| = 1$, $|s_{N}| \leq \frac{2}{\sqrt{(1 - \cos(x))^{2} + \sin^2(x)}} = \frac{2}{\sqrt{2 - 2\cos(x)}} = \frac{1}{|\sin(\frac{x}{2})|}$.
    
    Ограниченность сумм $C_{N} = \sum_{n = 1}^{N} \cos(nx)$ и $S_{N} = \sum_{n = 1}^{N} \sin(nx)$ следует из ограниченности $\{s_{N}\}$ и равенств $C_{N} = Re(s_{N})$, $S_{N} = Im(s_{N})$.
    
    Сходимость указанных рядов теперь следует из признака Дирихле.
\end{proof}

\begin{example}
    Найти сумму ряда $\sum_{n = 1}^{+\infty} \frac{(-1)^{n - 1}}{n}$.
\end{example}

\begin{solution}
    \[S_{2m} = 1 - \frac{1}{2} + \frac{1}{3} - \ldots + \frac{1}{2m - 1} - \frac{1}{2m} = 1 + \frac{1}{2} + \ldots + \frac{1}{2m - 1} + \frac{1}{2m} - 2\left(\frac{1}{2} + \frac{1}{4} + \ldots + \frac{1}{2m}\right) = \]
    \[= H_{2m} - H_{m} = (\ln 2m + \gamma + o(1)) - (\ln m + \gamma + o(1)) = \ln 2 + o(1), \ m \to +\infty.\]
    Значит искомая сумма равна $\ln 2$.
\end{solution}

\subsection{Перестановки рядов}

\begin{definition}
    Пусть дан ряд $\sum_{n = 1}^{+\infty} a_{n}$ и биекция $\phi: \N \to \N$. Тогда $\sum_{n = 1}^{+\infty}a_{\phi(n)}$ называется \textit{перестановкой} ряда $\sum_{n = 1}^{+\infty} a_{n}$.
\end{definition}

\begin{example}
    Рассмотрим следующую перестановку ряда $\sum_{n = 1}^{+\infty} \frac{(-1)^{n - 1}}{n}$:
    \[1, - \frac{1}{2}, -\frac{1}{4}, + \frac{1}{3}, - \frac{1}{6}, - \frac{1}{8}, + \frac{1}{5}, - \frac{1}{10}, \ldots\]
    Найдем сумму этой перестановки
    \[S_{p} = (1 - \frac{1}{2}) -\frac{1}{4} + (\frac{1}{3} - \frac{1}{6}) - \frac{1}{8} + (\frac{1}{5} - \frac{1}{10}) + \ldots = \frac{1}{2}(1 - \frac{1}{2} + \frac{1}{3} - \frac{1}{4} + \ldots) = \frac{1}{2} \ln 2.\]
    Заметим, что мы корректно нашли сумму, однако она отличается от ответа прошлой задачи. Это следует из условной сходимости ряда.
\end{example}

\begin{theorem}
    \label{convergence-9}
    Если ряд $\sum_{n = 1}^{+\infty} a_{n}$ сходится абсолютно, то любая его перестановка $\sum_{n = 1}^{+\infty} a_{\phi(n)}$ сходится абсолютно, причем к той же сумме.
\end{theorem}

\begin{proof}
    Абсолютная сходимость перестановки следует из оценки
    \[\sum_{n = 1}^{N}|a_{\phi(n)}| \leq \sum_{n = 1}^{\underset{1 \leq k \leq N}{\max\{\phi(k)\}}} |a_{n}| \leq \sum_{n = 1}^{+\infty}|a_{n}| < +\infty.\]
    Пусть $\epsilon > 0$. Выберем номер $m$ так, что $\sum_{n = m + 1}^{+\infty}|a_{n}| < \epsilon$. Выберем $M$ так, что $\{1, \ldots, m\} \subset \{\phi(1), \ldots, \phi(M)\}$ (достаточно положить $M = \max_{1 \leq j \leq m}\phi^{-1}(j)$). Тогда для любого $N \geq M$ имеем $\{1, \ldots, m\} \subset \{\phi(1), \ldots, \phi(N)\}$ и $\left|\sum_{n = 1}^{+\infty}a_{n} - \sum_{n = 1}^{N} a_{\phi(n)}\right| \leq \sum_{n = m + 1}^{+\infty}|a_{n}| < \epsilon$.
    Таким образом, частичные суммы перестановки сходятся у сумме исходного ряда.
\end{proof}

\begin{problem}[Теорема Римана]
    Если ряд с действительными членами $\sum_{n = 1}^{+\infty} a_{n}$ сходится условно, то для любого $L \in \overline{\R}$ существует такая перестановка $\sum_{n = 1}^{+\infty} a_{\phi(n)}$, что её сумма равна $L$.
\end{problem} %Предел последовательности. Неравенства, арифметические операции
    \textcolor{red}{$K = \Cm \vee \R$}

\begin{definition}
	Пространство $E$ называется \textit{линейно нормированным}, если выполнено 2 условия:
	\begin{enumerate}
		\item $E$ --- линейное пространство над $K$
		
		\item В пространстве $E$ существует \textit{оператор нормы} $\|\cdot\| \colon E \to \R_+$. Он удовлетворяет следующим условиям:
		\begin{enumerate}
			\item $\forall x \in E\ \ \|x\| \ge 0 \wedge \|x\| = 0 \Lra x = 0$
			
			\item $\forall x \in E,\ \alpha \in \R\ \ \|\alpha x\| = |\alpha| \cdot \|x\|$
			
			\item $\forall x, y \in E\ \ \|x  + y\| \le \|x\| + \|y\|$
		\end{enumerate}
	\end{enumerate}
\end{definition}

\begin{proposition}
	Любое линейно нормированное пространство является метрическим с индуцированной нормой метрикой $\rho(x, y) = \|x - y\|$.
\end{proposition}

\begin{proof}
	\textcolor{red}{Дописать}
\end{proof}

\begin{definition}
	Полное линейно нормированное пространство называется \textit{банаховым}.
\end{definition}

\begin{example}~
	\begin{enumerate}
		\item $\R^n$ является банаховым пространством
		
		\item $C[a; b]$ со своей собственной нормой $\|f\| = \min_{[a; b]} |f|$ является банаховым
		
		\item \textcolor{red}{Дописать}
	\end{enumerate}
\end{example}

\begin{note}
	Далее буква $E$ закрепляется за линейно нормированным пространством. 
\end{note}

\begin{definition}
	\textit{Линейным многообразием} $L \subseteq E$ называется линейная оболочка $\tbr{L}$. \textcolor{red}{Поправить формулировку}
\end{definition}

\begin{definition}
	Пространство $L \subseteq E$ называется \textit{подпространством в $E$}, если $L$ --- линейное многообразие.
\end{definition}

\begin{definition}
	\textit{Линейной оболочкой множества} $S \subseteq E$ называется множество всех конечных линейных комбинаций элементов из $S$. Обозначается как $[S]$ или $\Lin S$
\end{definition}

\begin{definition}
	Норма $\|\cdot\|_1$ \textit{слабее, чем} норма $\|\cdot\|_2$, если выполнено условие:
	\[
		\exists C > 0 \such \forall x \in E\ \ \|x\|_1 \le C\|x\|_2
	\]
\end{definition}

\begin{example}
	В пространстве $C[a; b]$ норма $\|\cdot\|_1$ слабее нормы $\|\|_{C[a; b]}$:
	\[
		\int_a^b |f(x)|dx \le \max_{x \in [a; b]} |f(x)| \cdot (b - a) = \|f\|_C \cdot (b - a)
	\]
\end{example}

\begin{definition}
	Нормы $\|\cdot\|_1$, $\|\cdot\|_2$ эквивалентны, если они слабее друг друга \textcolor{red}{Нужно явно другое слово для определения}
\end{definition}

\begin{definition}
	Пусть $E$ --- линейно нормированное пространство над $\R$. Тогда множество $S \subseteq E$ называется \textit{выпуклым}, если выполнено утверждение:
	\[
		\forall x, y \in S\ \forall \lambda \in [0; 1]\ \ \lambda x + (1 - \lambda)y \in S
	\]
\end{definition}

\begin{definition}
	\textit{Базисом} $E$ называется набор векторов $\{v_k\}_{k = 1}^\infty$ такой, что $\tbr{\{v_k\}_{k = 1}^\infty} = E$
\end{definition}

\begin{definition}
	\textcolor{red}{Базис Гамеля и базис Шаудера}
\end{definition}

\begin{definition}
	\textit{Размерностью} $E$ называется мощность базиса в пространстве $E$.
\end{definition}

\begin{definition}
	Пусть $E_1, E_2$ --- линейно нормированные пространства. Отображение $A \colon E_1 \to E_2$ называется \textit{оператором}.
\end{definition}

\begin{definition}
	Пусть $E$ --- линейно нормированное пространство над $K$. Тогда отображение $f \colon E \to K$ называется \textit{функционалом}.
\end{definition}

\begin{theorem}
	Пусть $\dim E < \infty$. Тогда на $E$ все нормы эквивалентны.
\end{theorem}

\begin{proof}
	\textcolor{red}{Ниже приведено доказательство над $\R$}. Рассмотрим ортонормированный базис $\{e_1, \ldots, e_n\}$. Покажем, что произвольная норма $\|\cdot\|_1$ является эквивалентной к норме, порождённой ортогональным базисом:
	\[
		\|x\| = \no{\sum_{k = 1}^n \alpha_k e_k} = \sqrt{\sum_{k = 1}^n \alpha_k^2}
	\]
	\begin{itemize}
		\item[$\Ra$] Обозначим $\kappa = \max_{k \in \range{1}{n}} \|e_k\|_1$. Имеет место цепочка неравенств:
		\[
			\|x\|_1 = \no{\sum_{k = 1}^n \alpha_ke_k}_1 \le \sum_{k = 1}^n \|\alpha_ke_k\|_1 \le \kappa \sum_{k = 1}^n |\alpha_k| \le \kappa \sqrt{n} \cdot \|x\|
		\]
		Последний переход --- неравенство Коши между средним арифметическим и квадратичным.
		
		\item[$\La$] Предположим противное. \textcolor{red}{Дописать}
	\end{itemize}
\end{proof}

\begin{proposition}
	Пусть $L = \tbr{e_1, \ldots, e_n}$. Тогда $L$ --- банахово пространство.
\end{proposition}

\begin{proof}
	$L$ --- конечномерное пространство. По доказанной теореме, все нормы в нём эквивалентны. В частности, можно рассмотреть $L$ как подпространство $\R^n$. Тогда норма на $L$ эквивалентна евклидовой норме, а стало быть есть полнота. \textcolor{red}{Последний переход не понял}
\end{proof}

\begin{theorem}
	Пусть $\{e_1, \ldots, e_n\} \subseteq E$. Тогда $\tbr{e_1, \ldots, e_n}$ является подпространством $E$.
\end{theorem}

\begin{theorem} (Ф. Рисса)
	Пусть $E$ --- бесконечномерное пространство. Тогда единичная сфера в $E$ не является вполне ограниченной.
\end{theorem}

\begin{corollary}
	Единичная сфера не компактна.
\end{corollary}

\begin{lemma} (о <<почти перпендикуляре>>)
	Пусть $E_1 \subset E$ --- подпространство. Тогда выполнено утверждение:
	\[
		\forall \eps > 0\ \exists y \in E \colon \System{
			&{\|y\| = 1}
			\\
			&{\rho(y, E_1) > 1 - \eps}
		}
	\]
\end{lemma}

\begin{note}
	Тот факт, что $E_1$ --- не просто линейное многообразие, а подпространство, существенно.
	
	Рассмотрим $E = C[0; 1]$, $E_1 = \mathcal{P}$. Тогда $E_1$ --- не подпространство.
\end{note}

\begin{proof}
	Коль скоро $E_1 \neq E$, то существует $y_0 \notin E_1$. Введём обозначение $d = \rho(y_0, E_1)$. Сразу понятно, что в силу замыкания не может быть $d = 0$, иначе $y_0 \in E_1$. Из определения расстояния, в частности, верно утверждение:
	\[
		\forall \eps > 0\ \exists z_0 \in E_1 \such d \le \|y_0 - z_0\| < d(1 + \eps)
	\]
	Посмотрим на вектор $y = \frac{y_0 - z_0}{\|y_0 - z_0\|}$. Обозначим коэффициент при векторе за $\alpha$. Сразу видно, что $\|y\| = 1$, поэтому осталось проверить только расстояние 
\end{proof} %Подпоследовательности. Частичные пределы
    \begin{corollary} \textit{(Формула интегрирования по частям)}
	Если $f$ и $g$ интегрируемы по Риману на $[a;b]$ вместе со своими производными, то верна формула интегрирования по частям:
	\[
		\int_a^bf(x)g'(x)dx = f(b)g(b) - f(a)g(a) - \int_a^bf'(x)g(x)dx
	\]
\end{corollary}

\begin{proof}
	$\\ f(x),\ g(x),\ f'(x),\ g'(x)\ \in R[a;b] \Ra (fg)'= f'g + fg' \in R[a;b]$.
	Очевидно, что $fg$ - первообразная этой функции. Тогда по формуле Ньютона-Лейбница:
	\[
		\int_a^bf'(x)g(x)dx + \int_a^bf(x)g'(x)dx = f(x)g(x)\Bigg |^b_a = f(b)g(b) - f(a)g(a)
	\]
\end{proof}

\begin{theorem} \textit{(Формула замены переменной)}
	Пусть $f(x)$ непрерывна на $[a;b],\ g(x)$ интегрируема по Риману на $[\alpha; \beta]$ вместе с $g'(x),\ \forall x \in [\alpha; \beta]\ a \leq g(x) \leq b$, причем $g(\alpha) = a\text{ и } g(\beta) = b$. Тогда 
	\[
		\int_a^bf(x)dx = \int_\alpha^\beta f(g(t))g'(t)dt
	\]
\end{theorem}

\begin{proof}
	Левый интеграл существует так как $f(x)$ непрерывна. Правый тоже существует, так как это произведение интегрируемых функций.
	Пусть $F(x) = \int_a^xf(t)dt$, тогда $\forall x \in [a;b]\ F'(x) = f(x)$.
	Рассмотрим производную $F(g(t))'$ (у $g$ она есть из условия):
	\[
		F(g(t))' = F'(g(t))\cdot g'(t) = f(g(t))\cdot g'(t) \in R[\alpha;\beta]
	\]
	Окончательно по теореме Ньютона-Лейбница:
	\[
		\int_\alpha^\beta f(g(t))g'(t)dt = F(g(t)) \Biggr |_\alpha^\beta = F(g(\beta)) - F(g(\alpha))= F(b) - F(a) = \int_a^bf(x)dx
	\]
\end{proof}

\begin{theorem} \textit{(Первая теорема о среднем)}
	Пусть $f(x)$ и $g(x)$ интегрируемы по Риману на $[a;b]$, причем $\forall x \in [a;b]\ g(x) \geq 0$, $m \leq f(x) \leq M$. Тогда 
	\[
		\exists \mu \in [m;M] \such \int_a^bf(x)g(x)dx = \mu \int_a^bg(x)dx
	\]
	Если также известно, что $f$ - непрерывна на $[a;b]$, то 
	\[
		\exists \xi \in [a;b]\such \int_a^bf(x)g(x)dx = f(\xi)\int_a^bg(x)dx
	\]
\end{theorem}

\begin{proof}
	Заметим, что
	\[
		\forall x\in [a;b]\ mg(x) \leq f(x)g(x) \leq Mg(x)
	\]
	По свойству монотонности:
	\[
		\int_a^b mg(x)dx \leq \int_a^b f(x)g(x)dx \leq \int_a^b Mg(x)dx
	\]
	Если $\int_a^bg(x)dx = 0$, то справедливость теоремы очевидна.
	Иначе можно разделить равенство на этот интеграл:
	\[
		m \leq \mu = \frac{\int_a^b f(x)g(x)dx}{\int_a^b g(x)dx} \leq M
	\]
	Считая, что $m$ и $M$ - точные грани непрерывной функции $f$, то по теореме о промежуточных значениях непрерывной функции $\exists \xi \in [a;b]\ f(\xi) = \mu$.
\end{proof}

\begin{theorem} \textit{(Вторая теорема о среднем)}
	Пусть $f$ интегрируема по Риману на $[a;b]$, $g(x)$ невозрастающая и неотрицательная функция на $[a;b]$. Тогда:
	\[
		\exists \xi \in [a;b] \such \int_a^bf(x)g(x)dx = g(a)\int_a^\xi f(x)dx
	\]
\end{theorem}


\begin{proof}
	Возьмем такое разбиение $P : a = x_0 < x_1 < \ldots < x_n = b$. Также пусть $M_k = \sup\limits_{x \in [x_{k - 1};x_k]} f(x),\ m_k = \inf\limits_{x \in [x_{k - 1};x_k]} f(x)$. Тогда есть следующее неравенство для интегральной суммы:
	\[
	 	\suml_{k = 1}^nm_kg(x_{k - 1})\Delta x_k \le	\suml_{k = 1}^nf(x_{k - 1})g(x_{k - 1})\Delta x_k \le \suml_{k = 1}^nM_kg(x_{k - 1})\Delta x_k
	\]
	Более того, верно следующее:
	\[
		\suml_{k = 1}^n (M_k - m_k)g(x_{k - 1})\Delta x_k \le g(a) \suml_{k = 1}^n (M_k - m_k) \Delta x_k
	\]
	Оценим величину справа:
	\begin{multline*}
		g(a)\suml_{k = 1}^n (M_k - m_k)\Delta x_k = g(a)\suml_{k = 1}^n (M_k - f(t'_k))\Delta x_k + g(a)\suml_{k = 1}^n f(t'_k)\Delta x_k -
		\\
		g(a)\suml_{k = 1}^n f(t''_k)\Delta x_k + g(a)\suml_{k = 1}^n (f(t''_k) - m_k)\Delta x_k
	\end{multline*}
	Где $t'_k,\ t''_k \in [x_{k - 1}; x_k]$.
	Обозначим слагаемые как $s_1, s_2, s_3, s_4$ соответственно. Случай $g(a) = 0$ очевиден, потому дальше будем говорить только об обратном. Из определения точных верхней и нижней граней:
	\begin{align*}
		&{\forall \eps > 0\ \exists t'_k \in [x_{k - 1}; x_k] \such M_k - f(t'_k) < \frac{\eps}{4g(a)(b - a)}}
		\\
		&{\forall \eps > 0\ \exists t''_k \in [x_{k - 1}; x_k] \such f(t''_k) - m_k < \frac{\eps}{4g(a)(b - a)}}
	\end{align*}
	Применим это к нашим суммам и получим, что:
	\[
		s_1,\ s_4 \such 0 \leq s_1 < \frac{\eps}{4},\ 0 \leq s_4 < \frac{\eps}{4}
	\]
	Для оставшихся двух сумм можем сказать из интегрируемости $f$ следующее:
	\[
		\forall \eps > 0\ \exists \delta_1 > 0 \such \forall P, \Delta P < \delta_1 \quad \left|s_j - g(a)\cdot \int_a^bf(x)dx\right| < \frac{\eps}{4},\ j = 2, 3
	\]
	Получаем в итоге утверждение
	\[
		\forall \eps > 0\ \exists \delta_1 > 0 \such \forall P, \Delta P < \delta_1 \quad \suml_{k = 1}^n(M_k - m_k)g(x_{k - 1})\Delta x_k < \eps
	\]
	Теперь распишем левый интеграл из теоремы по критерию:
	\[
		\forall \eps > 0\ \exists \delta_2 \such \forall P, \Delta P < \delta_2 \quad \left|\suml_{k = 1}^nf(x_{k - 1})g(x_{k - 1}) \Delta x_k - \int_a^bf(x)g(x)dx\right| < \eps
	\]
	Зафиксируем $\forall \mu_k \in [m_k; M_k]$. Где будет находиться сумма $\suml_{k = 1}^n \mu_k g(x_{k - 1})\Delta x_k$? Она зажата между верхней и нижней суммами, как и $\suml_{k = 1}^n f(x_{k - 1})g(x_{k - 1})\Delta x_k$. При этом мы ещё знаем теперь, что последняя сумма находится недалеко от интеграла. Значит, верно следующее:
	\[
		\forall \eps > 0\ \exists \delta = \min \{\delta_1, \delta_2\} \such \forall P, \Delta P < \delta \quad \left|\suml_{k = 1}^n\mu_kg(x_{k - 1}) \Delta x_k - \int_a^bf(x)g(x)dx\right| < 2\eps
	\]
	Или же
	\[
		\forall \mu_k \in [m_k; M_k] \quad \liml_{\Delta P \to 0} \suml_{k = 1}^n\mu_kg(x_{k - 1})\Delta x_k = \int_a^bf(x)g(x)dx
	\]
	Теперь найдём особые $\mu_k$. По первой теореме о среднем:
	\[
		\exists \mu_k \in [m_k; M_k] \such \int_{x_{k - 1}}^{x_k} f(x)dx = \mu_k \int_{x_{k - 1}}^{x_k} dx = \mu_k\Delta x_k
	\]
	Именно такие $\mu_k$ мы и будем брать. Обозначим за $S_i := \int_a^{x_i}f(x)dx,\ i \in \range{n}$. Причем $S_0 := 0$. Теперь можно переписать сумму выше, как:
	\[
		\suml_{k = 1}^n\mu_kg(x_{k - 1})\Delta x_k = \suml_{k = 1}^ng(x_{k - 1})(S_k - S_{k - 1})
	\]
	Воспользуемся \textit{преобразованием Абеля}:
	\begin{multline*}
		\suml_{k = 1}^ng(x_{k - 1})(S_k - S_{k - 1}) = g(x_0)S_1 +g(x_1)(S_2 - S_1) + \ldots + g(x_{n - 1})(S_n - S_{n - 1}) =
		\\
		 S_1(g(x_0) - g(x_1)) + S_2(g(x_1) - g(x_2)) + \ldots + S_{n - 1}(g(x_{n - 2}) - g(x_{n - 1})) + S_ng(x_{n - 1})
	\end{multline*}
	Полученное выражение объясняет, почему мы брали значение по $g(x_{k - 1})$: теперь каждая скобка выше неотрицательна. Значит, мы можем придумать какое-то неравенство на $S_i$, и оно не сломается от умножения на скобку с обеих сторон:
	\[
		S_i = F(x_i),\ F(x) = \int_a^xf(x)dx
	\]
	Она непрерывна, а значит ограничена. $\forall x \in [a;b]\ \exists m \leq F(x) \leq M$, где $m, M$ --- точные нижняя и верхняя грани соответственно. Это равнозначно $\forall i = \range{n}\ \ m \leq S_i \leq M$.
	\[
		\left(mg(a) \leq \suml_{k = 1}^n \mu_k g(x_{k - 1})\Delta x_k \leq Mg(a)\right) \Ra \left(m \leq \frac{1}{g(a)} \int_a^b f(x)g(x)dx \leq M\right)
	\]
	В силу непрерывности $F$ уже получаем
	\[
		\exists \xi \in [a;b] \such F(\xi) = \int_a^\xi f(x)dx =  \frac{1}{g(a)}\int_a^bf(x)g(x)dx
	\]
\end{proof}


\begin{corollary} \textit{(Формула Бонн\'{е})}
	Пусть $f(x)$  интегрируема по Риману на $[a;b]$, $g(x)$ - монотонна на $[a;b]$, тогда:
	\[
		\exists \xi \in [a;b] \such \int_a^b f(x)g(x)dx = g(a)\int_a^\xi f(x)dx + g(b)\int_\xi^b f(x)dx
	\]
\end{corollary}

\begin{proof}
	Пусть $g$  невозрастающая, то $g_1(x) = g(x) - g(b) \geq 0$. По 2 теореме о среднем:
	\[
		\exists \xi \in [a;b]\ \int_a^bf(x)g_1(x)dx = g_1(a)\int_a^\xi f(x)dx
	\]
	Подставляя $g_1$, получим:
	\[
		\int_a^bf(x)g(x)dx - g(b) \int_a^bf(x)dx = g(a)\int_a^\xi f(x)dx - g(b)\int_a^\xi f(x)dx
	\]
	В случае неубывающей $g(x)$ возьмем $g_1(x) = g(b) - g(x)$.
\end{proof}

\begin{theorem} \textit{(Формула Тейлора с остаточным членом в интегральной форме)}
	Пусть $f(x)$ непрерывна в $U_\eps(a)$ вместе со своими производными до порядка $n + 1$ включительно.
	Тогда:
	\[
		\forall x \in U_\eps(a) \quad f(x) = f(a) + \suml_{k = 1}^n\frac{f^k(a)}{k!}(x - a)^k + \frac{1}{n!}\int_a^xf^{(n + 1)}(t)(x - t)^ndt
	\]
\end{theorem}

\begin{proof} Проведём индукцию по $n$:
	\begin{itemize}
		\item База $n = 1$:
		\[
			\int_a^x f'(t)(x - t)dt = f'(t)(x - t) \Bigg |_{t = a}^{t = x} + \int_a^x f'(t)dt = - f'(a)(x - a) + f(x) - f(a)
		\]
		
		\item Переход $n = m \Ra m + 1$:
		\[
			\frac{1}{m!}\int_a^x f^{(m + 1)}(t)(x - t)^m dt = \frac{1}{m!}\left(-f^{(m + 1)}(t)\frac{(x - t)^{m + 1}}{(m + 1)!} \Bigg|_a^x + \int_a^x \frac{1}{m + 1} f^{(m + 2)}(t)(x - t)^{m + 1} dt\right)
		\]
		Равенство достигается путём интегрирования по частям с занесением $(x - t)^m$ под знак дифференциала
	\end{itemize}
\end{proof} %Верхний, нижний пределы. Фундаментальность. Число е
    \begin{theorem}
	$\mu^*$ --- аддитивная мера на $M$.
\end{theorem}

\begin{proof}
	Так как мы уже доказали, что $M$ является алгеброй, то нам достаточно показать следующий факт:
	\[
		\forall A, B, C \in M,\ A = B \sqcup C \Ra \mu^*(A) = \mu^*(B) + \mu^*(C)
	\]
	Аналогичное неравенство в одну сторону уже есть, остаётся в другую.
	
	Зафиксируем $\eps > 0$. Поскольку $B, C \in M$, то $\exists B_\eps, C_\eps \in R(S)$ - приближения по определению лебеговой измеримости. Увидим следующее включение:
	\[
		A \tr (B_\eps \cup C_\eps) \subseteq (B \tr B_\eps) \cup (C \tr C_\eps) \Ra \mu^*(A \tr (B_\eps \cup C_\eps)) \le 2\eps
	\]
	Дополнительно воспользуемся следствием из полуаддитивности для оценки снизу $\mu^*(A)$:
	\[
		\mu^*(A) \ge \mu^*(B_\eps \cup C_\eps) - \mu^*(A \tr (B_\eps \cup C_\eps)) \ge \mu^*(B_\eps \cup C_\eps) - 2\eps
	\]
	Тут заметим, что $B_\eps, C_\eps \in R(S)$, стало быть и $B_\eps \cup C_\eps \in R(S)$. Иначе говоря, для этих множеств $\mu^* = \nu$ --- индуцированная мера (реальная!). Значит, мы можем воспользоваться формулой включений и исключений:
	\[
		\mu^*(A) \ge \underbrace{\mu^*(B_\eps) + \mu^*(C_\eps) - \mu^*(B_\eps \cap C_\eps)}_{\mu^*(B_\eps \cup C_\eps)} - 2\eps
	\]
	Аналогичной оценкой через следствие полуаддитивности мы пользуемся для $\mu^*(B_\eps), \mu^*(C_\eps)$:
	\begin{multline*}
		\mu^*(A) \ge \mu^*(B) - \mu^*(B \tr B_\eps) + \mu^*(C) - \mu^*(C \tr C_\eps) - \mu^*(B_\eps \cap C_\eps) - 2\eps \ge
		\\
		\mu^*(B) + \mu^*(C) - \mu^*(B_\eps \cap C_\eps) - 4\eps
	\end{multline*}
	Осталось показать, что $\mu^*(B_\eps \cap C_\eps)$ достаточно малая величина. Для этого снова заметим вложение:
	\[
		B_\eps \cap C_\eps \subseteq (B_\eps \bs B) \cup (C_\eps \bs C) \subseteq (B \tr B_\eps) \cup (C \tr C_\eps) \Ra \mu^*(B_\eps \cap C_\eps) \le 2\eps
	\]
	Итого $\mu^*(A) \ge \mu^*(B) + \mu^*(C) - 6\eps$ для $\forall \eps > 0$, что и требовалось.
\end{proof}

\begin{definition}
	$\mu^*$ на $M$ называется \textit{мерой Лебега} и обозначается $\mu$.
\end{definition}

\begin{theorem}
	$M$ является $\sigma$-алгеброй.
\end{theorem}

\begin{proof}
	Пусть $\{\cA_i\}_{i = 1}^\infty \subseteq M$ и $\cA := \bigcup_{i = 1}^\infty \cA_i$. В силу того, что $M$ уже алгебра, мы можем заменить $\cA_i$ на $B_i \in M$ так, что $\cA = \bscup_{i = 1}^\infty B_i$.
	\begin{enumerate}
		\item $\forall n \in \N\ \bscup_{i = 1}^n B_i \subseteq \cA$. Тогда между мерами есть соотношение следующего вида:
		\[
			\sum_{i = 1}^n \mu(B_i) = \mu\ps{\bscup_{i = 1}^n B_i} = \mu^*\ps{\bscup_{i = 1}^n B_i} \le \mu^*(A)
		\]
		Отсюда следует, что $\sum_{i = 1}^\infty \mu(B_i) \le \mu^*(A) < \infty$. Благодаря этому ряд сходится.
		
		\item В частности, нас интересует следующее свойство сходимости ряда:
		\[
			\forall \eps > 0\ \exists N \in \N \such \sum_{i = N + 1}^\infty \mu(B_i) < \frac{\eps}{2}
		\]
		Для основной части мы можем воспользоваться определением измеримости (ибо конечное дизъюнктное объединение тоже измеримо по Лебегу):
		\[
			\forall \eps > 0\ \exists C_{\eps / 2} \in M \such \mu\ps{C_{\eps / 2} \tr \bscup_{i = 1}^N B_i} < \eps / 2
		\]
		Дело остаётся за малым --- нашим кандидатом на приближение является $C_{\eps / 2}$, нужно проверить меру симметрической разности с $\cA$:
		\begin{multline*}
			\cA \tr C_{\eps / 2} \subseteq \ps{C_{\eps / 2} \tr \bscup_{i = 1}^N B_i} \cup \ps{\bscup_{i = N + 1}^\infty B_i} \Longrightarrow
			\\
			\mu^*(\cA \tr C_{\eps / 2}) \le \mu^*\ps{C_{\eps / 2} \tr \bscup_{i = 1}^N B_i} + \underbrace{\mu^*\ps{\bscup_{i = N + 1}^\infty B_i}}_{\le \sum_{i = N + 1}^\infty \mu^*(B_i)} < \frac{\eps}{2} + \frac{\eps}{2} = \eps
		\end{multline*}
	\end{enumerate}
\end{proof}

\begin{theorem}
	$\mu$ --- $\sigma$-аддитивная мера на $M$.
\end{theorem}

\begin{proof}
	Пусть $\cA \in M, \{\cA_i\}_{i = 1}^\infty \subseteq M$, причём $\cA = \bscup_{i = 1}^\infty \cA_i$. Тогда нам нужно доказать лишь $\mu(\cA) \ge \sum_{i = 1}^\infty \mu(\cA_i)$. Заметим такую вещь: так как $M$ теперь $\sigma$-алгебра, то
	\[
		\forall n \in \N \quad \cA = A_1 \sqcup \ldots \sqcup A_n \sqcup \ps{\bscup_{i = n + 1}^\infty A_i} \wedge \bscup_{i = n + 1}^\infty A_i \in M
	\]
	Отсюда по аддитивности:
	\[
		\forall n \in \N \quad \mu(\cA) = \sum_{i = 1}^n \mu(\cA_i) + \mu\ps{\bscup_{i = n + 1}^\infty A_i} \ge \sum_{i = 1}^n \mu(\cA_i)
	\]
	устремляя $n$ в бесконечность, получаем требуемое неравенство.
\end{proof}

\begin{note}
	Подведём итоги проделанной работы:
	\begin{align*}
		&{\text{$S$ с единицей $E$} \to R(S) \to M \text{--- $\sigma$-алгебра}}
		\\
		&{\text{$m$ --- $\sigma$-аддитивная мера} \to \nu \to \mu \text{--- $\sigma$-аддитивная мера}}
	\end{align*}
	Мы смогли продолжить $m$ на $\sigma(S)$. Почему? Ну просто потому что $\sigma(S) \subseteq M$ по определению. Возникает вопрос: <<А единственна ли мера, определенная на $\sigma(S)$?>> Окажется, что да.
\end{note}

\begin{example} (мера Бореля)
	Борелевской $\sigma$-алгеброй на отрезке $[a; b]$ будет наименьшая $\sigma$-алгебра, содержащая все открытые множества из $[a; b]$:
	\[
		\B_{[a; b]} = \sigma(\{A \subset [a;b] \such A \text{--- открытое}\}) = \sigma(\{\tbr{c; d} \such a \le c \le d \le b\})
	\]
	Мы уже показывали, что последняя система множеств являеся полукольцом с единицей $E = [a; b]$, $m(\tbr{c; d}) = d - c$ задаёт $\sigma$-аддитивную меру на этом полукольце, а потому построенная нами $\mu$ будет $\sigma$-аддитивной мерой на $\B_{[a; b]}$ --- \textit{мерой Бореля}.
\end{example}

\begin{example} (мера Лебега-Стильтеса)
	Рассмотрим $\R = (-\infty; +\infty)$ и функцию $\phi(x)$, удовлетворяющую таким свойствам:
	\begin{enumerate}
		\item $\phi(x)$ неубывающая
		
		\item $\phi(x)$ непрерывна справа (в любой точке и в $+\infty$)
		
		\item $\phi(x)$ ограничена
	\end{enumerate}
	Положим за полукольцо $S$ следующую систему множеств (про которую, конечно же, читатель должен доказать свойства полукольца):
	\[
		S = \{0\} \cup \Big\{(a; b] \such a \in \R \cup \{-\infty\}, b \in \R\Big\} \cup \Big\{(a; +\infty) \such a \in \R \cup \{-\infty\}\Big\}
	\]
	Зададим меру на $S$ таким образом:
	\begin{align*}
		&{m(\{0\}) = 0}
		\\
		&{m((a; b]) = \phi(b) - \phi(a)}
		\\
		&{m((a; +\infty)) = \phi(+\infty) - \phi(a)}
	\end{align*}
	То, что этим задана хотя бы мера, должно быть очевидно (ну или домашнее задание читателю), а вот $\sigma$-аддитивность неясна. Для этого, мы отдельно разберём конечный и бесконечный случаи:
	\begin{enumerate}
		\item $(a; b] = \bscup_{i = 1}^\infty (a_i; b_i]$. Чтобы показать равенство мер, мы сведём ситуацию к компактности:
		\begin{align*}
			&{\forall \eps > 0\ \exists a < d \le b \such \phi(d) - \phi(a) < \frac{\eps}{2}}
			\\
			&{\forall i \in \N \forall \eps > 0\ \exists c_i > b_i \such \phi(c_i) - \phi(b_i) < \frac{\eps}{2^{i - 1}}}
		\end{align*}
		В силу этих определений и свойств, верно вложение $[d; b] \subseteq \bigcup_{i = 1}^\infty (a_i; c_i)$
	\end{enumerate}
\end{example} %Предел функции.Критерий Коши
    \begin{note}
	Поздравляю, мы доказали Усиленный Закон Больших Чисел в одной из самых приятных формулировок. Теперь встаёт не менее важный прикладной вопрос: <<А можно ли что-то сказать про скорость сходимости УЗБЧ? Иначе говоря, какое $n$ нужно взять, чтобы добиться определённой точности в отклонении от матожидания?>> Это мы и будем изучать далее, для ответа на вопрос нужна Центральная Предельная Теорема
\end{note}

\section{Слабая сходимость вероятностных мер}

\begin{definition}
	Пусть $\{F_n\}_{n = 1}^\infty$ --- функции распределения на $\R$. \textit{Последовательность $F_n$ слабо сходится к функции распределения $F$}, если выполнено условие:
	\[
		\forall f \colon \R \to \R \text{ --- непрерывная ограниченная}\ \ \int_\R f(x)dF_n(x) \xrightarrow[n \to \infty]{} \int_\R f(x)dF(x)
	\]
	Обозначается как $F_n \to^w F$
\end{definition}

\begin{anote}
	$w$ от слова $weak$, полагаю.
\end{anote}

\begin{definition}
	\textit{Последовательность $\{F_n\}_{n = 1}^\infty$ функций распределений на $\R$ сходится в основном к функции распределения $F$}, если выполнено утверждение:
	\[
		\forall x \in C(F)\ \ F_n(x) \xrightarrow[n \to \infty]{} F(x)
	\]
	где $C(F)$ --- точки непрерывности $F$. Сходимость обозначается как $F_n \Ra F$
\end{definition}

\begin{definition}
	\textit{Последовательность вероятностных мер $\{P_n\}_{n = 1}^\infty$ слабо сходится к вероятностной мере $P$ (все в пространстве $(\R^m, \B(\R^m))$)}, если выполнено условие:
	\[
		\forall f \colon \R^m \to \R \text{ --- непрерывная ограниченная}\ \ \int_{\R^m} f(x)dP_n(x) \xrightarrow[n \to \infty]{} \int_{\R^m} f(x)dP(x)
	\]
	Обозначается как $P_n \to^w P$
\end{definition}

\begin{definition}
	\textit{Последовательность вероятностных мер $\{P_n\}_{n = 1}^\infty$ сходится в основном к вероятностной мере $P$ (все в пространстве $(\R^m, \B(\R^m))$)}, если выполнено условие:
	\[
		\forall B \in \B(\R^m), P(\vdelta B) = 0\ \ \ P_n(B) \xrightarrow[n \to \infty]{} P(B)
	\]
	Обозначается как $P_n \Ra P$
\end{definition}

\begin{note}
	Методом пристального взгляда можно заметить, что если $\{\xi_n\}_{n = 1}^\infty$ --- последовательность случайных величин, то верна эквивалентность:
	\[
		\xi_n \to^d \xi \Lra F_{\xi_n} \to^w F_\xi \Lra P_{\xi_n} \to^w P_\xi
	\]
	Это верно в силу теоремы о замене переменных в интеграле:
	\[
		\E f(\xi) = \int_\R f(x)dP_\xi(x) = \int_\R f(x)dF_\xi(x)
	\]
\end{note}

\begin{note}
	Основной факт, который известен про описанные выше слабые и в основном сходимости, состоит в их эквивалентности. Часть этого факта можно получить в следующей мощной теореме.
\end{note}

\begin{theorem} (Александрова, без доказательства)
	Пусть $\{P_n\}_{n = 1}^\infty, P$ --- вероятностные меры на пространстве $(\R^m, \B(\R^m))$. Тогда следующие условия эквивалентны:
	\begin{enumerate}
		\item $P_n \to^w P$
		
		\item $\varlimsup_{n \to \infty} P_n(A) \le P(A)$ для любого замкнутого $A \subseteq \R^m$
		
		\item $\varliminf_{n \to \infty} P_n(G) \ge P(G)$ для любого открытого $G \subseteq \R^m$
		
		\item $P_n \Ra P$
	\end{enumerate}
\end{theorem}

\begin{theorem} (об эквивалентности сходимостей)
	Пусть $\{P_n\}_{n = 1}^\infty, P$ --- вероятностные меры на $(\R, \B(\R))$, а $\{F_n\}_{n = 1}^\infty, F$ --- соответствующие функции распределения. Тогда следующие условия эквивалентны:
	\begin{enumerate}
		\item $P_n \to^w P$
		
		\item $P_n \Ra P$
		
		\item $F_n \to^w F$
		
		\item $F_n \Ra F$
	\end{enumerate}
\end{theorem}

\begin{proof}
	По теореме Александрова, уже $1 \Lra 2$. Более того, $1 \Lra 3$ эквивалентно по определениям. Осталось присоединить последнюю сходимость к остальным:
	\begin{itemize}
		\item[$2 \Ra 4$] Пусть $x \in C(F)$. Рассмотрим множество $B = \rsi{-\infty; x}$, тогда $\vdelta B = \{x\} \Ra P(\vdelta B) = F(x) - F(x) = 0$. Стало быть, в силу условия
		\[
			\forall x \in C(F) \quad F_n(x) = P_n\rsi{-\infty; x} \xrightarrow[n \to \infty]{} P\rsi{-\infty; x} = F(x)
		\]
		
		\item[$4 \Ra 2$] Воспользуемся третьим эквивалентным свойством из теоремы Александрова. Пусть $G \subseteq \R$ --- произвольное открытое множество. Тогда известно, что $G$ есть не более чем счётное число непересекающихся интервалов $G = \bscup_{k = 1}^\infty (a_k; b_k)$. Зафиксируем $\eps > 0$ и для любого $k \in \N$ подберём полуинтервал $\rsi{a'_k; b'_k} \subset (a_k; b_k)$ такой, что $P(a_k; b_k) \le P\rsi{a'_k; b'_k} + \eps / 2^k$ и дополнительно $a'_k, b'_k \in C(F)$. Такой выбор возможен, коль скоро $P$ непрерывна, а точек разрыва $F$ не более чем счётное число (в силу монотонности). Осталось явно проверить требуемое ($N$ --- произвольное натуральное число):
		\begin{multline*}
			\varliminf_{n \to \infty} P_n(G) = \varliminf_{n \to \infty} \sum_{k = 1}^\infty P_n(a_k; b_k) \ge
			\\
			\varliminf_{n \to \infty} \sum_{k = 1}^N P_N(a_k; b_k) \ge \sum_{k = 1}^N \varliminf_{n \to \infty} P_n(a_k; b_k) \ge
			\\
			\sum_{k = 1}^N \varliminf_{n \to \infty} P_n\rsi{a'_k; b'_k} = \sum_{k = 1}^N \varliminf_{n \to \infty} (F_n(b'_k) - F_n(a'_k)) =
			\\
			\sum_{k = 1}^N (F(b'_k) - F(a'_k)) \ge \sum_{k = 1}^N \ps{P(a_k; b_k) - \frac{\eps}{2^k}} \ge \sum_{k = 1}^N P(a_k; b_k) - \eps
		\end{multline*}
		Устремляя $N$ в бесконечность, а затем и $\eps$ к нулю, получаем требуемое.
	\end{itemize}
\end{proof}

\begin{corollary}
	Имеет место эквивалентность (которое часто берут за определение сходимости по распределению): $\xi_n \to^d \xi \Lra \forall x \in C(F)\ \ F_{\xi_n}(x) \to F_\xi(x)$
\end{corollary}

\begin{note}
	В чём состоит смысл сходимости по распределению? Это позволяет делать аппроксимацию этих самых распределений.
	
	Пусть мы хотим вычислить распределение случайной величины $\xi$. Тогда, пусть $\eta_n \to^d \eta$, где распределение $\eta$ известно. Если окажется, что начиная с некоторого $n_0$ распределения $\eta_n$ и $\xi$ равны (или сильно близки), то можно считать, что $F_\xi(x) \sim F_\eta(x)$
\end{note}

\begin{example}
	В курсе ОВиТМа уже встречалось это соображение в виде теоремы Пуассона.
\end{example}

\section{Характеристические функции}

\begin{definition}
	Пусть $\xi$ --- случайная величина. Тогда \textit{характеристической функцией случайной величины $\xi$} называется комплексная функция $\phi_\xi(t) \colon \R \to \Cm$:
	\[
		\forall t \in \R\ \ \phi_\xi(t) := \E e^{i\xi t}
	\]
\end{definition}

\begin{note}
	Интеграл (матожидание) от комплексной функции берётся по такой формуле:
	\[
		\E e^{it\xi} = \E\cos(t\xi) + i\E\sin(t\xi)
	\]
\end{note}

\begin{definition}
	Пусть $\xi \colon \R^n \to \R$ --- случайный вектор. Тогда \textit{характеристической функцией случайного вектора $\xi$} называется комплекснозначная функция $\phi_\xi \colon \R^n \to \R$:
	\[
		\forall t \in \R^n\ \ \phi_\xi(t) := \E e^{i\tbr{\xi, t}}
	\]
\end{definition}

\subsubsection*{Примеры характеристических функций}

\begin{enumerate}
	\item Пусть $\xi \sim Bin(n, p)$. Тогда 
	\[
		\phi_\xi(t) = \E e^{it\xi} = \sum_{k = 0}^n e^{itk}P(\xi = k) = \sum_{k = 0}^n e^{itk} C_n^k p^k(1 - p)^{n - k} = (e^{it}p + 1 - p)^n
	\]
	
	\item Пусть $\xi \sim Exp(\alpha)$. Тогда
	\[
		\phi_\xi(t) = \E e^{it\xi} = \int_0^{+\infty} e^{itx}\alpha e^{-\alpha x}dx = \alpha \int_0^{+\infty} e^{(it - \alpha)x}dx = \frac{\alpha}{\alpha - it}
	\]
	
	\item Пусть $\xi \sim N(0, 1)$. Тогда
	\[
		\phi_\xi(t) = \E e^{it\xi} = \int_{-\infty}^{+\infty} e^{itx} \frac{1}{\sqrt{2\pi}} e^{-x^2 / 2}dx
	\]
	Если мы разложим интеграл на действительную и мнимую часть, то заметим, что в мнимой части подыинтегральная функция будет нечётной, стало быть ноль. Итого:
	\[
		\phi_\xi(t) = \int_{-\infty}^{+\infty} \cos(tx) \frac{1}{\sqrt{2\pi}} e^{-x^2 / 2}dx
	\]
	Получился очень хорошо параметризованный интеграл (подыинтегральная функция бесконечно гладкая), поэтому будем считать его через производную:
	\[
		\phi'_\xi(t) = \frac{1}{\sqrt{2\pi}}\int_{-\infty}^{+\infty} \sin(tx) \underbrace{(-x) e^{-x^2 / 2}}_{(e^{-x^2 / 2})'}dx = \frac{1}{\sqrt{2\pi}} \ps{0 - t\int_{-\infty}^{+\infty} \cos(tx)e^{-x^2 / 2}dx} = -t\phi_\xi(t)
	\]
	Отсюда $\phi_\xi(t) = Ce^{-t^2 / 2}$. Начальное условие $\phi_\xi(0) = 1$, в итоге $\phi_\xi(t) = e^{-t^2 / 2}$
\end{enumerate}

\begin{definition}
	Пусть $P$ --- вероятностная мера на $(\R^m, \B(\R^m))$. Тогда \textit{характеристической функцией меры $P$} называется $\phi_P \colon \R^m \to \R$, определённая следующим образом:
	\[
		\forall t \in \R^m\ \ \phi_P(t) := \int_{\R^m} e^{i\tbr{t, x}}dP(x)
	\]
\end{definition}

\begin{note}
	Несложно заметить, что определения характеристических функций для меры и случайных векторов согласованы, а именно (в случае случайных величин):
	\[
		\text{Хар-ая функция $\xi$} = \E e^{it\xi} = \int_\R e^{itx} dP_\xi(x) = \text{ хар-ая функция распределения $\xi$}
	\]
\end{note} %Односторонние пределы. Непрерывность. Точки разрыва
    \begin{proof}
	Пусть $\{d_1, \ldots, d_s\}$ --- это все делители числа $n \in \N$. Тогда множество $V$ можно представить как объединение множеств слов $V_i$ длины $n$ с одинаковым периодом $d_i$:
	\begin{align*}
		&V = V_1 \sqcup \ldots \sqcup V_s,
		\\
		&|V| = |V_1| + \ldots + |V_s|.
	\end{align*}
	
	Обозначим за $W_i$ --- множество слов длины $d_i$ с периодом $d_i$. Понятно, что
	\[
		|W_i| = |V_i| \Ra |V| = |W_1| + \ldots + |W_s|
	\]
	
	Ещё введём понятие $U_i$ --- это множество циклических слов длины $d_i$ и периодом $d_i$. Тогда
	\[
		|W_i| = d_i \cdot |U_i|.
	\]
	И обозначим $|U_i| =: M(d_i)$. Теперь $|V|$ можно записать как
	\[
		|V| = r^n = \suml_{i = 1}^s d_i |U_i| = \suml_{d \mid n} d \cdot M(d).
	\]
	Заметим, что если ввести функции
	\begin{align*}
		&f(n) = r^n,
		\\
		&g(n) = n \cdot M(n),
	\end{align*}
	то $M(n)$ можно посчитать через обращение Мёбиуса:
	\begin{align*}
		&g(n) = \suml_{d \mid n} \mu(d) \cdot f(n / d),
		\\
		&M(n) = \frac{1}{n}\suml_{d \mid n} \mu(d) \cdot r^{n / d}.
	\end{align*}
	Отсюда получаем
	\[
		T_r(n) = \suml_{d \mid n} M(d) = \suml_{d \mid n} \frac{1}{d} \left(\suml_{d' \mid d} \mu(d') r^{d / d'}\right)
	\]
\end{proof}

\subsection{Обобщённая Мёбиуса}

\begin{definition}
	 \textit{Функцией Мёбиуса на частично упорядоченном множестве} (ЧУМе) $\trbr{\mcP, \preceq}$ называется функция $\mu$, определяемая как
	 \[
	 	\mu(x, y) = \System{
	 		&{1,\ x = y,}
	 		\\
	 		&{-\suml_{x \preceq z \prec y} \mu(x, z),\ x \prec y.}
 		}
	 \]
	 При этом считается, что $\forall y \in \mcP$ существует лишь конечное число $x \in \mcP$ таких, что $x \preceq y$.
\end{definition}

\begin{theorem} (Связь между обобщённой и стандартной функцией Мёбиуса)
	Переобозначим стандартную функцию Мёбиуса за $\hat{\mu}$. Тогда, если $\trbr{\mcP, \preceq} = \trbr{\N, |}$, то
	\[
		\mu(y, x) = \hat{\mu}\left(\frac{x}{y}\right).
	\]
\end{theorem}

\begin{proof}
	Докажем теорему при помощи индукции по $\frac{x}{y}$:
	\begin{itemize}
		\item База: $\frac{x}{y} = 1 \Ra x = y$. Тогда
		\[
			\mu(x, x) = 1 = \hat{\mu}(1) \text{ --- верно}.
		\]
		
		\item Переход: $y \prec x \lra y | x$ и $y \neq x$. Значит
		\[
			x = y \cdot p_1^{\alpha_1} p_2^{\alpha_2} \ldots p_s^{\alpha_s},\ \alpha_i \ge 1.
		\]
		Тогда
		\[
			\mu(y, x) = -\suml_{y \preceq z \prec x} \mu(y, z).
		\]
		Из определения $z$ следует, что $\frac{z}{y} < \frac{x}{y}$. То есть мы можем применить предположение индукции:
		\[
			\mu(y, x) = -\suml_{y \preceq z \prec x} \hat{\mu}\left(\frac{z}{y}\right).
		\]
		Так как $y \preceq z$, то $z$ содержит в себе правую часть из выражения $x$ и можно записать следующее:
		\[
			\mu(y, x) = -\suml_{{0 \le \beta_i \le \alpha_i} \over {\exists j : \beta_j < \alpha_j}} \hat{\mu}(p_1^{\beta_1} \ldots p_s^{\beta_s}).
		\]
		Рассмотрим частный случай, когда $(a_1, \ldots, a_s) = (1, \ldots, 1)$. Тогда значения кортежа $(\beta_1, \ldots, \beta_s)$ являются битовым представлением числа от $0$ до $2^s - 2$. Отсюда имеем:
		\[
			\mu(y, x) = -\suml_{{0 \le \beta_i \le 1} \over {\exists \beta_j = 0}} \hat{\mu}(p_1^{\beta_1} \ldots p_s^{\beta_s}) = -\suml_{k = 0}^{s - 1} (-1)^k \cdot C_s^k = - \left(0 - (-1)^s \cdot C_s^s\right) = (-1)^s = \hat{\mu}\left(\frac{x}{y}\right).
		\]
		Теперь докажем другой случай, когда $\exists \alpha_j \ge 2$: если рассматривать $z$ такое, что в нём содержится $\beta_j \ge 2$, то всё слагаемое сразу будет ноль. Значит снова $0 \le \beta_i \le 1$, при этом кортеж уже может быть представлением числа $2^s - 1$:
		\[
			\mu(y, x) = -\suml_{{0 \le \beta_i \le 1}} \hat{\mu}(p_1^{\beta_1} \ldots p_s^{\beta_s}) = -\suml_{k = 0}^s (-1)^k C_s^k = 0 = \hat{\mu}\left(\frac{x}{y}\right).
		\]
	\end{itemize}
\end{proof}

\begin{theorem} (Обобщённое обращение Мёбиуса) \label{globMebius}
	Пусть $\trbr{\mcP, \preceq}$ --- некоторый ЧУМ. Пусть $f\colon \mcP \ra \Cm$, $g(x) := \suml_{y \preceq x} f(y)$. Тогда
	\[
		f(x) = \suml_{y \preceq x} \mu(y, x) g(y).
	\]
\end{theorem}

\begin{lemma}
	\[
		\suml_{x \preceq y \preceq z} \mu(y, z) = \mathbb{I}_{x = z} = \System{
			&{1,\ x = z,}
			\\
			&{0,\ x \neq z.}
		}
	\]
\end{lemma}

\begin{proof}~
	\begin{enumerate}
		\item Если $x = z$, то
		\[
			\suml_{x \preceq y \preceq z} \mu(y, z) = \mu(x, z) = 1.
		\]
		
		\item Иначе $x \prec z$. Проведём доказательство индукция по длине максимальной цепи вида
		\[
			x \prec \ldots \prec \ldots \prec z.
		\]
		\begin{enumerate}
			\item База: такая цепочка имеет длину 2, то есть вида $x \prec z$,
			\[
				\suml_{x \preceq y \preceq z} \mu(y, z) = \mu(x, z) + \mu(z, z) = \left(-\suml_{x \preceq w \prec z} \mu(x, w)\right) + 1 = 0.
			\]
			
			\item Переход: цепочка имеет длину $\ge 3$, а для меньших уже доказано:
			\begin{multline*}
				\suml_{x \preceq y \preceq z} \mu(y, z) = \suml_{x \preceq y \prec z} \mu(y, z) + \mu(z, z) = 1 - \suml_{x \preceq y \prec z} \suml_{y \preceq u \prec z} \mu(y, u) = 1 - \suml_{x \preceq y \preceq u \prec z} \mu(y, u) =
				\\ =
				1 - \suml_{x \preceq u \prec z} \suml_{x \preceq y \preceq u} \mu(y, u) = 1 - \suml_{x \preceq y \preceq x} \mu(y, x) - \suml_{x \prec u \prec z} \suml_{x \preceq y \preceq u} \mu(y, u).
			\end{multline*}
			Так как в последнем слагаемом $x \prec u \prec z$, то максимальная цепочка, соединяющая $x$ и $u$, короче, чем $x$ и $z$. Значит, можно воспользоваться предположением индукции:
			\[
				\suml_{x \preceq y \preceq u} \mu(y, u) = \mathbb{I}_{x = u} = 0,
			\]
			то есть
			\[
				\suml_{x \preceq y \preceq z} \mu(y, z) = 1 - 1 - 0 = 0.
			\]
		\end{enumerate}
	\end{enumerate}
\end{proof}

\begin{proof} (Теоремы \ref{globMebius})
	Поступаем аналогично стандартной функции Мёбиуса:
	\begin{multline*}
		\suml_{y \preceq x} \mu(y, x) g(y) = \suml_{y \preceq x} \mu(y, x) \suml_{z \preceq y} f(z) = \suml_{z \preceq y \preceq x} \mu(y, x)f(z) = \suml_{z \preceq x} f(z) \cdot \suml_{z \preceq y \preceq x} \mu(y, x) =
		\\ =
		f(x) + \suml_{z \prec x}f(z) \cdot  \underbrace{\suml_{z \preceq y \preceq x} \mu(y, x)}_0 = f(x).
	\end{multline*}
\end{proof} %Свойства непрерывности.
    \begin{corollary}
	Большинство интегральных соотношений, похожих на теорему Стокса-Пуанкаре, являются просто её следствием:
	\begin{enumerate}
		\item $m = 2$, $n = 3$. Тогда формула из теоремы Стокса-Пуанкаре называется просто формулой Стокса
		
		\item $m = 3 = n$. В этом особом случае мы говорим о 2-формах. Обозначим одну из таких $\Omega$ и пусть она имеет следующий вид:
		\[
			\Omega = P dy \wedge dz + Q dz \wedge dx + R dx \wedge dy
		\]
		Тогда дифференциал может быть лаконично записан так:
		\[
			d\Omega = \ps{\pd{P}{x} + \pd{Q}{y} + \pd{R}{z}}dx \wedge dy \wedge dz
		\]
		Если подставить эти выражения в формулу Стокса-Пуанкаре, то получим \textit{формулу Гаусса-Остроградского в терминах дифференциальных форм}:
		\[
			\int_M \ps{\pd{P}{x} + \pd{Q}{y} + \pd{R}{z}} dx \wedge dy \wedge dz = \int_{\vdelta M} P dy \wedge dz + Q dz \wedge dx + R dx \wedge dy
		\]
		
		\item $m = 2 = n$. В этом случае у нас всего лишь 1-формы. Пусть она имеет такой вид:
		\[
			\Omega = Pdx + Qdy \Lora d\Omega = \ps{\pd{Q}{x} - \pd{P}{y}} dx \wedge dy
		\]
		Если подставить эти выражения в формулу Стокса-Пуанкаре, то получится \textit{формула Грина в терминах дифференциальных форм}:
		\[
			\int_{\vdelta M} Pdx + Qdy = \int_M \ps{\pd{Q}{x} - \pd{P}{y}} dx \wedge dy
		\]
	\end{enumerate}
\end{corollary}

\begin{note}
	На время забудем о всяких многообразиях, а вернёмся к простому евклидову пространству $\R^m$ со стандартным скалярным произведением $\tbr{\cdot, \cdot}$ и ортонормированным базисом $e_0$.
	
	Если я хочу выяснить объём призмы $\Pi$, натянутой на векторы $\{\vv{H}_i\}_{i = 1}^m$, то это записывается следующим образом:
	\[
		\mu(\Pi) = \vol(\vv{H}_1, \ldots, \vv{H}_m) = |\det(H_i^j)|
	\]
	где $\vv{H}_i = H_i^j \vv{e}_0^j$ (используем соглашение Эйнштейна). С другой стороны, в курсе алгебры мы говорили о матрице Грама для системы векторов, заданной следующим образом:
	\[
		\Gamma(\vv{H}_1, \ldots, \vv{H}_m) = \Matrix{
			&\ntbr{\vv{H}_1, \vv{H}_1}& &\cdots& &\ntbr{\vv{H}_1, \vv{H}_m}
			\\
			&\quad\;\ \vdots& &\ddots& &\quad\;\ \vdots
			\\
			&\ntbr{\vv{H}_m, \vv{H}_1}& &\cdots& &\ntbr{\vv{H}_m, \vv{H}_m}
		}
	\]
	В силу того, что мы используем стандартное скалярное произведение, эта матрица является произведением матрицы столбцов системы векторов на саму себя транспонированную. Иначе говоря, имеет место равенство:
	\[
		(\det(H_i^j))^2 = \det\Gamma(\vv{H}_1, \ldots, \vv{H}_m)
	\]
	Следовательно, искомый объём призмы можно записать ещё и так:
	\[
		\mu(\Pi) = \vol{\vv{H}_1, \ldots, \vv{H}_m} = |\det(H_i^j)| = \sqrt{\det\Gamma(\vv{H}_1, \ldots, \vv{H}_m)}
	\]
\end{note}

\begin{anote}
	Неформальное определение многообразия заключается в том, что это пространство, локально схожее с евклидовым (есть скалярное произведение в каждой точке в касательном пространстве, в нашем случае). Совершенно не обязательно, что многообразие вложено в некоторое всеобъёмлющее пространство. Поэтому, сохраняя определённую общность рассуждений, мы зададим скалярное произведение в каждой точке многообразия по-отдельности.
	
	Словестное определение \textit{размерности многообразия} такое: <<Размерностью многообразия называется размерность евклидова пространства, с которым оно локально сходно>>. Самым простым примером может послужить сфера, чья поверхность локально напоминает плоскость (в физическом проявлении --- наша планета Земля).
	
	Поэтому, как я понимаю, размерность многообразия в наших ситуациях соответствует размерности касательного пространства в каждой точке.
\end{anote}

\begin{definition}
	Элементарное многообразие $M \subseteq \R^n$ размерности $m$ называется \textit{римановым}, если задана непрерывная (по $x \in M$) положительно определенная билинейная форма $\ntbr{\cdot, \cdot}_x$ на каждом касательном пространстве $T(x)$ --- \textit{риманова метрика}.
\end{definition}

\begin{definition}
	\textit{(Римановым) объёмом призмы}, натянутой на вектора $\{\vv{K}_i\}_{i = 1}^m$ пространства $T(x)$, называется следующая величина:
	\[
		\vol(\vv{K}_1, \ldots, \vv{K}_m) = \sqrt{\det\ntbr{\vv{K}_i, \vv{K}_j}_x}
	\]
\end{definition}

\begin{definition}
	\textit{Формой ориентированного объёма на римановом ориентированном многообразии} $M$ называется такая $m$-форма $V$, что
	\[
		\forall \{\vv{G}_i\}_{i = 1}^m \subset T(x)\ \ V(x)(\vv{G}_1, \ldots, \vv{G}_m) = \pm \vol(\vv{G}_1, \ldots, \vv{G}_m)
	\]
	причём знак выбирается в соответствии с ориентацией базиса $(\vv{G}_1, \ldots, \vv{G}_m)$ пространства $T(x)$.
\end{definition}

\begin{anote}
	Естественно, в определении выше мы не требуем, что $\{\vv{G}_i\}_{i = 1}^m$ является базисом в $T(x)$. Однако, если это не базис, то это линейно зависимая комбинация векторов, чей объём обязательно ноль. Для нуля, понятное дело, знак смысла не имеет.
\end{anote}

\begin{proposition}
	Если $M \subseteq \R^n$ --- риманово многообразие размерности $m$ с римановой метрикой, индуцированной стандартным скалярным произведением в $\R^n$, а также $\phi$ --- положительная параметризация, то имеет место формула:
	\[
		V(x) = \sqrt{\det\tbr{\pd{\phi}{u_i}(u), \pd{\phi}{u_j}(u)}_x} \cdot \psi^*(du^1 \wedge \ldots \wedge du^m),\ x = \phi(u)
	\]
\end{proposition}

\begin{anote}
	Отмечу явный вид вектора в определителе:
	\[
		\pd{\phi}{u_i}(u) = \ps{\pd{\phi_1}{u_i}(u), \ldots, \pd{\phi_n}{u_i}(u)}^T
	\]
	Из того, что $x = \phi(u)$ следует, что весь квадратный корень является просто функцией, зависящей только от $x$.
\end{anote}

\begin{proof}
	Заметим, что при любом фиксированном $x = \phi(u)$ линейное пространство $m$-форм над $T(x)$ одномерно. Это значит, что форма ориентированного объёма может быть выражена через базисную форму и какой-то коэффициент. Например, так:
	\[
		V(x) = \alpha(x) \cdot \psi^*(du^1 \wedge \ldots \wedge du^m)(x)
	\]
	Посмотрим на $\{\vv{H}_i\}_{i = 1}^m$ --- ортонормированный базис в пространстве $\R^m$, двойственный к $\{du^i\}_{i = 1}^m$. Тогда у нас есть репер $\{\phi'(u)\vv{H}_i\}_{i = 1}^m$, чей конкретный набор векторов в $x = \phi(u)$ образует положительный базис (в силу определения положительной параметризации). С одной стороны (по определению):
	\[
		V(x)(\phi'(u)\vv{H}_1, \ldots, \phi'(u)\vv{H}_m) = \sqrt{\det\ntbr{\phi'(u)\vv{H}_i, \phi'(u)\vv{H}_j}_x}
	\]
	где отдельно взятый вектор $\phi'(u)\vv{H}_i$ будет просто вектором $\pd{\phi}{u_i}(u)$. С другой стороны, можем подставить эти вектора в найденный вид формы $V$:
	\begin{multline*}
		V(x)(\phi'(u)\vv{H}_1, \ldots, \phi'(u)\vv{H}_m) = \alpha(x) \cdot \psi^*(du^1 \wedge \ldots \wedge du^m)(x)(\phi'(u)\vv{H}_1, \ldots, \phi'(u)\vv{H}_m) =
		\\
		\alpha(x) \cdot (du^1 \wedge \ldots \wedge du^m)(\psi'(x)(\phi'(u)\vv{H}_1), \ldots, \psi'(x)(\phi'(u)\vv{H}_m))
	\end{multline*}
	Рассмотрим отдельно взятый аргумент. По сути у нас записана сложная производная от композиции $\psi \circ \phi$, а они обратны друг к другу:
	\[
		\psi'(x)(\phi'(u)\vv{H}_i) = \psi'(\phi(u))(\phi'(u)\vv{H}_i) = \vv{H}_i
	\]
	Стало быть, так как $\{\vv{H}_i\}_{i = 1}^m$ был ортонормированным базисом, то значение формы $du^1 \wedge \ldots \wedge du^m$ на нём будет просто единицей. Таким образом, мы нашли коэффициент $\alpha(x)$:
	\[
		\alpha(x) = \sqrt{\det\tbr{\pd{\phi}{u_i}(u), \pd{\phi}{u_j}(u)}_x}
	\]
\end{proof}

\begin{note}
	Для отрицательной параметризации нужно поставить знак минус перед радикалом.
\end{note}

\begin{definition}
	\textit{Интегралом от функции $f$ по римановой ориентированной клетке} $M$ называется следующий интеграл:
	\[
		\int_M f := \int_M fV = \int_K f(\phi(u))\sqrt{\det\tbr{\pd{\phi}{u_i}(u), \pd{\phi}{u_j}(u)}_{\phi(u)}}d\mu(u)
	\]
	где $K$ --- параметризующий куб, $\phi$ --- положительная параметризация.
\end{definition}

\begin{note}
	Несмотря на то что финальная формула зависит от параметризации, понятие формы ориентированного объёма определялось независимо от неё, а потому определение интеграла по клетке инвариантно.
\end{note}

\begin{definition}
	\textit{Мерой Лебега куска клетки} $M' \subseteq M$ называется следующий интеграл:
	\[
		\int_M \chi_{M'}(x)d\mu_M(x)
	\]
	где $\chi_{M'}(x)$ --- характеристическая функция множества $M'$, определенная на клетке:
	\[
		\chi_{M'}(x) = \System{
			&{1,\ x \in M'}
			\\
			&{0,\ x \in M \bs M'}
		}
	\]
\end{definition}

\begin{definition}
	Если $m = 1$, $n = 3$, то $M$ становится одномерной клеткой $M = \{\phi(u) \in \R^n \colon u \in [0; 1]\}$. \textit{Криволинейным интегралом 1-го рода} называется следующий интеграл:
	\[
		\int_M f(x, y, z)ds := \int_M f(x, y, z)d\mu_M
	\]
\end{definition}

\begin{note}
	Продолжим существовать в рамках предыдущего определения. Рассмотрим положительную параметризацию $\phi(u) = (x(u), y(u), z(u))^T$. Выясним явную формулу в рамках этой параметризации для криволинейного интеграла первого рода. Определитель вырождается в следующее:
	\[
		\det\tbr{\frac{d\phi}{du}(u), \frac{d\phi}{du}(u)}_{\phi(u)} = (x'(u))^2 + (y'(u))^2 + (z'(u))^2
	\]
	Получается вот такая позитивная формула:
	\[
		\int_M f(x, y, z)ds = \int_0^1 f(x(u), y(u), z(u))\sqrt{(x'(u))^2 + (y'(u))^2 + (z'(u))^2}d\mu(u)
	\]
	Причём, если $f = 1$, то значение интеграла является \textit{длиной соответствующей кривой}.
\end{note}

\begin{note}
	Аналогично определяется криволинейный интеграл 2-го рода при $m = 1$, $n = 2$.
\end{note}

\begin{definition}
	Если $m = 2$, $n = 3$, то $M$ становится двумерной клеткой $M$. \textit{Поверхностным интегралом 1-го рода} называется следующий интеграл:
	\[
		\iint_M f(x, y, z)dS := \int_M f(x, y, z)d\mu_M
	\]
\end{definition}

\begin{note}
	Если $m = 2$, $n = 3$, то $M$ --- двумерная клетка. Положительную параметризацию $\phi$ запишем так:
	\[
		\phi(u, v) = (x(u, v), y(u, v), z(u, v))^T
	\]
	Тогда $g(u, v)$ будет функцией матрицы, от которой мы хотим посчитать определитель:
	\[
		g(u, v) = \Matrix{
			&\tbr{\pd{\phi}{u}(u, v), \pd{\phi}{u}(u, v)}& &\tbr{\pd{\phi}{u}(u, v), \pd{\phi}{v}(u, v)}
			\\
			&\tbr{\pd{\phi}{v}(u, v), \pd{\phi}{u}(u, v)}& &\tbr{\pd{\phi}{v}(u, v), \pd{\phi}{v}(u, v)}
		}
	\]
	Карл Гаусс ввёл следующие обозначения для частей определителя этой формулы (скобки $(u, v)$ опускаются):
	\begin{align*}
		&{E = \tbr{\pd{\phi}{u}, \pd{\phi}{u}} = \ps{\pd{x}{u}}^2 + \ps{\pd{y}{u}}^2 + \ps{\pd{z}{u}}^2}
		\\
		&{G = \tbr{\pd{\phi}{v}, \pd{\phi}{v}} = \ps{\pd{x}{v}}^2 + \ps{\pd{y}{v}}^2 + \ps{\pd{z}{v}}^2}
		\\
		&{F = \tbr{\pd{\phi}{u}, \pd{\phi}{v}} = \pd{x}{u} \cdot \pd{x}{v} + \pd{y}{u} \cdot \pd{y}{v} + \pd{z}{u} \cdot \pd{z}{v}}
	\end{align*}
	Тогда $\det g(u, v) = EG - F^2$, а соответствующий поверхностный интеграл имеет такой вид:
	\[
		\iint_M f(x, y, z)dS = \iint_K f(x(u, v), y(u, v), z(u, v))\sqrt{EG - F^2}d\mu(u, v)
	\]
	Причём, если $f = 1$, то значение интеграла является \textit{площадью соответствующей поверхности}.
\end{note}

\begin{note}
	Далее, если не оговорено обратного, мы живём в ситуации $m = 2$, $n = 3$ с двумерным многообразием $M$. При наличии параметризации $\phi$ мы используем следующие обозначения для координат:
	\[
		\phi(u, v) = (x, y, z)
	\]
\end{note}

\begin{definition}
	Пусть $(e_1(r), e_2(r))$ --- положительный базис касательного пространства $T(r)$. \textit{Положительной нормалью к поверхности в точке} $r \in M$ называется единичный вектор $n$, обладающий двумя свойствами:
	\begin{enumerate}
		\item $n \bot T(r)$
		
		\item $(n, e_1(r), e_2(r))$ --- положительный базис в $\R^3$
	\end{enumerate}
\end{definition}

\textcolor{red}{Тут должна быть картинка многообразия и этих векторочков}

\begin{proposition}
	Пусть $M$ --- ориентированная двумерная клетка, $V$ --- соответствующая форма ориентированного объёма, $(\vv{i}, \vv{j}, \vv{k})$ --- орты в $\R^3$ (ортонормированный базис) и $\vv{n}$ --- вектор положительной нормали. Тогда в каждой точке $r \in M$ верны следующие формулы:
	\begin{align*}
		&{dx \wedge dy|_M = \cos(\vv{n}, \vv{k})V}
		\\
		&{dy \wedge dz|_M = \cos(\vv{n}, \vv{i})V}
		\\
		&{dz \wedge dx|_M = \cos(\vv{n}, \vv{j})V}
	\end{align*}
\end{proposition}

\begin{proof}
	Проверим лишь первое равенство, ибо остальные делаются аналогично. Пусть $\phi$ --- положительная параметризация. Имеет место эквивалентность:
	\[
		dx \wedge dy|_M = \cos(\vv{n}, \vv{k})V \Longleftrightarrow \phi^*(dx \wedge dy|_M) = \phi^*(\cos(\vv{n}, \vv{k})V)
	\]
	Зафиксируем произвольную точку $(x_0, y_0, z_0) = \phi(u_0, v_0)$ (чтобы не запутаться в обозначениях). С одной стороны:
	\begin{multline*}
		\phi^*(dx \wedge dy|_M)(x_0, y_0, z_0) = dx(u_0, v_0) \wedge dy(u_0, v_0) = \ps{\pd{x}{u}du + \pd{x}{v}dv} \wedge \ps{\pd{y}{u}du + \pd{y}{v}dv} =
		\\
		\ps{\pd{x}{u} \cdot \pd{y}{v} - \pd{x}{v} \cdot \pd{y}{u}}du \wedge dv
	\end{multline*}
	где производные взяты в точке $(u_0, v_0)$, естественно (у $du \wedge dv$ первый аргумент, коим должна быть та же точка, по общему соглашению опускаем).
	
	В силу положительной параметризации, вектора $\pd{\phi}{u}(u_0, v_0)$ и $\pd{\phi}{v}(u_0, v_0)$ образуют положительный базис касательного пространства $T(x_0, y_0, z_0)$. Тогда положительную нормаль $\vv{n}$ в этой точке можно записать известным соотношением:
	\[
		\vv{n} = \frac{\sbr{\pd{\phi}{u}, \pd{\phi}{v}}}{\md{\sbr{\pd{\phi}{u}, \pd{\phi}{v}}}}
	\]
	С другой стороны, посчитаем косинус и выпишем явно форму ориентированного объёма:
	\begin{itemize}
		\item \[
			\cos(\vv{n}, \vv{k}) = \frac{\ntbr{\vv{n}, \vv{k}}}{\md{\sbr{\pd{\phi}{u}, \pd{\phi}{v}}} \cdot 1} = \frac{1}{\md{\sbr{\pd{\phi}{u}, \pd{\phi}{v}}}} \cdot \ps{\pd{x}{u} \cdot \pd{y}{v} - \pd{x}{v} \cdot \pd{y}{u}}
		\]
		Явно векторное произведение, как мы знаем, можно записать таким мнемоническим определителем:
		\[
			\sbr{\pd{\phi}{u}, \pd{\phi}{v}} = \Det{
				&\vv{i}& &\vv{j}& &\vv{k}
				\\
				&\pd{x}{u}& &\pd{y}{u}& &\pd{z}{u}
				\\
				&\pd{x}{v}& &\pd{y}{v}& &\pd{z}{v}
			}
		\]
		А из линейной алгебры мы знаем, что $|[\vv{a}, \vv{b}]|^2 = \ntbr{\vv{a}, \vv{a}}\ntbr{\vv{b}, \vv{b}} - \ntbr{\vv{a}, \vv{b}}^2$.
		
		\item Последнее равенство в точности совпадает с $EG - F^2$, выражающим квадрат определителя матрицы Грама для этих базисных векторов. При этом
		\[
			V = \sqrt{EG - F^2} \psi^*(du \wedge dv)
		\]
	\end{itemize}
	Осталось собрать всё вместе:
	\[
		\cos(\vv{n}, \vv{k})V = \frac{1}{\sqrt{EG - F^2}} \cdot \ps{\pd{x}{u} \cdot \pd{y}{v} - \pd{x}{v} \cdot \pd{y}{u}} \cdot \sqrt{EG - F^2} \psi^*(du \wedge dv)
	\]
	Что произойдёт при переносе $\phi^*$ с этим выражением, тривиально.
\end{proof}

\begin{note}
	Начиная отсюда $m = 1$.
\end{note}

\begin{reminder}
	Кривые изучались в конце первого семестра. Все связанные определения можно найти в соответствующем конспекте.
\end{reminder}

\begin{proposition}
	Пусть $M$ --- одномерная ориентированная клетка, $V$ --- соответствующая форма ориентированного объёма, $(\vv{i}, \vv{j}, \vv{k})$ --- орты в $\R^3$ и $\vv{\tau}$ --- вектор положительной единичной касательной к $M$. Тогда, имеют место следующие формулы:
	\begin{align*}
		&{dx|_M = \cos(\vv{\tau}, \vv{i})V}
		\\
		&{dy|_M = \cos(\vv{\tau}, \vv{j})V}
		\\
		&{dz|_M = \cos(\vv{\tau}, \vv{k})V}
	\end{align*}
\end{proposition}

\begin{proof}
	Повторим аналогичные действия, как и в предыдущем утверждении, например, для $dx|_M$. Теперь положительная параметризация $\phi$ удобно запишется так:
	\[
		\phi(t) = (x, y, z)
	\]
	Для конкретной точки $\phi(t_0) = (x_0, y_0, z_0)$ получим значение переноса формы:
	\[
		\phi^*(dx|_M)(x_0, y_0, z_0) = dx(t_0) = x'(t_0)dt
	\]
	При положительной параметризации вектор единичной касательной выражается как нормирванный положительный репер. Подойдёт такой:
	\[
		\vv{\tau}(t_0) = \frac{x'(t_0)\vv{i} + y'(t_0)\vv{j} + z'(t_0)\vv{k}}{\sqrt{(x'(t_0))^2 + (y'(t_0))^2 + (z'(t_0))^2}}
	\]
	Далее делаем те же самые действия.
\end{proof}

\begin{definition}
	Пусть $\vv{A}(r) = (P(r), Q(r), R(r))^T$ --- векторное поле, заданное на одномерной клетке $M$. Тогда \textit{криволинейным интегралом 2-го рода} называется следующий интеграл:
	\[
		\int_M \vv{A}^\# = \int_M Pdx + Qdy + Rdz
	\]
	где $dx = dx|_M$, $dy = dy|_M$ и $dz = dz|_M$.
\end{definition}

\begin{corollary} (из последнего утверждения)
	Если нужно посчитать криволинейный интеграл второго рода для векторного поля, то
	\[
		\int_M Pdx + Qdy + Rdz = \int_M (P\cos(\vv{\tau}, \vv{i}) + Q\cos(\vv{\tau}, \vv{j}) + R\cos(\vv{\tau}, \vv{k})) = \int_M \ntbr{\vv{A}, \vv{\tau}}ds
	\]
	Последнее выражение очень важно в физике, ибо является \textit{работой силы $\vv{A}$ по кривой $M$}.
\end{corollary}

\begin{definition}
	Если $\vv{A}(r)$ --- векторное поле, а $M$ --- замкнутая кривая, то криволинейный интеграл второго рода
	\[
		\int_M \ntbr{\vv{A}, \vv{\tau}}ds
	\]
	называется \textit{циркуляцией $A$ вдоль $M$}.
\end{definition}

\begin{note}
	Здесь мы снова возвращаемся к $m = 2$.
\end{note}

\begin{definition}
	Пусть $\vv{A} = (P, Q, R)^T$ --- векторное поле на двумерной клетке $M$. \textit{Поверхностным интегралом 2-го рода} называется следующий интеграл:
	\[
	\iint_M Pdydz + Qdzdx + Rdxdy := \int_M *\vv{A}^\# = \int_M Pdy \wedge dz + Q dz \wedge dx + Rdx \wedge dy
	\]
\end{definition}

\begin{corollary} (из предпоследнего утверждения)
	Если нужно посчитать поверхностный интеграл второго рода, то это можно сделать так:
	\begin{multline*}
	\iint_M Pdydz + Qdzdx + Rdxdy =
	\\
	\iint_M (P\cos(\vv{n}, \vv{i}) + Q\cos(\vv{n}, \vv{j}) + R\cos(\vv{n}, \vv{k}))dS = \iint_M \ntbr{\vv{A}, \vv{n}}dS
	\end{multline*}
\end{corollary}

\begin{definition}
	Если $\vv{A}$ --- векторное поле, а $M$ --- двумерная клетка, то поверхностный интеграл второго рода
	\[
	\iint_M \ntbr{\vv{A}, \vv{n}}dS
	\]
	называется \textit{потоком поля $\vv{A}$ через $M$}
\end{definition}

\begin{anote}
	Когда говорят об интегрировании на замкнутых объектах (замкнутая кривая, замкнутая поверхность), то используют значок $\oint$ и соответствующие с двумя/тремя интегралами.
\end{anote}

\begin{reminder}
	Нормаль, о которой мы говорили в теореме Стокса-Пуанкаре для куба, является \textit{внешней}. В более общем случае, это просто единичная нормаль, выходящая \textit{из} какого-то трехмерного объекта.
\end{reminder}

\begin{note}
	Отсюда $m = n = 3$.
\end{note}

\begin{lemma}
	Если $M$ --- трёхмерная клетка, причём граница $\vdelta M$ ориентирована по правилу выходящего вектора, то положительная нормаль клетки совпадает с внешней нормалью.
\end{lemma}

\textcolor{red}{Тут снова должна быть картинка, в этот раз с кубок и какой-то желешкой.}

\textcolor{red}{Дальнейшие рассуждения некорректны.}
 %
    \begin{corollary}
    В условиях ЦПТ верна ещё такая сходимость:
    \[
        \forall x \in \R \ \ F_n(x) := P \ps{\frac{S_n - \E S_n}{\sqrt{DS_n}} \le x} \xrightarrow[n \to \infty]{} \Phi(x) := \int_{-\infty}^{x} \frac{1}{\sqrt{2\pi}} e^{-y^2/2} dy
    \]
    Иными словами, можем записать сходимость в виде сходимости функций распределения.
\end{corollary}

\begin{proof}
    По Центральной Предельной Теореме имеем сходимость:
    \[
        \frac{S_n - \E S_n}{\sqrt{DS_n}} \xrightarrow{d} N(0, 1)
    \]
    $\Phi$ --- это функция распределения $N(0, 1)$. Из определения, у $\Phi$ есть плотность, то есть $\Phi$ абсолютно непрерывна (и просто непрерывна, в частности). Тогда, по теореме об эквивалентности сходимостей:
    \[
    	\frac{S_n - \E S_n}{\sqrt{DS_n}} \xrightarrow{d} N(0, 1) \Lra \Big(\forall x \in C(\Phi)\ \lim_{n \to \infty} F_n(x) = \Phi(x)\Big)
    \]
    В силу вышесказанного, $C(\Phi) = \R$, то есть мы показали требуемое.
\end{proof}

\begin{corollary}
    В условиях ЦПТ
    \[
        \sqrt{n} \ps{\frac{S_n}{n} - a} \xrightarrow{d} N(0, \sigma^2)
    \]
\end{corollary}

\begin{proof}
	Будем использовать то же обозначение $\eta_n = (\xi_n - a) / \sigma$. Тогда мы знаем, что сходимость по распределнию устроена так:
	\[
		\eta_n \xrightarrow{d} \eta \Lra \forall f \colon \R \to \R \text{ --- непрерывная ограниченная }\E f(\eta_n) \xrightarrow[n \to \infty]{} \E f(\eta)
	\]
	Так как для $\eta_n$ и $c\eta_n, c \in \R$ множества композиций с непрерывными ограниченными функциями одинаковы, то мы можем модифицировать сходимость из ЦПТ на любую наперёд заданную константу:
	\[
		\sqrt{n}\ps{\frac{S_n}{n} - a} = \frac{S_n - na}{\sqrt{n}} = \frac{S_n - an}{\sigma\sqrt{n}}\sigma \xrightarrow{d} \sigma N(0, 1) = N(0, \sigma^2)
	\]
\end{proof}

\begin{note}
    Такая формулировка осмыслена и при $\sigma = 0$: действительно, тогда $\xi_n$ --- константы, и в сходимости 
    \[
        \sqrt{n} \ps{\frac{S_n}{n} - a} \xrightarrow{d} N(0, \sigma^2)
    \]
    слева и справа стоят тождественные нули.
    
    Это будет полезно при обобщении на многомерный случай: в такой ЦПТ в роли $\sigma$ будет выступать матрица, которая может быть вырожденной и при этом не обязательно нулевой, тогда обратить её (<<делить на неё>>) мы не сможем, но сможем использовать в такой форме записи.
\end{note}

\begin{note}
    ЦПТ позволяет в определённом смысле оценить скорость сходимости УЗБЧ. Так, для УЗБЧ в форме Колмогорова нам были нужны независимые одинаково распределённые случайные величины, у которых конечное матожидание, а в ЦПТ к этому добавляется конечность дисперсии. Перепишем ЦПТ в терминах сходимости функции распределения:
    \[
    	\forall x \in \R\ \ P\ps{\sqrt{n}\ps{\frac{S_n}{n} - a} \le x} \xrightarrow[n \to \infty]{} P(\xi \le x),\ \xi \sim N(0, \sigma^2)
    \]
    Так как нормальное распределение абсолютно непрерывно, то можем выбрать такие $u' \le u$, что $P(u' \le |\xi| \le u) = 0.99$. Если применить эти числа в ЦПТ, то получится такой факт:
    \[
    	P\ps{\frac{u'}{\sqrt{n}} \le \md{\frac{S_n}{n} - a} \le \frac{u}{\sqrt{n}}} \xrightarrow[n \to \infty]{} 0.99
    \]
    Иными словами, есть такой номер $N_0 \in \N$, что, начиная с него, с огромной вероятностью будет выполнена оценка:
    \[
    	\md{\frac{S_n}{n} - a} = O\ps{\frac{1}{\sqrt{n}}}
    \]
    При этом данная оценка неулучшаема.
\end{note}

\begin{note}
    Теперь хотим оценить скорость сходимости в самой ЦПТ.
\end{note}

\begin{theorem} (Берри-Эссеена, без доказательства)
    Пусть $\{\xi_n\}_{n = 1}^\infty$ --- независимые одинаково распределённые случайные величины, причём $\E|\xi_1-\E\xi_1|^3 < +\infty$ и пусть $D\xi_1 \neq 0$. Обозначим $S_n = \xi_1 \plusdots \xi_n$ и $T_n = \frac{S_n - \E S_n}{\sqrt{DS_n}}$, $F_n$ --- функция распределения $T_n$, $\Phi$ --- функция распределения $N(0, 1)$. Тогда, имеет место следующее неравенство:
    \[
        \sup_{x \in \R} |F_n(x)-\Phi(x)| \le \frac{c \ \E|\xi_1-\E\xi_1|^3}{\sigma^3 \sqrt{n}}
    \]
    где $c > 0$ --- абсолютная константа
\end{theorem}

\begin{note}
    Здесь уже абсолютно явная оценка сходимости $O(1 / \sqrt{n})$. Про константу $c$ можно сказать следующее:
    \begin{align*}
    & c \le 0,478\ldots < \frac{1}{2}
    \\
    & c \ge \frac{1}{\sqrt{2\pi}} = 0,309\ldots
    \end{align*}
\end{note}

\begin{example}
    Складываются $10^4$ чисел, каждое из которых было вычислено с точностью $10^{-6}$. В каких пределах с вероятностью $0,98$ лежит суммарная ошибка, если считать ошибки независимыми?
\end{example}

\begin{anote}
	Хоть это нигде и не оговаривалось, но техника, которую мы развивали все эти страницы, теперь позволяет получить ответ на эту задачу для абсолютно любого вероятностного пространства $(\Omega, \F, P)$, которое способно реализовать всё условие задачи.
\end{anote}

\begin{solution}
    Пусть $\xi_1, \ldots, \xi_n \sim U[-10^{-6}, 10^{-6}]$, $n = 10^4$ --- независимые случайные величины. Посчитаем моменты различных порядков:
    \begin{align*}
        &{\E\xi_1 = \ldots = \E\xi_n = 0}
        \\
        &{D\xi_1 = \ldots = D\xi_n = \E|\xi_1 - \E\xi_1|^2 = \E\xi_1^2 = \int_{\R} x^2 \chi[-10^{-6}, 10^{-6}] \frac{1}{2 \cdot 10^{-6}}dx = \frac{10^{-12}}{3}}
        \\
        &{\E|\xi_1 - \E\xi_1|^3=\E|\xi_1|^3 =  \int_{\R} |x|^3 \chi[-10^{-6}, 10^{-6}] \frac{1}{2 \cdot 10^{-6}} dx = \frac{10^{-18}}{4}}
    \end{align*}

    Обозначим $S_n = \xi_1 \plusdots \xi_n$, Ф --- функция распределения $\xi \sim N(0, 1)$. Выполнены условия теоремы Берри-Эссеена, тогда:
    \[
        \forall x \in \R \ \ \md{P \ps{\frac{S_n-\E S_n}{\sqrt{DS_n}} \le x} - \Phi(x)} \le \frac{c \ \E|\xi_1-\E\xi_1|^3}{(D\xi_1)^{3/2} \sqrt{n}}
    \]

    Обозначим $\sigma = D\xi_1$ и оценим модуль отклонения вероятности из ЦПТ с вероятностью по нормальному распределению:
    \[
        \forall x \in \R\ \ \md{P\ps{\frac{S_n}{\sigma \sqrt{n}} \le x} - \Phi(x)} \le \frac{\frac{1}{2} \cdot \frac{10^{-18}}{4}}{(\frac{10^{-12}}{3})^{3/2}\sqrt{10^4}} = \frac{\frac{1}{2} \cdot \frac{10^{-18}}{4}}{\frac{10^{-18}}{3 \sqrt{3}}\sqrt{10^4}} = \frac{3\sqrt{3}}{8 \cdot 100}
    \]

    Теперь мы можем дать оценку на просто вероятность среднего оказаться в отрезке $[-x; x]$:
    \begin{multline*}
        P\ps{\md{\frac{S_n}{\sigma \sqrt{n}}} \le x} = \md{P\ps{\md{\frac{S_n}{\sigma \sqrt{n}}} \le x}} = \md{P\ps{\frac{S_n}{\sigma \sqrt{n}} \le x} - P\ps{\frac{S_n}{\sigma \sqrt{n}} < -x}} =
        \\
        = \md{P \ps{\frac{S_n}{\sigma \sqrt{n}} \le x} - P \ps{\frac{S_n}{\sigma \sqrt{n}} < -x} + \Phi(x) - \Phi(x) + P(\xi < -x) - \Phi(-x)} =
        \\
        = \md{\Phi(x) - \Phi(-x) + P\ps{\frac{S_n}{\sigma \sqrt{n}} \le x} - \Phi(x) - P\ps{\frac{S_n}{\sigma \sqrt{n}} < -x} + P(\xi < -x)} \ge
        \\
        |\Phi(x) - \Phi(-x)| - \md{P\ps{\frac{S_n}{\sigma \sqrt{n}} \le x} - \Phi(x)} - \md{P\ps{\frac{S_n}{\sigma \sqrt{n}} < -x} - P(\xi < -x)} \ge
        \\
        |\Phi(x) - \Phi(-x)| - \frac{3\sqrt{3}}{8 \cdot 100} - \frac{3\sqrt{3}}{8 \cdot 100} = \Phi(x) - \Phi(-x) - \frac{3\sqrt{3}}{4 \cdot 100}
    \end{multline*}

    Для функции распределения стандартного нормального распределения можем подобрать параметры:
    \[
        x = 2,807,\ \ \Phi(x) - \Phi(-x) \ge 0,995
    \]

    С учётом этого получим:
    \begin{align*}
        & P \ps{\md{\frac{S_n}{\sigma \sqrt{n}}} \le 2,807} \ge 0,995 - \frac{3\sqrt{3}}{4 \cdot 100}
        \\
        & P \ps{|S_n| \le \sqrt{\frac{10^{-12}}{3}} \sqrt{10^4} \cdot 2,807} \ge 0,98
        \\
        & P(|S_n| \le 1,7 \cdot 10^{-4}) \ge 0,98
    \end{align*}
\end{solution}

\section{Сходимости случайных векторов}

\begin{note}
	Далее мы резервируем обозначение $(\Omega, \F, P)$ под вероятностное пространство.
\end{note}

\begin{definition}
	Пусть $\{\xi_n\}_{n = 1}^\infty$, $\xi$ --- случайные векторы из $\R^m$. Тогда \textit{$\xi_n$ сходится к $\xi$ с вероятностью 1 ($P$-почти наверное)}, если выполнено равенство:
	\[
		P\ps{\lim_{n \to \infty} \xi_n = \xi} = 1
	\]
	Обозначается как $\xi_n \xrightarrow{P \text{ п.н.}} \xi$
\end{definition}

\begin{definition}
	Пусть $\{\xi_n\}_{n = 1}^\infty$, $\xi$ --- случайные векторы из $\R^m$. Тогда \textit{$\xi_n$ сходится к $\xi$ по вероятности}, если выполнено утверждение:
	\[
		\forall \eps > 0\ \ P(\|\xi_n - \xi\|_2 \ge \eps) \xrightarrow[n \to \infty]{} 0, \text{ где } \|x\|_2 = \sqrt{x_1^2 + \ldots + x_m^2}
	\]
	Обозначается как $\xi_n \xrightarrow{P \text{ п.н.}} \xi$
\end{definition}

\begin{definition}
	Пусть $\{\xi_n\}_{n = 1}^\infty$, $\xi$ --- случайные векторы из $\R^m$. Тогда \textit{$\xi_n$ сходится к $\xi$ по распределению}, если выполнено утверждение:
	\[
		\forall f \colon \R^m \to \R \text{ --- непрерывная ограниченная }\ \E f(\xi_n) \xrightarrow[n \to \infty]{} \E f(\xi)
	\]
	Обозначается как $\xi_n \xrightarrow{d} \xi$
\end{definition}

\begin{exercise}
    Пусть $\xi_n = (\xi_{1, n}, \ldots, \xi_{m, n}),\ n \in \N,\ \ \xi = (\xi_1, \ldots, \xi_m)$ --- случайные векторы. Тогда
    \begin{enumerate}
        \item $\xi_n \xrightarrow{P\text{ п.н.}} \xi \Lolra \Big(\forall i \in \range{1}{m}\ \xi_{i, n} \xrightarrow{P\text{ п.н.}} \xi_i\Big)$

        \item $\xi_n \xrightarrow{P} \xi \Lolra \Big(\forall i \in \range{1}{m}\ \xi_{i, n} \xrightarrow{P} \xi_i\Big)$

        \item $\xi_n \xrightarrow{d} \xi \Lolra \Big(\forall i \in \range{1}{m}\ \xi_{i, n} \xrightarrow{d} \xi_i\Big)$
    \end{enumerate}
\end{exercise}

\begin{theorem} (без доказательства)
    Пусть ${\xi_n}_{n = 1}^\infty$, $\xi$ --- случайные векторы в $\R^m$. Тогда эквивалентны следующие утверждения:
    \begin{enumerate}
        \item $\xi_n \xrightarrow{d} \xi$
        
        \item $\forall x \in C(F_\xi)\ F_{\xi_n}(x) \xrightarrow[n \to \infty]{} F_{\xi}(x)$
    \end{enumerate}
\end{theorem}

\begin{note}
	Просто по определению и по теореме о замене переменной в интеграле Лебега
	\[
		\xi_n \xrightarrow{d} \xi \Leftrightarrow P_{\xi_n} \xrightarrow{w} P_{\xi},
	\]
	где справа стоят соответствующие распределения случайных векторов. По теореме Александрова
	\[
		P_{\xi_n} \xrightarrow{w} P_{\xi} \Leftrightarrow P_{\xi_n} \Ra P_{\xi}.
	\]
	Это начало доказательства теоремы выше.
\end{note}

\begin{lemma} (О взаимоотношении видов сходимостей)
    Пусть $\{\xi_n\}_{n = 1}^\infty$, $\xi$ --- случайные векторы в $\R^m$. Тогда:
    \begin{enumerate}
        \item $\xi_n \xrightarrow{P\text{ п.н.}} \xi \Ra \xi_n \xrightarrow{P} \xi$
        
        \item $\xi_n \xrightarrow{P} \xi \Ra \xi_n \xrightarrow{d} \xi$
    \end{enumerate}
\end{lemma}

\begin{proof}~
    \begin{enumerate}
        \item С учётом упражнения достаточно доказать в одномерном случае, а это уже делали.
        \item Доказательство в точности повторяет доказательство одномерного случая, в виду того, что оно достаточно громоздкое, не будем проделывать ещё раз.
    \end{enumerate}
\end{proof}

\begin{theorem} (О наследовании сходимости)
    Пусть $\{\xi_n\}_{n = 1}^\infty$, $\xi$ --- случайные векторы в $\R^m$. Пусть $h \colon \R^m \to \R^k$ --- непрерывна почти всюду относительно распределения случайного вектора $\xi$ (то есть $\exists B \in \B(\R^m) \colon P_\xi(B) = 1 \wedge h$ непрерывна на $B$). Тогда:
    \begin{enumerate}
        \item $\xi_n \xrightarrow{P\text{ п.н.}} \xi \Ra h(\xi_n) \xrightarrow{P\text{ п.н.}} h(\xi)$
        \item $\xi_n \xrightarrow{P} \xi \Ra h(\xi_n) \xrightarrow{P} h(\xi)$
        \item $\xi_n \xrightarrow{d} \xi \Ra h(\xi_n) \xrightarrow{d} h(\xi)$
    \end{enumerate}
\end{theorem}

\begin{proof}~
    \begin{enumerate}
        \item Заметим, что так как $h$ непрерывна на $B$, то:
        \[
            \xi_n(\omega) \to \xi(\omega) \wedge \xi(\omega) \in B \Ra h(\xi_n(\omega)) \to h(\xi(\omega))
        \]
        Отсюда получаем:
        \[
            P(h(\xi_n) \to h(\xi)) \ge P(\xi_n \to \xi \wedge \xi \in B) = 1
        \]
        Последнее верно, ибо $P(\xi_n \to \xi) = P(\xi \in B) = 1$, а мы знаем формулу $P(A \cup B) + P(A \cap B) = P(A) + P(B)$, для любых событий $A, B$.

        \item Хотим доказать, что $h(\xi_n) \xrightarrow{P} h(\xi)$, то есть:
        \[
            \forall \eps > 0 \ \ P(||h(\xi_n) - h(\xi)||_2 \ge \eps) \xrightarrow[n \to \infty]{} 0
        \]
        Предположим противное. Тогда
        \[
            \exists \eps > 0 \ \ \exists \delta > 0 \ \ \exists \{n_k\}_{k = 1}^\infty \subseteq \N \such \forall k \in \N \ \ P(||h(\xi_{n_k}) - h(\xi)||_2 \ge \eps) \ge \delta
        \]
        Но по одному из результатов главы про сходимость случайных величин:
        \[
            \xi_{n_k} \xrightarrow{P} \xi \Ra \exists \{n_{k_l}\}_{l = 1}^\infty \subseteq \{n_k\}_{k = 1}^\infty \ \ \xi_{n_{k_l}} \xrightarrow{P\text{ п.н.}} \xi
        \]
        Отметим, что там результат был доказан для одномерного случая. Но так как сходимости почти наверное и по вероятности эквивалентны таким же покоординатным сходимостям, мы можем постепенно выбирать подпоследовательность: сначала сходящуюся почти наверное по первой координате, затем из неё сходящуюся почти наверное по первой и второй координате, и так далее. Теперь, согласно первому пункту и лемме о взаимоотношении видов сходимостей:
        \[
            \xi_{n_{k_l}} \xrightarrow{P\text{ п.н.}} \xi \Ra h(\xi_{n_{k_l}}) \xrightarrow{P\text{ п.н.}} h(\xi) \Ra h(\xi_{n_{k_l}}) \xrightarrow{P} h(\xi)
        \]
        Тогда получаем, что:
        \[
            0 < \delta \le P(||h(\xi_{n_{k_l}}) - h(\xi)||_2 \ge \eps) \xrightarrow[l \to \infty]{} 0 \text{ --- противоречие}
        \]

        \item Обозначим $Q_n$ --- распределение случайного вектора $h(\xi_n)$, $Q$ --- распределение случайного вектора $h(\xi)$. Хотим доказать, что $h(\xi_n) \xrightarrow{d} h(\xi)$. Как поняли в одном из замечаний выше, для этого нам достаточно доказать, что $Q_n \xrightarrow{w} Q$. По теореме Александрова достаточно проверить, что:
        $\varlimsup_{n \to \infty} Q_n(F) \le Q(F)$ для любого замкнутого $F \subseteq \R^m$. Также обозначим $P_n$ --- распределение случайного вектора $\xi_n$, $P_{\xi}$ --- распределение случайного вектора $\xi$. Знаем, что $\xi_n \xrightarrow{d} \xi$, по тому же замечанию и по теореме Александрова из этого следует, что:
        \[
            \forall F \subseteq \R^m \text{ --- замкнутое }\ \varlimsup_{n \to \infty} P_n(F) \le P_{\xi}(F)
        \]
        Далее будем обозначать $[A]$ --- замыкание множества $A$. Тогда получим:
        \begin{multline*}
            \varlimsup_{n \to \infty} Q_n(F) = \varlimsup_{n \to \infty} P(h(\xi_n) \in F) = \varlimsup_{n \to \infty} P(\xi_n \in h^{-1}(F)) =
            \\
            \varlimsup_{n \to \infty} P_n(h^{-1}(F)) \le \varlimsup_{n \to \infty} P_n([h^{-1}(F)]) \le P_{\xi}([h^{-1}(F)]) = P(\xi \in [h^{-1}(F)])
        \end{multline*}
        Теперь утверждается, что $[h^{-1}(F)] \subseteq h^{-1}(F) \cup (\R^m \setminus B)$ для того самого множества $B$ из условия теоремы. Действительно:
        \[
        	x \in [h^{-1}(F)] \Lra \exists \{x_n\}_{n = 1}^\infty \subseteq h^{-1}(F) \colon x = \lim_{n \to \infty} x_n
        \]
        Остаётся 2 варианта: либо $x \in \R^m \bs B$, либо $x \in B$. Во втором случае следующие 3 факта полностью обосновывают принадлежность к $h^{-1}(F)$:
        \begin{enumerate}
        	\item $h$ непрерывна на $B$, то есть $h(x) = \lim_{n \to \infty} h(x_n)$
        	
        	\item $\forall n \in \N\ x_n \in h^{-1}(F) \Lora \forall n \in \N\ h(x_n) \in F$
        	
        	\item $F$ --- замкнутое множество. Соединяя вышенаписанные факты, получаем требуемое: $h(x) = \lim_{n \to \infty} h(x_n) \in F$
        \end{enumerate}
		Значит, мы можем продолжить цепочку неравенств с верхним пределом так:
		\begin{multline*}
			\varlimsup_{n \to \infty} Q_n(F) \le P(\xi \in [h^{-1}(F)]) \le
			\\
			P(\xi \in h^{-1}(F)) + P(\xi \in \R^m \bs B) = P(h(\xi) \in F) + 0 = Q(F)
		\end{multline*}
        Это ровно то, что нам нужно было проверить.
    \end{enumerate}
\end{proof}

\begin{lemma}
    Пусть $\xi_n$ --- случайные величины, $c = const$, $c \in \R$. Тогда следующие утверждения эквивалентны:
    \begin{enumerate}
        \item $\xi_n \xrightarrow{P} c$
        \item $\xi_n \xrightarrow{d} c$
    \end{enumerate}
\end{lemma}

\begin{proof}~
    \begin{itemize}
        \item[$1 \Ra 2$] Уже доказано в более общем случае.

        \item[$2 \Ra 1$] Запишем сходимость по распределению в терминах функций распределения:
        \[
            \xi_n \xrightarrow{d} c \Lra \Big(\forall x \in C(F_c)\ \lim_{n \to \infty} F_{\xi_n}(x) = F_c(x)\Big)
        \]
		С учётом определения $F_c$ для константы
        \[
            F_c(x) = \System{
                        & 1,\ x \ge c,
                        \\
                        & 0,\ x < c
                    }
        \]
        Заключаем, что $C(F_c) = \R \bs \{c\}$. А показать нам надо следующее:
        \[
            \xi_n \xrightarrow{P} c \Lra \Big(\forall \eps > 0\ \ P(|\xi_n - c| \ge \eps) \xrightarrow[n \to \infty]{} 0\Big)
        \]
		Зафикируем $\eps > 0$ и выразим вероятность выше через функции распределения:
        \begin{multline*}
            P(|\xi_n - c| \ge \eps) = P(\xi_n - c \ge \eps) + P(\xi_n - c \le -\eps) =
            \\
            1 - P(\xi_n < c + \eps) + P(\xi_n \le c - \eps) \le 1 - P \ps{\xi_n \le c + \frac{\eps}{2}} + P(\xi_n \le c - \eps) =
            \\
            1 - F_{\xi_n} \ps{c + \frac{\eps}{2}} + F_{\xi_n}(c - \eps) \xrightarrow[n \to \infty]{} 1 - F_c \ps{c + \frac{\eps}{2}} + F_c(c - \eps) = 1 - 1 + 0 = 0
        \end{multline*}
    \end{itemize}
\end{proof}

\begin{theorem} (Лемма Слуцкого)
    Пусть $\xi_n \xrightarrow{d} \xi$, $\eta_n \xrightarrow{d} c = const$ --- случайные величины. Тогда
    \begin{enumerate}
        \item $\xi_n + \eta_n \xrightarrow{d} \xi + c$
        
        \item $\xi_n \cdot \eta_n \xrightarrow{d} \xi \cdot c$
    \end{enumerate}
\end{theorem}

\begin{note}
    Если бы была дана сходимость случайных векторов $(\xi_n, \eta_n) \xrightarrow{d} (\xi, c)$, то утверждение леммы Слуцкого мгновенно бы следовало из теоремы о наследовании сходимости.
\end{note}

\begin{proof}~
    \begin{enumerate}
        \item Необходимо и достаточно доказать сходимость функций распределения:
        \[
        	\forall x \in C(F_{\xi + c})\ \ \lim_{n \to \infty} F_{\xi_n + \eta_n}(x) = F_{\xi + c}(x)
        \]

		Основная идея доказательства состоит в том, чтобы хорошо оценить верхние и нижние пределы для $F_{\xi_n + \eta_n}(x)$ (не без помощи леммы выше). Пусть $x \in C(F_{\xi + c})$. Тогда просто по определению $x - c$ --- точка непрерывности $F_{\xi}$:
		\[
			F_{\xi + c}(x) = P(\xi + c \le x) = P(\xi \le x - c) = F_\xi(x - c)
		\]
		В силу той же непрерывности, мы можем выбрать сколь угодно малое $\eps > 0$, так, что $x - c \pm \eps$ --- точки непрерывности функции $F_\xi$. Действительно, можем так сделать, так как $F_\xi$ монотонна, имеет не более, чем счётное множество точек разрыва. А если бы не могли так выбрать, получили бы, что в какой-то окрестности точки $x - c$ из любой пары точек $x - c \pm \eps$ хотя бы одна является точкой разрыва, то есть в этой окрестности мощность множества точек разрыва не меньше мощности множества точек непрерывности, то есть множество точек разрыва континуально. Итак, зафиксируем нужное $\eps > 0$.
		\begin{itemize}
			\item[$\varlimsup$] Сделаем оценку сверху на $F_{\xi_n + \eta_n}(x)$ следующим образом:
			\begin{multline*}
				F_{\xi_n + \eta_n}(x) = P(\xi_n + \eta_n \le x) =
				\\
				P(\xi_n + \eta_n \le x \wedge c - \eta_n \ge \eps) + P(\xi_n + \eta_n \le x \wedge c - \eta_n < \eps) \le
				\\
				P(c - \eta_n \ge \eps) + P(\xi_n + c < x + \eps) \le
				\\
				P(|c - \eta_n| \ge \eps) + P(\xi_n \le x - c + \eps)
			\end{multline*}
			По доказанной лемме, $\eta_n \xrightarrow{d} c \Ra \eta_n \xrightarrow{P} c$. Это даёт следующий предел:
			\[
				\forall \eps > 0\ \ P(|c - \eta_n| \ge \eps) = P(|\eta_n - c| \ge \eps) \xrightarrow[n \to \infty]{} 0
			\]
			Также вспомним, что $x - c + \eps \in C(F_\xi)$ и $\xi_n \xrightarrow{d} \xi$ по условию:
			\[
				\lim_{n \to \infty} P(\xi_n \le x - c + \eps) = \lim_{n \to \infty} F_{\xi_n}(x - c + \eps) = F_\xi(x - c + \eps)
			\]
			Теперь мы можем навесить верхний предел на исходное неравенство и получить желаемую оценку:
			\begin{multline*}
				\varlimsup_{n \to \infty} F_{\xi_n + \eta_n}(x) \le \varlimsup_{n \to \infty} P(|c - \eta_n| \ge \eps) + \varlimsup_{n \to \infty} P(\xi_n \le x - c + \eps) =
				\\
				\lim_{n \to \infty} P(|c - \eta_n| \ge \eps) + \lim_{n \to \infty} P(\xi_n \le x - c + \eps) = F_\xi(x - c + \eps)
			\end{multline*}
			
			\item[$\varliminf$] Чтобы получить оценку снизу на $F_{\xi_n + \eta_n}(x)$, получим оценку сверху на $1 - F_{\xi_n + \eta_n}(x)$:
			\begin{multline*}
				1 - F_{\xi_n + \eta_n}(x) = P(\xi_n + \eta_n > x) =
				\\
				P(\xi_n + \eta_n > x \wedge c - \eta_n \le -\eps) + P(\xi_n + \eta_n > x \wedge c - \eta_n > -\eps) \le
				\\
				P(c - \eta_n \le -\eps) + P(\xi_n + c > x - \eps) \le P(|c - \eta_n| \ge \eps) + P(\xi_n > x - c - \eps)
			\end{multline*}
			Из этого получаем неравенство для только $F_{\xi_n + \eta_n}(x)$:
			\[
				F_{\xi_n + \eta_n}(x) \ge 1 - P(|c - \eta_n| \ge \eps) - P(\xi_n > x - c - \eps) = P(\xi_n \le x - c - \eps) - P(|c - \eta_n| \ge \eps)
			\]
			Теперь нужно воспользоваться всё тем же фактом $\eta_n \xrightarrow{d} c \Ra \eta_n \xrightarrow{P} c$ и непрерывностью $F_\xi$ в точке $x - c - \eps$:
			\[
				\lim_{n \to \infty} P(\xi_n \le x - c - \eps) = \lim_{n \to \infty} F_\xi(x - c - \eps) = F_\xi(x - c - \eps)
			\]
			Навешиваем нижние пределы и получаем последнюю оценку:
			\begin{multline*}
				\varliminf_{n \to \infty} F_{\xi_n + \eta_n}(x) \le \varliminf_{n \to \infty} P(\xi_n \le x - c - \eps) - \varliminf_{n \to \infty} P(|c - \eta_n| \ge \eps) =
				\\
				\lim_{n \to \infty} P(\xi_n \le x - c - \eps) - \lim_{n \to \infty} P(|c - \eta_n| \ge \eps) = F_\xi(x - c - \eps)
			\end{multline*}
		\end{itemize}
		В итоге мы получили такое утверждение:
		\[
			\forall \eps > 0\ \ F_\xi(x - c - \eps) \le \varliminf_{n \to \infty} F_{\xi_n + \eta_n}(x) \le \varlimsup_{n \to \infty} F_{\xi_n + \eta_n}(x) \le F_\xi(x - c + \eps)
		\]
		Так как $x - c \in C(F_\xi)$, то можем устремить $\eps$ к нулю и получить равенство. Стало быть:
		\[
			\lim_{n \to \infty} F_{\xi_n + \eta_n}(x) = F_\xi(x - c) = F_{\xi + c}(x)
		\]

        \item Доказательство второй части теоремы очень похоже на доказательство первой части. Вместо $x - c$ возникнет $x/c$, сложение заменится на умножение. Нужно только аккуратно разобрать случаи, где возникает деление на ноль. Тем не менее, эту часть оставляем без доказательства.
    \end{enumerate}
\end{proof}

\begin{theorem} (Обобщение леммы Слуцкого, без доказательства)
    Пусть $\xi_n \xrightarrow{d} \xi,\ \eta_n \xrightarrow{d} c = const$ --- случайные величины. Тогда $(\xi_n, \eta_n) \xrightarrow{d} (\xi, c)$.
\end{theorem}

\begin{example}
    Пусть известно, что $X_1, \ldots, X_n$ --- независимые одинаково распределённые случайные величины, причём $X_1 \sim Bin(1, p) = Bern(p)$, но мы не знаем $p$. Хотим как раз оценить $p$. Покажем, как это можно организовать, если мы знаем просто значения $X_1, \ldots, X_n$ на каких-то хороших исходах. Обозначим среднее арифметическое $n$ величин за $\ol{X}$ 
    \[
        \ol{X} := \frac{X_1 + \ldots + X_n}{n}
    \]
    За счёт известного распределения, мы знаем матожидание и дисперсию: $\E X_1 = p$, $DX_1 = p(1 - p)$. Применим ЦПТ:
    \[
        \frac{n\ol{X}-np}{\sqrt{np(1 - p)}} = \frac{\sqrt{n}(\ol{X} - p)}{\sqrt{p(1 - p)}} \xrightarrow[n \to \infty]{d} N(0, 1)
    \]
    Хочется заменить знаменатель последней дроби на что-то, зависящее от $X_1, \ldots, X_n$, чтобы сходимость при этом сохранилась (тогда мы можем перейти к вероятностям и выразить границы оценки $p$). Итак, по УЗБЧ в форме Колмогорова:
    \[
        \ol{X} \xrightarrow{P\text{ п.н.}} p
    \]
    По теореме о наследовании сходимости:
    \[
        \sqrt{\ol{X}(1 - \ol{X})} \xrightarrow{P\text{ п.н.}} \sqrt{p(1 - p)}
    \]
    По этой же теореме, так как $p=const$:
    \[
        \frac{\sqrt{p(1 - p)}}{\sqrt{\ol{X}(1 - \ol{X})}} \xrightarrow{P\text{ п.н.}} 1
    \]

    Сходимость почти наверное влечёт сходимость по распределению, поэтому, применяя лемму Слуцкого для произведения, имеем уже такую сходимость:
    \[
        \frac{\sqrt{n}(\ol{X} - p)}{\sqrt{\ol{X}(1 - \ol{X})}} = \frac{\sqrt{n}(\ol{X} - p)}{\sqrt{p(1 - p)}} \frac{\sqrt{p(1 - p)}}{\sqrt{\ol{X}(1 - \ol{X})}} \xrightarrow{d} N(0, 1) \cdot 1 = N(0, 1)
    \]

    Нормальное распределение абсолютно непрерывно, поэтому, в терминах функций распределения:
    \[
        \forall u \in \R\ \ P \ps{\frac{\sqrt{n}(\ol{X} - p)}{\sqrt{\ol{X}(1 - \ol{X})}} \le u} \xrightarrow[n \to \infty]{} \Phi(u)
    \]
    Здесь $\Phi$ --- функция распределения стандартного нормального распределения.

    Далее, применяя оценку из примера к теореме Берри-Эссеена, получим:
    \[
        P \ps{\md{\frac{\sqrt{n}(\ol{X} - p)}{\sqrt{\ol{X}(1 - \ol{X})}}} \le 2,807} \xrightarrow[n \to \infty]{} \Phi(2,807) -  \Phi(-2,807) \ge 0,995
    \]

    Таким образом, с вероятностью, стремящейся к чему-то большему, чем $0,995$:
    \[
        \ol{X} - \frac{2.807}{\sqrt{n}} \sqrt{\ol{X}(1 - \ol{X})} \le p \le \ol{X} + \frac{2.807}{\sqrt{n}} \sqrt{\ol{X}(1 - \ol{X})}
    \]
\end{example} %Производная, арифметика, производная обратной.
    \textcolor{red}{Тут начало 14й лекции. Покуда этот текст тут, далее я не гарантирую качество материала.}

\textcolor{red}{Есть понятие $\sigma$-конечного простанства. Надо дописать}

\begin{theorem} (<<Теорема Эрлиха>>, критерий сходимости по мере)
	Пусть задано измеримое пространство $(X, M, \mu)$, причём $\mu(X) < \infty$. Если $f_n, f$ --- это измеримые функции, то имеет место эквивалентность:
	\[
		f_n \Ra^\mu f \Longleftrightarrow \forall \{n_k\}_{k = 1}^\infty\ \exists \{n_{k_m}\}_{m = 1}^\infty \such f_{n_{k_m}} \xrightarrow[m \to \infty]{} f \text{ почти всюду}
	\]
\end{theorem}

\textcolor{red}{Дописать, откуда куда действуют функции}

\begin{proof}
	Проведём доказательство в каждую из сторон по отдельности:
	\begin{itemize}
		\item $\Ra$ Так как $f_n \Ra f$, то по теореме Рисса уже следует требуемое.
		
		\item $\La$ Предположим противное, то есть $f_n \centernot\Ra f$. По определению сходимости по мере это означает
		\[
			\exists \eps_0 > 0\ \such \delta_0 > 0\ \such \exists \{n_k\}_{k = 1}^\infty
		\]
		\textcolor{red}{Ничего не понял по кванторам}
	\end{itemize}
\end{proof}

\begin{definition}
	Измеримые функции $f, g$ на измеримом пространстве $(X, M, \mu)$ называются \textit{эквивалентными}, если выполнено равенство:
	\[
		\mu\{x \in X \colon f(x) \neq g(x)\} = 0
	\]
\end{definition}

\begin{note}
	Вполне справедливо ждать от сходимости как \textit{явления} следующих свойств:
	\begin{enumerate}
		\item Единственность предела с точностью до эквивалентности измеримых функций
		
		\item Линейность сходимости
		
		\item Непрерывность композиции
		
		\item Непрерывность произведения и деления (второе с оговоркой про нули)
	\end{enumerate}
\end{note}

\begin{theorem}
	Для поточечной сходимости почти всюду верны все свойства
\end{theorem}

\begin{proof}
	\textcolor{red}{В одну сторону очевидно, в обратную тривиально.}
\end{proof}

\begin{note}
	Верны ли все свойства для сходимости по мере? Оказывается, что не всё так просто, это показывает следующий пример.	
\end{note}

\begin{example}
	Рассмотрим измеримое пространство $(\R, M, \mu)$ ($M$ --- измеримые множества, $\mu$ --- классическая мера Лебега). Проверим выполнение непрерывности композиции на последовательности $f_n = x + 1 / n$. Она, идейно, должна сходится к $f(x) = x$, что и наблюдается:
	\[
		\forall \eps > 0\ \ \mu\{x \in X \colon |f_n(x) - f(x)| > \eps\} \xrightarrow[n \to \infty]{} 0
	\]
	Посмотрим композицию с непрерывной функцией $h(x) = x^2$. Тогда
	\[
		h(f_n(x)) = x^2 + 2x\frac{1}{n} + \frac{1}{n^2}
	\]
\end{example}

\begin{theorem}
	Если задано измеримое пространство $(X, M, \mu)$ и верно, что $\mu(X) < \infty$, то для сходимости по мере верны все свойства из замечания.
\end{theorem}

\begin{theorem}
	Если задано измеримое пространство $(X, M, \mu)$, то для сходимости по мере верны первые 2 свойства из замечания.
\end{theorem}

\begin{proof}
	\begin{enumerate}
		\item Пусть $f_n \Ra f$ и $f_n \Ra g$. Заметим такое вложение:
		\begin{multline*}
			\forall \eps > 0\ \forall n \in \N\ \ \{x \in X \colon |f(x) - g(x)| > \eps\} \subseteq
			\\
			\set{x \in X \colon |f(x) - f_n(x)| > \frac{\eps}{2}} \cup \set{x \in X \colon |f_n(x) - g(x)| > \frac{\eps}{2}}
		\end{multline*}
	\end{enumerate}
\end{proof}

\subsection{Интеграл Лебега}

\begin{note}
	Интеграл Лебега можно вводить разными методами. Конкретно у нас, зафиксируем $(X, M, \mu)$ --- $\sigma$-конечное пространство. Более того, любая рассматриваемая функция --- измеримая.
\end{note}

\begin{definition}
	\textit{Носителем функции} $f \colon D \to \R$ называется множество точек, где она принимает ненулевые значения.
\end{definition}

\begin{definition}
	Функция $f$ называется \textit{простой}, если она представима в следующем виде:
	\[
		f(x) = \sum_{k = 1}^n c_k \chi_{E_k} (x)
	\]
	при этом $c_k \neq 0$, $\mu(E_k) < \infty$ и $E_i \cap E_j = \emptyset$
\end{definition}

\begin{note}
	Если $f$ проста, то её носитель имеет конечную меру.
\end{note}

\begin{definition}
	\textit{Интегралом Лебега} для простой функции $f$ на множестве $X$ называется такая величина:
	\[
		(L)\int_X fd\mu := \sum_{k = 1}^n c_k \mu(E_k)
	\]
\end{definition}

\begin{note}
	Буковку $(L)$ обычно опускают.
\end{note}

\begin{note}
	Наше определение самую малость безосновательно: почему разные суммы дают одинаковый результат? Это необходимо обосновать.
\end{note}

\begin{lemma}
	Для любой простой функции $f$ существует \textit{каноническое представление}, при котором
	\[
		f = \sum_{k = 1}^n c_k \chi_{E_k}(x)
	\]
	где $c_i \neq c_j$ и $c_1 < \ldots < c_n$.
\end{lemma}

\begin{proof}
	\textcolor{red}{Дописать}
\end{proof}

\begin{proposition}
	Определение интеграла Лебега для простой функции корректно.
\end{proposition}

\begin{proof}
	\textcolor{red}{Дописать}
\end{proof}

\begin{theorem}
	Если $f$ и $g$ --- простые функции, то для интеграла Лебега имеет место \textit{линейность}:
	\[
		\forall \alpha, \beta \in \R\ \ \int_X (af(x) + bg(x))d\mu = a \int_X fd\mu + b \int_X gd\mu
	\]
\end{theorem}

\begin{proof}
	Есть 2 факта, которые нужно объяснить:
	\begin{enumerate}
		\item Простота функции-линейной комбинации. 
		
		\item Равенство интегралов.
	\end{enumerate}
\end{proof}

\begin{theorem} (Свойства интеграла Лебега от простой функции)
	Пусть $f$ и $g$ --- простые функции. Тогда имеют место такие свойства:
	\begin{enumerate}
		\item Если $f \ge 0$, то $\int_X f(x)d\mu \ge 0$
		
		\item Если $f \ge g$, то $\int_X f(x)d\mu \ge \int_X g(x)d\mu$
		
		\item $\left|\int_X f(x)d\mu\right| \le \int_X |f(x)|d\mu$
		
		\item Если $X = A \sqcup B$, где $A, B \in M$, тогда
		\[
			\int_X fd\mu = \int_A fd\mu + \int_B fd\mu
		\]
	\end{enumerate}
\end{theorem}

\begin{theorem}
	Пусть есть такие условия:
	\begin{enumerate}
		\item $E \in M$
		
		\item $g$ --- простая функция
		
		\item $\{g_n\}_{n = 1}^\infty$ --- неубывающая на $E$ последовательность простых функций, то есть
		\[
			\forall n \in \N\ \forall x \in E\ \ g_n(x) \le g_{n + 1}(x)
		\]
		
		\item $\forall x \in E\ \ \lim_{n \to \infty} g_n(x) \ge g(x)$ (допускается бесконечное значение предела)
	\end{enumerate}
	Тогда имеется неравенство:
	\[
		\lim_{n \to \infty} \int_E g_n(x)d\mu \ge \int_E g(x)d\mu
	\]
\end{theorem}

\begin{proof}
	Если предел интегралов бесконечен, то теорема очевидна. Рассмотрим конечный случай. Запишем каноническое разложение $g$:
	\[
		g(x) = \sum_{k = 1}^N a_k \chi_{E_k}
	\]
	где, из определения и условия, $0 < a_1 < \ldots < a_N$. Обозначим $F = \bscup_{k = 1}^N E_k$. Так как мы объединяем конечное число непересекающихся множеств, 
\end{proof} %
    %29.03.23

\subsection{Подпространства метрического пространства}

\begin{definition}
Пусть $(X, \rho)$ --- метрическое пространство, $E \subset X, E \neq \emptyset$. Сужение $\rho\vert_{E \times E}$ является метрикой на $E$. Пара $(E, \rho\vert_{E \times E})$ называется \emph{подпространством} $(X, \rho)$, а функция $\rho\vert_{E \times E}$ --- \emph{индуцированной метрикой}.
\end{definition}

Рассмотрим $B_r^E (x) = \{y \in E \ | \ \rho(x, y) < r\} = B^X_r(x) \cap E$.

\begin{lemma}
    Пусть $(X, \rho)$ --- метрическое пространство, $E \subset X$.
    \[
        \underbrace{U}_{\text{откр. в $E$}} \Leftrightarrow \exists \underbrace{V}_{\text{откр. в $X$}} \ (U = V \cap E).
    \]


    \begin{proof}
        \emph{($\Rightarrow$)} Пусть $U$ открыто в $E$. Тогда $\forall x \in U \, \exists B_{\epsilon_x}^E(x) \subset U$ и, значит, $U = \bigcup_{x \in U} B_{\epsilon_x}^E (x)$. Положим $V = \bigcup_{x \in U} B_{\epsilon_x}^X (x)$. Тогда $V$ открыто в $X$ и $V \cap E = \bigcup_{x \in U} (B_{\epsilon_x}^X(x) \cap E) = \bigcup_{x \in U} B_{\epsilon_x}^E(x) = U$.

        \emph{($\Leftarrow$)} Пусть $x \in U$ и $U = \underbrace{V}_{\text{откр. в $X$}} \cap E$, тогда $x \in V \Rightarrow \exists B_\epsilon^X(x) \subset V \Rightarrow B_\epsilon^E(x) = B_\epsilon^X(x) \cap E \subset V \cap E = U$, то есть $U$ открыто в $E$.
    \end{proof}
\end{lemma}

\begin{corollary}
    \[
        \underbrace{Z}_{\text{замк. в $E$}} \Leftrightarrow \exists \underbrace{F}_{\text{замк. в $X$}} \ (Z = F \cap E).
    \]
\end{corollary}

\begin{example}
    $X = \R, E = (0, 10], A = (0, 1], B = (2, 3], C = (9, 10]$.

    \begin{itemize}
        \item $A$ замкнуто в $E$, $A = [-1, 1] \cap E$;
        \item $C$ открыто в $E$, $C = (9, 11) \cap E$;
        \item $B$ не открыто и не замкнуто в $E$.
    \end{itemize}
\end{example}

\subsection{Компакты в метрических пространствах}

\begin{definition}
    Пусть $X$ --- множество, $Y \subset X$. Семейство $\{X_\alpha\}_{\alpha \in A}$ подмножеств $X$ называется \emph{покрытием $Y$}, если $Y \subset \bigcup_{\alpha \in A} X_\alpha$.

    Если $B \subset A$ и $\{X_\alpha\}_{\alpha \in B}$ также является покрытием $Y$, то оно называется \emph{подпокрытием}.
\end{definition}

\begin{definition}
    Пусть $(X, \rho)$ --- метрическое пространство, $K \subset X$. $K$ называется \emph{компактом} (в $X$), если из любого его открытого покрытия $\{G_\lambda\}_{\lambda \in \Lambda}$ можно выделить конечное подпокрытие, то есть $\exists \lambda_1, \ldots \lambda_m \in \Lambda \ (K \subset G_{\lambda_1} \cup \ldots \cup G_{\lambda_m})$.
\end{definition}

\begin{example}
    $X = \R \Rightarrow [a, b]$ --- компакт по теореме Гейне-Бореля.
\end{example}

\begin{note}
    K --- компакт в $X$ тогда и только тогда, когда $K$ --- компакт <<в себе>>, то есть в $(K, \rho)$.
\end{note}

\begin{lemma}
    \label{lem_lim_closed}
    Пусть $(X, \rho)$ --- метрическое пространство, $K \subset X$. Если $K$ --- компакт, то $K$ ограничено и замкнуто в $X$.

    \begin{proof}
        Пусть $a \in K$. Так как $\bigcup_{n = 1}^\infty B_n(a) = X$, то $\{B_n(a)\}_{n \in \N}$ --- открытое покрытие $K$. Следовательно, $K \subset B_{n_1}(a) \cup \ldots \cup B_{n_m}(a) = B_N(a)$, где $N = \max_{1 \le i \le m}\{n_i\}$, и, значит, $K$ ограничено.

        Пусть $a \in X \setminus K$. Так как $\bigcup_{n=1}^{\infty}\left(X \setminus \overline{B}_{\frac{1}{n}}(a)\right) = X \setminus \{a\}$, то $\{X \setminus \overline{B}_{\frac{1}{n}} (a)\}_{n \in \N}$ --- открытое покрытие $K$. Следовательно, $K \subset \left(X \setminus \overline{B}_{\frac{1}{n_1}}(a)\right) \cup \ldots \cup \left(X \setminus \overline{B}_{\frac{1}{n_m}}(a)\right) = X \setminus \overline{B}_{\frac{1}{N}}(a)$, где $N = \max_{1 \le i \le m}\{n_i\}$. Тогда $\overline{B}_{\frac{1}{N}}(a) \subset X \setminus K$ и, значит, $X \setminus K$ открыто, а значит, $K$ -- замкнуто.
    \end{proof}
\end{lemma}

\begin{lemma}
    \label{lem_comp_subset}
    Замкнутое подмножество компакта --- компакт.

    \begin{proof}
        Пусть $K$ --- компакт в $X$, $\underbrace{F}_{\text{замк. в $X$}} \subset K$. Покажем, что $F$ -- компакт. Рассмотрим открытое покрытие $\{G_\lambda\}_{\lambda \in \Lambda}$ множества $F$, тогда $\{G_{\lambda}\}_{\lambda \in \Lambda} \cup \{X \setminus F\}$ --- открытое покрытие $K$, так как $\left(\bigcup_{\lambda \in \Lambda} G_\lambda\right) \cup (X \setminus F) = X$. Поскольку $K$ --- компакт, то $K \subset G_{\lambda_1} \cup \ldots \cup G_{\lambda_m} \cup (X \setminus F) \overset{F \subset K}{\Rightarrow} F \subset G_{\lambda_1} \cup \ldots \cup G_{\lambda_m}$. Значит, $F$ -- компакт.
    \end{proof}
\end{lemma}

\begin{problem}
    Пусть $\{F_n\}$ --- непустые компакты в $X$, $F_1 \supset F_2 \supset \ldots$. Покажите, что $\bigcap_{n = 1}^\infty F_n \neq \emptyset$.
\end{problem}

\begin{theorem}
    \label{compact-criterion}
    Пусть $(X, \rho)$ --- метрическое пространство, $K \subset X$. $K$ --- компакт тогда и только тогда, когда из любой последовательности элементов $K$ можно выделить сходящуюся в $K$ подпоследовательность.

    \begin{proof}
        \emph{($\Rightarrow$)} Пусть $\forall n \in \N \ x_n \in K$. Предположим, что из $\{x_n\}$ нельзя выделить сходяющуюся подпоследовательность в $K$. Тогда $\forall a \in K \ \exists \delta_a > 0 \ \exists N_a \ \forall n \ge N_a \ (x_n \not\in B_{\delta_a}(a))$.

        Рассмотрим $\{B_{\delta_a}(a)\}_{a \in K}$ --- открытое покрытие $K$. Следовательно, $K \subset B_{\delta_{a_1}}(a_1) \cup \ldots \cup B_{\delta_{a_m}}(a_m)$.
    
        Положим $N = \max_{1 \le i \le m} \{N_{a_i}\}$. Так как $N \ge N_{a_i}$, то $x_N \not\in B_{\delta_{a_i}}(a_i)$ $i = 1, \ldots, m \Rightarrow x_N \not\in K$ --- противоречие.
    
        \emph{($\Leftarrow$)} Пусть из любой последовательности элементов $K$ можно выделить сходящуюся в $K$ подпоследовательность \emph{(секвенциальная компактность)}.

        \begin{enumerate}
            \item Покажем, что для любого $\epsilon > 0$ $K$ можно покрыть конечным набором открытых шаров радиуса $\epsilon$.
    
            Докажем от противного -- пусть нельзя покрыть. Индуктивно построим последовательность $\{x_n\}$, $x_1 \in K, x_n \in K \setminus \bigcup_{i = 1}^{n - 1} B_\epsilon(x_i)$.
    
            По построению $\rho(x_i, x_j) \geq \epsilon$, и, значит, из $\{x_n\}$ нельзя выделить сходящуюся подпоследовательность -- противоречие.
    
            \item Пусть $\{G_\lambda\}_{\lambda \in \Lambda}$ --- открытое покрытие $K$, тогда $\exists \epsilon > 0 \, \forall x \in K \, \exists \lambda \in \Lambda \ \left(B_\epsilon(x) \ \subset G_\lambda\right)$. Предположим, что это не выполняется, тогда $\forall n \in \N \, \exists x_n \in K \, \forall \lambda \in \Lambda \left(B_{\frac{1}{n}}(x_n) \not\subset G_\lambda\right)$.
    
            Имеем $\{x_n\} \subset K \Rightarrow \exists x_{n_k} \rightarrow x \in K$, следовательно, $\exists \lambda_0 \in \Lambda (x \in \underbrace{G_{\lambda_0}}_{\text{откр.}}) \Rightarrow \exists B_{\alpha}(x) \subset G_{\lambda_0}$. Выберем $k$ так, чтобы $x_{n_k} \in B_{\frac{\alpha}{2}}(x)$ и $\frac{1}{n_k} < \frac{\alpha}{2}$. Если $z \in B_{\frac{1}{n_k}}(x_{n_k}) \Rightarrow \rho(z, x) \le \rho(z, x_{n_k}) + \rho(x_{n_k}, x) < \frac{\alpha}{2} + \frac{\alpha}{2} = \alpha$.
    
            Следовательно, $z \in B_\alpha(x)$, $B_{\frac{1}{n_k}}(x_{n_k}) \subset B_\alpha(x) \subset G_{\lambda_0}$ --- противоречие.
    
            \item Пусть $\{G_{\lambda}\}_{\lambda \in \Lambda}$ -- открытое покрытие $K$. Тогда по (2):
            \[\exists \epsilon > 0 \ \forall x \in K \ \exists \lambda \in \Lambda \ (B_{\epsilon}(x) \subset G_{\lambda})\]
    
            По (1) $\exists x_{1}, x_{2}, ..., x_{m} \in K$, что $K \subset B_{\epsilon}(x_{i}) \cup ... \cup B_{\epsilon}(x_{m}) \subset G_{\lambda_{1}} \cup ... \cup G_{\lambda_{m}}$, где $\lambda_{i}$ удовлетворяет условию $B_{\epsilon}(x_{i}) \subset G_{\lambda_{i}}$.
    
            Следовательно, $K$ -- компакт.
        \end{enumerate}
    \end{proof}
\end{theorem} %Производные и дифф. высших порядкой. Французские теоремы (свойства производной)
    \begin{theorem} (Фату)
	Пусть $\mu$ --- полная мера, а $\{f_n\}_{n = 1}^\infty$ --- последовательность неотрицательных функций, сходящихся почти всюду на $E$ к $f$. Тогда имеет место неравенство:
	\[
		\int_E f(x)d\mu \le \varliminf_{n \to \infty} \int_E f_n(x)d\mu
	\]
\end{theorem}

\begin{proof}
	Рассмотрим функции инфинумов $\phi_n(x) = \inf_{k \ge n} f_k(x)$ на $E$. Тогда понятно, что эти функции неуывают в каждой точке на $E$ и более того:
	\[
		\lim_{n \to \infty} \phi_n(x) = \varliminf_{n \to \infty} f_n(x) = f(x)
	\]
	Обозначим $E_1$ --- множество точек, где последовательность сходится. Тогда $\mu(E \bs E_1) = 0$ и можно записать следующие равенства (равенство с пределом обусловлено теоремой Леви):
	\[
		\int_E f(x)d\mu = \int_{E_1} f(x)d\mu = \lim_{n \to \infty} \int_{E_1} \phi_n(x)d\mu = \varliminf_{n \to \infty} \int_E \phi_n(x)d\mu
	\]
	Остаётся заметить, что $\forall n \in \N\ \int_E \phi_n(x)d\mu \le \int_E f_n(x)d\mu$
\end{proof}

\begin{theorem} (Лебега)
	Пусть $\mu$ --- полная мера, заданы $\{f_n\}_{n = 1}^\infty$ и $F \colon E \to \R$ на $E \in M$ со следующими свойствами:
	\begin{enumerate}
		\item $F \in L(E)$
		
		\item $F$ --- неотрицательная функция
		
		\item $\forall n \in \N\ \forall x \in E\ \ |f_n(x)| \le F(x)$
		
		\item $f_n \to f$ почти всюду на $E$
	\end{enumerate}
	Тогда имеет место равенство:
	\[
		\int_E f(x)d\mu = \lim_{n \to \infty} \int_E f_n(x)d\mu
	\]
\end{theorem}

\begin{proof}
	Коль скоро $|f_n(x)| \le F(x)$, причём последняя интегрируема по Лебегу, то и $f_n \in L(E)$. Более того, то же самое верно и про $f$ на множестве сходимости $E_1$. Так как мера полна и $\mu(E \bs E_1) = 0$ (то есть $f$, взятая на $E_1$ и дополненная какими-то значениями до $E$, будет эквивалентна $f$ на всей $E$). Пользуясь пунктом из теоремы об абсолютной непрерывности интеграла Лебега, устанавливаем, что $f \in L(E)$.
	
	Осталось установить равенство. Для этого рассмотрим такие последовательности функций:
	\begin{align*}
		&{\phi_n(x) = F(x) + f_n(x)}
		\\
		&{\psi_n(x) = F(x) - f_n(x)}
	\end{align*}
	Про них по условию мы уже можем сказать, что они неотрицательны и почти всюду есть пределы:
	\begin{align*}
		&{\lim_{n \to \infty} \phi_n(x) = F(x) + f(x)}
		\\
		&{\lim_{n \to \infty} \psi_n(x) = F(x) - f(x)}
	\end{align*}
	Применим к ним теорему Фату и получим оценки с двух сторон для равенства:
	\begin{multline*}
		\int_E F(x)d\mu + \int_E f(x)d\mu = \int_E (F(x) + f(x))d\mu \le
		\\
		\varliminf_{n \to \infty} \int_E (F(x) + f_n(x))d\mu = \int_E F(x)d\mu + \varliminf_{n \to \infty} \int_E f_n(x)d\mu
	\end{multline*}
	И второе аналогично:
	\begin{multline*}
		\int_E F(x)d\mu - \int_E f(x)d\mu = \int_E (F(x) - f(x))d\mu \le
		\\
		\varliminf_{n \to \infty} \int_E (F(x) - f_n(x))d\mu = \int_E F(x)d\mu - \varlimsup_{n \to \infty} \int_E f_n(x)d\mu
	\end{multline*}
	Итого:
	\[
		\varlimsup_{n \to \infty} \int_E f_n(x)d\mu \le \int_E f(x)d\mu \le \varliminf_{n \to \infty} \int_E f_n(x)d\mu
	\]
	Стало быть, предел существует и равенство выполнено.
\end{proof}

\begin{reminder}
	Функциональная последовательность $\{f_n\}_{n = 1}^\infty$ \textit{сходится равномерно} к $f \colon E \to \R$ на $E$, если выполнено утверждение:
	\[
		\forall \eps > 0 \exists N \in \N \such \forall n > N, x \in E\ \ |f_n(x) - f(x)| < \eps
	\]
\end{reminder}

\begin{theorem} (Егорова)
	Если $\mu(E) < \infty$ и $f_n \to f$ почти всюду на $E$, то
	\[
		\forall \eps > 0\ \exists E_\eps \subseteq E \such E_\eps \in M \wedge \mu(X \bs E_\eps) < \eps \wedge f_n \text{ сходится равномерно на } E_\eps
	\]
\end{theorem}

\begin{proof}
	По критерию сходимости почти всюду:
	\[
		\forall m \in \N\ \ \lim_{n \to \infty} \ps{\bigcup_{k = n}^\infty \set{x \in X \colon |f_k(x) - f(x)| > \frac{1}{m}}} = 0
	\]
	Отсюда получим такое утверждение:
	\begin{multline*}
		\forall \eps > 0\ \forall m \in \N\ \exists n_m \in \N \such n_m > n_{m - 1} \wedge \mu\ps{\bigcup_{k = n_m}^\infty \set{x \in X \colon |f_k(x) - f(x)| > \frac{1}{m}}} < \frac{\eps}{2^m}
	\end{multline*}
\end{proof} %Дарбу. Лопиталь. Равномерная непрерывность
    %01.03.23

\section{Непрерывные функции}

\subsection{Предел функции в точке}

Пусть $(X, \rho_{x}), (Y, \rho_{y})$ -- метрические пространства, $a$ -- предельная точка $X$, и задана функция $f: X \setminus \{a\} \to Y$.

\begin{definition}[Коши]
    Точка $b \in Y$ называется \textit{пределом} функции $f$ в точке $a$, если
    
    \[\forall \epsilon > 0 \ \exists \delta > 0 \ \forall x \in X (0 < \rho_{X}(x, a) < \delta \Rightarrow \rho_{Y}(f(x), b) < \epsilon)\]
    
    или, что эквивалентно,
    \[\forall \epsilon > 0 \ \exists \delta > 0 \ \forall x \in X (x \in \mathring{B}_{\delta}(a) \Rightarrow f(x) \in B_{\epsilon}(b).\]
\end{definition}

\begin{definition}[Гейне]
    Точка $b \in Y$ называется \textit{пределом} функции $f$ в точке $a$, если
    
    \[\forall \{x_{n}\}, x_{n} \in X \setminus \{a\} (x_{n} \to a \Rightarrow f(x_{n}) \to b).\]
    
\end{definition}

Как и в случае числовых функций, доказывается равносильность определений по Коши и по Гейне, поэтому в обоих случаях пишут $\lim_{x \to a}f(x) = b$, или $f(x) \to b$ при $x \to a$.

\begin{property}[единственность]
    Если $\lim_{x \to a}f(x) = b$ и $\lim_{x \to a}f(x) = c$, то $b = c$.
\end{property}

\begin{proof}
    Пусть $x_{n} \to a$ и $x_{n} \neq a$. По определению Гейне $f(x_{n}) \to b$ и $f(x_{n}) \to c$. Так как последовательность в метрическом пространстве имеет не более одного предела, то $b = c$.
\end{proof}

\begin{note}
    Пусть $E \subset \R^{n}$, $a$ -- предельная точка $E$, функция $f: E \to \R^{m}$. Если $x \in E$, то $f(x) = (y_{1}, \ldots, y_{m})$, и значит, для каждого $i = 1, \ldots, m$ определена \textit{$i$-я координатная функция} $f_{i}: E \to \R$, $f_{i}(x) = y_{i}$. Пишут $f = (f_{1}, \ldots, f_{m})$.
\end{note}

\begin{lemma}[о покоординатной сходимости]
    $\lim_{x \to a}f(x) = b \lra \lim_{x \to a} f_{i}(x) = b_{i}$.
\end{lemma}

\begin{proof}
    Следует из неравенств $|x_{i} - b_{i}| \leq \rho_{2}(x, b) \leq \sqrt{m}\max_{1 \leq i \leq m}|x_{i} - b_{i}|$.
\end{proof}

\begin{example}
    \begin{enumerate}
        \item $f(x, y) = \frac{x^{3} + y^{3}}{x^{2} + y^{2}}$.

        $|f(x, y) - 0| = \frac{|x^{3} + y^{3}|}{x^{2} + y^{2}} \leq \frac{|x^3| + |y^3|}{x^2 + y^2} \leq 2\cdot \frac{\left(\sqrt{x^2 + y^2}\right)^{3}}{x^2 + y^2} = 2\sqrt{x^2 + y^2} < \epsilon \Rightarrow \rho_{2}((x, y), (0, 0)) < \frac{\epsilon}{2}$. $(\delta = \frac{\epsilon}{2})$.

        \item $f(x, y) = \frac{xy + y^2}{x^2 + y^2}$.

        $f(x, 0) = 0$, $f(0, y) = 1 \Rightarrow$ предела в $(0, 0)$ нет.
    \end{enumerate}
\end{example}

\begin{property}
    $f, g: X \setminus \{a\} \to \R$ и $\lim_{x \to a} f(x) = b$, $\lim_{x \to a} g(x) = c$. Тогда $\lim_{x \to a} (f(x) + g(x)) = b + c$, $\lim_{x \to a} f(x)g(x) = bc$.
\end{property}

\begin{proof}
    $x_{n} \in X \setminus \{a\}$, $x_{n} \to a \Rightarrow f(x_{n}) \to b$, $g(x_{n}) \to c \Rightarrow f(x_{n}) + g(x_{n}) \to b + c$, $f(x_{n})g(x_{n}) \to bc$. Утверждение следует по определению Гейне.
\end{proof}

В дальнейшем, говоря о <<пределе по подможеству>>, всегда будем иметь в виду подпространство с индуцированной метрикой.

\begin{property}[предел по подмножеству]
    \label{proper3}
    Пусть $E \subset X$, $a$ -- предельная точка множества $E$. Если $\lim_{x \to a} f(x) = b$, то $\lim_{x \to a} (f|_{E})(x) = b$.
\end{property}

\begin{proof}
    Пусть $x_{n} \subset E$, $x_{n} \to a$ и $x_{n} \neq a$. Тогда $(f|_{E})(x_{n}) = f(x_{n}) \to b$. По определению Гейне $\lim_{x \to a}(f|_{E})(x) = b$.
\end{proof}

Пусть $f: D \to \R$, $D \subset \R^{n}$, $a, u \in \R^{n}$ и $|u| = 1$.
$\{a + tu: 0 < t < \Delta\} \subset D$ для некоторого $\Delta > 0$.
Тогда $\lim_{t \to +0}f(a + tu)$ называется \textit{пределом $f$ в точке $a$ по направлению $u$}. По свойству (\ref{proper3}) $\lim_{t \to +0}f(a) = b \Rightarrow \lim_{t \to +0} f(a + tu) = b$. Обратное утверждение неверно.

\begin{example}
    $f(x, y) = \begin{cases}
    1, y = x^2, x \geq 0. \\
    0, \text{иначе}.
    \end{cases}$
    Рассмотрим $f(\frac{1}{n}, \frac{1}{n^{2}}) = 1$, $f(\frac{1}{n}, 0) = 0 \Rightarrow$ нет предела.
\end{example}

\begin{property}[локальная ограниченность]
    Если существует $\lim_{x \to a} f(x)$, то $\exists \delta > 0: f(\mathring{B}_{\delta}(a))$ ограничено.
\end{property}

\begin{proof}
    Достаточно положить в определении Коши $\epsilon = 1$.
\end{proof}

\begin{problem}
    Пусть $X, Y$ -- метрические пространства, причем $Y$ полное, $a$ -- предельная точка $X$, и $f: X \setminus \{a\} \to Y$. Покажите, что $\lim_{x \to a} f(x)$ существует тогда и только тогда, когда
    \[\forall \epsilon > 0 \ \exists \delta > 0 \ \forall x, x' \in X (x, x' \in \mathring{B}_{\delta}(a) \Rightarrow \rho_{Y}(f(x), f(x')) < \epsilon).\]
\end{problem}

Положим $a = (x_{0}, y_{0}), f: \mathring{B}_{\Delta}(x_{0}, y_{0}) \to \R$.
\begin{definition}
    Пусть $\exists \sigma > 0 \ \forall x \in (x_{0} - \sigma, x_{0} + \sigma) \setminus \{x_{0}\}$ существует $\lim_{y \to y_{0}} f(x, y) = \phi(x)$. Предел функции $\phi$ в точке $x_{0}$ называется \textit{повторным пределом}:
    \[\lim_{x \to x_{0}} \phi(x) = \lim_{x \to x_{0}}\lim_{y \to y_{0}}f(x, y).\]
\end{definition}

\begin{lemma}
    Пусть $f: \mathring{B}_{\Delta}(x_{0}, y_{0}) \to \R$, такая что
    \begin{enumerate}
        \item $\underset{y \to y_{0}}{\underset{x \to x_{0}}{\lim}} f(x, y) = b$; 
        \item $\exists \sigma > 0 \ \forall x \in (x_{0} - \sigma, x_{0} + \sigma) \setminus \{x_{0}\}$ существует $\lim_{y \to y_{0}} f(x, y) = \phi(x)$ (конечный).
    \end{enumerate}
    Тогда $\lim_{x \to x_{0}}\lim_{y \to y_{0}}f(x, y) = b$.
\end{lemma}

\begin{proof}
    Положим $\delta_{0} = \min\{\Delta, \sigma\}$. Зафиксируем $\epsilon > 0$. Тогда
    \[\exists \delta \in (0, \delta_{0}) \ \forall (x, y) \ \mathring{B}_{\delta}(x_{0}, y_{0})\left(|f(x, y) - b| < \frac{\epsilon}{2}\right).\]

    $\forall x \in (x_{0} - \delta, x_{0} + \delta) \setminus \{x_{0}\}$ существует $\phi(x)$. Перейдем к пределу при $y \to y_{0}$:
    \[|\phi(x) - b| \leq \frac{\epsilon}{2} < \epsilon.\]

    Это доказывает, что $\lim_{x \to x_{0}}\phi(x) = b$, что и требовалось доказать.
\end{proof}

\subsection{Непрерывные функции}

Пусть $(X, \rho_{X})$ и $(Y, \rho_{Y})$ -- метрические пространства и задана функция $f: X \to Y$.

\begin{definition}
    Функция $f$ \textit{непрерывна в точке} $a \in X$, если
    \[\forall \epsilon > 0 \ \exists \delta > 0 \ \forall x \in X \left(\rho_{X}(x, a) < \delta \Rightarrow \rho_{Y}(f(x), f(a)) < \epsilon\right)\]
    или, что эквивалентно,
    \[\forall \epsilon > 0 \ \exists \delta > 0 \ \forall x \in X \left(x \in B_{\delta}(a) \Rightarrow f(x) \in B_{\epsilon}(f(a))\right).\]
\end{definition}

\begin{example}
    Координатная функция $p_{i}: \R^{n} \to \R$, $p_{i}(x_{1}, \ldots, x_{n}) = x_{i}$, непрерывна в каждой точке $\R^{n}$. Это следует из неравенства $|x_{i} - a_{i}| \leq \rho_{2}(x, a)$.
\end{example}

\begin{lemma}
    Пусть $f: X \to Y$, $a \in X$. Следующие условия эквивалентны:
    \begin{enumerate}
        \item функция $f$ непрерывна в точке $a$;
        \item $\forall \{x_{n}\}$, $x_{n} \in X \left(x_{n} \to a \Rightarrow f(x_{n}) \to f(a)\right)$;
        \item $a$ -- изолированная точка множества $X$ или $a$ -- предельная точка $X$ и $\lim_{x \to a} f(x) = f(a)$.
    \end{enumerate}
\end{lemma}

\begin{proof}
    $(1) \Rightarrow (2)$ Выберем $\epsilon > 0$ и соответствующее $\delta > 0$ из определения непрерывности. Если $x_{n} \to a$ (в $X$), то существует такой номер $N$, что $\rho_{X}(x_{n}, a) < \delta$ при всех $n \geq N$, но тогда $\rho_{Y}(f(x_{n}), f(a)) < \epsilon$ при $n \geq N$. Это означает, что $f(x_{n}) \to f(a)$.

    $(2) \Rightarrow (3)$ Если $a$ -- предельная точка $X$, то из условия $\lim_{x \to a} f(x) = f(a)$ по определению Гейне.

    $(3) \Rightarrow (1)$ Если $a$ изолирована, то $B_{\delta_{0}}(a) \cap X = \{a\}$ для некоторого $\delta_{0} > 0$. Тогда для любого $\epsilon > 0$ определение непрерывности в точке $a$ выполняется при $\delta = \delta_{0}$. Пусть $a$ предельная для $X$. По определению предела по Коши $\forall \epsilon > 0 \ \exists \delta > 0 \ \forall x \in E \left(0 < \rho_{X}(x, a) < \delta \Rightarrow \rho_{Y}(f(x), f(a)) < \epsilon\right)$. Но последняя импликация верна и для $x = a$. Значит, функция $f$ непрерывна в точке $a$.
\end{proof}

\begin{theorem}[о непрерывности композиции]
    Пусть $(X, \rho_{X})$, $(Y, \rho_{Y})$ и $(Z, \rho_{Z})$ -- метрические пространства. Если функция $f: X \to Y$ непрерывна в точке $a \in X$, и функция $g: Y \to Z$ непрерывна в точке $f(a) \in Y$, то их композиция $g \circ f: X \to Z$ непрерывна в точке $a$.
\end{theorem}

\begin{proof}
    Пусть $x_{n} \to a$, тогда $f(x_{n}) \to f(a)$ и, значит, $g(f(x_{n})) \to g(f(a))$.
\end{proof}

\begin{corollary}
    Если функции $f, g: X \to \R$ непрерывны в точке $a$, то в этой точке также непрерывны функции $f + g$, $fg: X \to \R$.
\end{corollary}

\begin{definition}
    Функция $f \: X \to Y$ \textit{непрерывна} (на $X$), если $f$ непрерывна в каждой точке $X$.
\end{definition}

\begin{example}
    \textit{Многочленом} называется функция $P: \R^{n} \to \R$, $P(x) = \sum_{(k_{1}, \ldots, k_{n})}a_{k_{1}\ldots k_{n}}x_{1}^{k_{1}}\ldots x_{n}^{k_{n}}$, где суммирование ведется по конечному множеству наборов $(k_{1}, \ldots, k_{n})$ целых неотрицательных чисел. Многочлен $P$ непрерывен как линейная комбинация непрерывных функций $p_{1}^{i_{1}}\ldots p_{n}^{i_{n}}$, где $p_{i}(x) = x_{i}$.
\end{example}
    \subsubsection*{Некоторые приёмы разложения функций по формуле Тейлора}

\begin{enumerate}
	\item Если $f$ дифференцируема $n + 1$ раз в точке $x_0$ и 
	\[
	f'(x - x_0) = \suml_{k = 0}^n b_k (x - x_0)^k + o((x - x_0)^n),\ x \to x_0
	\]
	то
	\[
		f(x) = f(0) + \suml_{k = 0}^n \frac{b_k}{k + 1} (x - x_0)^{k + 1} + o((x - x_0)^{n + 1}),\ x \to x_0
	\]
	\begin{proof}
		Разложим $f(x)$ по формуле Тейлора:
		\[
		f(x) = f(0) + \suml_{k = 0}^n a_k (x - x_0)^{k + 1} + o((x - x_0)^{n + 1}),\ x \to x_0
		\]
		При этом $a_k = \frac{f^{(k + 1)}(x_0)}{(k + 1)!} = \frac{(f')^{(k)}(x_0)}{k!} \cdot \frac{1}{k + 1} = \frac{b_k}{k + 1}$
	\end{proof}
	\begin{example}
		\[
			(\arcctg x)' = -\frac{1}{1 + x^2}
		\]
		Разложим производную $\arcctg x$ в ряд Маклорена:
		\[
			-\frac{1}{1 + x^2} = -(1 + x^2)^{-1} = \suml_{k = 0}^n C_{-1}^k \cdot k! \cdot x^{2k} + o(x^{2n + 1}) = \suml_{k = 0}^n (-1)^{k + 1} x^{2k} + o(x^{2n + 1})
		\]
		Отсюда получаем, что
		\[
			\arcctg x = \frac{\pi}{2} + \suml_{k = 0}^n \frac{(-1)^{k + 1}}{2k + 1} x^{2k + 1} + o(x^{2n + 2}),\ x \to 0
		\]
	\end{example}

	\item Метод неопределённых коэффициентов
	\begin{example}
		Пусть $f$ имеет вид
		\[
			f(x) = \System{
				&{x \ctg x,\ x \neq 0}
				\\
				&{1,\ x = 0}
			}
		\]
		Тогда $f(x)$ при $x \neq 0$ имеет ещё вид
		\begin{align*}
			&f(x) = \frac{\cos x}{\frac{\sin x}{x}} = \frac{1 - \frac{x^2}{2!} + \frac{x^4}{4!} + o(x^5)}{1 - \frac{x^2}{3!} + \frac{x^4}{5!} + o(x^5)} = a_0 + a_2x^2 + a_4x^4 + o(x^5),\ x \to 0
			\\
			&\left(1 - \frac{x^2}{2} + \frac{x^4}{24} + o(x^5)\right) = \left(a_0 + a_2x^2 + a_4x^4 + o(x^5)\right) \cdot \left(1 - \frac{x^2}{6} + \frac{x^4}{120} + o(x^5)\right),\ x \to 0
		\end{align*}
		Чтобы равенство выполнялось, должны быть равны коэффициенты при одинаковых степенях у приведённых многочленов. Отсюда имеем
		\begin{align*}
			&{1 = a_0}
			\\
			&{-\frac{1}{2} = a_2 - \frac{a_0}{6} \Ra a_2 = -\frac{1}{3}}
			\\
			&{\frac{1}{24} = a_4 - \frac{a_2}{6} + \frac{a_0}{120} \Ra a_4 = -\frac{1}{45}}
		\end{align*}
		То есть
		\[
			f(x) = 1 - \frac{x^2}{3} - \frac{x^4}{45} + o(x^5)
		\]
		Но опять же, нужно доказать, что это формула Тейлора. Иначе это просто асимптотическое разложение
	\end{example}

	\item Применение формулы Тейлора для подсчёта пределов
	\begin{example}
		Вычислим следующий предел:
		\[
			\liml_{x \to 0} \left(e^{x^2 \ctg x} + \ln(1 - x)\right)^{1 / \left(\arcctg (\sh x) + \sin x - \frac{\pi}{2}\right)}
		\]
		Заметим, что он представим в виде
		\[
			\liml_{x \to 0} \left(1 + u(x)\right)^{1 / v(x)} = \liml_{x \to 0} e^{ \frac{\ln (1 + u(x))}{v(x)}}
		\]
		где $u, v = o(1),\ x \to 0$. Тогда, в силу эквивлентности
		\[
			\liml_{x \to 0} e^{ \frac{\ln (1 + u(x))}{v(x)}} = \liml_{x \to 0} e^{\frac{u(x)}{v(x)}}
		\]
		Распишем $u(x)$:
		\begin{multline*}
			u(x) = e^{x\left(1 - \frac{1}{3}x^2 - \frac{1}{45}x^4 + o(x^5)\right)} - x - \frac{x^2}{2} - \frac{x^3}{3} - \frac{x^4}{4} - \frac{x^5}{5} + o(x^5) - 1 =
			\\
			1 + x\left(1 - \frac{1}{3}x^2 + o(x^3)\right) + \frac{x^2\left(1 - \frac{1}{3}x^2 + o(x^3)\right)^2}{2!} + \frac{x^3\left(1 - \frac{1}{3}x^2 + o(x^3)\right)^3}{3!} + 
			\\
			o\left(x^3\left(1 - \frac{1}{3}x^2 + o(x^3)\right)^3\right) - 1 - x - \frac{x^2}{2} - \frac{x^3}{3} + o(x^3) = \left(-\frac{1}{3} + \frac{1}{6} - \frac{1}{3}\right)x^3 + o(x^3) = 
			\\
			-\frac{1}{2}x^3 + o(x^3)
		\end{multline*}
		Теперь $v(x)$:
		\[
			\sh x = \frac{e^x - e^{-x}}{2} = x + \frac{x^3}{3!} + \frac{x^5}{5!} + o(x^6)
		\]
		\begin{multline*}
			\arcctg(\sh x) = \frac{\pi}{2} - \left(x + \frac{x^3}{3!} + \frac{x^5}{5!} + o(x^6)\right) + \frac{1}{3}\left(x + \frac{x^3}{3!} + \frac{x^5}{5!} + o(x^6)\right)^3 -
			\\
			\frac{1}{5}\left(x + \frac{x^3}{3!} + \frac{x^5}{5!} + o(x^6)\right)^5 + o\left(\left(x + \frac{x^3}{3!} + \frac{x^5}{5!} + o(x^6)\right)^6\right) =
			\\
			\frac{\pi}{2} - x + \left(\frac{1}{3} - \frac{1}{6}\right)x^3 + \left(\frac{1}{6} - \frac{1}{120} - \frac{1}{5}\right)x^5 + o(x^6),\ x \to 0
		\end{multline*}
		В итоге имеем
		\begin{multline*}
			v(x) = \frac{\pi}{2} - x + \frac{1}{6}x^3 - \frac{1}{24}x^5 + o(x^6) + x - \frac{1}{6}x^3 + \frac{1}{120}x^5 + o(x^6) - \frac{\pi}{2} =
			\\
			-\frac{1}{30}x^5 + o(x^6),\ x \to 0
		\end{multline*}
		Отсюда предел получает вид
		\[
			\liml_{x \to 0} e^{\frac{u(x)}{v(x)}} = \liml_{x \to 0} e^{\frac{-\frac{1}{2}x^3 + o(x^3)}{-\frac{1}{30}x^5 + o(x^6)}} = +\infty
		\]
	\end{example}
\end{enumerate}

\subsection{Исследование функции с помощью производной}

\begin{theorem} \label{monoF} (Необходимое и достаточное условия монотонности функции)
	Если $f$ дифференцируема на $(a; b)$, то
	\begin{enumerate}
		\item $\forall x \in (a; b)\  f'(x) \ge 0 \lra f(x)$ неубывающая на $(a; b)$
		
		\item $\forall x \in (a; b)\ f'(x) \le 0 \lra f(x)$ невозрастающая на $(a; b)$
		
		\item $\forall x \in (a; b)\  f'(x) > 0 \Ra f(x)$ возрастающая на $(a; b)$
		
		\item $\forall x \in (a; b)\ f'(x) < 0 \Ra f(x)$ убывающая на $(a; b)$
	\end{enumerate}
\end{theorem}

\begin{proof}
	Докажем первый случай. Начнём с утверждения $\Ra$:
	
	Рассмотрим $\forall a < x_1 < x_2 < b$. Тогда, по теореме Лагранжа
	\[
		\exists c \in (x_1; x_2) \such \frac{f(x_2) - f(x_1)}{x_2 - x_1} = f'(c)
	\]
	Отсюда
	\[
		f(x_2) - f(x_1) = f'(c) (x_2 - x_1) \ge 0
	\]
	
	Теперь докажем $\La$: посчитаем производную в некоторой точке $x_0 \in (a; b)$:
	\[
		f'(x_0) = f'_+(x_0) = \liml_{\Delta x \to 0+} \frac{f(x_0 + \Delta x) - f(x_0)}{\Delta x}
	\]
	Так как $x_0 + \Delta x > x_0 \Ra f(x_0 + \Delta x) \ge f(x_0)$. Отсюда
	\[
		f'(x_0) = f'_+(x_0) \ge 0
	\]
\end{proof}

\begin{note}
	В случаях 3 и 4 утверждение верно в одну сторону. Контрпример: 
	\[
		y = \pm x^3,\ x \in (-1; 1)
	\]
\end{note}

\begin{note}
	Если дополнительно потребовать непрерывности $f$ на $[a; b]$, то в теорема будет верна на $[a; b]$.
\end{note}

\begin{theorem} (Первое достаточное условие локального экстремума)
	Пусть $f$ непрерывна в $U_\delta(x_0)$, дифференцируема в $\mc{U}_\delta(x_0)$. Тогда
	\begin{enumerate}
		\item Если $\forall x \in (x_0 - \delta, x_0)\ f'(x) < 0$ и $\forall x \in (x_0, x_0 + \delta)\ f'(x) > 0$, то $x_0$ является точкой строгого локального минимума.
		
		\item Если $\forall x \in (x_0 - \delta, x_0)\ f'(x) > 0$ и $\forall x \in (x_0, x_0 + \delta)\ f'(x) < 0$, то $x_0$ является точкой строгого локального максимума.
	\end{enumerate}
\end{theorem}

\begin{proof}
	Докажем первый случай. Выберем $x_1 \in (x_0 - \delta; x_0)$. Тогда, $f$ непрерывна на $[x_1; x_0]$ и $\forall x \in (x_1; x_0)\ f'(x) < 0$. То есть, $f$ убывает на $[x_1; x_0]$ по теореме \ref{monoF}. Аналогично доказывается, что $f$ возрастает на $[x_0; x_2]$. Значит
	\[
		\exists \delta' = \min(x_0 - x_1, x_2 - x_0) \such \forall x \in \mc{U}_{\delta'}(x_0)\ f(x) > f(x_0)
	\]
	$x_0$ - локальный минимум.
\end{proof}

\begin{theorem} (Второе достаточное условие локального экстремума)
	Пусть $f^{(n)}(x_0) \neq 0$, а $f'(x_0) = \ldots = f^{(n - 1)}(x_0) = 0$. Тогда
	\begin{enumerate}
		\item При $n$ - чётном 
		\[
			\System{
				&{f^{(n)}(x_0) > 0 \Ra x_0 \text{ - точка строгого локального минимума}}
				\\
				&{f^{(n)}(x_0) < 0 \Ra x_0 \text{ - точка строгого локального максимума}}
			}
		\]
		
		\item При $n$ - нечётном $x_0$ не является точкой локального экстремума
	\end{enumerate} 
\end{theorem}

\begin{proof}
	По формуле Тейлора с остаточным членом в форме Пеано можно записать разложение:
	\[
		f(x) = f(x_0) + \frac{f^{(n)}(x_0)}{n!}(x - x_0)^n + o((x - x_0)^n),\ x \to x_0
	\]
	Перепишем это выражение в другом виде:
	\[
		n! \cdot \frac{f(x) - f(x_0)}{f^{(n)}(x_0)} = (x - x_0)^n + o((x - x_0)^n),\ x \to x_0
	\]
	
	Если $n$ чётно, то справа стоит положительное число. То есть
	\[
		\frac{f(x) - f(x_0)}{f^{(n)}(x_0)} > 0,\ x \to x_0
	\]
	Если $f^{(n)}(x_0) > 0$, то и $f(x) > f(x_0),\ \forall x \in \mc{U}_\delta(x_0)$. Следовательно $x_0$ - точка строгого локального минимума. Аналогично при $f^{(n)}(x_0) < 0$ $x_0$ - точка строгого локального максимума.
	
	Если $n$ нечётно, то выражение справа положительно при $x \to x_0+0$ и отрицательно при $x \to x_0-0$. Это значит, что какой бы знак мы не выбрали для $f^{(n)}(x_0)$, разность $f(x) - f(x_0)$ принимает разные знаки по разные стороны от $x_0$, то есть $x_0$ не является точкой локального экстремума.
\end{proof}

\begin{definition}
	Функция $f$ называется \textit{выпуклой вниз(вогнутой вверх)} на $(a; b)$, если её график лежит не выше любой хорды, стягивающей две точки графика.
\end{definition}

\begin{definition}
	Функция $f$ называется \textit{выпуклой вверх(вогнутой вниз)} на $(a; b)$, если её график лежит не ниже любой хорды, стягивающей две точки графика.
\end{definition}

%\subsubsection*{Геометрический смысл выпуклости}

%% Нарисовать. 1:15:42 16я лекция 2021г

\subsubsection*{Аналитический смысл выпуклости}

Возьмём 2 точки $a < x_1 \le x_2 < b$. Координаты точки на хорде можно выразить параметрически:
\[
	\System{
		&x_0 = tx_1 + (1 - t)x_2
		\\
		&y_0 = tf(x_1) + (1 - t)f(x_2)
	}
	,\ t \in [0; 1]
\]
Выпуклость вниз по определению означает, что
\[
	f(tx_1 + (1 - t)x_2) \le tf(x_1) + (1 - t)f(x_2)
\]

\begin{note}
	Если неравенство - строгое $\forall t \in (0; 1),\ \forall x_1, x_2 \in (a; b)$, то $f$ \textit{строго выпукла вниз (вверх)}
\end{note}
    \begin{theorem} (Конечная аддитивность мер Лебега и Жордана)
	Если $A_1, \ldots, A_N$ - измеримые по Лебегу (Жордану) непересекающиеся подмножества $K_I$, то 
	\[
		\jlm \left(\bscup_{i = 1}^N A_i\right) = \sum_{i = 1}^N \jlm(A_i)
	\]
\end{theorem}

\begin{proof}
	Чтобы показать верность равенства, мы воспользуемся критерием измеримости и, следовательно, покрытиями $A_i$ через элементарные множества. Основная трудность состоит в том, что пересечения этих покрытий не внесут какого-то значимого вклада.
	
	Проведём индукцию по $N$:
	\begin{itemize}
		\item База $N = 2$: рассматриваем $A_1 \sqcup A_2$. По критерию измеримости
		\[
			\forall \eps > 0\ \exists M_{1, \eps}, M_{2, \eps} \such \jlm(A_1 \tr M_{1, \eps}) < \eps,\ \ \jlm(A_2 \tr M_{2, \eps}) < \eps
		\]
		Мы уже знаем, что $\jlm(A_1 \sqcup A_2) \le \jlm(A_1) + \jlm(A_2)$ по свойствам мер. Нужно доказать неравенство в другую сторону, этим мы и займёмся. Для начала, отметим следующие включения:
		\begin{align*}
			&{A_i \subset (A_i \tr M_{i, \eps}) \cup M_{i, \eps} \Lora \jlm(A_i) \le \jlm(A_i \tr M_{i, \eps}) + |M_{i, \eps}|}
			\\
			&{M_{i, \eps} \subset (M_{i, \eps} \tr A_i) \cup A_i \Lora |M_{i, \eps}| \le \jlm(A_i \tr M_{i, \eps}) + \jlm(A_i)}
		\end{align*}
		Отсюда приятное неравенство:
		\[
			\left|\jlm(A_i) - |M_{i, \eps}|\right| \le \jlm(A_i \tr M_{i, \eps}) < \eps
		\]
		Теперь, напишем цепочку включений для дизъюнктного объединения и его приближения:
		\begin{align*}
			&{A_1 \sqcup A_2 \subset (A_1 \tr M_{1, \eps}) \cup (A_2 \tr M_{2, \eps}) \cup (M_{1, \eps} \cup M_{2, \eps})}
			\\
			&{M_{1, \eps} \cup M_{2, \eps} \subset (A_1 \tr M_{1, \eps}) \cup (A_2 \tr M_{2, \eps}) \cup (A_1 \sqcup A_2)}
		\end{align*}
		Аналогичное следствие, как и из предыдущих вложений:
		\[
			\big|\jlm(A_1 \sqcup A_2) - |M_{1, \eps} \cup M_{2, \eps}|\big| < 2\eps
		\]
		Вернёмся к уже упомянутому неравенству выше и продолжим его (воспользовались первой оценкой на разность мер):
		\[
			\jlm(A_1 \sqcup A_2) \le \jlm(A_1) + \jlm(A_2) < |M_{1, \eps}| + |M_{2, \eps}| + 2\eps
		\]
		По свойствам мер мы знаем ещё это равенство:
		\[
			|M_{1, \eps} \cup M_{2, \eps}| + |M_{1, \eps} \cap M_{2, \eps}| = |M_{1, \eps}| + |M_{2, \eps}|
		\]
		Нужно оценить пересечение, потому что на объединение неравенство написано уже выше:
		\begin{multline*}
			M_{1, \eps} \cap M_{2, \eps} \subset \big((M_{1, \eps} \tr A_1) \cup A_1\big) \cap \big((M_{2, \eps} \tr A_2) \cup A_2\big) =
			\\
			\Big(\big((M_{1, \eps} \tr A_1) \cup A_1\big) \cap (M_{2, \eps} \tr A_2)\Big) \cup \Big(\big((M_{1, \eps} \tr A_1) \cup A_1\big) \cap A_2\Big) =
			\\
			\Big(\big((M_{1, \eps} \tr A_1) \cup A_1\big) \cap (M_{2, \eps} \tr A_2)\Big) \cup \big((M_{1, \eps} \tr A_1) \cap A_2\big) \subset
			\\
			(M_{1, \eps} \tr A_1) \cup (M_{2, \eps} \tr A_2) \Lora |M_{1, \eps} \cap M_{2, \eps}| \le \jlm(M_{1, \eps} \tr A_1) + \jlm(M_{2, \eps} \tr A_2) < 2\eps
		\end{multline*}
		Отсюда рождается ещё один переход в неравенстве:
		\[
			\jlm(A_1) + \jlm(A_2) < |M_{1, \eps} \cup M_{2, \eps}| + 4\eps
		\]
		Используя неравенство с $|M_{1, \eps} \cup M_{2, \eps}|$ выше, получаем требуемое:
		\[
			\forall \eps > 0 \quad \jlm(A_1) + \jlm(A_2) < \jlm(A_1 \sqcup A_2) + 6\eps
		\]
		
		\item Переход $N > 2$: просто полагаем первые $N - 1$ множество за одно, применяем базу и раскрываем предположение индукции.
	\end{itemize}
\end{proof}

\begin{theorem}
	Если $\{A_i\}_{i = 1}^\infty$ - измеримые по Лебегу подмножества $K_I$, то $\bigcup_{i = 1}^\infty A_i$ также измеримо по Лебегу
\end{theorem}

\begin{proof}
	Сначала рассмотрим $\bscup_{i = 1}^\infty A_i$. Поскольку
	\[
		\forall m \in \N \quad \bscup_{i = 1}^m A_i \subset K_I \Ra \sum_{i = 1}^m \mu(A_i) \le 1
	\]
	то имеет место неравенство $0 \le \sum_{i = 1}^\infty \mu(A_i)\le 1$ и ряд сходится. Это даёт нам возможность оценить его конец:
	\[
		\forall \eps > 0\ \exists r \in \N \such \sum_{i = r + 1}^\infty \mu(A_i) < \frac{\eps}{2}
	\]
	Плюс отметим следующий факт, следующий из измеримости множеств по Лебегу:
	\[
		\forall i \in \N\ \forall \eps > 0\ \exists M_{i, \eps} \such \mu(A_i \tr M_{i, \eps}) < \frac{\eps}{2 \cdot r}
	\]
	Положим за $M = \bigcup_{i = 1}^r M_{i, \eps}$, которое по понятным причинам будет элементарным множеством. Это наш кандидат для критерия измеримости:
	\begin{multline*}
		\left(\bscup_{i = 1}^\infty A_i\right) \tr M \subset \left(\bscup_{i = 1}^r A_i \bs M\right) \cup \left(\bscup_{i = r + 1}^\infty A_i\right) \cup \left(M \bs \bscup_{i = 1}^r A_i\right) \subset
		\\
		\bscup_{i = 1}^r (A_i \bs M_{i, \eps}) \cup \left(\bscup_{i = r + 1}^\infty A_i\right) \subset \bigcup_{i = 1}^r (A_i \tr M_{i, \eps}) \cup \left(\bscup_{i = r + 1}^\infty A_i\right)
	\end{multline*}
	Отсюда вытаскиваем неравенство на меры:
	\[
		\mu^* \left(\left(\bscup_{i = 1}^\infty A_i\right) \tr M\right) \le \sum_{i = 1}^r \mu^* (A_i \tr M_{i, \eps}) + \sum_{i = r + 1}^\infty \mu^* (A_i) < \eps
	\]
	
	Теперь скажем про общий случай. Он очень просто сводится к уже рассмотренному:
	\[
		\bigcup_{i = 1}^\infty A_i = \bscup_{i = 1}^\infty \wt{A}_i,\ \ \wt{A}_i = A_i \bs \bigcup_{j = 1}^{i - 1} \wt{A}_j
	\]
\end{proof}

\begin{note}
	Последняя теорема означает, что совокупность измеримых по Лебегу подмножеств $K_I$ образует \textit{$\sigma$-алгебру} множеств (то есть счётная алгебра).
\end{note}

\begin{theorem} ($\sigma$-аддитивность меры Лебега)
	Если $\{A_i\}_{i = 1}^\infty$ - измеримые по Лебегу непересекающиеся подмножества $K_I$, то
	\[
		\mu\left(\bscup_{i = 1}^\infty A_i\right) = \sum_{i = 1}^\infty \mu(A_i)
	\]
\end{theorem}

\begin{proof}
	Положим $A := \bscup_{i = 1}^\infty A_i$. Так как измеримость $A$ установлена предыдущей теоремой, то остаётся доказать равенство мер. Будем делать это через неравенства в две стороны:
	\begin{itemize}
		\item $\ge$
		\[
			\forall m \in \N \quad \bscup_{i = 1}^m A_i \subset A \Ra \mu\left(\bscup_{i = 1}^m A_i\right) = \sum_{i = 1}^m \mu(A_i) \le \mu(A) \Ra \sum_{i = 1}^\infty \mu(A_i) \le \mu(A)
		\]
		
		\item $\le$
		
		\[
			A \subset \bscup_{i = 1}^\infty A_i \Ra \mu^*(A) \le \sum_{i = 1}^\infty \mu^*(A_i) \Ra \mu(A) \le \sum_{i = 1}^\infty \mu(A_i)
		\]
	\end{itemize}
\end{proof}

\begin{definition}
	$\{A_i\}_{i = 1}^\infty$ - последовательность множеств. Определим супремум и инфинум следующим образом:
	\begin{align*}
		&{\sup_i \{A_i\} := \bigcup_{i = 1}^\infty A_i}
		\\
		&{\inf_i \{A_i\} := \bigcap_{i = 1}^\infty A_i}
	\end{align*}
\end{definition}

\begin{definition}
	Пределы для множств введём следующим образом:
	\begin{itemize}
		\item Если $\forall i \in \N\ \ A_i \subset A_{i + 1}$, то
		\[
			\liml_{i \to \infty} A_i := \sup \{A_i\} = \bigcup_{i = 1}^\infty A_i
		\]
		
		\item Если $\forall i \in \N\ \ A_i \supset A_{i + 1}$, то
		\[
			\liml_{i \to \infty} A_i := \inf \{A_i\} = \bigcap_{i = 1}^\infty A_i
		\]
		
		\item Верхний предел определим через эквивалентное свойство:
		\[
			\varlimsup_{i \to \infty} A_i := \liml_{i \to \infty} \sup_{k \ge i} \{A_k\}
		\]
		
		\item Аналогично с нижним пределом:
		\[
			\varliminf_{i \to \infty} A_i := \liml_{i \to \infty} \inf_{k \ge i} \{A_k\}
		\]
		
		\item Отсюда можно дать определение предела в общем случае как общее значение нижнего и верхнего предела:
		\[
			\liml_{i \to \infty} A_i = \varliminf_{i \to \infty} A_i = \varlimsup_{i \to \infty} A_i
		\]
	\end{itemize}
\end{definition}

\begin{theorem} (Непрерывность меры Лебега) \label{4lemth}
	Если последовательность $\{A_i\}_{i = 1}^\infty$ измеримых по Лебегу подмножеств $K_I$ имеет предел $A := \liml_{i \to \infty} A_i$, то $A$ тоже измеримо по Лебегу, причём
	\[
		\mu(A) = \liml_{i \to \infty} \mu(A_i)
	\]
\end{theorem}

\begin{lemma}
	Пусть последовательность $\{A_i\}_{i = 1}^\infty$ такова, что $\forall i \in \N\ \ A_i \subset A_{i + 1}$. При этом $A_i$ - измеримое по Лебегу подмножество $K_I$. Тогда
	\[
		\mu\left(\bigcup_{i = 1}^\infty A_i\right) = \liml_{i \to \infty} \mu(A_i)
	\]
\end{lemma}

\begin{proof}
	Отметим уже известный факт:
	\[
		\bigcup_{i = 1}^\infty A_i = \bscup_{i = 1}^\infty \wt{A}_i, \quad \wt{A}_i = A_i \bs \bigcup_{j = 1}^{i - 1} A_j
	\]
	Благодаря этому мы можем применить $\sigma$-аддитивность меры Лебега и записать следующую цепочку:
	\[
		\mu\left(\bigcup_{i = 1}^\infty A_i\right) = \sum_{i = 1}^\infty \mu(\wt{A}_i) = \liml_{N \to \infty} \sum_{i = 1}^N \mu(\wt{A}_i) = \liml_{N \to \infty} \mu\left(\bscup_{i = 1}^N \wt{A}_i\right) = \liml_{N \to \infty} \mu(A_N)
	\]
\end{proof}

\begin{lemma} \label{justLemma}
	Пусть последовательность $\{A_i\}_{i = 1}^\infty$ такова, что $\forall i \in \N\ \ A_i \supset A_{i + 1}$. При этом $A_i$ - измеримое по Лебегу подмножество $K_I$. Тогда
	\[
		\mu\left(\bigcap_{i = 1}^\infty A_i\right) = \liml_{i \to \infty} \mu(A_i)
	\]
\end{lemma}

\begin{proof}
	Введём последовательность $\{B_i\}_{i = 1}^\infty$ следующим образом:
	\[
		\forall i \in \N \quad B_i = A_1 \bs A_i
	\]
	Тогда $\forall i \in \N\ \ B_i \subset B_{i + 1}$, что позволяет применить к ней предыдущую лемму:
	\[
		\mu\left(\bigcup_{i = 1}^\infty B_i\right) = \liml_{i \to \infty} \mu(B_i) = \mu(A_1) - \liml_{i \to \infty} \mu(A_i)
	\]
	При этом верно равенство:
	\[
		\bigcup_{i = 1}^\infty B_i = \bigcup_{i = 1}^\infty (A_1 \bs A_i) = A_1 \bs \bigcap_{i = 1}^\infty A_i
	\]
	Отсюда получаем следующее:
	\[
		\mu\left(\bigcup_{i = 1}^\infty B_i\right) = \liml_{i \to \infty} \mu(B_i) = \mu(A_1) - \liml_{i \to \infty} \mu(A_i) = \mu(A_1) - \mu\left(\bigcap_{i = 1}^\infty A_i\right)
	\]
\end{proof}

\begin{lemma} \ref{lem1}
	Если последовательность $\{A_i\}_{i = 1}^\infty$ - это измеримые по Лебегу подмножества $K_I$, то
	\[
		\varlimsup_{i \to \infty} \mu(A_i) \le \mu(\varlimsup_{i \to \infty} A_i)
	\]
\end{lemma}

\begin{proof}
	Распишем меру справа в неравенстве:
	\[
		\mu(\varlimsup_{i \to \infty} A_i) = \mu(\liml_{i \to \infty} \sup_{k \ge i} \{A_k\})
	\]
	Супремумы образуют невозрастающую последовательность. Отсюда по предыдущей лемме
	\[
		\mu(\liml_{i \to \infty} \sup_{k \ge i} \{A_k\}) = \liml_{i \to \infty} \mu(\sup_{k \ge i} \{A_k\}) = \liml_{i \to \infty} \mu\left(\bigcup_{k = i}^\infty A_k\right) \ge \liml_{i \to \infty} \sup_{k \ge i} \mu(A_k)
	\]
\end{proof}

\begin{lemma}
	Если последовательность $\{A_i\}_{i = 1}^\infty$ - это измеримые по Лебегу подмножества $K_I$, то
	\[
		\varliminf_{i \to \infty} \mu(A_i) \ge \mu(\varliminf_{i \to \infty} A_i)
	\]
\end{lemma}

\begin{proof}
	Аналогично третьей лемме, только ссылаться будем ещё и на первую:
	\[
		\mu(\varliminf_{i \to \infty} A_i) = \mu(\liml_{i \to \infty} \inf_{k \ge i} \{A_k\}) = \liml_{i \to \infty} \mu(\inf_{k \ge i} \{A_k\}) = \liml_{i \to \infty} \mu\left(\bigcap_{k = 1}^\infty A_k\right) \le \liml_{i \to \infty} \inf_{k \ge i} \mu(A_k)
	\]
\end{proof}

\begin{proof} (теоремы \ref{4lemth})
	Запишем неубывающую цепочку:
	\[
		\varlimsup_{i \to \infty} \mu(A_i) \le \mu(\varlimsup_{i \to \infty} A_i) = \mu(\varliminf_{i \to \infty} A_i) \le \varliminf_{i \to \infty} \mu(A_i) \le \varlimsup_{i \to \infty} \mu(A_i)
	\]
\end{proof}

\begin{note}
	Всё это время мы жили внутри куба $K_I$. Однако, пришло время выйти за его рамки: замостим всё пространство параллельными сдвигами $K_I$. Тогда, для любого множества $A$ мы можем определить меры Лебега и Жордана следующим образом:
	\[
		\jlm(A) := \sum_{\vv{m} \in \Z^n} \jlm\big((\vv{m} + K_I) \cap A\big)
	\]
	где $\vv{m} + K_I$ нужно воспринимать как сумму Минковского (каждая точка $K_I$ - это вектор. Новые точки получаются прибавлением $\vv{m}$ к этим векторам).
	
	Стоит отметить, что с мерой Жордана не всё так просто. Для Лебега мы уже доказали, что если каждый кусочек внутри своего куба будет измерим, то и всё множество тоже будет измеримо. Однако, для Жордана это неверно. Либо ряд сойдётся, либо нет (при этом порядок суммирования неважен в силу положительности каждого слагаемого, была теорема о таких рядах). Поэтому, когда говорят о мере Жордана, то предпочитают рассматривать только ограниченные множества $A$.
\end{note}

\begin{note}
	Отдельно отметим, что далеко не все свойства остаются верными для любых множеств из $\R^n$. В частности, ломается непрерывность:
	
	Рассмотрим $A_n = [n; +\infty)$. Для этой последовательности ломается лемма \ref{justLemma}, ибо мера $A_1$ - бесконечность.
\end{note}
    \section{Вектор-функции и топология пространства $\R^n$}

\subsection{Пространство $\R^n$}

\subsubsection*{Алгебраические структуры}

\begin{definition}
	\textit{Линейным пространством} над полем действительных
	чисел (линейным действительным пространством) называется
	множество $X$, на котором определены операции
	$+: X \times X \ra X$ и $\cdot: \R \times X \ra X$,
	удовлетворяющие аксиомам линейного пространства:
	\begin{enumerate}
		\item $(\forall \vec{x}, \vec{y} \in X)\ \ \vec{x} + \vec{y}
			= \vec{y} + \vec{x}\ \ $ \textit{(ассоциативность сложения)}
		
		\item $(\forall \vec{x}, \vec{y}, \vec{z} \in X)\ \ 
			(\vec{x} + \vec{y}) + \vec{z} = \vec{x} + 
			(\vec{y} + \vec{z})\ \ $ \textit{(ассоциативность сложения)}
		
		\item $(\exists \vec{0} \in X)(\forall \vec{x} \in X)
			\ \ \vec{x} + \vec{0} = \vec{x}\ \ $ \textit{(нейтральный 
			элемент относительно сложения)}
		
		\item $(\forall \vec{x} \in X)(\exists
			(-\vec{x}) \in X)\ \ \vec{x} + (-\vec{x}) = \vec{0}\ \ $
			\textit{(обратный элемент относительно сложения)}
		
		\item $(\forall \alpha, \beta \in \R)(\forall
			\vec{x} \in X)\ \ \alpha(\beta \vec{x}) =
			(\alpha \beta) \vec{x}\ \ $ \textit{(ассоциативность
			умножения на скаляр)}
		
		\item $(\forall \alpha, \beta \in \R)(\forall \vec{x}
			\in X)\ \ (\alpha + \beta) \vec{x} =
			\alpha \vec{x} + \beta \vec{x}\ $ \textit{(дистрибутивность
			относительно скаляра)}
		
		\item $(\forall \alpha \in \R)(\forall \vec{x},
			\vec{y} \in X)\ \ \alpha(\vec{x} + \vec{y}) =
			\alpha\vec{x} + \alpha\vec{y}\ \ $ \textit{(дистрибутивность
			относительно вектора)}
		
		\item $(\forall \vec{x} \in X)\ \ 1 \cdot \vec{x} = \vec{x}\ \ $
			\textit{(нейтральный элемент относительно умножения)}
	\end{enumerate}
\end{definition}

\begin{definition}
	\[
		\R^n = \underbrace{\R \times \R \times \dots \times \R}_{n}	
	\]
	$\R^n$ является линейным пространством над
	полем $\R$ (линейным действительным пространством).
	\[
		\vec{x} = (x_1, \ldots, x_n) \text{ - вектор};\ 
		\ \Matrix{&\xi^1 \\ &\vdots \\ &\xi^n}
		\text{ - координатный столбец}
	\]
\end{definition}

\begin{proposition}
	Ноль и обратный элемент единственны
\end{proposition}

\begin{proof}
	Единственность обратного элемента:
	\[
		(\vec{x} + (-\vec{x})_1) + (-\vec{x})_2 =
		(-\vec{x})_2 = (-\vec{x})_1 =
		(\vec{x} + (-\vec{x})_2) + (-\vec{x})_1
	\]
	Единственность нуля:
	\[
		\vec{0}_1 + \vec{0}_2 = \vec{0}_1 = \vec{0}_2 =
		\vec{0}_2 + \vec{0}_1
	\]
\end{proof}

\begin{proposition}
	\[
		0 \cdot \vec{x} = \vec{0}
	\]
\end{proposition}

\begin{proof}
	\[
		(0 + 0)\vec{x} = 0\vec{x} = 0\vec{x} + 0\vec{x}
	\]
	К обеим частям добавим обратный элемент к $0\vec{x}$ и
	получим:
	\[
		\vec{0} = 0\vec{x}
	\]
\end{proof}

\begin{proposition}
	\[
		(-1)\vec{x} = -\vec{x}
	\]
\end{proposition}

\begin{proof}
	Рассмотрим выражение
	\[
		\vec{x} + (-1)\vec{x} = 1\vec{x} + (-1)\vec{x} =
		(1 - 1)\vec{x} = 0\vec{x} = \vec{0}
	\]
	То есть $(-1)\vec{x}$ является обратным к $\vec{x}$
	по сложению. Отсюда по единственности
	\[
		(-1)\vec{x} = -\vec{x}
	\]
\end{proof}

\begin{anote}
	В 2023 году Алексей Леонидович не строил теории для комплексных чисел,
	поэтому весь материал, связанный с ними, остался нетронутым с 2021 года.
	К этому же числу относятся линейные отображения.
\end{anote}

\begin{definition}
	\textit{Комплексным} линейным пространством
	называется линейное пространство над $\Cm$.
	Определяется аналогично вещественному.
\end{definition}

\begin{definition}
	Отображение $L: X_1 \ra X_2$ линейного пространства $X_1$ над
	$\R(\Cm)$ на линейное пространство $X_2$ над
	$\R(\Cm)$ называется
	\textit{линейным отображением (оператором)}, если
	\begin{itemize}
		\item $(\forall \vec{x}, \vec{y} \in X_1)\ \ 
			L(\vec{x} + \vec{y}) = L(\vec{x}) + L(\vec{y})$
		
		\item $(\forall \alpha \in \R(\Cm))(\forall \vec{x}
			\in X_1)\ \ L(\alpha \vec{x}) = \alpha L(\vec{x})$
	\end{itemize}
\end{definition}

\begin{definition}
	Если существует биекция линейного пространства
	$X_1$ на $X_2$, являющаяся линейным оператором
	вместе со своим обратным, то $X_1 \cong X_2$ (изоморфны)
\end{definition}

\begin{definition}
	Говорят, что на действительном линейном пространстве
	$X$ задана \textit{комплексная структура}, если
	существует линейный оператор $\goth{j}: X \to X$
	такой, что
	\[
		\goth{j}^2  = -\id_X
	\]
\end{definition}

\begin{example}
	На $\R^2$ комплексная структура задаётся оператором
	\[
		\goth{j}: (x, y) \ra (-y, x)
	\]
\end{example}

\begin{example}
	\[
		\{(x_1, x_2) \in \R^2 \such x_2 = 0\} \cong \R
	\]
\end{example}

\begin{example}
	\[
		\{(z_1, z_2) \in \Cm^2 \such z_2 = 0\} \cong \Cm
	\]
\end{example}

\begin{lemma}
	Комплексная структура на $\R^{2n}$ задаётся оператором с матрицей
	\[
		\Matrix{
		0& & -1& & \cdots& & & & 0 \\
		1& & 0& & -1& & \cdots& & \vdots \\
		\vdots& & 1& & 0& & \ddots& & \\
		& & \vdots& & \ddots& & \ddots& & -1 \\
		0& & \cdots& & & & 1& & 0
		}
	\]
\end{lemma}

\begin{definition}
	Линейное действительное пространство называется \textit{евклидовым},
	если определена функция $\trbr{\cdot, \cdot}: X \times X \ra
	\R$, обладающая свойствами:
	\begin{itemize}
		\item $(\forall \vec{x} \in X)\ \ \trbr{\vec{x},
			\vec{x}} \ge 0$, причём $\trbr{\vec{x}, \vec{x}} =
			0 \lra \vec{x} = \vec{0}$
		
		\item $(\forall \vec{x}, \vec{y} \in X)\ \ 
			\trbr{\vec{x}, \vec{y}} = \trbr{\vec{y}, \vec{x}}$
		
		\item $(\forall \alpha, \beta \in \R)\ \ 
			\trbr{\alpha\vec{x} + \beta\vec{y}, \vec{z}} =
			\alpha\trbr{\vec{x}, \vec{z}} + \beta\trbr{\vec{y}, \vec{z}}$
	\end{itemize}
	Эта функция $\trbr{\vec{x}, \vec{y}}$ называется скалярным
	произведением
\end{definition}

\begin{lemma}
	$\R^n$ является вещественным евклидовым
	пространством если определить скалярное произведение как:
	\[
		\trbr{\vec{x}, \vec{y}} = \suml_{i = 0}^n x_iy_i
	\]
	где $\vec{x} = (x_1, \ldots, x_n);\ \vec{y} = (y_1, \ldots, y_n)$
\end{lemma}

\begin{addition}
	Если в определении вещественного евклидового пространства
	заменить первое свойство на
	\[
	(\forall \vec{y} \in X\ \trbr{\vec{x}, \vec{y}} = 0)
		\lra \vec{x} = \vec{0}
	\]
	то получим определение \textit{псевдоевклидового} пространства.
\end{addition}

\begin{example}
	$R^4$ - псевдоевклидово пространство, где для любых векторов
	$\vec{x} = (x_0, x_1, x_2, x_3)$ и $\vec{y} = (y_0, y_1, y_2, y_3)$
	скалярное произведение имеет вид
	\[
		\trbr{\vec{x}, \vec{y}} = x_0 y_0 - x_1 y_1 - x_2 y_2 - x_3 y_3
	\]
	Это пространство носит имя \textit{пространства Минковского} и
	играет большую роль в Специальной Теории Относительности.
\end{example}

\begin{idea}
	Доказательство проводится рутинной проверкой каждого
	условия из определения евклидова пространства.
\end{idea}

\begin{theorem} (Неравенство Коши-Буняковского-Шварца) \label{Cauchy–Schwarz}
	Если $X$ - вещественное евклидовое пространство, то
	\[
		(\forall \vec{x}, \vec{y} \in X)\ \ 
		|\trbr{\vec{x}, \vec{y}}|^2 \le
		\trbr{\vec{x}, \vec{x}} \cdot
		\trbr{\vec{y}, \vec{y}}
	\]
	причём равенство имеет место ТиТТК $\ x = 0$ или $y = 0$, или
	$(\exists \lambda \in \R)\ \vec{x} =
	\lambda \vec{y}\ $ (по сути, когда $\vec{x}\ ||\ \vec{y}$).
\end{theorem}

\begin{proof}
	Если $\vec{y} = \vec{0}$, то
	\[
		\trbr{\vec{x}, \vec{0}} = \trbr{\vec{x},
		\vec{x} + (-\vec{x})} = \trbr{\vec{x}, \vec{x}} - 
		\trbr{\vec{x}, \vec{x}} = 0 = \trbr{\vec{x}, \vec{x}}
		\cdot \trbr{\vec{0}, \vec{0}}
	\]
	Теперь пусть $\vec{y} \neq \vec{0}$.
	Рассмотрим скалярное произведение
	$\trbr{\vec{x} + \lambda \vec{y},
	\vec{x} + \lambda \vec{y}},\ \lambda \in \R$:
	\begin{multline*}
		\trbr{\vec{x} + \lambda \vec{y},
		\vec{x} + \lambda \vec{y}} =
		\trbr{\vec{x}, \vec{x} + \lambda\vec{y}} +
		\lambda\trbr{\vec{y}, \vec{x} + \lambda\vec{y}} =
		\trbr{\vec{x} + \lambda\vec{y}, \vec{x}} +
		\lambda\trbr{\vec{x} + \lambda\vec{y}, \vec{y}} =
		\\
		\trbr{\vec{x}, \vec{x}} + \lambda\trbr{\vec{y},
		\vec{x}} + \lambda^2\trbr{\vec{y}, \vec{y}} +
		\lambda\trbr{\vec{x}, \vec{y}} =
		\trbr{\vec{x}, \vec{x}} + 2\lambda
		\trbr{\vec{x}, \vec{y}} +
		\lambda^2\trbr{\vec{y}, \vec{y}} \ge 0
	\end{multline*}
	Раз у квадратного трёхчлена относительно $\lambda$
	коэффициент при старшей степени положителен и весь
	он неотрицателен, то дискриминант должен быть
	неположителен (чтобы было не более одного корня вследствие
	геометрического положения параболы):
	\[
		\frac{D}{4} = \trbr{\vec{x}, \vec{y}}^2 -
		\trbr{\vec{x}, \vec{x}} \cdot \trbr{\vec{y}, \vec{y}}
		\le 0
	\]
	При этом если $D = 0$ и выражение выше обращается в
	равенство, то 
	\[
		(\exists \lambda \in \R)\ 
		\trbr{\vec{x} + \lambda \vec{y},
		\vec{x} + \lambda \vec{y}} = 0
	\]
	Тогда
	$\vec{x} + \lambda\vec{y} = \vec{0} \lra \vec{x} =
	(-\lambda)\vec{y}$
\end{proof}

\begin{corollary}
	\[
		(\forall \vec{x}, \vec{y} \in \R^n)\ \ \ 
		\Big|\suml_{i = 1}^n x_i y_i\Big| \le
		\sqrt{\suml_{i = 1}^n x_i^2} \cdot
		\sqrt{\suml_{i = 1}^n y_i^2}
	\]
\end{corollary}

%-------------------------- Не было ----------------------

\begin{definition}
	Комплексным евклидовым (унитарным) пространством называется комплексное линейное пространство $X$, для любых двух элементов которого $\vec{x}, \vec{y} \in X$ определено число $\trbr{\vec{x}, \vec{y}} \in \Cm$ так, что
	\begin{itemize}
		\item $\forall \vec{x}, \vec{y} \in X\ \ \trbr{\vec{x}, \vec{x}} \ge 0$, причём $\trbr{\vec{x}, \vec{x}} = 0 \lra \vec{x} = \vec{0}$
		
		\item $\forall \vec{x}, \vec{y} \in X\ \ \trbr{\vec{x}, \vec{y}} = \overline{\trbr{\vec{y}, \vec{x}}}$
		
		\item $\forall \vec{x}, \vec{y}, \vec{z} \in X\ \forall \alpha, \beta \in \Cm\ \ \trbr{\alpha\vec{x} + \beta\vec{y}, \vec{z}} = \alpha\trbr{\vec{x}, \vec{z}} + \beta\trbr{\vec{y}, \vec{z}}$
	\end{itemize}
	$\trbr{\vec{x}, \vec{y}}$ называется \textit{эрмитовым} скалярным произведением.
\end{definition}

\begin{lemma}
	$\Cm^n$ является унитарным с $\trbr{\vec{z}, \vec{w}} = \suml_{i = 0}^n z_i \bar{w_i}$, для $\vec{z} = (z_1, \ldots, z_n)$ и $\vec{w} = (w_1, \ldots, w_n)$.
\end{lemma}

%-------------------------- Не было ----------------------

\begin{theorem} (Неравенство Коши-Буняковского-Шварца для унитарных пространств)
	Если $X$ - унитарное пространство, то для любых $\vec{z}, \vec{w} \in X$ верно неравенство
	\[
		|\trbr{\vec{z}, \vec{w}}| \le \sqrt{\trbr{\vec{z}, \vec{z}}} \cdot \sqrt{\trbr{\vec{w}, \vec{w}}}
	\]
	причём равенство имеет место тогда и только тогда, когда $\exists \lambda \in \Cm \such \vec{z} = \lambda\vec{w}$ или $\vec{w} = \vec{0}$
\end{theorem}

\begin{proof}
	Обозначим $\trbr{\vec{z}, \vec{w}} = |\trbr{\vec{z}, \vec{w}}|e^{i\varphi}$. Рассмотрим $\trbr{\vec{z} + \lambda e^{i\varphi} \vec{w}, \vec{z} + \lambda e^{i\varphi} \vec{w}},\ \lambda \in \R$:
	\begin{multline*}
		\trbr{\vec{z} + \lambda e^{i\varphi} \vec{w}, \vec{z} + \lambda e^{i\varphi} \vec{w}} = \trbr{\vec{z}, \vec{z} + \lambda e^{i\varphi} \vec{w}} + \lambda e^{i\varphi} \trbr{\vec{w}, \vec{z} + \lambda e^{i\varphi} \vec{w}} =
		\\
		\overline{\trbr{\vec{z} + \lambda e^{i\varphi} \vec{w}, \vec{z}}} + \lambda e^{i\varphi} \overline{\trbr{\vec{z} + \lambda e^{i\varphi} \vec{w}, \vec{w}}} = \overline{\trbr{\vec{z}, \vec{z}}} + \overline{\lambda e^{i \varphi}} \cdot \overline{\trbr{\vec{w}, \vec{z}}} + \lambda e^{i\varphi} \overline{\trbr{\vec{z}, \vec{w}}} + \lambda e^{i\varphi} \cdot \overline{\lambda e^{i\varphi}} \cdot \overline{\trbr{\vec{w}, \vec{w}}} =
		\\
		\trbr{\vec{z}, \vec{z}} + 2\lambda |\trbr{\vec{z}, \vec{w}}| + \lambda^2 \trbr{\vec{w}, \vec{w}} \ge 0
	\end{multline*}
	И снова получили квадратный трёхчлен относительно $\lambda$:
	\[
		\frac{D}{4} = |\trbr{\vec{z}, \vec{w}}|^2 - \trbr{\vec{w}, \vec{w}} \cdot \trbr{\vec{z}, \vec{z}} \le 0
	\]
	Отсюда
	\[
		|\trbr{\vec{z}, \vec{w}}| \le \sqrt{\trbr{\vec{w}, \vec{w}}} \cdot \sqrt{\trbr{\vec{z}, \vec{z}}}
	\]
\end{proof}

\begin{corollary}
	Для любых комплексных чисел
	\[
		\left|\suml_{i = 1}^n z_i w_i\right| \le \sqrt{\suml_{i = 1}^n |z_i|^2} \cdot \sqrt{\suml_{i = 1}^n |w_i|^2}
	\]
\end{corollary}

\subsubsection*{Топология $\R^n$}

\begin{definition}
	Линейное действительное пространство $X$ называется
	\textit{линейным нормированным пространством (ЛНП)},
	если на нём определена функция $\|\cdot\|: X \ra \R$
	(норма), обладающая свойствами:
	\begin{enumerate}
		\item $(\forall \vec{x} \in X)\ \|\vec{x}\| \ge 0$,
			причём $\|\vec{x}\| = 0 \lra \vec{x} = \vec{0}$
		
		\item $(\forall \alpha \in \R(\Cm))(\forall \vec{x}
			\in X)\ \ \|\alpha \vec{x}\| = |\alpha| \cdot \|\vec{x}\|$
		
		\item $(\forall \vec{x}, \vec{y} \in X)\ \ \|\vec{x}
			+ \vec{y}\| \le \|\vec{x}\| + \|\vec{y}\|\ $
			\textit{(неравенство треугольника)}
	\end{enumerate}
\end{definition}

\begin{lemma} 
	Любое евклидово пространство является линейным 
	нормированным пространством (ЛНП) с $\|\vec{x}\| =
	\sqrt{\trbr{\vec{x}, \vec{x}}}$.
\end{lemma}

\begin{proof}~
	\begin{enumerate}
		\item В обоих случаях следует из определения:
		\begin{align*}
			\vec{x} = \vec{0} \lra \trbr{\vec{x}, \vec{x}} = 0 \lra \|\vec{x}\| = 0
			\\
			\vec{x} \neq \vec{0} \lra \trbr{\vec{x}, \vec{x}} > 0 \lra \|\vec{x}\| > 0
		\end{align*}
		
		\item Для действительных:
			\[
				\|\alpha\vec{x}\| = \sqrt{\trbr{\alpha\vec{x},
				\alpha\vec{x}}} = \sqrt{\alpha\trbr{\vec{x},
				\alpha\vec{x}}} = \sqrt{\alpha\trbr{\alpha\vec{x},
				\vec{x}}} = \sqrt{\alpha \cdot
				\alpha\trbr{\vec{x}, \vec{x}}} =
				|\alpha|\sqrt{\trbr{\vec{x}, \vec{x}}} = |\alpha|
				\cdot \|\vec{x}\|
			\]
			Докажем комплексный случай:
			\[
				\|\alpha\vec{x}\| = \sqrt{\trbr{\alpha\vec{x},
				\alpha\vec{x}}} = \sqrt{\alpha\trbr{\vec{x},
				\alpha\vec{x}}} = \sqrt{\alpha\overline{\trbr{\alpha\vec{x},
				\vec{x}}}} = \sqrt{\alpha \cdot
				\overline{\alpha}\overline{\trbr{\vec{x}, \vec{x}}}} =
				|\alpha|\sqrt{\trbr{\vec{x}, \vec{x}}} = |\alpha| \cdot
				\|\vec{x}\|
			\]
		
		\item Также докажем для действительных, используя неравенство
			Коши-Буняковского-Шварца:
			\begin{multline*}
				\|\vec{x} + \vec{y}\|^2 = \trbr{\vec{x} +
				\vec{y}, \vec{x} + \vec{y}} =
				\trbr{\vec{x}, \vec{x} + \vec{y}} + \trbr{\vec{y},
				\vec{x} + \vec{y}} =
				\\
				= \trbr{\vec{x}, \vec{x}} + \trbr{\vec{x}, \vec{y}}
				+ \trbr{\vec{y}, \vec{x}} + \trbr{\vec{y}, \vec{y}} =
				\\
				= \trbr{\vec{x}, \vec{x}} + 2 \trbr{\vec{x}, \vec{y}} +
				\trbr{\vec{y}, \vec{y}}
				\le \trbr{\vec{x}, \vec{x}} + 2 |\trbr{\vec{x}, \vec{y}}| +
				\trbr{\vec{y}, \vec{y}} \le
				\\
				\le \trbr{\vec{x}, \vec{x}} + 2
				\sqrt{\trbr{\vec{x}, \vec{x}}} \sqrt{\trbr{\vec{y}, \vec{y}}}
				+ \trbr{\vec{y}, \vec{y}} =
				\left(\sqrt{\trbr{\vec{x}, \vec{x}}}
				+ \sqrt{\trbr{\vec{y}, \vec{y}}}\right)^2
				= \left(\|\vec{x}\| + \|\vec{y}\|\right)^2
			\end{multline*}
			Пусть Вас здесь не смутит, что мы доказывали для квадрата нормы:
			поскольку норма неотрицательна (доказано только что),
			то все переходы верны и под корнем.
			
			Для комплексных:
			\begin{multline*}
				\|\vec{x} + \vec{y}\|^2 = \trbr{\vec{x} +
				\vec{y}, \vec{x} + \vec{y}} =
				\trbr{\vec{x}, \vec{x} + \vec{y}} + \trbr{\vec{y},
				\vec{x} + \vec{y}} =
				\\
				= \overline{\trbr{\vec{x} + \vec{y},
				\vec{x}}} + \overline{\trbr{\vec{x} + \vec{y}, \vec{y}}}
				= \trbr{\vec{x}, \vec{x}} + \overline{\trbr{\vec{y}, \vec{x}}}
				+ \overline{\trbr{\vec{x}, \vec{y}}} + \trbr{\vec{y}, \vec{y}} =
				\\
				= \trbr{\vec{x}, \vec{x}} + 2\re(\trbr{\vec{x}, \vec{y}}) +
				\trbr{\vec{y}, \vec{y}}
				\le \trbr{\vec{x}, \vec{x}} + 2 |\trbr{\vec{x}, \vec{y}}| +
				\trbr{\vec{y}, \vec{y}} \le
				\\
				\le \trbr{\vec{x}, \vec{x}} + 2
				\sqrt{\trbr{\vec{x}, \vec{x}}} \sqrt{\trbr{\vec{y}, \vec{y}}}
				+ \trbr{\vec{y}, \vec{y}} =
				\left(\sqrt{\trbr{\vec{x}, \vec{x}}}
				+ \sqrt{\trbr{\vec{y}, \vec{y}}}\right)^2
				= \left(\|\vec{x}\| + \|\vec{y}\|\right)^2
			\end{multline*}
	\end{enumerate}
\end{proof}

\begin{corollary} (Неравенство Минковского)
	\[
		\sqrt{\suml_{i = 1}^n (x_i + y_i)^2} \le \sqrt{\suml_{i = 1}^n x_i^2} + \sqrt{\suml_{i = 1}^n y_i^2},\ \ x_i, y_i \in \R 
	\]
\end{corollary}

\begin{lemma}
	$\R^n$ - ЛНП с $\|\vec{x}\| = \sqrt{\suml_{i = 1}^n x_i^2}$. При этом в $\R^2$ и $\R^3$ норма совпадает с длиной вектора.
\end{lemma}

\begin{note}
	Для удобства, в $\R^n$ будем обозначать норму просто как $|\vec{x}|$.
\end{note}

\begin{lemma}
	$\Cm^n$ - ЛНП с $\|\vec{z}\| = \sqrt{\suml_{i = 1}^n |z_i|^2}$
\end{lemma} %
    %13.04.23

\begin{note}[Геометрический смысл дифференцируемости ($n = 2$)]

    Пусть $f: U \to \R$, $U$ -- открыто в $\R^{2}$, $f$ -- дифференцируема в точке $(x_{0}, y_{0})$, то есть 
    \[f(x, y) = f(x_{0}, y_{0}) + \frac{\partial f}{\partial x}(x_{0}, y_{0})(x - x_{0}) + \frac{\partial f}{\partial y}(x_{0}, y_{0})(y - y_{0}) + o(\rho),\]
    где $\rho = \sqrt{(x - x_{0})^{2} + (y - y_{0})^{2}}$.
    
    $G_{f} = \{(x, y, f(x, y)): (x, y) \in U\}$ -- график $f$.
    
    $\pi: z = f(x_{0}, y_{0}) + \frac{\partial f}{\partial x}(x_{0}, y_{0})(x - x_{0}) + \frac{\partial f}{\partial y}(x_{0}, y_{0})(y - y_{0})$.
    
    $\overline{n}(\frac{\partial f}{\partial x}(x_{0}, y_{0}), \frac{\partial f}{\partial y}(x_{0}, y_{0}), -1)$ -- вектор нормали.
    
    $\overline{a} (x - x_{0}, y - y_{0}, f(x, y) - f(x_{0}, y_{0}))$ -- непрерывный вектор в $MM_{0}$
    
    $\cos(\phi) = \frac{(\overline{a}, \overline{n})}{|\overline{a}||\overline{n}|}$
    
\end{note}

\begin{definition}
    Вектор $(\frac{\partial f}{\partial x_{1}}(a), \ldots, \frac{\partial f}{\partial x_{n}}(a))^{T}$ называется \textit{градиентом} функции $f$ в точке $a$ и обозначается $grad f(a)$ или $\nabla f(a)$.
\end{definition}

\begin{corollary}
    Пусть $f$ дифференцируема в точке $a$, и $grad f(a) \neq 0$, то для любого $v \in \R^{n}$ с $|v| = 1$ выполнено
    \[\left|\frac{\partial f}{\partial v}(a)\right| \leq |grad f(a)|,\]
    причем равенство достигается лишь при $v = \pm \frac{grad f(a)}{|grad f(a)|}$.
\end{corollary}

\begin{proof}
    Так как $\frac{\partial f}{\partial v}(a) = df_{a}(v) = (grad f(a), v)$, то по неравенству Коши-Буняковского-Шварца $\left|\frac{\partial f}{\partial v}(a)\right| \leq |grad f(a)| \cdot |v| = |grad f(a)|$, причем равенство достигается лишь в случае коллинеарности $grad f(a)$ и $v$, то есть $v = \pm \frac{grad f(a)}{|grad f(a)|}$.
\end{proof}

\begin{example}
    Пусть $f: \R^{2} \to \R$,
    \[f(x, y) = \begin{cases}
        1, \ y = x^{2}, \ x > 0 \\
        0, \text{ иначе. }
    \end{cases}\]
    Тогда $\frac{\partial f}{\partial v}(0, 0) = 0$ для любого $v \in \R^{2}$, но функция $f$ разрывна в точке $(0, 0)$.

    Тем не менее, в терминах частных производных можно получить довольно простой признак дифференцируемости.
\end{example}

\begin{theorem}[Достаточное условие дифференцируемости]
    Пусть $f: U \subset \R^{n} \to \R$, точка $a \in U$. Если все частные производные $\frac{\partial f}{\partial x_{k}}$ определены в окрестности а и непрерывны в точке $a$, то $f$ дифференцируема в точке $a$.
\end{theorem}

\begin{proof}
    Пусть все $\frac{\partial f}{\partial x_{k}}$ определены в $B_{r}(a) \subset U$. Рассмотрим $h = (h_{1}, \ldots, h_{n})^{T}$ с $|h| < r$, и определим точки $x_{0} = a$, $x_{k} = a + \sum_{j = 1}^{k} h_{j}e_{j}$. Тогда приращение
    \[f(a + h) - f(a) = \sum_{k = 1}^{n}(f(x_{k}) - f(x_{k - 1})) = \sum_{k = 1}^{n}(f(x_{k - 1} + h_{k}e_{k}) - f(x_{k - 1})).\]

    Функция $g(t) = f(x_{k - 1} + te_{k}) - f(x_{k - 1})$ на отрезке с концами $0$ и $h_{k}$ (при $h_{k} \neq 0$) имеет производную $g'(t) = \frac{\partial f}{\partial x_{k}}(x_{k - 1} + t_{e_{k}})$. По теореме Лагранжа о среднем $g(h_{k}) - g(0) = g'(\xi_{k})h_{k}$ для некоторого $\xi_{k}$ между $0$ и $h_{k}$. Положим $c_{k}(h) = x_{k - 1} + \xi_{k}e_{k}$, тогда последнее равенство перепишется в виде $f(x_{k}) - f(x_{k - 1}) = \frac{\partial f}{\partial x_{k}}(c_{k})h_{k}$, причем $c_{k} \to a$ при $h \to 0$. Поэтому 
    \[f(a + h) - f(a) - \sum_{k = 1}^{n} \frac{\partial f}{\partial x_{k}}(a)h_{k} = \sum_{k = 1}^{n}\left(\frac{\partial f}{\partial x_{k}}(c_{k}) - \frac{\partial f}{\partial x_{k}}(a)\right)h_{k} =\]
    \[=\sum_{k = 1}^{n} \left(\frac{\partial f}{\partial x_{k}}(c_{k}) - \frac{\partial f}{\partial x_{k}}(a)\right)\frac{h_{k}}{|h|}|h| =: \alpha(h)|h|.\]

    В силу непрерывности $\frac{\partial f}{\partial x_{k}}$ в точке $a$ и неравенства $|h_{k}| \leq |h|$ функция $\alpha(h) \to 0$ при $h \to 0$. Следовательно, $f$ дифференцируема в точке $a$.
\end{proof}

\textit{Случай функций из $\R^{n}$ в $\R^{m}$.}

Пусть $U \subset \R^{n}$ открыто, и функция $f: U \to \R^{m}$, $f = (f_{1}, \ldots, f_{m})^{T}$.

\begin{lemma}
    \label{dif-lem1}
    Функция $f$ дифференцируема в точке $a$ тогда и только тогда, когда все координатные функции $f_{i}$ дифференцируемы в точке $a$.
\end{lemma}

\begin{proof}
    Пусть $f$ дифференцируема в точке $a$. Распишем формулу (1) покоординатно:
    \[f_{i}(a + h) = f_{i} + L_{i}(h) + \alpha_{i}(h)|h|.\]
    Координатные функции $L_{i}$ дифференциала $L_{a}$ линейны, а условие "$\alpha(h) \to 0$ при $h \to \overline{0}$"\ равносильно "$\alpha_{i}(h) \to 0$ при $h \to 0$"\ , где $i = 1, \ldots, m$, поэтому функция $f_{i}$ дифференцируема в точке $a$ и ее дифференциал $d(f_{i})_{a} = L_{i}$.

    Обратно, если все функции $f_{i}$ дифференцируемы, то верна и формула (1) с $L_{a} = (L_{1}, \ldots, L_{m})^{T}$ и $\alpha = (\alpha_{1}, \ldots, \alpha_{m})^{T}$.
\end{proof}

Поскольку действие линейного отображения из $\R^{n}$ в $\R^{m}$ на вектор есть умножение этого вектора слева на матрицу, поэтому найдется такая матрица $Df_{a}$ размера $m \times n$, что $df_{a}(h) = D f_{a} \cdot h$ для всех $h \in \R^{n}$.

\begin{definition}
    Матрица $Df_{a}$ называется \textit{матрицей Якоби} функции $f$ в точке $a$.
\end{definition}

\begin{note}
    По лемме 1 следует, что $df(h) = (df_{1}(h), \ldots, df_{m}(h))^{T}$, поэтому $ij$-й элемент матрицы Якоби в точке $a$ равен значению $d(f_{i})_{a}(e_{j})$, то есть $\frac{\partial f_{i}}{\partial x_{j}}(a)$. Таким образом, строками матрицы Якоби являются градиенты ее координатных функций в этой точке.
\end{note}
    \subsection{Топология пространства $\R^n$ и непрерывные отображения}

\subsubsection*{Пределы в частных случаях}

\begin{proposition}
	Если $X$ - метрическое пространство, то
	\[
		\liml_{n \to \infty} x_n = x_0 \lra \forall \eps > 0\ \exists N \in \N \such \forall n > N\ \rho(x_n, x_0) < \eps
	\]
\end{proposition}

\begin{proposition}
	Если $X$ - линейное нормированное пространство, то
	\[
		\liml_{n \to \infty} x_n = x_0 \lra \forall \eps > 0\ \exists N \in \N \such \forall n > N\ \|x_n - x_0\| < \eps
	\]
\end{proposition}

\begin{theorem} (Основные свойства предела последовательностей в $\R^n$)
	\begin{enumerate}
		\item (Единственность предела) В метрическом пространстве последовательность $\{x_n\}_{n = 1}^\infty$ не может иметь более одного предела
		
		\item (Ограниченность сходящейся последовательности) Если последовательность $\{x_n\}_{n = 1}^\infty \subset \trbr{X, \rho}$ - сходящаяся к $x_0 \in X$, то она ограничена. То есть
		\[
			\exists R > 0 \such \forall n \in \N\ \rho(x_n, x_0) < R
		\] 
		
		\item (Отделимость от нуля) Если последовательность $\{x_n\}_{n = 1}^\infty \subset E$ (ЛНП) сходится к $x_0 \neq 0 \in E$, то она отделена от нуля. То есть
		\[
			\exists c > 0\ \exists N \in \N \such \forall n > N\ ||x_n|| \ge c
		\]
		
		\item (Предел и арифметические операции) Если последовательности $\{x_n\}_{n = 1}^\infty$, $\{y_n\}_{n = 1}^\infty \subset E$ (ЛНП) - сходящиеся к $x_0, y_0 \in E$ соответственно, $\{\alpha_n\}_{n = 1}^\infty \subset \R(\Cm)$ сходится к $\alpha_0 \in \R(\Cm)$, то
		\begin{enumerate}
			\item $\liml_{n \to \infty} (x_n + y_n) = x_0 + y_0$
			
			\item $\liml_{n \to \infty} (\alpha_n \cdot x_n) = \alpha_0 \cdot x_0$
		\end{enumerate}
	
	\item (Предел и скалярное произведение) Если последовательности $\{x_n\}_{n = 1}^\infty, \{y_n\}_{n = 1}^\infty \subset E$ (евклидово) - сходящиеся к $x_0, y_0$ соответственно, то
	\[
		\liml_{n \to \infty} \trbr{x_n, y_n} = \trbr{x_0, y_0}
	\]
	
	\item (Предел и векторное произведение) Если последовательности $\{\vec{x}_n\}_{n = 1}^\infty, \{\vec{y}_n\}_{n = 1}^\infty \subset \R^n$ - сходящиеся к $\vec{x}_0, \vec{y}_0$ соответственно, то
	\[
		\liml_{n \to \infty} [\vec{x}_n, \vec{y}_n] = [\vec{x}_0, \vec{y}_0]
	\]
	\end{enumerate}
\end{theorem}

\begin{proof}~
	\begin{enumerate}
		\item От противного. Пусть $\liml_{n \to \infty} x_n = x_0,\ \liml_{n \to \infty} x_n = y_0,\ x_0 \neq y_0$. Из условия сразу следует, что $\rho(x_0, y_0) > 0$. Рассмотрим $\eps := \frac{1}{2}\rho(x_0, y_0)$:
		\begin{align*}
			&\exists N_1 \in \N \such \forall n > N_1\ \rho(x_n, x_0) < \eps
			\\
			&\exists N_2 \in \N \such \forall n > N_2\ \rho(x_n, y_0) < \eps
		\end{align*}
		Следовательно, если положить $N := \max(N_1, N_2)$, то
		\[
			\forall n > N\ \rho(x_0, y_0) \le \rho(x_0, x_n) + \rho(x_n, y_0) < 2\eps = \rho(x_0, y_0)
		\]
		Противоречие.
		
		\item Положим $\eps := 1$. По условию
		\[
			\exists N \in \N \such \forall n > N\ \rho(x_n, x_0) < 1
		\]
		Обозначим за $R$ следующую величину:
		\[
			R := \max(\rho(x_1, x_0), \rho(x_2, x_0), \ldots, \rho(x_N, x_0)) + 1
		\]
		Из определения следует, что
		\[
			\forall n \in \N\ \rho(x_n, x_0) < R
		\]
		
		\item По определению
		\[
			\forall \eps > 0\ \exists N \in \N \such \forall n > N\ \|x_n - x_0\| < \eps
		\]
		Положим $\eps := \frac{\|x_0\|}{2}$. По неравенству треугольника имеем
		\[
			\|x_0\| \le \|x_0 - x_n\| + \|x_n - 0\| = \|x_n - x_0\| + \|x_n\| < \frac{\|x_0\|}{2} + \|x_n\|
		\]
		А уже отсюда
		\[
			\|x_n\| \ge \frac{\|x_0\|}{2}
		\]
		
		\item
		\begin{enumerate}
			\item Раз исходные последовательности сходятся, то справедливы утверждения
			\begin{align*}
				&{\forall \eps > 0\ \exists N_1 \in \N \such \forall n > N_1\ \ \|x_n - x_0\| < \frac{\eps}{2}}
				\\
				&{\forall \eps > 0\ \exists N_2 \in \N \such \forall n > N_2\ \ \|y_n - y_0\| < \frac{\eps}{2}}
			\end{align*}
			Ну и как обычно: $N := \max(N_1, N_2)$ и тогда $\forall n > N$ оба неравенства верны одновременно. Отсюда
			\[
				\|(x_n + y_n) - (x_0 + y_0)\| = \|(x_n - x_0) + (y_n - y_0)\| \le \|x_n - x_0\| + \|y_n - y_0\| < \eps
			\]
			
			\item По уже доказанному свойству, сходящаяся последовательность ограничена:
			\[
				\exists C > 0 \such \forall n \in \N\ \ \|x_n\| < C
			\]
			Из условия можем также заключить два утверждения:
			\begin{align*}
				&{\forall \eps > 0\ \exists N_1 \in \N \such \forall n > N_1\ \ |\alpha_0| \cdot \|x_n - x_0\| < \frac{\eps}{2}}
				\\
				&{\forall \eps > 0\ \exists N_2 \in \N \such \forall n > N_2\ \ |\alpha_n - \alpha_0| < \frac{\eps}{2C}}
			\end{align*}
			В итоге имеем $N := \max(N_1, N_2)$ и $\forall n > N$:
			\begin{multline*}
				\|\alpha_n x_n - \alpha_0 x_0\| \le \|\alpha_n x_n - \alpha_0 x_n\| + \|\alpha_0 x_n - \alpha_0 x_0\| =
				\\
				|\alpha_n - \alpha_0| \cdot \|x_n\| + |\alpha_0| \cdot \|x_n - x_0\| < \frac{\eps}{2C} \cdot C + \frac{\eps}{2} = \eps
			\end{multline*}
		\end{enumerate}
	
		\item Аналогично предыдущему пункту
		\begin{align*}
			&{\exists C > 0 \such \forall n \in \N\ \ \|x_n\| < C}
			\\
			&{\forall \eps > 0\ \exists N_1 \in \N \such \forall n > N_1\ \ \|y_0\| \cdot \|x_n - x_0\| < \frac{\eps}{2}}
			\\
			&{\forall \eps > 0\ \exists N_2 \in \N \such \forall n > N_2\ \ \|y_n - y_0\| < \frac{\eps}{2C}}
		\end{align*}
		Теперь $N := \max(N_1, N_2)$ и тогда $\forall n > N$:
		\begin{multline*}
			|\trbr{x_n, y_n} - \trbr{x_0, y_0}| \le |\trbr{x_n, y_n} - \trbr{x_n, y_0}| + |\trbr{x_n, y_0} - \trbr{x_0, y_0}| =
			\\
			|\trbr{x_n, y_n - y_0}| + |\trbr{x_n - x_0, y_0}| \le \|x_n\| \cdot \|y_n - y_0\| + \|x_n - x_0\| \cdot \|y_0\| <
			\\
			C \cdot \frac{\eps}{2C} + \frac{\eps}{2} = \eps
		\end{multline*}
		
		\item Снова аналогично пункту про скалярное произведение
		\begin{align*}
			&{\exists C > 0 \such \forall n \in \N\ \ |x_n| < C}
			\\
			&{\forall \eps > 0\ \exists N_1 \in \N \such \forall n > N_1\ \ |y_0| \cdot |x_n - x_0| < \frac{\eps}{2}}
			\\
			&{\forall \eps > 0\ \exists N_2 \in \N \such \forall n > N_2\ \ |y_n - y_0| < \frac{\eps}{2C}}
		\end{align*}
		Положим $N := \max(N_1, N_2)$ и рассмотрим $\forall n > N$:
		\[
			|[x_n, y_n] - [x_0, y_0]| \le |[x_n, y_n] - [x_n, y_0]| + |[x_n, y_0] - [x_0, y_0]| \le |x_n| \cdot |y_n - y_0| + |y_0| \cdot |x_n - x_0| < \eps
		\]
		Предпоследний переход получен из тех соображений, что
		\[
			|[a, b]| = |a| \cdot |b| \cdot \sin \angle(a, b) \le |a| \cdot |b|
		\]
	\end{enumerate}
\end{proof}

\begin{lemma} (Критерий сходимости последовательности в $\R^n$)
	Последовательность $\{\vec{x}_m = (x_m^{(1)}, \ldots, x_m^{(n)})\}_{m = 1}^\infty$ сходится к $\vec{x}_0 = (x_0^{(1)}, \ldots, x_0^{(n)})$ тогда и только тогда, когда $\forall j \in \{1, \ldots, n\}$ последовательность $\{x_m^{(j)}\}_{m = 1}^\infty$ сходится к $x_0^{(j)}$.
\end{lemma}

\begin{proof}
	Докажем необходимость. По условию
	\[
		\forall \eps > 0\ \exists N \in \N\ \such \forall m > N\ \ |\vec{x}_m - \vec{x}_0| < \eps
	\]
	При этом
	\[
		|\vec{x}_m - \vec{x}_0| = \sqrt{(x_m^{(1)} - x_0^{(1)})^2 + \ldots + (x_m^{(n)} - x_0^{(n)})^2}
	\]
	Отсюда
	\[
		\forall j \in \{1, \ldots, n\}\ \ |x_m^{(j)} - x_0^{(j)}| \le \sqrt{(x_m^{(1)} - x_0^{(1)})^2 + \ldots + (x_m^{(n)} - x_0^{(n)})^2} = |\vec{x}_m - \vec{x}_0| < \eps
	\]
	
	Теперь докажем достатовность. Из условия
	\[
		\forall j \in \range{n}\ \left(\forall \eps > 0\ \exists N_j \in \N \such \forall n > N_j\ \ |x_m^{(j)} - x_0^{(j)}| < \frac{\eps}{\sqrt{n}}\right)
	\]
	Снова распишем метрику:
	\[
		|\vec{x}_m - \vec{x}_0| = \sqrt{(x_m^{(1)} - x_0^{(1)})^2 + \ldots + (x_m^{(n)} - x_0^{(n)})^2} < \left(\max\limits_{j \in \{1, \ldots, n\}} |x_m^{(j)} - x_0^{(j)}|\right) \cdot \sqrt{n} < \frac{\eps}{\sqrt{n}} \cdot \sqrt{n} = \eps
	\]
\end{proof}

\begin{theorem} (Больцано-Верейштрасса в $\R^n$)
	Из каждой ограниченной последовательности в $\R^n$ можно выделить сходящуюся подпоследовательность.
\end{theorem}

\begin{proof}
	Пусть $\{\vec{x}_m\}_{m = 1}^\infty$ - ограниченная последовательность. Это означает, что
	\[
		\exists C > 0 \such \forall m \in \N\ \ |\vec{x}_m| < C \Ra \forall j \in \range{n}\ |x_m^{(j)}| < C
	\]
	Рассмотрим произвольное $j \in \range{n}$. Тогда последовательность $\{x_m^{(j)}\}_{m = 1}^\infty$ - ограниченная, а значит, по теореме Больцано-Вейерштрасса, существует $\{x_{m_k}^{(j)}\}_{k = 1}^\infty$ - сходящаяся подпоследовательность. Применим доказанную выше лемму, получим:
	\[
		\liml_{k \to \infty} \vec{x}_{m_k, l_k, \ldots, p_k} = \vec{x}_0
	\]
	где $l_k, \ldots, p_k$ - натуральные последовательности для других координат. Что и требовалось доказать.
\end{proof}

\begin{definition}
	\textit{Фундаментальной последовательностью в метрическом пространстве} $\trbr{X, \rho}$ называется такая последовательность $\{x_n\}_{n = 1}^\infty \subset X$, что
	\[
		\forall \eps > 0\ \exists N \in \N \such \forall n > N,\ \forall p \in \N\ \ \rho(x_n, x_{n + p}) < \eps
	\]
\end{definition}

\begin{theorem} (Критерий Коши в $\R^n$)
	Последовательность $\{\vec{x}_m\}_{m = 1}^\infty \subset \R^n$ сходится тогда и только тогда, когда $\{\vec{x}_m\}_{m = 1}^\infty$ фундаментальная
\end{theorem}

\begin{proof}~
\begin{itemize}
	\item Сходимость $\Ra$ Фундаментальность (эта часть верна в \textbf{любом} метрическом пространстве) По условию
	\[
		\forall \eps > 0\ \exists M \in \N \such \forall m > M\ \ \rho(x_m, x_0) < \frac{\eps}{2}
	\]
	Оценим $\rho(x_m, x_{m + p})$ для $\forall p \in \N$ при уже зафиксированных $\eps$ и $M$:
	\[
		\rho(x_m, x_{m + p}) \le \rho(x_m, x_0) + \rho(x_0, x_{m + p}) < \frac{\eps}{2} + \frac{\eps}{2} = \eps
	\]
	
	\item Фундаментальность $\Ra$ Сходимость (эта часть верна \textbf{только для} $R^n$) По определению фундаментальности
	\[
		\forall \eps > 0\ \exists M \in \N \such \forall m > M, p \in \N\ \ |\vec{x}_m - \vec{x}_{m + p}| < \eps
	\]
	Следовательно, для $\forall j \in \range{n}$ верно неравенство $|x_m^{(j)} - x_{m + p}^{(j)}| < \eps$. Воспользовавшись критерием Коши из $\R$ получим, что
	\[
		\forall j \in \range{n}\ \ \exists \liml_{m \to \infty} x_m^{(j)} = x_0^{(j)}
	\]
	Отсюда по критерию сходимости в $\R^n$ уже получаем, что
	\[
		\exists \liml_{m \to \infty} \vec{x}_m = \vec{x}_0
	\]
\end{itemize}
\end{proof}

\begin{definition}
	Метрическое пространство, в котором каждая фундаментальная последовательность сходится, называется \textit{полным метрическим пространством}.
	
	Полное линейное нормированное пространство называется \textbf{банаховым}, в честь Стефана Банаха.
	
	Полное евклидово пространство называется \textit{гильбертовым} (не конечномерное), в честь Гильберта.
\end{definition}

\begin{theorem} (Критерий замкнутости множества)
	Множество $F$ в метрическом пространстве является \textit{замкнутым} тогда и только тогда, когда
	\[
		\left(\forall \{x_n\}_{n = 1}^\infty \subset F,\ \liml_{n \to \infty} x_n = x_0\right) x_0 \in F
	\]
\end{theorem}

\begin{proof}
	Докажем необходимость. Из замкнутости следует, что
	\[
		\cl F \subset F
	\]
	Запишем предел $\{x_n\}$ по определению:
	\[
		\liml_{n \to \infty} x_n = x_0 \lra \forall \eps > 0\ \exists N \in \N \such \forall n > N\ \ \rho(x_n, x_0) < \eps
	\]
	Коль скоро $\forall n \in \N\ x_n \in F$, то
	\[
		\forall \eps > 0\ (U_\eps(x_0) \cap F) \supset \{x_{N + 1}, \ldots\}
	\]
	То есть $x_0$ - точка прикосновения, а значит $x_0 \in F$
	
	Докажем достаточность. Пусть $x_0 \in \cl F$. Тогда по определению
	\[
		\forall \eps > 0\ U_\eps(x_0) \cap F \neq \emptyset
	\]
	Последовательно будем рассматривать $\eps := 1, \frac{1}{2}, \ldots, \frac{1}{n}, \ldots$ и выбирать произвольную точку $x_n \in (U_\eps(x_0) \cap F)$. Получим $\{x_n\}$, удовлетворяющую условию
	\[
		\forall n \in \N\ \ \rho(x_n, x_0) < \frac{1}{n}
	\]
	Следовательно
	\[
		\liml_{n \to \infty} x_n = x_0
	\]
	А отсюда по условию получаем, что $x_0 \in F$
\end{proof}

\subsubsection*{Примеры метрических пространств}

\begin{example}~
\begin{itemize}
	\item $R^n$ с \textbf{манхэттенской метрикой}:
	\[
		\rho(\vec{a}, \vec{b}) = \suml_{j = 1}^n |a_i - b_i|
	\]
	
	\item Связный взвешенный граф с положительными весами
\end{itemize}
\end{example}
    \section{Приложения интеграла Римана в геометрии}

\subsection{Площадь плоских фигур}

\begin{definition}
	\textit{Площадью} называется мера Жордана в $\R^2$.
\end{definition}

\begin{definition}
	Фигуры, измеримые по Жордану в $\R^2$, называются \textit{квадрируемыми}.
\end{definition}

\begin{theorem}
	Пусть $f$ - неотрицательная интегрируемая по Риману функция на $[a; b]$. Тогда её подграфик (криволинейная трапеция) $G_f = \{(x, y) \colon a \le x \le b,\ 0 \le y \le f(x)\}$ квадрируем, причём его площадь равна $\int_a^b f(x)dx$ и наоборот.
\end{theorem}

\begin{proof}
	Заметим, что $L(P, f) \le \downjm(G_f) \le \upjm(G_f) \le U(P, f)$, ибо нижняя и верхняя суммы Дарбу соответствуют площадям элементарных множеств. Отсюда уже видно, что при интегрируемости $f$ будет совпадение площади с интегралом.
	
	Аналогично в обратную сторону: если верхняя и нижняя меры Жордана совпадают, то мы можем приблизить их с любой точностью, а отсюда уже можно взять элементарные множества для нижней и верхней сумм Дарбу.
\end{proof}

\begin{exercise}
	Докажите, что площадь сектора с центральным углом $0 \le \alpha \le 2\pi$ на окружности радиуса $R$ равна
	\[
		S_\alpha = \frac{\alpha R^2}{2}
	\]
\end{exercise}

\begin{theorem} (Площадь фигуры в полярных координатах)
	Если фигура $G$ задана как $G = \{(r, \phi) \colon \psi_1 \le \phi \le \psi_2,\ 0 \le r \le r(\phi)\}$, где $r, \phi$ - полярные координаты, причём $r(\phi) \in R[\psi_1; \psi_2]$ (предполагаем, что $\psi_2 - \psi_1 \le 2\pi$), то $G$ квадрируема, причём
	\[
		\jm(G) = \frac{1}{2} \int_{\psi_1}^{\psi_2} r^2(\phi)d\phi
	\]
\end{theorem}

\textcolor{red}{Сюда бы картиночку с 23 лекции весны 2022, 1:04:30}

\begin{proof}
	Положим $f(\phi) = \frac{1}{2}r^2(\phi)$. Запишем, например, нижнюю сумму Дарбу для этой функции:
	\[
		L(P, f) = \sum_{k = 1}^n m_k \Delta \phi_k
	\]
	Что есть слагаемое из этой суммы? Это площадь сектора с центральным углом $\Delta \phi_k$ на окружности радиуса $m_k$. Стало быть, можно применить аналогичные рассуждения, как и в предыдущей теореме (только элементарные множества немного искривляются).
\end{proof}

\subsection{Длина кривой}

\begin{proposition}
	Если кривая задана параметризацией $\vv{r}(t),\ t_0 \le t \le T$, где $\vv{r}$ - непрерывно дифференцируемая вектор-функция, то её длина равна
	\[
		\int_{t_0}^T |\vv{r'}(t)|dt
	\]
\end{proposition}

\begin{proof}
	Уже известен тот факт, что $s'(t) = |\vv{r'}(t)|$. Тогда
	\[
		\int_{t_0}^T s'(t)dt = \int_{t_0}^T |\vv{r'}(t)|dt = s(T) - s(t_0)
	\]
\end{proof}

\subsection{Объём тела вращения}

\begin{definition}
	\textit{Объём} --- это мера Жордана в $\R^3$.
\end{definition}

\begin{definition}
	Тело, измеримое по Жордану в $\R^3$, называется \textit{кубируемым}.
\end{definition}

\begin{lemma}
	Если $G$ --- квадрируемая фигура, то $\mathcal{G} = \{(x, y, z) \colon (x, y) \in G,\ z \in [a; b]\}$ --- кубируемая фигура, причём
	\[
		\jm(\mathcal{G}) = (b - a)\jm(G)
	\]
\end{lemma}

\begin{proof}
	По условию известно, что
	\[
		\forall \eps > 0\ \exists E_1 \subset G \subset E_2 \text{ -- элементарные} \colon |E_2 \bs E_1| < \frac{\eps}{b - a}
	\]
	Тогда $\mathcal{E}_1 = E_1 \times [a; b]$, $\mathcal{E}_2 = E_2 \times [a; b]$ --- тоже элементарные множества, причём $\mathcal{E}_1 \subset \mathcal{G} \subset \mathcal{E}_2$ и $|\mathcal{E}_2 \bs \mathcal{E}_1| = |E_2 \bs E_1| \cdot (b - a) < \eps$. Отсюда тривиально следует измеримость $\mathcal{G}$ и равенство в мере.
\end{proof}

\begin{theorem} (Объём тела вращения)
	Если $f$ - неотрицательная интегрируемая по Риману функция на $[a; b]$, то тело, полученное вращением её подграфика $G_f$ вокруг оси $ox$, кубируемо, причём его объём вычисляется следующим образом:
	\[
		\jm(\mathcal{G}_f) = \pi \int_a^b f^2(x)dx
	\]
\end{theorem}

\begin{proof}
	Обозначим $F(x) = \pi f^2(x)$ и заметим, что слагаемое в сумме Дарбу для $f$ - это объём цилиндра, который измерим по уже доказанной лемме. К этому слагаемому можно приблизиться сколь угодно близко при помощи элементарных множеств сверху/снизу, стандартная тактика.
\end{proof}
    %26.04.23

\begin{corollary}
    Функция $f$ дифференцируема $k$ раз в точке $a$, тогда и только тогда, когда все частные производные до порядка $k - 2$ дифференцируемы в некоторой окрестности точки $a$, а все частные производные порядка $k - 1$ дифференцируемы в точке $a$.
\end{corollary}

\begin{theorem}[формула Тейлора с остаточным членом в форме Лагранжа]
    \label{taylor-lagrange}
    Пусть $f: \underbrace{U}_{\text{откр.}} \to \R$ дифференцируема $(p + 1)$ раз на $U$. Если $a \in U$, $h \in \R^{n}$, такие что $[a, a+h] \subset U$, то $\exists \Theta \in (0, 1)$, что

    \[f(a + h) = f(a) + \sum_{k = 1}^{p}\frac{1}{k!}d^{k}f_{a}(h) + \frac{1}{(p+1)!}d^{p+1}f_{a + \Theta h}(h).\]
\end{theorem}

\begin{proof}
    $[a, a + h] = \{a + th \ | \ t \in [0, 1]\}$ --- отрезок с концами $a$ и $a + h$.

    Рассмотрим функцию $g(t) = f(a + th)$, определённую на интервале, содержащем $[0, 1]$. Так как $t \mapsto \underbrace{a}_{\text{пост.}} + \underbrace{th}_{\text{линейн.}} \Rightarrow \forall \tau \in \R \ d(a + th)_t(\tau) = \tau h$. Тогда по теореме о дифференцировании композиции
    \[
        dg_t(\tau) = df_{a + th}(\tau h).
    \]
    По индукции
    \[
        d^kg_t(\tau) = d^kf_{a + th}(\tau h) \quad k = 1, \ldots, p + 1.
    \]
    Имеем $d^kg_t(\tau) = g^{(k)}(t)\tau^k \overset{\tau = 1}{\Rightarrow} g^{(k)}(t) = d^k f_{a + th}(h), \quad k = 1, \ldots, p + 1$.

    По формуле Тейлора с остаточным членом в форме Лагранжа
    \[
        g(t) = g(0) + \sum_{k = 1}^p \frac{g^{(k)}(0)}{k!}t^k + \frac{g^{(p + 1)}(\theta_t)}{(p + 1)!}t^{p + 1}.
    \]
    При $t = 1$ и $\theta = \theta_1$ получаем искомую формулу.
\end{proof}

\begin{lemma}
    \label{peano_lem}
    Пусть $\phi: \R^{n}\times \ldots \times \R^{n} \to \R$ -- $k$-линейное симметрическое отображение, и $\Phi: \R^{n} \to \R$, $\Phi(x) = \phi(x, \ldots, x)$. Тогда функция $\Phi$ дифференцируема и $d\Phi_{x}(h) = k\phi(x^{k - 1}, h)$.
\end{lemma}

\begin{proof}
    Имеем $\Phi(x + h) - \Phi(x) = \phi(x + h, \ldots, x + h) - \phi(x, \ldots, x) = k\phi(x, \ldots, x, h) + $ слагаемые $\phi(x^{p}, h^{q})$, где $p + q = k$, $q \geq 2$. 

    Покажем, что найдется такое $C \geq 0$, что $|\phi(x^{p}, h^{q})| \leq C|x|^{p}|h|^{q}$. Если оба $x$, $h$ ненулевые, то $|\phi(x^{p}, h^{q})| = \left|\phi\left((\frac{x}{|x|})^{p}, (\frac{h}{|h|})^{q}\right)\right||x|^{p}|h|^{q} \leq C|x|^{p}|h|^{q}$ для $C = \max_{|x| = 1}|\phi(x^{k})|$. Оценка очевидно выполняется, когда хотя бы один из векторов нулевой.

    Так как $q \geq 2$, то из полученной оценки следует, что $\phi(x^{p}, h^{q}) = o(|h|)$ при $h \to 0$, что доказывает утверждение.
\end{proof}

\begin{theorem}[остаточный член в форме Пеано]~
    Если функция $f: \underbrace{U}_{\text{откр.}} \to \R$ дифференцируема $p$ раз в точке $a$, то 

    \[f(a + x) = f(a) + \sum_{k = 1}^{p}\frac{1}{k!}d^{k}f_{a}(h) + o(|h|^{p}), \ h \to 0.\]
\end{theorem}

\begin{proof}
    Индукция по $p$. При $p = 1$ равенство верно по определению дифференцируемости. Предположим, утверждение верно при $p - 1$.

    Рассмотрим функцию $g(x) = f(a + x) - f(a) - df_{a}(x) - \ldots - \frac{1}{p!}d^{p}f_{a}(x)$. Зафиксируем $v \in \R^{n}$. Тогда по лемме \ref{peano_lem} имеем
    \[d(d^{k}f_{a}(x))(v) = kd^{k}f_{a}(x)(v)\]
    и, значит, 
    \[d g_{x}(v) = df_{a + x}(v) - df_{a}(v) - \ldots - \frac{1}{(p - 1)!}d^{p}f_{a}(x, \ldots, x, v).\]

    Применим предположение индукции к $y \mapsto df_{y}(v)$:
    \[df_{a + x}(v) = df_{a}(v) + d^{2}f_{a}(x, v) + \ldots + \frac{1}{(p - 1)!}d^{p}f_{a}(x, \ldots, x, v) + o(|x|^{p - 1}).\]
    Заключаем, что $|dg_{x}(v)| = o(|x|^{p - 1})$ при $x \to 0$.

    Зафиксируем $\epsilon > 0$. Найдем такое $\delta > 0$, что $\|dg_{h}\| \leq \epsilon|h|^{p - 1}$ при всех $h \in \R$ с $|h| < \delta$. В шаре $B_{\delta}(0)$ применим теорему \ref{taylor-lagrange} (для  $p = 1$), получим
    \[|g(h)| = |g(h) - g(o)| \leq \epsilon|h|^{p - 1}|h|,\]
    то есть $g(h) = o(|h|^{p})$, $h \to 0$.
\end{proof}

\begin{definition}
    Будем говорить, что $f$ $k$ раз непрерывно дифференцируема на $U$ и писать $f \in C^{k}(U)$, если $d^{k - 1}f_{x} \in C^{1}(U)$.
\end{definition}

\begin{note}
    Пусть $\phi: \R^{n} \times \ldots \times \R^{n} \to \R$ -- $k$-линейное отображение. Тогда $\|\phi\|=\underset{|v_{1}| = 1, \ldots, |v_{k}| = 1}{\max}|\phi(v_{1}, \ldots, v_{k})|$ -- норма на пространстве $k$-линейных отображений. Тогда из леммы \ref{dif-lem1} $f \in C^{k}(U) \lra $ все частные производные до $k$-го порядка непрерывны на $U$.
\end{note}

\section{Мера Лебега}

\subsection{Объем бруса}

\begin{definition}
    \textit{Брусом} в $\R^{n}$ называется множество вида $B = I_{1} \times \ldots \times I_{n}$, где $I_{k}$ -- ограниченный промежуток. Если $a_{k} \leq b_{k}$ -- концы $I_{k}$, то $|B| = (b_{1} - a_{1})\cdot \ldots \cdot(b_{n} - a_{n})$ называется \textit{объемом} бруса $B$.

    Если хотя бы один из промежутков $I_k$ вырожденный, то брус $B$ называется \emph{вырожденным}, в частности, $\emptyset$ --- вырожденный брус. Объём вырожденного бруса равен 0.

    Если все $I_{k}$ -- отрезки, то брус называется \textit{замкнутым}.
    
    Если все $I_{k}$ -- интервалы, то брус называется \textit{открытым}.

\end{definition}

\begin{problem}
    Докажите, что пересечение двух брусов --- брус, а разность двух брусов --- объединение не более чем $2n$ брусов.
\end{problem}

\begin{property}
    \label{brus-prop1}
    Если $B, B_1, \ldots, B_m$ --- брусы и $B \subset \bigcup_{i = 1}^m B_i$, то $|B| \le \sum_{i = 1}^m |B_i|$.

    \begin{proof}
        Если $I \subset \R$ --- ограниченный промежуток, то
        \begin{gather*}
            |I| - 1 \le \#(I \cap \Z) \le |I| + 1,\\
            N|I| - 1 \le \#(NI \cap \Z) \le N|I| + 1,\\
            |I| - \frac{1}{N} \le \frac{1}{N}\#\left(I \cap \frac{1}{N}\Z\right) \le |I| + \frac{1}{N},\\
            |I| = \lim_{N \rightarrow \infty} \frac{1}{N}\#\left(I \cap \frac{1}{N}\Z\right).
        \end{gather*}

        Пусть $B = I_1 \times \ldots \times I_n$, тогда
        \[
            |B| = \lim_{N \rightarrow \infty} \bigsqcap_{j = 1}^n \frac{1}{N} \#\left(I_j \cap \frac{1}{N}\Z\right) = \lim_{N \rightarrow \infty} \frac{1}{N^n} \#\left(B \cap \frac{1}{N}\Z^n\right).
        \]

        Если $B \subset \bigcup_{i = 1}^m B_i$, то
        \[
            \frac{1}{N^n} \#\left(B \cap \frac{1}{N}\Z^n\right) \le \frac{1}{N^n} \sum_{i = 1}^n \#\left(B_{i} \cap \frac{1}{N}\Z^n\right).
        \]

        Предельный переход $N \rightarrow \infty$ завершает доказательство.
    \end{proof}
\end{property}

\begin{property}
    \label{brus-prop2}
    Для любого бруса $B$ и $\epsilon > 0$ найдутся замкнутый брус $B'$ и открытый брус $B^o$, так что $B' \subset B \subset B^o$ и $|B'| > |B| - \epsilon$, $|B^o| < |B| + \epsilon$.

    \begin{proof}
        Пусть $B = I_1 \times \ldots \times I_n$, где $I_k$ --- ограниченный промежуток с концами $a_k \le b_k$.

        Если $|B| > 0$, то положим
        \begin{gather*}
            B'_\delta = [a_1 + \delta, b_1 - \delta] \times \ldots \times [a_n + \delta, b_n - \delta]\\
            B_\delta^o = (a_1 - \delta, b_1 + \delta) \times \ldots \times (a_n - \delta, b_n + \delta)
        \end{gather*}

        Так как $|B_{\delta}'|$, $|B_{\delta}^{o}| \to |B|$ при $\delta \to +0$, то искомые брусы существуют и определяются выбором $\delta$. Если же $B$ -- вырожденный брус, то положим $B' = \emptyset$, $B_{\delta}^{o}$ как выше.
    \end{proof}
\end{property}
    \begin{note}
	До сего момента мы говорили лишь об уравнениях вида $F(\vv{x}, y) = 0$. Но что делать, если мы захотим разобраться в такой системе:
	\[
		\System{
			&{F_1(\vv{x}, \vv{y}) = 0}
			\\
			&{\vdots}
			\\
			&{F_m(\vv{x}, \vv{y}) = 0}
		};
		\ \ \vv{x} \in \R^n,\ \vv{y} \in \R^m
	\]
\end{note}

\begin{definition}
	\textit{Якобианом отображения} $\vv{F} \colon D \to \R,\ D \subset \R^{n + m}$ по переменным $\vv{y} \in \R^m$ в точке $(\vv{x}_0, \vv{y}_0) \in D$ называется следующий определитель:
	\[
		\frac{D(F_1, \ldots, F_m)}{D(y_1, \ldots, y_m)} = \pd{(F_1, \ldots, F_m)}{(y_1, \ldots, y_m)} = \Det{
			&{\pd{F_1}{y_1}} & &\cdots &{\pd{F_1}{y_m}}
			\\
			&{\vdots} & &\ddots &\vdots
			\\
			&{\pd{F_m}{y_1}} & &\cdots &{\pd{F_m}{y_m}} 
		}
	\]
	где все производные существуют и взяты в точке $(\vv{x}_0, \vv{y}_0)$.
\end{definition}

\begin{note}
	Так как определитель можно записать в виде многочлена, то якобианом также можно называть этот многочлен с частными производными.
\end{note}

\begin{theorem} (О неявных функциях, заданных системой уравнений)
	Если выполнены следующие требования:
	\begin{enumerate}
		\item $\forall i \in \range{m}\ F_i \colon D \to \R,\ D \subset \R^{n + m}$ - дифференцируемые функции в некоторой окрестности точки $(\vv{x}_0, \vv{y}_0) \in \R^{n + m}$
		
		\item $\forall i, j \in \range{m}\ \pd{F_i}{y_j}$ - непрерывные частные производные в этой же окрестности
		
		\item $\forall i \in \range{m}\ F_i(\vv{x}_0, \vv{y}_0) = 0$
		
		\item $\pd{(F_1, \ldots, F_m)}{(y_1, \ldots, y_m)}(\vv{x}_0, \vv{y}_0) \neq 0$
	\end{enumerate}
	Тогда, для достаточно малых $\eps_1, \ldots, \eps_m > 0$ существует $\delta > 0$ такая, что в кубе $K_{\delta, \vv{x}_0}$ единственным образом найдутся дифференцируемые в $\vv{x}_0$ функции $y_i = \phi_i(\vv{x})$, удовлетворяющие следующему условию:
	\begin{multline*}
		\forall (\vv{x}, \vv{y}) \in K_{\delta, \vv{x}_0} \times \prod_{j = 1}^m (y_{j, 0} - \eps_j; y_{j, 0} + \eps_j)
		\\
		\big(\forall t \in \range{m}\ F_t(\vv{x}, \vv{y}) = 0\big) \lra \big(\forall t \in \range{m}\ y_t = \phi_t(\vv{x})\big)
	\end{multline*}
\end{theorem}

\begin{proof}
	Проведём индукцию по $m$:
	\begin{itemize}
		\item База $m = 1$: просто предыдущая теорема
		
		\item Переход $m > 1$: без ограничения общности, потребуем ненулевой главный минор порядка $m - 1$ ненулевым. Действительно, если что, то мы можем перенумеровать функции/аргументы в нужную сторону. Чтобы применить индукцию, введём функции $F'_t \colon \R^{(n + 1) + (m - 1)} \to \R$:
		\[
			\forall t \in \range{m - 1}\ \ F'_t(\vv{x}, y_m, y_1, \ldots, y_{m - 1}) = F_t(\vv{x}, y_1, \ldots, y_m)
		\]
		Ну а теперь запишем предположение индукции (оно работает, потому что мы просто переставили аргументы):
		\begin{multline*}
			\forall_{\text{д.м.}} \eps_1, \ldots, \eps_{m - 1} > 0\ \exists \delta_1 > 0 \such \forall t \in \range{m - 1}
			\\
			\exists! \psi_t(\vv{x}, y_m) \such \forall (\vv{x}, y_m, y_1, \ldots, y_{m - 1}) \in K_{\delta_1, (\vv{x}_0, y_{m, 0})} \times \prod_{j = 1}^{m - 1} (y_{j, 0} - \eps_j; y_{j, 0} + \eps_j)
			\\
			\big(\forall t \in \range{m - 1}\ \underbrace{F'_t(\vv{x}, y_m, y_1, \ldots, y_{m - 1})}_{F_t(\vv{x}, \vv{y})} = 0\big) \lra \big(\forall t \in \range{m - 1}\ y_t = \psi_t(\vv{x}, y_m)\big)
		\end{multline*}
		Рассмотрим функцию $\Phi(\vv{x}, y_m) = F_m(\vv{x}, \psi_1(\vv{x}, y_1), \ldots, \psi_{m - 1}(\vv{x}, y_{m - 1}), y_m)$. Покажем, что она удовлетворяет требованиям теоремы о неявной функции, заданной одним уравнением: уже из индукции и просто свойств дифференцирования мы имеем дифференцируемость $\Phi$, а также и непрерывность частной производной. Нужно проверить, что она не зануляется. Для этого, мы подставим $\psi_t$ в тождества с $F_t(\vv{x}, \vv{y})$ и продифференцируем их по $y_m$:
		\[
			\pd{F_t}{y_1} \cdot \pd{\psi_1}{y_m} + \ldots + \pd{F_t}{y_{m - 1}} \cdot \pd{\psi_{m - 1}}{y_m} + \pd{F_t}{y_m} = 0
		\]
		Обозначим $\Delta = \pd{(F_1, \ldots, F_m)}{y_1, \ldots, y_m}$, а $\forall i \in \range{m}\ \Delta_i$ --- это алгебраическое дополнение элемента из $m$-го столбца и $i$-й строки. Теперь, сложим все неравенства выше по $t \in \range{m - 1}$, предварительно домножив каждое на $\Delta_t$, и добавим $\Delta_m\pd{\Phi}{y_m}$ с каждой стороны. Это даст нам следующее:
		\begin{multline*}
			\Delta_m \pd{\Phi}{y_m} = \pd{\psi_1}{y_m}\underbrace{\ps{\Delta_m \pd{F_m}{y_1} + \sum_{j = 1}^{m - 1} \Delta_j \pd{F_j}{y_1}}}_0 + \ldots + \pd{\psi_{m - 1}}{y_m} \underbrace{\ps{\Delta_m \pd{F_m}{y_{m - 1}} + \sum_{j = 1}^{m - 1} \Delta_j \pd{F_j}{y_{m - 1}}}}_0 +
			\\
			\underbrace{\Delta_m \pd{F_m}{y_m} + \sum_{j = 1}^{m - 1} \Delta_j \pd{F_j}{y_m}}_{\Delta}
		\end{multline*}
		Выражения в скобках равны нулю, потому что в форме детерминанта будет 2 одинаковых столбца. Отсюда
		\[
			\pd{\Phi}{y_m} = \frac{\Delta}{\Delta_m} \neq 0
		\]
		Стало быть
		\[
			\forall_{\text{д.м.}} \eps_m > 0\ \exists \delta > 0, \exists! \psi_m(\vv{x}) \such \forall \vv{x} \in K_{\delta, \vv{x}_0}\ (\Phi(\vv{x}, y_m) = 0) \lra (y_m = \psi_m(\vv{x}))
		\]
		Наконец-то можно определить искомые $\phi_j$:
		\[
			\forall j \in \range{m - 1}\ \phi_j(\vv{x}) = \psi_j(\vv{x}, \psi_m(\vv{x}))
		\]
		Так как $F_t(\vv{x}, \vv{y}) = 0$ для $t \in \range{m - 1}$ верно при $y_t = \psi_t(\vv{x}, y_m)$, то $y_m$ можно варьировать. В частноти, вместо него можно подставить $\psi_m(\vv{x})$. Таким образом, последняя функция $\phi_m = \psi_m$ и равносильности верны. Осталось показать, почему есть единственность функций (дифференцируемость получается по теореме о сложной функции). Пойдём от противного:
		\[
			\forall j \in \range{m}\ \exists \hat{\phi}_j \such (\forall t \in \range{m}\ F_t(\vv{x}, \vv{y}) = 0) \lra (\forall t \in \range{m}\ y_t = \hat{\phi}_j(\vv{x}))
		\]
		Но тогда автоматически $y_j = \psi_j(\vv{x}, \hat{\phi}_m(\vv{x}))$. Отсюда и из равенства $\Phi(\vv{x}, y_m) = 0$ мы придём к тому, что $\hat{\phi}_m(\vv{x}) = \phi_m(\vv{x})$. Значит, имеет место единственность.
	\end{itemize}
\end{proof}

\begin{definition}
	Пусть имеются функции $\forall j \in \range{m}\ F_j \colon E \to \R,\ E \subset \R^{n + m}$. Они задают отображение $\vv{F} \colon E \to \R^m$, которое называется \textit{дифференцируемым в $E$}, если для каждого $F_j$ существуют частные производные в каждой точке $E$:
	\[
		\forall t \in \range{m}\ \forall (\vv{x}, \vv{y}) \in \R^{n + m}, i \in \range{n}, j \in \range{m}\ \ps{\exists \pd{F_t}{x_i}(\vv{x}, \vv{y}) \wedge \exists \pd{F_t}{y_j}(\vv{x}, \vv{y})}
	\]
\end{definition}

\begin{definition}
	Если к предыдущему определению добавить непрерывность частных производных, то $\vv{F}$ будет \textit{непрерывно дифференцируема}.
\end{definition}

\begin{theorem} (О локальной обратимости отображения)
	Если $\vv{\phi} \colon D \to \R,\ D \subset \R^n$ --- непрерывно дифференцируемая функция в окрестности точки $\vv{x}_0 \in D$ и $\pd{(\phi_1, \ldots, \phi_n)}{x_1, \ldots, x_n} \neq 0$ в этой же точке, то в некоторой окрестности $\vv{y}_0 = \vv{\phi}(\vv{x}_0)$ существует обратное отображение $\vv{\psi} = \vv{\phi}^{-1}$, дифференцируемое в этой окрестности
\end{theorem}

\begin{proof}
	Положим $F_j$ для предыдущей теоремы следующим образом:
	\[
		\forall j \in \range{n}\ F_j(\vv{x}, \vv{y}) = \phi_j(\vv{x}) - y
	\]
	Тогда автоматически $\pd{(F_1, \ldots, F_n)}{x_1, \ldots, x_n} = \pd{(\phi_1, \ldots, \phi_n)}{x_1, \ldots, x_n} \neq 0$. Значит, применима предыдущая теорема, и в какой-то окрестности
	\begin{multline*}
		\forall i \in \range{n}\ \exists! x_i = \psi_i(y_1, \ldots, y_n) \such (\forall j \in \range{n} F_j(\vv{x}, \vv{y}) = 0) \lra
		\\
		(\forall j \in \range{n}\ x_j = \psi_j(\vv{y}))
	\end{multline*}
	Что и устанавливает обратное отображение.
\end{proof}

\begin{note}
	Для того, чтобы продифференцировать функции, построенные в предыдущих теоремах, достаточно продифференцировать тождества, получаемые при их подстановке в соответствующие равенства.
	
	Если $\vv{y} = \vv{\phi}(\vv{x})$ и $\vv{x} = \vv{\psi}(\vv{y})$, то $\vv{y} = \vv{\phi}(\vv{\psi}(\vv{y}))$. Если пристально посмотреть на частные производные каждой координаты, то заметим следующее равенство:
	\[
		E = \Matrix{\grad \phi_1 \\ \cdots \\ \grad \phi_n}^{\square} \cdot \Matrix{\grad \psi_1 \\ \cdots \\ \grad \psi_n}^{\square}
	\]
	где каждый градиент первой матрицы --- это строчка из всех частных производных соответствующей функции по $x_i$. У второй матрицы --- это частные производные по $y_j$. Ячейка $(i, j)$ в единичной матрице символизирует производную $y_i$ по $y_j$.
\end{note}
    %29.04.23

\begin{note}
    При проверке измеримости достаточно установить, что $\mu^{*}(A) \geq \mu^{*}(A \cap E) + \mu^{*}(A \cap E^{c})$, так как противоположное неравенство следует из счетной аддитивности.
\end{note}

\begin{example}
    Если $\mu^{*}(E) = 0$, то $E$ измеримо. 

    Действительно, $\mu^{*}(A \cap E) \leq \mu^{*}(E) = 0$, $\mu^{*}(A \cap E) \leq \mu^{*}(A)$ из монотонности $\mu^{*}$. Тогда $\mu^{*}(A) \geq \mu^{*}(A \cap E) + \mu^{*}(A \cap E^{c})$.
\end{example}

\begin{example}
    \label{lebeg-ex2}
    Для всякого $a \in \R$ и $k \in \{1, \ldots, n\}$ полупространство $H = H_{a, k} = \{x = (x_{1}, \ldots, x_{n})^{T}: x_{k} < a\}$ измеримо.

    Рассмотрим $A \subset \R^{n}$ и произвольное покрытие $\{B_{i}\}_{i = 1}^{\infty}$. Брусами определим
    \[B_{i}^{1} = B_{i} \cap H, \ B_{i}^{2} = B_{i} \cap H^{c}.\]
    Тогда $B_{i}^{1}, B_{i}^{2}$ -- брусы. $\{B_{i}^{1} \cap H\}_{i = 1}^{\infty}$ -- покрытие $A \cap H$. $\{B_{i}^{2} \cap H^{c}\}_{i = 1}^{\infty}$ -- покрытие $A \cap H^{c}$.

    \[\sum_{i = 1}^{\infty}|B_{i}| = \sum_{i = 1}^{\infty}|B_{i}^{1}| + \sum_{i = 1}^{\infty}|B_{i}^{2}| \geq \mu^{*}(A \cap H) + \mu^{*}(A \cap H^{c}).\]

    Следовательно, $\mu^{*}(A) \geq \mu^{*}(A \cap H) + \mu^{*}(A \cap H^{c})$.

    Аналогичное утверждение верно и для других неравенств между $x_{k}$ и $a$.
\end{example}

\begin{theorem}[Каратеодори]
    Совокупность $\mathcal{M}$ всех измеримых множеств в $\R^{n}$ образует $\sigma$-алгебру. Сужение $\mu^{*}\lvert_{\mathcal{M}}$ счетно аддитивно.
\end{theorem}

\begin{proof}
    $\emptyset \in \mathcal{M}$, $E \in \mathcal{M} \Rightarrow E^{c} \in \mathcal{M}$.

    \begin{enumerate}
        \item Пусть $E, F \in \mathcal{M}$. Покажем, что $E \cup F \in \mathcal{M}$.

        Пусть $A \subset \R^{n}$, тогда
        \[\mu^{*}(A \cap (E \cup F)) + \mu^{*}(A \cap (E \cup F)^{c}) = \mu^{*}(A \cap (E \cup F) \cap E) + \mu^{*}(A \cap (E \cup F) \cap E^{c}) + \mu^{*}(A \cap (E \cup F)^{c}) =\] \[= \mu^{*}(A \cap E) + \mu^{*}(A \cap E^{c} \cap F) + \mu^{*}(A \cap E^{c} \cap F^{c}) = \mu^{*}(A \cap E) + \mu^{*}(A \cap E^{c}) = \mu^{*}(A).\]

        \item Пусть $\{E_{k}\} \subset \mathcal{M}$, причем $E_{i} \cap E_{j} = \emptyset$ при $i \neq j$. Покажем, что $F = \bigcup_{k = 1}^{\infty} E_{k} \in \mathcal{M}$.

        Положим $F_{n} = \bigcup_{k = 1}^{n}E_{k}$. Если $A \subset X$, то
        \[\mu^{*}(A \cap F_{n}) = \mu^{*}(A \cap F_{n} \cap E_{n}) + \mu^{*}(A \cap F_{n} \cap E_{n}^{c}) = \mu^{*}(A \cap E_{n}) + \mu^{*}(A \cap F_{n - 1}).\]
        Продолжая процесс, получим $\mu^{*}(A \cap F_{n}) = \sum_{k = 1}^{n}\mu^{*}(A \cap E_{k})$.

        Поскольку $F_{n} \in \mathcal{M}$, то
        \[\mu^{*}(A) = \mu^{*}(A \cap F_{n}) + \mu^{*}(A \cap F_{n}^{c}) \geq \sum_{k = 1}^{n} \mu^{*}(A \cap E_{k}) + \mu^{*}(A \cap F^{c}).\]
        Переходя к пределу при $n \to \infty$, получим $\mu^{*}(A) \geq \sum_{k = 1}^{\infty}\mu^{*}(A \cap E_{k}) + \mu^{*}(A \cap F^{c})$. Откуда по свойству счетной полуаддитивности
        \[\mu^{*}(A) \geq \sum_{k = 1}^{\infty}\mu^{*}(A \cap E_{k}) + \mu^{*}(A \cap F_{c}) \geq \mu^{*}(A \cap F) + \mu^{*}(A \cap F^{c}) \geq \mu^{*}(A).\]
        Это доказывает, что $F \in \mathcal{M}$. Если еще положить $A = F$, то $\mu^{*}(F) = \sum_{k = 1}^{\infty}\mu^{*}(E_{k})$.

        \item Пусть $\{A_{k}\} \subset \mathcal{M}$. Покажем, что $A = \bigcup_{k = 1}^{\infty}A_{k} \in \mathcal{M}$.

        Положим $E_{1} = A_{1}$, $E_{k} = A_{k} \setminus \bigcup_{i < k} E_{i}$. Тогда $E_{k}$ попарно не пересекаются, и $A = \bigcup_{k = 1}^{\infty}E_{k} \in \mathcal{M}$ по предыдущему пункту.
    \end{enumerate}
\end{proof}

\begin{corollary}
    $\mathcal{B}(\R^{n}) \subset \mathcal{M}$.
\end{corollary}

\begin{proof}
    Брус измерим, так как его можно записать в виде пересечения конечного числа подпространств (измеримы по примеру \ref{lebeg-ex2}). По лемме \ref{lebeg-lem1} тогда всякое открытое множество измеримо.
\end{proof}

\begin{definition}
    $\mu = \mu^{*}\lvert_{\mathcal{M}}$ -- мера Лебега.
\end{definition}

\begin{theorem}[непрерывность меры]
    \begin{enumerate}
        \item $A_{i} \in \mathcal{M}$, $A_{1} \subset A_{2} \subset \ldots$, $A = \bigcup_{i = 1}^{\infty} A_{i}$. Тогда $\mu(A) = \lim_{i \to \infty}\mu(A_{i})$ (непрерывность снизу).
        \item $A_{i} \in \mathcal{M}$, $A_{1} \supset A_{2} \supset \ldots$, $A = \bigcap_{i = 1}^{\infty}A_{i}$, $\mu(A_{1}) < \infty$. Тогда $\mu(A) = \lim_{i \to \infty}\mu(A_{i})$ (непрерывность сверху).
    \end{enumerate}
\end{theorem}

\begin{proof}
    \begin{enumerate}
        \item Положим $B_{1} = A_{1}$, $B_{i} = A_{i} \setminus A_{i - 1}$. Тогда $B_{i} \in \mathcal{M}$, $B_{i} \cap B_{j} = \emptyset$ при $i \neq j$, и $\bigcup_{i = 1}^{m}B_{i} = \bigcup_{i = 1}^{m}A_{i}$ для всех $m \in \N \cup \infty$. Поэтому
        \[\mu(A) = \mu\left(\bigcup_{i = 1}^{\infty}B_{i}\right) = \sum_{i = 1}^{\infty}\mu(B_{i}) = \lim_{m \to \infty}\sum_{i = 1}^{m} \mu(B_{i}) = \lim_{m \to \infty}\mu(A_{m}).\]
        \item Рассмотрим $A_{1} \setminus A_{i}$. Применим прошлый пункт к этим множествам. Тогда $\bigcup_{i = 1}^{\infty}(A \setminus A_{i}) = A_{1} \setminus A$ и 
        \[\mu(A_{1}) - \mu(A) = \mu(A_{1} \setminus A) = \lim_{m \to \infty}\mu(A_{1} \setminus A_{m}) = \mu(A_{1}) - \lim_{m \to \infty}\mu(A_{m}).\]
        Осталось из обоих частей вычесть $\mu(A_{1})$ и изменить знак.
    \end{enumerate}
\end{proof}

\begin{problem}
    Покажите, что $\mu(A_{1}) < \infty$ -- существенно.
\end{problem}

\begin{example}[инвариативность меры относительно сдвигов]
    Пусть $E \in \mathcal{M}$ и $y \in \R^{n}$. Тогда $E + y = \{x + y: x \in E\} \in \mathcal{M}$ и $\mu(E + y) = \mu(E)$.
\end{example}

\begin{proof}
    Пусть $A \subset \R^{n}$ и $\{B_{i}\}_{i = 1}^{\infty}$ -- покрытие $A$ брусами.
    \[A \subset \bigcup_{i = 1}^{\infty} B_{i} \Rightarrow A + y \subset \bigcup_{i = 1}^{\infty}(B_{i} + y).\]
    Ясно, что $B_{i} + y$ -- брус, $|B_{i} + y| = |B_{i}|$. Тогда $\mu^{*}(A + y) \leq \sum_{i = 1}^{\infty}|B_{i}| \Rightarrow \mu^{*}(A + y) \leq \mu^{*}(A)$. Так как $A = (A + y) - y \Rightarrow \mu^{*}(A) \leq \mu^{*}(A + y)$, то есть $\mu^{*}(A) = \mu^{*}(A + y)$.
    
    Пусть $E \in \mathcal{M}$. Тогда
    \[\mu^{*}(A \cap (E + y)) + \mu^{*}(A \cap (E + y)^{c}) = \mu^{*}\left(((A - y) \cap E) + y\right) + \mu^{*}\left(((A - y) \cap E^{c}) + y\right) =\]\[= \mu^{*}\left((A - y) \cap E\right) + \mu^{*}\left((A - y) \cap E^{c}\right) = \mu^{*}(A - y) = \mu^{*}(A),\]
    так что $E + y$ также измеримо.
\end{proof}

\begin{lemma}[регулярность меры]
    Если $E \in \mathcal{M}$, то $\forall \epsilon > 0 \ \exists \underbrace{G}_{\text{откр.}} \supset E \left(\mu(G \setminus E) < \epsilon\right)$.
\end{lemma}

\begin{proof}
    Рассмотрим случай, когда $E$ ограничено, а значит, $\mu^{*}(E) < \infty$. Для $\epsilon > 0$ рассмотрим покрытие $E$ счетным семейством брусов $\{B_{k}\}$ с $\sum_{i = 1}^{\infty}|B_{i}| < \mu(E) + \frac{\epsilon}{2}$. По свойству брусов $\exists \underbrace{B_{i}^{o}}_{\text{откр.}} \supset B_{i}\left(|B_{i}^{o}| < |B_{i}| + \frac{\epsilon}{2^{i + 1}}\right)$. Определим $G = \bigcup_{i = 1}^{\infty} B_{i}^{o}$. Тогда $G$ -- открытое, $G \supset E$ и 
    \[\mu(G \setminus E) = \mu(G) - \mu(E) \leq \sum_{i = 1}^{\infty}|B_{i}^{0}| - \mu(E) < \epsilon.\]

    Перейдем к общему случаю. Поскольку $\R^{n} = \bigcup_{k = 1}^{\infty}A_{k}$, где $A_{k} = \{x \in \R^{n}: k - 1 \leq |x| < k\}$, то $E$ есть счетное объединение непересекающихся играниченных измеримых множеств $E_{k} = E \cap A_{k}$. По доказанному существует такое открытое множество $G_{k} \supset E_{k}$, что $\mu(G_{k} \setminus E_{k}) \leq \frac{\epsilon}{2^{k}}$. Тогда множество $G = \bigcup_{k = 1}^{\infty}G_{k}$ открыто, содержит $E$ и 
    \[\mu(G \setminus E) = \mu\left(\bigcup_{k = 1}^{\infty} G_{k} \setminus E\right) \leq \sum_{k = 1}^{\infty}\mu(G_{k} - E_{k}) < \epsilon.\]
\end{proof}

\begin{corollary}
    Если $E \in \mathcal{M}$, то $\forall \epsilon > 0 \ \exists \underbrace{F}_{\text{замк.}} \subset E \left(\mu(E \setminus F) < \epsilon\right)$.
\end{corollary}
    \begin{theorem}
	Пусть $\Gamma = \{\vec{r}(t), a \le t \le b\}$ - гладкая кривая.
	
	Тогда \(s(t) = V(\vec{r}, [a; t])\) - возрастающая функция, которая дифференцируема на $(a; b)$, имеет конечные производные слева в $b$, справа в $a$ и \(s'(t) = |\vec{r'}(t)|,\ t \in (a; b)\)
\end{theorem}

\begin{proof}
	$\vec{r}(t)$ удовлетворяет условиям признака спрямляемости на $[a; b]$. Применим доказанную оценку для отрезка $[t_0; t_0 + \Delta t],\ \Delta t > 0$:
	\[
		|\vec{r}(t_0 + \Delta t) - \vec{r}(t_0)| \le V(\vec{r}, [t_0; t_0 + \Delta t]) \le \max\limits_{t \in [t_0; t_0 + \Delta t]} |\vec{r'}(t)| \cdot \Delta t
	\]
	Максимум вместо супремума уместен в силу гладкости кривой. При этом среднее число в неравенстве можно записать как
	\[
		V(\vec{r}, [t_0; t_0 + \Delta t]) = V(\vec{r}, [a; t_0 + \Delta t]) - V(\vec{r}, [a; t_0]) = s(t_0 + \Delta t) - s(t_0)
	\]
	Поделим неравенство на $\Delta t$ и получим
	\[
		\left|\frac{\vec{r}(t_0 + \Delta t) - \vec{r}(t_0)}{\Delta t}\right| \le \frac{s(t_0 + \Delta t) - s(t_0)}{\Delta t} \le \max\limits_{t \in [t_0; t_0 + \Delta t]} |\vec{r'}(t)|
	\]
	Остаётся заметить, что левая и правая части стремятся к $|\vec{r'}(t_0)|$ при $\Delta t \to 0+$. Стало быть
	\[
		s'_+(t_0) = \liml_{\Delta t \to 0+} \frac{s(t_0 + \Delta t) - s(t_0)}{\Delta t} = |\vec{r'}(t_0)| > 0
	\]
	Аналогично доказывается для $\Delta t < 0$. Из этого следуют все свойства $s(t)$, описанные в теореме.
\end{proof}

\begin{corollary}
	В силу строго возрастания $s(t)$, определена также обратная к ней функция $s^{-1}(\tau): [0; V(\vec{r})] \to [a; b]$. Она является допустимой заменой параметра для гладкой кривой.
\end{corollary}

\begin{definition}
	Полученная величина $s(t)$ называется \textit{натуральным параметром кривой} $\Gamma$.
	
	При этом кривая $\Gamma = \{\vec{r}(s), 0 \le s \le L\}$ - натуральная параметризация $\Gamma$.
\end{definition}

\begin{lemma}
	Пусть $\Gamma = \{\vec{r}(t), a \le t \le b\}$ - гладкая кривая. Тогда $\vec{r}(t)$ является натуральной параметризацией тогда и только тогда, когда
	\[
		\forall t \in [a; b]\ \ |\vec{r'}(t)| = 1
	\]
\end{lemma}

\begin{proof}~
	\begin{itemize}
		\item $\Ra$ Если $\vec{r}(t)$ - натуральная параметризация, то $t = s$, $s'(t) = 1 = |\vec{r'}(t)|$.
		
		\item $\La$ Теперь выполнено, что $\forall t \in [a; b]\ \ |\vec{r'}(t)| = 1$. Тогда отсюда $\forall t \in [a; b]\ \ s'(t) = 1$. При этом $s(0) = 0$. Значит, для любого $t \in (a; b]$ имеем
		\[
			s(t) - s(0) = s'(\xi) \cdot (t - 0) \Ra s(t) = t
		\]
	\end{itemize}
\end{proof}

\begin{lemma}
	Пусть $\vec{f}(t),\ t \in [a; b]$ - непрерывно дифференцируемая вектор-функция такая, что $|\vec{f}(t)| = const$. Тогда
	\[
		\forall t \in (a; b] \ \ \trbr{\vec{f'}(t), \vec{f}(t)} = 0
	\]
\end{lemma}

\begin{proof}
	Известно равенство:
	\[
		|\vec{f}(t)|^2 = \trbr{\vec{f}(t), \vec{f}(t)} = const
	\]
	Продифференцируем его с обеих сторон. Получится следующее выражение:
	\[
		\trbr{\vec{f'}(t), \vec{f}(t)} + \trbr{\vec{f}(t), \vec{f'}(t)} = 2\trbr{\vec{f'}(t), \vec{f}(t)} = 0
	\]
	Из него уже очевидно следует утверждение теоремы.
\end{proof}

\begin{note}
	Последующая теория построена для пространства $\R^3$.
\end{note}

\begin{theorem} (Второе приближение кривой в $\R^3$)
	Пусть $\Gamma = \{\vec{r}(s), 0 \le s < L\}$ - дважды дифференцируемая гладкая кривая в натуральной параметризации. Тогда $\forall s_0 \in [0; L]$ верно, что
	\[
		\vec{r}(s) = \vec{\rho}(s) + \vec{o}\left((s - s_0)^2)\right),\ s \to s_0
	\]
	где $\vec{\rho}(s)$ - окружность с центром в точке $\vec{r}(s_0) + \frac{1}{k}\vec{\nu}$ и при этом
	\begin{align*}
		&{k = \left|\frac{d^2 \vec{r}}{ds^2}(s_0)\right|}
		\\
		&{\vec{\nu} = \frac{1}{k} \cdot \frac{d^2 \vec{r}}{ds^2}(s_0)}
	\end{align*}
	Радиус этой окружности $R = \frac{1}{k}$, а сама она находится в плоскости $\Pi = \{\vec{r}(s_0) + \alpha \vec{\tau} + \beta \vec{\nu},\ \alpha, \beta \in \R\}$, где $\vec{\tau} = \frac{d\vec{r}}{ds}(s_0)$
\end{theorem}

\begin{definition}
	Определённые выше величины имеют свои названия. Так, в точке $\vec{r}(s_0)$ мы знаем, что
	\begin{itemize}
		\item $\vec{\tau}$ - единичный вектор касательной
		
		\item $\vec{\nu}$ - единичный вектор главной нормали
		
		\item $k$ - кривизна кривой $\Gamma$
		
		\item $R$ - радиус кривизны
		
		\item $\vec{\rho}$ - вектор-функция соприкасающейся окружности
		
		\item $\Pi$ - соприкасающаяся плоскость
	\end{itemize}
\end{definition}

\begin{proof}
	Начнём с небольших фактов:
	\begin{itemize}
		\item $|\vec{\tau}| = 1$, так как по условию кривая в натуральной параметризации.
		
		\item $\trbr{\vec{\nu}, \vec{\tau}} = 0$, по уже доказанной лемме. Значит, они действительно образуют базис для плоскости $\Pi$.
	\end{itemize}
	Воспользуемся формулой Тейлора для $\vec{r}$:
	\[
		\vec{r}(s) = \vec{r}(s_0) + \frac{d\vec{r}}{ds}(s_0)(s - s_0) + \frac{1}{2}\frac{d^2\vec{r}}{ds^2}(s_0)(s - s_0)^2 + \vec{o}\left((s - s_0)^2\right),\ s \to s_0
	\]
	Попытаемся доказать, что расстояние между выбранной точкой и центром окружности "примерно постоянно". Для этого посмотрим на это расстояние:
	\begin{multline*}
		\vec{r}(s) - \left(\vec{r}(s_0) + \frac{1}{k}\vec{\nu}\right) = \frac{d\vec{r}}{ds}(s_0)(s - s_0) + \frac{1}{2} \frac{d^2 \vec{r}}{ds^2}(s_0)(s - s_0)^2 - \frac{1}{k^2}\frac{d^2\vec{r}}{ds^2}(s_0) + \vec{o}\left((s - s_0)^2\right) =
		\\
		(s - s_0)\vec{\tau} + \left(\frac{1}{2}k(s - s_0)^2 - \frac{1}{k}\right)\vec{\nu} + \vec{o}\left((s - s_0)^2\right),\ s \to s_0
	\end{multline*}
	Теперь взглянем на длину этого вектора. Для этого возьмём правый ортонормированный базис, два вектора из которых - это $\vec{\tau}$ и $\vec{\nu}$, а третий - их векторное произведение:
	\begin{multline*}
		\left|\vec{r}(s) - \vec{r}(s_0) - \frac{1}{k}\vec{\nu}\right| =
		\\
		\sqrt{\left(s - s_0 + o_1\left((s - s_0)^2\right)\right)^2 + \left(\frac{1}{2}k(s - s_0)^2 - \frac{1}{k} + o_2\left((s - s_0)^2\right)\right)^2 + \left(o_3\left((s - s_0)^2\right)\right)^2} =
		\\
		\sqrt{\frac{1}{k^2} + o\left((s - s_0)^2\right)},\ s \to s_0
	\end{multline*}
	Как записать вектор-функцию окружности в $\R^3$? Например так:
	\[
		\vec{r}_{\text{окр}}(\phi) = \vec{r}_0 + (\vec{i} \cdot (-R\cos \phi) + \vec{j} \cdot R \sin \phi)
	\]
	Тогда в нашем случае уравнение окружности имеет вид ($\vec{i} := \vec{\nu},\ \vec{j} := \vec{\tau}$):
	\[
		\vec{r}_\text{окр}(\phi) = \vec{r}(s_0) + \frac{1}{k}\vec{\nu} - \vec{\nu} \cdot R\cos \phi + \vec{\tau} \cdot R \sin \phi
	\]
	Для окружности несложно догадаться о натуральном параметре:
	\[
		\phi = \frac{s - s_0}{R}
	\]
	Подставим его в функцию и получим вектор-функцию окружности в натуральной параметризации:
	\begin{multline*}
		\vec{\rho}(s) = \vec{r}(s_0) + \vec{\tau}R \sin \frac{s - s_0}{R} + \vec{\nu}R\left(1 - \cos \frac{s - s_0}{R}\right) =
		\\
		\vec{r}(s_0) + \vec{\tau}R \left(\frac{s - s_0}{R} + o\left((s - s_0)^2\right)\right) + \vec{\nu}R \left(1 - \left(1 - \frac{1}{2}\left(\frac{s - s_0}{R}\right)^2 + o\left((s - s_0)^2\right)\right)\right) =
		\\
		\vec{r}(s_0) + (s - s_0)\vec{\tau} + \frac{1}{2R}(s - s_0)^2\vec{\nu} + \vec{o}\left((s - s_0)^2\right),\ s \to s_0
	\end{multline*}
	Если сравнить данное выражение с формулой Тейлора для $\vec{r}(s)$, то получится нужное утверждение:
	\[
		\vec{r}(s) = \vec{\rho}(s) + o\left((s - s_0)^2\right),\ s \to s_0
	\]
\end{proof}

\begin{theorem} (О вычислении кривизны)
	Если $\Gamma = \{\vec{r}(t), a \le t \le b\}$ - дважды дифференцируемая гладкая кривая, то кривизна в точке $\vec{r}(t)$ может быть подсчитана по формуле:
	\[
		k = \frac{\left|\left[\vec{r'}(t), \vec{r''}(t)\right]\right|}{\left|\vec{r'}(t)\right|^3}
	\]
\end{theorem}

\begin{proof}
	По правилу дифференцирования сложной функции вектор $\tau$ можно расписать так:
	\[
		\vec{\tau} = \frac{d\vec{r}}{ds} = \frac{\vec{r'}(t)}{\frac{ds}{dt}} = \frac{\vec{r'}(t)}{|\vec{r'}(t)|}
	\]
	Что такое кривизна $k$? Она выражается как
	\[
		k = \left|\frac{d\vec{\tau}}{ds}\right| = \left|\left[\frac{d\vec{\tau}}{ds}, \vec{\tau}\right]\right| = \left|\left[\frac{d\vec{\tau}}{ds}, \frac{d\vec{r}}{ds}\right]\right| = \frac{1}{|\vec{r'}(t)|^2} \cdot \left|\left[\vec{\tau'}(t), \vec{r'}(t)\right]\right|
	\]
	Отдельно распишем $\vec{\tau'}(t)$:
	\[
		\vec{\tau'}(t) = \frac{\vec{r''}(t)}{\left|\vec{r'}(t)\right|} - \frac{1}{\left|\vec{r'}(t)\right|^2} \cdot \frac{d}{dt}\left(\left|\vec{r'}(t)\right|\right) \cdot \vec{r'}(t)
	\]
	Если подставить $\vec{\tau'}$ в выражение кривизны, то при раскрытии по линейности второе слагаемое не внесёт вклада. Стало быть
	\[
		k = \frac{1}{\left|\vec{r'}(t)\right|^2} \cdot \left|\left[\frac{\vec{r''}(t)}{\left|\vec{r'}(t)\right|}, \vec{r'}(t)\right]\right| = \frac{\left|\left[\vec{r''}(t), \vec{r'}(t)\right]\right|}{\left|\vec{r'}(t)\right|^3}
	\]
\end{proof}

\begin{corollary}~
	\begin{enumerate}
		\item Если $\vec{r}(t) = (x(t), y(t), z(t))$, то кривизна выражается как
		\[
			k = \frac{\sqrt{(y'z'' - y''z')^2 + (x'z'' - z'x'')^2 + (x'y'' - y'x'')^2}}{\left((x')^2 + (y')^2 + (z')^2\right)^{3/2}}
		\]
		
		\item Если кривая плоская $(z(t) = 0)$, то
		\[
			k = \frac{|x'y'' - y'x''|}{\left((x')^2 + (y')^2\right)^{3/2}}
		\]
		
		\item Для функции $y = f(x)\ (x = t)$ кривизна в точке $x_0$ вычисляется как
		\[
			k(x_0) = \frac{|f''(x_0)|}{\left(1 + (f'(x_0))^2\right)^{3/2}}
		\]
	\end{enumerate}
\end{corollary}

\begin{definition}
	Вектором \textit{бинормали} называется $\vec{\beta}$:
	\[
		\vec{\beta} = [\vec{\tau}, \vec{\nu}]
	\]
\end{definition}

%%% Нарисовать. Здесь нужна картинка с трансляции матана 2го декабря. 1:34:00

\begin{theorem} (Формулы плоскостей сопровождающего трехгранника кривой)
	\begin{itemize}
		\item Уравнение нормальной плоскости имеет вид:
		\[
			(\vec{r} - \vec{r}_0, \vec{r'}(t_0)) = 0
		\]
		
		\item Уравнение соприкасающейся плоскости имеет вид:
		\[
			(\vec{r} - \vec{r}_0, \vec{r'}(t_0), \vec{r''}(t_0)) = 0
		\]
		
		\item Уравнение спрямляющей плоскости имеет вид:
		\[
			\left(\vec{r} - \vec{r}_0, \vec{r'}(t_0), \left[\vec{r'}(t), \vec{r''}(t_0)\right]\right) = 0
		\]
	\end{itemize}
\end{theorem}

\begin{proof}
	Достаточно посмотреть на рисунок.
\end{proof}
    \begin{definition}
	Давайте посмотрим, что из себя представляет $\frac{d\vec{\beta}}{ds}$:
	\[
		\frac{d\vec{\beta}}{ds} = \left[\frac{d\vec{\tau}}{ds}, \vec{\nu}\right] + \left[\vec{\tau}, \frac{d\vec{\nu}}{ds}\right] = \left[\frac{d^2\vec{r}}{ds^2}, \frac{1}{k} \cdot \frac{d^2\vec{r}}{ds^2}\right] + \left[\vec{\tau}, \frac{d\vec{\nu}}{ds}\right]
	\]
	Первое слагаемое обнуляется. Для второго вектора во втором слагаемом мы знаем, что он будет перпендикулярен самому $\vec{\nu} \Ra $ будет находиться в плоскости $\vec{\tau}$ и $\vec{\beta}$. То есть
	\[
		\exists \alpha_1, \alpha_2 \in \R \such \frac{d\vec{\nu}}{ds} = \alpha_1 \vec{\tau} + \alpha_2 \vec{\beta}
	\]
	Отсюда получаем, что
	\[
		\frac{d\vec{\beta}}{ds} = \alpha_2\left[\vec{\tau}, \vec{\beta}\right] = -\ae \vec{\nu}
	\]
	Коэффициент $\ae$ называется \textit{кручением пространственной кривой}. Его можно выразить как
	\begin{multline*}
		\ae = -\trbr{\frac{d\vec{\beta}}{ds}, \vec{\nu}} = -\frac{1}{k} \trbr{\left[\vec{\tau}, \frac{d\vec{\nu}}{ds}\right], \frac{d^2\vec{r}}{ds^2}} = \frac{1}{k^2} \trbr{\frac{d\vec{r}}{ds}, \frac{d^2\vec{r}}{ds^2}, \frac{d^3\vec{r}}{ds^3}} = \frac{1}{k^2} \cdot \frac{1}{|\vec{r'}|^6} \cdot \trbr{\vec{r'}, \vec{r''}, \vec{r'''}} =
		\\
		\frac{(\vec{r'}, \vec{r''}, \vec{r'''})}{|[\vec{r'}, \vec{r''}]|^2}
	\end{multline*}
\end{definition}

\begin{lemma} (Физический смысл кривизны и кручения)
	Если $\Gamma = \{\vec{r}(t)\}$ (т.е. любая) - трижды дифференцируемая кривая, то $k$ и $|\ae|$ - угловые скорости вращения векторов $\vec{\tau}$ и $\vec{\beta}$ соответственно.
\end{lemma}

\begin{proof}
	Пусть $\vec{a}(s)$ - вектор-функция такая, что $\forall s\ \ |\vec{a}(s)| = 1$. Распишем модуль производной этой функции:
	%%% Нарисовать. Внутрь доказательства поместить, наверное. Трансляция Мат.анализ 08.12 время 28:50
	\[
		\left|\frac{d\vec{a}}{ds}\right| = \liml_{\Delta s \to 0} \left|\frac{\vec{a}(s_0 + \Delta s) - \vec{a}(s_0)}{\Delta s}\right|
	\]
	В стремления к нулю $\Delta s$ справедливо приближение:
	\[
		|\Delta \vec{a}(s_0)| = |\vec{a}(s_0 + \Delta s) - \vec{a}(s_0)| = 2\left|\sin \frac{\Delta \phi}{2}\right|
	\]
	То есть предел на самом деле
	\[
		\left|\frac{d\vec{a}}{ds}\right| = \liml_{\Delta s \to 0} \left|\frac{\vec{a}(s_0 + \Delta s) - \vec{a}(s_0)}{\Delta s}\right| = \liml_{\Delta s \to 0} \left|\frac{2\sin \frac{\Delta \phi}{2}}{\Delta s}\right| = \liml_{\Delta s \to 0} \left|\frac{\Delta \phi}{\Delta s}\right|
	\]
	В итоге получаем, что производная по модулю совпадает со скоростью вращения единичного вектора, что и требовалось доказать.
\end{proof}

\begin{theorem} (Формулы Френе)
	Из проведённого анализа кривых можно выделить 3 формулы, носящие имя французского математика Жана Фредерика Френе.
	\begin{enumerate}
		\item \[
			\frac{d\vec{\tau}}{ds} = k\vec{\nu}
		\]
		
		\item \[
			\frac{d\vec{\nu}}{ds} = -k\vec{\tau} + \ae\vec{\beta}
		\]
		
		\item \[
			\frac{d\vec{\beta}}{ds} = -\ae\vec{\nu}
		\]
	\end{enumerate}
	Их все можно записать в матричном виде:
	\[
		\frac{d}{ds}\Matrix{\vec{\tau} \\ \vec{\nu} \\ \vec{\beta}} = \Matrix{0& & k& & 0 \\ -k& & 0& & \ae \\ 0& & -\ae& & 0} \cdot \Matrix{\vec{\tau} \\ \vec{\nu} \\ \vec{\beta}}
	\]
\end{theorem}

\begin{proof}
	Первая и последняя уже доказаны, осталось разобраться со второй:
	\[
		\frac{d\vec{\nu}}{ds} = \frac{d}{ds}[\vec{\beta}, \vec{\tau}] = \left[\frac{d\vec{\beta}}{ds}, \vec{\tau}\right] + \left[\vec{\beta}, \frac{d\vec{\tau}}{ds}\right] = -\ae [\vec{\nu}, \vec{\tau}] + k[\vec{\beta}, \vec{\nu}] = \ae \vec{\beta} - k\vec{\tau}
	\]
\end{proof}

\begin{definition}
	В случае плоской кривой ($\R^2$), у нас всего 2 формулы Френе:
	\begin{enumerate}
		\item \[
			\frac{d\vec{\tau}}{ds} = k\vec{\nu}
		\]
		
		\item \[
			\frac{d\vec{\nu}}{ds} = -k\vec{\tau}
		\]
	\end{enumerate}
	\textit{Эволютой} плоской кривой называется геометрическое место центров кривизны кривой. Исходная кривая называется \textit{эвольвентой} для эволюты.
\end{definition}

\begin{theorem} (Уравнения эволюты плоской кривой)
	Уравнения эволюты в случае $\R^2$ имеет вид:
	\[
		\vec{\rho}(s) = \xi(s)\vec{i} + \eta(s)\vec{j}
	\]
	\begin{align*}
		&{\xi = x - \frac{(x')^2 + (y')^2}{x'y'' - x''y'}y'}
		\\
		&{\eta = y + \frac{(x')^2 + (y')^2}{x'y'' - x''y'}x'}
	\end{align*}
\end{theorem}

\begin{proof}
	Движение вдоль кривой - это с точностью до бесконечно малых второго порядка движение в каждый момент по какой-то окружности. Радиус окружности - это радиус кривизны. Вооружившись полученными формулами, мы можем записать уравнение эволюты:
	\[
	\vec{\rho}(s) = \vec{r}(s) + R(s) \vec{\nu}(s)
	\]
	где $\vec{\rho}(s)$ - это радиус-вектор текущей касательной окружности.
	
	Подставим $\vec{\nu}$ в уже известное уравнение. Что получим?
	\[
		\vec{\rho}(s) = \vec{r}(s) + R^2(s) \cdot \frac{d^2\vec{r}}{ds^2}(s)
	\]
	При этом
	\[
		\frac{d^2\vec{r}}{ds^2} = \frac{\vec{r''}}{|\vec{r'}|^2} - \frac{1}{|\vec{r'}|^2} \frac{d}{ds}(|\vec{r'}|) \cdot \vec{r'} = \frac{\vec{r''}}{(s')^2} - \frac{s''}{(s')^3}\vec{r'} = \frac{(s')^2 \vec{r''} - s's''\vec{r'}}{(s')^4}
	\]
	Если $\vec{i}, \vec{j}$ - это орты плоскости, то тогда имеем
	\begin{align*}
		&{\vec{r} = x\vec{i} + y\vec{j}}
		\\
		&{\vec{r'} = x'\vec{i} + y'\vec{j}}
		\\
		&{\vec{s'} = \sqrt{(x')^2 + (y')^2}}
		\\
		&{s'' = \frac{x'x'' + y'y''}{\sqrt{(x')^2 + (y')^2}}}
	\end{align*}
	Подставим всё, что выписали в выражение выше:
	\begin{multline*}
		\frac{d^2\vec{r}}{ds^2} = \frac{\left((x')^2 + (y')^2\right)(x''\vec{i} + y''\vec{j}) - (x'x'' + y'y'')(x'\vec{i} + y'\vec{j})}{(s')^4} =
		\\
		\frac{(x''(y')^2 - x'y'y'')\vec{i} + ((x')^2y'' - x'x''y')\vec{j}}{(s')^4} = \frac{x'y'' - x''y'}{(s')^4}(-y'\vec{i} + x'\vec{j})
	\end{multline*}
	Теперь распишем $R^2$:
	\[
		R^2 = \frac{1}{k^2} = \frac{(s')^6}{(x'y'' - x''y')^2}
	\]
	где последнее было получено по лемме о выражении кривизны.
	
	Эволюту можно записать через орты как
	\[
		\vec{\rho}(s) = \xi(s)\vec{i} + \eta(s)\vec{j}
	\]
	Если подставим полученные для всех частей формулы в исходное векторное уравнение, то сможем выделить $\xi$ и $\eta$ в том виде, в котором они даны в теореме.
\end{proof}

\begin{theorem} (Свойства эволюты и эвольвенты)
	Если $\Gamma = \{\vec{r}(s), 0 \le s \le L\}$ - трижды непрерывнод дифференцируемая кривая с натуральным параметром $s$, причём $\frac{dR}{ds} \neq 0$, то
	\begin{enumerate}
		\item Касательная к эволюте в любой точке является нормалью к эвольвенте
		
		\item Длина дуги эволюты равна соответствующему изменению радиуса кривизны эвольвенты.
	\end{enumerate}
\end{theorem}

%%% Нарисовать. Тут точно стоит пояснение рисунком сделать, с эволютой/эвольвентой любой кривой

\begin{proof}	
	Для исходной кривой $s$ - это натуральный параметр, но для эволюты - совершенно не обязательно. Поэтому штрих будет обозначать $\frac{d}{ds}$:
	\[
		\vec{\rho'}(s) = \frac{d\vec{r}}{ds} + \frac{dR}{ds}(s) \cdot \vec{\nu}(s) + R(s) \cdot \frac{d\vec{\nu}}{ds}(s)
	\]
	Для удобства опустим $(s)$:
	\[
		\vec{\rho'} = \vec{\tau} + \frac{dR}{ds}\vec{\nu} - kR\vec{\tau} = \frac{dR}{ds} \vec{\nu}
	\]
	Этим мы доказали первый пункт: касательная к эволюте в любой точке выражается через $\vec{\nu}$ - нормаль к эвольвенте.
	
	Обозначим за $\sigma$ - натуральный параметр для эволюты. Тогда
	\[
		\sigma' = |\vec{\rho'}| = \left|\frac{dR}{ds}\right| = \pm \frac{dR}{ds}
	\]
	Последний переход имеет место из-за того, что мы имеем дело с непрерывной функцией, которая не обращается в 0. В итоге получили, что производная длины эволюты с точностью до знака совпадает во всех точках с производной радиуса кривизны. По теореме Лагранжа можем заключить, что
	\[
		\sigma(s_2) - \sigma(s_1) = \pm(R(s_2) - R(s_1))
	\]
	Это и требовалось доказать.
\end{proof}

\begin{note}
	Эвольвента - это развёртка эволюты. Представить себе это можно, если взять нить, натянуть её вдоль эволюты и постепенно отодвигать её от кривой: она будет идти по касательной и отклонение будет изменяться так же, как и радиус кривизны.
\end{note}

\end{document}