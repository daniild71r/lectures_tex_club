\subsection{Числа Каталана}

Обозначим за $T_n$ - количество правильных скобочных выражений из $2n$ скобок вида $()$. Тогда $T_n$ - это $n$-ое \textit{число Каталана}.

\begin{proposition}
	Для чисел Каталана имеет место следующее рекуррентное соотношение:
	\[
		T_n = T_{n - 1}T_0 + T_{n - 2}T_1 + \ldots + T_0T_{n - 1}
	\]
	При этом $T_0 = T_1 = 1$
\end{proposition}

Само рекуррентное соотношение уже даёт подсказку для производящей функции $f(x)$ чисел Каталана. Составим её:
\[
	f(x) = T_0 + T_1 x + T_2 x^2 + \ldots + T_n x^n + \ldots
\]
А теперь посмотрим на $f^2(x)$:
\[
	f^2(x) = \underbrace{T_0^2}_{T_1} + \underbrace{(T_0T_1 + T_1T_0)}_{T_2}x + \underbrace{(T_0T_2 + T_1T_1 + T_2T_0)}_{T_3}x^2 + \ldots + \underbrace{(T_0T_n + \ldots + T_nT_0)}_{T_{n + 1}}x^n + \ldots
\]
То есть верно равенство:
\[
	f(x) = T_0 + xf^2(x) = 1 + xf^2(x) \lra xf^2 - f + 1 = 0
\]
Получаем квадратное уравнение относительно $f(x)$. Решая его, получим корни:
\[
	f = \frac{1 \pm \sqrt{1 - 4x}}{2x}
\]
Как отобрать нужный корень? Для этого домножим на $x$ и так как выражение должно оставаться корректным при всех $|x| < \rho$, то оно должно быть верным для $x = 0$. Отсюда получаем, что нам нужен корень с минусом (или из других соображений о том, что деление на $x$ должно быть корректным. Значит, единица должна сократиться с той, что появится в ряду корня):
\[
	f(x) = \frac{1 - \sqrt{1 - 4x}}{2x}
\]
Осталось, собственно, вычислить корень. Это можно сделать последовательным вычислением $\frac{1}{\sqrt{1 - x}}$ и подстановкой $4x$ вместо $x$:
\[
	\sqrt{1 - x} = 1 - C_{1/2}^1 x + C_{1/2}^2 x^2 - \ldots + (-1)^n C_{1/2}^n x^n + \ldots
\]
где $C_{1/2}^n$ можно переписать так:
\begin{multline*}
	C_{1/2}^n = \frac{\frac{1}{2} \cdot \left(\frac{1}{2} - 1\right) \cdot \ldots \cdot \left(\frac{1}{2} - n + 1\right)}{n!} = \frac{\frac{1}{2} \cdot \left(-\frac{3}{2}\right) \cdot \ldots \cdot \left(-\frac{2n - 3}{2}\right)}{n!} =
	\\
	(-1)^{n - 1} \frac{1 \cdot 3 \cdot \ldots \cdot (2n - 3)}{n! \cdot 2^n} = \frac{(-1)^{n - 1}}{2^n} \cdot \frac{(2n - 2)!}{n! \cdot 2 \cdot 4 \cdot \ldots \cdot (2n - 2)} =
	\\
	\frac{(-1)^{n - 1}}{2^n} \cdot \frac{(2n - 2)!}{n! \cdot 2^{n - 1} \cdot (n - 1)!} = \frac{(-1)^{n - 1}}{2^{2n - 1}} \cdot \frac{C_{2n - 2}^{n - 1}}{n}
\end{multline*}
Тогда коэффициент при $x^n$ в ряду $\sqrt{1 - 4x}$ имеет вид:
\[
	C_{1/2}^n \cdot (-4)^n = \frac{-2C_{2n - 2}^{n - 1}}{n}
\]
Отсюда уже получаем общий вид для $\sqrt{1 - 4x}$:
\[
	\sqrt{1 - 4x} = 1 - \frac{2C_0^0}{1} x - \frac{2C_2^1}{2} x^2 - \ldots - \frac{2C_{2n - 2}^{n - 1}}{n} x^n - \ldots
\]
Стало быть
\[
	f(x) = \frac{1 - \sqrt{1 - 4x}}{2x} = \frac{C_0^0}{1} + \frac{C_2^1}{2} x^2 + \ldots + \frac{C_{2n - 2}^{n - 1}}{n} x^{n - 1} + \frac{C_{2n}^n}{n + 1} x^n + \ldots
\]
\begin{theorem}
	$n$-е число Каталана можно записать в явном виде:
	\[
		T_n = \frac{C_{2n}^n}{n + 1}
	\]
\end{theorem}

\subsection{Предварительные сведения о теории чисел}

\subsubsection*{Сравнения по модулю}

\begin{definition}
	Пусть есть произвольные $a, b, m \in \Z$. Тогда говорят, что \textit{$a$ сравнимо с $b$ по модулю $m$}, если $m$ является делителем разности $a - b$.
	\[
		a \equiv b \pmod m \lra m \mid (a - b)
	\]
\end{definition}

\begin{definition}
	Полученное сравнение является отношением эквивалентности на $\Z$ и соответственно разбивает его на классы эквивалентности. Число, взятое за представителя класса эквивалентности сравнения по модулю, называется \textit{вычетом по модулю} $m$.
	
	Типично берут набор либо $1, 2, \ldots, m$, либо $0, 1, \ldots, m - 1$. Если набор, как выше, содержит представителей всех классов эквивалентностей, то его также называют \textit{полной системой вычетов}. Если из полной системы мы выбираем только такие элементы, которые взаимно просты с $m$, то новая система называется \textit{приведённой системой вычетов}.
\end{definition}

\begin{definition}
	Пусть есть $a, b \in \Z$. Тогда \textit{Наибольший Общий Делитель} этих чисел мы будем обозначать как $(a, b)$, а \textit{Наименьшее Общее Кратное} как $[a, b]$.
	
	Числа $a$ и $b$ называются \textit{взаимно простыми}, если $(a, b) = 1$.
\end{definition}

\begin{definition}
	\textit{Функцией Эйлера} называется $\phi(m)$, которая возвращает количество взаимно простых с $m$ чисел из множества $\{1, \ldots, m\}$:
	\[
		\phi(m) = \left|\{a \in \{1, \ldots, m\} \such (a, m) = 1\}\right|
	\]
\end{definition}

\begin{theorem} (Малая теорема Ферма)
	Пусть $p$ - простое число. Пусть $a \in \N$ такое, что $(a, p) = 1$. Тогда
	\[
		a^{p - 1} \equiv 1 \pmod p
	\]
\end{theorem}

\begin{corollary}
	Для любого $a \in \N$, удовлетворяющего условию теоремы, будет верно также и следующее утверждение:
	\[
		a^p \equiv a \pmod p
	\]
\end{corollary}

\begin{proof} Докажем теорему двумя способами:
\begin{enumerate}
	\item Чтобы не облегчить себе жизнь, будем доказывать её следствие, которое равносильно исходной теореме (в силу $(a, p) = 1$). Представим $a^p$ в следующем виде:
	\[
		a^p = (\underbrace{1 + \ldots + 1}_{a})^p = \underbrace{1 + \ldots + 1}_{a} + \suml_{n_1 + \ldots + n_a = p \over \forall i \in [1; a]\ n_i < p} P(n_1, \ldots, n_a) 1^{n_1} \cdot \ldots \cdot 1^{n_a}
	\]
	Кроме единиц останутся только слагаемые, содержащие полиномиальные коэффициенты и произведение единиц (которое, естественно, можно просто убрать). Распишем его:
	\[
		P(n_1, \ldots, n_a) = \frac{p!}{n_1! \cdot \ldots \cdot n_a!}
	\]
	При этом $\forall i \in [1; a]\ n_i < p$. Отсюда следует, что $n_i!$ не будет делиться на $p$ для любого $i$. При этом числитель на $p$ делится. Следовательно, все слагаемые с полиномиальным коэффициентом обнуляются в кольце вычетов по $p$ и мы получаем необходимое тождество:
	\[
		a^p \equiv a \pmod p
	\]
	
	\item Возьмём $\{1, 2, \ldots, p - 1\}$ за приведенную систему вычетов по модулю $p$. Тогда утверждается, что множество $\{a \cdot 1, a \cdot 2, \ldots, a \cdot (p - 1)\}$ - тоже приведённая система вычетов по модулю $p$. Почему это так? Предположим противное:
	\[
		\exists i, j \in \{1, \ldots, p - 1\} \such i \neq j,\ \ a \cdot i \equiv a \cdot j \pmod p
	\]
	В таком случае верно следующее:
	\[
		a \cdot (i - j) \equiv 0 \pmod p
	\]
	Но разность $i - j$ заведомо меньше $p$ и даже не может равняться $-p$ или $0$. Значит, что $a$ делится на $p$. Противоречие. Существование такой приведённой системы вычетов даёт нам следующий факт:
	\[
		1 \cdot 2 \cdot \ldots \cdot (p - 1) \equiv (a \cdot 1) \cdot (a \cdot 2) \cdot \ldots \cdot (a \cdot (p - 1)) \pmod p
	\]
	Так как одинаковые сомножители с обеих сторон взаимно просты с $p$, то сократив их получим утверждение теоремы:
	\[
		1 \equiv a^{p - 1} \pmod p
	\]
\end{enumerate}
\end{proof}

\begin{theorem} (Теорема Эйлера)
	Пусть $m \in \N$, а также есть $a \in \N$ такое, что $(a, m) = 1$. Тогда
	\[
		a^{\phi(m)} \equiv 1 \pmod m
	\]
\end{theorem}

\begin{note}
	Теорема Эйлера является полным обобщением Малой теоремы Ферма из-за того факта, что
	\[
		\phi(p) = p - 1
	\]
\end{note}

\begin{proof}
	По аналогии со вторым доказательством Малой теоремы Ферма, выберем приведённую систему вычетов по модулю $m$: $\{b_1, \ldots, b_{\phi(m)}\}$. Тогда и $\{ab_1, \ldots, ab_{\phi(m)}\}$ будет приведённой системой вычетов по модулю $m$. Следовательно
	\[
		a^{\phi(m)} \cdot b_1 \cdot \ldots \cdot b_{\phi(m)} \equiv b_1 \cdot \ldots \cdot b_{\phi(m)} \pmod m
	\]
	Сокращая сомножители с обеих сторон снова получим нужное тождество:
	\[
		a^{\phi(m)} \equiv 1 \pmod m
	\]
\end{proof}