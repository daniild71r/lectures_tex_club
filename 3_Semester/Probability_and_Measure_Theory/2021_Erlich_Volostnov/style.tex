\documentclass[a4paper,12pt]{article}
\usepackage{cmap}
\usepackage{amssymb}
\usepackage{amsmath}
\usepackage[T2A]{fontenc}
\usepackage[utf8]{inputenc}
\usepackage[russian]{babel}
\usepackage{indentfirst}
\usepackage{epigraph}
\usepackage{framed}
\renewcommand{\epigraphsize}{\small}
% \pagestyle{empty}

\usepackage[left=2cm,right=2cm,
    top=2cm,bottom=2cm]{geometry}
\usepackage{graphicx}
\graphicspath{ {./images/} }
\newcommand{\RomanNumeralCaps}[1] {\MakeUppercase{\romannumeral #1}}

\newcommand{\mysection}[2]{\setcounter{section}{#1}\addtocounter{section}{-1}\section{#2}}

% код
\DeclareRobustCommand{\svdots}{% s for `scaling'
\, \vcenter{%
\offinterlineskip
\hbox{.}
\vskip0.25\normalbaselineskip
\hbox{.}
\vskip0.25\normalbaselineskip
\hbox{.}%
}%
\,
}
\usepackage{listings}
\usepackage[unicode, pdftex]{hyperref}
\usepackage{xcolor}

\usepackage{dsfont}
\usepackage{mathabx}
\usepackage{multirow}
\usepackage{hhline}
\usepackage{float}

\definecolor{linkcolor}{HTML}{50006b} % цвет ссылок
%\definecolor{urlcolor}{HTML}{107896} % цвет гиперссылок
\definecolor{urlcolor}{HTML}{50006b} % цвет гиперссылок
 
\hypersetup{pdfstartview=FitH,  linkcolor=linkcolor,urlcolor=urlcolor, colorlinks=true}

\definecolor{codegreen}{rgb}{0,0.6,0}
\definecolor{codegray}{rgb}{0.5,0.5,0.5}
\definecolor{codepurple}{rgb}{0.58,0,0.82}
\definecolor{backcolour}{cmyk}{0,0,0,0.05}

\lstdefinestyle{mystyle}{
backgroundcolor=\color{backcolour},
commentstyle=\color{codegreen},
keywordstyle=\color{magenta},
numberstyle=\tiny\color{codegray},
stringstyle=\color{codepurple},
basicstyle=\ttfamily\footnotesize,
breakatwhitespace=false,
breaklines=true,
captionpos=b,
keepspaces=true,
numbers=left,
numbersep=5pt,
showspaces=false,
showstringspaces=false,
showtabs=false,
tabsize=2,
texcl=true
}

\lstset{extendedchars=\true, style=mystyle}

\usepackage[pages = some]{background}
\backgroundsetup{
	scale = 1,
	angle = 0,
	opacity = 1,
	contents = {\includegraphics[height = \paperheight, keepaspectratio]{background.jpg}}}

\newcommand{\Def}{\textbf{Определение:} }  % объявление новых макрокоманд
\newcommand{\Statement}{\textbf{Утверждение:} }
\newcommand{\Lemma}{\textbf{Лемма:} }
\newcommand{\LemmaN}[1]{\textbf{Лемма \emph{#1}:}}
\newcommand{\Th}{\textbf{Теорема:} }
\newcommand{\Conseq}{\textbf{Следствие:} }
\newcommand{\Task}{\textbf{Задача:} }
\newcommand{\Solution}{\textbf{Решение:} }
\newcommand{\Example}{\textbf{Пример:} }
\newcommand{\Examples}{\textbf{Примеры:} }
\newcommand{\Note}{\textbf{Замечание:} } 
\newcommand{\Vars}{\textbf{Введем обозначения:} } 
\newcommand{\Idea}{\textbf{Идея:} } 
\newcommand{\Proof}{$\blacktriangle$ }
\newcommand{\EndProof}{$\blacksquare$ }

\renewenvironment{leftbar}[2][\hsize]
{
    \def\FrameCommand
    {
        {\hspace{20pt}
        \color{gray}\vrule width 3pt}
        \hspace{0pt}
    }
    \MakeFramed{\hsize#1\advance\hsize-\width\FrameRestore}
}
{\endMakeFramed}

\newcommand{\norm}{\triangleleft}
\newcommand{\bnorm}{\triangleright}

% скобки, модуль
\newcommand{\angles}[1]{\left\langle{#1}\right\rangle}
\newcommand{\abs}[1]{\left|{#1}\right|}
\newcommand{\brackets}[1]{\left({#1}\right)}

% Натуральные, целые, действительные и борелевское что-то
\newcommand{\N}{\mathbb{N}}
\newcommand{\Z}{\mathbb{Z}}
\newcommand{\R}{\mathbb{R}}
\newcommand{\B}{\mathfrak{B}}
\newcommand{\M}{\mathcal{M}}
\newcommand{\e}{\varepsilon}

%сходимость по мере
\newcommand{\convme}{\xrightarrow{\mu}}
% сходимость почти всюду
\newcommand\eqae{\stackrel{\text{п.в.}}{=}}
% равенство почти всюду
\newcommand{\convae}{\xrightarrow{\text{п.в.}}}

\setcounter{page}{-2}