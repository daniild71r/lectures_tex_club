\section{Системы множеств (полукольца, кольца, алгебры, сигма-алгебры). Примеры. Минимальное кольцо, содержащее полукольцо. Понятие наименьшего кольца, алгебры, сигма-алгебры, содержащей систему множеств.}
\setcounter{section}{8}
\section{Проверка правильности скобочной последовательности с несколькими типами скобок.}
\par \textbf{Определение}: \textit{Правильные скобочные последовательности с несколькими типами скобок} (рассмотрим с двумя)
\begin{enumerate}
    \item $\varepsilon$ (пустое слово) - ПСП
    \item $S$ - ПСП $\Rightarrow (S), [S]$ - ПСП
    \item $S_1, S_2$ - ПСП $\Rightarrow S_1 S_2$ - ПСП 
\end{enumerate}
\par \textbf{Задача:} Проверить, является ли последовательность из нескольких типов скобок правильной скобочной последовательностью
\par \textbf{Решение:} Храним стек незакрытых открывающих скобок
\lstinputlisting[language=C++,
emph={int,char,double,float,unsigned},
emphstyle={\color{blue}}
]{code/9_psp.cpp}
\par \textbf{Утверждение:} Данный алгоритм корректен, то есть ПСП $\Leftrightarrow$ алгоритм вывел true
\begin{itemize}
    \item[$\blacktriangle \Rightarrow$] Индукция по построению \begin{enumerate}
    \item База: $\varepsilon$ - обработается корректно
    \item $T=(u)$, $u$ - ПСП. По предположению индукции, всё $u$ удалится из стека к моменту прихода закрывающей скобки (можно считать, что стек начинается после первой скобки, она никак не влияет на применение алгоритма к $u$), а внешние скобки обработаются корректно (аналогично для других типов скобок)
    \item $T=T_1 T_2; T_1, T_2$ - ПСП. По предположению индукции, после того как считается $T_1$ стек опустошится $\Rightarrow$ после $T_2$ - тоже $\Rightarrow$ алгоритм сработает корректно $\blacksquare$
    \end{enumerate}
    \item[$\Leftarrow$] Доказываем индукцией по количеству действий, "обращая" предыдущий пункт.
    \par Рассмотрим скобочную последовательность, на которую алгоритм выдаёт true. Алгоритм сопоставил каждой открывающейся скобке одного типа закрывающуюся скобку того же типа. Причём они обязательно идут в правильном порядке. 
\parДля доказательства факта используем индукцию.
Рассмотрим пары соответсвующих скобок в порядке закрытия пары:
\begin{enumerate}
    \item База: Если между парой скобок нет других скобок, то последовательность от одной скобки до другой - правильная.
    \item Шаг: Если между парой скобок (назовём их $a$ и $b$ соответственно) есть непустая подстрока, то все скобки из подстроки уже были рассмотрены индукцией, так как открывающиеся скобки в подстроке были позже, чем $a$ добавлены в стек и по правилу стека должны были раньше из него выйти, а значит они уже были рассмотрены индукцией. Аналогично с закрывающимеся скобками из подстроки - они идут раньше, чем $b$, следовательно, по правилу стека им ставили в соответствие открывающиеся скобки, которые были добавленны позже $a$. (окрывающаяся скобка не могла быть добавленны раньше $a$, потому что $a$ перегородила бы ей выход.) По предположению индукции получаем, что подстрока состоит из одной или нескольких ПСП $\Rightarrow$ сама подстрока ПСП. $\Rightarrow$ Подпоследовательность от $a$ до $b$ - правильная. $\blacksquare$
\end{enumerate}
\end{itemize} 

\newpage

\section{Меры на полукольцах. Классическая мера Лебега на полукольце промежутков и ее сигмааддитивность.}
\includegraphics[width=1\linewidth]{sections/Polina/imgs/21.jpg}

\newpage

\section{Продолжение меры с полукольца на минимальное кольцо. Наследование сигма-аддитивности при продолжении меры. Внешние меры Лебега и Жордана. Мера Лебега. Свойства. Сигмаалгебра измеримых по Лебегу множеств. Сигма-аддитивность меры Лебега на сигмаалгебре измеримых по Лебегу множеств.}

\par \Th Пусть $S$ - полукольцо и $m$ - мера на $S$. Тогда $\exists! \nu$ - мера на $R(S)$, такая что $\forall A \in S \hookrightarrow m(A)=\nu(A)$. Кроме того, если $m$ сигма-аддитивна, то и $\nu$ сигма-аддитивна.
\par \Proof Пусть $A \in R(S), A=\bigsqcup_{i=1}^n A_i, A_i \in S$. Тогда определим $\nu(A):=\sum_{i=1}^n m(A_i)$ (по-другому определить не можем, то есть мера не более чем единственна)
\begin{enumerate}
    \item Корректность (независимость от представления $A$): Пусть $A=\bigsqcup_{i=1}^n A_i=\bigsqcup_{j=1}^m B_j$. Определим $C_{ij}=A_i \cap B_j \in S$. Тогда $A=\bigsqcup_{ij}C_{ij}$.
    $$\sum_{i=1}^n m(A_i)=\sum_{i=1}^n \sum_{j=1}^m m(C_{ij})=\sum_j \sum_i m(C_{ij})=\sum_j m(B_j)$$
    \item Аддитивность: $A=\bigsqcup_{i=1}^n A_i; A, A_i \in R(S)$. Тогда $A_i=\bigsqcup_{j=1}^m B_{ij}$ и $A=\bigsqcup_{ij} B_{ij}$. 
    $$\nu(A)=\sum_{i=1}^n \sum_{j=1}^m m(B_{ij})=\sum_{i=1}^n \nu(A_i)$$. Неотрицательность очевидна, значит $\nu$ - это мера
    \item Сигма-аддитивность: $A=\bigsqcup_{i=1}^\infty A_i,  A_i=\bigsqcup_{j=1}^{l_i} B_{ij}, A=\bigsqcup_{k=1}^m C_k$. Определим $D_{ijk}=C_k \cap B_{ij}$.
    $$\nu(A)=\sum_{k=1}^m m(C_k)=\sum_{k=1}^m \sum_{i=1}^\infty \sum_{j=1}^{l_i} m(D_{ijk})=\sum_{i=1}^\infty \sum_{k=1}^m \sum_{j=1}^{l_i} m(D_{ijk})=\sum_{i=1}^\infty \nu(A_i)$$
    \par В предпоследнем равенстве имеем право поменять знаки суммы так как ряд сходится абсолютно. \EndProof
\end{enumerate}

\par \textbf{Следствия:} Пусть $m$ - мера на полукольце $S$. Тогда
\begin{enumerate}
    \item Если $\bigsqcup_{i=1}^\infty A_i \subset A \Rightarrow \sum_{i=1}^\infty m(A_i) \leq m(A)$
    \item Если $A \subset \cup_{i=1}^n A_i \Rightarrow m(A) \leq \sum_{i=1}^n m(A_i)$
    \item Если $m$ сигма-аддитивна и $A \subset \cup_{i=1}^\infty A_i \Rightarrow m(A) \leq \sum_{i=1}^\infty m(A_i)$
\end{enumerate}
\par \Proof Продолжим меру $m$ до $\nu$ на $R(S)$. \begin{enumerate}
    \item $$m(A)=\nu(A)=\nu(\bigsqcup_{i=1}^n A_i \bigsqcup (A \setminus (\bigsqcup_{i=1}^n A_i)))=\sum_{i=1}^n \nu(A_i) + \nu(A \setminus (\bigsqcup_{i=1}^n A_i)) \geq \sum_{i=1}^n m(A_i)$$
    \par Раскрыть по аддитивности (третье равенство) можем в силу замкнутости кольца относительно операций. Остается только перейти к пределу в полученном неравенстве.
    \item Будем доказывать что $\nu(A) \leq \sum_{i=1}^n \nu(A_i)$. Введем $B_k=(A_k \setminus \cup_{i=1}^{k-1} A_i) \cap A$. Тогда $A=\bigsqcup_{k=1}^n B_i$. Так как $B_k \in R(S)$ можем записать $\nu(A)=\sum_{i=1}^n \nu(B_i) \leq \sum_{i=1}^n \nu(A_i)$ (последнее неравенство справедливо, так как $B_i \subset A_i$)
    \item Доказывается аналогично прошлому пункту (теперь в последнем равенстве можем перейти к ряду в силу сигма-аддитивности) \EndProof
\end{enumerate}
\par Пусть $S$ - полукольцо с единицей $E$, $m$ - сигма-аддитивная мера на полукольце

\par \Def $\mu^*(A)=\underset{A\subset \cup_{i=1}^\infty A_i, A_i \in S}{inf}\sum_{i=1}^\infty m(A_i)$ - \textit{внешняя мера} ($\mu^*: 2^E \rightarrow [0;+\infty)$ - обязательно конечна, так как живем в полукольце с единицей)

\leftbar
\par \textbf{Альтернативное определение:} $\overline{\mu^*}(A)=\underset{A \subset \bigsqcup A_i, A_i \in S}{inf} \sum_{i=1}^\infty m(A_i)$

\par \Statement $\mu^*=\overline{\mu^*}$
\par \Proof Покрытие непересекающимися множествами так же учитывается в определении $\mu^* \Rightarrow \mu^*(A) \leq \overline{\mu^*}(A)$
\par Пусть $A \subset \cup_{i=1}^\infty A_i$. Положим $B_k=A_k \setminus \cup_{i=1}^{k-1}A_i$. Тогда $\cup_{i=1}^\infty A_i=\bigsqcup_{k=1}^\infty B_k$. Так как $B_k \in R(S)$, оно представляется в виде $B_k=\bigsqcup C_{jk}$. Тогда $$\overline{\mu^*}(A) \leq \sum_j \sum_k m(C_{jk})=\sum_k \nu(B_k) \leq \sum_k \nu(A_k)=\sum_k m(A_k)$$
\par Остается только перейти к $inf$ в этом неравенстве $\Rightarrow \overline{\mu^*}(A) \leq \mu^*(A)$
\endleftbar

\par \textbf{Свойства внешней меры:} \begin{enumerate}
    \item $\mu^*(A) \leq m(E) < +\infty$
    \item Пусть $\nu$ - продолжение $m$ на $R(S)$. Тогда $\forall A \in R(S) \hookrightarrow \mu^*(A)=\nu(A)$
    \par \Proof $A=\bigsqcup_{i=1}^n A_i, A_i \in S$. Тогда $\mu^*(A) \leq \sum_{i=1}^n m(A_i)=\nu(A)$
    \par Пусть $A \subset \cup_{i=1}^\infty A_i$. Тогда по следствию 3 из прошлой теоремы: $\nu(A) \leq \sum_{i=1}^\infty \nu(A_i)=\sum_{i=1}^\infty m(A_i) \Rightarrow \nu(A) \leq \mu^*(A)$ (перешли к $inf$) \EndProof
    \item $A \subset \cup_{i=1}^\infty A_i, \forall A, A_i \subset E \Rightarrow \mu^*(A) \leq \sum_{i=1}^\infty \mu^*(A_i)$
    \par \Proof Рассмотрим такое покрытие $A_i \subset \cup_{j=1}^\infty B_{ij}$, что $\sum_{j=1}^\infty m(B_{ij}) \leq \mu^*(A_i)+\frac{\varepsilon}{2^i}$. Тогда $A \subset \cup_{i, j} B_{ij}$, а значит $\mu^*(A) \leq \sum_i \sum_j m(B_{ij}) \leq \sum_{i=1}^\infty \mu^*(A_i) + \varepsilon$. В силу произвольности выбора $\varepsilon$ получаем искомое неравенство при устремлении его к 0. \EndProof
\end{enumerate}

\par \Conseq $|\mu^*(A)-\mu^*(B)| \leq \mu^*(A \Delta B)$
\par \Proof $A \subset B \cup (A \Delta B) \Rightarrow \mu^*(A) \leq \mu^*(B)+\mu^*(A \Delta B) \Rightarrow \mu^*(A)-\mu^*(B) \leq \mu^*(A \Delta B)$. Записав аналогичное выражение для $B$, получим $\mu^*(B)-\mu^*(A) \leq \mu^*(A \Delta B)$. Так как для любой из разностей это неравенство верно, то оно верно и для модуля \EndProof

\Def $A \subset E$ называется \textit{измеримым по Лебегу}, если $\forall \varepsilon > 0 \: \exists A_\varepsilon \in R(S)$, такое что $\mu^*(A \Delta A_\varepsilon)<\varepsilon$. $\M$ - множество всех измеримых по Лебегу множеств из $2^E$

\par \Def \textit{Лебеговым продолжением $m$ (мерой Лебега)} называется $\mu: \M \rightarrow [0;+\infty)$, такая что $\forall A \in \M \hookrightarrow \mu(A)=\mu^*(A)$

\par \Statement $R(S) \subset \M$
\par \Proof В качестве $A_\varepsilon$ можно выбрать $A$ и тогда определение будет выполнено. Кроме того, $\forall A \in R(S) \hookrightarrow \mu(A)=\nu(A)$ (по второму свойству внешней меры) \EndProof

\par \Th $\M$ - сигма-алгебра, $\mu$ - сигма-аддитивная мера на $\M$
\par \begin{itemize}
    \item[\Proof 1.] $\M$ - алгебра: $\varnothing, E \in R(S) \subset \M$
    \par $A, B \in \M$. Тогда $\forall \varepsilon > 0 \exists A_{\varepsilon/2}, B_{\varepsilon/2}: \mu^*(A \Delta A_{\varepsilon/2})<\frac{\varepsilon}{2}, \mu^*(B \Delta B_{\varepsilon / 2}) < \frac{\varepsilon}{2}$.
    \par Из определения множественных операций справедливы включения
    $$(A \cap B) \Delta (A_{\varepsilon/2} \cap B_{\varepsilon/2}) \subset (A \Delta A_{\varepsilon/2}) \cup (B \Delta B_{\varepsilon/2}) \supset (A \Delta B) \Delta (A_{\varepsilon/2} \Delta B_{\varepsilon/2})$$
    \par Тогда из свойства 3 внешней меры получаем
    $$\mu^*((A \cap B) \Delta (A_{\varepsilon/2} \cap B_{\varepsilon/2})) \leq \mu^*(A \Delta A_{\varepsilon/2})+\mu^*(B \Delta B_{\varepsilon/2}) < \varepsilon \Rightarrow A \cap B \in \M$$
    \par Аналогично получаем $A \Delta B \in \M \Rightarrow \M$ - алгебра.
    
    \item[2.] $\mu$ - мера на $\M$: Достаточно показать, что $\forall B, C \in \M: B \cap C=\varnothing \hookrightarrow \mu(B \bigsqcup C)=\mu(B)+\mu(C)$. Обозначим $A=B \bigsqcup C$
    \par Из третьего свойства внешней меры: $\mu(A) \leq \mu(B)+\mu(C)$. Докажем в обратную сторону.
    \par Так как $B,C \in \M$, то $\forall \varepsilon>0 \exists B_\varepsilon, C_\varepsilon\in R(S): \mu^*(B \Delta B_\varepsilon) < \varepsilon, \mu^*(C \Delta C_\varepsilon)< \varepsilon$
    \par $B_\varepsilon \cap C_\varepsilon \subset (C_\e \setminus C) \cup (B_\e \setminus B)$ (так как $B \cap C = \varnothing$, а значит если элемент был слева то он не выкинется хотя бы из одной скобки справа) $\subset (C_\e \Delta C) \cup (B_\e \Delta B) \Rightarrow \mu^*(B_\e \cap C_\e) \leq \mu^*(C_\e \Delta C)+\mu^*(B_\e \Delta B) < 2\e$
    \par $A \Delta (B_\e \cup C_\e)\subset(B \Delta B_\e) \cup (C \Delta C_\e) \Rightarrow \mu^*(A \Delta (B_\e \cup C_\e)) \leq \mu^*(B \Delta B_\e)+\mu^*(C \Delta C_\e)<2\e$
    \par $|\mu(A)-\mu(B_\e \cup C_\e)| \leq \mu^*(A \Delta (B_\e \cup C_\e)) < 2\e$ (по следствию), а значит $\mu(A) \geq \mu(B_\e \cup C_\e)-2\e$.
    \par Так как $B_\e, C_\e \in R(S)$, получаем $\mu(B_\e \cup C_\e)=\nu(B_\e \cup C_\e)=\nu(B_\e)+\nu(C_\e)-\nu(B_\e \cap C_\e)=\mu(B_\e)+\mu(C_\e)-\mu(B_\e \cap C_\e) \geq \mu(B)+\mu(C)-4\e$. Тогда $\mu(A) \geq \mu(B)+\mu(C)-4\e$ и в силу произвольности выбора $\e$ верное $\mu(A)\geq \mu(B)+\mu(C)$ \EndProof
    \item[3.] $\M$ - сигма-алгебра: Пусть $A_1, \ldots, A_n \ldots \in \M$. Покажем, что $A=\cup_{i=1}^\infty A_i \in \M$. Определим $B_k=A_k \setminus (\cup_{i=1}^{k-1} A_i)$. Тогда $A=\bigsqcup_{k=1}^\infty B_k \Rightarrow A \supset \bigsqcup_{k=1}^n B_k \Rightarrow \mu^*(A) \geq \mu^*(\bigsqcup_{k=1}^n B_k)=\mu(\bigsqcup_{k=1}^n B_k)=\sum_{k=1}^n \mu(B_k)$ (так как уже доказали что $\mu^*$ - это мера на $\M$) $\Rightarrow \sum_{k=1}^\infty \mu(B_k)$ сходится $\Rightarrow \exists N: \sum_{k=N+1}^\infty \mu(B_k) < \e$.
    \par $A=(\bigsqcup_{k=1}^N B_k)\bigsqcup(\bigsqcup_{k=N+1}^\infty B_k)=C \bigsqcup D=C \Delta D$. $\mu^*(D) \leq \sum_{k=N+1}^\infty \mu^*(B_k) < \e$, $C \in \M$ (так как представляется в виде объединения множеств $\M$, а $\M$ - алгебра). Тогда $\exists C_\e \in R(S): \mu^*(C \Delta C_\e)<\e$. 
    $$A \Delta C_\e=(C \Delta C_\e) \Delta D \subset (C \Delta C_\e) \cup D \Rightarrow \mu^*(A \Delta C_\e) \leq \mu^*(C \Delta C_\e)+\mu^*(D)<2\e \Rightarrow A\in \M$$
    \item[4.] $\mu$ сигма-аддитивна: Пусть $A=\bigsqcup_{i=1}^\infty A_i$. Тогда $\mu(A) \leq \sum_{i=1}^\infty \mu(A_i)$ (из полуаддитивности внешней меры) и $\mu(A) \geq \sum_{i=1}^\infty \mu(A_i)$ (по следствию 1 из теоремы о продолжении меры на кольцо) \EndProof
\end{itemize}

\par \Def $\mu^*_J(A)=\underset{A\subset \cup_{i=1}^n A_i, A_i \in S}{inf}\sum_{i=1}^n m(A_i)$ - \textit{внешняя мера Жордана}. Аналогично можно определить алгебру (не сигма) измеримых по Жордану множеств ($\M_J \subset \M$), меру Жордана.

\par \textbf{Теорема (б/д):} $\M_J$ - алгебра, $\mu_J$ - сигма-аддитивная мера на $\M_J$. $\M_J \subset \M$ и $\forall A \in \M_J \hookrightarrow \mu_J(A)=\mu(A)$.
\newpage{}

\section{Структура измеримых множеств. Теорема Каратеодори.}
\begin{Def}
	Косинус-преобразованием Фурье функции $f\in L_1[0,+\infty)$ называется $f_c(\lambda)=\sqrt{\frac{2}{\pi}}\int\limits_0^{+\infty}f(t)\cos\lambda td\mu(t)$. Синус-преобразованием Фурте функции $f\in L_1[0,+\infty)$ называется $f_s(\lambda)=\sqrt{\frac{2}{\pi}}\int\limits_0^{+\infty}f(t)\sin\lambda td\mu(t)$.
\end{Def}

\begin{corollary}[из признака Дини]\ \\
	\begin{enumerate}
		\item Если $f\in L_1(\R)$ и удовлетворяет уловию Гельдера в точке $x$, то $\tilde{\hat{f}}=f(x)$ или обозначается $F^{-1}[F[f]](x)=f(x)$.
		\item Если $f\in L_1([0,+\infty))$ и удовлетворяет уловию Гельдера в точке $x$, то $(f_c)_c(x)=f(x), (f_s)_s(x)=f(x)$.
	\end{enumerate}
\end{corollary}

\begin{linkthm}{https://youtu.be/qRAS2-XHKws?t=1009}[Свойства преобразования Фурье функций из $L_1(\R)$]\ \\
	\begin{enumerate}
		\item (Линейность) Если $f,g\in L_1(\R), \alpha, \beta\in\R$, то $\widehat{\left(\alpha f+\beta g\right)}(\lambda)=\alpha\hat{f}(\lambda)+\beta\hat{g}(\lambda)$.
		\item Если $f\in L_1(\R)$, то $\hat{f}$ непрерывна на $\R, \lim\limits_{\lambda\to\infty}\hat{f}(\lambda)=0$.
		\item Если $f\in L_1(\R), \varphi(x)=f(\alpha x), \alpha>0$, то $\hat{\varphi}(\lambda)=\frac{1}{\alpha}\hat{f}\left(\frac{\lambda}{\alpha}\right)$.
		\item Если $f\in L_1(\R) \psi(x)=f(x+a), a\in\R$, то $\hat{\psi}(\lambda)=e^{ia\lambda}\hat{f}(\lambda)$. 
	\end{enumerate}
\end{linkthm}
\begin{proof}
	TODO
\end{proof}

\begin{linkthm}{https://youtu.be/qRAS2-XHKws?t=1512}[Преобразование Фурье производной]\ \\
	Если $f$ абсолютно непрерывна на любой отрезке $[a,b]\subset\R$, и $f,f'\in L_1(\R)$, то $\widehat{f'}(\lambda)=(-i\lambda)\hat{f}(\lambda), \forall\lambda\in\R$.
\end{linkthm}

\begin{proof}
	TODO
\end{proof}

\begin{corollary}
	Если $f^{(n-1)}$ --- абсолютно непрерывна на любом отрезке $[a,b]\subset\R$ и $f,f',\ldots,f^{(n)}\in L_1(\R)$, то $\widehat{f^{(n)}}(\lambda)=(-i\lambda)^n\hat{f}(\lambda),\lambda\in\R$.
\end{corollary}

\begin{linkthm}{https://youtu.be/qRAS2-XHKws?t=2370}[Производная преобразования Фурье]\ \\ 
	Если $f(t),t\cdot f(t)\in L_1(\R)$, то преобразование Фурье $\hat{f}(\lambda)$ дифференцируемо, причем $\left(\hat{f}\right)'(\lambda)=\widehat{(-itf(t))}(\lambda), \lambda\in\R$.
\end{linkthm}

\begin{proof}
	TODO
\end{proof}

\begin{Def}
	Сверткой функций $f,g\in L_1(\R)$ называется $(f\ast g)(t)=\int\limits_{-\infty}^{+\infty}f(x)g(t-x)d\mu(x)$.
\end{Def}
\begin{linkthm}{https://youtu.be/qRAS2-XHKws?t=3297}[Преобразование Фурье свертки]\ \\
	Если $f,g\in L_1(\R)$, то $\widehat{(f\ast g)}(\lambda)=\sqrt{2\pi}\hat{f}(\lambda)\hat{g}(\lambda)$.
\end{linkthm}
\begin{proof}
	TODO
\end{proof}










\newpage{}

\section{Полнота и непрерывность мер. Теоремы о связи непрерывности и сигма-аддитивности.}
\Def

Заданная на кольце $R$ подмножеств некоторого множества $X$ мера $\mu$ называется \textit{полной}, если из того, что $A \in R$, $\mu(A) = 0$ и $B \subset A$, следует, что $B \in R$ и $\mu(B) = 0$.

\Statement Меры Лебега и Жордана является полными.

\Proof
Пусть $A$ измеримо по Лебегу и $\mu(A) = 0$. Так как $B \subset A$, то $\mu^*(B) \leq \mu^*(A) = \mu(A) = 0$. Докажем, что если $\mu^*(B) = 0$, то $B$ измеримо по Лебегу.

$\forall \varepsilon > 0 \ \exists B_\varepsilon = \emptyset : \mu^*(B \triangle B_\varepsilon) = 0 < \varepsilon$, а значит $B$ измеримо по Лебегу и $\mu(B) = 0$.

Доказательство для меры Жордана аналогично.
\EndProof

\Note Мера Бореля не является полной 
Можно взять континуальное множество меры нуль (например, множество Кантора) -- обозначим его $E$. Тогда $2^E$ более, чем континуально, а значит найдется неборелевское подмножество $E$ (так как борелевских множеств континуальное количество).

\Def

Пусть на кольце $R$ задана конечная мера $\mu$, и дана такая последовательность элементов кольца $A_1 \supset A_2 \supset \ldots$, что
$A = \bigcap \limits_{i = 1}^{\infty} A_i, A_i \in R$.

Если для такой последовательности $\{ A_i \}$ верно, что $\mu(A) = \lim \limits_{i \to \infty} \mu(A_i)$, то $\mu$ называется \textit{непрерывной}.

\Th Заданная на кольце $R$ мера $\mu$ непрерывна тогда и только тогда, когда она $\sigma$-аддитивна.

\Proof

Пусть $\mu$ $\sigma$-аддитивна и $A = \bigcap \limits_{i = 1}^{\infty} A_i$, где множества $A_i$ вложены и $A, A_1, A_2, \ldots \in R$. 
Положим $B_i = A_i \setminus A_{i + 1}$ при $i \geq 1$. Тогда

$$
    A_1 \setminus A = \bigsqcup_{n = 1}^{\infty} B_n
$$, откуда

$$
    \mu(A_1 \setminus A) = \mu(A_1) - \mu(A) = \sum_{n = 1}^{\infty} \mu(B_n)
    = \lim_{n \to \infty} \sum_{k = 1}^{n - 1} \mu(B_k) = 
$$
$$
    = \lim_{n \to \infty} \left( \sum_{k = 1}^{n - 1} \left( \mu(A_k) - \mu(A_{k + 1}) \right) \right) = 
    \mu(A_1) - \lim_{n \to \infty} \mu(A_n)
$$, то есть

$$
    \mu(A) = \lim_{n \to \infty} \mu(A_n)
$$

Пусть теперь мера $\mu$ непрерывна и $A = \bigsqcup \limits_{n = 1}^{\infty} A_n$, где $A, A_1, A_2, \ldots \in R$. Положим 

$$
    B_n = \bigsqcup_{i = n}^{\infty} A_i = 
    A \setminus \bigsqcup_{i = 1}^{n - 1} A_i.
$$

Тогда $B_1 \supset B_2 \supset \ldots$ и $\bigcap \limits_{n = 1}^{\infty} B_n = \emptyset$. Поэтому

$$
    0 = \lim_{n \to \infty} \mu(B_n) = 
    \lim_{n \to \infty} \mu \left( A \setminus \bigsqcup_{i = 1}^{n - 1} A_i \right) = 
    \lim_{n \to \infty} \left( \mu(A) - \sum_{k = 1}^{n - 1} \mu(A_k) \right) = 
    \mu(A) - \lim_{n \to \infty} \sum_{k = 1}^{n - 1} \mu(A_k)
$$.

Это и означает, что $\mu(A) = \sum \limits_{n = 1}^{\infty} \mu(A_n)$.
\EndProof

\section{Мера Бореля. Меры Лебега-Стилтьеса на прямой и ее сигма-аддитивность.}
M - множество всех измеримых множеств.

\Def \textbf{Лебегово продолжение m (мера Лебега)} - $\mu: M \rightarrow [0; +\infty): \forall A \in M \mu (A) = \mu^* (A)$ (Классический случай: полукольцо - полукольцо промежутков, мера - разность концов -> классическая мера Лебега).

(\Idea для задания меры нам нужно задать $\sigma$-аддитивную меру на полукольце с единицей, дальше продолжаем её до кольца с единицей, а дальше мы можем, используя внешнюю меру, перейти к измеримым множествам. Этот переход и есть лебегово продолжение)

\vspace{5pt}

\Def \textbf{Мера Лебега-Стильтеса}.

Полукольцо $S = \{ (a; b] | -\infty \leqslant a \leqslant b \leqslant +\infty \}$; значит, есть нейтральный элемент относительно $\cdot$: $\mathbb{R} = (-\infty; +\infty)$ - значит, это полукольцо с единицей.

$\varphi: \mathbb{R} \rightarrow \mathbb{R}$: неубывающая; непрерывна справа; ограниченная; $\Rightarrow \exists \varphi(-\infty), \varphi(+\infty)$; тогда $m((a; b]) = \varphi(b) - \varphi(a)$; мера Лебега-Стильтеса - лебегово продолжение m.

\Th m $\sigma$-аддитивная мера.

\Proof m - мера (очевидно); $(a; b] = \cup_{i = 1}^{\infty} (a_i; b_i]$.
Зафиксируем $\varepsilon > 0$; $[c; d] \subseteq (a; b]: |\varphi(a) - \varphi(c)| < \varepsilon; |\varphi(b) - \varphi(d)| < \varepsilon$. Аналогично находим $c_i, d_i$: $(a_i, b_i] \subseteq (c_i, d_i); $ $|\varphi(a_i) - \varphi(c_i)| < \varepsilon / 2^i$; $|\varphi(b_i) - \varphi(d_i)| < \varepsilon / 2^i$

Отсюда $[c; d] \subseteq \cup_{i = 1}^{\infty} (c_i; d_i)$. Тогда по лемме о конечном покрытии (принцип Гейне-Бореля) $\Rightarrow [c; d] \subseteq \cup_{i = 1}^{n} (c_i; d_i)$ $\Rightarrow (c; d] \subseteq \cup_{i = 1}^{n} (c_i; d_i]$

m - мера. Тогда $m((a; b]) - 2\varepsilon \leqslant m((c; d]) \leqslant \sum_{i=1}^{n} m((c_i; d_i]) \leqslant \sum_{i=1}^{n} (m((a_i; b_i]) + \varepsilon / 2^i)$. Итого $m(a; b] \leqslant \sum_{i=1}^\infty (a_i; b_i] + 3 \varepsilon. $

Неравенство в обратную сторону следует из свойств меры, значит, имеет место равенство.\EndProof

\Note (б/д) любая мера на полукольце является мерой Лебега-Стильтеса для некоторого $\varphi$.

\Def Мера Бореля - классическая мера Лебега на $\B_{a, b} = \{ A \cap [a. b] | A \in \B(\R)\}$
\newpage{}

\section{Сигма-конечные меры.}

\textbf{Определение} Пусть $S$ - полукольцо множеств $X$. $X \not\in S, \ m$ - $\sigma-$аддитивная мера на $S$. Если $\exists \{X_i\}_{i=1}^{\infty} \in S \left( X = \bigcup_{i=1}^{\infty}X_i\right)$, то m -  $\sigma-$конечная мера
\\
\\
\textbf{Продолжим меру m}
\\
\\
\textbf{Утверждение}\\
X = $\bigsqcup_{i=1}^{\infty} B_i, B_i \in S$
\\
$\blacktriangle$ Определим $B_i = X_i\textbackslash (\bigcup_{j=1}^{i-1}X_i)$. Любой элмент  R(S) является дизъюнктным объединением элементов полукольца по доказанной теореме, откуда  $B_i = \bigsqcup_{j=1}^{n_i} C_{ij}, C_{ij} \in S \ \blacksquare$
\\
\\
Далее будем считать, что X = $\bigsqcup_{i=1}^{\infty} X_i, X_i \in S$.
\\
Определим $S_i = S \cap X_i$. $S_i$ - это $\sigma-$алгебра с единицей $X_i$. Отсюда делаем вывод, что существует лебегово продолжение m до $\sigma-$аддитивной меры $\mu_i$ на $\sigma-$алгебре $M_i$
\\
\\
\textbf{Определение} $A \subset X$ - измеримо, если $\forall i (A \cap X_i)$ - измерима, то есть $(A \cap X_i) \in M_i$. Тогда можно определить меру:
\\
\\
\textbf{Определение} 
\begin{center}
    $\mu(A) = \sum\limits_{i=1}^{\infty} \mu_i(A \cap X_i)$
\end{center}
\textbf{Определение} $\mu$ - лебегово продолжение $\sigma$-конечной меры на M
\\
\\
\textbf{Замечание} 
\\
$\mu: M \rightarrow [0, +\infty]$, то есть эта мера не является мерой в классическом определении
\\
\\
\textbf{Утверждение}
\\
$\forall A \in S, A\in M$(А - измеримо) и $\mu(A) = m(A) \\ \blacktriangle \\$
Для того чтобы A было измеримым, нужно чтобы $A \cap X_i \in S$ - очевидно, как пересечение лежащих в S множеств.
\\
$\mu(A) = \sum\limits_{i=1}^{\infty} \mu_i(A \cap X_i)$ = (продолжение меры и мера совпадают) $\sum\limits_{i=1}^{\infty} m(A \cap X_i) = m(A)$, так как $\bigsqcup_{i=1}^{\infty} (A \cap X_i) = A \ \blacksquare$
\\
\\
\textbf{Теорема 1} $\mu$ корректно определена, то есть ее определение не зависит от разбиения множества $X$ - \textbf{б\textbackslash д}
\\
\\
\textbf{Теорема 2} Множество $M$ измеримых подмножеств $X$ - $\sigma$-алгебра
\\
$\blacktriangle$
\begin{itemize}
    \item [1.] $X \in M$, так как  X = $\bigsqcup_{i=1}^{\infty} X_i \in M$
     \item [2.] $A, B \in M$, тогда $A \cap X_i \in M_i, B \cap X_i \in M_i$, откуда  $(A \cap X_i) \cap (B \cap X_i) \in M_i \Longrightarrow (A \cap B) \cap X_i \in M_i$
     \item [3.] Для  $\Delta$ - аналогично
     \item [4.] $A_1, A_2,..., A_n, ... \in M$. Тогда $(\bigcup_{i=1}^{\infty} (A_i)) \cap X_j = \bigcup_{i=1}^{\infty} (A_i \cap X_j) \in M_j$, откуда $(\bigcup_{i=1}^{\infty} (A_i)) \in M$ 
\end{itemize}
$\blacksquare$
\\
\\
\textbf{Теорема 3} $\mu$ на M является $\sigma-$аддитивной мерой
\\
\\
$\blacktriangle \ \ A = \bigsqcup_{i=1}^{\infty} A_i \in M$ \\ \\
$\mu(A) = \sum\limits_{i=1}^{\infty} \mu_i(A \cap X_i) = \sum\limits_{i=1}^{\infty} \mu_i (\bigsqcup_{j=1}^{\infty} (A_j \cap X_i)) = \sum\limits_{i=1}^{\infty}\sum\limits_{j=1}^{\infty} \mu_i(A_j \cap X_i) = \sum\limits_{j=1}^{\infty}\sum\limits_{i=1}^{\infty} \mu_i(A_j \cap X_i) = \sum\limits_{j=1}^{\infty}\mu(A_j) \blacksquare$
\section{Неизмеримые множества.}
В этом билете мы будем строить множество Витали
\\
\textbf{Теорема}
\\
Пусть А - измеримое по классической мере Лебега подмножество [0,1]. Тогда существует неизмеримое $B \subset A \\ \blacktriangle$ Введем на $A$ отношение эквивалентности на отрезке [0,1]: $a \sim b$ , если $a-b \in Q$. Используя аксиому выбора можем выбрать в каждом классе эквивалентноси представителя. Положим Е = $\{x_{\alpha}\}$ - множество этих представителей.
\\
Занумеруем все рациональные числа в отрезке [-1, 1]. $\{r_n\} = Q \cap [-1, 1];$ Рассмотрим $E_n - E + r_n$.
\\
Множества $E_n$ не пересекаются. Пусть это не так, тогда $\exists x_{\alpha} + r_{\alpha} = x_{\beta} + r_{\beta}$, тогда $x_{\alpha} \sim x_{\beta}$. Противоречие
\\
Покажем, что $\forall C_n \subset E_n \rightarrow \lambda(C_n) = 0$.
\\
Пусть $\lambda(C_n) = d  > 0, C_m = C_n - r_n + r_m, C_n \subset [0, 1] \Longrightarrow \forall m C_m \in [-1, 2]$. 
\\
$C_m \subset E_m \Longrightarrow \bigsqcup_{m=1}^{\infty}C_m \subset \bigsqcup_{m=1}^{\infty}E_m \leq [-1,2] \Longrightarrow \infty = \sum\limits_{m=1}^{\infty}\lambda(C_m)  \leq 3$. Противоречие
\\
\\
$A = \bigsqcup_{n=1}^{\infty} A\cap E_n$ в силу того, что $E_n$ попарно не пересекаются и их объединение содержит отрезок [0,1], так как $\forall \ x \in [0,1] x = x_{\alpha} + r_n$.
\\
Положим $F_n = A\cap E_n$. Если какое-то множество $F_n$ неизмеримо по Лебегу, то мы уже победили, так как тогда $A$ неизмеримо. Предположим, что все $F_n$ измеримы, тогда в силу того, что $\lambda(A) \neq 0$. Тогда $\exists n: \lambda(F_n) \neq 0$. Противоречие с доказанным \ \ $\blacksquare$

\newpage{}

\section{Измеримые функции. Их свойства. Измеримые функции и предельный переход.}
\textbf{Определение} $(X, M) $ - \textit{измеримое пространство}, где $X$ - единица, $M$ - $\sigma$-алгебра. Элементы $M$ называются \textit{измеримыми множествами}
\\
\\
\textbf{Определение} Функция $f: X \rightarrow \overline{R}$ \textit{измеримая}, если $\forall c \in \overline{R} (f^{-1}((c, +\infty]) \in M)$
\\
\\
\textbf{Лемма} Множества $f^{-1}(+- \infty), f^{-1}(R), f^{-1}<a, b>$ - измеримы
\\
\\
\textbf{Теорема}
\\
Если $f$ - измерима на $(X, M)$, то $\forall B \in B(R)$(борелевская $\sigma$-алгебра) ($f^{-1}(B) \in M$)
\\
$\blacktriangle \\ S = \{A \subset R:\ f^{-1}(A) \in M\}$
\begin{itemize}
    \item $R \in S$
     \item $\forall A, B \in S : \ A\cap B \in S, A\Delta B \in S$ в силу того, что $f^{-1}(A)\cap f^{-1}(B) = f^{-1}(A\cap B)$
     \item $\forall A_1, ..., A_n, ... \ \bigcup_{i=1}^{\infty} A_i \in S$ в силу того, что $f^{-1}(\bigcup_{i=1}^{\infty} A_i) = \bigcup_{i=1}^{\infty} f^{-1}(A_i)$
\end{itemize}
Следовательно, $S -$ $\sigma$-алгебра, содержащая все полуинтервалы, а значит, $B(R) \subset S$, то есть прообраз любого борелевского множества измерим
$\blacksquare$
\\
\\
\textbf{Определение} \textit{Борелевская функция} - это отображение из $G \subset B(R)$ в $R$, для которого верно следующее: $\forall c \in R (f^{-1}((c, +\infty)) \in B(R))$
\\
\\
\textbf{Замечание} Если  $f$ - борелевская функция и $B \in B(R)$, то $f^{-1}(B)\in B(R)$
\\
\textbf{Замечание} Если  $f$ - непрерывная функция, определенная на открытом $G \in R$, то $f$ - борелевская
\\
\\
\textbf{Теорема(о композиции)} Пусть f - измерима и конечна на $(X, M)$. $f(x) \subset G \subset R$($G \in B(R)$) и g - борелевская, тогда $gf$ - измеримая на $(X, M)$
\\
$\blacktriangle (gf)^{-1}(c, +\infty) = f^{-1}(g^{-1}\underset{\in B(R)}{(c, +\infty)})$ - измеримо $\blacksquare$
\\
\textbf{Теорема (об арифметических операциях)}
\\
$f, g: X \rightarrow R $  - измеримые функции, $(X, M)$ - измеримое пространство. Тогда $\forall a,b \in R :$
\begin{itemize}
    \item [1] $af + bп$ - измерима
     \item [2] $fg$ - измерима
     \item [3] $\frac{f}{g}, g \neq 0$ - измерима
\end{itemize}
$\blacktriangle$
\begin{itemize}
    \item [1] $a*f, f + c$ - измеримые функции так как
    \begin{itemize}
        \item $h(y) = ay$
        \item $h(y) = y + c$
    \end{itemize}
    $h(f(x))$ - измеримая по предыдущей теореме
    \\
    $\{x: f(x) < g(x)\} = \bigcup_{i=1}^{\infty}( \{x : f(x) < r_n\} \cap \{x : g(x) > r_n\}$ - измеримое, тогда
    \\
    $\{x: f(x) + g(x) < a\} = \{x: f(x) < a - g(x)\}$ - измеримо
    \item[2] $f*g = \frac{1}{4}((f+g)^2 - (f-g)^2)$ - измерима
    \item[3] $\frac{f}{g} = f *\frac{1}{g}$. Положим $h(y) = \frac{1}{y}$. Если $y \neq 0$, то $h$ - непрерывна на области определения, поэтому если g измерима, то и $\frac{1}{g}$ - измерима. После применяем п2 и получаем, что $\frac{f}{g}$ - измерима
\end{itemize}
$\blacksquare$
\\
\\
\textbf{Теорема} (о переходе к пределу)
\\
$\{f_n\}_{n=1}^{\infty}$ - измеримые функции. $f_n: X \rightarrow \overline{R}$. Тогда
\begin{itemize}
    \item [1] $\underset{n \geq 1}{sup}f_n$ и $\underset{n \geq 1}{inf}f_n$ - измеримые
     \item [2] $\overline{\underset{n\rightarrow\infty}{lim}}(f_n(x))$ и  $\underset{n\rightarrow\infty}{\underline{lim}}(f_n(x))$ - измеримы
     \item [3] (Следствие)  E =$\{x: \exists \underset{n\rightarrow\infty}{lim}(f_n(x)) \} \in M$ \\
     $g(x) = \underset{n\rightarrow\infty}{lim}(f_n(x)), g: E \rightarrow R$ - измеримая функция из $(E, M_E) : M_E = \{A \subset E: A\in M\}$
\end{itemize}
$\blacktriangle$
\begin{itemize}
    \item [1]  $\{ x:\underset{n \geq 1}{sup}(f_n) \leq a\} = \cap_{i=1}^{\infty} \{x:f_n(x)\leq a\} \in M$. Для inf аналогчино 
     \item [2]$\overline{\underset{n\rightarrow\infty}{lim}}(f_n(x))$ = $\underset{n \geq 1}{inf}\ \underset{k \geq n}{sup}(f_k)$
     \\
     $\underset{n\rightarrow\infty}{\underline{lim}}(f_n(x)) = $ $\underset{n \geq 1}{sup}\ \underset{k \geq n}{inf}(f_k)$
     \item [3] $\phi(x) = \overline{\underset{n\rightarrow\infty}{lim}}(f_n(x)),  \\ 
     \psi(x) = \underset{n\rightarrow\infty}{\underline{lim}}(f_n(x)) 
    \\
    E = \{x: \phi(x) = \psi(x)\} = \overline{\{ x: \phi(x) > \psi(x)\} \cup \{x: \psi(x) > \phi(x)\}}$ - измеримо
    \\
    \\
    $\forall x \in E \ \underset{n \rightarrow \infty}{lim}(f_n) = \phi(x)\\ \\ \phi: X\rightarrow\overline{R} \\ \phi|_E :E\rightarrow\overline{R}
    \\
    \\
    \forall A \in B(R) \ \phi^{-1}|_E(A) = \phi^{-1}(A)\cap E$
\end{itemize}
$\blacksquare$
\newpage{}

\section{Множество Кантора и кривая Кантора. Теорема о существовании композиции измеримой от непрерывной, не являющейся измеримой функцией.}

\Idea рассмотрим отрезок $[0, 1]$, выкидываем центральную треть, а потом выкидываем центральную треть из оставшихся третей и так далее.

Формально: пусть $C_0 = [0, 1]$, $C_1 = [0, \frac{1}{3}]\cup[\frac{2}{3}, 1]$, ...: $C_n = \cup [a_n, b_n] \rightarrow C_{n+1} = \cup([a_n, c_n]\cup[d_n, b_n])$, где $c_n = \frac{2a_n+b_n}{3}$ и $d_n = \frac{a_n+2b_n}{3}$.

\Def \textbf{Канторово множество $C$} - пересечение всех $C_n$: $\cap_{n=1}^\infty C_n$.

Пользуясь непрерывностью меры и вложенностью $C_i$, вычислим Лебегову меру:\\
1. $C_0 \supset C_1 \supset C_2 \supset ...$, тогда $\lim\limits_{n\to +\infty} (\frac{2}{3})^n=0$\\
2. Замкнуто как пересечение замкнутых\\
3. Континуально (изоморфно $\mathds{R}$), так как когда мы выкидываем трети от отрезков, мы выкидываем все числа, у которых последовательно первое, второе, третье и далее число в троичной записи соответственно равно единице. А значит, в результате, в множестве кантора лежат все числа, которые записываются нулями и двойками, значит, имеем континуальное количество.

Построим \textbf{канторову лестницу}.\\
Начнём с определения 'заготовки'. Пусть $T$ \textendash\; множество концов отрезков $[a_n, b_n]$, тогда пусть $\overline{\varphi}: T \rightarrow [0, 1]$: \\
1. База: $\overline{\varphi}(0) = 0,\; \overline{\varphi}(1) = 1$ \\
2. $\overline{\varphi}(c) =  \overline{\varphi}(d) = \frac{\varphi(a) + \varphi(b)}{2}$\\
3. $\overline{\varphi}(x)$ не убывает\\
4. $\overline{\varphi}(x)$ принимает все значения вида $\frac{k}{2^n}$.

Доопределим для остальных точек интервала $[0, 1]$:
$\varphi(x) = \sup\limits_{y \leq x,\; y \in T} \; \overline{\varphi}(y)$. \\
1. Неубывающая \\
2. В точках из T: $\overline{\varphi}(x) = \varphi(x)$ (что следует из монотонности вспомогательной функции).

\Th \\
1. $\mathds{B}(\mathds{R}) \subsetneq \mathds{M}$: измеримые по Борелю не совпадают с измеримыми по Лебегу \\
2. $\exists$ непрерывная $f: [0, 1]  \rightarrow  [0, 1]$ и  $\exists$ измеримая по Лебегу $g: g(f(x))$ не измерима по Лебегу \\
Замечание: $f(g(x))$ измерима по теореме о композиции.\\
3. $\exists$ измеримое по Лебегу множество $E$, такое что его прообраз $f^{-1}(E)$ не измерим по Лебегу. 

\Proof Рассмотрим $\psi = \frac{x+\phi(x)}{2}$, непрерывная и монотонная, а значит и биекция. Обозначим обратную $\psi^{-1} = f$. Тогда: \\
1. $\lambda(C_0) = \frac{1}{2}$ \\
2. $\lambda(\psi(\bar{C})) = \frac{1}{2}\lambda(\bar{C}) = \frac{1}{2}$ \\
Имеем, что $\psi$ переводит интервал в интервал $[0,1]$, который лежит в каком-то подмножестве $C`$ с мерой $\frac{1}{2}$, поскольку множество имеет положительную меру Лебега, воспользуемся утверждением о том, что оно содержит неизмеримое подмножество $E`$. Тогда: \\
1. $E$ измеримо, но не Борелевское: $\psi(E) = E`$ (доказали пункты 1 и 3).\\
2. В качестве $g$ выберем индикатор множества $E$, тогда $g$ измерима, но композиция $g(f)$ это индикатор незимеримого $E`$, а приведен пример для пункта 2. \EndProof


\section{Сходимость по мере и почти всюду. Их свойства (критерий Коши сходимости по мере, арифметические, связь сходимостей, Теорема Рисса).}
$(X, M, \mu)$ - измеримое пространство с мерой $\mu$, $\mu$ - $\sigma$-аддитивная, $\sigma$-конечная,

\Def Последовательность измеримых функций $\{ f_n \}$ \textbf{сходится по мере} к измеримой функции f ($f_n \convme f$): $\forall \varepsilon$ $lim_{n \to \infty}$ $\mu \{x \in X | |f_n(x) - f(x)| \geqslant \varepsilon \} = 0$

\Def Последовательность измеримых функций $\{ f_n \}$ \textbf{сходится почти всюду} к измеримой функции f ($f_n \convae f$): $\mu \{ x \in X | f_n \nrightarrow f(x)\} = 0$

\Note \textit{в теории вероятностей первое - сходимость по вероятности, второе - сходимость почти наверное}

\Note Здесь и далее ВСЕ функции измеримые.

\vspace{5pt}

\textbf{Теорема.} $f_n \convae f$; $f_n \convae g$ $\Rightarrow$ $f \eqae g$

\Proof Наследуется из свойств предела в определении сходимости почти всюду. \EndProof

\textbf{Теорема.} \textbf{Арифметические свойства сходимости п.в.} $f_n \convae f$; $g_n \convae g$ Тогда:

1) $\forall a, b \in \R$ $(af_n + bg_n) \convae (af + bg)$

2) $h: \R \to \R$ непрерывна, тогда $h(f_n(x)) \convae h(f(x))$

3)$f_n g_n \convae fg$

4) $g_n, g \neq 0$ ни в одной точке, тогда $\frac{f_n}{g_n} \convae \frac{f}{g}$

\Proof Пункт 2): Непрерывность по Гейне: $x_n \to x \Rightarrow h(x_n) \to h(x)$ $\Rightarrow$ $\{x \in X: h(f_n(x)) \nrightarrow h(f(x)) \} \subseteq \{x \in X: f_n(x) \nrightarrow f(x)\}$; аналогично остальные пункты наследуются из определения предела. \EndProof

\vspace{5pt}

\textbf{Теорема.} \textbf{критерий сходимости почти всюду}

Если $\mu(X) < \infty$, то $f_n \convae f$ $\Leftrightarrow$ $\forall \varepsilon$ $\lim_{n \to \infty}$ $\mu (\cup_{k=n}^{\infty} \{ x: \abs{f_k(x) - f(x)} \geqslant \varepsilon \}) = 0$

\Proof
$f_n(x) \nrightarrow f(x)$ $\Rightarrow$ $\exists m \forall n \exists k > n: \abs{f_k(x) - f(x)} \geqslant \frac{1}{m}$ 

$\{x: f_n(x) \nrightarrow f(x)\} = \cup_{m=1}^{\infty}\cap_{n=1}^{\infty}\cup_{k=n}^{\infty} \{ x: \abs{f_k(x) - f(x)} \geqslant \frac{1}{m} \} = C$; 

$\mu(C) = 0$ $\Leftrightarrow$ $\forall m$ $\mu(\cap_{n=1}^{\infty}\cup_{k=n}^{\infty} \{ x: \abs{f_k(x) - f(x)} \geqslant \frac{1}{m} \}) = 0$ (в одну сторону очевидно, в другую - мера объединения больше или равна мере каждого из элементов)

События вложены друг в друга, значит, можно перейти к пределу: 

$\forall m$ $\lim_{n \to \infty}$ $\mu (\cup_{k=n}^{\infty} \{ x: \abs{f_k(x) - f(x)} \geqslant \frac{1}{m} \}) = 0$ ($\sigma$-аддитивность, конечность меры)
\EndProof

\vspace{5pt}

\textbf{Теорема.} \textbf{связь сходимостей} Если $\mu(X) < \infty$, то $f_n \convae f$ $\Rightarrow$ $f_n \convme f$

\Proof
Кр-ий сх. п.в.: $\forall \varepsilon$ $\mu (\cup_{k=n}^{\infty} \{ x: \abs{f_k(x) - f(x)} \geqslant \varepsilon \}) \geqslant \mu (\{ x: \abs{f_n(x) - f(x)} \geqslant \varepsilon \})$; 
\\ $\lim_{n \to \infty}$ $\mu (\cup_{k=n}^{\infty} \{ x: \abs{f_k(x) - f(x)} \geqslant \varepsilon \}) = 0$ $\Rightarrow$ $\lim_{n \to \infty}$ $\mu (\{ x: \abs{f_k(x) - f(x)} \geqslant \varepsilon \}) = 0$
\EndProof

\Note Обратное неверно, пример Рисса: $\forall n \in \N, k \in [1; n]$; $\varphi_{n, k} = I_{[(k-1)/n; k/n]}(x)$

\vspace{5pt}

\textbf{Теорема.} \textbf{теорема Рисса}: Пусть $(X, M, \mu)$ - $\sigma$-конечное измеримое пространство. Тогда $f_n \convme f$ $\Rightarrow$ $\exists n_k$: $f_{n_k} \convae f$ (можно выбрать подпоследовательность).

\Proof
1) Пусть мера конечная. $ \forall \varepsilon$ $lim_{n \to \infty}$ $\mu \{x \in X | |f_n(x) - f(x)| \geqslant \varepsilon \} = 0$ $\Rightarrow$ $\forall k \exists n_k$: $\mu \{x \in X | |f_{n_k}(x) - f(x)| \geqslant \frac{1}{k} \} \leqslant \frac{1}{2^k}$; Считаем $\varepsilon > 1/k$.
Хотим доказать, что $f_{n_k} \convae f$: $\forall \varepsilon$  $\mu (\cup_{i=k}^{\infty} \{ x: \abs{f_{n_i}(x) - f(x)} \geqslant \varepsilon \}) \leqslant \mu (\cup_{i=k}^{\infty} \{ x: \abs{f_{n_i}(x) - f(x)} \geqslant 1/i \}) \leqslant \sum_{i=k}^{\infty} \mu (\{ x: \abs{f_{n_i}(x) - f(x)} \geqslant 1/i \}) \leqslant \sum_{i=k}^{\infty} \frac{1}{2^i} = \frac{1}{2^{k-1}}$. Таким образом, по т. о 2х миллиционерах изначальная функция стремится к нулю.

2) Мера $\sigma$-конечная. $X = \cup_{i=1}^{\infty} X_i$, $\mu(X_i) < \infty$.

$X_1$: $f_n \convme f$ $\Rightarrow$ $\exists n_{1, k}: f_{n_{1, k}} \convae f$

$X_2$: $f_{n_{1, k}} \convme f$ $\Rightarrow$ $\exists n_{2, k}: f_{n_{2, k}} \convae f$

и.т.д. Тогда имеем $f_{n_{m, k}} \convae f$ на $X_1, X_2, \dots, X_m$. Берём диагональ: $g_k = f_{n_{k, k}}$ - искомая подпоследовательность, т.к. $\{ x: g_k \nrightarrow f = \cup_{m=1}^{\infty} \{x \in X_m: g_k \nrightarrow f\}$, а меры таких мн-в 0. \EndProof

\vspace{6pt}

\textbf{Теорема.} \textbf{следствие: критерий сходимости по мере}

Если $\mu(X) < \infty$, то $f_n \convme f$ $\Leftrightarrow$ $\forall n_k \exists n_{k_m}: f_{n_{k_m}} \convae f$

\Proof
В одну сторону: если посл-ть сходится по мере, то и подпосл-ть сходится по мере, а значит, по т. Рисса можно выбрать искомую подпосл-ть.

В другую: от противного. Пусть $f_n \nrightarrow f$ по мере. Тогда $\exists \varepsilon > 0 \exists \delta > 0 \exists n_k \mu(x \in X: \abs{f_{n_k}(x) - f(x)} \geqslant \varepsilon) \geqslant \delta$; с другой стороны, выделим из этой последовательности $n_k$ подпоследовательность $n_{k_m}$: $f_{n_{k_m}} \convae f$ $\Rightarrow$ $f_{n_{k_m}} \convme f$, а это противоречие.
\EndProof

\vspace{5pt}

\textbf{Теорема.} $f_n \convme f$; $f_n \convme g$ $\Rightarrow$ $f \eqae g$

\Proof 
Неравенство треугольника: $\abs{f_n - f} + \abs{f_n - g} \geqslant \abs{f - g} \geqslant \varepsilon$ $\Rightarrow$ $\forall \varepsilon$ $\{x: \abs{f(x) - g(x)} \geqslant \varepsilon\} \subseteq \{x: \abs{f(x) - f_n(x)} \geqslant \varepsilon/2\} \cup \{x: \abs{f_n(x) - g(x)} \geqslant \varepsilon/2\}$ получается неравенство по мере, причём для правых множеств мера стремится к нулю, значит, по т. о 2-х миллицонерах, левая последовательность так же стремится к нулю; однако она не зависит от n (это константа), значит, $\forall \varepsilon \mu(\{x: \abs{f(x) - g(x)} \geqslant \varepsilon\}) = 0$; $\mu(x: f(x) \neq g(x)) = \mu(x: \abs{f(x) - g(x)} > 0) = \mu(\cup_{m=1}^{\infty} \{x: \abs{f(x) - g(x)} \geqslant 1/m\}) \leqslant \sum_{m=1}^{\infty} \mu( \{x: \abs{f(x) - g(x)} \geqslant 1/m\}) = 0$
\EndProof

\vspace{5pt}

\textbf{Теорема.} \textbf{арифметические св-ва сходимости по мере} $f_n \convme f$; $g_n \convme g$, $\mu(X) < \infty$ Тогда:

1) $\forall a, b \in \R$ $(af_n + bg_n) \convme (af + bg)$

2) $h: \R \to \R$ непрерывна, тогда $h(f_n(x)) \convme h(f(x))$

3)$f_n g_n \convme fg$

4) $g_n, g \neq 0$ ни в одной точке, тогда $\frac{f_n}{g_n} \convme \frac{f}{g}$

\Proof
Докажем пункт 3 (остальные аналогично). Воспользуемся критерий сходимости по мере: $\forall n_k \exists n_{k_m}: f_{n_{k_m}}g_{n_{k_m}} \convae fg$; Знаем, что $f_{n_k} \convme f$, $g_{n_k} \convme g$; воспользуемся теоремой Рисса сначала для $f_{n_k}$, получу $n_{k_l}$, и уже из него выберу такую подпоследовательность $n_{k_{l_m}}$, чтобы и $g_{...}$ сходилось почти всюду. Применим аналогичную теорему из сходимости почти всюду.
\EndProof

\vspace{5pt}

\Def $\{f_n\}$ фундаментальна почти всюду, если $\mu(x: \{f_n(x)\}$ не фундаментальна $) = 0$

\textbf{Теорема.} \textbf{критерий Коши для сходимости почти всюду}
$\{f_n\}$ сходится почти всюду $\Leftrightarrow$ $\{f_n\}$ фундаментальна почти всюду.
\Proof
Прямое наследование свойств обычных пределов: в тех точках, где она фундаментальна, она сходится почти всюду, и наоборот.

$\Rightarrow$: $\mu\{f_n\}$ не фунд. $) \leqslant \mu(x: f_n(x) \nrightarrow f(x)) = 0$

$\Leftarrow$: $\{f_n\}$ фунд. п.в.; $E = \{x : \{f_n(x)\}$ фундаментальна $\}$. Тогда $\forall x \in E \exists \lim f_n(x) = f(x)$, f(x) измерима как предел измеримых функций; для x не из Е положим $f(x) = 0$. Тогда $f_n(x) \convae f(x)$
\EndProof

\vspace{5pt}

\Def $\{f_n\}$ фундаментальна по мере, если 

$\forall \varepsilon > 0 \lim_{n, m \to \infty}\mu(x \in X: \abs{f_n(x) - f_m(x)} \geqslant \varepsilon) = 0$

\textbf{Теорема.} \textbf{критерий Коши для сходимости по мере}

$\{f_n\}$ сходится по мере $\Leftrightarrow$ $\{f_n\}$ фундаментальна по мере.

\Proof
$\Rightarrow$: $f_n \convme f$: $\mu(x \in X: \abs{f_n(x) - f_m(x)} \geqslant \varepsilon) \leqslant \mu(x \in X: \abs{f_n(x) - f(x)} \geqslant \varepsilon/2) + \mu(x \in X: \abs{f_m(x) - f(x)} \geqslant \varepsilon/2)$, каждая из мер справа $\to 0$, значит, по т. о 2 миллиционерах левая часть $\to 0$.

$\Leftarrow$: План: 1) найти $n_k$: $f_{n_k}$ фунд. п.в. Тогда 2) $\exists f: f_{n_k} \convae f$ 3) Докажем, что $f_{n_k} \convme f$ 4) докажем $f_n \convme f$

1) Зададим явно $n_k$: $\mu(x \in X: |f_{n_{k+1}} - f_{n_k}| \geqslant 2^{-k}) \leqslant 2^{-k}$. Существование следует из фундаментальности по мере.

$A_k = \{x \in X: |f_{n_{k+1}} - f_{n_k}| \geqslant 2^{-k}\}$; $A = \cap_{m=1}^{\infty} \cup_{k=m}^{\infty}A_k$; $\mu(A) = 0$, т.к. $\mu(A) \leqslant \mu(\cup_{k=m}^{\infty}A_k) \leqslant \sum_{k=m}^{\infty} \mu(A_k) \leqslant \frac{1}{2^{m-1}}$, а это $\to 0$.

$x \in A$: $\forall m \exists k \geqslant m: \abs{f_{n_{k+1}} - f_{n_k}} \geqslant 2^{-k}$

Отрицание: $\exists m \forall k \geqslant m: \abs{f_{n_{k+1}} - f_{n_k}} < 2^{-k}$ - выполнено почти всюду, значит, $f_{n_k}(x)$ фундаментальна (через неравенство треугольника) на $X\backslash A$

2) $f_{n_{k+1}}$ фундаментально, значит, зададим $f(x)$ как предел, если $x \in A$, иначе 0; $f_{n_k} \convae f$.

3) Выкладка: $\abs{f_{n_i}(x) - f(x)} = \abs{\sum_{k=i}^{\infty}f_{n_k}(x) - f_{n_{k+1}}(x)} \leqslant \sum_{k=i}^{\infty} \abs{f_{n_k}(x) - f_{n_{k+1}}(x)} \leqslant 2^{-i+1}$ (1), $x \not\in A$, а последнее неравенство из предположения, что на соответствующие модули есть оценка виде $<2^{-k}$.

$\forall \varepsilon > 0$ $\mu (x \in X: \abs{f_{n_i}(x) - f(x)} \geqslant \varepsilon)$; для больших i точно $\varepsilon \geqslant 2^{-i+1}$, а значит, на тех точках, где это выполняется, не выполняется неравенство (1), значит, $\mu (x \in X: \abs{f_{n_i}(x) - f(x)} \geqslant \varepsilon) \leqslant \sum_{k=i}^{\infty} \mu(x \in \overline{A}: \abs{f_{n_k} - f_{n_{k+1}}} \geqslant 2^{-k}) \leqslant \sum_{k=i}^{\infty} 2^{-k} = \frac{1}{2^{-i+1}} \to 0, i \to \infty$; доказали сходимость по мере: $f_n \convme f$

4) По определению и т. о 2 миллиционерах: $\forall \varepsilon > 0$ $\mu(x \in X: \abs{f_n(x) - f(x)} \geqslant \varepsilon) \leqslant \mu(x \in X: \abs{f_n(x) - f_{n_k}(x)} \geqslant \varepsilon/2) + \mu(x \in X: \abs{f(x) - f_{n_k}(x)} \geqslant \varepsilon/2)$; 1-ая мера стремится к нулю из фундаментальности по мере (если $n, n_k$ устремить к бесконечности), а вторая - так как последовательность сходится по мере. Значит, вся левая часть стремится к нулю, значит, $f_n \convme f$.
\EndProof

\section{Теорема Егорова.}
\Def $f_n$ сходится равномерно к $f$ ($f_n \rightrightarrows f$) -  $\forall \delta >0 \exists N: \forall n > N \forall x \in E_{\varepsilon} \abs{f_n(x) - f(x)} < \delta$.

\Th \textbf{теорема Егорова}

$\mu(X) < \infty, f_n \convae f$. Тогда $\forall \varepsilon > 0$ $\exists E_{\varepsilon}: \mu(X \backslash E_{\varepsilon}) < \varepsilon$ $f_n \rightrightarrows f$ на $E_{\varepsilon}$

\Proof
$f_n \convae f$ $\Leftrightarrow$ $\forall m > 0$ $\mu( \cup_{k=n}^{\infty}\{x \in X: |f_k(x) - f(x)| \geqslant \frac{1}{m}) \to 0, n \to \infty$

$\varepsilon > 0$: $\exists n_m$ $\mu(G_m) < \frac{\varepsilon}{2^m}$ ($\varepsilon$ из условия теоремы)

$G_m = \cup_{k=n_m}^{\infty}\{x \in X: |f_k(x) - f(x)| \geqslant \frac{1}{m}\}$; $E_{\varepsilon} = X \backslash (\cup_{m=1}^{\infty} G_m)$ (заметим, что множества не вложены)

$\mu(\cup G_m) \leqslant \sum \mu(G_m) < \sum \frac{\epsilon}{2^m} = \varepsilon$; 1-ое условие выполнено.

Докажем, что на $E_{\varepsilon}$ есть равномерная сходимость:

$x \in \cup G_m$: $\exists m \exists k > n_m$ $\abs{f_k(x) - f(x)} \geqslant \frac{1}{m}$. Тогда отрицание:

$x \in E_{\varepsilon}$ $\forall m \forall k > n_m$ $\abs{f_k(x) - f(x)} < \frac{1}{m}$; верно $\forall m \in \N$, значит, заменим на произвольное положительное число: $\forall \delta > 0 \exists n_m \forall k > n_m$ $\abs{f_k(x) - f(x)} < \delta$; причём заметим, что $n_m$ не зависит от точки х - это и есть равномерная сходимость на $E_{\varepsilon}$
\EndProof
\newpage{}

\section{Интеграл Лебега для конечно-простых функций и его свойства. Определение интеграла Лебега в общем случае. Основные свойства интеграла Лебега.}
Рассмотрим $\sigma$-конечное пространство $(X, M, \mu)$.

\Def Функция $f(x)$ называется простой, если $f(x) = \sum_{k = 1}^{n} c_k\; I_{E_k(x)}$, причём $E_k \in M$, $E_k \bigcap E_i = \varnothing$. Отметим, что $c_k \neq 0 \Rightarrow \mu(E_k) < +\infty$. \\

\Note \\
1. принимает конечное множество значений.\\
2. простая функция обязана быть измеримой, как сумма измеримых.\\
3. из конечности меры понимаем, что носитель должен иметь конечную меру.\\

\Note если $c_1< c_2< ... <c_n$ и $\sqcup  E_k = X$, то $f(x) = \sum_{k = 1}^{n} c_k\; I_{E_k(x)}$ назовём каноническим видом.

\Def Интеграл лебега от простой функции
$\int_{X}^{} f(x) d \mu(x) = \sum_{k = 1}^{n} c_k \mu(E_k)$ \\
\Note $0\cdot\infty = 0$

\Lemma Интеграл Лебега от простой функции не зависит от её представления.\\

Свойства Интеграла Лебега от простой функции :\\
1. $\forall a, b \in \R: af+bg$ ~--- простая, причём  $\int\limits_X a f(x) + b(g(x) d \mu = a  \int\limits_X  f(x) d \mu +b  \int\limits_X  g(x) d \mu $ \\
2. $\forall x \; f(x) \geq 0 \Rightarrow \int\limits_x f(x) d \mu \geq 0$ \\
3. $\forall x \; f(x) \geq g(x) \Rightarrow \int\limits_x f(x) d \mu \geq  \int\limits_x  g(x)  d \mu$ \\
4. $|\int\limits_x f(x) d \mu| \leq \int\limits_x |f(x)| d \mu$\\
5. Аддитивность по областям интегрирования\\
\Proof
1. Рассмотрим f и g как сумму индикаторов, тогда имеем, что $(af+bg)(x) = \sum_k \sum_j (ac_k+ bd_j) I_{E_k \cap D_j}$, тогда распишем по определению интеграла Лебега:
\[\int (af+bg)(x) d\mu = \sum_k \sum_j (ac_k+ bd_j) \mu(E_k \cap D_j)\]
И воспользуемся линейностью суммы.
2. Слагаемые суммы неотрицательны
3. Переходим к неотрицательной функции $(f-g)(x)$ и обращаемся к первым двум пунктам
4. Неравенство треугольника
5. Очевидно в силу измеримости подмножеств, по которым интегрируем.
\EndProof 

\Def Интеграл Лебега от неотрицательной функции 
Пусть $f(x) \geq 0$ и эта функция измерима. Тогда рассмотрим множество $Q_f = \{ h(x): h - \text{простая и } \forall x \; 0 \leq h(x) \leq f(x) \}$
\[\int_X f(x) d\mu = sup_{h \in Q_f }\int h(x) d \mu \geq 0\]Если супремум конечен, то будем называть такую функцию интегрируемой по Лебегу.
\Def Пусть $f$ измерима, тогда она представима в виде разности $f = f^+ - f^-$, где $f^+ = \max(0, f(x)), \; f^- = \max(0, -f(x))$:
\[\int_X f(x) d\mu = \int_X f^+(x) d\mu -  \int_X f^-(x) d\mu\]
\textbf{Теорема.} Линейности интеграла Лебега для неотрицательной функции. $\forall a, b \in \R+$, f и g неотрицательные измеримые, тогда:
\[\int\limits_X a f(x) + bg(x) d \mu = a  \int\limits_X  f(x) d \mu +b  \int\limits_X  g(x) d \mu\]
\Proof
По лемме о приближении функции простыми, строим $\{f_n\}$ и $\{g_n\}$, тогда $a f_n + b g_n\uparrow a f + b g$, причем левая часть это неотрицательная простая функция. Воспользуемся свойством предела:
\[\int\limits_X a f(x) + bg(x) d \mu = a  \lim\limits_{n\rightarrow \infty} \int\limits_X f_n(x) d \mu +b  \lim\limits_{n\rightarrow \infty} \int\limits_X g_n(x) d \mu = a   \int\limits_X f(x) d \mu +b  \int\limits_X g(x) d \mu \]\EndProof


\textbf{Теорема.} Аддитивность интеграла Лебега для неотрицательных функций.
Пусть $f$ измерима и неотрицательна, $A, B, C \in M$ и $C = A \sqcup B\; (A\cap B = \varnothing)$, тогда
\[\int\limits_C  f(x)  d \mu =   \int\limits_A  f(x) d \mu +  \int\limits_B  f(x)  d \mu\]
\Proof
Вспомним, что $\int\limits_E  f(x)  d \mu =   \int\limits_X  f(x) I_E d \mu$, тогда $f I_C = f I_A + f I_B$ и мы свели задачу к предыдущей теореме. \EndProof

\textbf{Теорема.} Свойства, связанные с нулевой мерой.\\
1. Если $\mu(x) = 0$, $f$ измерима, тогда функция всегда интегрируема и её интеграл по пространству равен нулю.\\
2. Пусть $f$  и $g$ равны почти всюду, тогда их интегралы по $X$ равны.\\
3.  Пусть $f$ интегрируема, тогда $\mu(\{x: f(x) = \pm \infty\}) = 0$.\\

\textbf{Теорема.} Линейность интеграла Лебега в общем случае.
Пусть $f, g \in I(x)$, тогда\\
1.  $f+g \in I(x)$ \\
2. $\int\limits_X a f(x) + bg(x) d \mu = a  \int\limits_X  f(x) d \mu +b  \int\limits_X  g(x) d \mu$

\Proof
Пусть $f$ неотрицательна и $g$ неположительна. Пусть $X = E_1 \sqcup E_2$, где  $E_1 = \{x: f(x) + g(x) \geq 0\}$ и $E_2 = \{x: f(x) + g(x) < 0\}$. Значит, $\int\limits_{E_1}  f(x)  d \mu =   \int\limits_{E_1}  f  + g \; d \mu +  \int\limits_{E_1}  (-g) d\mu$, значит $\int\limits_{E_1}  f +g \; d \mu =   \int\limits_{E_1}  f  d \mu +  \int\limits_{E_1} g(x) d\mu$. Аналогичное равенство запишем и для $E_2$. Сложив данные неравенства, получаем:
 \[\int\limits_{E_1}  (f+g)  d \mu + \int\limits_{E_2}  (f+g)  d \mu =   \int\limits_{E_1}  f\; d \mu  + \int\limits_{E_1}  g \; d \mu+ \int\limits_{E_2}  f\; d \mu  + \int\limits_{E_2}  g \; d \mu =  \int\limits_{X}  f\; d \mu  + \int\limits_{X}  g \; d \mu \] \EndProof

\textbf{Теорема.} Если функция интегрируема, то её модуль тоже интегрируем, а так же 
\[|\int\limits_{X}  f(x)  d \mu| \leq   \int\limits_{X}  f  (x)\; d \mu\]
\textbf{Теорема.} Монотонность интеграла Лебега. Если обе функции интегрируемы и $f \leq g  \Rightarrow \int\limits_{X}  f(x)  d \mu \leq   \int\limits_{X} g (x)\; d \mu $
\Proof
Воспользуемся тем, что  $h = g - f \geq 0$ и $|\int\limits_{X}  f(x)  d \mu| \leq  C\cdot \mu(X)$, если $f$ интегрируема и меньше некоего $C$.\EndProof

























\newpage{}

\section{Теоремы о предельном переходе под знаком интеграла Лебега.}
\textbf{Теорема} Пусть $\ g_n(x), g$ - простые функции. Последовательность$\ g_n(x)$ неубывающая неотрицательная (поточечно) на $E \in M$ и $\underset{n\rightarrow \infty}{lim}(g_n) \geq g(x)$. Тогда 
\begin{center}
    $$\underset{n\rightarrow \infty}{lim}\int_E g_n \ d\mu \geq \int_E g \ d\mu$$
\end{center}
$\blacktriangle $
Если $$\underset{n\rightarrow \infty}{lim}\int_E g_n \ d\mu = +\infty$$то все сразу выпoлняется, так как конечен интеграл $$\int_E g \ d\mu$$
\\
Пусть  $$\underset{n\rightarrow \infty}{lim}\int_E g_n \ d\mu <+\infty$$
\\
 $g(x) = \sum\limits_k c_kI_{E_k}(x), c_i > 0$ Определим $\forall \varepsilon > 0 \ F_n = \{x\in E: g_n \leq g(x) - \varepsilon \} \Longrightarrow F_i \supset F_{i+1}, \  \bigcap_{n=1}^{\infty} F_n = \emptyset$. Определим $F = \bigsqcup E_k \\ \mu(F) < \infty$, отсюда по непрерывности меры $\underset{n\rightarrow \infty}{lim} \ \mu(F_n) = 0$
 $$\int_E g \ d\mu = \int_{F_n} g \ d\mu + \int_{F\textbackslash F_n}g \ d\mu \ \leq \ c_m\mu(F_n) + \int_{F\textbackslash F_n}(g_n + \varepsilon) \ d\mu \ \leq \ c_m\mu(F_n) + \varepsilon\int_{F\textbackslash F_n} 1 \ d \mu + \int_{F\textbackslash F_n} g_n \ d \mu$$
 $c_m\mu(F_n) \rightarrow 0$, откуда переходя к пределу $$\int_E g \ d\mu \leq \underset{n\rightarrow \infty}{lim} \int_E g_n \ d\mu \blacksquare$$
 \\
 \textbf{Утверждение} Пусть $\{g_n\}$ - неубывающая неотрицательная последовательность простых функций. $E \in M, \ g(x) = \underset{n\rightarrow \infty}{lim}(g_n(x)) \ \forall x \in E$. Тогда 
   $$\underset{n\rightarrow \infty}{lim}\int_E g_n \ d\mu \ = \ \int_E g \ d\mu$$
 $\blacktriangle $ $$\\ \int_E g \ d\mu \ = \ \underset{h \in Q_g}{sup} \int_E h \ d\mu$$ \\  $$\underset{n\rightarrow \infty}{lim}(g_n(x)) \geq h(x)$$ \\$$ \int_E h \ d\mu \leq \underset{n\rightarrow \infty}{lim} \int_E g_n \ d\mu $$ Переходя к супремуму в левой части \\ $$ \int_E g \ d\mu \leq \underset{n\rightarrow \infty}{lim} \int_E g_n \ d\mu $$
 В силу того, что $g_n \in Q_g$, очевидно выполняется неравенство в обратную сторону (g - супремум) $\blacksquare$
 \\
 \textbf{Утверждение}
Для любой неотрицательной измеримой функции $f$ и $E \in M$ существует неубывающая последовательность $\{f_n\}$ неотрицательных простых функций, которая поточечно сходится к $f$ на Е.
\\
$\blacktriangle$ Пусть $E = \bigsqcup_{k=1}^{\infty}E_k, \ \mu(E_k) < \infty$
\begin{equation*}
f_m(x) = 
 \begin{cases}
   0 & x \in \bigsqcup_{k = m+1}^{\infty} E_k\\
   2^m & x \in \bigsqcup_{k = 1}^{m} E_k, f(x) \geq 2^m \\
   \frac{k-1}{2^m} & f(x) \in [\frac{k-1}{2^m}, \frac{k}{2^m}), 1 \leq k \leq 2^m
   
 \end{cases}
\end{equation*}
\textbf{Теорема Леви} Пусть $\{f_n\}$ - неубывающая последовательность неотрицательных измеримых функций, $f = \underset{n\rightarrow \infty}{lim} \ f_n$. Тогда
$$\int_E f \ d\mu = \underset{n\rightarrow \infty}{lim} \int_E f_n \ d\mu $$
$\blacktriangle$ Рассмотрим  $g_n = f_n - f_{n-1}, g_1 = f_1$. Применяя предыдущее утверждение для неотрицательной измеримой функции $g_n$ получим, что $\exists \psi_{mn}$ - последовательность простых неотрицательных функций, сходящаяся к $g_n$.
Определим $F_m = \sum\limits_{n=1}^{m} \psi_{mn}$. Заметим, что $F_{m+1} - F_m = \sum\limits_{n=1}^{m+1}\psi_{m+1 n} - \sum\limits_{n=1}^{m}\psi_{mn} = \sum\limits_{n=1}^{m+1}(\psi_{m+1 n} - \psi_{m n}) + \psi_{m+1 m+1} \geq 0$\\
$F_m \leq \sum\limits_{n=1}^{m}g_n = f_m \leq f$\\
Заметим, что в силу этого равенства $F_m = \sum\limits_{n=1}^{m} \psi_{mn}$ можно зафиксировать достаточно большое N и получить  оценку $\underset{m\rightarrow \infty}{lim} F_m \geq \underset{m\rightarrow \infty}{lim}\sum\limits_{n=1}^{N} \psi_{mn} = \sum\limits_{n=1}^{N} \underset{m\rightarrow \infty}{lim} \psi_{mn} = \sum\limits_{n=1}^{N} g_n = f_N$\\
Следовательно $\underset{m\rightarrow \infty}{lim} F_m \geq f(x)$ (но $F_m \leq f$, откуда $\underset{m\rightarrow \infty}{lim} F_m = f(x)$ )\\
Так как $F_m$ - простые функции
$$\int_E f \ d\mu = \underset{m\rightarrow \infty}{lim} \int_E F_m \ d\mu $$
\\
С другой стороны для любого m $$\int_E F_m \ d\mu \leq \int_E f_m \ d\mu \leq \int_E f \ d\mu \ \blacksquare$$
\textbf{Следствие} Пусть $\{f_n\}$ - неубывающая последовательность интегрируемых на Е функций,  $$\int_E f_n \ d\mu\leq K $$ Тогда $\underset{n\rightarrow \infty}{lim} f_n = f$ - интегрируема на E и $$\int_E f \ d\mu = \underset{n\rightarrow \infty}{lim} \int_E f_n \ d\mu $$
\\
\textbf{Лемма Фату}
Пусть $\{f_n\}$ - последовательность неотрицательных измеримых функций,  $f_n \convae f$ - измерима, тогда 
$$\int_E f \ d\mu \leq \underset{n\rightarrow \infty}{\underline{lim}} \int_E f_n$$
$\blacktriangle$ Пусть $\phi_n = \underset{k\geq n}{inf} f_k$, $\phi_n$ - возрастающая последовательность. Заметим, что $\phi_n \leq f_n$ и $\underset{n\rightarrow \infty}{lim} \phi_n = f$ на $E_1 = \{x\in E: f_n \rightarrow f\}$
$$\int_E f \ d\mu  =\ \int_{E_1} f \ d\mu = \underset{n\rightarrow \infty}{lim} \int_{E_1} \phi_n \ d\mu = \underset{n\rightarrow \infty}{\underline{lim}} \int_{E} \phi_n \ d\mu \leq \underset{n\rightarrow \infty}{\underline{lim}} \int_{E} f_n \ d\mu $$
$\blacksquare$\\
\textbf{Теорема Лебега} Пусть $\{f_n\}$ - последовательность измеримых функций, $f_n \convae f$ - измерима, F интегрируема на Е и $|f_n| \leq F $ на Е. Тогда f интегрируема на Е и $$\int_E f \ d\mu = \underset{n\rightarrow \infty}{lim} \int_E f_n \ d\mu = \underset{n\rightarrow \infty}{\underline{lim}}$$
$\blacktriangle$ Заметим, что все $f_n$ интегрируемы, так как ограничены по модулю интегрируемой функцией. Определим $\phi_n = F + f_n, \ \psi_n = F - f_n$
$$\int_E F\ d\mu - \int_E f\ d\mu = \int_E (F - f)\ d\mu \leq \underset{n\rightarrow \infty}{\underline{lim}} \int_E (F - f_n) \ d\mu \  =  \int_E F \ d\mu \ - \underset{n\rightarrow \infty}{\overline{lim}} \int_E f_n\ d\mu \ $$
$$\int_E F\ d\mu + \int_E f\ d\mu \leq \int_E F \ d\mu \ + \underset{n\rightarrow \infty}{\underline{lim}} \int_E f_n \ d\mu \ $$
$$\underset{n\rightarrow \infty}{\overline{lim}} \int_E f_n \ d\mu \leq \int_E f\ d\mu \leq \underset{n\rightarrow \infty}{\underline{lim}} \int_E f_n \ d\mu \Longrightarrow \underset{n\rightarrow \infty}{lim} \int_E f_n \ d\mu = \int_E f\ d\mu$$
$\blacksquare$\\ 
\textbf{Следствие}
Теорема Лебега верна для сходимости по мере.
\\
$\blacktriangle$
Предположим, что теорема Лебега неверна, т. е. существует подпоследовательность интегралов, которая не сходится к интегралу предела. По теореме Рисса у
соответствующей подпоследовательности функций существует подпоследовательность, сходящаяся к пределу всей последовательности функций почти всюду, значит, по теореме Лебега интегралы подпоследовательности сходятся к интегралу предела — противоречие.
$\blacksquare$
\newpage{}