\section{Основные понятия, простейшие типы дифференциальных уравнений.}

\subsection{Основные понятия. Простейшие типы уравнений первого порядка: уравнения с разделяющимися переменными, однородные, линейные.}
Рассмотрим функцию $y(x)$, определённую вместе с $n$ производными на промежутке $I$. \newline Также рассмотрим функцию $F(x,y,p_1, \ldots,p_n)$, определённую и непрерывную на некотором $\Omega\subset \R^{n+2}$

\Def Уравнение вида 
\setcounter{equation}{0}
\begin{equation}\label{eq1} 
    F(x, y, y', \ldots, y^{(n)})=0
\end{equation}
называется \textit{дифференциальным уравнением $n$-го порядка}.

\Def Функция $\varphi(x)$, определённая на $I$ вместе со своими $n$ производными,
называется решением уравнения (\ref{eq1}), если:
\begin{enumerate}
    \item $\varphi$ и все её $n$ производных непрерывны на $I$.
    \item $\forall x \in I \quad (x, \varphi(x), \varphi'(x), \ldots, \varphi^{n}(x)) \in \Omega$
    \item $\forall x \in I \quad F(x, \varphi(x), \varphi'(x), \ldots, \varphi^{n}(x)) = 0$
\end{enumerate}

\Def \textit{Обыкновенным дифференциальным уравнением I-го порядка} называется уравнение вида $F(x,y,y') = 0$

\Def $y'=\frac{dy}{dx}=f(x,y)$ -- \textit{уравнение, разрешённое относительно производной}

\Def Функция $y = \varphi(x)$, определенная на промежутке $I$, называется \textit{решением дифференциального уравнения} $y'=f(x,y)$, если 
\begin{enumerate}
    \item $\varphi$ имеет непрерывную производную $\varphi'(x)$ на $I$.
    \item $\forall x \in I \quad (x, \varphi(x)) \in \Omega$
    \item $\varphi'(x) = f(x,\varphi(x))$ на $I$
\end{enumerate}

\Def $M(x,y)dx + N(x,y)dy = 0$ -- \textit{уравнение в дифференциалах}. Подмножество его решений:
\begin{equation*}
    \left[ 
      \begin{gathered} 
        y_x' = -\frac{M(x,y)}{N(x,y)} \\ 
        x_y' = -\frac{N(x,y)}{M(x,y)} \\ 
      \end{gathered} 
\right.
\end{equation*}





\subsection*{Уравнения с разделяющимися переменными}

Уравнения с разделяющимися переменными -- это уравнения, которые могут быть записаны в виде
\begin{equation}\label{eq2} 
    y' = f(x)g(y) \qquad f(x) \in C(I_1), g(y) \in C(I_2)
\end{equation}
\begin{center}
    или же в виде
\end{center}
\begin{equation}\label{eq3} 
    M(x)N(y)dx + P(x)Q(y)dy = 0
\end{equation}

\Note Если же $y_k\in I_2$ решение уравнение $g(y) = 0$, то $y\equiv y_k$ -- решение дифф. уравнения.
\bigbreak
Если же $y(x)$ нигде не принимает значение $y_k$, то $g(y) \neq 0$, а потому мы можем делить на него. Значит, чтобы решить исходное уравнение, необходимо \textit{разделить переменные}, то есть, привести уравнение к такой форме, чтобы при дифференциале $dx$ стояла
функция, зависящая лишь от $x$, а при дифференциале $dy$ -- функция, зависящая от $y$. 

Для этого уравнение вида (\ref{eq2}) или (\ref{eq3}) следует переписать в форме:

\begin{equation*}
    \frac{y'}{g(y)} = f(x) \qquad \qquad \frac{M(x)}{P(x)}dx + \frac{Q(y)}{N(y)}dy = 0
\end{equation*}

Будем работать с первым вариантом, так как он более общий. Проинтегрируем обе части по $x$:
\begin{align*}
    \int\frac{y'dx}{g(y)} &= \int f(x)dx \\
    \int\frac{dy}{g(y)} &= \int f(x)dx \\
    H(y) &= F(x) + C \\
    y &= H^{-1}(F(x) + C)
\end{align*}

Поскольку $g(y)$ знакопостоянна, то $H(y)$ строго монотонна, а следовательно, обратима

\subsection*{Уравнения, приводящиеся к уравнениям с разделяющимися переменными}
\begin{equation*}
    y' = f(ax + by + c)
\end{equation*}
Сделав в таком уравнении замену $z = ax + by + c$, получим уравнение с разделяющимися переменными $\frac{dz}{dx} = bf(z) + a$.


\subsection*{Однородные уравнения}
\Def Функция двух переменных $f(x, y)$ называется \textit{однородной степени} $m$ (еще
говорят, с показателем однородности $m$), если для всех $t$ (или хотя бы для
$t > 0$) справедливо соотношение: 
\begin{center}
    $f(tx, ty) = t^m f(x, y)$
\end{center} 

\textit{Однородным дифференциальным уравнением} называется уравнение вида 
\begin{equation}
    M(x, y) dx + N(x, y) dy = 0
\end{equation}
если $M(x, y)$ и $N(x, y)$ -- однородные функции одной и той же степени $m$. 

Можно показать, что однородное уравнение может
также быть записано в виде 
\begin{center}
    $y' = f(\frac{y}{x}), \qquad f(z)\in C(I)$
\end{center}

Однородное уравнение приводится к уравнению с разделяющимися переменными с помощью замены искомой функции $y(x)$ по формуле:
\begin{center}
    $t(x) = \frac{y(x)}{x}$
\end{center}

Тогда производная $y'$ и дифференциал $dy$ заменяются по формулам:
\begin{equation*}
    y' = t'x + t, \qquad dy = tdx + xdt
\end{equation*}
После решения полученного уравнения нужно сделать обратную подстановку $t = \frac{y}{x}$


\subsection*{Уравнения, приводящиеся к однородным}
\begin{equation*}
    y' = f\Big(\frac{ax + by + c}{a_1x + b_1y + c_1}\Big), \qquad f(z)\in C(I)
\end{equation*}
приводится к однородному уравнению заменой $u = x - x_0, \; v = y - y_0$ , где $(x_0, y_0)$ — точка пересечения прямых $ax+by+c = 0$ и $a_1x+b_1y+c_1 = 0$. Если
же эти прямые не пересекаются, то $a_1x + b_1y = k(ax + by)$ для некоторого $k \in \R$ и уравнение имеет вид $y' = f_1(ax+by)$.

\Def Уравнение называется \textit{обобщенно-однородным}, если его можно привести к
однородному заменой $y = z^m$ , где $m$ -- некоторое действительное число.

\Example $9yy' - 18xy + 4x^3 = 0 \;\; \Rightarrow \;\; 9mz^{2m-1}z'-18xz^m+4x^3 = 0$. 

Оно однородно, если $2m-1 = 1+m = 3 \;\; \Rightarrow \;\; m = 2 \;\; \Rightarrow \;\; 9z^3z' - 9xz^2 + 2x^3 = 0.$




\subsection*{Линейные уравнения}

\textit{Линейным уравнением первого порядка} называется уравнение, линейное относительно искомой функции $y(x)$ и ее производной, то есть, уравнение вида
\begin{equation}\label{eq5}
    y' + a(x)y = b(x) \qquad a(x), b(x) \in C(I)
\end{equation}

Функция $b(x)$ называется \textit{свободным членом} уравнения (4). Уравнение 
\begin{equation}\label{eq6}
    y' + a(x)y = 0
\end{equation}
называется \textit{линейным однородным уравнением}, соответствующим линейному уравнению (\ref{eq5}).

Покажем, что однородное уравнение является уравнением с разделяющимися переменными (далее подразумевается, что $y \neq 0$):
\begin{align*}
    y' + a(x)y &= 0 \qquad \Rightarrow \qquad
    \frac{y'}{y} = -a(x) \qquad \Rightarrow \qquad
    \int\frac{dy}{y} = -\int a(x)dx \qquad \Rightarrow\\
    \Rightarrow \qquad \ln|y| &= -\int\limits_{x_0}^xa(t)dt + C \qquad \Rightarrow \qquad
    |y| = e^{C}\cdot e^{-\int\limits_{x_0}^xa(t)dt}
\end{align*}
Работаем с интегралом с переменным верхним пределом ($x_0$ -- любая точка из промежутка непрерывности $a(t)$). Таким образом, мы получили одну из возможных первообразных, а все остальные с помощью прибавления константы.

Объединяя все решения, получаем \textit{общее решение}:
\begin{equation*}
    y_0=C\exp{\Big[-\int\limits_{x_0}^xa(t)dt\Big]}
\end{equation*}
Будем искать частное решение исходного линейного уравнения в виде 

\textbf{метод вариации постоянной}:
\begin{equation*}
    y_{\text{ч}}=C(x)\cdot\exp{\Big[-\int\limits_{x_0}^xa(t)dt\Big]}
\end{equation*}

Подставим его в левую часть уравнения:
\begin{align*}
    y' + a(x)y = \exp{\Big[-\int\limits_{x_0}^xa(t)dt\Big]}\cdot \Big(C'(x) - C(x)a(x) + C(x)a(x)\Big) 
    = C'(x)\cdot \exp{\Big[-\int\limits_{x_0}^xa(t)dt\Big]}
\end{align*}
Тогда получаем, что исходное уравнение (\ref{eq5}) имеет вид:
\begin{align*}
    C'(x) &= b(x) \cdot \exp{\Big[\int\limits_{x_0}^xa(t)dt\Big]}\\
    C(x) &= \int\limits_{x_1}^{x}b(t)\exp{\Big[\int\limits_{t_0}^ta(\tau)d\tau\Big]}dt + C \\
    y &= \underbrace{\Big(\int\limits_{x_1}^{x}b(t)\exp{\Big[\int\limits_{t_0}^ta(\tau)d\tau\Big]}dt\Big)\cdot \exp{\Big[-\int\limits_{x_0}^xa(t)dt\Big]}}_{\text{частного решение линейного}} + \underbrace{C\exp{\Big[-\int\limits_{x_0}^xa(t)dt\Big]}}_{\text{общее решение однородного}}
\end{align*}

\Example $y' + y = 4x$

Решение однородного: $y = Ce^{-x}$

Подстановка: $C'(x)e^{-x} - C(x)e^{-x} + C(x)e^{-x} = 4x \;\; \Rightarrow \; C'(x) = 4xe^{x}$
\begin{align*}
    C(x) &= 4(xe^x - \smallint e^xdx) = 4(x-1)e^x + C \\
    y &= 4(x-1) + Ce^{-x}
\end{align*}

\newpage

\subsection{Уравнения в полных дифференциалах. Интегрирующий множитель. Уравнения Бернулли и Риккати}
\subsection*{Уравнения в полных дифференциалах}
Рассмотрим уравнение первого порядка, записанное в дифференциалах. Это уравнение
\begin{equation}\label{eq7}
    M(x, y) dx + N(x, y) dy = 0
\end{equation}
называется \textit{уравнением в полных дифференциалах}, если его левая часть является дифференциалом некоторой гладкой функции $F(x, y)$. 
Тогда это уравнение
можно переписать в виде $dF(x, y) = 0$, так что его решение будет иметь вид
\begin{equation*}
    F(x, y) = C
\end{equation*}

\Statement Если функции $M(x, y)$ и $N(x, y)$ определены и непрерывны в некоторой односвязной области $\Omega$ и имеют в ней \underline{непрерывные частные производные} по $x$ и по $y$, то уравнение (\ref{eq7}) будет уравнением в полных дифференциалах тогда и только тогда, когда выполняется тождество
\begin{equation}\label{eq8}
    \frac{\partial M(x,y)}{\partial y} \equiv \frac{\partial N(x,y)}{\partial x}
\end{equation}

Легко понять откуда берется это тождество: если $F$ — решение уравнения, то 
\[
M(x, y) = \frac{\partial F}{\partial x}, \;\; N(x, y) = \frac{\partial F}{\partial y}.
\]
Тогда 
\[
\frac{\partial M(x,y)}{\partial y} =  \frac{\partial^2 F}{\partial x \partial y} = \frac{\partial N(x,y)}{\partial x}.
\]

Если это условие выполнено, то криволинейный интеграл
\begin{equation*}
    \int\limits_{(x_0, y_0)}^{(x,y)}M dx + N dy
\end{equation*}
не зависит от выбора пути интегрирования, поэтому функцию $F(x, y)$ можно восстановить по любой из формул
\begin{equation}\label{eq9}
    F(x, y) = \int\limits_{x_0}^xM(x, y) dx + \int\limits_{y_0}^yN(x_0, y) dy \qquad \text{или} \qquad F(x, y) = \int\limits_{y_0}^yN(x, y) dx + \int\limits_{x_0}^xM(x, y_0) dy
\end{equation}

При этом нижние пределы $x_0$ и $y_0$ можно выбирать произвольно, лишь бы точка $(x_0, y_0)$ принадлежала области $D$ (области определения функций $M$
и $N$). За счет правильного выбора чисел $x_0$ и $y_0$ иногда удается упростить
вычисления интегралов (\ref{eq9}). Например, если функции $M$ и $N$ являются многочленами от $x$ и $y$, целесообразно выбирать $x_0 = y_0 = 0$.

\subsection*{Интегрирующий множитель}
Пусть дано уравнение~(\ref{eq7}), для которого не выполнено условие (\ref{eq8}).

\Def Функция $\mu(x, y) \neq 0$ называется \textit{интегрирующим множителем} для уравнения (\ref{eq7}), если уравнение
\begin{equation*}
    \mu(x, y)\big(M(x, y) dx + N(x, y) dy\big) = 0,
\end{equation*}
является уравнением в полных дифференциалах. Отсюда следует, что функция $\mu$ удовлетворяет условию
\begin{equation*}
    \frac{\partial(\mu M)}{\partial y} \equiv \frac{\partial(\mu N)}{\partial x}
\end{equation*}

Это равенство дает уравнение в частных производных первого порядка для $\mu(x,y)$:
\begin{equation*}
    N\frac{\partial\mu}{\partial x} - M\frac{\partial\mu}{\partial y} = \Big(\frac{\partial M}{\partial y} - \frac{\partial N}{\partial x}\Big)\mu
\end{equation*}
Поделив обе части последнего уравнения на $\mu$, перепишем его в виде:
\begin{equation*}
    N\frac{\partial(ln\mu)}{\partial x} - M\frac{\partial(ln\mu)}{\partial y} = \frac{\partial M}{\partial y} - \frac{\partial N}{\partial x}
\end{equation*}

Несмотря на то, что эти уравнения, как
правило, имеют бесконечно много решений, задача их нахождения в общем случае ничуть не легче решения исходного уравнения.

Рассмотрим два случая, когда уравнение (\ref{eq7}) имеет интегрирующий множитель, зависящий только от $x$ или только от $y$:
\begin{enumerate}
    \item $\mu = \mu(x)$. Тогда 
    \begin{equation*}
        \frac{d(ln\mu)}{d x} = \frac{\frac{\partial M}{\partial y} - \frac{\partial N}{\partial x}}{N}
    \end{equation*}
    и такой множитель существует, если правая часть зависит только от $x$ или
является постоянной.
    \item $\mu = \mu(y)$. Тогда 
    \begin{equation*}
        \frac{d(ln\mu)}{d y} = \frac{\frac{\partial M}{\partial y} - \frac{\partial N}{\partial x}}{-M}
    \end{equation*}
    и правая часть должна зависеть только от $y$ или быть постоянной.
\end{enumerate}

\subsection*{Уравнение Бернулли}
Нелинейное уравнение первого порядка вида
\begin{equation*}
    y' + a(x)y = b(x)y^m, \qquad m \neq 0, m \neq 1,\;\; a,b\in C(I)
\end{equation*}
называется \textit{уравнением Бернулли}. 
\bigbreak
Заметим, что $y=0$ -- решение
уравнения Бернулли при $m > 0$. 

Если $y \neq 0$, то, разделив уравнение на
$y^m$ и вводя новую неизвестную функцию $z = y^{1-m}$, относительно функции $z$ получаем линейное уравнение:
\begin{align*}
    \frac{y'}{y^m}+a(x)\frac{1}{y^{m-1}}=b(x)\qquad \text{при этом \;} (y^{1-m})'=(1-m)y^{-m}y' \\ 
    \text{Делаем замену: $z = y^{1-m}$} \quad \Rightarrow \qquad \frac{z'}{1-m}+a(x)z=b(x)
\end{align*}

\subsection*{Уравнение Риккати}
Нелинейное уравнение первого порядка вида
\begin{equation*}
    y' = a(x)y^2 + b(x)y + c(x), \qquad a,b,c\in C(I)
\end{equation*}
называется \textit{уравнением Риккати}. 
\bigbreak
В отличие от всех уравнений, рассматривавшихся ранее, уравнение Риккати не всегда интегрируется в квадратурах.
Чтобы решить его, необходимо знать хотя бы одно частное решение $y = y_1(x)$
этого уравнения. Тогда замена $y = y_1 + z$ приводит это уравнение к уравнению Бернулли. Однако, проще сразу сделать замену:
\begin{equation*}
    z = \frac{1}{y-y_1} \quad \Rightarrow \quad y = y_1+\frac{1}{z} \quad y'_x = y'_{1x} - \frac{z'_x}{z^2}
\end{equation*}
которая сводит уравнение Риккати к линейному.
\newpage

\subsection{Метод введения параметра для уравнения первого порядка, не разрешенного относительно производной}
\textbf{Определение} Уравнение первого порядка, не разрешенное относильно производной~--- это уравнение вида 
\begin{equation}\label{firstorder-parapm-eq}
    F(x, y, y') = 0
\end{equation}
где \(F(x, y, p)\)~--- заданная непрерывная функция в некоторой непустой окрестности $G$ евклидового пространства $\mathbb{R}_{(x, y, p)}^3$ с декартовыми прямоугольными координатами $x, y, p$. Где $x$~--- аргумент, $y = y(x)$~--- неизвестная функция.

В общем случае для решения уравнения~(\ref{firstorder-parapm-eq}) применяется метод введения параметра, который позволяет свести решение~(\ref{firstorder-parapm-eq}) к решению некоторого уравнения первого порядка в симметричной форме.

Сам метод: положим $y' = p$ и рассмотрим систему
\begin{equation}\label{firstorder-parapm-sys}
    \begin{cases}
    F(x, y, p) = 0 \\
    dy = pdx.
    \end{cases}
\end{equation}

\begin{theorem}[Не доказывалось]
Уравнение~(\ref{firstorder-parapm-eq}) эквивалентно системе~(\ref{firstorder-parapm-sys}).
\end{theorem}
\begin{proof}
Проверяется непосредственной подстановкой решений.
\end{proof}

Наиболее важным является случай когда уравнение~(\ref{firstorder-parapm-eq}) разрешимо относительно $y$ или $x$. Тогда система~(\ref{firstorder-parapm-sys}) принимает вид 

\begin{equation}\label{firstorder-parapm-sys-simp}
    \begin{cases}
    y = f(x, p) \\
    dy = pdx.
    \end{cases}
\end{equation}
откуда получается уравнение
\[\left(\frac{\partial f}{\partial x} - p\right) dx + \frac{\partial f}{\partial p} dp = 0.\]

\begin{lemmanote}
При таком типе решения ошибкой является переход от параметра $p$ обратно к $y'$. Важно понимать, что это не замена, а именно введение параметра и решения могут быть записаны как в параметрической так и в явной форме.
\end{lemmanote}

\begin{lemmanote}[О возникновении неоднозначности]
Откуда берутся особые решения? При переходе к параметру $p$ неявно применяется теорема о неявной функции:
\begin{align*}
    F(x, y, p) &= 0 \; \Longleftrightarrow \; \underbrace{y' = p = f(x, y)}_{\text{т. о неявной функ.}}\\
    F(x, y, f(x, y)) &= 0.
\end{align*}
Но понятно, что у теоремы есть условия которые не всегда выполняются. В качестве примера можно рассмотерть уравнение $(y')^2 - x^2 = 0$.
\end{lemmanote}

\begin{theorem}
Пусть в области $G \subset \mathbb{R}^3$ функция $F$ непрерывна вместе с $\frac{\partial F}{\partial y}$, $\frac{\partial F}{\partial p}$, а также $\frac{\partial F}{\partial p}|_{(x_0, y_0, p_0) \in G} \neq 0$.
Тогда существует $\delta > 0$ на $[x_0 - \delta, x_0 + \delta]$ существует единственное решение задачи Коши
\[
\begin{cases}
F(x, y, y') = 0\\
F(x_0, y_0, p_0) = 0\\
y(x_0) = y_0\\
y'(x_0) = p_0
\end{cases}
\]
\end{theorem}
\begin{proof}
Так как $\frac{\partial F}{\partial p}|_{(x_0, y_0, p_0) \in G} \neq 0$, то по теореме о неявной функции $\exists U(x_0, y_0), U(p_0)$, что $p_0 = f(x_0, y_0)$ и 
\[\forall (x, y) \in U(x_0, y_0)\;\; \exists f(x, y): U(x_0, y_0) \rightarrow U(p_0), \; p = f(x, y).\]

Итого имеем
\[
\begin{cases}
y' = f(x, y)\\
y(x_0) = y_0.
\end{cases}
\]

А также
\[
\frac{\partial F}{\partial y} + \frac{\partial F}{\partial p} \cdot \frac{\partial f}{\partial y} = 0,
\]
\[
\frac{\partial f}{\partial y} = - \frac{\frac{\partial F}{\partial y}}{ \frac{\partial F}{\partial p} }.
\]

Выполнены условия теоремы существования и единственности решения задачи Коши для уравнения, разрешённого относительно производной.
\end{proof}
Эквивалентное доказательство можно найти в пункте~\ref{zk-notsolved}.

\subsection{Методы понижения порядка дифференциальных уравнений}
\subsection{Теорема о полноте метода резолюций: из невыполнимой КНФ всегда можно вывести $\perp$.}

\textbf{Теорема.} Метод резолюций всегда заканчивает свою работу, причём для невыполнимых КНФ выводится $\perp$ (полнота), а для выполнимых не выводится (корректность).

\textbf{Доказательство.}

\includegraphics[width=0.9\textwidth]{images/1.4}

\newpage
\section{Задача Коши.}

\subsection{Принцип сжимающих отображений.}
\Def \textit{Линейное пространство} $L$ называется нормированным, если каждому его элементу $x$ поставлено в соответствие неотрицательное действительное число, называемое \textit{нормой} $x$ (обзначается $||x||$), обладающее свойствами:

1) $||x|| \geq 0, ||x|| = 0 \Longleftrightarrow x = 0_L$;

2) для любого $x \in L$ и $\lambda \in \R$ верно: $||\lambda x|| = |\lambda|\cdot ||x||$;

3) для любых $x, y \in L \quad ||x + y||\; \leqslant ||x|| + ||y||$.
\bigbreak
\noindent \Def Последовательность $\{ x_n \} \subset L$ называется \textit{сходящейся к $x \in L$ \underline{по норме}}, если \\ $\lim \limits_{n \rightarrow \infty} ||x_n - x|| = 0$.
\\
\Def Последовательность $\{ x_n \} \subset L$ называется \textit{фундаментальной} в норме, если \\ $||x_n - x_k|| \rightarrow 0$ при $n,k \rightarrow \infty$.
\bigbreak
\noindent
\Def Нормы $||\cdot||_1$ и $||\cdot||_2$ одного и того же нормированного пространства $L$ \textit{эквивалентны}, если $\exists c_1, c_2 > 0$, такие что $\forall x \in L \;\; c_1 {||x||}_2 \leqslant {||x||}_1 \leqslant c_2 {||x||}_2$.
\\
\Example Пространство функций, непрерывных на $[a, b]$, является линейным. Введём
две нормы:
\begin{equation*}
    ||f||_c = \sup_{x\in[a,b]}|f(x)|\ \text{(цэ-норма)},\\
    ||f||_{L_1} = \frac{1}{b - a} \int_a^b |f(x)|dx\ \text{($L_1$ норма)}.
\end{equation*}
Отметим, что равномерная сходимость является сходимостью по первой норме (цэ-норме), причём предел также непрерывен, а значит, принадлежит пространству.
\bigbreak
\noindent \Def Нормированное пространство, в котором каждая фундаментальная последовательность является сходящейся, является \textit{полным}. Но! Не во всяком нормированном пространстве фундаментальная последовательность сходится.
\\
\Def Полное линейное нормированное пространство называется \textit{банаховым}.
\\
\Def Открытый шар: $U_\varepsilon(a) = \{x \in L \;:\; ||x - a|| \,< \varepsilon\}$. Замкнутый: $\overline{U}_\varepsilon(a) =
\{\ldots \leqslant \varepsilon\}$
\bigbreak
Пусть $L_1$ и $L_2$ — банаховы пространства, $X\subseteq L_1$.
\\
\Def Оператор $\Phi: X \rightarrow L_2$ называется \textit{непрерывным} в точке $x_0$, если \\
\[\forall\varepsilon > 0 \;\; \exists\delta=\delta_{\varepsilon} \;\; \forall x \in U_\delta (x_0) \cap X \;\Rightarrow\; {||\Phi(x) - \Phi(x_0)||_2} < \varepsilon.\]
\bigbreak
Переходим в ситуацию, где $L_2 \equiv L_1$.
\\
\Def Точка $x^* \in X$ называется \textit{неподвижной точкой отображения} $\Phi$, если $\Phi(x^*) = x^*$.
\\
\Def Оператор $\Phi$ называется \textit{сжимающим на множестве $X$}, если $\exists q \in (0;1)$, такое что $\forall x_1, x_2 \in X \; \mapsto \; ||\Phi(x_1) - \Phi(x_2)|| \leqslant q||x_1 - x_2||$. Число $q$ — коэффициент сжатия.
\\
\Statement Сжимающее отображение является непрерывным:
\begin{equation*}
    \forall \varepsilon > 0 \;\; \exists \delta \leqslant \frac{\varepsilon}{q} \;\; \forall x_1,x_2 \; : \; x_1 \neq x_2 \mapsto ||x_1-x_2|| \;< \delta \quad \Rightarrow \quad ||\Phi(x_1)-\Phi(x_2)|| \;\leqslant q||x_1 - x_2|| \;\leqslant \varepsilon 
\end{equation*}

\subsection*{Теорема Банаха о неподвижной точке (принцип сжимающих отображений)}

Пусть $\Phi \,:\; \overline{U}_r (x_0) \to L$, причем $\Phi$ является сжимающим на $\overline{U}_r (x_0)$ с некоторым коэффициентом $q$. Тогда если выполнено условие $||\Phi(x_0) - x_0|| \leqslant (1-q)r$, то  в $\overline{U}_r (x_0)$ существует единственная неподвижная точка отображения.

\Proof Покажем, что $\Phi(\overline{U}_r(x_0))\subseteq\overline{U}_r(x_0)$. Пусть $x\in\overline{U}_r(x_0)$. Тогда 
\begin{equation*}
    ||\Phi(x) - x_0||\; = ||\Phi(x) -\Phi(x_0) + \Phi(x_0) - x_0||\; \leqslant ||\Phi(x) -\Phi(x_0)||\; +\; ||\Phi(x_0) - x_0||\; \leqslant
\end{equation*}
Так как отображение сжимающие, оценим первый модуль. Дополнительное условие в теореме используем для второго модуля.
\begin{equation*}
    \leqslant q||x - x_0|| \;+\; (1-q)\cdot r \leqslant q\cdot r + (1-q)\cdot r = r \qquad \Rightarrow \; \forall x \in \overline{U}_r(x_0) \;\; \Phi(x)\subseteq\overline{U}_r(x_0)
\end{equation*}
Рассмотрим последовательность $\{x_n\}\in\overline{U}_r(x_0)$, такую что $x_n = \Phi(x_{n-1})$ при $n \geqslant 1$. Также для удобства обозначим $\rho = ||\Phi(x_0) - x_0|| = ||x_1 - x_0||$. Покажем, что эта последовательность фундаментальная:
\begin{align*}
    ||x_2 - x_1|| \;= ||\Phi(x_1) - \Phi(x_0)||\; \leqslant q||x_1 - x_0||\; = q \cdot\rho \qquad \Rightarrow \;\ldots \; \Rightarrow\qquad
    ||x_{n+1} - x_n|| \;\leqslant q^n \rho
\end{align*}

Используем полученную оценку для того, чтобы оценить модули в сумме:
\begin{align*}
    \forall p \;\; ||x_{n+p} - x_n||\; \leqslant ||x_{n+p} - x_{n+p-1}|| \;+\; ||x_{n+p-1} - x_{n+p-2}|| \;+ \;\ldots + ||x_{n+1} - x_n||\; \leqslant \\ 
    \leqslant \rho (q^{n+p-1} + q^{n+p-2} + \dots + q^n) = \rho q^n (q^{p-1} + q^{p-2} + \dots + 1) = \frac{\rho q^n(1-q^p)}{1-q} < \frac{\rho q^n}{1 - q} \xrightarrow[n\to\infty]{}0
\end{align*}
А так как мы в банаховом пространстве (т.е. полном), то из фундаментальности получили сходящуюся последовательность.

$\exists \;x^* = \lim \limits_{n \rightarrow \infty} x_n$. А так как $\overline{U}_r (x_0)$ -- замкнутый шар, значит $x^* \in \overline{U}_r (x_0)$.
\\
Докажем, что $x^*$ является неподвижной точкой нашего оператора $\Phi$. Воспользуемся тем, что сжимающее отображение является неприрывным.

В $x_n = \Phi(x_{n-1})$ перейдём к пределу: $x^* = \lim \limits_{n \rightarrow \infty} x_n = \Phi(\lim \limits_{n \rightarrow \infty} x_{n-1}) = \Phi(x^*)$.
\\
Докажем единственность неподвижной точки.
Допустим, что существует $x^{**} \in \overline{U}_r (x_0): x^{**}= \Phi(x^{**})$, такое что $x^* \neq x^{**}$.

Тогда $||{x^* - x^{**}}|| \;=||\Phi(x^*) - \Phi(x^{**})|| \;\leqslant q\cdot ||x^* - x^{**}||$, где $q<1$, и получаем противоречие. \EndProof



\newpage

\subsection{Теоремы существования и единственности решения задачи Коши для нормальной системы дифференциальных уравнений и для уравнения \texorpdfstring{$n$}-го порядка в нормальном виде}
\setcounter{equation}{0}

\textbf{Определение} 
Пусть $n\geqslant 2$, $f_1,\ldots,f_n$ -- непрервные функции, определенные на $G \subseteq \R_{(x, \vec{y})}^{n+1}$. 
\newline Назовём \textit{нормальной системой дифференциальных уравнений первого порядка} следующую систему
\begin{equation}\label{Koshi1}
    \begin{cases}
    y_1'(x) = f_1(x, y_1(x), \dots, y_n(x)) = f_1 (x; \vec{y}(x)) \\
    \vdots \qquad \vdots \\
    y_n'(x) = f_n(x, y_1(x), \dots, y_n(x)) = f_n (x; \vec{y}(x)) \\
\end{cases}
\end{equation}
\begin{equation*}
    \text{Векторная форма: $\vec{y'} = \vec{f}(x; \vec{y})$, где $\vec{y}(x) = \left( \begin{matrix} y_1(x), \ldots, y_n(x) \end{matrix} \right)^T$}
\end{equation*}

\Def Вектор-функция $\vec{\varphi}(x)$ называется \textit{решением нормальной системы (\ref{Koshi1})} на некотором промежутке $I \subseteq \R$, если:

\begin{enumerate}
    \item $\vec{\varphi} (x) \in C^1(I)$
    \item $\forall x \in I \quad (x; \vec{\varphi}(x)) \in G$
    \item $\forall x \in I \quad \vec{\varphi'}(x) = f(x; \vec{\varphi}(x))$
\end{enumerate}
График решения $\vec{\varphi}(x)$ в пространстве $\R^{n+1}$ -- это интегральная кривая.
\bigbreak
\Def \textit{Задача Коши} -- это \begin{equation}\label{Koshi2}
    \begin{cases}
    \text{$\vec{y'} = \vec{f}(x; \vec{y})$ \;--\; нормальная система уравнений I-го порядка} \\
    \text{$\vec{y}(x_0) = \vec{y_0}$ \;\;\;--\; начальное условие} 
    \end{cases}
\end{equation}
Рассмотрим уравнение $\vec{y'} = f(x; \vec{y})$. Проинтегрируем его покомпонентно. Получим слева искомое $y(x)$, справа ищем одну из первообразных как интеграл с переменным верхним пределом: 
\begin{equation}\label{Koshi3}
    y(x) = \int \limits_{x_0}^{x} f(\tau ; \vec{y}(\tau))d\tau + \vec{y_0}
\end{equation}
\par \Def Вектор-функция $\vec{\varphi}(x)$ из пространства $C_n(I)$ называется \textit{решением интегрального уравнения (\ref{Koshi3})} на $I\subseteq\R$, если:

\begin{enumerate}
    \item $\vec{\varphi}(x) \in C^1(I)$
    \item $\forall x \in I \quad (x; \vec{\varphi}(x)) \in G$
    \item $\forall x \in I \quad \vec{\varphi}(x) = \int \limits_{x_0}^x f(\tau; \vec{y}(\tau))d\tau + \vec{y_0}$
\end{enumerate}

\Lemma (об эквивалентности) Вектор-функция $\vec{\varphi}(x)$ является решением задачи Коши (\ref{Koshi2}) на $I$ тогда и только тогда, когда $\vec{\varphi}(x)$ на том же $I$ является решением (\ref{Koshi3}).

\Proof

\fbox{$\Longrightarrow$} Пусть $\vec{y}=\vec{\varphi}(x)$ -- решение ЗК на $I$. Тогда $\vec{y'}(x) = f(x; \vec{y}(x))$, значит, $\vec{y}(x) = \int \limits_{x_0}^{x} \vec{f}(\tau; \vec{y}(\tau))d\tau + C$. Из начального условия $y_0 = \vec{y}(x_0) = \int \limits_{x_0}^x \dots d\tau + C \;\;\Rightarrow\;\; C = \vec{y_0} \;\;\Rightarrow\;\; \vec{y}(x) = \int \limits_{x_0}^{x} \vec{f}(\tau; \vec{y}(\tau))d\tau + \vec{y_0}$.

\fbox{$\Longleftarrow$} Пусть $\vec{y}=\vec{\varphi}(x)$ -- решение интегрального уравнения. $\forall x \in I$ $\vec{\varphi}(x) = 
\int \limits_{x_0}^{x} \vec{f}(\tau; \vec{\varphi}(\tau))d\tau + \vec{y_0}$.
\newline Дифференцируем по $x$, получим 
\(\vec{\varphi'} = \vec{f}(x;\vec{\varphi}(x))\) \;и\; 
\(\vec{\varphi}(x_0) = \int \limits_{x_0}^{x_0} \dots d\tau + \vec{y_0} = \vec{y_0}\). \quad \EndProof


Возьмём кубическую норму: $|\vec{y}| = \max \limits_{1 \leqslant i \leqslant n} |y_i|$
\\
\Def Вектор-функция $\vec{f}(x; \vec{y})$, определённая в области $G \subseteq \R_{(x, \vec{y})}^{n + 1}$ называется \textit{удовлетворяющей условию Липшица} относительно $\vec{y}$ равномерно по $x$, если $\exists L > 0$ такой, что $\forall (x; \vec{y_1})$ и $(x; \vec{y_2})$ верно, что $|\vec{f}(x; \vec{y_1}) - \vec{f}(x; \vec{y_2})|\; \leqslant L|\vec{y_1} - \vec{y_2}|$.
\bigbreak
\Lemma Вектор-функция $\vec{f}(x; \vec{y})$ удовлетворяет условию Липшица по $\vec{y}$ равномерно по $x$ при выполнении следующих условий:

\begin{enumerate}
    \item $G$ -- выпуклая область в $\R^{n + 1}$;
    \item $\vec{f} (x; \vec{y}) \in C_n(G)$, т.е. непрерывна (от $n$ аргументов) и $\forall i, j = \overline{1,n}$\; $\frac{\partial f_i}{\partial y_j} \in C(G)$
    \item $\exists K > 0:$ $\forall i, j = \overline{1,n}$ \; $\forall (x, \vec{y}) \in G:\;$ $|\frac{\partial \vec{f}_i}{\partial y_i} (x; \vec{y})| \leqslant K$.
\end{enumerate}

\Proof Фиксируем $i = 1, \dots, n$. Рассмотрим $(x; \vec{y_1})$, где $\vec{y_1} = (y_{1_1}, \dots, y_{1_n})$, а также $\vec{y_2} = (y_{2_1}, \dots, y_{2_n})$.
\begin{align*}
    |f_i (x; \vec{y_1}) - f_i (x; \vec{y_2})| \,= \Big|f_i (x; \vec{y_2} + \theta(\vec{y_1} - \vec{y_2}))|_{\theta = 0}^{\theta = 1}\Big| \stackrel{\footnotesize{\text{Ньютон-Лейбниц}}}{=} \bigg| \int \limits_0^1 \Big[\frac{d}{d \theta} f_i (x; \vec{y_2} + \theta (\vec{y_1} - \vec{y_2})) \Big]d \theta\bigg| = \\
    = \bigg|\int \limits_0^1 \sum \limits_{j = 1}^n \frac{\partial f_i (x; \vec{y_2} +  \theta(\vec{y_1} - \vec{y_2}))}{\partial y_j}(y_{1_j} - y_{2_j}) d\theta\bigg| \; \leqslant \sum \limits_{j = 1}^{n} \int \limits_0^1 \underbrace{\bigg|\frac{\partial f_i (x; \vec{y_2} +  \theta(\vec{y_1} - \vec{y_2}))}{\partial y_j}\bigg|}_{\leqslant K}\cdot \underbrace{|y_{1_j} - y_{2_j}|}_{\leqslant |\vec{y_1} - \vec{y_2}|} d\theta \leqslant \\
    \leqslant \underbrace{n\cdot K}_{=L} \cdot |\vec{y_1} - \vec{y_2}| \qquad \Rightarrow \qquad |\vec{f}(x; \vec{y_1}) - \vec{f}(x; \vec{y_2})| = \max \limits_{1 \leqslant i \leqslant n} |f_i (x; \vec{y_1}) - f_i  (x; \vec{y_2})| \leqslant L \cdot |\vec{y_1} - \vec{y_2}| \quad \text{\EndProof }
\end{align*}

\subsection*{Теорема о существовании и единственности решения задачи Коши для системы уравнений $n$-го порядка в нормальном форме}

\noindent Пусть вектор-функция $\vec{f} \brackets{x, \vec{y}}$ непрерывна в области $G$ вместе со своими производными по $y_j \;(j = \overline{1, n})$, точка $(x_0, \vec{y_0})$ тоже лежит в $G$. Тогда задача Коши локально разрешима единственным образом:

\begin{enumerate}
    \item \(\exists \delta > 0\), такое что на $[x_0 - \delta, x_0 + \delta]$ решение задачи Коши существует;
    \item Решение единственно в следующем смысле: \\
    Если $\vec{y_1} (x) \equiv \vec{\varphi}(x)$ — решение задачи Коши в $\delta_1$-окрестности точки $x_0$, а $\vec{y_2} \equiv \vec{\psi}(x)$ — решение задачи Коши в $\delta_2$-окрестности точки $x_0$, то в окрестности точки $x_0$ с радиусом $\delta = \min (\delta_1, \delta_2)\newline \vec{\varphi}(x) \equiv \vec{\psi}(x)$. 
\end{enumerate}

\noindent \Proof Рассмотрим множество $\overline{H_{\delta, r}} (x_0) = \{ (x, \vec{y}) \in G: x \in [x_0 - \delta, x_0 + \delta] \;\; \text{и} \;\; |\vec{y} - \vec{y_0}| \leqslant r \} \subset G$. \newline Заметим, что в силу компактности этого множества (что следует из ограниченности и замкнутости) применима теорема Вейерштрасса (непрерывная на компакте функция ограничена): $\exists M > 0: \forall (x, \vec{y}) \in \overline{H_{\delta, r}} \;\; |\vec{f} (x, \vec{y})| \leqslant M$ \; и \; $\forall i, j =  \overline{1, n}$\;\; $\Big|\mathlarger{\frac{\partial f_i}{\partial y_j}}\Big| \leqslant M$. \newline Значит, $\vec{f}(x, \vec{y})$ на $\overline{H_{\delta, r}} (x_0)$ удовлетворяет условию Липшица относительно $\vec{y}$ равномерно по $x$. \newline Рассмотрим интегральное уравнение, которое как мы доказали ранее эквивалетно ЗК:

\begin{equation*}
    \vec{y}(x) = \vec{y_0} + \int \limits_{x_0}^x \vec{f}\brackets{\tau, \vec{y}(\tau)} d\tau \;\;\Longleftrightarrow \;\;\vec{y} = \Phi(\vec{y})
\end{equation*}

Рассмотрим в $C_n [x_0 - \delta, x_0 + \delta]$ замкнутый шар $\overline{D_{\delta, r}}(\vec{y_0}) = \{ \vec{y} \in C_n [x_0 - \delta, x_0 + \delta]: {||\vec{y} - \vec{y_0}||}_{C_n} \leqslant r \}$, где ${||\vec{y}||}_{C_n} = \max \limits_{1 \leqslant i \leqslant n} \sup \limits_{|x - x_0| < \delta} |y_i (x)|$. 

\noindent Докажем, что существуют $\delta$ и $r$ такие, что 
\begin{itemize}
    \item $\Phi$ является сжимающим
    \item отображает шар $\overline{D_{\delta, r}} (\vec{y_0})$ в себя 
\end{itemize}
Тогда мы сможем применить теорему Банаха о сжимающем отображении. Получим единственную неподвижную точку отображения $\Leftrightarrow$ интегральное уравнение имеет единственное решение $\Leftrightarrow$ ЗК имеет единственное решение.
\bigbreak
Докажем, что $\Phi$ является сжимающим. Рассмотрим $\vec{y}, \vec{z} \in \overline{D_{\delta, r}} (\vec{y_0})$ (функции): 
\begin{align*}
    ||\Phi(\vec{y}) - \Phi(\vec{z})|| \; &= \max \limits_{i = 1, \dots, n} \sup \limits_{|x - x_0| \leqslant \delta} \bigg|\int \limits_{x_0}^{x} \brackets{f_i(\tau, \vec{y}(\tau)) - f_i (\tau, \vec{z}(\tau))}d\tau\bigg| \; \leqslant \\
    &\leqslant \max \limits_{i = 1, \dots, n} \sup \limits_{|x - x_0| < \delta} \int \limits_{x_0}^x \underbrace{\Big|f_i(\tau, \vec{y}(\tau)) - f_i (\tau, \vec{z}(\tau))\Big|}_{\mathlarger{\leqslant |\vec{f}(\tau; \vec{y}) - \vec{f}(\tau; \vec{z})|}} d\tau \leqslant \\
    &\leqslant \sup \limits_{|x - x_0| < \delta} \int \limits_{x_0}^{x} L |\vec{y}(\tau) - \vec{z}(\tau)|d\tau \;\leqslant\; \sup \limits_{|x - x_0| < \delta} \int \limits_{x_0}^{x} L ||\vec{y} - \vec{z}||_{C_n} d\tau \; \leqslant \; \underbrace{\delta L}_{=q < 1} {||\vec{y} - \vec{z}||}_{C_n}
\end{align*}
Положим $\delta = \frac{q}{L}$, получим требуемое.
\bigbreak
Теперь докажем вторую часть:
\begin{align*}
    ||\Phi(\vec{y_0}) - \vec{y_0}|| = \max \limits_{i = 1, \dots, n} \sup \limits_{|x - x_0| < \delta} \bigg|\int \limits_{x_0}^x f_i \brackets{\tau, \vec{y_0}} d\tau\bigg|
    \leqslant \int \limits_{x_0}^x \brackets{\max \limits_{i = 1, \dots, n} \sup \limits_{|x - x_0| < \delta} |f_i (\tau, \vec{y_0})| d\tau} = \\
    = \int \limits_{x_0}^{x} {||\vec{f}(\tau, \vec{y_0})||}_{C_n} d\tau \;\leqslant \; \delta M = (1 - q) r
\end{align*}

Получили, что $\begin{cases} q = \delta L \\ (1 - q)r = \delta M \end{cases} \Longrightarrow \;\; \begin{cases} r - rq = \delta M \\ r = \delta L r + \delta M \end{cases} \Longrightarrow  \;\;\delta_r = \mathlarger{\frac{r}{M + Lr}}$ \qquad \EndProof

\subsection*{Теорема о существовании и единственности решения задачи Коши для уравнения $n$-го порядка в нормальном виде}

\noindent Нормальный вид -- уравнение разрешённо относительно старшей производной

\begin{equation*}
    \begin{cases}
    y^{(n)}=f(x,y,y',\ldots,y^{(n-1)}) \\
    y(x_0) = y_0\\
    y'(x_0) = y_0^1\\
    \ldots\\
    y^{(n-1)}(x_0)=y_0^{(n-1)}
    \end{cases}
\end{equation*}

Пусть функция $f(x, y, p_1, \ldots , p_{n-1})$ определена и непрерывна по совокупности переменных
вместе с частными производными по переменным $y, p_1, \ldots , p_{n-1}$ в некоторой области $G \subseteq \R^{n+1}$, и точка $(x_0, y_0, y^1_0, \ldots , y^{n-1}_0) \in G$, тогда существует замкнутая $\delta$-окрестность точки $x_0$, в
которой существует единственное (в ранее указанном смысле) решение задачи Коши.

\Proof Пусть $\vec{z} = (y,y',\ldots,y^{(n-1)})^T$ -- вектор-функция. Запишем: 
\begin{figure}[h]
    \vspace{-4ex}
    \hspace{-4ex} \begin{minipage}[h]{0.4\linewidth}
        \begin{align*}
            z_1'=z_2 \\
            z_2'=z_3 \\
            \ldots \\
            z_{n-1}'=z_n \\
            z_n'=f(x,z_1,z_2,\ldots,z_n)=f(x,\vec{z})
        \end{align*}
    \end{minipage}
    \hfill
    \hspace{-4ex} \begin{minipage}[h]{0.6\linewidth}
    Введём обозначение: $\vec{g}(x,\vec{z})=(z_2,z_3,\ldots,z_n, f(x,\vec{z}))^T$.
    \\
    Перепишем задачу Коши: $
        \begin{cases}
        \vec{z'}=\vec{g}(x,\vec{z}) \\
        \vec{z}(x_0)=\vec{z_0}
        \end{cases}
    $
    \end{minipage}
\end{figure}

\noindent Согласно предыдущей теореме, существует единственное решение полученной задачи
Коши в некоторой замкнутой $\delta$-окрестности точки $x_0$; эта же окрестность подходит и для
исходной ЗК. \; \EndProof
\bigbreak
Интегральная кривая для данной задачи определяется как множество точек вида:
$$\Big(x, y(x), y'(x), \ldots,y^{(n-1)}(x)\Big)$$

\newpage

\subsection{Теоремы о продолжении решения для нормальной системы дифференциальных уравнений}
\subsection{Теорема о структуре вполне упорядоченного множество: оно представляется как $\omega \cdot L + F$, где $L$ — множество предельных элементов (кроме, возможно, наибольшего), $F$ — конечное множество.}

\par $\blacktriangle$ Пусть $P$ - множество предельных элементов нашего ВУМа. Заметим, что $P$ - ВУМ (как подмножество ВУМа). Рассмотрим элемент $x \in P$. Пусть $Sx=y$ (следующий элемент). Построим биекцию между $\omega$ и $[x; y)$. Числу $n$ из $\omega$ поставим в соответствие число $\underbrace{SS\ldots S}_\text{$n$ раз}x$. Очевидно, что это инъекция ($x+n=x+m \Leftrightarrow n=m$).
\par Докажем, что это сюръекция. Рассмотрим элемент $t$ лежащий в $[x;y)$. Бесконечно уменьшать его на 1 (то есть брать предыдущий) нельзя по одному из эквивалентных определений фундированности $\Rightarrow$ существует предельный элемент $k$ (у которого нет предыдущего), такой что  $S\ldots Sk=t$. $k$ лежит на в $[x,y)$, но единственный предельный элемент, лежащий в этом множестве - это $x \Rightarrow k=x \Rightarrow t=S\ldots Sk$ будет получен. 
\par Повторим такие действия для всех $x$ (кроме наибольшего). Затем возможны 2 случая
\begin{enumerate}
    \item В исходном ВУМе нет наибольшего элемента. Тогда аналогично прошлым шагам строим изоморфизм между $\omega$ и оставшимися элементами. Получаем, что наш ВУМ равен $\omega \cdot P$
    \item В исходном ВУМе есть наибольший элемент. Тогда осталось лишь конечное число нерассмотренных элементов. Докажем это
    \par Обозначим наибольший элемент всего ВУМа как $a$. По определению фундированности, мы не сможем бесконечно брать предыдущий элемент $\Rightarrow$ существует $k$ - предельный, такой что $a=\underbrace{S\ldots S}_\text{$m$ раз}k$. $k \geq x,$ но $x$ - наибольший из предельных элементов $\Rightarrow$ $k=x \Rightarrow |[x;a]|=m+1$. Построим биекцию между этим отрезком и множеством $F=[0;m]$.
\end{enumerate}

\par Таким образом, получаем, что наше ВУМ равномощно $\omega \cdot L + F$, где $L$ - множество предельных элементов кроме, возможно, наибольшего, а $F$ - конечное множество $\blacksquare$
\newpage

\subsection{Непрерывная зависимость от параметров решения задачи Коши для нормальной системы дифференциальных уравнений (б/д)}
\subsection{Теорема о трансфинитной рекурсии.}

\textbf{Теорема.} Пусть A — вполне упорядоченное множество, B — произвольное множество. Пусть имеется некоторое рекурсивное правило (отображение F, которое ставит в соответствие элементу $x \in A$ и функции $g : [0, x) \rightarrow B$ некоторый элемент B). Тогда $\exists !$ функция $f : A \rightarrow B$: $f(x) = F(x, f|_{[0,x)})$ $\forall x \in A$. (Здесь $f|_{[0,x)}$ обозначает ограничение функции f на начальный отрезок [0, x) — мы отбрасываем все значения функции на элементах, больших или равных x.)

$\blacktriangle$
Идея доказательства: значение f на минимальном элементе определено однозначно, так как предыдущих значений нет (сужение $f|_{[0,0)}$ пусто). Тогда и на следующем элементе значение функции f определено однозначно, поскольку на предыдущих (точнее, единственном предыдущем) функция f уже задана, и т. д.

Строгое док-во:

1. Утверждение о произвольном $a \in A$: существует и единственно отображение f отрезка [0, a] в множество B, для которого рекурсивное определение (равенство, приведённое в условии) выполнено при всех $x \in [0, a]$.

Пусть отображение $f : [0, a] \rightarrow B$, обладающее указанным свойством - "корректное". Таким образом, мы хотим доказать, что $\forall a \in A$ $\exists!$ корректное отображение отрезка [0, a] в B. Поскольку мы рассуждаем по индукции, можно предполагать, что для всех $c < a$ это утверждение выполнено, то есть существует и единственно корректное отображение $f_c : [0, c] \rightarrow B$. (Корректность $f_c$ означает, что при всех $d \leqslant c$ значение $f_c(d)$ совпадает с предписанным по рекурсивному правилу.)

Рассмотрим отображения $f_{c_1}$ и $f_{c_2}$ для двух различных $c_1$ < $c_2$. Отображение $f_{c_2}$ определено на большем отрезке $[0, c_2]$. Если ограничить $f_{c_2}$ на меньший отрезок $[0, c_1]$, то оно совпадёт с $f_{c_1}$, поскольку ограничение корректного отображения на меньший отрезок корректно (это очевидно), а мы предполагали единственность на отрезке $[0, c_1]$.

Таким образом, все отображения $f_c$ согласованы друг с другом (принимают одинаковое значение, если определены одновременно). Объединив их, мы получаем некоторое единое отображение h, определённое на $[0, a)$. Применив к a и h рекурсивное правило, получим некоторое значение $b \in B$. Доопределим h в точке a, положив $h(a) = b$. Получится отображение $h: [0, a] \rightarrow B$; легко понять, что оно корректно.

Чтобы завершить индуктивный переход, надо проверить, что на отрезке $[0, a]$ корректное отображение единственно. В самом деле, его ограничения на отрезки $[0, c]$ при $c < a$ должны совпадать с $f_c$, поэтому осталось проверить однозначность в точке a — что гарантируется рекурсивным определением (выражающим значение в точке a через предыдущие). На этом индуктивное доказательство заканчивается.

2. Осталось лишь заметить, что для разных a корректные отображения отрезков $[0, a]$ согласованы друг с другом (сужение корректного отображения на меньший отрезок корректно, применяем единственность) и потому вместе задают некоторую функцию $f : A \rightarrow B$, удовлетворяющую рекурсивному определению. Существование доказано; единственность тоже понятна, так как ограничение этой функции на любой отрезок [0, a] корректно и потому однозначно определено, как мы видели.
$\blacksquare$
\bigbreak
\bigbreak
\subsection{Дифференцируемость решения по параметрам, уравнение в вариациях (б/д)}
\subsection{Сравнимость любых двух вполне упорядоченных множеств.}
\textbf{Теорема}. Если А и В - в.у.м., то верно ровно одно из трёх:

1)  $A \simeq B$
	
2) $A \simeq [0, b)$, $b \in B$
	
3) $B \simeq [0, a)$, $a \in A$

$\blacktriangle$

1. Покажем, что 2 и 3 не могут быть выполнены одновременно. $A \simeq [0, b)$, $B \simeq [0, a)$  $\Rightarrow$ начальный отрезок B изоморфен начальному отрезку начального отрезка A, а начальный отрезок начального отрезка так же является начальным отрезком. Получили что \emph{A изоморфно своему начальному отрезку}, что невозможно по следствию, противоречие. Аналогичными рассуждениями можно понять, что 1 и 2, 1 и 3 тоже не могут быть выполнены одновременно. 
Таким образом, понимаем, что не больше одного из этих пунктов может быть выполнено. 

2. Покажем, что хотя бы один из этих пунктов будет выполнен (будем использовать трансфинитную рекурсию): постепенно построим функцию с аргументами в A и значениями в B. Строим функцию $g: A  \rightarrow B \cup\{\perp\}$, где $\perp$ - специальный символ неопределённости (любую частично определённую функцию можно переделать во всюду определённую, если добавить специальный символ неопределённости)

Строим функцию рекурсивно:
$g(a) = \{min\{y \in B: y \neq g(x)$ для $x < a \}\}$ (1), если это множество не пусто, иначе - $\perp$.

Корректность определения: \emph{функция g  существует и единственна}.
Скажем, что $g|_{[0, a)}: [0,a) \rightarrow B\cup\{\perp\}$ корректна, если она удовлетворяет соотношению (1). Докажем по трансфинитной индукции, что $g|_{[0, a)}$ существует и единственна. Пусть $\forall x < a$ $g|_{[0, a)}$ существует и единственна. Тогда при $x < a$  $g|_{[0, a)}(x)$ определено однозначно. 

Пусть a < c . Тогда $g|_{[0, a)}$ и $g|_{[0, c)}$ совпадают на $[0, a)$ (ввиду однозначности). Можно рассмотреть $g: A  \rightarrow B \cup\{\perp\}$, которая продолжает все $g|_{[0, a)}$. Если в множестве А есть максимальный элемент, то он не попадёт ни в один из полуинтервалов, но он ровно один, и для него всё доопределится по (1). Если же максимального элемента нет, то нужно всё объединить.

I. $\exists a: g(a) = \perp$ $\Rightarrow$ при всех $c > a$ $g(c) = \perp$

Если $g(c) = \perp$, то пусть $a = min\{x| g(x) = \perp\}$. Тогда $B \simeq [0, a)$. Доказывается, что при $x < a$ начальный отрезок $[0, x) \simeq [0, g(x))$, g - изоморфизм. Пусть при  $y < x [0, y) \simeq [0, g(y))$.

	инъекция: $y_1 < y_2 < x \Rightarrow g(y_2) = min\{z \in B: z \neq g(x)$ для $x<y_2\}$ $\Rightarrow$ $g(y_1) \neq g(y_2)$
	
	
	сюръекция: $z < g(x) \Rightarrow z = g(v)$ при $v < x$
	

Сохранение порядка: $y_1 < y_2 < x \Rightarrow g(y_1) < g(y_2)$ . По написанному выше $g(y1) \neq g(y2)$. Но $g(y_2)$ не может быть меньше, чем $g(y_1)$, иначе бы получилось, что до $g(y_1)$ есть какие-то пустые места, и $g(y_1)$ бы определилось не так, как оно определилось, а занято было бы то пустое место.

II. $\nexists a: g(a) = \perp$: 

- все значения в B принимаются. Тогда $A \simeq B$

- не все значения в B принимаются. Тогда $b = min\{y | y \neq g(x), x \in A\}$, и $A \simeq [0, b)$
$\blacksquare$
\newpage

\subsection{Задача Коши для уравнения первого порядка, не разрешенного относительно производной, теорема существования и единственности решения задачи Коши.}
\label{zk-notsolved}
Путь у нас есть задача Коши
\begin{center}
    $\begin{cases}
    F \brackets{x, y, y'} = 0; \\
    y\brackets{x_0} = y_0; \\
    y'\brackets{x_0} = p_0
    \end{cases}$
\end{center}

\textbf{Теорема($\exists ! \text{решение ЗК для ур-ния 1-го порядка, не разр. отн. } y'$):}
\newline Пусть $F(x, y, p)$ опрелена и непреывна вместе с $ \frac{\del F}{\del y} $ и $ \frac{\del F}{\del p} $ в некоторой области $G\subseteq \R^3$ и в точке $(x_0, y_0, p_0) \in G $ справедливо $\mathlarger{\frac{\del F}{\del p} \bigg|_{\brackets{x_0, y_0, p_0}}}\!\!\! \neq 0$. \newline Тогда существует $\delta > 0$, такое что на отрезке $[x_0 - \delta; x_0 + \delta]$ существует и единственно решение ЗК. 
\bigbreak
\Proof Так как $\frac{\del F}{\del p} \neq 0$ и частные производные непрерывны в $G$, то по теореме о неявной функции существует окрестность точки $\brackets{x_0, y_0}$ и существует непрерывно дифференцируемая функция $f\brackets{x,y}$, определённая на окрестности $(x_0, y_0)$, такая что $f\brackets{x_0, y_0} = p_0$ и для любой точки из $(x_0, y_0)$ верно равенство $p = f(x, y)$. Тогда задача Коши будет формулироваться так:

\begin{center}
    $\begin{cases}
    y' = f\brackets{x, y}; \\
    y\brackets{x_0} = y_0
    \end{cases}$
\end{center}

Вновь применим теорему о неявной функции, получим $\mathlarger{\frac{\del f}{\del y} = - \frac{\frac{\del F}{\del y} \brackets{x, y, p}}{\frac{\del F}{\del p} \brackets{x, y, p}} \bigg|_{p = f \brackets{x, y}}}$.

Внутри окрестности точки $(x_0, y_0)$ можно взять выпуклую область, на которой для $f(x,y)$ будет выполняться условие Липшица. Тогда получим требуемое по теореме о существовании и единственности решения задачи Коши для уравнения, разрешённого относительно производной. \EndProof

\subsection{Особые решения}
\noindent \Def Точка $\brackets{x_0, y_0, p_0} \in G$ называется \textit{особой точкой уравнения}, если в окрестности этой точки решение задачи Коши либо не существует, либо не единственно. В таких точках $\frac{\del F}{\del p} = 0$
\\
\Def Если для особой точки решений задачи Коши не менее двух, то такая точка называется \textit{точкой локальной неединственности}.
\\
\Def Множество точек локальной неединственности $-$ \textit{дискриминантное множество}.
\\
\Def \textit{Особым решением} уравнения $F \brackets{x, y, y'} = 0$ называется такое решение уравнения, для которого любая точка $\brackets{x, y}$ является точкой локальной неединственности. График особого решения в каждой точке касается графика некоторого другого решения. Ясно, что такие решения могут быть найдены только среди дискриминантных кривых.
\bigbreak
\Th Если $\varphi(x)$ -- особое решение $F \brackets{x, y, y'} = 0$, то в каждой точке его интегральной кривой справедлива система:
\begin{equation*}
    \begin{cases}
    F(x,y,p) = 0 \\
    \frac{\partial F}{\partial p} = 0
    \end{cases}
\end{equation*}
\Proof Возьмем точку $(x_0, y_0 = \varphi(x_0), p_0=\varphi'(x_0))$. Первое условие системы выполнено автоматически, так как $\varphi$ -- решение по условию. Пусть второе уравнение не выполнено, то есть $\frac{\partial F}{\partial p}(x_0, y_0, p_0) \neq 0$. Но тогда выполнена теорема $\exists ! \text{решение ЗК}$, иначе говоря, есть окрестность $(x_0, y_0, p_0)$, в которой через нее проходит ровно 1 интегральная кривая, что противоречит определению $\varphi$. А значит, второе уравнение тоже выполнено. \; \EndProof
\bigbreak 
Таким образом, особое решение принадлежит дискриминантному множеству, но не наоборот.
\newpage

\section{Линейные дифференциальные уравнения и линейные системы дифференциальных уравнений с постоянными коэффициентами}

\subsection{Фундаментальная система решений и общее решение линейного однородного уравнения n-го порядка}
\label{firstthird-anchor}
\subsection{Эквивалентность следующих утверждений: множество перечислимо, полухарактеристическая функция множества вычислима, множество является областью
определения вычислимой функции, множество является проекцией разрешимого
множества пар.}

\textbf{Теорема.} Следующие утверждения для непустого $S \subseteq \mathbb{N}$ эквивалентны:

1) S перечислимо (существует печатающая машина, такая, что $\forall x \in S$ x встречается в потоке вывода, $\forall x \notin S$ x не встречается в потоке вывода);

2) Полухарактеристическая функция множества (равная 0 на элементах S и не определённая вне S) вычислима;

3) S - область определения вычислимой функции (если существует алгоритм, её вычисляющий, то
есть такой алгоритм A, что $\forall f(n)$ определённых для некоторого n алгоритм А остановится на входе n и напечатает f(n), иначе - не остановится на входе n);

4) S - проекция разрешимого (существует алгоритм, который по любому натуральному n определяет, принадлежит ли оно множеству) множества пар.

$\blacktriangle$
(1) $\Rightarrow$ (2). Запускаем эту печатающую машину. Если она выдаёт x, то значение полухарактеристической функции 1, иначе - $\perp$.

(2) $\Rightarrow$ (3). S - область определения характеристической функции, описанной ранее.

(3) $\Rightarrow$ (1). Пусть S - область определения вычислимой функции f, вычисляемой алгоритмом B. Тогда есть алгоритм, перечисляющий A: параллельно запускать B на входах 0, 1, 2, ..., делая всё больше шагов (1 шаг на входах 0 и 1, 2 шага - на входах 0, 1, 2, и.т.д.); напечатать все номера, на которых B остановился. 

(1) $\Rightarrow$ (4). S = $\{ x | \exists n (x, n) \in B\}$ - проекция множества $B = \{ (x, n):$ x в первых n шагах алгоритма, перечисляющего S$\}$

(4) $\Rightarrow$ (1). for (x=0;; ++x) \\
for (y=0;; ++y) 

    $\{ if ((x, y) \in B)$ cout $<<$ x;

    $if ((y, x) \in B)$ cout $<<$ y; $\}$
$\blacksquare$

\newpage

\subsection{Линейное неоднородное уравнение \texorpdfstring{$n$}-го порядка с постоянными коэффициентами и правой частью квазимногочленом}
\Def Эти уравнения имеют вид
\[y^{(n)}+a_1y^{(n-1)}+a_2y^{(n-2)}+...+a_n y = f(x),\]
где $f(x)$ квазимногочлен: $f(x) = e^{\mu x} P_m(x)$, $\mu \in \mathbf{C}$ $P_m(x)$ - заданный многочлен степени $m$ с комплексными коэффициентами. 

\Def Характеристическим многочленом $L(x)$ назовём многочлен
\[L(X) = a_n x^n + a_{n-1} x^{n-1} + ... + a_0\]

\Def Дифференциальным оператором $D$ назовём оператор
\[D = \frac{d}{dx}\]

\Note $D^n y = y^{(n)}$ 

Существование и единственность решения следуют из таковых для системы

\begin{equation*}
    \begin{cases}
        \vec{y'}_1 = y_2&\\
        \vec{y'}_2 = y_3&\\
        \ldots&\\
        \vec{y'}_{n-1} = y_n&\\
        \vec{y'}_n = f - a_1 y_1 - a_{2}y_2 - \ldots - a_n y_n
    \end{cases}
\end{equation*}

(здесь $y_1 = y$)
\bigbreak
\Def Если число $\mu$ является корнем характеристического уравнения 
\[L(\lambda) = \lambda^n + a_1 \lambda^{n-1}+...+ a_n = 0\]
то говорят, что в уравнении резонансный случай. Если же $\mu$ не является корнем, то имеем нерезонансный случай.

\Def Дифференциальным многочленом назовём многочлен вида 
\[L(D) = (D-\lambda_1)^{k_1} (D-\lambda_2)^{k_2} ... (D-\lambda_s)^{k_s},\]
Где $k_s$ соответствующие кратности корней характеристического уравнения
\bigbreak
Рассмотрим ЛОУ. Покажем, что если известно некоторое решение $y_0(x)$ ЛНУ, то замена $y = z + y_0$ приводит уравнение к ЛОУ. Воспользуемся представлением левой части через дифференциальный многочлен:

\[L(D)y=L(D)(z+y_0) =L(D)z + L(D) y_0 = L(D)z + f(x) = f(x)\]

Откуда следует, что $L(D)z = 0$, т.е. решение.

Рассмотрим $L(D) y(x) = e^{\mu x} P_m(x)$.
\bigbreak
\Statement $(P_m(x)e^{\lambda x})'_x = Q_m(x)e^{\lambda x}$

\begin{theorem}[О структуре ФСР]
Пусть $\lambda_1, ..., \lambda_k$ корни характеристического многочлена кратности $l_1, ..., l_k$. Тогда набор функций $x^s e^{\lambda_i x}$, где $s = 0,..., l_1-1$, $i = 1, ..., k$ является ФСР для рассматриваемого уравнения
\end{theorem}
Доказано в пункте~<<\nameref{firstthird-anchor}>>.

\begin{theorem}[О структуре решения ЛНУ c правой частью в виде квазимногочлена]
Для рассматриваемого уравнения существует и единственно решение вида
\[y(x) = x^k e^{\mu x} Q_m(x) \]
где $ Q_m(x)$ - многочлен одинаковой с $P_m(x)$ степени $m$, а число $k$ равно кратности корня $\mu$ в уравнении $L(\lambda)=0$ в резонансном случае и $k=0$ в нерезонансном
\end{theorem}

\Proof
Если $\mu \neq 0$, то заменой $y = z e^{\mu x}$ всегда можно избавиться от $e^{\mu x}$ в правой части. В самом деле, по формуле сдвига после замены имеем что \[L(D) y = L(D)(e^{\mu x} z) = e^{\mu x} L(D+\mu) z = e^{\mu x}  P_m(x),\]
откуда $L(D+\mu)z=P_m(x)$.

Таким образом, доказательство теоремы осталось провести для уравнения вида
\[L(D)y=P_m(x)\]

\begin{enumerate}
    \item Нерезонансный случай: $L(\mu) \neq 0$. Пусть
    \[P_m(x) = p_m x^m + ... + p_0\]
    \[Q_m(x) = q_m x^m + ... + q_0\]
    Если подставить и приравнять коэффициенты при одинаковых степенях $x$, получим линейную алгебраическую систему уравнения для опредения неизвестных коэффициентов $q_0, ... q_m$. Матрица системы треугольная с числами $a_n = L(0) \neq 0$, таким образом, все коэффициенты определяются из неё одназначно.
    \item В резонансном случае имеем 
    \[L(\lambda) = \lambda^k (\lambda^{n-k} + a_1 \lambda^{n-k-1} + ...+a_{n-k}) \]
    Следовательно,
    \begin{equation*}
        L(D) = 
        \begin{cases}
           D^n + a_1 D^{n-1} + ... + a_{n-k} D^k, k < n\\
           D^n, k = n
        \end{cases}
    \end{equation*}
    В первом случае замена $D^k y = z$ приводит к уравнению с нерезонансным случаем. Рассмотрим уравнение
    \begin{equation*}
        D^k(y) = 
        \begin{cases}
           R_m(x), k < n\\
           P_m(x), k = n
        \end{cases}
    \end{equation*}
    Взяв нулевые начальные условия для этого уравнения
        \[y(x) = y'(0) = ... = y^{(k-1)} (0) = 0\]
        получим единственное решение вида 
    \[y(x) = x^k Q_m(x)\]
\end{enumerate}
\EndProof

\textbf{О вещественнозначной ФСР}
Для уравнения с вещественнозначными коэффициентами комплексные корни $M(\lambda)$ распадаются на сопряжённые пары одинаковой кратности. Соответствующие им решения легко
заменяются на вещественные функции (по аналогии с однократными действительными корнями $M(\lambda)$).


\newpage

\subsection{Уравнение Эйлера}
Уравнением Эйлера называется уравнение вида
\[
x^n y^{(n)} + a_1 x^{n-1} y^{(n-1)} + a_2 x^{n-2} y^{(n-2)} + ... + a_{n-1} x y^{'} + a_n y = 0
\]
Данное уравнение сводится к уравнению с постоянными коэффициентами при замене $x = e^t$ при $x > 0$ и  $x = -e^t$ при $x < 0$.
 
Докажем это индукцией по порядку:

\begin{align*}
    \frac{dy}{dx} &= \frac{y^{'}_t}{x^{'}_t} = e^{-t} y^{'}_t\\
    \frac{d^2 y}{dx^2} &= e^{-2t} ( y^{''}_t - y^{'}_t)\\
    \frac{d^3 y}{dx^3} &= e^{-3t} ( y^{'''}_t - 3y^{''}_t + 2 y^{'}_t)\\
    &\ldots\\
    \frac{d^n y}{dx^n} &=  e^{-nt} \varphi(y^{(n)}_t, y^{(n-1)}_t, \ldots, y^{'}_t)
\end{align*}

Подставим найденные выражения в определение и получим уравнение вида, где $y^{(n)}$ зависит от нового параметра $t$:

\[a_0 y^{(n)} + b_1 y^{(n-1)} + b_2 y^{(n-2)} + ... + b_{n-1} y^{'} + b_n y = 0\]

Поскольку мы получили линейное однородное с постоянными коэффициентами, то его фундаментальная система решений может содержать лишь функции вида

\[ e^{\lambda t}, \;  t^k e^{\lambda t}, \;  e^{\lambda t} cos(\gamma t), \;  e^{\lambda t} sin(\gamma t), \; 
 t^k e^{\lambda t}  cos(\gamma t), \; 
  t^k e^{\lambda t}  sin(\gamma t) \]
\newpage

\subsection{Фундаментальная система решений и общее решение нормальной линейной однородной системы уравнений}

\Def \textit{Нормальной системой дифференциальных уравнений} называется система дифференциальных уравнений первого порядка, разрешённых относительно производной:
\begin{equation*}
 \begin{cases}
   y_1' = f_1(x, y_1, ..., y_n) \\
   \dots \\
   y_n' = f_n(x, y_1, ..., y_n)
 \end{cases}
\end{equation*}
где $x$ -- независимая переменная,\\
$y_1(x), \dots, y_n(x)$ -- неизвестные функции


\textbf{Построение фундаментальной системы решений}

$$
\Vec{x}(t) = \begin{pmatrix}
x_{1}(t)\\
\vdots\\
x_{n}(t)
\end{pmatrix},\ \ \ 
A_{n\times n} = (a_{ij}),\ \ \ 
\Vec{f} = \begin{pmatrix}
f_{1}(t)\\
\vdots\\
f_{n}(t)
\end{pmatrix} 
$$
\begin{equation}
\dot{\Vec{x}} = A\Vec{x} + \Vec{f}(t)
\end{equation}

\begin{equation}\label{dotxax}
\dot{\Vec{x}} = A\Vec{x}
\end{equation}

\Th Если $\Vec{h_1}, \dots,\Vec{h_n}$ - базис из собственных векторов матрицы $A$, то $\Vec{x_i} = e^{\lambda_it}\Vec{h_i}$ - ФСР для уравнения (\ref{dotxax})

\Proof
Заметим, что $A(e^{\lambda t}\Vec{h}) = e^{\lambda t} (A\vec{h}) = e^{\lambda t}\lambda \Vec{h} = (e^{\lambda t} \Vec{h})'$, значит собственный вектор является решением (\ref{dotxax}). Их линейная независимость следует из того, что их вронскиан в точке t=0 равен определителю из координатных столбцов этого базиса, а значит не равен нулю.
\EndProof

\textbf{Замечание.} Если $\lambda - $комплексное собственное значение матрицы $A$, то $\Vec{\lambda}$, и соответствующие им собственные векторы покомпонентно сопряжены. Это позволяет нам перейти в базис, содержащий только действительнозначные функции (экспонента, синус, косинус).

\subsubsection*{Жорданова нормальная форма}

\Def. Пусть $\overline{h_1}$ - собственный вектор матрицы $A$ для собственного значения $\lambda$:
$$(A - \lambda E) \overline{h_1} = \overline{0}$$

Последовательность $\{\overline{h_i}\}^k_{i=1}$, определяемая соотношением $(A - \lambda E) \overline{h_{i+1}} = \overline{h_i}$, причём
уравнение $(A - \lambda E) \overline{h} = \overline{h_k}$ не имеет решений, называется \textit{жордановой цепочкой}, а её элементы (кроме $\overline{h_1}$) — \textit{присоединёнными (к $\overline{h_1}$) векторами}.

Матрица следующего вида называется \textit{жордановой клеткой}:

\begin{equation*}
\begin{pmatrix}
\lambda & 1 & 0 & \dots & 0\\
0 & \lambda & 1 & \dots & 0\\
0 & 0 & \lambda & \dots & 0\\
\dots & \dots & \dots & \dots & \dots\\
0 & 0 & 0 & \dots & \lambda
\end{pmatrix}
\end{equation*}

Блочно-диагональная матрица, на диагонали которая стоят жордановы клетки, называется \textit{жордановой.}

Пусть $S$ — (числовая) матрица перехода, переводящая $A$ в жорданову матрицу $J$. Соответствующий базис называется жордановым, а произведение $SJS^{-1}$ — \textit{жордановой нормальной формой}.

\Lemma Пусть $S(t)$ — матрица-функция размера $n \times n$, $\overline{x}(t)$ — $n$-мерная вектор-функция,
тогда
$$\frac{d}{dt}(S(t) \overline{x}(t)) = \frac{dS}{dt} \overline{x} + S \dot{\overline{x}}$$

\Proof 
Посчитаем явно:
\[(S(t) \overline{x}(t))'_i = \left( \sum_{k=1}^n s_{ik}(t) x_k(t) \right)' = \sum_{k=1}^n \dot{s}_{ik}(t) x_k(t) + \sum_{k=1}^n s_{ik}(t) \dot{x}_k(t).\]
Это и есть требуемое
\EndProof
\\

Общее решение однородной системы $\dot{\overline{x}} = A\overline{x}$: Введём $\overline{y}$ следующим образом: $\overline{x} = S\overline{y}$. Заметим, что $\dot{\overline{x}} = A\overline{x}$ можно преобразовать в $S \dot{\overline{y}} = AS\overline{y}$, а затем (в силу невырожденности $S$) в
$$\dot{\overline{y}} = J\overline{y}$$

(J - ЖНФ)

Полученная система уравнений решается “поблочно”.

Рассмотрим один блок системы уравнений:

\begin{equation*}
 \begin{cases}
    \dot{y}_1 = \lambda y_1 + y_2\\
    \dot{y}_2 = \lambda y_2 + y_3\\
    \dots\\
    \dot{y}_{k-1} = \lambda y_{k-1} + y_k\\
    \dot{y}_k = \lambda y_k
 \end{cases}
\end{equation*}

Выполним следующую замену: $y_i = e^{\lambda t} z_i$

\begin{equation*}
 \begin{cases}
    \lambda e^{\lambda t} z_i + e^{\lambda t} \dot{z}_i = \lambda e^{\lambda t} z_i +  e^{\lambda t} z_{i+1}\\
    \dots\\
    \lambda e^{\lambda t} z_k + e^{\lambda t} \dot{z}_k = \lambda e^{\lambda t} z_k
 \end{cases}
\end{equation*}

\begin{equation*}
 \begin{cases}
    \dot{z_i} = z_{i+1}\\
    \dots\\
    \dot{z_k} = 0
 \end{cases}
\end{equation*}

Следовательно,
$$z_i = \sum_{j=i}^k c_j \frac{t^{j-i}}{(j - i)!}$$
$$y_i = e^{\lambda t} z_i = e^{\lambda t} \sum_{j=i}^k c_j \frac{t^{j-i}}{(j - i)!}$$

Для векторов из одной жордановой цепочки константы $c_j$ одинаковые.

Объединим все компоненты в вектор и перейдём в исходный базис. Получим общее решение однородной системы $\dot{\overline{x}} = A \overline{x}$:

$$\overline{x} = \sum_{i=1}^n y_i \overline{h}_i$$

\subsection{Линейная неоднородная система уравнений в случае, когда неоднородность представлена векторным квазимногочленом (б/д)}

\Def \textit{Вектор-квазимногочленом} размерности $n$ и степени $m$ назовём $n$-мерную вектор-функцию, компонетами которой являются квазимногочлены, а максимальная степень хоты бы одного квазимногочлена равна $m$: $\overline{x}(t) = e^{\mu t} \overline{P_{m}}(t)$.

\Th (О частном решении СЛДУ с неоднородностью в виде векторного квазимногочлена)
Если в системе уравнений $\dot{\overline{x}} = A \overline{x} + \overline{f(t)}, \overline{f(t)} = e^{\mu t} \overline{P_{m}}(t)$, то $\exists !$ решение вида $\overline{x}(t) = e^{\mu t} \overline{Q_{m+l}}(t),$ где $l =0$, если $\mu$ не является собственным значением $A$ или $l$ не превосходит длины наибольшей жордановой цепочки для $\mu$.





\newpage

\subsection{Матричная экспонента, ее свойства и применение к решению нормальных линейных систем}

\Def Пусть $t$ - действительная переменная, $A_{n \times n}$ - комплекснозначная квадратная матрица. Матричной экспонентой называется ряд: $$e^{tA} = E_{n \times n} + \sum_{k=1}^{\infty} \frac{t^k}{k!} A^k$$
где $a_{ij}^{(k)}$ - элемент матрицы $A^k$ на месте $ij$ (верхний индекс $a$ это не возведение в степень)

Введём обозначение частичных сумм:
$$S_m = E_{n \times n} + \sum_{k=1}^{m} \frac{t^k}{k!} A^k$$
$$(S_m)_{ij} = \delta_{ij} + \sum_{k=1}^{m} \frac{t^k}{k!} a^{(k)}_{ij}$$

\par Корректность определения.

\Def Матричный ряд 
$$e^{tA} = E_{n \times n} + \sum_{k=1}^{\infty} \frac{t^k}{k!} A^k$$ 
называется \textit{сходящимся} при $t_0 \in \R$, если степенной ряд $$(S_m)_{ij} = \delta_{ij} + \sum_{k=1}^{m} \frac{t^k}{k!} a^{(k)}_{ij}$$ сходится для всех $i$ и $j$.

\Lemma $\forall A \in M_{n \times n}(\R)$ верно, что ряд $e^{tA} = E + \sum_{k=1}^{\infty} \frac{t^k}{k!} A^k$ сходится абсолютно.

\Proof
Пусть $M = \max_{i, j}|a_{ij}|$. 

1) Докажем по индукции: $|a_{ij}^{(k)}| \leqslant n^{k-1}M^k$. База:$ |a_{ij}^{(1)}| \leqslant n^0 M$

2) $|a_{ij}^{(k)}| = |\sum_{l=1}^n a_{il}^{(1)}a_{lj}^{(k-1)}| \leqslant \sum_{l=1}^n |a_{il}^{(1)}a_{lj}^{(k-1)}| \leq M n (n^{k-2} M^{k-1}) \leqslant n^{k-1} M^k$\\
Рассмотрим ряд 

$1 + \frac{|t|}{1!} M + \frac{|t|^2}{2!} n M^2 + \dots + \frac{|t|^k}{k!} n^{k-1} M^k + \dots$ (*)

$\overline{\lim}_{k \rightarrow \infty} \frac{\left(\frac{n^k M^{k+1}}{(k+1)!}\right)}{\left(\frac{n^{k-1} M^k}{k!}\right)} = \overline{\lim}_{k \rightarrow \infty} \frac{nM}{k+1} = 0 \rightarrow$ ряд (*) сходится по признаку Даламбера и мажорирует каждый компонентный ряд.

\EndProof

\textbf{Замечание}

Матричная экспонента сходится равномерно на $\forall [\alpha, \beta] \in \R_t^1$

\Lemma (формула матричного бинома)

Если $A$ и $B$ перестановочны, то $\forall n \in \N: (A + B)^n = \sum_{k=0}^n C_n^k A^k B^{n-k}$

\Lemma Если $A$ и $B$ перестановочны (т.е. $AB = BA$), то $\forall t \in \R$
$$e^{tA} e^{tB} = e^{tB} e^{tA} = e^{t(A+B)}$$

\Proof

$$e^{t(A+B)} = \sum_{n=0}^{\infty} \frac{t^n}{n!} (A+B)^n = \sum_{n=0}^{\infty} \sum_{k+m=n} \frac{t^k A^k}{k!} \frac{t^m B^m}{m!} =\text{(сходится абсолютно)}=\sum_{k=0}^{\infty} \sum_{m=0}^{\infty} \frac{t^k A^k}{k!} \frac{t^m B^m}{m!} =$$ 
$$= \sum_{k=0}^{\infty} \frac{t^k A^k}{k!} \sum_{m=0}^{\infty} \frac{t^m B^m}{m!} = e^{tB} e^{tA} = e^{tA} e^{tB}$$

\EndProof

\textbf{Следствие} 

$e^{tA}$ невырождена $\forall t \in R,$ и $(e^{tA})^{-1} = e^{-tA}$

\Proof

$$E = e^{t(A-A)} = e^{tA} e^{-tA}$$

\EndProof

\Lemma (свойства матричной экспоненты)

1) Если $S$-невырожденная и $A = SBS^{-1}$, то $e^{tA} = Se^{tB}S^{-1}, \forall t \in \R$

2) $\frac{d}{dt}(e^{tA}) = Ae^{tA} = e^{tA}A$

\Proof

1) Заметим, что $A^k = SB^kS^{-1}$

$$\sum_{k=0}^{\infty} \frac{t^k}{k!}A^k = S (\sum_{k=0}^{\infty} \frac{t^k}{k!}B^k) S^{-1} \rightarrow_{(k \rightarrow \infty)} S e^{tB} S^{-1}, \sum_{k=0}^{\infty} \frac{t^k}{k!}A^k \rightarrow_{(k \rightarrow \infty)} e^{tA}$$

2) $$\frac{d}{dt} (\sum_{k=0}^n \frac{t^k}{k!} A^k) = (\sum_{k=1}^n \frac{t^{k-1}}{(k-1)!} A^k) = \sum_{k=0}^{n-1} \frac{t^{k}}{k!} A^{k+1} = A \sum_{k=0}^{n-1} \frac{t^{k}}{k!} A^{k} \rightarrow A e^{tA}$$

\EndProof

\Th (матричная экспонента для ФСР)\\
Матрица $e^{tA}$ является фундаментальной матрицей для системы линейных уравнений $\dot{\overline{x}} = A \overline{x}$.

\Proof

$(e^{tA})' = Ae^{tA}$, следовательно, каждый столбец матрицы  $Ae^{tA}$ является решением системы $\dot{\overline{x}} = A \overline{x}$. Поскольку $det\  e^{tA} \neq 0$ при любом $t$ (т.е. столбцы содержат независимые решения), то $e^{tA}$ фундаментальна.
\EndProof

Общее решение системы $\dot{\overline{x}} = A \overline{x}$ это $\overline{x} = e^{tA} \overline{c}$, где $\overline{c}$ - вектор констант.

\Th общее решение системы $\dot{\overline{x}} = A \overline{x} + \overline{f}(t)$ задаётся следующей формулой:

$$\overline{x} = e^{tA} \left( \int_{t_0}^t e^{-\tau A} \overline{f}(\tau) d\tau + \overline{c}_0 \right)$$

\Proof
Метод вариации постоянных:

$$\overline{x} = e^{tA}\overline{c}(t)$$
$$(e^{tA}\overline{c}(t))' = Ae^{tA}\overline{c}(t) + e^{tA} \dot{\overline{c}}(t) = Ae^{tA}\overline{c}(t) + \overline{f}(t)$$
$$e^{tA} \dot{\overline{c}}(t) = \overline{f}(t)$$
$$\dot{\overline{c}}(t) = e^{-tA} \overline{f}(t)$$
$$\overline{c}(t) = \left( \int_{t_0}^t e^{-\tau A} \overline{f}(\tau) d\tau + \overline{c}_0 \right)$$

\EndProof

\textbf{Следствие.} Решение задачи Коши
$$\dot{\overline{x}} = A\overline{x} + \overline{f}(t) \quad \overline{x}(t_0) = \overline{x}_0$$

выражается в следующем виде:

$$\overline{x} = e^{tA} \left( \int_{t_0}^t e^{-\tau A} \overline{f}(\tau) d\tau + e^{-t_0 A} \overline{x}_0 \right) = e^{tA} \left( \int_{t_0}^t e^{-\tau A} \overline{f}(\tau) d\tau \right) + e^{(t - t_0) A} \overline{x}_0$$

\Proof 

Воспользуемся формулой 

$$\overline{x} = e^{tA} \left( \int_{t_0}^t e^{-\tau A} \overline{f}(\tau) d\tau + \overline{c}_0 \right)$$

положив \(t = t_0\), получим \(\overline{c}_0 = e^{-t_0 A}\overline{x}_0\).

\EndProof

\textbf{Пример.} 
Если $t_0 = 0$ и $\overline{f}(t) \equiv \overline{0}$, то $\overline{x} = e^{tA}\overline{x}_0$.


\newpage